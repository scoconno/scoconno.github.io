\documentclass{exam}
\usepackage[margin=1in]{geometry}
\newcommand{\class}{Critical Thinking}
\newcommand{\examnum}{Exam 1}
\newcommand{\term}{Fall, 2015 }
\newcommand{\examdate}{10/26/2015}
\parindent 0ex
\setlength\answerlinelength{3in}
\begin{document}
\pagestyle{head}
\firstpageheader{}{}{}
\runningheader{\class}{\examnum\ - Page \thepage\ of \numpages}{\examdate}
\runningheadrule

\begin{flushright}
\begin{tabular}{p{2.8in} r l}
\textbf{\class} & \textbf{Name (Print):} & \makebox[2in]{\hrulefill}\\
\textbf{\term} &&\\
\textbf{\examnum} &&\\
\textbf{\examdate} &&\\
\end{tabular}\\
\end{flushright}
\rule[1ex]{\textwidth}{.1pt}

\begin{minipage}[t]{3.7in}
\vspace{0pt}
\begin{itemize}
\item Enter all requested information on the top of this page.
\item You may use your textbook. 
\item You may not use notes, handouts, etc.
\item Do not write in the table to the right.
\item Write clearly. Poor handwriting may lead to loss in points.


\end{itemize}
 \rule[1ex]{\textwidth}{.1pt}

\end{minipage}
\hfill
\begin{minipage}[t]{2.3in}
\vspace{0pt}
%\cellwidth{3em}
\gradetablestretch{2}
\vqword{Problem}
\addpoints % required here by exam.cls, even though questions haven't started yet.	
\gradetable[v]%[pages]  % Use [pages] to have grading table by page instead of question

\end{minipage}


\begin{questions}
\addpoints
\section*{Multiple Choice Questions}

\question Circle the correct answer. 
\begin{parts}
\part[2]  A group of statements in which some of them (the premises) are intended to support another of them (the conclusion) is known as a(n)
\begin{choices}
\choice Chain argument		
\choice Claim	
\correctchoice Argument 
\choice Reason
\end{choices}

\part[2] The statements (reasons) given in support of another statement are called
\begin{choices}
\choice An argument
\choice The conclusion	
\correctchoice The premises 	
\choice The complement
\end{choices}

\part[2] These two statements ``The Wall Street Journal says that people should invest heavily in stocks. Therefore, investing in stocks is a smart move'' constitute
\begin{choices}
\choice No argument	
\choice An explanation	
\correctchoice An argument 	
\choice  Two conclusions	
\end{choices}

\part[2] A deductive argument is intended to provide 
\begin{choices}
\choice Probable support for its conclusion	Incorrect	
\choice Persuasive support for its conclusion	
\correctchoice Logically conclusive support for its conclusion 	
\choice Tentative support for its conclusion	
\end{choices} 

\part[2] An inductive argument is intended to provide
\begin{choices}
\choice Valid support for its conclusion	
\correctchoice Probable support for its conclusion 	
\choice Weak support for its conclusion	
\choice Truth-preserving support for its conclusion	
\end{choices}
\part[2] A deductively valid argument cannot have
\begin{choices}
\correctchoice  True premises and a false conclusion 	
\choice  False premises and a false conclusion	
\choice False premises and a true conclusion	
\choice True premises and a true conclusion	
\end{choices}

\part[2] This classic argument ``All men are mortal. Socrates is a man. Therefore, Socrates is mortal'' is
\begin{choices}
\choice  Inductively strong
\choice  Deductively cogent	
\choice Deductively invalid	
\correctchoice Deductively valid 	
\end{choices}

\part[2] ``If the Yankees win, they will be in the World Series'' is a…
\begin{choices}
\choice Negation	
\choice Disjunction
\correctchoice Conditional
\choice Conjunction
\end{choices}

\part[2] Modus ponens has this argument pattern
\begin{choices}
\choice If p, then q. q. Therefore, p.	
\choice If p, then q. If q, then r. Therefore, if p, then r.	
\choice Either p or q. Not p. Therefore, q.	
\correctchoice If p, then q. p. Therefore, q. 	
\end{choices}

\part[2] The invalid argument form known as denying the antecedent has this pattern:	
\begin{choices}
\choice If p, then q. p. Therefore, q.	
\choice If p, then q. q. Therefore, p.	
\correctchoice If p, then q. Not p. Therefore, not q. 	
\choice If p, then q. If q, then r. Therefore, if p, then r.	
\end{choices}

\part[2] An argument with this form, ``If p, then q. If q, then r. Therefore, if p, then r'' is known as
\begin{choices}
\choice Modus tollens	
\correctchoice Hypothetical syllogism 	
\choice Modus ponens	
\choice Disjunctive syllogism	
\end{choices}
\part[2] An argument with this form, ``Either p or q. Not p. Therefore, q'' is known as
\begin{choices}
\correctchoice Disjunctive syllogism 	
\choice Hypothetical syllogism	
\choice Modus tollens	
\choice Dual syllogism	
\end{choices}

\part[2] This argument, ``If Buffalo is the capital of New York, then Buffalo is in the state of New York. Buffalo is in the state of New York. Therefore, Buffalo is the capital of New York'', is an example of
\begin{choices}
\choice Denying the antecedent	
\choice Disjunctive syllogism	
\choice Affirming the antecedent	
\correctchoice Affirming the consequent
\end{choices}
\part[2] It is not reasonable to believe a claim when
\begin{choices}
\choice  It is criticized	
\choice Most people reject it
\correctchoice There is no good reason for doing so
\choice There is no good reason for examining it
\end{choices}

\part[2] The truth-table test of validity is based on the fact that it is impossible for a valid argument to have true premises and…
\begin{choices}
\choice A true conclusion	
\choice A negated conclusion	
\choice A conditional	
\correctchoice A false conclusion	
\end{choices}




\end{parts}





\subsection*{Translations}
\question For each of the following arguments, translate it into symbols. Use the standard symbols for the logical connectives and use  letters for the claims. 

\begin{parts}
\part[5] If there is no rain soon, the crops will die. If the crops die, there will be no food for the coming winter. The crops will not die. Therefore, there will be rain soon. 
\part[5] If we give kidnappers the money that they demand, then further kidnappings will be encouraged. If we do not give kidnappers the money that they demand, the kidnappers will kill the hostages. We will not give kidnappers the money that they demand. Therefore, the kidnappers will kill the hostages.

\part[5] Either Emilio walks or he takes the train. And either Joann takes the train or she does not take the train. If Emilio walks, then Joann takes the train. Emilio takes the train. So Joann will not take the train. 
\part[5] Either the herbal remedy alleviated the symptoms, or the placebo effect alleviated the symptoms. If the placebo effect is responsible for easing the symptoms, then the herbal remedy is worthless. The herbal remedy alleviated the symptoms. So the herbal remedy is not worthless. 
\part[5] If then things are either the result of coincidence or for an end, and these cannot be the result of coincidence, it follows that they must be for an end.
 
\end{parts}

\subsection*{Truth-Tables}

\question For each of the following arguments, do the following:
\begin{enumerate}
\item Compose a truth table. (4 points)
\item State whether the argument is valid or invalid. (1 point)
\end{enumerate}
\begin{parts}
\part[5] \begin{enumerate}
\item[ ] 
\item[P1.] $(p \lor q) \rightarrow (p \& q)$
\item[P2.] p \& q
\item[C.] $ p \lor q $
\end{enumerate}
\part[5] \begin{enumerate}
\item[ ] 
\item[P1.] $ a \lor (b \& c)$
\item[P2.] $ \neg(b \& c)$
\item[C.] a
\end{enumerate}
\part[5]  \begin{enumerate} 
\item[ ]
\item[P1.] $a \lor (b \rightarrow c)$
\item[P2.] $b \& \neg c$
\item[C.] $\neg c$
\end{enumerate}
\part[5] 
\begin{enumerate}
\item[ ] 
\item[P1.] $d \lor \neg e$
\item[P2.] $f \rightarrow e$
\item[C.] $d \rightarrow \neg f$
\end{enumerate}
\part[5]
\begin{enumerate}
\item[ ] 
\item[P1.] $d \rightarrow e$
\item[P2.] $e \lor f$
\item[P3.] e
\item[C.] $d \& f$
\end{enumerate}

\end{parts}


\section*{Integrative Exercises}

\question For each of the following arguments, do following:  
\begin{enumerate}
\item Identify the conclusion. (1 point)
\item Identify the premises. (1 point)
\item Translate the premises (2 points) and conclusion into symbols (1 point). 
\item Construct a truth-table for the argument. (4 points)
\item State whether the argument is valid or invalid? (1 point)
\end{enumerate}


\begin{parts}
\part[10] If the allies accidentally damage any holy sites when they attack enemy forces, the local people will never give the allies any respect. The allies, though, will not damage any holy sites. Therefore, the locals will respect the allies. 

\part[10] People do not have free will. For, if they did, they could be held morally responsible for what they do. But---as our judicial system demonstrates---people cannot really be held morally responsible for their actions. 


\end{parts}




\end{questions}



\end{document}