\documentclass[11pt,article,oneside]{memoir}

%%% custom style file with standard settings for xelatex and biblatex. Note that when [minion] is present, we assume you have minion pro installed for use with pdflatex.
%\usepackage[minion]{org-preamble-pdflatex} 

%%% alternatively, use xelatex instead
\usepackage{org-preamble-xelatex} 



\def\myauthor{Author}
\def\mytitle{Title}
\def\mycopyright{\myauthor}
\def\mykeywords{}
\def\mybibliostyle{plain}
\def\mybibliocommand{}
\def\mysubtitle{}
\def\myaffiliation{NJCU}
\def\myaddress{102 Phil}
\def\myemail{soconnor@njcu.edu}
\def\myweb{scottoconnor.org}
\def\myphone{}
\def\myversion{}
\def\myrevision{}
\def\myaffiliation{NJCU}
\def\myauthor{Dr. Scott O'Connor}
\def\mykeywords{}
\def\mysubtitle{Syllabus}
\def\mytitle{{\normalsize Phil 102, 3 Credit, Fall 2015, Somewhere. \newline} \HUGE Critical Thinking}


\begin{document}

%%% If using xelatex and not pdflatex
%%% xelatex font choices
\defaultfontfeatures{}
\defaultfontfeatures{Scale=MatchLowercase}    
% You will need to buy these fonts, change the names to fonts you own, or comment out if not using xelatex.      
\setromanfont[Mapping=tex-text]{Georgia} 
\setsansfont[Mapping=tex-text]{Georgia} 
\setmonofont[Mapping=tex-text,Scale=0.8]{Georgia} 

%% blank label items; hanging bibs for text
%% Custom hanging indent for vita items
\def\ind{\hangindent=1 true cm\hangafter=1 \noindent}
\def\labelitemi{$\cdot$}
%\renewcommand{\labelitemii}{~}

%% RCS info string for version tracking
\chapterstyle{article-3}  % alternative styles are defined in latex-custom-kjh/needs-memoir/
\pagestyle{kjh}

\title{\LARGE \mytitle}     
\author{\Large\myauthor \newline \footnotesize\texttt{\noindent\myemail}}
\date{09/02/2015--12/14/2015}

\published{\,}

\maketitle

% \thispagestyle{kjhgit}

% Copyright Page
%\textcopyright{} \mycopyright


%
% Main Content
%

\section{Copyright}
The materials used in this class, including, but not limited to, lectures, exams, quizzes, and homework assignments are copyright protected works.  Any unauthorized copying of the class materials or recording of lectures is a violation of federal law and may result in disciplinary actions being taken against the student.  Additionally, the sharing of class materials without the specific, express approval of the instructor may be a violation of the University's Student Honor Code and an act of academic dishonesty, which could result in further disciplinary action.  This includes, among other things, uploading class materials to websites for the purpose of sharing those materials with other current or future students. 

\section{Course Description}

In this course, we will be learning how how to identify, analyze, and
evaluate arguments. We will begin by discussing what critical thinking
is and why we should improve our ability to do it. Since critical
thinking concerns arguments, the bulk of the course will focus on
reconstructing arguments and learning how to diagnose them. Topics to be covered include argument patterns, argument types, fallacies,
introductory formal logic, probabilistic reasoning, and reasoning in the
law.


\section{Learning Objectives}

Upon completing this course, students will be able to (i) identify the
difference between deductive and inductive arguments, (ii) assess the
former for validity and soundness, the latter for strength, (iii)
identify argument patterns, (iv) diagnoze their own thoughts, (v) reason
better.

\section{Required Textbook}

By 09/08/2015, the following textbook must be purchased or rented:

\begin{itemize}
\item
  \href{http://www.amazon.com/Power-Critical-Thinking-Effective-Extraordinary/dp/0199856672/ref=sr_1_1?s=books\&ie=UTF8\&qid=1421936130\&sr=1-1\&keywords=critical+thinking+vaughn}{`The
  Power of Critical Thinking: Effective Reasoning about Ordinary and
  Extraordinary Claims', 4th edition, by Lewis Vaughn} (Available from
  the campus book store as well as online retailers)
\item
  \href{http://global.oup.com/us/companion.websites/9780199856671/student/}{Book's
  Companion website}
\end{itemize}

\section{Requirements}

\begin{itemize}
\item \textit{Workload:} Successfully completing this course requires a minimum commitment of 6 hours per week. 

\item \textit{6 Problem Sets:} administered through Blackboard. First problem set due in Week 3.

\item \textit{2 Exams:} There is a mid-term and cumulative final. 

\item \textit{Attendance:} Roll call will be taken from Week 1. 0.5 point per class up to a maximum of 10 points. Excludes Week 1 \& 2. 
 

\item \textit{Grade Distribution:} 6 Problem sets---10 points each (60 total); Mid-term---20; Final---25; Attendance--0.5 point per class (10 total).

\item \textit{Grade Breakdown:}

 \begin{tabular}{ | l | l | p{2cm} | l | l | }
    \hline 
96--115 & A  & &  77--79 &  C+ \\  
90--95 & A- & &  73--76 & C \\
87-89 & B+ &  &  70--72 & C- \\ 
83--86 & B  & &  60--69 & D\\
80--82 & B - & & 0--59 & F\\ \hline
    \end{tabular}


\end{itemize}




\section{Policies}
\begin{itemize}

\item \textit{Student Responsibility:} This syllabus outlines the required text, assignments, requirements, and policies for this course. By taking this course, you agree to read this syllabus and be bound by those requirements and policies. 

\item \textit{Late work \& Make-up Policy:} 
\begin{itemize}
\item All assignments must be submitted through Blackboard by 1:00 pm on the due date (see assignment schedule below).
\item  No make-ups or late work accepted under any circumstances. No exceptions. 
\item Blackboard difficulties are rare and automatically reported to instructors. Under no circumstance will a student's report of a Blackboard difficulty be reason for an extension. It is your responsibility to contact blackboard support for help: \href{dlsupport@njcu.edu}{dlsupport@njcu.edu}. 

\end{itemize}

\item \textit{Attendance:} You are considered absent if you are (i) not present during roll call, or (ii) leave early, or (iii) leave without permission, or (iv) leave for an extended period of time.

\item \textit{Electronic devices:} Use of electronic device, including, but not limited, to smartphones, dictaphones, tablets, laptops, is prohibited. Recording a lecture is in violation of Copyright. Penalties include, but are not limited to, a lost of attendance grade for the day of violation. Repeat offenders will be reported to the Dean of Students. 

\item \textit{Conduct:} Distracting and disrespectful behavior, including but not limited to eating, leaving your seat, talking out of turn, aggression is prohibited. Penalties include, but are not limited to, a lost of attendance grade for the day of violation. Repeat offenders will be reported to the Dean of Students. 

\item \textit{Communication:} All communication will be through Blackboard. Messages will be responded to within two days of receiving them. 

\item \textit{Grading Schedule:} Grades will be available within 1 week of an assignment being submitted.

\item \textit{Statement for students with disabilities:} If you are a student
with a disability and wish to receive consideration for reasonable
accommodations, please register with the Office of Specialized Services
and Supplemental Instruction (OSS/SI). To begin this process, complete
the registration form available on the OSS/SI website at
\href{http://www.njcu.edu/Specialized_Services.aspx}{www.njcu.edu/Specialized\_Services.aspx}
(listed under Student Resources-Forms). Contact OSS/SI at 201-200-2091
or visit the office in Karnoutsos Hall, Room 102 for additional
information.
\end{itemize}

\section{Plagiarism}

\begin{itemize} 
\item You are bound by \href{http://www.njcu.edu/uploadedFiles/About_NJCU/Governance_and_Organization/University_Senate/Policies/Academic\%20INTEGRITY\%20POLICY\%20FINAL\%202-04.pdf}{NJCU's Academic Integrity Policy}
\item Penalty for plagiarism:
\begin{itemize}
\item 1st infraction: 0 for the assignment. 
\item 2nd infraction: 0 for the entire course \& application for permanent record on student's transcript. (Repeated violations can lead to expulsion from NJCU). 
\end{itemize}
\end{itemize}

\section{Weekly Course Schedule}
Dates refer to the first day of the week: 
\begin{enumerate}
\item \textit{08/31} Introduction, no class on Mon. ch.1.
\item \textit{09/07} Continued, no class on Mon. ch.1
\item \textit{09/14} Judging Arguments, ch.3
\item \textit{09/21} Argument Patterns, ch.3
\item \textit{09/28} Deductive Reasoning, ch.6
\item \textit{10/05} Connectives \& Truth Values, ch.6
\item \textit{10/12} Truth Tables, ch.6
\item \textit{10/19} Mid-Term Review
\item \textit{10/26} Mid-term (Mon), Induction (wed) ch.8
\item \textit{11/02} Enumerative Induction, ch.8
\item \textit{11/09} Analogical and Causal Arguments, ch.8
\item \textit{11/16} Fallacies and Persuaders, ch.5
 \item \textit{11/23} Continued, ch.5
 \item \textit{11/30} Inference to the Best Explanation. ch.9
\item \textit{12/07} Continued,  ch.9
\item \textit{12/14} Exam Review
\end{enumerate}


\section{Assignment \& Exam Schedule}
Dates refer to the due date. All assignments must be submitted through Blackboard by 1:00pm. No late work accepted. No make-ups. No exceptions. Final exam will be determined by NJCU's centralized exam schedule.

\begin{enumerate}
\item \textit{09/11/2015,} Problem Set 1 (covers ch.1)
\item \textit{09/25/2015,} Problem Set 2 (covers ch. 3)
\item \textit{10/16/2015,} Problem Set 3 (covers ch. 6)
\item \textit{10/26/2015,} Mid-term 
\item \textit{11/13/2015,} Problem Set 4 (covers ch.8)
\item \textit{11/25/2015,} Problem Set 5 (covers ch.5)
\item \textit{12/11/2015,} Problem Set 6 (covers ch.9)
\item \textit{Exam Week.} Final exam date to be determined. 
\end{enumerate}


%% Uncomment if you want a printed bibliography.
%\printbibliography 

\end{document}