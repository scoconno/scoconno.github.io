\documentclass[11pt]{article}

\usepackage{fontspec}
\defaultfontfeatures{Mapping=tex-text}
\setmainfont[BoldFont={Minion Pro Bold}]{Minion Pro}

\usepackage[usenames,dvipsnames]{color}
\usepackage{graphicx}
\usepackage{fullpage}

\usepackage{hyperref}

\usepackage{lastpage, fancyhdr}
\pagestyle{fancy}
\lhead{}
\chead{Handout} 
\rhead{}
\lfoot{}
\cfoot{\thepage\space of \pageref{LastPage}} 
\rfoot{}
\footskip=30 pt
\headsep=20pt

\thispagestyle{empty}

\hypersetup{colorlinks=true, linkcolor=Sepia, urlcolor=Sepia, citecolor=BrickRed}

\usepackage{polyglossia}
\setdefaultlanguage{english}
\setotherlanguage{greek}
\newcommand{\gk}[1]{\textgreek{#1}}
\newcommand{\gloss}[1]{(\textgreek{#1})}

\begin{document}
\author{Phil 234}
\title{Plato's Euthyphro}
\maketitle
\thispagestyle{empty}


\section*{Plato: Life and Works}
\vspace*{-2mm}
\section*{Basic Structure of \emph{Euthyphro}}

\begin{itemize}\item{2a--4a: Socrates encounters Euthyphro, who is about to prosecute his (E's) father for murder}\item{4a--5d: S claims that only someone with expertise about piety, crucially including knowledge of its nature, should be confident that he were not acting impious in prosecuting a relative}\item{5d--15d: E offers various candidates for the definition of piety, all of which S rejects}\item{15d--end: S requests to start from the beginning, E refuses, leaves}\end{itemize}
\vspace*{-8mm}
\section*{Ethical action without knowledge}
\begin{itemize}

\item{[1] E's prosecution of his father is based on his belief that he has expertise concerning piety}

\item{[2] S seems to show that E does \emph{not} have that expertise}\begin{itemize}\item{What conception of expertise is at work here? Why does such expertise require knowledge of the nature of piety? Is it reasonable to demand such expertise? (for all actions? for some?) How does S test whether E has such expertise?}\end{itemize}

\item{[3] If E becomes aware of [2], how would it be rational for him to act?}

\begin{itemize}\item{By the end, S seems to think that E shouldn't prosecute his father. Why is this?}\end{itemize}
\end{itemize}
\vspace*{-8mm}
\section*{Socrates' method}

\begin{itemize}
\item{S gets E to offer a candidate, C, as a definition of piety}\item{S then elicits further claims that seem to entail that C is not, in fact, the definition of piety}\item{What do S's rejections of E's various candidates tell us about what S thinks would be a \emph{satisfactory} answer to the question ``What is Piety?''?}\end{itemize}
\vspace*{-8mm}
\section*{Candidates for the definition of piety (\emph{to hosion}) and Socrates' objections}
\begin{itemize}

\item{[1] 5e: Prosecuting the wrongdoer (regardless of personal relationship to wrongdoer)}

\begin{itemize}\item{This is an \emph{example} of a (kind of) pious action, not a specification of that in virtue of which all pious actions are pious}\end{itemize}

\item{[2] 7a: What is dear to the gods}

\begin{itemize}\item{If the gods disagree on important ethical matters (as Euth. agrees they do at 6c and 7b--8a), then one and the same thing could be both pious (because dear to some god(s)) and impious (because hated by some other god(s)), but that is impossible}\end{itemize}

\item{[3] (modification of [2]) 9e: What is dear to \emph{all} the gods}

\begin{itemize}\item{This only gives us a quality or affection of piety, it does not tell us what piety is}\end{itemize}

\item{[4] 12e: The part of justice concerned with the care of the gods}

\begin{itemize}\item{Care for \emph{X} aims at benefitting \emph{X} or making \emph{X} better; But gods cannot be made better}\end{itemize}

\item{[5] 13d: The part of justice concerned with service to the gods}

\begin{itemize}\item{Service aims at some goal (e.g. service to generals aims to help them win wars, service to house builders aims to help them build houses) but Euth. can't specify what ``fine thing'' gods achieve such that service could aim to help them achieve that goal}\end{itemize} 

\item{[6] (modification of [5]) 14d: [5] = Knowledge of how to sacrifice and pray}

\begin{itemize}\item{This definition reduces to [3] and the claim that sacrifices and prayers are dear to all the gods}\end{itemize}

\end{itemize}
\vspace*{-8mm}
\section*{Socrates' rejection of [3]: Piety = What is dear to all the gods (10a-11b)}
\begin{itemize}
\item{Socrates asks ``Is the pious loved by the gods because it's pious? Or is it pious because it's loved?''}
\item{Distinction between something's being \emph{X}ed and something's \emph{X}ing (e.g. something's being carried and something's carrying; being led and leading; being seen and seeing)}

\begin{itemize}\item{S claims that something \emph{X}ing \emph{Y} is prior to \emph{Y}'s being \emph{X}ed}
\begin{itemize}\item{E.g. If we ask ``Why is this piece of chalk being carried (led, seen)?'' the answer is ``Because Professor Schwab is carrying (leading, seeing) it''}\item{If we ask ``Why is Professor Schwab carrying this piece of chalk?'' the answer is \emph{not} ``Because the piece of chalk is being carried by Professor Schwab''}\end{itemize}\end{itemize}
\item{So, by analogy, something's being loved by the gods is \emph{posterior} to the gods' loving it (i.e. the fact that the gods love it explains why it is being loved, not the other way around)}

\item{If we ask, ``Why do the gods love it?'', the answer is, ``Because it is pious'' and \emph{not} ``Because it is loved by the gods'' (remember the carrying of the chalk: the answer to ``Why is Professor Schwab carrying the chalk'' was not ``Because the chalk is being carried by Professor Schwab'')}

\item{But, as we saw above, the answer to ``Why is it god-loved?'' is ``Because the Gods love it''} \item{But if ``To be pious'' and ``To be God-loved'' were the same thing (as E claims), then:}\begin{itemize}\item{If the pious were loved because it's pious, [by substituting identicals], the god-loved would be loved because it's god-loved}\item{But, we've shown that to be impossible}\end{itemize}\item{Therefore: To be pious $\neq$ To be dear to the gods}

\end{itemize}

\end{document}
