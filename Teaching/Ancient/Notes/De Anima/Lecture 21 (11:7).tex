% !TEX encoding = UTF-8 Unicode
% !TEX TS-program = xelatex

\documentclass[11pt]{article}
\usepackage{fontspec}
\defaultfontfeatures{Mapping=tex-text}
\usepackage{xunicode}
\usepackage{xltxtra}
\usepackage{verbatim}
\usepackage[margin= 1 in]{geometry} % see geometry.pdf on how to lay out the page. There's lots.
\geometry{letterpaper} % or letter or a5paper or ... etc
%\usepackage[parfill]{parskip}    % Activate to begin paragraphs with an empty line rather than an indent 
\usepackage{mathrsfs}
\usepackage{bbding}
\usepackage[usenames,dvipsnames]{color}
\usepackage{natbib}
\usepackage{stmaryrd}
%\usepackage{mathpartir}
\usepackage{txfonts}
\usepackage{graphicx}
\usepackage{fullpage}
\usepackage{hyperref}
\usepackage{amssymb}
\usepackage{epstopdf}
\usepackage{fontspec}
%\setmainfont{Hoefler Text}
\setmainfont[BoldFont={Minion Pro Bold}]{Minion Pro}
\usepackage{hyperref}
\usepackage{lastpage, fancyhdr}
%\usepackage{setspace}
\pagestyle{fancy}
\lhead{}
\chead{Lecture 21, Aristotle's \emph{De Anima} Book 2\space---\space Handout} 
\rhead{}
\lfoot{}
\cfoot{\thepage\space of \pageref{LastPage}} 
\rfoot{}
\footskip=30 pt
\headsep=20pt
\thispagestyle{empty}
\hypersetup{colorlinks=true, linkcolor=Sepia, urlcolor=Sepia, citecolor=BrickRed}
\DeclareGraphicsRule{.tif}{png}{.png}{`convert #1 `dirname #1`/`basename #1 .tif`.png}
\usepackage{polyglossia}
\setdefaultlanguage{english}
\setotherlanguage{greek}
\newfontfamily\greekfont{Gentium Plus}
\newcommand{\gk}[1]{\textgreek{#1}}
\newcommand{\gloss}[1]{(\textgreek{#1})}

\usepackage{covington}
\usepackage{fixltx2e}
\usepackage{graphicx}
\begin{document}

%\maketitle
\thispagestyle{empty}
\begin{center} \LARGE{PHIL 321\\ Lecture 21: Aristotle's \emph{De Anima}, 2.1-6, 11-12}\\ \vspace*{2mm}
\large{11/7/2013}\end{center}
\thispagestyle{empty}\vspace*{3mm}
\vspace*{-8mm}

\section*{Aristotle's potentiality / actuality distinction}

\noindent Central to A's metaphysics is a distinction between potential beings and actual beings
\vspace*{2mm}

For example: being potentially a house and a being actually a house; being potentially a dog and being \\\hspace*{6mm}actually a dog
\vspace*{2mm}

\noindent A does not think that facts about potential beings can be reduced to facts solely about actual beings
\vspace*{2mm}

For example: he does not think that the fact that a collection of bricks constitutes a potential house\\\hspace*{6mm}can be reduced solely to facts about bricks
\vspace*{2mm}

\noindent Rather, a being (or collection of beings) X is (are) a potential \emph{F} if and only if there is a single process such that, as a result of undergoing that process, X is actually \emph{F}
\vspace*{2mm}

For example: a collection of bricks constitutes a potential house because there is a single process, namely\\\hspace*{6mm}an exercise of the art of housebuilding, that it can undergo such that it becomes an actual house
\vspace*{2mm}

\noindent A's general account of soul employs a tiered notion of the potentiality / actuality distinction
\vspace*{2mm}

The example A gives is the difference between someone who can acquire knowledge / a person who has\\\hspace*{6mm}acquired knowledge / and a person who is employing that knowledge
\vspace*{2mm}

The first person is in a state of ``first potentiality'' / the second in a state of ``second potentiality'' and ``first\\\hspace*{6mm}actuality'' / the third in a state of ``second actuality''
\vspace*{-1mm}

\section*{The general account of soul}

\noindent In Bk. 2, Ch. 1 A offers a general definition that covers all souls
\vspace*{2mm}

Such a definition, while it will be true of all souls, will not be particularly informative, and so he also\\\hspace*{6mm}proceeds, in Chs. 4 and following, to discuss the various kinds of soul in detail 
\vspace*{2mm}

\noindent At the beginning of Ch. 1, A tells us that the soul is a particular kind of nature, namely the form of a living organism; thus, the relation of body--soul an instance of the more general form--matter relationship
\vspace*{2mm}

\noindent The soul is a second potentiality / first actuality of a certain kind of body
\vspace*{2mm}

\noindent The kind of body at question is ``organic'' (composed of organs), i.e. one's whose parts are capable of functioning in integrated ways
\vspace*{2mm}

For example, the leaves, roots, stem etc., of plants are capable of functioning in various integrated ways
\vspace*{2mm}

\noindent Thus, A's general account of soul: the first actuality of a natural body that is potentially alive
\vspace*{2mm}

\noindent In other words, the soul just is the integrated set of abilities the possession of which makes a living organism the kind of organism it is
\newpage

\noindent A thinks this account renders the question ``Are the soul and body one?'' easily answerable: of course they are different, since a set of abilities is not the same thing as a body; but, since the presence of soul makes the body the kind of body it is, you can't have that kind of body without it having that kind of soul

\section*{Homonymy}

\noindent A's general account of homonymy: A and B are ``homonyms'' iff the same name ``N'' applies to both A and B but the reason why it applies is different
\vspace*{2mm}

For example: the eye of a living organism and the eye of a statue are ``homonyms'' because the same name\\\hspace*{6mm}applies to them (i.e. ``eye'') but the reason why that name applies is different for each
\vspace*{1mm}

For the eye of a living organism, it applies because it has the power of sight; for the eye of the statue it\\\hspace*{6mm}applies because it ``looks like'' or ``resembles'' the eye of a living organism in shape (or maybe location)
\vspace*{2mm}

\noindent So, for A, a body that lacks the ability to perform the activities characteristic of an X (e.g. a dog, cat, human being) can only be the body of an X homonymously (so a corpse is only homonymously a ``human body'')

\section*{Integrated abilities}

\noindent The ability to perform the activities of nutrition can be had without the abilities to perceive and think (e.g. in plants); the ability to perceive can be had without the ability to think but \emph{not} without the ability to nourish (e.g. in non-human animals); the ability to think requires all the others (at least for earthly organisms)
\vspace*{2mm}

\noindent As noted above, A thinks that the general definition of soul is not that informative
\vspace*{2mm}

\noindent In part this is because the possession of higher-order abilities ``colors'' or ``affects'' the lower-order abilities of an organism
\vspace*{1mm}

For example: possessing the ability to think means that human beings' ability to nourish themselves is\\\hspace*{6mm}quite different from plants and non-human animals ability to nourish themselves
\vspace*{1mm}

This is part of the reason why A thinks that virtue is something that only human beings can possess; it\\\hspace*{6mm}isn't just that we possess the ability to think; its because our possessing that ability means we can perform\\\hspace*{6mm}the lower-order abilities in a manner not available to non-human animals

\section*{The soul as \emph{aitia}}

\noindent In Ch. 4 A says that the soul is the \emph{aitia} in three ways of the organism whose soul it is: as source of motion (efficient), what it is for (final), and substance (formal)
\vspace*{2mm}

\noindent This is controversial, but I do \emph{not} think that we should understand him to mean that the soul is the \emph{aitia} in these three senses of one and the same thing
\vspace*{2mm}

Efficient \emph{aitia} of the locomotion, growth, and alteration of the organism
\vspace*{1mm}

Formal \emph{aitia} of the organism (i.e. it makes the organism the kind of organism it is)
\vspace*{1mm}

Final \emph{aitia} of the organism's ``body'' (i.e. its parts are for the sake of performing those abilities)

\end{document}
