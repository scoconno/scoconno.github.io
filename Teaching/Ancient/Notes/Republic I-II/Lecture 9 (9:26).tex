% !TEX encoding = UTF-8 Unicode
% !TEX TS-program = xelatex

\documentclass[11pt]{article}
\usepackage{fontspec}
\defaultfontfeatures{Mapping=tex-text}
\usepackage{xunicode}
\usepackage{xltxtra}
\usepackage{verbatim}
\usepackage[margin= 1 in]{geometry} % see geometry.pdf on how to lay out the page. There's lots.
\geometry{letterpaper} % or letter or a5paper or ... etc
%\usepackage[parfill]{parskip}    % Activate to begin paragraphs with an empty line rather than an indent 
\usepackage{mathrsfs}
\usepackage{bbding}
\usepackage[usenames,dvipsnames]{color}
\usepackage{natbib}
\usepackage{stmaryrd}
%\usepackage{mathpartir}
\usepackage{txfonts}
\usepackage{graphicx}
\usepackage{fullpage}
\usepackage{hyperref}
\usepackage{amssymb}
\usepackage{epstopdf}
\usepackage{fontspec}
%\setmainfont{Hoefler Text}
\setmainfont[BoldFont={Minion Pro Bold}]{Minion Pro}
\usepackage{hyperref}
\usepackage{lastpage, fancyhdr}
%\usepackage{setspace}
\pagestyle{fancy}
\lhead{}
\chead{Lecture 9, Plato's \emph{Republic}, Books I and II\space---\space Handout} 
\rhead{}
\lfoot{}
\cfoot{\thepage\space of \pageref{LastPage}} 
\rfoot{}
\footskip=30 pt
\headsep=20pt
\thispagestyle{empty}
\hypersetup{colorlinks=true, linkcolor=Sepia, urlcolor=Sepia, citecolor=BrickRed}
\DeclareGraphicsRule{.tif}{png}{.png}{`convert #1 `dirname #1`/`basename #1 .tif`.png}
\usepackage{polyglossia}
\setdefaultlanguage{english}
\setotherlanguage{greek}

\usepackage{covington}
\usepackage{fixltx2e}
\usepackage{graphicx}
\begin{document}

%\maketitle
\thispagestyle{empty}
\begin{center} \LARGE{PHIL 321\\ Lecture 9: Plato's \emph{Republic}, Books I and II}\\ \vspace*{2mm}
\end{center}
\thispagestyle{empty}\vspace*{3mm}
\vspace*{-8mm}

\section*{Plato: Life and Works}
\begin{itemize}

\item{Born in 428 BCE to an aristocratic and politically powerful Athenian family; so would have been expected to go into politics; In one of his letters he talks about how he was tempted to partake in public life, but the events surrounding the end of the Peloponnesian war, and in particular the treatment of Socrates, led him to believe that there you could not better Athenian society (or any actually existing society) ``from the inside''; so, he embarked on a philosophical life; but, one thing to bear in mind is that, if his autobiography is correct, he had a very practical aim; he did want to make society better}\item{In his youth, he fell in with Socrates. He was, in fact, so enamored of Socrates that he uses him as the main character in many of his dialogues}\item{Socrates himself seems to have written nothing, so his influence in the later tradition was through his personal interactions with people}\item{Plato seems to have left Athens upon the execution of Socrates in 399 BCE; returned and founded his Academy in around 387 BCE}\item{At the academy he both wrote his own philosophical dialogues and oversaw various research programs in philosophy, mathematics, and natural science}\item{He also seems to have taken trips to Sicily to try and bring his vision of the Philosopher-King into fruition, but that was ultimately a failure}\item{Died in 348 or 347 BCE in Athens}\item{His main philosophical form was the dialogue, in which various characters engage in philosophical discussion; we must consider why Plato wrote in this manner, and how this affects our understanding of his work and his philosophy}\item{His work is conventionally divided into three or four ``groups'', often chronological; but we must note that it is difficult to be certain about the chronology:}\begin{itemize}\item{``Socratic'' (sometimes called ``early'')--Socrates is the main character; some moral concept is investigated (e.g. piety, courage, temperance, friendship; often ends in \emph{aporia} or failure to come to a decisive conclusion}\item{``Middle'': the \emph{Republic}--here we get more positive theses stated and argued for, loses some of the aporetic nature of the Socratic dialogues}\item{Some people identify a distinct group of ``transitional'' (e.g. the \emph{Meno}) which they don't want to classify as Socratic nor as Middle}\item{Late}\end{itemize}
\item{\textbf{Stephanus}}
\end{itemize}

\section*{Book 1: Conventional views of justice}

\noindent Cephalus: Justice is telling the truth and returning what one owes (331)
\vspace*{2mm}

\noindent Polemarchus: Justice is doing good to one's friends and harm to one's enemies (332)
\vspace*{2mm}

\noindent Thrasymachus: Justice is the advantage of the stronger (338c)
\begin{itemize}\item{[A] ``the stronger'' = the established power in a community, or ruler}\item{[B] Being just = serving the good of another person or class, as defined in [A]}\begin{itemize}\item{As such, T claims that justice is actually harmful to oneself}\end{itemize}\item{[C] Being unjust = serving one's self, \emph{either} as a citizen by not performing [B] actions, \emph{or} as a ruler or tyrant by setting up the standards of justice for others, but ignoring them oneself (343-44)}\end{itemize}

\noindent Justice is, in fact, a vice---i.e. it is unprofitable to the agent to be just; injustice is, in fact, a virtue---i.e. it is profitable to the agent to be unjust (348)

\begin{itemize}\item{Remember the background notion that happiness (\emph{eudaimonia}) is the goal in life and that virtue is supposed to contribute to/ensure the attainment of that goal}\end{itemize}

\noindent The perfectly unjust person would, if he/she were clever, courageous, etc. be happy
\vspace*{2mm}

\noindent The reaction of S and the other interlocutors indicates that T's claims go against the generally received view, but are there elements of his claims that are widely accepted?

\section*{Division of goods (Book II, 357)}

\noindent\underline{Kind of good}\hspace*{55mm}\underline{Examples}
\vspace*{2mm}

\noindent [1] Good for own sake only\hspace*{32mm} Joy, harmless pleasures

\noindent [2] Good for own sake and consequences\hspace*{11mm} Knowing, seeing, being healthy

\noindent [3] Good for consequences only\hspace*{26mm}Physical training, medical treatment, medicine,\\\hspace*{75mm}making money
\vspace*{2mm}

\noindent One of the central questions of the dialogue, ``What kind of good is justice?''
\begin{itemize}\item{None of these senses = T's view in Book I}\item{Sense [3] = As Glaucon argues in Book II (although note that G says he himself isn't persuaded by the argument but wants to hear S's response to it (358c6))}\item{Sense [2] = Socrates (but note that in the ensuing discussion he focuses on arguing that justice is good for its own sake, irrespective of its consequences)}\end{itemize}
\vspace*{-6mm}

\section*{Glaucon}

\noindent Origin of justice: A contract produces laws that are mutually beneficial (358e-59b)

\begin{itemize}
\item{Doing injustice is naturally good \& suffering injustice is naturally bad (in which sense of good?)}
\item{Justification of contract: I act justly\hspace*{5mm} I act unjustly\\\hspace*{14mm} You act justly\hspace*{8mm}2\hspace*{20mm} 1\\\hspace*{14mm}You act unjustly\hspace*{4mm} 1\hspace*{21mm}3}\begin{itemize}\item{The situation in which we both act unjustly is worse than when we both act justly because the cost of suffering injustice outweighs the benefits of acting unjustly oneself}\end{itemize}
\end{itemize}

\noindent Justice is practiced unwillingly---i.e. is only good in sense [3] (or, at least, people \emph{think} it's good in sense [3])
\begin{itemize}\item{Proof: Gyges' ring---supposed to show that if the consequences of justice were attainable for an agent without actually being just, and if the outcome of this were better than the outcome when the agent is just, the agent would have better justification for acting unjustly than justly}\begin{itemize}\item{Two assumptions: i) the consequences of others' acting justly are beneficial\\\hspace*{28mm}ii) happiness is an independent criterion by which we can judge which outcome\\\hspace*{32mm}is better for the agent}\end{itemize}\end{itemize}

\noindent Injustice is in fact better than justice for the agent

\begin{itemize}\item{G argues that a comparison of justice without its consequences and of injustice without its consequences would show not just that injustice is better justified, but that justice without its consequences has no value---i.e. doesn't contribute to happiness and so is not a good in sense [1] or [2]}\end{itemize}
\vspace*{-8mm}

\section*{Socrates}

\noindent S wants to show that justice is good for its own sake, even without its consequences
\vspace*{2mm}

\noindent He also wants to show that justice is better than any other combination of goods without it
\begin{itemize}\item{Are either of these necessary to support a reasonable theory of justice?}\end{itemize}

\noindent S sets out to argue that:
\begin{itemize}\item{[X] Inter-personal justice is strictly analogous to intra-personal justice. So, by examining the nature of political justice, S thinks he can show:}\item{[Y] Intra-personal justice is a state of character which is both intrinsically good and a necessary condition for any kind of ordered life, and hence for happiness, AND}\item{[Z] Intra-personal justice is a dominant constituent of happiness itself, and hence happiness cannot be specified or achieved independently of it, and justice is better for the agent than any combination of goods without it}\end{itemize}

\noindent [X], [Y], and [Z] are problematic. Is it reasonable to \emph{define} happiness by reference to the virtues? What relation is there between justice in a city and an individual? Why think that ``psychic justice'' would produce recognizably just acts?
\end{document}
