% !TEX encoding = UTF-8 Unicode
% !TEX TS-program = xelatex

\documentclass[11pt]{article}
\usepackage{fontspec}
\defaultfontfeatures{Mapping=tex-text}
\usepackage{xunicode}
\usepackage{xltxtra}
\usepackage{verbatim}
\usepackage[margin= 1 in]{geometry} % see geometry.pdf on how to lay out the page. There's lots.
\geometry{letterpaper} % or letter or a5paper or ... etc
%\usepackage[parfill]{parskip}    % Activate to begin paragraphs with an empty line rather than an indent 
\usepackage{mathrsfs}
\usepackage{bbding}
\usepackage[usenames,dvipsnames]{color}
\usepackage{natbib}
\usepackage{stmaryrd}
%\usepackage{mathpartir}
\usepackage{txfonts}
\usepackage{graphicx}
\usepackage{fullpage}
\usepackage{hyperref}
\usepackage{amssymb}
\usepackage{epstopdf}
\usepackage{fontspec}
%\setmainfont{Hoefler Text}
\setmainfont[BoldFont={Minion Pro Bold}]{Minion Pro}
\usepackage{hyperref}
\usepackage{lastpage, fancyhdr}
%\usepackage{setspace}
\pagestyle{fancy}
\lhead{}
\chead{Lecture 10, Plato's \emph{Republic}, Books III and IV\space---\space Handout} 
\rhead{}
\lfoot{}
\cfoot{\thepage\space of \pageref{LastPage}} 
\rfoot{}
\footskip=30 pt
\headsep=20pt
\thispagestyle{empty}
\hypersetup{colorlinks=true, linkcolor=Sepia, urlcolor=Sepia, citecolor=BrickRed}
\DeclareGraphicsRule{.tif}{png}{.png}{`convert #1 `dirname #1`/`basename #1 .tif`.png}
\usepackage{polyglossia}
\setdefaultlanguage{english}
\setotherlanguage{greek}
\newfontfamily\greekfont{Gentium Plus}
\newcommand{\gk}[1]{\textgreek{#1}}
\newcommand{\gloss}[1]{(\textgreek{#1})}

\usepackage{covington}
\usepackage{fixltx2e}
\usepackage{graphicx}
\begin{document}

%\maketitle
\thispagestyle{empty}
\begin{center} \LARGE{PHIL 321\\ Lecture 10: Plato's \emph{Republic}, Books III and IV}\\ \vspace*{2mm}
\large{10/1/2013}\end{center}
\thispagestyle{empty}\vspace*{3mm}
\vspace*{-8mm}

\section*{For board}

\noindent How does Socrates think the ``power of appearance'' leads people to make mistaken judgments about what is best overall?
\vspace*{2mm}

\noindent ``You realize that most people aren't going to be convinced by us. They are going to say that most people are unwilling to do what is best, even though they know what it is and are able to do it. And when I have asked them the reason for this, they say that those who act that way do so because they are overcome by pleasure or pain or are being ruled by one of the things I referred to just now.''

\section*{Context of the tri-partition argument}

\noindent The city has three ``kinds'' or ``classes'' of people in it: workers (money-lovers), guardians (honor-lovers), and rulers (wisdom-lovers). They have the distinct functions of: producing, guarding, and ruling respectively.
\vspace*{2mm}

\noindent Some of the city's virtues are ``located'' in its parts: \emph{wisdom} in rulers, \emph{courage} in guardians. Some consist in a certain relation ``between'' its parts: \emph{temperance} is a certain concord between workers and rulers. \emph{Justice} consists in \emph{each part }performing its proper function.
\vspace*{2mm}

\noindent If the city is a good model for the individual, both should have three ``parts'' with distinct functions.
\vspace*{-3mm}

\section*{The three kinds of desires or faculties in the \emph{Republic}}

\noindent Appetite: Desires for food, drink, sex, etc. These are ``physiological'' desires (439)
\vspace*{.5mm}

\noindent Spirit: Emotions of anger, self-disgust, shame and desires for honor or respect (440-41)
\vspace*{.5mm}

\noindent Reason: Rational desires are for the overall good or good of the whole (441e)
\vspace*{2mm}

\noindent Either we always ``go for'' things with part of the soul \emph{or} the whole soul---e.g. we learn with one part of it, get angry with another, and desire pleasures of food with a third (436a).
\vspace*{2mm}

\noindent\textbf{SOC TAKES IT AS OBVIOUS THAT HUMANS HAVE THESE THREE KINDS OF DESIRE IN THE SOUL. HE POINTS TO THE VARIOUS GENERAL TENDENCIES AMONG DIFFERENT SOCIETIES AND ASKS WHERE THOSE COULD COME FROM IF NOT FROM THE PEOPLE LIVING WITHIN THEM (E.G. SCYTHIANS ARE SPIRITED; GREEKS LOVE WISDOM; PHOENICIANS AND EGYPTIANS LOVE MONEY)}
\vspace*{2mm}

\noindent\textbf{SOC'S QUESTION IS ABOUT THE PROPER SUBJECT OF THESE THREE MOTIVATING CONDITIONS}
\vspace*{-3mm}

\section*{Argument for one non-rational part (i.e. the appetitive part)}

\noindent [P1] \textbf{Principle of Opposites}: A thing cannot undergo opposites in the same part of itself, in relation to the same thing, at the same time (436b-37a)
\vspace*{1mm}

\noindent\textbf{TO CLARIFY: SOC ASKS WHETHER SOMETHING CAN STAND STILL AND MOVE AT THE SAME TIME IN THE SAME PART OF ITSELF--E.G. A PERSON STANDING STILL BUT MOVING HIS HANDS; THIS MAY SEEM LIKE ONE AND THE SAME THING MOVING AND STANDING STILL, BUT ONE PART IS MOVING AND ANOTHER PART IS STANDING STILL}
\vspace*{1mm}

\noindent\textbf{ALSO CLARIFY: THE CASE OF THE TOP: HERE IT SEEMS THAT THE WHOLE TOP IS MOVING AND STANDING STILL: BUT IT'S MOVING IN RESPECT TO IT'S VERTICAL AXIS IT'S STILL AND HORIZONTAL ITS MOVING}
\vspace*{1mm}

\noindent [P2] Going (assent, wishing) for X and rejecting (dissent, not wishing for) X, are opposites
\vspace*{1mm}

\noindent [P3] The desire for a drink is an unqualified desire $\neq$ the qualified desire for a good drink
\vspace*{1mm}

[P3.i] Thirst \emph{as such} is a desire for drink \emph{as such} (437b-39a)
\vspace*{1mm}

[P3.ii] Unqualified desires are for unqualified objects
\vspace*{1mm}

\noindent\textbf{THIS IS THE ANTI-PROTAGOREAN POINT; ON THAT VIEW, ALL DESIRES ARE FOR THE GOOD. HERE SOC IS CLAIMING THAT THERE ARE SOME DESIRES THAT ARE NOT FOR SOMETHING QUA GOOD BUT JUST FOR THAT THING. HE IS IN PART DOING THIS TO FORESTALL THE WORRY THAT AN APPARENT CONFLICT BETWEEN REASON AND APPETITE IS ACTUALLY JUST DIFFERENT RESPECTS OF THE OBJECT: YOU DESIRE TO DRINK IT FOR THE GOOD YOU WILL GET FROM SLAKING YOUR THIRST, BUT YOU DON'T DESIRE TO DRINK IT FOR THE BAD THAT THE POISON WILL CAUSE}
\vspace*{1mm}

\noindent [P4] Sometimes we have a desire for drink, but choose not to drink
\vspace*{1mm}

\noindent\underline{[P5] Having this desire to drink and not wanting to drink are opposites}
\vspace*{1mm}

\noindent [C] So there are two ``things'' in the person involved in this event---one thing is the proper subject of the desire to drink, a \emph{distinct} thing is the proper subject of the rejection of drink
\vspace*{1mm}

\noindent The former is, properly speaking, a motivation generated by appetite, the latter by reason
\vspace*{-3mm}

\section*{Arguments for a second non-rational part (i.e. the spirited part)}

\noindent\textbf{[1] Spirit is distinct from appetite (439e-40)}
\vspace*{1mm}

\hspace*{3mm} Evidence: Leontius---wants to look at corpses and does, but is angry with himself
\vspace*{1mm}

\noindent\textbf{[2] Spirit is distinct from reason (440e-41)}
\vspace*{1mm}

\hspace*{3mm} Evidence 1: Children and animals do not have reason but do act contrary to their appetites
\vspace*{.5mm}

\hspace*{3mm} Evidence 2: Odysseus---wants to take vengeance on his unfaithful servants, but is restrained by reason
\vspace*{2mm}

\noindent\textbf{[3] Spirit is ``allied'' with reason (440)}
\vspace*{1mm}

\hspace*{3mm} Evidence: The cases of a good person undergoing just and unjust punishment

\section*{\emph{Akrasia} (lack of self-control) on this theory}

\noindent S here seems to allow, against the view presented in the \emph{Protagoras}, that lack of self-control is possible
\vspace*{2mm}

\noindent X does B, i) \emph{thinking} B is bad, ii) when able not to do B, and iii) overwhelmed by pleasure

\begin{itemize}\item{In the \emph{Protagoras} S maintained that in such cases it is actually a judgment that B is the best course of action that causes X to do B (so ``being overcome by passion'' does not, in fact, occur)}\item{In the \emph{Republic}, S maintains that in such a case the judgment of the rational part of the soul (i.e. a rational desire) is overcome by the non-rational appetitive desire of the soul}\end{itemize}

\noindent Question: On the \emph{Republic}'s theory, is \emph{akrasia} possible when X knows that B is bad, or only when X \emph{merely believes} that B is bad?

\section*{Justice}

\noindent Given the tri-partition of the soul, and the city-soul analogy, S maintains that justice is the harmonious functioning of the three parts of the soul, with each part fulfilling its proper ``function'' or ``work'': 

\begin{itemize}\item{Reason rules, making judgments about the overall good for the person}\item{Spirit and appetite ``obey'' reason in the sense that they only desire things that, in fact, accord with reason's determinations about what is good overall}\begin{itemize}\item{Spirit and appetite, however, \emph{do not} do this because they can judge what is best, or because they can judge that reason ``knows best,'' or because they can judge anything at all (they are \emph{non-rational}---they cannot grasp reasons)}\item{They do it because they have been trained or conditioned only to generate such desires}\end{itemize}\end{itemize}

\noindent So, only the just person's soul is genuinely unified, with no internal conflicts. This is why G agrees that it is always better to be just than unjust (although S himself says that the argument is not done)

\section*{Problems}

\noindent [P3] makes a logical point about qualified and unqualified things: how could this show that we actually have non-rational desires?
\vspace*{2mm}

\noindent It may be possible to re-describe [P4] along the lines of the view presented by Socrates in the \emph{Protagoras} as conflict between non-simultaneous rational beliefs
\vspace*{2mm}

\noindent Does [P1] entail indefinite partition of the soul if two desires of a single soul part can conflict with each other?
\vspace*{1mm}

\noindent What does justice, in the sense described above, have to do with justice as G \& A were discussing it at the beginning of Book II? Has S changed the subject?

\end{document}
