\documentclass[article,oneside]{memoir}

%%% custom style file with standard settings for xelatex and biblatex. Note that when [minion] is present, we assume you have minion pro installed for use with pdflatex.
%\usepackage[minion]{org-preamble-pdflatex} 

%%% alternatively, use xelatex instead
\usepackage{org-preamble-xelatex} 



\def\myauthor{Author}
\def\mytitle{Title}
\def\mycopyright{\myauthor}
\def\mykeywords{}
\def\mybibliostyle{plain}
\def\mybibliocommand{}
\def\mysubtitle{}
\def\myaffiliation{NJCU}
\def\myaddress{Phil 234}
\def\myemail{soconnor@njcu.edu}
\def\myweb{\href{http://scoconno.github.io/Teaching/Ancient}{http://scoconno.github.io/Teaching/Ancient}}
\def\myphone{}
\def\myversion{}
\def\myrevision{}
\def\myaffiliation{NJCU}
\def\myauthor{Dr. Scott O'Connor}
\def\mykeywords{}
\def\mysubtitle{Syllabus}
\def\mytitle{{\normalsize \myweb  \newline} \HUGE Ancient Philosophy}


\begin{document}

%%% If using xelatex and not pdflatex
%%% xelatex font choices
\defaultfontfeatures{}
\defaultfontfeatures{Scale=MatchLowercase}    
% You will need to buy these fonts, change the names to fonts you own, or comment out if not using xelatex.      
\setromanfont[Mapping=tex-text]{Minion Pro} 
\setsansfont[Mapping=tex-text]{Myriad Pro} 
\setmonofont[Mapping=tex-text,Scale=0.8]{Minion Pro} 

%% blank label items; hanging bibs for text
%% Custom hanging indent for vita items
\def\ind{\hangindent=1 true cm\hangafter=1 \noindent}
\def\labelitemi{$\cdot$}
%\renewcommand{\labelitemii}{~}

%% RCS info string for version tracking
\chapterstyle{article-3}  % alternative styles are defined in latex-custom-kjh/needs-memoir/
%\pagestyle{kjh}

\title{\LARGE \mytitle}     
\author{\Large\myauthor \newline \footnotesize\texttt{\noindent Office hours: \href{http://scoconno.github.io/Contact/Office/}{http://scoconno.github.io/Contact/Office/}}}
\date{1/19/2016--5/9/2016}

\published{\textbf{Phil 234-1 (2693), 3 credits, Spring 2016, M\&W 16:00--17:15, H220}}


\maketitle

%\thispagestyle{kjhgit}

% Copyright Page
%\textcopyright{} \mycopyright


%
% Main Content
%

\section{Copyright}
The materials used in this class, including, but not limited to, lectures, exams, quizzes, and homework assignments are copyright protected works.  Any unauthorized copying of the class materials or recording of lectures is a violation of federal law and may result in disciplinary actions being taken against the student.  Additionally, the sharing of class materials without the specific, express approval of the instructor may be a violation of the University's Student Honor Code and an act of academic dishonesty, which could result in further disciplinary action.  This includes, among other things, uploading class materials to websites for the purpose of sharing those materials with other current or future students. 

\section{Catalog Description}

What do you know? Do you know anything? What exists? Are there objective truths about what’s right and wrong for you to do, or is it all a matter of convention? Does being a moral person go against your self-interest? If so, why should you be a moral person? What is happiness? Will being a moral person contribute to your happiness? These questions were raised by philosophers speaking and writing in Greek over two millennia ago. In this course, we will think hard about these questions and try to identify how they were answered by three of the most influential philosophers of all time---Socrates, Plato, Aristotle, and the Hellenestics. 

\section{Learning Objectives}

Upon completing this course students will be able to (i) read
philosophical texts, (ii) clearly and charitably explain viewpoints that
are not their own, (iii) think critically and philosophically, (iv)
write well-structured prose in which they clearly state a thesis and
persuasively defend it, (v) demonstrate an understanding of the
philosophies of Socrates, Plato, Aristotle, and the Hellenistics.


\section{Readings}
\subsection{Required}
Available in the campus book store and online retailers. Please bring the readings to class.
\begin{itemize}
\item \href{http://www.amazon.com/Plato-Dialogues-Euthyphro-Apology-Classics/dp/0872206335/ref=sr_1_1?ie=UTF8&qid=1452099006&sr=8-1&keywords=plato+five+dialogues}{`Plato: Five Dialogues', Hackett Classics, 2nd Edition}
\item \href{http://www.amazon.com/Hellenistic-Philosophy-Hackett-Classics-Inwood/dp/0872203786/ref=sr_1_1?ie=UTF8&qid=1452099186&sr=8-1&keywords=hellenistic+philosophy}{`Hellenistic Philosophy', Hackett Classics, 2nd Edition}
\item \href{http://www.amazon.com/Aristotle-Introductory-Readings-Hackett-Classics/dp/0872203395/ref=sr_1_1?ie=UTF8&qid=1452102830&sr=8-1&keywords=aristotle+hackett}{`Aristotle: Introductory Readings', Hackett Classics}
\end{itemize}
\subsection{Optional}
Students who wish to improve their writing might wish to purchase and work through the 10 lessons contained in the following: 
\begin{itemize}
\item \href{http://www.amazon.com/Style-Lessons-Clarity-Grace-11th/dp/0321898680/ref=sr_1_1?ie=UTF8&qid=1452356026&sr=8-1&keywords=lessons+in+clarity+and+grace}{`Style: Lessons in Clarity and Grace', Joseph Williams and Joseph Bizup}
\end{itemize}
\section{Course Website}
There is both a Blackboard site and website for this course (link on first page). Clicking the first link on the left panel within the Blackboard site will bring you to the course website. All assignments will be submitted through Blackboard. Readings, notes, etc. will be posted on the course website. Note that Blackboard difficulties are rare and automatically reported to instructors. Under no circumstance will a student's report of a Blackboard difficulty be reason for an extension. It is your responsibility to contact Blackboard support for help.


\section{Requirements}

\begin{itemize}
\item \textit{Workload:} Expect to spend an average of 5--6 hours per week  completing the readings and assignments.

\item \textit{Course evaluations} completed online. 5 points extra credit for successful completion.

\item \textit{Attendance:} Roll call will be taken. 0.5 point will be awarded per class up to a maximum of 10 points. Points will not be awarded during weeks 1 \& 2. No extra credit will be offered to make up missed attendance.
\item \textit{Reading:} You are required to prepare the assigned readings before class. This will be determined through discussion, pop quizzes, and reading responses. You begin with 10 points. You lose 2 points for every class you come to unprepared. No extra credit will be offered to make up lost points in this category. 

\item \textit{Short essays} submitted through Blackboard. 400--600 words long. 5 will be assigned. You must complete 3. 1 must be completed before the mid-term break. If you complete more than 3, the lowest grades will be dropped.

 
\item \textit{Long essays} submitted through Blackboard. 1200--1500 words long. 3 will be assigned. You must complete 2. If you complete more than 2, the lowest grade will be dropped.


\item \textit{Grade Distribution:} Attendance--0.5 point per class (10 total); Reading--10 points; 3 short essays---10 points each (30 total); 2 long essays--25 points each (50 total).

\item \textit{Grade Breakdown:}

 \begin{tabular}{ | l | l | p{2cm} | l | l | }
    \hline 
96--100 & A  & &  77--79 &  C+ \\  
90--95 & A- & &  73--76 & C \\
87-89 & B+ &  &  70--72 & C- \\ 
83--86 & B  & &  60--69 & D\\
80--82 & B - & & 0--59 & F\\ \hline
    \end{tabular}


\end{itemize}





\section{Policies}

\begin{itemize}

\item \textbf{Student Responsibility:} This syllabus outlines the required text, assignments, requirements, and policies for this course. By taking this course, you agree to read this syllabus and be bound by those requirements and policies. 

 \item \textit{Academic Integrity:} All the work you turn in (including papers, drafts, and discussion board posts) must be written by you specifically for this course. It must originate with you in form and content with all contributory sources fully and specifically acknowledged. Being a student at NJCU requires you to follow \href{http://www.njcu.edu/uploadedFiles/About_NJCU/Governance_and_Organization/University_Senate/Policies/Academic\%20INTEGRITY\%20POLICY\%20FINAL\%202-04.pdf}{NJCU's Academic Integrity Policy.} Penalties for violations are as follows: 1st infraction will result in a 0 for the assignment.  2nd infraction will result in a 0 for the entire course \& application for permanent record on student's transcript. (Repeated violations can lead to expulsion from NJCU). 

\item \textit{Attendance:} You are considered absent if you are (i) not present during roll call, (ii) leave early, (iii) leave without permission, or (iv) leave for an extended period of time. No excuses. No exceptions.



\item \textit{Communication:} To comply with Federal Privacy Laws (FERPA) and NJCU policies, all communication will be through Blackboard and/or official NJCU e-mail. Check both your NJCU e-mail and Blackboard daily. For further information see \href{http://scoconno.github.io/Contact/}{http://scoconno.github.io/Contact/}.

\item \textit{Conduct:} Distracting and disrespectful behaviors, including but not limited to eating, leaving your seat, talking out of turn, and aggression are prohibited. Penalties include, but are not limited to, a loss of attendance points for the day of violation. Repeat offenders will be reported to the Dean of Students. 

\item \textit{Electronic devices:} Use of electronic device, including, but not limited, to smartphones, dictaphones, tablets, and laptops, is prohibited. Recording a lecture is in violation of Copyright. Penalties include, but are not limited to, a loss of attendance points for the day of violation. Repeat offenders will be reported to the Dean of Students.

\item \textit{Format for Written Work:} Submit work to Blackboard either as a rich text or Microsoft Word file. All work must be typed. Your papers should be in 12-point Times New Roman font, double-spaced with margins set to one inch on all sides. If hard copies are requested, please staple or paperclip copies of papers and drafts.



\item \textit{Grading:} Grades will be available within 1--2 weeks of an assignment being submitted. See: \href{http://scoconno.github.io/Teaching/Grading}{http://scoconno.github.io/Teaching/Grading} for further information.


\item \textit{Late work \& Make-up Policy:} See the assignment schedule below. No late work accepted under any circumstances. No exceptions. If you miss a short or long essay, you can complete an additional one. 

\item \textit{Statement for students with disabilities:} If you are a student with a disability and wish to receive consideration for reasonable accommodations, please register with the Office of Specialized Services and Supplemental Instruction (OSS/SI). To begin this process, complete the registration form available on the OSS/SI website at
\href{http://www.njcu.edu/Specialized_Services.aspx}{www.njcu.edu/Specialized\_Services.aspx}
(listed under Student Resources-Forms). Contact OSS/SI at 201-200-2091
or visit the office in Karnoutsos Hall, Room 102 for additional
information.

\end{itemize}



\section{Weekly Course Schedule}
Dates refer to the first day of the week. Complete the readings before the first class of the week. Readings marked with a `**' can be found on the course website. Handouts can be found under the relevant modules on the course website. All other listed readings can be found in the required textbooks. Changes to the syllabus will be announced in class and \emph{via} your NJCU email address.


\begin{description}
\item[Module 0:] {Introduction}
\begin{enumerate}
\item[\textit{Week 1}] Introduction 
\end{enumerate}

\item[Module 1:] Socrates, readings from `Five Dialogues'
\begin{enumerate}
\item[\textit{Week 2}] Socrates in action
\begin{enumerate}
\item `Socrate', 1st half, directed by Rossalini (in class)
\item `Euthyphro'
\end{enumerate}

\item[ \textit{Week 3}] Is Socrates a bad influence? 
\begin{enumerate}
\item `Meno'
\end{enumerate}

\item[ \textit{Week 4}] Trial \& Defense
\begin{enumerate}
\item `Apology'
\end{enumerate}

\item[ \textit{Week 5}] Should Socrates escape? (no class on Mon)
\begin{enumerate}
\item `Crito'
\end{enumerate}
\item[ \textit{Week 6}] Socrates' death? 
\begin{enumerate}
\item `Phaedo' (selections)
\item `Socrate', 2nd half (in class) 
\end{enumerate}
\end{enumerate}


\item[Module 2:] Plato, readings online
\begin{enumerate}
\item[\textit{Week 7}] `The Republic'
\begin{enumerate}
\item \emph{Mon.,} Glaucon's Challenge, `Republic', Book 2 (up until 368c) 
\item \emph{Wed.,} Justice in the City, TBD
\end{enumerate}
\item[\textbf{Spring Break}] \

\item[\textit{Week 8}] `The Republic'
\begin{enumerate}
\item \emph{Mon.,} Justice in the Soul, `Republic', TBD %434d-445e
\item \emph{Wed..} Philosopher Kings, `Republic', TBD
\end{enumerate}
\end{enumerate}

\item[Module 3:] Aristotle, readings from `Aristotle: Introductory Readings'
\begin{enumerate}
\item[\textit{Week 9}]  The Good Life
\begin{enumerate}
\item Nicomachean Ethics', Book 1
\item Nicomachean Ethics', Book 2.1-6
\item Nicomachean Ethics', Book 3.1-5 

\end{enumerate}
\item[\textit{Week 10}] The Best Life
\begin{enumerate}
\item Nicomachean Ethics', Book 2.1-6
\item 'Nicomachean Ethics', Book 3.1-5, 
\item 'Nicomachean Ethics', Book 10.6-9

\end{enumerate}

\item[\textit{Week 11}] Natural Philosophy
\begin{enumerate}
\item `Physics', Book 2
\end{enumerate}
\item[\textit{Week 12}] The Soul
\begin{enumerate}
\item `De Anima', Book 2, Ch.1--6, 11--12
\end{enumerate}
\end{enumerate}

\item[Module 4:] Hellenistic Philosophy
\begin{enumerate}
\item[ \textit{Week 13}] Epicurus \& The Stoics
\begin{enumerate}
\item TBD
\end{enumerate}
\item[ \textit{Week 14}] The Stoics \& The Skeptics
\begin{enumerate}
\item TBD
\end{enumerate}
\item[\textit{Week 15}] The Skeptics
\begin{enumerate}
 \item TBD
\end{enumerate}

\end{enumerate}
\end{description}

\section{ Assignment Schedule}

Dates refer to the due date. All assignments must be submitted through Blackboard by 1:00pm. No late work accepted. No exceptions. You must complete 3 short essays and 2 long essays. If you complete more than the required number, the lowest grades will be dropped.  
\begin{enumerate}
\item \textit{02/01/2016,} Short essay 1
\item \textit{02/15/2016,} Short essay 2
\item \textit{02/29/2016,} Long essay 1
\item \textit{03/14/2016,} Short essay 3
\item \textit{03/28/2016,} Short essay 4
\item \textit{04/11/2016,} Long essay 2
\item \textit{04/25/2016,} Short essay 5
\item \textit{5/09/2016,} Long essay 3
\end{enumerate}



%% Uncomment if you want a printed bibliography.
%\printbibliography 

\end{document}
