% !TEX encoding = UTF-8 Unicode
% !TEX TS-program = xelatex

\documentclass[11pt]{article}
\usepackage{polyglossia}
\usepackage{amssymb}
\usepackage{fullpage}
\usepackage{hyperref}
\setdefaultlanguage{english}
\setotherlanguage{greek}
\newfontfamily\greekfont{Gentium Plus}
\newcommand{\gk}[1]{\textgreek{#1}}

\title{Rosalind Hursthouse: `Normative Virtue Ethics'}
\author{}
\date{}

\begin{document}

\maketitle

\noindent VE often accused of `not telling us what to do'
\vspace*{2mm}

\noindent Often takes the form of saying that it is `agent-centered' rather than `act-centered'
\vspace*{2mm}

\noindent Thus, taken not to be a normative rival to consequentialism or deontology, but a supplement (reminding us that our normative theories leave a lot of what we are interested in when asking questions about our lives out)
\vspace*{2mm}

\noindent But anyone who wants to espouse virtue ethics as a rival to deon­tological or utilitarian ethics (finding it distinctly bizarre to suppose that Aristotle espoused either of the latter) will find this com m on belief voiced against her as an objection: `Virtue ethics does not, because it can­ not, tell us what we should do. Hence it cannot be a normative rival to deontology and utilitarianism.'
\vspace*{2mm}

\noindent So, she does seem to be saying that this is what Aristotle would say; otherwise, why say this?

\begin{itemize}\item{Act-utilitarian An action is right iff (and because) it promotes the best consequences}\begin{itemize}\item{To actually get substance about what action to perform, we need something about best consequences}\begin{itemize}\item{The best consequences are those in which hap­piness is maximized}\end{itemize}\end{itemize}\item{Deontology: An action is right iff and because it is in accordance with a correct moral rule or principle.}\begin{itemize}\item{Need some specification of correct moral rule}\begin{itemize}\item{Correct moral rule is one that}\begin{itemize}\item{is on this list}\item{is laid on us by God}\item{is universalizable}\item{Would be the object of choice of all rational beings}\end{itemize}\end{itemize}\end{itemize}\end{itemize}

\noindent The big question, then, is `How can virtue ethics define both the Good and the Right in terms of the (virtuous) agent?'

\begin{itemize}\item{An action is right iff (and because) it is what a virtuous agent would characteristically (i.e. acting in character) do in the circumstances}\begin{itemize}\item{Don't respond: but, that doesn't tell us \emph{what} to do, `Who are the virtuous agents?'}\item{Because, both act-utilitarianism and deontology needed a second claim to get us to what actions are right/wrong}\end{itemize}\end{itemize}

\noindent Now, we need to be careful to define the virtuous agent in neither deontological nor consequentalist terms
\vspace*{2mm}

\noindent If you say, `the VA is someone whose actions maximize best outcomes' or `who acts in accordance with the correct moral rules', VE \emph{as a normative theory} collapses into one or ther other
\vspace*{2mm}

\noindent\textbf{THAT'S NOT QUITE RIGHT IF YOU HAVE THE `BECAUSE' CLAIM; IT MIGHT BE TRUE THAT THE VIRTUOUS AGENT IS ONE WHOSE ACTIONS MAXIMIZE GOOD CONSEQUENCES, BUT THE ACTIONS AREN'T RIGHT BECAUSE OF THAT}
\vspace*{2mm}

\noindent Second premise of VE

\begin{itemize}\item{A virtuous agent is one who acts virtuously, that is, one who has and exercises the virtues}\item{A virtue is a character trait that..}\begin{itemize}\item{Is on this list}\item{Is necessary for \emph{eudaimonia}}\end{itemize}\end{itemize}

\noindent But, even so, whereas deontology can give you things like: don't lie, don't steal and so on; VE just says `do what the honest person would do'; but, if I'm not honest, how on is is that helpful?
\vspace*{2mm}

\noindent So, one background question she has is whether it really is that difficult to know what the virtuous action is in many circumstances
\vspace*{2mm}

\noindent \textbf{SO, WOULD ARISTOTLE ACCEPT THAT BICONDITIONAL?}
\vspace*{2mm}

\noindent Texts

\begin{itemize}\item{Maybe}\begin{itemize}\item{2.4, 1105b7}\item{2.6, 1107a1; connection with 6.1}\item{3.4, 1113a30}\end{itemize}\item{Maybe not}\begin{itemize}\item{2.5, \emph{hexis}}\item{1.7}\end{itemize}\end{itemize}

\noindent Is there a problem with a normative theory specifying right action in explicitly evaluative terms?
\vspace*{2mm}

\noindent She makes a lot of hay over child-rearing; saying things like `Don't do that it hurts, you mustn't be cruel', `be kind to your brother, he's only little'
\vspace*{2mm}

\noindent But, isn't this just instructing by ostenison and then analogy as to what these kinds of actions are
\vspace*{2mm}

\noindent So, at times VE looks somewhat like deontology, but the backing is different

\section*{Conflict}

\noindent VE is supposed to be particularly vulnerable to conflicts, perhaps even irresolvable conflicts
\vspace*{2mm}

\noindent Consequentialism is often thought not to be, since the maximal value is what you ought to pursue
\vspace*{2mm}

\noindent Is deontology?
\vspace*{2mm}

\noindent Charity prompts me to kill the person who would (truly) be better off dead, but justice forbids it. Honesty points to telling the hurtful truth, kindness and compassion to remaining silent or even lying. And so on. So virtue ethics lets us down just at the point where we need it, where we are faced with the really difficult dilemmas and do not know what to do.
\vspace*{2mm}

\noindent Are there really conflicts? Can't we really examine a lot of them more closely and see that they don't really conflict?
\vspace*{2mm}

\noindent One quite interesting question: why do there not seem to be moral whiz kids, like there are in math or (maybe) science?

\section*{Moral dilemma}

\noindent Differs from conflict, not necessarily between the virtues but just, reasons for and against doing A and B seem to balance out; how does VE say we should think about such matters?
\vspace*{2mm}

\noindent First, what is a normative \emph{theory}; is it a mistake to hope from ethics what we want from science?
\vspace*{2mm}

\noindent Good thing about a theory is that it systematizes things and helps us see things we did not know; in the ethical sphere, this would be most important precisely at those points where we don't really know what to do; it's not good for it just to capture the obvious cases
\vspace*{2mm}

\noindent Must moral dilemmas be resolvable? Is that what we should expect

\begin{itemize}\item{Dilemmas must have a resolution}\item{Allow possibility of nothing counting as \emph{the} reasonable practical answer}\item{Sufficiently flexible to allow for comprehensible disagreement on this issue between two proponents of the normative ethics in question}\end{itemize}

\noindent The third allows that two agents in the same circumstances who do different things could \emph{each} have done the virtuous action
\vspace*{2mm}

\noindent And, this seems natural (sort of), since here we have two virtuous agents who do different things: so the if and only if gets covered (but, can it be characteristically?)

\end{document}