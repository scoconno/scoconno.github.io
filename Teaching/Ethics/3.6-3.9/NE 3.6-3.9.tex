% !TEX encoding = UTF-8 Unicode
% !TEX TS-program = xelatex

\documentclass[11pt]{article}
\usepackage{polyglossia}
\usepackage{amssymb}
\usepackage{fullpage}
\usepackage{hyperref}
\setdefaultlanguage{english}
\setotherlanguage{greek}
\newfontfamily\greekfont{Gentium Plus}
\newcommand{\gk}[1]{\textgreek{#1}}

\title{\emph{Nicomachean Ethics} 3.6-3.9}
\author{}
\date{}

\begin{document}

\maketitle

\noindent\underline{3.6}
\vspace*{4mm}

\noindent Recall the general account of virtues we got in Book 2: there are `ranges of objects' towards which we naturally have certain reactive feelings, those feelings fall on a continuum of intensity
\vspace*{2mm}

\noindent Fear and confidence are the reactive feelings we have towards `bad things'
\vspace*{2mm}

\noindent A thinks that all bad things are naturally feared; but the brave person fears some bad things more than others (bad reputation is the correct kind of thing to fear; other things is not; things that are not the results of vice or caused by ourselves)
\vspace*{2mm}

\noindent He runs through what seem to be a lot of intuitive claims about what kinds of fearful things concern the brave person
\vspace*{2mm}

\noindent Death might seem most frightening of all; it is a boundary beyond which we know not what; but not all death
\vspace*{2mm}

\noindent It is death in war; war presents the `greatest and finest' danger
\vspace*{2mm}

\noindent\textbf{READ 1115a33}; Aristotle seems to have a weird obsession with the sea
\vspace*{4mm}

\noindent\underline{3.7}
\vspace*{4mm}

\noindent So, while it's true that different people find different things frightening; there are some things that are `too frightening' for a human being to resist, which are frightening for anyone
\vspace*{2mm}

\noindent\textbf{READ 1115b12} Here we settle an issue raised in Book 2, the brave person is `unperturbed' but will fear even the sorts of things that are not irresistible
\vspace*{2mm}

\noindent\textbf{READ next two graphs} Here we get something we've gotten a lot; the `right way stuff'; how are we to understand this?
\vspace*{2mm}

\noindent Excessively fearless person has no name; some sort of madman (or Celt?)
\vspace*{2mm}

\noindent Excessively confident person about frightening things is rash (pretender to bravery)
\vspace*{2mm}

\noindent Do the claims he makes about rash vs. brave people ring true?
\vspace*{2mm}

\noindent Excessively fearful: coward; also deficient in confidence
\vspace*{2mm}

\noindent\textbf{READ 1116a10-16} How narrow do we think we should take the scope here?
\vspace*{4mm}

\noindent\underline{3.8}
\vspace*{4mm}

\noindent Conditions that resemble bravery (but, presumably, are not actually braver)

\begin{itemize}\item{Civic bravery}\begin{itemize}\item{Most like bravery}\item{Also include those compelled by superiors; less like bravery (and less good), because they are motivated by fear, not shame, and to avoid pain not to avoid disgrace}\item{\textbf{ARE THESE MIXED ACTIONS? NO, BUT IMPORTANT TO SEE WHY NOT}}\end{itemize}\item{Experience}\begin{itemize}\item{This is an odd one; especially given his example of professional soldiers (\textbf{NB: MERCENARIES})}\item{Aren't these people who have been habituated to feel and act the right way? What's lacking then?}\item{Clearly there is a level of understanding required for someone to be virtuous}\item{It seems it can't just be a matter of having been habituated}\item{But, they only fight when they have the advantage}\item{They abandon their post when the danger arises}\item{So, presumably, this is `mere' experience that resembles but isn't actually bravery?}\end{itemize}\item{Spirit}\begin{itemize}\item{Brave people are spirited; and so those who are spirited often seem brave, whether or not they are actually brave}\item{But, this is actually most like bravery, since, once the person acquires `decision and the goal' they would actually become brave}\item{He is not saying that such people don't decide to do what they do; presumably what he means is that they don't decide for the right goal}\end{itemize}\item{Hopeful people}\begin{itemize}\item{We do call such people, maintaining hope in the face of dim prospects, brave}\item{Given how he describes these people, I'm not sure `hopeful' is the best rendering}\item{These are people whose `bravery' depends on their feeling of superiority}\end{itemize}\item{People who act in ignorance}\begin{itemize}\item{Not quite sure what the idea is here, but I guess people who are just unaware of what the danger is, might appear brave when they stand to face it}\end{itemize}\end{itemize}

\noindent\underline{3.9}
\vspace*{4mm}

\noindent Bravery concerns feelings of fear and confidence; but more centrally fear
\vspace*{2mm}

\noindent\textbf{READ 1117a33-end of graph} This raises an interesting issue; the end of some difficult thing seems fine; what does that lead us to think about the action (connect this with, if you do something which ultimately results in some state you take pleasure in, does that entail you did it for the sake of pleasure?
\vspace*{2mm}

\noindent Brave person finds death and wounds painful, and suffers them unwillingly, but does so because it's fine and/or failure is shameful
\vspace*{2mm}

\noindent This also shows that the active exercise of bravery is not necessarily pleasant; it's pleasant insofar as we attain the end
\vspace*{2mm}

\noindent\textbf{READ LAST GRAPH} Interesting; here we get that the brave might not be the best soldiers; maybe it's those who are less brave but have `nothing to lose'

\end{document}

