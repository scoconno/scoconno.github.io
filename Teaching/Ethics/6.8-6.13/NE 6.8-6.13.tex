% !TEX encoding = UTF-8 Unicode
% !TEX TS-program = xelatex

\documentclass[11pt]{article}
\usepackage{polyglossia}
\usepackage{amssymb}
\usepackage{fullpage}
\usepackage{hyperref}
\setdefaultlanguage{english}
\setotherlanguage{greek}
\newfontfamily\greekfont{Gentium Plus}
\newcommand{\gk}[1]{\textgreek{#1}}

\title{\emph{Nicomachean Ethics} 6.8-6.13}
\author{}
\date{}

\begin{document}

\maketitle

\noindent\underline{6.6}
\vspace*{4mm}

\noindent `Understanding' (nous): strikes me as a bad translation, since we can clearly understand things that have explanations
\vspace*{2mm}

\noindent Basic problem is that the principles don't have explanations or proofs; but scientific knowledge involves grasping a proof
\vspace*{2mm}

\noindent So, can't have episteme of the principles; this is a good point, there does seem to be a different cognitive grasp of principles than derived facts
\vspace*{2mm}

\noindent Well, by elimination, then, can't be episteme, phronesis, or techne
\vspace*{2mm}

\noindent It's also not wisdom, since mark of wisdom is ability to prove
\vspace*{2mm}

\noindent So, only thing left is nous
\vspace*{2mm}

\noindent Important: we've learned that episteme requires grasp of principles, but here we learn that grasp of principles as such isn't (partially) constitutive of episteme
\vspace*{2mm}

\noindent\underline{6.7}
\vspace*{4mm}

\noindent Wisdom (sophia)
\vspace*{2mm}

\noindent Most exact form of scientific knowledge
\vspace*{2mm}

\noindent\textbf{READ 1141a18} Wisdom = understanding + scientific knowledge
\vspace*{2mm}

\noindent Here we also seem to get a view on which quality of cognitive state in some sense depends on quality of its object
\vspace*{2mm}

\noindent Seems like it's gonna concern divine beings (1141b1)
\vspace*{2mm}

\noindent\underline{6.5}
\vspace*{4mm}

\noindent Prudence (phronesis)
\vspace*{2mm}

\noindent Who do we call prudent people? Those who can reason well about living well in general; who grasp and can reason about the sorts of things that promote living well (not some specialized area)
\vspace*{2mm}

\noindent Prudence can't be either scientific-knowledge nor craft, since the former deals with what can't be otherwise, and people are prudent insofar as the reason well about what can be otherwise
\vspace*{2mm}

\noindent\textbf{READ 1140b5}
\vspace*{2mm}

\noindent\textbf{READ 1140b21}
\vspace*{2mm}

\noindent\underline{6.8}
\vspace*{4mm}

\noindent Talk generally about the idea that there seems to be prudence concerning big-scale political decisions and concerning an individual
\vspace*{2mm}

\noindent Need experience to acquire prudence, since it involves knowledge of particulars, which comes from experience
\vspace*{2mm}

\noindent Focus on last paragraph
\vspace*{2mm}

\noindent\underline{6.9} 
\vspace*{4mm}

\noindent This chapter deals with good deliberation (euboulia); there's an important question of what the relationship between euboulia and phronesis is
\vspace*{2mm}

\noindent Euboulia is not scientific knowledge
\vspace*{2mm}

\noindent Not good guessing; since that doesn't actually involve reasoning at all; nor quick thinking, since quick thinking is a type of good guessing
\vspace*{2mm}

\noindent Nor is it belief; for weird reasons
\vspace*{2mm}

\noindent Remaining possibility is that euboulia belongs to thought; because it doesn't involve commitment (unlike belief)
\vspace*{2mm}

\noindent There is a normative flavor to good deliberation: it isn't just getting well to the end you set; needs a good end
\vspace*{2mm}

\noindent Nor is it just getting to the right end, since you can do that by accident
\vspace*{2mm}

\noindent\textbf{READ 1142b30}
\vspace*{2mm}

\noindent So, euboulia seems to be the correctness that phronesis issues in
\vspace*{2mm}

\noindent\underline{6.10}
\vspace*{4mm}

\noindent This chapter is on comprehension (sunesis): a difficult word to translate; it suggests something like quick to get what's going on; we might use the term `perceptive' in the way that we describe someone as perceptive in, say, social situations; being quick at getting `what's going on'
\vspace*{2mm}

\noindent Here, Aristotle says that comprehension and good comprehension (here sunesis and eusunesia) are the same; and we do often speak this way; in describing someone as perceptive we mean they are good at it
\vspace*{2mm}

\noindent So, what A says is that comprehension is `about what we might be pulled about and might deliberate about'; so its about the same things as prudence (this is why `being perceptive' is a pretty good translation)
\vspace*{2mm}

\noindent\underline{6.11}
\vspace*{4mm}

\noindent Many of the states just outlined concern grasp of particulars; those are what actions concern
\vspace*{2mm}

\noindent Understanding (nous) is the state in which we grasp things of which there is no rational account: this occurs w/r/t both first terms and last terms; first terms (of demonstrations); last terms , in action, that this is thus and such (no proof here, just immediate grasp)
\vspace*{2mm}

\noindent Need perception of the particulars; this perception is nous
\vspace*{2mm}

\noindent\textbf{READ 1143b15} So, maybe it is right to say that there are two virtues: prudence and wisdom
\vspace*{2mm}

\noindent\underline{6.12}
\vspace*{4mm}

\noindent In this chapter and the next he raises some puzzles about prudence: what good is it, for those who are already good; just as, if you are healthy, you don't need to know anything about medicine; if you are good, you don't need to know anything relevant to prudence
\vspace*{2mm}

\noindent So, maybe the answer is: it helps us become good
\vspace*{2mm}

\noindent But, again, it doesn't seem like we must have it \emph{ourselves} to become good; can't we just follow the advice of those who have it? Just as all we care about in becoming healthy is following the doctor's advice
\vspace*{2mm}

\noindent This gets to a core issue of philosophy asked often in the ancient world, but still important today: are there benefits to philosophy that require one to oneself engage in it? It seems like science, for the most part, does not require someone to be a scientist to engage in it
\vspace*{2mm}

\noindent Lastly: prudence is inferior to sophia, insofar as it concerns less valuable matters
\vspace*{2mm}

\noindent Responses

\begin{itemize}\item{[1] Prudence and wisdom are choice worthy in themselves because each is the virtue of a respective part of the soul}\begin{itemize}\item{Again, the worry though is, so what}\end{itemize}\item{[2] Wisdom produces happiness, not as a product, but just as being a constituent}\item{[3] Virtue makes the goal correct, prudence the things that promote the goal correct}\begin{itemize}\item{Lorenz translation: Decision, then, is made correct by virtue. But as for those things that are naturally done for the sake of that, that task belongs not to virtue, but to another capacity.}\item{The `for the sake of \emph{that}' is often taken to pick up `decision'}\item{But, it's much more plausible that `that' picks up virtue}\item{\textbf{READ} 1139a31-3}\item{So, the `that' here is virtue; meaning, actions done for the sake of becoming virtuous}\item{Go to 2.9; here, the aim is to become generous; if one recognizes that one is naturally incline to cheapness, cleverness is a matter of figuring out how to go in the opposite extreme; prodigality}\end{itemize}\item{Cleverness, is a distinct capacity, but one required for prudence: it involves being good at achieving a goal, whatever that goal is}\item{So, virtue requires prudence, which requires cleverness}\end{itemize}

\noindent\underline{6.13}
\vspace*{4mm}

\noindent Just as prudence relates to cleverness; genuine virtue relates to natural virtue
\vspace*{2mm}

\noindent Natural virtue has caused a lot of problems: given that, since the very beginning of book 2, we have been learning that we are not virtuous by nature, it's hard to see what `natural virtue' could be; but, it seems to be something like, having an unreflective, `natural' disposition to do virtuous actions
\vspace*{2mm}

\noindent\textbf{READ 1144b4-7}: NB, should take `prone' to cover all examples
\vspace*{2mm}

\noindent Full virtue cannot be acquired without prudence
\vspace*{2mm}

\noindent Soc. was wrong to think that all virtues were prudence (or instances of prudence), but he was right to think that all virtues require prudence
\vspace*{2mm}

\noindent\textbf{READ 144b23}: Here we get the answer to `what is the correct reason'
\vspace*{2mm}

\noindent Now, raise the issue: what is the word `logos' supposed to mean?
\vspace*{2mm}

\noindent We cannot be fully good without prudence, or prudent without virtue of character
\vspace*{2mm}

\noindent So, while people think: yes, you can be courageous without the other virtues, they are mistaken; you can have natural tendencies in line with one virtue and not others, but can't have full virtue (i.e. real virtue)
\vspace*{2mm}

\noindent So, here the basic idea is that, the agent must actually understand why what they are doing is the virtuous thing, to be virtuous
\vspace*{2mm}

\noindent So, even if prudence were useless in action (\textbf= WE COULD GET THE SAME RESULTS EVEN IF WE WEREN'T PRUDENT), it would still be valuable


\end{document}

