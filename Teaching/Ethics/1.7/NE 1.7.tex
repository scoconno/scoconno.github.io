% !TEX encoding = UTF-8 Unicode
% !TEX TS-program = xelatex

\documentclass[11pt]{article}
\usepackage{amssymb}
\usepackage{fullpage}
\usepackage{hyperref}

\title{\emph{Nicomachean Ethics} 1.7}
\author{}
\date{}

\begin{document}

\maketitle

\noindent\underline{1.5}

\begin{itemize}\item{At first seems an alternative to the end of the political life (i.e. not honor); but, it also can't be it, since someone can be virtuous yet have horrible misfortunes befall him}\item{\textbf{READ: 1096a, ALLUSION TO PLATO'S \emph{Republic}}}
\end{itemize}

\noindent\underline{1.6}

\begin{itemize}\item{A here argues against the notion of the Platonic Form of the Good}\item{I.e. a good that is good in and of its self; not good for something, or good as a kind of thing; just good}\item{Aristotle does have some objections to the very idea that there is such a thing, but he is primarily concerned to argue that, even if there were such a thing, it wouldn't be any help in action}\item{Just as a carpenter can perfectly well get around in their carpentry business just by considering what is good in that domain (not helped at all in knowing what the good in itself is), so too the person pursuing the human good can do just fine without knowing what the good in itself would be}\end{itemize}

\noindent\underline{1.7}

\begin{itemize}

\item{Having rejected Plato, notes that there are many goods; good for each X is the end X pursues}\item{Unclear yet whether there is one ultimate end or several, and so one highest good or several}\item{Formal Criteria for the highest good: conditions we can lay down that any candidate must meet to count as the HG, before we even consider any particular candidates)}\begin{itemize}

\item{Completeness}\begin{itemize}\item{Whatever the highest good is, it is complete; if there are multiple complete ends, it will be the most complete}\begin{itemize}\item{This raises a big question that will concern us throughout: is the highest end inclusive or exclusive}\item{\textbf{Exclusive}: the single most complete end, excluding all others that are less complete}\item{\textbf{Inclusive}: The most complete end is the one that includes other ends, and is composed of them as parts}\item{NOTE: `For the sake of' can cover both end that is external and part-whole}\item{Use good vacation example}\end{itemize}\item{\textbf{READ 1097a31} An end X is more complete than and end Y if and only if we pursue X for its own sake and Y for the sake of something else}\item{\textbf{READ 1097a4-b4}: we get a test for determining whether or not some end X, which is choice worthy for the sake of some further end Y is \emph{also} choice worthy in itself}\item{This again raises the \emph{Inclusive vs. Exclusive issue}}\item{Thus, an end that is choice worthy only for its own sake and \emph{never} for the sake of something else, is complete without qualification}\item{He thinks that \textbf{HAPPINESS} fits the bill}\end{itemize}

\item{Self-sufficiency}\begin{itemize}\item{Not like a hermit but in the sense that a life that possesses it is automatically choice worthy}\item{\textbf{DISCUSS: HOW SHOULD WE UNDERSTAND THIS. CAN'T YOU MAKE ANY LIFE BETTER JUST BY ADDING SOME GOOD. PRESUMABLY IT CAN'T MEAN LACKING IN LITERALLY NOTHINg}}\end{itemize}

\item{Most choice worthy}\begin{itemize}\item{Not just one among others?}\end{itemize}

\end{itemize}

\end{itemize}

\noindent Ergon argument

\begin{itemize}\item{For things with `\emph{erga}' (`work', `function'), the good for X depends on its \emph{ergon}}\begin{itemize}\item{Support: abduction (flautist, sculptor, craftsman)}\item{Living organisms?}\end{itemize}

\item{Human beings have \emph{ergon}}\begin{itemize}\item{Support: Every part of the human being has an \emph{ergon}}\item{Thus the whole should have an \emph{ergon}}\begin{itemize}\item{Abductive support again?}\end{itemize}\end{itemize}

\item{\emph{Ergon} of human being is activity of the soul in accord with reason or requiring reason: \emph{based on} reason}\begin{itemize}\item{Support: argument by elimination}\begin{itemize}\item{Living is shared by many things, so it can't be any of the activities concerned with nutrition and growth}\item{Perception is shared by animals so it can't be the activities connected with perception}\item{All that's left is reason}\begin{itemize}\item{`Reason' has two senses: properly and obediently}\item{`Life' has two senses: capacity and activity; the latter is more properly \emph{life}}\end{itemize}\end{itemize}
\end{itemize}

\item{The human good is activity of the soul based on reason expressing virtue}
\begin{itemize}\item{X vs. Good X is not a difference in activity, but in way of activity}\item{Good X performs the activity well}\begin{itemize}\item{Support: Abductive or conceptual (i.e. that's just what the concept of Good gets us?)}\end{itemize}\item{S performs X well when S performs X based on virtue (expressive of virtue)}\end{itemize}\end{itemize}

\noindent So, here's how I would probably represent the argument for what the highest human good is succinctly

\begin{itemize}\item{[P1] The Human \emph{ergon} is activity of the soul based on reason}
\item{[P2] The good of something is achieved when its \emph{ergon} is performed well}\item{\underline{[P3] An \emph{ergon} is performed well when it is performed according to virtue}}
\item{[C] Human Good is activity of the soul based on reason expressing virtue}

\end{itemize}

\noindent Read the immediately following line about complete life
\vspace*{5mm}

\noindent This is just an `outline': we have no substantive account either of what `reason' is, nor of what `virtue' is: for the former we just have `that activity or ability that is distinctive of human beings as such'; for the latter we have, `whatever, when an action is expressed in accordance with, the action is done well; this all has to get filled out


\end{document}

