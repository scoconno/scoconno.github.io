% !TEX encoding = UTF-8 Unicode
% !TEX TS-program = xelatex

\documentclass[11pt]{article}
\usepackage{amssymb}
\usepackage{fullpage}
\usepackage{hyperref}

\title{Situationism}
\author{}
\date{}

\begin{document}

\maketitle

\section{The basic problem}

\noindent Aristotelian approaches to ethics presuppose distinctive commitments in descriptive psychology; thus they may be subject to empirical criticism
\vspace*{2mm}

\noindent\textbf{DISTINGUISH Empirical vs. Normative inadequacy}; talk about problems in just jumping from descriptive psychological claims to normative claims
\vspace*{2mm}

\noindent\textbf{Character trait}: (1) relatively long-term stable disposition to act in distinctive ways (Harman); (2) If someone possess a CT, she will exhibit relevant behavior in a given behavior relevant eliciting condition with some markedly above chance probability \emph{p} (Doris)
\vspace*{2mm}

\noindent Honesty is or includes reliable disposition not to steal, lie, or cheat
\vspace*{2mm}

\noindent Two dimensions of reliability

\begin{itemize}\item{Cross-temporal: in similar situations, similar behavior}\item{Cross-situational: similar behavior in different situations}\end{itemize}

\noindent Situationists don't really have a beef with cross-temporal reliability (since they are situationists, it would seem, in fact, that they need this)
\vspace*{2mm}

\noindent Their beef is with cross-situational reliability
\vspace*{2mm}

\noindent\textbf{NB}: Even if there is C-S reliability, doesn't entail that only/best explanation is that people have CTs (but, Sists deny C-S reliability so that doesn't matter here)
\vspace*{2mm}

\noindent We speak in ways, and attribute traits to people, that suggest we take people's behavior to be cross-situationally predictive (`courageous', `honest', `deceptive' and so on)
\vspace*{2mm}

\noindent But `Trait attribution is often surprisingly inefficacious in predicting behavior in particular novel situations, because differing behavioral outcomes often seem a function of situational variation more than individual disposition'
\vspace*{2mm}

\noindent For Aristotle: `we can say that Aristotelian virtues are robust, or substantially resistant to contrary situational pressures, in their behavioral manifestations.'
\vspace*{2mm}

\noindent\textbf{Fundamental attribution error}: Inflated belief in the importance of personality traits and dispositions, together with [a] failure to recognize the importance of situational factors in affecting behavior'
\vspace*{2mm}

\section{The Studies}

\noindent\textbf{Phone booth study}:
\vspace*{2mm}

\hspace*{17mm} Helped\hspace*{5mm} Did not help

Dime\hspace*{10mm} 14\hspace*{17mm} 2

No Dime\hspace*{5.5mm} 1\hspace*{17mm} 24
\vspace*{5mm}

\noindent\textbf{Helping stranger study}
\vspace*{4mm}

\noindent In the experiment, which both Harman(1999)and Ross and Nisbett (1991) adduce, Darley and Batson had students from the Princeton Theological Seminary prepare a short talk that was to be recorded in another building. The students also completed a questionnaire concerning the basis of their interest in religion. As they left for the other building, some students were told to hurry because they were late; others were told they had just enough time to get there; and yet others were told they would arrive a little early. En route, each student encountered a man slumped in a doorway, who coughed twice and groaned. Which of the students stopped to help? There were three variables-content of assigned talk, religious orientation, and degree of lateness. But only the degree of lateness-that is, a situational variable-turned out to be of any significance. 63 percent of those running early stopped to help, as did 45 percent of those running just on time, whereas only 10 percent of those running late stopped to help. The suggestion is that, in considering the parable,we are apt not to recognize the extent to which the priest and the Levite may simply have been running late.
\vspace*{4mm}

\noindent In one study of conscientiousness, for example, in which there were 19 different behavioural measures, observed between 2 and 12 times each, the mean temporal stability coefficient after aggregation was .65 (Mischel and Peake 1982, pp. 734-35). 
\vspace*{4mm}

\noindent The average correlation between different behavioral measures specifically
designed to tap the same personality trait (for example, impulsivity, honesty, dependency, or the like) was typically in the range between .10 and .20, and often was even lower... Virtually no coefficients, either between individual pairs of behavioral measures or between personality scale scores and individual behavioral measures, exceeded the .30 'barrier'
\vspace*{4mm}

\noindent According to Hartshorne and May (1982), the average correlation between stealing and lying was .13, between stealing and cheating .13, and between lying and cheating .31. Overall, the average correlation between any given pair of the various behavioural measures of honesty they studied was .23.
\newpage

\section{Situationism}

\noindent Three main features of Situationism

\begin{itemize}\item{(i) Behavioral variation across a population owes more to situational differences than dispositional differences among persons. Individual dispositional differences are not as strongly behaviorally individuating as we might have supposed; to a surprising extent we are safest predicting, for a particular situation, that a person will behave pretty much as most others would.}\item{(ii) Empirical evidence problematizes the attribution of robust traits. Whatever behavioral reliability we do observe may be readily short-circuited by situational variation:in a run of trait-relevant situations with diverse features, an individual to whom we have attributed a given trait will often behave inconsistently with regard to the behavior expected on attribution of that trait. Note that this is not to deny the possibility of temporal stability in behavior;the situationist acknowledges that individuals may exhibit behavioral regularity over time across a run of substantially similar situations.}\item{(iii) Personality structure is not typically evaluatively consistent. For a given person, the dispositions operative in one situation may have a very different evaluative status than those manifested in another situation-evaluatively inconsistent dispositions may "cohabitate"in a single personality.}\end{itemize}

\noindent Generally, human behavior is too fine-grained for Aristotle's picture
\vspace*{2mm}

\noindent ``It is important to notice that situationism is not embarrassed by the considerable behavioral regularity we do observe: because the preponderance of our life circumstances may involve a relatively structured range of situations, behavioral patterns are not, for the most part, haphazard. At bottom, the question is whether the behavioral regularity we observe is to be primarily explained by reference to robust dispositional structures or situational regularity. The situationist insists that the striking variability of behavior with situational variation favors the latter hypothesis.''

\section{Sreenivasan's response} 

\noindent First issue, people tend to attribute character traits on the basis of one action; clearly too quick (and, Aristotle is not prone to this; remember the third condition on acting virtuously: given what he says, he would think we are only justified in attributing virtue to someone on the basis of seeing many, many instances of behavior)
\vspace*{2mm}

\noindent First beef: FAE is besides the point; people's mistakes don't tell us whether there are CTs, but whether there is such reliable behavior
\vspace*{2mm}

\noindent But, this does raise issues for a main source of \emph{prima facie} evidence for the existence of CTs, namely, ordinary judgments
\vspace*{2mm}

\noindent Second beef (minor): But, virtue isn't supposed to be something a lot of people have (Doris addresses this); what would it mean if only virtuous people had CTs?
\vspace*{2mm}

\noindent Main beef: these studies don't operationalize the right kind of behavior for studying virtue
\vspace*{2mm}

\noindent But, one thing we have to focus on is not only what would constitute `honesty-relevant behavior' but what situations count as `honesty-eliciting' situations.
\vspace*{2mm}

\noindent Three major issues

\begin{itemize}\item{Whose construal matters: if the agent doesn't think that the situation is one where honesty is at issue (e.g. finders-keepers with loose change), then is that really a test of whether honesty is CS reliable}\begin{itemize}\item{Would Aristotle think that agent's construal matters?}\end{itemize}\item{Degree of relevance to the relevant CT: pocketing loose change is hardly a central case of theft (if it is one)}\item{Normative sensitivity of the specification: it's not just `sameness of situation' that is relevant, but `sameness of reason'}\begin{itemize}\item{E.g. if you take `lying' just to be, telling a falsehood with the intent to deceive, you can get two situations which are lying situation, but the ultimate aim promoted by lying in one differs from the other; so it's not obvious that lying in both is dishonest, if, say, a good end is promoted by telling a falsehood}\end{itemize}\end{itemize}

\end{document}

