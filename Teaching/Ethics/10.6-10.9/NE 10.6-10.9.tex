% !TEX encoding = UTF-8 Unicode
% !TEX TS-program = xelatex

\documentclass[11pt]{article}
\usepackage{polyglossia}
\usepackage{amssymb}
\usepackage{fullpage}
\usepackage{hyperref}
\setdefaultlanguage{english}
\setotherlanguage{greek}
\newfontfamily\greekfont{Gentium Plus}
\newcommand{\gk}[1]{\textgreek{#1}}

\title{\emph{Nicomachean Ethics} 10.6-10.9}
\author{}
\date{}

\begin{document}

\maketitle

\section{10.6}

\noindent Declares end to specific discussion; return to discussion of \emph{eudaimonia}
\vspace*{2mm}

\noindent \textbf{RECAP BRIEFLY HOW IT WENT}

\begin{itemize}\item{Began by talking generally about the highest good achievable in action}\item{argued that it is a certain type of activity performed a certain kind of way: specifically rational activity expressing virtue}\item{Two main kinds of virtue: of character and of the intellect}\item{2-5 discussed the first; 6 discussed the second}\item{Then we got a discussion of \emph{akrasia}, a very important ethical phenomenon, friendship (which he argued is necessary to be happy), and pleasure (which he thinks is a good, but not the good}\end{itemize} 

\noindent So, we would likely expect him to say that happiness somehow involves much of what he has talked about up until now
\vspace*{2mm}

\noindent And, that seems to come through in 10.6:

\begin{itemize}\item{\textbf{READ 1176b3}: Reiteration that happiness lacks nothing, is self-sufficient and most choice worthy}\item{\textbf{READ 1176b7}: Virtuous activity, i.e. \emph{all} virtuous activity, seems to be trumpeted--no reason to suspect this is limited to only certain kinds of virtuous activity}\item{Spends time ruling out the life of mere amusements; it would be bizarre to say that the highest good}\end{itemize}

\noindent Recall, also, \textbf{NE 1.5} and the discussion of the three lives; these are three ideals offered for the best life possible for human beings to live
\vspace*{2mm}

\noindent Here in 10.6 he rules out the life of pleasure (because that could be had merely by enjoying meaningless amusements)
\vspace*{2mm}

\noindent So, this all seems to be good
\vspace*{2mm}

\noindent But, when we turn to 10.7, things become problematic

\begin{itemize}\item{\textbf{READ 1177a13}}\begin{itemize}\item{A little jarring, the proper virtue of the theoretical intellect is \emph{sophia}}\item{So, it seems that `complete happiness' is theoretical contemplation expressing wisdom}\end{itemize}\item{This activity is best because:}\begin{itemize}\item{Its objects are the best}\item{it is the most continuous}\item{It has pure and firm pleasures}\end{itemize}\item{\textbf{READ 1177a28}}\begin{itemize}\item{This activity is also self-sufficient}\item{Requires external goods, just like the just and otherwise virtuous people; but less so}\item{\textbf{READ 1.7 1097b7}: How are these consistent?}\end{itemize}\item{\textbf{READ 1177b4-12}}\begin{itemize}\item{So, study is pursued for its own sake; but virtuous activity to a greater or lesser degree for further things}\end{itemize}\item{\textbf{READ 1177b17}}\end{itemize}

\noindent A acknowledges that a life that consisted solely of contemplation is not possible for a human being: we have physical, emotional, and social realities; but still, \textbf{READ 1177b35}
\vspace*{2mm}

\noindent So, this is somewhat shocking; it seems that activities in accord with the virtue of character don't have a role in the notion of eudaimonia and, so, in the good life (WTF?)
\vspace*{2mm}

\noindent This seems to be further confirmed at beginning of 10.8:
\vspace*{2mm}

\noindent \textbf{READ 1178a8}
\vspace*{2mm}

\noindent He also denigrates the life of virtue of character as human (as opposed to divine)
\vspace*{2mm}

\noindent \textbf{READ 1178a25} Again a comparison where VoC falls short
\vspace*{2mm}

\noindent The gods are supposed supremely blessed, but it seems ludicrous to think they engage in activities of virtue of character; they engage in contemplation; a theistic bent here
\vspace*{2mm}

\noindent \textbf{READ 1178b25}

\section{Problems}

\begin{itemize}\item{Seems to advocate a value monism (\textbf{concerning final value}), against the earlier value pluralism}\begin{itemize}\item{Either solely contemplation or solely virtuous activity, not both}\end{itemize}\item{Seems to demand immorality: the supreme goodness of contemplation seems to entail that we should sacrifice everything on the altar of contemplation}\begin{itemize}\item{Even if you insist, no, Aristotle still accords virtuous activity, what he says at \textbf{1178a1} suggests that won't matter}\item{The \emph{Ethics} is, in a real sense, theistically grounded, but `its theological pigeons must come home to roost in human immorality'}\end{itemize}\item{No sense of tradeoff between virtuous activity and contemplation}\end{itemize}

\section{Reactions}

\begin{itemize}\item{Accept it and criticize Aristotle}\item{Attempt some sort of reconciliation:}\begin{itemize}\item{Argue that one of the two lives is, in fact, value pluralist}\item{Argue that the two lives are complementary \emph{aspects} of a single life}\end{itemize}\end{itemize}

\section{My Response: Greatly indebted to Gavin Lawrence}

\noindent Two ways asking, `what's the best life to live?'

\begin{itemize}\item{However circumstanced ideal}\begin{itemize}\item{What's the best life to live, in whatever circumstances you find yourself in?}\end{itemize}\item{Ideally circumstanced ideal}\begin{itemize}\item{What's the best life to live in the best circumstances}\end{itemize}\end{itemize}

\noindent A basic thought is that, someone can say, the best thing to do in a situation is give to charity, without thinking that giving to charity is the best thing to do in optimal circumstances
\vspace*{2mm}

\noindent Indeed, it seems we \emph{should} say this: someone who says otherwise seems to be making a serious moral mistake
\vspace*{2mm}

\noindent At bottom: you shouldn't think that the absolutely best life a human being could conceivably live requires actions in response to natural or human desires depredation, disease etc.
\vspace*{2mm}

\noindent One way to get at this: a doctor can be perfectly confident that gardening is the best possible activity; yet, if he gets a call, he recognizes that, given the circumstances, tending the patient is the best activity \emph{in those circumstances}; he can tend the patient while also wishing that the circumstance hadn't arisen at all
\vspace*{2mm}

\noindent Difference between regretting \emph{having} done something and regretting \emph{having had to do} something
\vspace*{2mm}

\noindent Everyone should be aware of both, especially politicians
\vspace*{2mm}

\noindent A shows himself to be keenly aware of this: \textbf{1177b6, 1178b5}
\vspace*{2mm}

\noindent So, A thinks that contemplation is the ideally circumstanced ideal and such a life is the best life for a human being to live
\vspace*{2mm}

\noindent But, he acknowledges, such a life is, in a real sense, supra-human; we have to eat, so temperance arises; we need other things we can't provide for ourselves, so interactions with other people are necessary
\vspace*{2mm}

\noindent So, we need the virtues of character, all of them, because that is the only way to ensure that we do what is best in whatever circumstances befall us; and if we are raising kids, we raise them to have the virtues of character, because this puts them in the best position to do the best they can in the circumstances in which they find themselves
\vspace*{2mm}

\noindent So, the issue in the two lives isn't one of value, but activity: what activities are best to perform; the person still has all the right scheme of values
\vspace*{2mm}

\noindent Solves problems as follows:
\vspace*{2mm}

\noindent [1] Not an issue of value but activity; it isn't that the agent \emph{values} only contemplation; the question is the activities the agent engages in
\vspace*{2mm}

\noindent [2] Also realize that here virtuous activity retains its final value; but it is \emph{also} valuable because it contributes to producing those circumstances that are more ideal, and in which people can contemplate
\vspace*{2mm}

\noindent [3] No, you can perfectly well think that contemplation is the best activity to engage it yet recognize that it cannot be engaged in \emph{given the current circumstances}; Now, the tradeoff will be difficult, but that's not immediately the problem
\vspace*{2mm}

\noindent [4] No algorithmic tradeoff; but that's to be expected, how could there be, given A's general approach?

\section{10.9}

\noindent Declares another end; but, the aim all along wasn't just to be able to answer some academic questions, but to become good; to act on our knowledge
\vspace*{2mm}

\noindent So, he closes the work by discussing how to put it into action
\vspace*{2mm}

\noindent The role of argumentation in fostering good behavior is limited

\begin{itemize}\item{Might stimulate and encourage already well-raised people}\item{Might be able to effect a final transition for someone well brought up, to appreciate why what he values is, in fact, valuable}\item{But, for people who aren't already far along the path to being good, arguments seem limited}\end{itemize}

\noindent The many, unfortunately, aren't motivated by shame but fear; they focus on prudential, rather than honor-bound, moral reasons; \textbf{APPIAH}; \textbf{PROPAGANDA}
\vspace*{2mm}

\noindent If, generally speaking, you are the kind of person led on by your feelings, \emph{rational} arguments aren't likely to have much force; they respond to brute force, near Pavlovian-condition
\vspace*{2mm}

\noindent So, for arguments to be effective, you already have to be inclined in the right direction
\vspace*{2mm}

\noindent Laws have a huge role to play in this

\begin{itemize}\item{Corrective punishments}\item{Banishing incurables}\item{Orderliness}\end{itemize}

\noindent Tension between influences of a particular individual (e.g. a father) and the laws, or the state as a whole; the latter, Aristotle thinks, is so much more important
\vspace*{2mm}

\noindent There is sense to this; insofar as the main influences that are likely to lead a child away from the instruction of his parents is the prevailing societal mores
\vspace*{2mm}

\noindent Only Sparta have they attended to upbringing and practices

\begin{itemize}\item{This gives some indication of what sort of thing he is talking about}\item{Sparta was heavily regimented}\item{The agoge}\end{itemize}

\noindent Aristotle is attuned to the fact that individualized treatment is often more effective; tailor-made to the individual; so it may be better to have the parent do it, but only if the state isn't willing to do its job
\vspace*{2mm}

\noindent But still, you'll get the best attention if you have the universal knowledge and bring it to bear; experience can work in circumscribed cases, but not in the vast range that beset human life
\vspace*{2mm}

\noindent By analogy with medicine; A advocates that politicians need legislative science \emph{and} experience; we don't become doctors just by reading textbooks; we can't judge which laws are best from examining laws alone
\vspace*{2mm}

\noindent But, theoretical study of health is beneficial to those with experience (even if the latter themselves lack it), so there is urgency here
\vspace*{2mm}

\noindent So too with law; so, let's discuss Politics


\end{document}

