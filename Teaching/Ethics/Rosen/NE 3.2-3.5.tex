% !TEX encoding = UTF-8 Unicode
% !TEX TS-program = xelatex

\documentclass[11pt]{article}
\usepackage{polyglossia}
\usepackage{amssymb}
\usepackage{fullpage}
\usepackage{hyperref}
\setdefaultlanguage{english}
\setotherlanguage{greek}
\newfontfamily\greekfont{Gentium Plus}
\newcommand{\gk}[1]{\textgreek{#1}}

\title{\emph{Nicomachean Ethics} 3.3-3.5 and Rosen}
\author{}
\date{}

\begin{document}

\maketitle

\section*{Recap}

\noindent\textbf{READ 1112b12-16}
\vspace*{2mm}

\noindent The last thing found in the analysis (i.e. deliberation?) is the first that comes into being (DOES THIS REQUIRE DECISION TO BE PARTICULAR?)
\vspace*{2mm}

\noindent We do not deliberate about particulars, questions such as `what is this'? aren't deliberated about
\vspace*{2mm}

\noindent\textbf{READ 1113a}: Again, general or particular?
\vspace*{2mm}

\noindent\textbf{READ 1113a10}

\section*{3.4}

\noindent Aristotle has maintained that wish is for the end; now he takes up a question that dogged a lot of ancient thinkers: do we only wish for the good (or can we also wish for the apparent good)
\vspace*{2mm}

\noindent\textbf{READ 1113a22}: I think that we are supposed to read `wished by nature' to mean `by nature worthy of being wished'; similar to choice worthy (I would take a statement like `chosen by nature' to be, but nature such as to be chosen)
\vspace*{2mm}

\noindent Aristotle says that, in reality and without qualification, what is wish for is the good; but for each person on each occasion, what is wished for is what appears to them to be good; for the good person, the two will line up
\vspace*{2mm}

\noindent\textbf{READ 1113a33}: Don't take `measure or standard' to be metaphysical

\section*{3.5}

\noindent\textbf{READ a lot of the beginning of the chapter}
\vspace*{2mm}

\noindent Aristotle takes supporting evidence from the fact that legislators 
\vspace*{2mm}

\noindent\textbf{READ 1113b30}: Here we do get an indication that Aristotle is concerned with something we have wondered; what about being ignorant but being responsible for the ignorance
\vspace*{2mm}

\noindent So, A is basically here trying to argue that we are responsible for our characters; and the fact that we have a character cannot get us off the hook for the actions we do
\vspace*{2mm}

\noindent And this is because A thinks that we acquire our characters through habituation, but that just means repeated particular actions, and in each case, what we do is up to us, so the character we acquire is up to us
\vspace*{2mm}

\noindent This is incredibly difficult territory: look at \textbf{TEXT 1114a12-13}; But, do people know what actions will make them what way? presumably they have some info, but enough?
\vspace*{2mm}

\noindent A major objection comes at \textbf{1114b}
\vspace*{2mm}

\noindent\textbf{READ 1114b15}: What is Aristotle saying here?

\section*{Rosen}

\noindent Distinguish Factual from Moral ignorance
\vspace*{2mm}

\noindent Factual ignorance exculpates when it itself is blameless; when it isn't blameless, it doesn't exculpate
\vspace*{2mm}

\noindent Factual ignorance is culpable when it is the upshot of epistemic irresponsibility
\vspace*{2mm}

\noindent What about moral ignorance?
\vspace*{2mm}

\noindent Case of Hittite slave owner:

\begin{itemize}\item{Had he known slavery was morally wrong, he would have acted otherwise}\item{Question: Is he culpable for being ignorant of the moral wrongness of slavery?}\item{Answer: maybe not; he might not have demonstrated any epistemic irresponsibility}\end{itemize}

\noindent Note: Rosen is not saying we shouldn't be upset or condemn the Hittite's actions; just that we shouldn't blame him for it (we can shake our fist at the universe for `serving up injustice on so grand a scale')

\noindent Case of Smith the (1952) sexist
\vspace*{2mm}

\noindent The basic upshot is that, if seeing through to the `moral truth' requires going above and beyond the epistemic call of duty, moral ignorance is blameless (and so is any action that stems from or because of that ignorance)
\vspace*{2mm}

\noindent Otherwise, we have to say that someone, who is blameless for being ignorant of what he ought to do, and who blamelessly thinks that phi-ing is permissible, is nevertheless blameworthy for phi-ing
\vspace*{2mm}

\noindent This even applies to `ignorance about the reason-giving force of moral considerations'
\vspace*{2mm}

\noindent Here's where he most directly ties it up with Aristotle

\end{document}

