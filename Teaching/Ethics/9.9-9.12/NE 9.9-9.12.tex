% !TEX encoding = UTF-8 Unicode
% !TEX TS-program = xelatex

\documentclass[11pt]{article}
\usepackage{polyglossia}
\usepackage{amssymb}
\usepackage{fullpage}
\usepackage{hyperref}
\setdefaultlanguage{english}
\setotherlanguage{greek}
\newfontfamily\greekfont{Gentium Plus}
\newcommand{\gk}[1]{\textgreek{#1}}

\title{\emph{Nicomachean Ethics} 9.9-9.12}
\author{}
\date{}

\begin{document}

\maketitle

\section{Recap}

\noindent Recall: Friendship for Aristotle is reciprocated goodwill of which both parties are aware
\vspace*{2mm}

\noindent Also recall: Aristotle thinks there are three causes of friendship: good, pleasure, utility
\vspace*{2mm}

\noindent The big question was whether character friendship is the only one that actually meets the definition, or whether relationships based on pleasure or utility can as well
\vspace*{2mm}

\noindent A big sticking point was whether two people whose relationship is based on mutually derived pleasure can wish well for the other for their own sake (and whether the latter were required for genuine friendship)
\vspace*{2mm}

\noindent Example of pleasure and utility friendship
\vspace*{2mm}

\noindent So, Aristotle thinks that character friendship can only form between good people; the recognition that someone else is of good character is a necessary condition for wishing them well solely for their own sake
\vspace*{2mm}

\noindent\textbf{DISCUSS IN WHAT SENSE THIS CAN MAKE THESE KINDS OF FRIENDSHIPS AN INTRINSIC GOOD, RATHER THAN AN INSTRUMENTAL ONE}

\section{9.8}

\noindent Self-love seems problematic: we conventionally denounce the self-lover
\vspace*{2mm}

\noindent We also think it's a mark of the vicious person to care more about themselves than anyone else
\vspace*{2mm}

\noindent And, we often take it to be the mark of a decent person to act for his friend's sake even at his own detriment
\vspace*{2mm}

\noindent One puzzle: self-love seems to fit the definition of friendship: you wish good for yourself for your own sake
\vspace*{2mm}

\noindent You could say: OK, fine, but friendship requires two people; but that doesn't remove the interesting question about the appropriate relationship to oneself
\vspace*{2mm}

\noindent The important point is the kind of things you wish for yourself: if it is money, pleasure, power etc. than it is deplorable
\vspace*{2mm}

\noindent If it is the fine, then it is commendable
\vspace*{2mm}

\noindent Because, what is it to wish that the fine happens to yourself, it's that you do virtuous actions for their own sake; and that is a commendable disposition
\vspace*{2mm}

\noindent So, it's wrong just to condemn self-love as bad

\section{9.9}

\noindent At the end of the book, he raises an important question: why are friends needed?
\vspace*{2mm}

\noindent One serious difficulty pervading this question in general is: we want to explain why a person needs friends, without answering that question in a way that makes genuine friendships (i.e. complete friendships) impossible
\vspace*{2mm}

\noindent The important question is whether the happy person will need friends
\vspace*{2mm}

\noindent In the preceding chapter, he argued that the virtuous person will be concerned with his friends because he is concerned with the fine; but, if friends are just an instrumental good, and hence, not fine but dispensable, what is fine about being concerned for our friends?
\vspace*{2mm}

\noindent Happiness is supposed to be self-sufficient: it alone makes a life choice worthy and lacking in nothing; if friends provide what you yourself can't, what need of friends?

\begin{itemize}\item{One way to ask: can a person be happy without friends?}\item{Other: if a person is happy, does that person need friends?}\end{itemize}

\noindent\textbf{READ 1169b9-14} So, it seems that there are some activities that are constitutive of \emph{eudaimonia} that can only be engaged in with friends
\vspace*{2mm}

\noindent\textbf{DOES THAT SOUND RIGHT TO PEOPLE?}
\vspace*{2mm}

\noindent Friendship, Aristotle thinks, is a necessary expression of human nature: humans are social animals; to live a solitary life is to go, in some sense, against human nature
\vspace*{2mm}

\noindent The objection rests on thinking that friendship is merely an instrumental good (pleasure or utility)
\vspace*{2mm}

\noindent So, A's response is to say that, to lack character friends is to lack one of the intrinsic goods, and that's why a happy person needs friends
\vspace*{2mm}

\noindent \textbf{READ 1169B29}: So, A seems to think that having character friends 
\vspace*{2mm}

\noindent\textbf{READ 1170a13ff}: very difficult argument here

\section{9.10}

\noindent How many friends are needed?
\vspace*{2mm}

\noindent With utility friendships, not too many, not none; it's hard to reciprocate utility a lot
\vspace*{2mm}

\noindent It just isn't possible to have character friendships with too many people

\section{9.11}

\noindent There is, running throughout this, a worry that Aristotle just has a bad view of friends: often he speaks as if, the point of having friends is to be able to engage in activities you otherwise wouldn't be able to
\vspace*{2mm}

\noindent The worry, is that if you think the point of friends (or even a point of friends) is that doing good to friends is finer or nobler than doing good to non-friends, that just seems like the wrong reason
\vspace*{2mm}

\noindent What do we think: we do speak as if part of the point of having friends is that, when we are down, they help bring us up; is that a necessary condition for being a friend?
\vspace*{2mm}

\noindent We do have the phrase `fair-weather' friends

\section{9.12}



\end{document}

