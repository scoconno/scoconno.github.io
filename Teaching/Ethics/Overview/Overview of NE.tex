% !TEX encoding = UTF-8 Unicode
% !TEX TS-program = xelatex

\documentclass[11pt]{article}
\usepackage{amssymb}
\usepackage{fullpage}
\usepackage{hyperref}

\title{Outline of the \emph{Nicomachean Ethics}}
\author{}
\date{}

\begin{document}

\maketitle

\noindent\textbf{Book 1}: Aims primarily to identify the highest good human beings can achieve in action

\begin{itemize}\item{Maintains that it in someway involves or requires virtue (\emph{aret\^{e}})}\item{Claims that virtue comes in two kinds: virtue of character (\emph{ethik\^{e}}), virtue of thought (\emph{diano\^{e}tik\^{e}})}\end{itemize}

\noindent\textbf{Book 2}: Discusses character virtue at a general level, trying to identify what kind of thing it is
\vspace
\noindent\textbf{Book 3.1--3.5}

\begin{itemize}\item{Discusses some of the `pre-conditions' for virtue}\begin{itemize}\item{3.1: Voluntariness}\item{3.2: Decision}\item{3.3: Deliberation}\item{3.4: Wish}\item{3.5: Whether it's in our power to be virtuous or vicious}\end{itemize}\end{itemize}

\noindent\textbf{Book 3.6--5.11}

\begin{itemize}\item{Discusses particular character virtues and corresponding vices}\begin{itemize}\item{3.6--3.9 Bravery}\item{3.10--3.12: Temperance}\item{4.1: Generosity}\item{4.2: Magnificence}\item{4.3: Magnanimity}\item{4.4: Virtue concerned with small honors}\item{4.5: Mildness}\item{4.6: Friendliness}\item{4.7: Truthfulness}\item{4.8: Wit}\item{4.9: Shame (not actually a virtue)}\item{5.10-

\end{document}

