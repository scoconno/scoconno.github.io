% !TEX encoding = UTF-8 Unicode
% !TEX TS-program = xelatex

\documentclass[11pt]{article}
\usepackage{polyglossia}
\usepackage{amssymb}
\usepackage{fullpage}
\usepackage{hyperref}
\setdefaultlanguage{english}
\setotherlanguage{greek}
\newfontfamily\greekfont{Gentium Plus}
\newcommand{\gk}[1]{\textgreek{#1}}

\title{\emph{Nicomachean Ethics} 6.1-6.7}
\author{}
\date{}

\begin{document}

\maketitle

\noindent\underline{6.1}
\vspace*{4mm}

\noindent\textbf{READ FIRST SENTENCE}: Careful that the Greek doesn't say `let us now determine what it says', but, `let us now determine this' (the former would suggest a level of specificity that we've seen Aristotle isn't willing to give; the latter can read that way, but also allows just what the correct reason is
\vspace*{2mm}

\noindent Analogy: it's true to say that someone should prescribe the medicine that medical science prescribes, but, what is that

\begin{itemize}\item{Here might be a place to press the analogy: it would be bizarre to say that correct/healthy medicine is what it is \emph{because} medical science prescribes it}\end{itemize}

\noindent Back to the soul

\begin{itemize}\item{Distinction within the rational part of the soul (this is the part that has reason strictly speaking}\begin{itemize}\item{That with which we study things whose first principles cannot be otherwise: the scientific part}\item{Things whose principles admit of being otherwise: the rationally calculating part}\end{itemize}\item{\textbf{DISCUSS THIS DIFFERENCE: IT'S CLEARLY RELATED TO THEORETICAL VS. PRACTICAL RATIONALITY; HOW IS A DRAWING THE DIFFERENCE?}}\end{itemize}

\noindent So, what's the best state of the scientific part and best state of rationally calculating part; that is the virtue of each
\vspace*{2mm}

\noindent Reiterates that virtue of \emph{X} is relative to proper \emph{ergon} of \emph{X}
\vspace*{2mm}

\noindent\underline{6.2}
\vspace*{4mm}

\noindent Three things in the soul control action and truth

\begin{itemize}\item{Sense-perception: not the principle of any action, since beasts have perception but no share in action}\begin{itemize}\item{So, `action' must be meant in a very restricted manner}\item{Note, here, that being `principle' of action is more than just being involved in action; since perception certainly is necessary for acting}\end{itemize}\item{Desire: assertion and denial: though :: pursuit/avoidance: desire}\begin{itemize}\item{Desire is definitely a principle, because decision is a kind of desire (a deliberative desire); but, truth is the good state of the scientific part}\item{Ergon of rationally calculating part is truth agreeing with correct desire}\item{Here is a problem: this seems to be a definition of the \emph{good}, not just the \emph{ergon} of this part}\end{itemize}\item{Understanding (\emph{nous}): things get tricky here; this is sometimes used for the intellectual part all together}\begin{itemize}\item{Thought as such isn't a principle; but goal-directed thought}\item{Here is an interesting question that re-sparks the Humean issue:}\begin{itemize}\item{Thought only motivates us when we use it to achieve some end we \emph{antecedently} desire}\item{Thought can move us just by being aimed at some end we recognize as worthy of desire}\end{itemize}\item{If the latter, we don't need to posit a distinct capacity that generates the desire}\end{itemize}\end{itemize}

\noindent One question: Does A here suggest that virtue of character is separable from correct reasoning; or, is it compatible with what he says that one must have correct reasoning to have virtue of character?
\vspace*{2mm}

\noindent\textbf{READ 1139b11-14}
\vspace*{2mm}

\noindent\underline{6.3}
\vspace*{4mm}

\noindent Five states of the soul concerned with truth: Scientific knowledge, craft, prudence, wisdom, understanding
\vspace*{2mm}

\noindent One question: How many VoIs are there? Two? All of these?
\vspace*{2mm}

\noindent Scientific knowledge (episteme)
\vspace*{2mm}

\begin{itemize}\item{Deals with what cannot be otherwise}\item{Proceeds from necessities}\item{Spelled out in detail in \emph{Posterior Analytics}}\item{Demonstrative state}\item{So, this can't be the virtue of the rationally calculative part, since that deals with things that can be otherwise}\item{\textbf{GO THROUGH A DEMONSTRATION AND SHOW WHY THE CONCLUSION COULDN'T BE Φ IS THE MEAN}}\end{itemize}

\noindent\underline{6.4}
\vspace*{4mm}

\noindent Craft (techne)

\noindent Two kind of things admit of being otherwise (or, at least, two belong to that group): what is produced, what is achieved in action
\vspace*{2mm}

\noindent So, here we get an indication that there is a serious disanalogy between craft and virtue
\vspace*{2mm}

\noindent\textbf{SO, WHAT IS THE DIFFERENCE BETWEEN PRODUCTION AND ACTION SUCH THAT THIS MAKES SENSE?}
\vspace*{2mm}

\noindent We've had three kinds of `action'

\begin{itemize}\item{Voluntary behavior (in this sense, animals and children perform action); \textbf{1111a25}}\item{Rational behavior on a decision (only adult humans); \textbf{1139a20}}\item{Rational behavior on a decision that is its own end; here, distinguished from production}\begin{itemize}\item{So, it's in this last sense that A wants to distinguish action from production}\item{This might be an example where A thinks that, in the strict sense, (3) is action, but the others are called `action' by similarity}\end{itemize}\end{itemize}

\noindent The basic idea seems clear and correct: the state of intellect or what's involved in being a good carpenter, is quite different from the state involved in excellence of practical reasoning
\vspace*{2mm}

\noindent\underline{6.5}
\vspace*{4mm}

\noindent Prudence (phronesis)
\vspace*{2mm}

\noindent Who do we call prudent people? Those who can reason well about living well in general; who grasp and can reason about the sorts of things that promote living well (not some specialized area)
\vspace*{2mm}

\noindent Prudence can't be either scientific-knowledge nor craft, since the former deals with what can't be otherwise, and people are prudent insofar as the reason well about what can be otherwise
\vspace*{2mm}

\noindent\textbf{READ 1140b5}
\vspace*{2mm}

\noindent\textbf{READ 1140b21}
\vspace*{2mm}

\noindent\underline{6.6}
\vspace*{4mm}

\noindent `Understanding' (nous): strikes me as a bad translation, since we can clearly understand things that have explanations
\vspace*{2mm}

\noindent Basic problem is that the principles don't have explanations or proofs; but scientific knowledge involves grasping a proof
\vspace*{2mm}

\noindent So, can't have episteme of the principles; this is a good point, there does seem to be a different cognitive grasp of principles than derived facts
\vspace*{2mm}

\noindent Well, by elimination, then, can't be episteme, phronesis, or techne
\vspace*{2mm}

\noindent It's also not wisdom, since mark of wisdom is ability to prove
\vspace*{2mm}

\noindent So, only thing left is nous
\vspace*{2mm}

\noindent Important: we've learned that episteme requires grasp of principles, but here we learn that grasp of principles as such isn't (partially) constitutive of episteme
\vspace*{2mm}

\noindent\underline{6.7}
\vspace*{4mm}

\noindent Wisdom (sophia)
\vspace*{2mm}

\noindent Most exact form of scientific knowledge
\vspace*{2mm}

\noindent\textbf{READ 1141a18} Wisdom = understanding + scientific knowledge
\vspace*{2mm}

\noindent Here we also seem to get a view on which quality of cognitive state in some sense depends on quality of its object
\vspace*{2mm}

\noindent Seems like it's gonna concern divine beings (1141b1)


\end{document}

