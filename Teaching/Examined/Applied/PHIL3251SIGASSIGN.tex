\documentclass[]{article}

\usepackage{fancyhdr}
 \pagestyle{fancy}
\rhead{\textsc{Scott O`Connor}}

\usepackage{lmodern}
\usepackage{amssymb,amsmath}
\usepackage{ifxetex,ifluatex}
\usepackage{fixltx2e} % provides \textsubscript
\ifnum 0\ifxetex 1\fi\ifluatex 1\fi=0 % if pdftex
  \usepackage[T1]{fontenc}
  \usepackage[utf8]{inputenc}
\else % if luatex or xelatex
  \ifxetex
    \usepackage{mathspec}
    \usepackage{xltxtra,xunicode}
  \else
    \usepackage{fontspec}
  \fi
  \defaultfontfeatures{Mapping=tex-text,Scale=MatchLowercase}
  \newcommand{\euro}{€}
\fi
% use upquote if available, for straight quotes in verbatim environments
\IfFileExists{upquote.sty}{\usepackage{upquote}}{}
% use microtype if available
\IfFileExists{microtype.sty}{%
\usepackage{microtype}
\UseMicrotypeSet[protrusion]{basicmath} % disable protrusion for tt fonts
}{}
\ifxetex
  \usepackage[setpagesize=false, % page size defined by xetex
              unicode=false, % unicode breaks when used with xetex
              xetex]{hyperref}
\else
  \usepackage[unicode=true]{hyperref}
\fi
\usepackage[usenames,dvipsnames]{color}
\hypersetup{breaklinks=true,
            bookmarks=true,
            pdfauthor={},
            pdftitle={Signature Assignment for The Examined Life PHIL 140 (2945)},
            colorlinks=true,
            citecolor=blue,
            urlcolor=blue,
            linkcolor=magenta,
            pdfborder={0 0 0}}
\urlstyle{same}  % don't use monospace font for urls
\setlength{\parindent}{0pt}
\setlength{\parskip}{6pt plus 2pt minus 1pt}
\setlength{\emergencystretch}{3em}  % prevent overfull lines
\providecommand{\tightlist}{%
  \setlength{\itemsep}{0pt}\setlength{\parskip}{0pt}}
\setcounter{secnumdepth}{0}

\title{Signature Assignment for The Examined Life PHIL 140 (3251)}
\author{Scott O’Connor}


% Redefines (sub)paragraphs to behave more like sections
\ifx\paragraph\undefined\else
\let\oldparagraph\paragraph
\renewcommand{\paragraph}[1]{\oldparagraph{#1}\mbox{}}
\fi
\ifx\subparagraph\undefined\else
\let\oldsubparagraph\subparagraph
\renewcommand{\subparagraph}[1]{\oldsubparagraph{#1}\mbox{}}
\fi

\begin{document}
\maketitle

\subsection{Final Projects}\label{final-projects}

This assignment will be completed in several parts. Please see the
syllabus for the relevant deadlines. You will work in groups of 3--4.
Your group will pick a topic from the list below. You will research the
current status of the topic in question and then use the ethical
theories introduced in Module 6 to address it.

\subsection{Pick a Topic (Class
Assignment)}\label{pick-a-topic-class-assignment}

Each group picks one of the questions from the list below. Note that
each group must pick a different topic:

\begin{enumerate}
\def\labelenumi{\arabic{enumi}.}
\tightlist
\item
  Is it permissible to abort a fetus?
\item
  Ought a doctor help her terminally ill patient die if the patient
  requests it?
\item
  Should we ever execute people for their crimes?
\item
  Should we allow people to reproduce by cloning?
\item
  Should prostitution be legal?
\item
  To what extent ought we give to charity?
\item
  May I pirate music?
\item
  Should recreational drug use be illegal?
\item
  Is it permissible to eat meat?
\item
  On whom are we allowed to perform medical trials?
\item
  Should corporations be considered persons?
\item
  Should we curb climate change for the sake of future generations?
\end{enumerate}

\subsection{Draft 1: Background
Research}\label{draft-1-background-research}

In the first part of your project, you will be collecting and thinking
through the information needed to write your final paper. Decide between
yourselves how to fairly proportion the work, but clearly indicate in
your submission how you divided up the tasks. Note that you will be
grading each others' level of participation.

Applied Ethicists are more concerned with particular, practical cases
than with more abstract theoretical questions. They want to know how, if
at all, a hospital should distribute donated organs, whether it is
permissible to bribe officials in foreign states to do business, etc.
The questions listed above are all hot topics in the United States. Your
job will be to answer them as an Applied Ethicist. In this first draft,
write 5 short documents:

\begin{enumerate}
\def\labelenumi{\arabic{enumi}.}
\tightlist
\item
  After researching articles in the NY Times from the last 3 years,
  outline the legal and regulatory status of your chosen issue.
  (400--600 words)
\item
  Identify and summarize the major public concerns relating to the issue
  at hand discussed in those articles. (400--600 words)
\item
  Explain both act utilitarianism and rule utilitarianism. How are they
  different? (400--600 words)
\item
  How would the act utilitarian answer your chosen question? Give
  reasons for your answer. (400--600 words)
\item
  How would the rule utilitarian answer your chosen question? Give
  reasons for your answer. (400--600 words)
\end{enumerate}

\subsection{Oral Presentation}\label{oral-presentation}

Each group will prepare a power point presentation on their chosen
topic. Each presentation should be 10--15 minutes long with a minimum of
5 slides and a maximum of 10.

\begin{itemize}
\tightlist
\item
  The presentation should \textbf{summarize} the findings of the five
  documents prepared in the research phase.
\item
  Each member of the group must present at least one slide.
\end{itemize}

\subsection{Final Submission}\label{final-submission}

You must prepare your final submission by yourself, not in groups. You
can refer to the documents prepared by your group as background
research, but you must cite those group members who were responsible for
those documents.

\textbf{Prompt:} Compare and contrast how the act and rule utilitarian
would answer your chosen question. Which theory has the most unsettling
consequences? Give reasons for your answer. (1250--1500 words)

\subsection{Grade Breakdown}\label{grade-breakdown}

\begin{enumerate}
\def\labelenumi{\arabic{enumi}.}
\tightlist
\item
  Background Research: 10 pts.
\item
  Oral Presentation: 10 pts.
\item
  Final Submission: 10 pts.
\item
  Peer Evaluation of Work Contribution: 5 pts.
\item
  \textbf{Total Points: 35pts}
\end{enumerate}

The first two grades will be awarded to your group contribution, the
third to your individual final paper, and the fourth by averaging each
group member's evaluation of your level of participation.

\end{document}
