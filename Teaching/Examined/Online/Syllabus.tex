\documentclass[article,oneside]{memoir}

%%% custom style file with standard settings for xelatex and biblatex. Note that when [minion] is present, we assume you have minion pro installed for use with pdflatex.
%\usepackage[minion]{org-preamble-pdflatex} 

%%% alternatively, use xelatex instead
\usepackage{org-preamble-xelatex} 



\def\myauthor{Author}
\def\mytitle{Title}
\def\mycopyright{\myauthor}
\def\mykeywords{}
\def\mybibliostyle{plain}
\def\mybibliocommand{}
\def\mysubtitle{}
\def\myaffiliation{NJCU}
\def\myaddress{Phil 101}
\def\myemail{soconnor@njcu.edu}
\def\myweb{\href{http://scoconno.github.io/Teaching/Examined/Online}{http://scoconno.github.io/Teaching/Examined/Online}}
\def\myphone{}
\def\myversion{}
\def\myrevision{}
\def\myaffiliation{NJCU}
\def\myauthor{Dr. Scott O'Connor}
\def\mykeywords{}
\def\mysubtitle{Syllabus}
\def\mytitle{{\normalsize \myweb \newline} \HUGE Persons \& Problems}


\begin{document}

%%% If using xelatex and not pdflatex
%%% xelatex font choices
\defaultfontfeatures{}
\defaultfontfeatures{Scale=MatchLowercase}    
% You will need to buy these fonts, change the names to fonts you own, or comment out if not using xelatex.      
\setromanfont[Mapping=tex-text]{Minion Pro} 
\setsansfont[Mapping=tex-text]{Myriad Pro} 
\setmonofont[Mapping=tex-text,Scale=0.8]{Georgia} 

%% blank label items; hanging bibs for text
%% Custom hanging indent for vita items
\def\ind{\hangindent=1 true cm\hangafter=1 \noindent}
\def\labelitemi{$\cdot$}
%\renewcommand{\labelitemii}{~}

%% RCS info string for version tracking
\chapterstyle{article-3}  % alternative styles are defined in latex-custom-kjh/needs-memoir/
%\pagestyle{kjh}

\title{\LARGE \mytitle}     
\author{\Large\myauthor \newline \footnotesize\texttt{\noindent Office hours: \href{http://scoconno.github.io/Contact/Office/}{http://scoconno.github.io/Contact/Office/}}}
\date{1/19/2016--5/9/2016}

\published{\textbf{Phil 101-1 (1701), 3 credits, Spring 2016, Online}}

\maketitle

%\thispagestyle{kjhgit}

% Copyright Page
%\textcopyright{} \mycopyright


%
% Main Content
%

\section{Copyright}
The materials used in this class, including, but not limited to, lectures, exams, quizzes, and homework assignments are copyright protected works.  Any unauthorized copying of the class materials or recording of lectures is a violation of federal law and may result in disciplinary actions being taken against the student.  Additionally, the sharing of class materials without the specific, express approval of the instructor may be a violation of the University's Student Honor Code and an act of academic dishonesty, which could result in further disciplinary action.  This includes, among other things, uploading class materials to websites for the purpose of sharing those materials with other current or future students. 

\section{Catalog Description}

This course teaches students to identify and evaluate those beliefs that guide their thoughts and actions. Reflecting on different sources, students identify those philosophical beliefs that play a role in their own lives. By developing their critical thinking skills, they learn how to clarify, systematize, and assess these beliefs. 

\section{Course Description}

Does God exist? Are you free? Why live? What should you do with your life?  In this course, we'll be asking some of these deep philosophical questions. We begin by discussing the meaning of life, especially why some philosophers have connected a meaningful life with God's existence. This raises the question as to whether God exists. We will examine some classic arguments for the existence of God as well as concerns that God's existence is incompatible with the existence of evil. Many respond to the problem of evil by claiming that evil is a by-product of our free-will, a gift endowed by God to use as we see fit. But are we free? We will discuss why some think our actions are completely pre-determined by causal factors outside of our control. If they are right, free-will is a mere illusion. This raises deep questions about the nature of moral responsibility; can you be held responsible for an action that was out of your control? In the final part of the course, we will ask what determines the moral character of our actions. Do the ends justify the means? After studying the main ethical theories, you will get a chance to work in groups to apply them to a current controversy of your choosing, e.g., the death penalty, euthanasia, abortion, etc. 

\section{Learning Objectives}

Upon completing this course, students will be able to (i) read
philosophical texts, (ii) clearly and charitably explain viewpoints that
are not their own, (iii) think critically and philosophically, (iv)
write well-structured prose in which they clearly state a thesis and
persuasively defend it, (v) demonstrate an understanding of several core
philosophical topics, (vi) manage their studies in a responsible and timely manner. 


\section{Required Textbook}

\begin{itemize}
\item
  \href{http://www.amazon.com/Philosophy-Here-Now-Powerful-Everyday/dp/0190207035/ref=dp_ob_title_bk}{`Philosophy  Here and Now: Powerful Ideas in Everyday Life', 2nd Edition, by Lewis Vaughn}  (Available in the campus book store and online retailers)
\end{itemize}


\section{Course Website}
There is both a Blackboard site and website for this course (link on first page). Clicking the first link on the left panel within the Blackboard site will bring you to the course website. All assignments will be submitted through Blackboard. Readings, notes, etc. will be posted on the course website. Note that Blackboard difficulties are rare and automatically reported to instructors. Under no circumstance will a student's report of a Blackboard difficulty be reason for an extension. It is your responsibility to contact Blackboard support for help.




\section{Requirements}

\begin{itemize}
\item \textit{Workload:} Expect to spend an average of 5--6 hours per week  completing the readings and assignments.

\item \textit{Course evaluations} completed online. 5 points extra credit for successful completion.

\item \textit{Reading quizzes} administered through Blackboard. 8 will be assigned. You must complete 5. If you complete more than 5, the lowest grades will be dropped. 

\item \textit{Short essays} submitted through Blackboard. 5 will be assigned. You must complete 3. If you complete more than 3, the lowest grades will be dropped. 

\item \textit{Independent project} submitted through Blackboard. You will select and write a paper of 1200--1500 words about a controversial ethical topic. 
 
\item \textit{Grade Distribution:} Quiz---10 points each (50 total);  Short Writing Assignments---10 points each (30 total); Final Project---20 points 


\item \textit{Grade Breakdown:}

 \begin{tabular}{ | l | l | p{2cm} | l | l | }
    \hline 
96--100 & A  & &  77--79 &  C+ \\  
90--95 & A- & &  73--76 & C \\
87-89 & B+ &  &  70--72 & C- \\ 
83--86 & B  & &  60--69 & D\\
80--82 & B - & & 0--59 & F\\ \hline
    \end{tabular}


\end{itemize}


\section{Policies}

\begin{itemize}

\item \textbf{Student Responsibility:} This syllabus outlines the required text, assignments, requirements, and policies for this course. By taking this course, you agree to read this syllabus and be bound by those requirements and policies. 

 \item \textit{Academic Integrity:} All the work you turn in (including papers, drafts, and discussion board posts) must be written by you specifically for this course. It must originate with you in form and content with all contributory sources fully and specifically acknowledged. Being a student at NJCU requires you to follow \href{http://www.njcu.edu/uploadedFiles/About_NJCU/Governance_and_Organization/University_Senate/Policies/Academic\%20INTEGRITY\%20POLICY\%20FINAL\%202-04.pdf}{NJCU's Academic Integrity Policy.} Penalties for violations are as follows: 1st infraction will result in a 0 for the assignment.  2nd infraction will result in a 0 for the entire course \& application for permanent record on student's transcript. (Repeated violations can lead to expulsion from NJCU). 


\item \textit{Communication:} To comply with Federal Privacy Laws (FERPA) and NJCU policies, all communication will be through Blackboard and/or official NJCU e-mail. Check both your NJCU e-mail and Blackboard daily. For further information see \href{http://scoconno.github.io/Contact/}{http://scoconno.github.io/Contact/}.



\item \textit{Format for Written Work:} Submit work to Blackboard as a Microsoft Word file. All work must be typed. Your papers should be in 12-point Times New Roman font, double-spaced with margins set to one inch on all sides.



\item \textit{Grading:} Grades will be available within 1-2 weeks of an assignment being submitted. See: \href{http://scoconno.github.io/Teaching/Grading}{http://scoconno.github.io/Teaching/Grading} for further information.


\item \textit{Late work \& Make-up Policy:} See the assignment schedule below. No make-ups or late work accepted under any circumstances. No exceptions under any imaginable circumstances. 


\item \textit{Statement for students with disabilities:} If you are a student
with a disability and wish to receive consideration for reasonable
accommodations, please register with the Office of Specialized Services
and Supplemental Instruction (OSS/SI). To begin this process, complete
the registration form available on the OSS/SI website at
\href{http://www.njcu.edu/Specialized_Services.aspx}{www.njcu.edu/Specialized\_Services.aspx}
(listed under Student Resources-Forms). Contact OSS/SI at 201-200-2091
or visit the office in Karnoutsos Hall, Room 102 for additional
information.

\end{itemize}



\section{Weekly Course Schedule}
Dates refer to the first day of the week. Readings marked with a `**' can be found on the course website. All other listed readings can be found in the required textbook. Changes to the syllabus will be announced through Blacboard and \emph{via} your NJCU email address.

\begin{description}

\item[Module 1:] \href{http://scoconno.github.io/Teaching/Examined/Intro/}{Introduction}
\begin{enumerate}
\item[\textit{Week 1}] What is Philosophy?
\begin{enumerate}
\item `The Trial and Death of Socrates', Plato, pp.44-53
\item Ch.1.1--1.2.
\end{enumerate}
\end{enumerate}

\item[Module 2:] \href{http://scoconno.github.io/Teaching/Examined/CT/}{Thinking Philosophically}
\begin{enumerate}
\item[\textit{Week 2}] Critical Thinking Tools 
\begin{enumerate}
\item Ch.1.3
\item **Worksheet
\end{enumerate}
\end{enumerate}

\item[Module 3:] \href{http://scoconno.github.io/Teaching/Examined/Meaning/}{The Meaning of Life}

\begin{enumerate}
\item[\textit{Week 3}] Pessimism 
\begin{enumerate}
\item 'The Good Brahmin', Voltaire, pp.408--409
\item Ch.9.1--9.2 
\item **`A Confession', Leo Tolstoy (optional)
\end{enumerate}

\item[\textit{Week 4}] Optimism with God
\begin{enumerate}
\item Ch.9.3, 'Meaning from Above'
\item **`A Confession', Leo Tolstoy (optional)

\end{enumerate}

\item[\textit{Week 5}] Optimism without God
\begin{enumerate}
\item Ch.9.3 `Meaning from Below'
\item Ch.2.6, pp.106--112
\end{enumerate}
\end{enumerate}


\item[Module 4: ] \href{http://scoconno.github.io/Teaching/Examined/God/}{The Existence of God}

\begin{enumerate}

\item[\textit{Week 6}] The Design Argument
\begin{enumerate}
\item **`Intelligent Design Has No Place in the Science Curriculum', Harold Morowitz, Robert Hazen, and James Trefil
 \item **`Design for Living', Michael J. Behe
\item Ch.2.1-2.2. 

 \end{enumerate}
 
\item[\textit{Week 7}] Further Arguments for God's Existence
\begin{enumerate}
\item `The Star', Arthur C. Clarke, pp.126--128
\item Ch.2.2
\end{enumerate}
\end{enumerate}
\item[\textbf{Spring Break}]\

\begin{enumerate}
\item[\textit{Week 8}] The Problem of Evil 
\begin{enumerate}
\item Short story to be announced.
\item Ch.2.3
\end{enumerate}
\end{enumerate}


\item[Module 5:] \href{http://scoconno.github.io/Teaching/Examined/FreeWill/}{Free Will}

\begin{enumerate}

\item[\textit{Week 9}] Free Will and Determinism
\begin{enumerate}
\item `A Little Omniscience Goes a Long Way', Thomas Davis, pp.258--261
\item Ch.5.1--5.2
\end{enumerate}

\item[\textit{Week 10}] Compatibilism
\begin{enumerate}
\item `Science and Free Will', p.244
\item Ch.5.3
\end{enumerate}
\end{enumerate}

\item[Module 6:] \href{http://scoconno.github.io/Teaching/Examined/Ethics/}{Normative Ethics}
\begin{enumerate}

\item[\textit{Week 11}] Cultural Relativism
\begin{enumerate}
\item **`Kirinyaga;, Mike Resnik, ch.1
\item Ch.3.1--3.2
\end{enumerate}
\item[\textit{Week 12}]  Consequentialism 
\begin{enumerate}
\item  `The Ones Who Walk Away from Omelas', Ursula Le Guin, pp.191--194 
\item Ch.3.3
\end{enumerate}
\item[\textit{Week 13}]  Deontology
\begin{enumerate}
\item **`Billy Budd', Herman Melville, (extracts)
\item Ch.3.4

\end{enumerate}

\item[\textit{Week 14}] Virtue Ethics
\begin{enumerate}
\item Ch.3.4--3.5
\end{enumerate}
\end{enumerate}

\item[Module 7:] \href{http://scoconno.github.io/Teaching/Examined/Applied/}{Applied Ethics}
\begin{enumerate}

\item[\textit{Week 15}] Independent Project
\begin{enumerate}
\item Sundry
\end{enumerate}
\end{enumerate}

\end{description}





\section{ Assignment Schedule}
Dates refer to the due date. All assignments must be submitted through Blackboard by 11:59pm. No late work accepted. No exceptions. You must complete 5 quizzes, 3 short essays, and the final independent project. If you complete more than the required number, the lowest grades will be dropped.
\begin{itemize}
\item \textit{02/01/2016,} Quiz 1 
\item \textit{02/08/2016,} Quiz 2 
\item \textit{02/15/2016,} \href{http://scoconno.github.io/Teaching/Examined/Meaning/SW1/}{Short essay 1---The Meaning of Life} 
\item \textit{02/22/2016,} Quiz 3 
\item \textit{02/29/2016,} \href{http://scoconno.github.io/Teaching/Examined/God/SW2/}{Short essay 2---God's Existence}
\item \textit{03/07/2016,} Quiz 4
\item \textit{03/21/2016,}  \href{http://scoconno.github.io/Teaching/Examined/God/SW3}{Short essay 3---The Problem of Evil}
\item \textit{03/28/2016,} Quiz 5 
\item \textit{04/04/2016,}  \href{http://scoconno.github.io/Teaching/Examined/FreeWill/SW/}{Short essay 4---Free will}
\item \textit{04/11/2016,} Quiz 6
\item \textit{04/18/2016,} Quiz 7
\item \textit{04/25/2016,} \href{http://scoconno.github.io/Teaching/Examined/Ethics/DQ3/}{Short essay 5---Ethics}
\item \textit{05/02/2016,} Quiz 8
\item \textit{5/09/2016,} \href{http://scoconno.github.io/Teaching/Examined/Applied/Essay/}{Independent Project}
\end{itemize}






%% Uncomment if you want a printed bibliography.
%\printbibliography 

\end{document}