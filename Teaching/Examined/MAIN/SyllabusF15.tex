\documentclass[article,oneside]{memoir}
%%% custom style file with standard settings for xelatex and biblatex. Note that when [minion] is present, we assume you have minion pro installed for use with pdflatex.
%\usepackage[minion]{org-preamble-pdflatex} 

%%% alternatively, use xelatex instead
\usepackage{org-preamble-xelatex} 



\def\myauthor{Author}
\def\mytitle{Title}
\def\mycopyright{\myauthor}
\def\mykeywords{}
\def\mybibliostyle{plain}
\def\mybibliocommand{}
\def\mysubtitle{}
\def\myaffiliation{NJCU}
\def\myaddress{140 Phil}
\def\myemail{soconnor@njcu.edu}
\def\myweb{\href{http://scoconno.github.io/Teaching/Examined/MAIN}{http://scoconno.github.io/Teaching/Examined/MAIN}}
\def\myphone{}
\def\myversion{}
\def\myrevision{}
\def\myaffiliation{NJCU}
\def\myauthor{Dr. Scott O'Connor}
\def\mykeywords{}
\def\mysubtitle{Syllabus}
\def\mytitle{{\normalsize Phil 140 (2945), 3 credits, Fall 2015, M\&W, 9:55am, K 554\newline} \HUGE The Examined Life}


\begin{document}

%%% If using xelatex and not pdflatex
%%% xelatex font choices
\defaultfontfeatures{}
\defaultfontfeatures{Scale=MatchLowercase}    
% You will need to buy these fonts, change the names to fonts you own, or comment out if not using xelatex.      
\setromanfont[Mapping=tex-text]{Georgia} 
\setsansfont[Mapping=tex-text]{Georgia} 
\setmonofont[Mapping=tex-text,Scale=0.8]{Georgia} 

%% blank label items; hanging bibs for text
%% Custom hanging indent for vita items
\def\ind{\hangindent=1 true cm\hangafter=1 \noindent}
\def\labelitemi{$\cdot$}
%\renewcommand{\labelitemii}{~}

%% RCS info string for version tracking
\chapterstyle{article-3}  % alternative styles are defined in latex-custom-kjh/needs-memoir/
\pagestyle{kjh}

\title{\LARGE \mytitle}     
\author{\Large\myauthor \newline \footnotesize\texttt{\noindent\myweb}}
\date{09/02/2015--12/14/2015}

\published{\,}

\maketitle

%\thispagestyle{kjhgit}

% Copyright Page
%\textcopyright{} \mycopyright


%
% Main Content
%

\section{Copyright}
The materials used in this class, including, but not limited to, lectures, exams, quizzes, and homework assignments are copyright protected works.  Any unauthorized copying of the class materials or recording of lectures is a violation of federal law and may result in disciplinary actions being taken against the student.  Additionally, the sharing of class materials without the specific, express approval of the instructor may be a violation of the University's Student Honor Code and an act of academic dishonesty, which could result in further disciplinary action.  This includes, among other things, uploading class materials to websites for the purpose of sharing those materials with other current or future students. 

\section{Catalog Description}

This course teaches students to identify and evaluate those beliefs that guide their thoughts and actions. Reflecting on different sources, students identify those philosophical beliefs that play a role in their own lives. By developing their critical thinking skills, they learn how to clarify, systematize, and assess these beliefs. 

\section{Course Description}

Does God exist? Are you free? Why live? What should you do with your life?  In this course, we'll be asking some of these deep philosophical questions. We begin by discussing the meaning of life, especially why some philosophers have connected a meaningful life with God's existence. This raises the question as to whether God exists. We will examine some classic arguments for the existence of God as well as concerns that God's existence is incompatible with the existence of evil. Many respond to the problem of evil by claiming that evil is a by-product of our free-will, a gift endowed by God to use as we see fit. But are we free? We will discuss why some think our actions are completely pre-determined by causal factors outside of our control. If they are right, free-will is a mere illusion. This raises deep questions about the nature of moral responsibility; can you be held responsible for an action that was out of your control? In the final part of the course, we will ask what determines the moral character of our actions. Do the ends justify the means? After studying the main ethical theories, you will get a chance to work in groups to apply them to a current controversy of your choosing, e.g., the death penalty, euthanasia, abortion, etc. 

\section{Learning Objectives}

Upon completing this course, students will be able to (i) read
philosophical texts, (ii) clearly and charitably explain viewpoints that
are not their own, (iii) think critically and philosophically, (iv)
write well-structured prose in which they clearly state a thesis and
persuasively defend it, (v) demonstrate an understanding of several core
philosophical topics, (vi) manage their studies in a responsible and timely manner. 

\section{General Education Information} 
Successfully completing this course satisfies one Tier 1 Language, Literary, and Cultural Studies requirement. It teaches the following two University-wide Learning Goals: (1) Critical Thinking and Problem Solving, (2) Written Communication. For further information about the General Education Program see \href{http://www.njcu.edu/cas/general-education/}{http://www.njcu.edu/cas/general-education/}.

 \section{First Year Experience Program}
Students in this course are also enrolled in the same section of English Composition I--ALP (ECI--ALP) with Dr. Joshua Fausty.  This section of PHIL 140 coordinates with ECI--ALP, in part, by giving students opportunities to practice and reinforce the reading, writing, argument-making and analytical skills necessary to perform well in all other college courses. 


\section{Required Textbook}

By 09/09/2015, the following textbook,  which you should bring to class, must be purchased or rented:

\begin{itemize}
\item
  \href{http://www.amazon.com/Philosophy-Here-Now-Powerful-Everyday/dp/0199765227}{Philosophy
  Here and Now: Powerful Ideas in Everyday Life', by Lewis Vaughn}
  (Available in the campus book store and online retailers)
\end{itemize}

\section{Course Website}
There is both a Blackboard site and website for this course (link on first page). Clicking the first link on the left panel within the Blackboard site will bring you to the course website. All assignments will be submitted through Blackboard. Readings, notes, etc. will be posted on the course website. Note that Blackboard difficulties are rare and automatically reported to instructors. Under no circumstance will a student's report of a Blackboard difficulty be reason for an extension. It is your responsibility to contact Blackboard support for help.


\section{Requirements}

\begin{itemize}
\item \textit{Workload:} Expect to spend an average of 5--6 hours per week  completing the readings and assignments.

\item \textit{Attendance:} Roll call will be taken. 0.5 point will be awarded per class up to a maximum of 10 points. Points will not be awarded during weeks 1 \& 2. 

\item \textit{1 Critical Thinking Quiz} administered in class. 

\item \textit{2 Short Writing Assignments} submitted through Blackboard. 

\item \textit{2 Essays} submitted through Blackboard. 

\item \textit{Final group project} consisting of an oral presentation and a written submission in two drafts. Each member must present at least one slide. Groups will grade their members' participation to ensure equal participation. 

\item \textit{Grade Distribution:}  Quiz---5 points; Attendance---0.5 points per class (max 10 total);  Short Writing Assignments---10 points each (20 total); Essays---20 points each (40 points total); Presentation---10 oral, 20 points written, 5 points peer evaluation (35 points total).

\item \textit{Grade Breakdown:}

 \begin{tabular}{ | l | l | p{2cm} | l | l | }
    \hline 
96--\textbf{110} & A  & &  77--79 &  C+ \\  
90--95 & A- & &  73--76 & C \\
87-89 & B+ &  &  70--72 & C- \\ 
83--86 & B  & &  60--69 & D\\
80--82 & B - & & 0--59 & F\\ \hline
    \end{tabular}


\end{itemize}




\section{Policies}

\begin{itemize}

\item \textbf{Student Responsibility:} This syllabus outlines the required text, assignments, requirements, and policies for this course. By taking this course, you agree to read this syllabus and be bound by those requirements and policies. 

 \item \textit{Academic Integrity:} All the work you turn in (including papers, drafts, and discussion board posts) must be written by you specifically for this course. It must originate with you in form and content with all contributory sources fully and specifically acknowledged. Being a student at NJCU requires you to follow \href{http://www.njcu.edu/uploadedFiles/About_NJCU/Governance_and_Organization/University_Senate/Policies/Academic\%20INTEGRITY\%20POLICY\%20FINAL\%202-04.pdf}{NJCU's Academic Integrity Policy.} Penalties for violations are as follows: 1st infraction will result in a 0 for the assignment.  2nd infraction will result in a 0 for the entire course \& application for permanent record on student's transcript. (Repeated violations can lead to expulsion from NJCU). 

\item \textit{Attendance:} You are considered absent if you are (i) not present during roll call, (ii) leave early, (iii) leave without permission, or (iv) leave for an extended period of time. No excuses. No exceptions.



\item \textit{Communication:} To comply with Federal Privacy Laws (FERPA) and NJCU policies, all communication will be through Blackboard and/or official NJCU e-mail. Check both your NJCU e-mail and Blackboard daily. For further information see \href{http://scoconno.github.io/Contact/}{http://scoconno.github.io/Contact/}.

\item \textit{Conduct:} Distracting and disrespectful behaviors, including but not limited to eating, leaving your seat, talking out of turn, and aggression are prohibited. Penalties include, but are not limited to, a loss of attendance points for the day of violation. Repeat offenders will be reported to the Dean of Students. 

\item \textit{Electronic devices:} Use of electronic device, including, but not limited, to smartphones, dictaphones, tablets, and laptops, is prohibited. Recording a lecture is in violation of Copyright. Penalties include, but are not limited to, a loss of attendance points for the day of violation. Repeat offenders will be reported to the Dean of Students.


\item \textit{Format for Written Work:} Submit work to Blackboard either as a rich text or Microsoft Word file. All work must be typed. Your papers should be in 12-point Times New Roman font, double-spaced with margins set to one inch on all sides. If hard copies are requested, please staple or paperclip copies of papers and drafts.


\item \textit{General Education Program Assessment:} General Education courses participate in programmatic assessment of the six University-wide student learning goals. They include instruction in, and assessment of, at least two of these learning goals. Signature assignments, which may include document, picture, sound, or video files, are uploaded to a secure server for anonymous distribution to the NJCU assessment team, which scores them using approved program rubrics. While instructors also grade their own students’ signature assignments, which count toward the course grade, assessment team results are aggregated to provide information about the Gen Ed program as a whole. Your name will not be included in any programmatic assessment data.

\item \textit{Grading:} Grades will be available within 1--2 weeks of an assignment being submitted. See: \href{http://scoconno.github.io/Teaching/Grading}{http://scoconno.github.io/Teaching/Grading} for further information.


\item \textit{Late work \& Make-up Policy:} See the assignment schedule below. No make-ups or late work accepted under any \textbf{imaginable circumstances.} No exceptions. This policy will be applied equally and fairly to all students. \emph{Requests for special treatment will be ignored.}


\item \textit{Statement for students with disabilities:} If you are a student with a disability and wish to receive consideration for reasonable accommodations, please register with the Office of Specialized Services and Supplemental Instruction (OSS/SI). To begin this process, complete the registration form available on the OSS/SI website at
\href{http://www.njcu.edu/Specialized_Services.aspx}{www.njcu.edu/Specialized\_Services.aspx}
(listed under Student Resources-Forms). Contact OSS/SI at 201-200-2091
or visit the office in Karnoutsos Hall, Room 102 for additional
information.

\end{itemize}



\section{Weekly Course Schedule}
Dates refer to the first day of the week. Complete the readings before the first class of the week. Readings marked with a `**' can be found on the course website. Handouts can be found under the relevant modules on the course website. All other listed readings can be found in the required textbook. Changes to the syllabus will be announced in class and \emph{via} your NJCU email address.

\begin{description}

\item[Module 1:] \href{http://scoconno.github.io/Teaching/Examined/Intro/}{Introduction}
\begin{enumerate}

\item \textit{08/31/2015,} What is Philosophy?
\begin{enumerate}
\item Ch.1.1--1.2. 
\item `The Trial and Death of Socrates', Plato, pp.44-53
\end{enumerate}



\item \textit{09/07/2015,} Continued

\begin{enumerate}
\item Ch.1.1--1.2.
\item `The Trial and Death of Socrates', Plato, pp.44-53
\end{enumerate}
\end{enumerate}



\item[Module 2:] \href{http://scoconno.github.io/Teaching/Examined/CT/}{Thinking Philosophically}
\begin{enumerate}

\item \textit{09/14/2015,} Critical Thinking Tools
\begin{enumerate}
\item Ch.1.3
\item **Worksheet
\end{enumerate}

\end{enumerate}

\item[Module 3:] \href{http://scoconno.github.io/Teaching/Examined/Meaning/}{The Meaning of Life}

\begin{enumerate}
\item \textit{09/21/2015,} Pessimism 
\begin{enumerate}
\item 'The Good Brahmin', Voltaire, pp.381--382
\item Ch.8.1--8.2 
\item **`A Confession', Leo Tolstoy (optional)
\end{enumerate}

\item \textit{09/28/2015,} Optimism
\begin{enumerate}
\item Ch.8.3
\item **TBD 
\end{enumerate}
\end{enumerate}

\item[Module 4: ] \href{http://scoconno.github.io/Teaching/Examined/God/}{The Existence of God}

\begin{enumerate}
\item \textit{10/05/2015,} The Design Argument
\begin{enumerate}
\item Ch.2.1-2.2. 
\item **`Intelligent Design Has No Place in the Science Curriculum', Harold Morowitz, Robert Hazen, and James Trefil
 \item **`Design for Living', Michael J. Behe
 \end{enumerate}
 
\item \textit{10/12/2015,}Further Arguments for God's Existence
\begin{enumerate}
\item Ch.2.2
\item `The Star', Arthur C. Clarke, pp.118--120
\end{enumerate}

\item \textit{10/19/2015,} God: The Problem of Evil 
\begin{enumerate}
\item Ch.2.3.
\item ``A Little Omniscience Goes a Long Way'', Thomas Davis, pp.248--250
\item News Report TBD. 
\end{enumerate}
\end{enumerate}

\item[Module 5:] \href{http://scoconno.github.io/Teaching/Examined/FreeWill/}{Free Will}

\begin{enumerate}
\item \textit{10/26/2015,}  Causal Determinism
\begin{enumerate}
\item Ch.5.1--5.2
\item **`Do Androids Dream of Electric Sheep', Philip, K. Dick (extract TBD)
\end{enumerate}

\item \textit{11/03/2015,} Compatibilism
\begin{enumerate}
\item Ch.5.3
\item `Science and Free Will', p.234
\end{enumerate}

\end{enumerate}

\item[Module 6:] \href{http://scoconno.github.io/Teaching/Examined/Ethics/}{Normative Ethics}
\begin{enumerate}

\item \textit{11/09/2015,} Relativism, \textbf{no class 11/11}
\begin{enumerate}
\item Ch.3.1--3.2.
\item **`Kirinyaga', Mike Resnik, ch.1
\end{enumerate}
\item \textit{11/16/2015,} Consequentialism \& Deontology 
\begin{enumerate}
\item Ch.3.3--3.4
\item  `The Ones Who Walk Away from Omelas', Ursula Le Guin, pp.181--184 
\end{enumerate}
\item \textit{11/23/2015,} Deontology \& Virtue Ethics
\begin{enumerate}
\item Ch.3.4--3.5
\end{enumerate}

\end{enumerate}

\item[Module 7:] \href{http://scoconno.github.io/Teaching/Examined/Applied/}{Applied Ethics}
\begin{enumerate}
\item \textit{11/30/2015,} Group Projects
\item \textit{12/07/2015,} Group Projects
\item \textit{12/14/2015,} Group Projects
\end{enumerate}

\end{description}





\section{Assignment Schedule}
Dates refer to the due date. All assignments must be submitted through Blackboard by 1:00pm. No late work accepted. No make-ups. No exceptions. 

\begin{enumerate}
\item \textit{09/21/2015,} Critical Thinking Quiz (in class)
\item \textit{10/05/2015,} \href{http://scoconno.github.io/Teaching/Examined/Meaning/SW1/}{SW1---The Meaning of Life}
\item \textit{10/19/2015,} \href{http://scoconno.github.io/Teaching/Examined/God/SW2/}{SW2---God's Existence}
\item \textit{11/09/2015,} \href{http://scoconno.github.io/Teaching/Examined/God/Essay1}{Essay 1---The Problem of Evil}
\item \textit{11/23/2015,} \href{http://scoconno.github.io/Teaching/Examined/Ethics/Essay/}{Essay 2---Ethics}

 \item \textit{12/07/2015,} \href{(http://scoconno.github.io/Teaching/Examined/Applied/MAINSIG.pdf/)}{Final Project--Research Documents in-class peer review}

\item \textit{12/09/2015,} \href{(http://scoconno.github.io/Teaching/Examined/Applied/MAINSIG.pdf/)}{Final Project Research Documents due through Blackboard by 9:55am} 

\item \textit{12/14/2015,} \href{(http://scoconno.github.io/Teaching/Examined/Applied/MAINSIG.pdf/)}{Final Project--Group Presentations}


\item \textit{12/18/2015,} \href{(http://scoconno.github.io/Teaching/Examined/Applied/MAINSIG.pdf/)}{Final Project--Final Submission Due}

\end{enumerate}




%% Uncomment if you want a printed bibliography.
%\printbibliography 

\end{document}