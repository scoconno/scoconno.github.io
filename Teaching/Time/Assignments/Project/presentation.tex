\documentclass[]{article}

\usepackage{fancyhdr}
 \pagestyle{fancy}
\rhead{\textsc{Scott O`Connor}}

\usepackage{lmodern}
\usepackage{amssymb,amsmath}
\usepackage{ifxetex,ifluatex}
\usepackage{fixltx2e} % provides \textsubscript
\ifnum 0\ifxetex 1\fi\ifluatex 1\fi=0 % if pdftex
  \usepackage[T1]{fontenc}
  \usepackage[utf8]{inputenc}
\else % if luatex or xelatex
  \ifxetex
    \usepackage{mathspec}
    \usepackage{xltxtra,xunicode}
  \else
    \usepackage{fontspec}
  \fi
  \defaultfontfeatures{Mapping=tex-text,Scale=MatchLowercase}
  \newcommand{\euro}{€}
\fi
% use upquote if available, for straight quotes in verbatim environments
\IfFileExists{upquote.sty}{\usepackage{upquote}}{}
% use microtype if available
\IfFileExists{microtype.sty}{%
\usepackage{microtype}
\UseMicrotypeSet[protrusion]{basicmath} % disable protrusion for tt fonts
}{}
\ifxetex
  \usepackage[setpagesize=false, % page size defined by xetex
              unicode=false, % unicode breaks when used with xetex
              xetex]{hyperref}
\else
  \usepackage[unicode=true]{hyperref}
\fi
\usepackage[usenames,dvipsnames]{color}
\hypersetup{breaklinks=true,
            bookmarks=true,
            pdfauthor={},
            pdftitle={Independent Project-Stage 3},
            colorlinks=true,
            citecolor=blue,
            urlcolor=blue,
            linkcolor=magenta,
            pdfborder={0 0 0}}
\urlstyle{same}  % don't use monospace font for urls
\setlength{\parindent}{0pt}
\setlength{\parskip}{6pt plus 2pt minus 1pt}
\setlength{\emergencystretch}{3em}  % prevent overfull lines
\providecommand{\tightlist}{%
  \setlength{\itemsep}{0pt}\setlength{\parskip}{0pt}}
\setcounter{secnumdepth}{0}

\title{Independent Project-Stage 3}
\author{Scott O’Connor}


% Redefines (sub)paragraphs to behave more like sections
\ifx\paragraph\undefined\else
\let\oldparagraph\paragraph
\renewcommand{\paragraph}[1]{\oldparagraph{#1}\mbox{}}
\fi
\ifx\subparagraph\undefined\else
\let\oldsubparagraph\subparagraph
\renewcommand{\subparagraph}[1]{\oldsubparagraph{#1}\mbox{}}
\fi

\begin{document}
\maketitle

\subsection{Independent Project-Stage
3}\label{independent-project-stage-3}

\subsubsection{Introduction}\label{introduction}

Great progress! You have a question. You have identified a dozen or so
readings that are relevant to that question. You have explained the
background to your questions and sketched out some standard answers in
the literature. Congratulate yourself on how much research you have now
done.

Now that you have surveyed the literature, it is time for you to form
your own opinions on the question. Do you think one of the answers that
has been defended is correct? If so, why that one as opposed to another?
Do you think there is an ignored alternative? Now is the time for you to
advance the conversation.

I hope that you are beginning to realize that research is a very gradual
process. Forming your own views is also a gradual process. Researchers
don't come up with something new overnight. They take the germ of an
idea, mull over it, talk about it with peers and colleagues, and slowly
develop it.

This presentation provides you the opportunity to formulate your views
before writing your final paper. Researchers often do present their
findings at conferences before publishing them. So too you will be
presenting to your colleagues your views on your topic.

Since you might not have given a presentation before, I am going to give
you very structured instructions.

\subsubsection{Requirements}\label{requirements}

Write 6 slides. Each slide should contain a striking picture, or quote,
or graph, or other visual cue. It must also provide in bullet form a
summary of the main point of the slide. You will be using your slides as
cues---do not put the entire text on the slide. The 6 slides must be
written as follows (each number corresponds to a slide):

\begin{enumerate}
\def\labelenumi{\arabic{enumi}.}
\tightlist
\item
  Relevant background on your topic.
\item
  Question raised.
\item
  Answer 1.
\item
  Answer 2.
\item
  Your novel contribution introduced.
\item
  Evidence/support/argument for your novel contribution.
\end{enumerate}

\subsubsection{Grading}\label{grading}

This assignment is worth 10 points. The breakdown is as follows:

\begin{itemize}
\tightlist
\item
  1 point per slide (6 points total)
\item
  4 points peer review (average response)
\end{itemize}

\subsubsection{Schedule}\label{schedule}

\textbf{Mon 4/25}

\begin{itemize}
\tightlist
\item
  Khadiyah
\item
  Benjamin
\item
  Anthony
\item
  David
\item
  Kelechi
\item
  Claudia
\end{itemize}

\textbf{Wed 4/27}

\begin{itemize}
\tightlist
\item
  Gabriella
\item
  Megan
\item
  Shaira
\item
  Yan
\item
  Rachel
\item
  Erika
\item
  Scott
\end{itemize}

\textbf{Mon 5/2}

\begin{itemize}
\tightlist
\item
  Caitlin
\item
  Joseph
\item
  Rodney
\item
  Alysha
\item
  Laura
\item
  Zoe
\end{itemize}

\end{document}
