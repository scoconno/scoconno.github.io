\documentclass[10pt, oneside]{article}

\usepackage[T1]{fontenc}

\usepackage{fancyhdr}
 \pagestyle{fancy}
 \lhead{\textbf{\textsc{\small Generation and Destruction (Phil 1111.107)\\Scott O'Connor}}}
 \chead{}
 \rhead{\large\textbf{\textsc{Bibliography}}}
 \lfoot{\footnotesize\thepage}
 \cfoot{}
 \rfoot{\footnotesize\today}
\tolerance=700

\begin{document}
\thispagestyle{fancy}

\section*{Annotated Bibliography Assignment}

In most of your majors, I qualify merely as an educated bystander. Therefore, it is crucial that you write your research papers in a way that can readily be comprehended by someone with little experience in your field. A similar point applies, of course, to your in-class presentations, which you should address to an intelligent but ``uninitiated'' audience. By the way, being able to present technical information accessibly is a mark of high quality academic writing; you should strive for this even when communicating with other specialists in your field.

\subsection*{Topic}: Every discipline asks some questions about time. That's what makes it such a good topic. Hopefully, you have already started asking some of these questions yourself. The purpose of this assignment is to teach you how to research topics that interest you. That means, in large part, teaching you how to find information that will help clarify issues you find difficult. 

But where do you start?  For this exercise, you will pose one clear question about time and begin to prepare an annotated bibliography that addresses that topic. 

\subsection*{What is an annotated bibliography?}
An annotated bibliography is a list of citations to books, articles, and documents. Each citation is followed by a brief (usually about 150 words) descriptive and evaluative paragraph, the annotation. The purpose of the annotation is to inform the reader of the relevance, accuracy, and quality of the sources cited. The idea here is that you are creating a small collection of resources that you can use to address your question. 

\subsection*{The Process}
\noindent Creating an annotated bibliography calls for the application of a variety of intellectual skills: concise exposition, succinct analysis, and informed library research.

\begin{enumerate}

\item{Locate and record citations to books, periodicals, and documents that may contain useful information and ideas on your topic. While you can use digital resources, you should also go into the library. If you locate a book you think is useful, look at the rest of the shelf. Often you will find other relevant material.}
\begin{itemize}
\item \textbf{NB:} Use librarians! One of the main jobs of an academic librarian is to help users find research material. Tell them what you are interested in and they will help you figure out what to read. 
\end{itemize}
\item{Briefly examine and review the actual items. While you only need to skim these sources, you need to read them closely enough to decide whether they are relevant.} 

\item{Choose those works that provide a variety of perspectives on your topic. Discard those that are irrelevant.}

\item{Cite the book, article, or document using the your preferred citation style, e.g., APA, Chicago, etc.}

\item{Write a concise annotation that summarizes the central theme and scope of the book or article. Include one or more sentences that (a) evaluates the authority or background of the author, (b) comment on the intended audience, (c) compare or contrast this work with another you have cited, or (d) explain how this work illuminates your bibliography topic.}
\end{enumerate}

\subsection*{Requirements} Your annotated bibliography should include citations and annotations for 1--2 background sources, whether they are encyclopedias, news stories, or blog entries, and 3--5 scholarly sources, whether they be articles from scholarly journals, books published by academic publishers, etc. 

\subsection*{Grading:} I will grade the assignment based on how relevant the sources are, how well you describe those sources, and whether you complete and format the citations correctly. 




\end{document}  