\documentclass[article,oneside]{memoir}

%%% custom style file with standard settings for xelatex and biblatex. Note that when [minion] is present, we assume you have minion pro installed for use with pdflatex.
%\usepackage[minion]{org-preamble-pdflatex} 

%%% alternatively, use xelatex instead
\usepackage{org-preamble-xelatex} 



\def\myauthor{Author}
\def\mytitle{Title}
\def\mycopyright{\myauthor}
\def\mykeywords{}
\def\mybibliostyle{plain}
\def\mybibliocommand{}
\def\mysubtitle{}
\def\myaffiliation{NJCU}
\def\myaddress{Phil 235}
\def\myemail{soconnor@njcu.edu}
\def\myweb{\href{http://scoconno.github.io/Teaching/Time}{http://scoconno.github.io/Teaching/Time}}
\def\myphone{}
\def\myversion{}
\def\myrevision{}
\def\myaffiliation{NJCU}
\def\myauthor{Dr. Scott O'Connor}
\def\mykeywords{}
\def\mysubtitle{Syllabus}
\def\mytitle{{\normalsize HON 202--1 (3289), 3 credits, Spring 2016, M\&W 12:45--14:00, K237 \newline} \HUGE Time}


\begin{document}

%%% If using xelatex and not pdflatex
%%% xelatex font choices
\defaultfontfeatures{}
\defaultfontfeatures{Scale=MatchLowercase}    
% You will need to buy these fonts, change the names to fonts you own, or comment out if not using xelatex.      
\setromanfont[Mapping=tex-text]{Minion Pro} 
\setsansfont[Mapping=tex-text]{Myriad Pro} 
\setmonofont[Mapping=tex-text,Scale=0.8]{Georgia} 

%% blank label items; hanging bibs for text
%% Custom hanging indent for vita items
\def\ind{\hangindent=1 true cm\hangafter=1 \noindent}
\def\labelitemi{$\cdot$}
%\renewcommand{\labelitemii}{~}

%% RCS info string for version tracking
\chapterstyle{article-3}  % alternative styles are defined in latex-custom-kjh/needs-memoir/
%\pagestyle{kjh}

\title{\LARGE \mytitle}     
\author{\Large\myauthor \newline \footnotesize\texttt{\noindent\myweb}}
\date{1/19/2016--5/9/2016}

%\published{\today}

\maketitle

%\thispagestyle{kjhgit}

% Copyright Page
%\textcopyright{} \mycopyright


%
% Main Content
%

\section{Copyright}
The materials used in this class, including, but not limited to, lectures, exams, quizzes, and homework assignments are copyright protected works.  Any unauthorized copying of the class materials or recording of lectures is a violation of federal law and may result in disciplinary actions being taken against the student.  Additionally, the sharing of class materials without the specific, express approval of the instructor may be a violation of the University's Student Honor Code and an act of academic dishonesty, which could result in further disciplinary action.  This includes, among other things, uploading class materials to websites for the purpose of sharing those materials with other current or future students. 



\section{Course Description}

We are temporal creatures. We structure our lives around a sense of our history, of what we were in the past, and what we hope to be in the future. Our societies are also essentially temporal. Political, economic, and cultural institutions value certain units of time over others. Work weeks must not exceed a certain length. Elections must happen on certain fixed dates. The measuring of time is both essential to us and the societies we live in. But what exactly is it that we are measuring?

  In this course, we will examine the nature of time and the role it plays in our lives and societies more generally. We begin by briefly discussing the history of time-measurement and various struggles peoples have encountered in measuring it. We then move to discuss the nature of time. Could time stop? Does it has a beginning and end? Does the future exist already? Does time exist at all?  After the spring break, we will discuss what it is for humans to exist in time. How do we perceive time? Do we have free-will? How do we survive time? Might we be able to time-travel? The course will conclude with presentations on your independent project that you will have begun earlier in the semester. You will have the opportunity to identify some question about time that interests you and explore it throughout the semester.
  
   



\section{Learning Objectives}

Upon completing this course, students will be able to (i) find the argument of a text and clearly restate it, (ii) clearly and charitably explain viewpoints that are not their own, (iii) think critically, (iv) write well-structured prose in which they clearly state a thesis and persuasively defend it, (v) demonstrate a substantive grasp of how time is an  important topic of study in several distinct disciplines. 


\section{Reading}
\subsection{Required}
Available in the campus book store and online retailers. All other readings will be distributed on the course website. 
\begin{itemize}
\item \href{http://www.amazon.com/Travels-Four-Dimensions-Enigmas-Space/dp/0198752555/ref=sr_1_1?ie=UTF8&qid=1452098846&sr=8-1&keywords=robin+le+poidevin+enigmas}{Robin LePoidevin, `Travels in Four-Dimensions: The Enigmas of Space and Time', OUP, 2005 (TRAVELS)}
\end{itemize}
\subsection{Optional}
Students who wish to improve their writing might wish to purchase and work through the 10 lessons contained in the following: 
\begin{itemize}
\item \href{http://www.amazon.com/Style-Lessons-Clarity-Grace-11th/dp/0321898680/ref=sr_1_1?ie=UTF8&qid=1452356026&sr=8-1&keywords=lessons+in+clarity+and+grace}{`Style: Lessons in Clarity and Grace', Joseph Williams and Joseph Bizup}
\end{itemize}

\section{Course Website}
There is both a Blackboard site and website for this course (link on first page). Clicking the first link on the left panel within the Blackboard site will bring you to the course website. All assignments will be submitted through Blackboard. Readings, notes, etc. will be posted on the course website. Note that Blackboard difficulties are rare and automatically reported to instructors. Under no circumstance will a student's report of a Blackboard difficulty be reason for an extension. It is your responsibility to contact Blackboard support for help.

\section{Honors Information}
An Honors student who receives an F for an Honors course will be dismissed from the program at the discretion of the HPC.

\section{Requirements}

\begin{itemize}
\item \textit{Workload:} Expect to spend an average of 5--6 hours per week  completing the readings and assignments.

\item \textit{Attendance:} Roll call will be taken. 0.5 point will be awarded per class up to a maximum of 10 points. Points will not be awarded during weeks 1 \& 2. 

\item \textit{Reading:} You are required to prepare the assigned readings before class. This will be determined through discussion, pop quizzes, and reading responses. You begin with 10 points. You lose 2 points for every class you come to unprepared. No extra credit will be offered to make up lost points in this category. 

\item \textit{Short essays} submitted through Blackboard. 500--750 words long. 5 will be assigned. You must complete 4. If you complete more than 4, the lowest grade will be dropped.
 
\item \textit{1 final project} comprising a proposal, 1 progress report (750--1000 words), short presentation, and written submission (2000--2500 words.)



\item \textit{Grade Distribution:} Attendance--0.5 point per class (10 total); Reading--10 points; 3 short Essays---10 points each (30 total); 1 final project--10 points for the proposal, 10 points for the presentation, 20 points for the final submission (40 points total).

\item \textit{Grade Breakdown:}

 \begin{tabular}{ | l | l | p{2cm} | l | l | }
    \hline 
96--100 & A  & &  77--79 &  C+ \\  
90--95 & A- & &  73--76 & C \\
87-89 & B+ &  &  70--72 & C- \\ 
83--86 & B  & &  60--69 & D\\
80--82 & B - & & 0--59 & F\\ \hline
    \end{tabular}


\end{itemize}





\section{Policies}

\begin{itemize}

\item \textbf{Student Responsibility:} This syllabus outlines the required text, assignments, requirements, and policies for this course. By taking this course, you agree to read this syllabus and be bound by those requirements and policies. 

 \item \textit{Academic Integrity:} All the work you turn in (including papers, drafts, and discussion board posts) must be written by you specifically for this course. It must originate with you in form and content with all contributory sources fully and specifically acknowledged. Being a student at NJCU requires you to follow \href{http://www.njcu.edu/uploadedFiles/About_NJCU/Governance_and_Organization/University_Senate/Policies/Academic\%20INTEGRITY\%20POLICY\%20FINAL\%202-04.pdf}{NJCU's Academic Integrity Policy.} Penalties for violations are as follows: 1st infraction will result in a 0 for the assignment.  2nd infraction will result in a 0 for the entire course \& application for permanent record on student's transcript. (Repeated violations can lead to expulsion from NJCU). 

\item \textit{Attendance:} You are considered absent if you are (i) not present during roll call, (ii) leave early, (iii) leave without permission, or (iv) leave for an extended period of time. No excuses. No exceptions.



\item \textit{Communication:} To comply with Federal Privacy Laws (FERPA) and NJCU policies, all communication will be through Blackboard and/or official NJCU e-mail. Check both your NJCU e-mail and Blackboard daily. For further information see \href{http://scoconno.github.io/Contact/}{http://scoconno.github.io/Contact/}.

\item \textit{Conduct:} Distracting and disrespectful behaviors, including but not limited to eating, leaving your seat, talking out of turn, and aggression are prohibited. Penalties include, but are not limited to, a loss of attendance points for the day of violation. Repeat offenders will be reported to the Dean of Students. 

\item \textit{Electronic devices:} Use of electronic device, including, but not limited, to smartphones, dictaphones, tablets, and laptops, is prohibited. Recording a lecture is in violation of Copyright. Penalties include, but are not limited to, a loss of attendance points for the day of violation. Repeat offenders will be reported to the Dean of Students.

\item \textit{Format for Written Work:} Submit work to Blackboard either as a rich text or Microsoft Word file. All work must be typed. Your papers should be in 12-point Times New Roman font, double-spaced with margins set to one inch on all sides. If hard copies are requested, please staple or paperclip copies of papers and drafts.



\item \textit{Grading:} Grades will be available within 1 week of an assignment being submitted. See: \href{http://scoconno.github.io/Teaching/Grading}{http://scoconno.github.io/Teaching/Grading} for further information.


\item \textit{Late work \& Make-up Policy:} See the assignment schedule below. No make-ups or late work accepted under any circumstances. No exceptions.


\item \textit{Statement for students with disabilities:} If you are a student with a disability and wish to receive consideration for reasonable accommodations, please register with the Office of Specialized Services and Supplemental Instruction (OSS/SI). To begin this process, complete the registration form available on the OSS/SI website at
\href{http://www.njcu.edu/Specialized_Services.aspx}{www.njcu.edu/Specialized\_Services.aspx}
(listed under Student Resources-Forms). Contact OSS/SI at 201-200-2091
or visit the office in Karnoutsos Hall, Room 102 for additional
information.

\end{itemize}



\section{Weekly Course Schedule}
Dates refer to the first day of the week. Complete the readings before the first class of the week. Readings marked with a `**' can be found on the course website. Handouts can be found on the course website. All other listed readings can be found in the required textbook. Changes to the syllabus will be announced in class and \emph{via} your NJCU email address.


\begin{description}

\item[Week 1:] Introduction

\item[Week 2:]  Struggling with time
\begin{enumerate}
\item Geography of time? 
\item `Time, Work-Discipline, and Industrial Capitalism,' E.P. Thompson*
\item `Old Time and Ancient Chronometers,' Tony Roark
\end{enumerate}

\item[Week 3:] What do clocks measure? 
\begin{enumerate}
\item TRAVELS, Ch. 1
\item `Mind, the Metric, and Conventionality', Bradley Dowden
\end{enumerate}

\item[Week 4:] Time and Change
\begin{enumerate}
\item Project brain-storming
\item TRAVELS, Ch. 2
\item  `Time Without Change,' Sydney Shoemaker*
\end{enumerate}

\item[Week 5:] Does time have a beginning? (no class on Mon.)
\begin{enumerate}
\item TRAVELS Ch 5
\end{enumerate}

\item[Week 6:] Does Time Pass? 
\begin{enumerate}
\item `Slaughterhouse-Five,' Kurt Vonnegut (extracts)*
\item TRAVELS, Ch. 8
\end{enumerate}

\item[Week 7:] Does the future exist? (Zeno's paradoxes, eternalism, etc.)
\begin{enumerate}
\item TRAVELS, Ch. 9
\end{enumerate}


\item [Spring Break]



\item[Week 7:] Our perception of time
\begin{enumerate}
\item TBD
\item `Confessions' (extracts),  St. Augustine
\item `The Principles of Psychology' (extracts), William James
\end{enumerate}


\item[Week 9:] Free-Will
\begin{enumerate}
\item TBD
\end{enumerate}

\item[Week 10:] How do we survive time? 
\begin{enumerate}
\item TBD
\end{enumerate}

\item[Week 11:] Solutions 
\begin{enumerate}
\item TBD
\end{enumerate}

\item[Week 12:]  Time-Travel
\begin{enumerate}
\item `All You Zombies,' by Robert Heinlein
\item TRAVELS, Ch. 10 
\item Lewis and the paradoxes of time travel
\end{enumerate}

\item[Week 13:]  The future of time-travel
\begin{enumerate}
\item Interstellar (move)
\item TBD
\end{enumerate}

\item[Week 14:] Presentations


%\item[Week 13:] Time's Arrow
%\begin{enumerate}
%\item TRAVELS, Ch. 12
%\item `The Arrow of Time', Bradley Dowden
%\end{enumerate}



\item[Week 15:] Presentations
 \end{description}

\section{ Assignment Schedule}
Dates refer to the due date. All assignments must be submitted through Blackboard by 1:00pm. No late work accepted. No exceptions. You must complete 2 short essays and 1 long essay. If you complete more than the required number, the lowest grades will be dropped.
\begin{enumerate}
\item \textit{02/01/2016,} Short essay 1
\item \textit{02/15/2016,} Short essay 2
\item \textit{02/29/2016,} Independent Project: Proposal
\item \textit{03/21/2016,} Short essay 3
\item \textit{03/28/2016,} Independent Project: Progress Report
\item \textit{04/11/2016,} Short essay 4
\item \textit{04/25/2016,} Short essay 5
\item \textit{5/06/2016,} Final project: Written submission.
\end{enumerate}




%% Uncomment if you want a printed bibliography.
%\printbibliography 

\end{document}
