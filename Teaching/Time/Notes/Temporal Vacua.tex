\documentclass[oneside]{article}
 \headheight = 25pt
\footskip = 20pt
\usepackage{mdwlist}
\usepackage[T1]{fontenc}
\renewcommand{\rmdefault}{ppl}
\usepackage{fancyhdr}
 \pagestyle{fancy}
 \lhead{\textbf{\textsc{\small Scott O'Connor}}}
 \chead{}
 \rhead{\large\textbf{\textsc{Temporal Vacua}}}
 \lfoot{\footnotesize{\thepage}}
 \cfoot{}
 \rfoot{\footnotesize{\today}}
\tolerance=700


\begin{document}
\thispagestyle{fancy}

\section{Some Basics}

We are interested in the relationship between time and change. Let us distinguish two distinct views (these are rough characterizations):
\begin{description}
\item[Relationalism:] Time is one and the same as change, and so time cannot exist independently of change. 
\item[Absolutism:] Time is different from change.
\end{description}
If time can exist independently of change, then time is different from change, i.e. absolutism is true. If time cannot exist independently of change, then this is some evidence for relationalism. 

A temporal vacuum is a period of time in which absolutely nothing happens. If temporal vacua can exist, then time can exist independently of change.


\section{An argument that time is distinct from change}

If time is identical to change, then time should have the various properties that change possesses. 
\subsection{Argument: Rate of Change}
\begin{enumerate*}
\item Change can go at different rates, speed up, slow down, etc. 
\item If time is a change, then time can speed up, slow down, etc. 
\item We measure how fast things move against time:  rate of change is variation in some dimension in so many units of time.
\item If time can speed up or slow down, we measure how fast it changes against time. 
\item But we cannot measure how fast time changes against time, e.g. five minutes will always last five minutes.  
\item So time is not a change (from 1-5).
\end{enumerate*}
\subsection{Response}
\begin{enumerate*}
\item Particular changes vs the sum of all change. 
\item The argument shows that time is distinct from each particular change. It does not show that it is distinct from the sum of all changes. 
\begin{enumerate*}

\item If the sum of all changes could go at different rates, the sum could double its speed. 
\item But the sum of all changes cannot double its speed.
\item The sum of all changes cannot go at different rates (from (a) and (b)).  
\end{enumerate*}
\item So the rate of change argument does not show that time is distinct from the sum of all change.
\end{enumerate*}

\subsection{Argument that time is distinct from the sum of all change.}

Suppose that all processes were to come to an end. It seems that we can make sense of the idea of time existing even though everything has stopped. For instance, we could imagine a difference between everything having stopped for 1 year as opposed to 10 years. If we can measure how long all changes have stopped, then time can exist independently from the sum of all change. Therefore, it is different from the sum of all change.

\section{Arguments against temporal vacua}

\subsection{Time and experience}
\begin{enumerate*}
\item To notice anything is to undergo a change in mental state.
\item The cessation of all change is the cessation of any experience. 
\item So it is impossible to experience a temporal vacuum (in the sense of experiencing anything as a temporal vacuum) (from 1-2). 
\item So it is impossible to imagine the sum of all change stopping and time continuing to exist (from 3). 
\end{enumerate*}

\subsection{Time and measure}
\begin{enumerate*}
\item Periods of time are measured by changes. 
\item Since, by definition, nothing happens in a temporal vacuum, there is no possible means of determining its length. 
\item If there is no means of determining the length of a temporal interval, it has no specific length. 
\item Every interval of time has a specific length. 
\item There cannot be a temporal vacuum (from 1-4). 

\end{enumerate*}


\subsection{The Principle of Sufficient Reason}
\begin{description}
\item[Principle of Sufficient Reason:] For everything that occurs at a given moment, there is always an explanation of why it occurred at precisely \emph{that} moment and not at some other moment.
\end{description} 

\begin{enumerate*}
\item If there have been temporal vacua in the past, then there have been times when change has resumed after a period of no change. 
\item For every change that occurs at a given moment, there is always an explanation, in terms of an immediately preceding change, of why it occurred at precisely that moment and not at some other moment. 
\item There is no explanation of why a change occuring immediately after a temporal vacuum occurred when it did (since no change immediately preceded it). 
\item There have been no temporal vacua in the past. 
\end{enumerate*}

\end{document}

