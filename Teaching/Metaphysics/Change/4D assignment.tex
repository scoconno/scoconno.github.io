\documentclass[oneside]{article}
 \headheight = 25pt
\footskip = 20pt
\usepackage[T1]{fontenc}
\renewcommand{\rmdefault}{ppl}

\usepackage{fancyhdr}
 \pagestyle{fancy}
 \lhead{\textbf{\textsc{\small Scott O'Connor\\Generation and Destruction (Phil1111)}}}
 \chead{}
 \rhead{\Large\textbf{\textsc{Lit. Review 1}}}
 \lfoot{\footnotesize{\thepage}}
 \cfoot{}
 \rfoot{\footnotesize{\today}}
\tolerance=700

\begin{document}
\thispagestyle{fancy}


\section*{Temporal Parts}

Over the next two weeks, we will think about how those who believe in temporal parts try solve the paradox of increase. You will be filling out the skeleton of your paper by both considering this solution, and also considering problems for it. Before you can write those sections, you need to understand the literature on temporal parts. We begin by reviewing 4 VERY SHORT papers from MET. Don't be startled that there are 4 papers. Together, they total c.15 pages: 

\begin{enumerate}
\item Quine, `Identity, Ostension, and Hypostasis', MET, c. 26
\item Lewis, `In Defense of Stages: Postscript B to ``Survival and Identity'', MET, c. 27
\item Lewis, `The Problem of Temporary Intrinsics', MET, c.28
\item Zimmerman, `Temporary Intrinsics and Presentism', MET, c.29. 
\end{enumerate}

\section*{Assignment}

\begin{itemize}  
\item Write a summary of each article, i.e., summarize the main point and the argument for that point. 
\item Think of the summary as the material that you will use when turning to write your paper. You would prefer not to re-read those articles. Having a succinct summary with choice quotes will save you time and energy. 
\item For 1-3, write 100-150 words for each - no more!
\item For 4, write 150-300 words - no more!
\item Include the word count. 
\item This is due by 6pm. on Sun. Please submit it through turnitin on the Blackboard site. 
\item I expect you to write and submit clear summaries rather than mere notes. Make sure to apply each of the STYLE lessons. 
\end{itemize}

\end{document}
