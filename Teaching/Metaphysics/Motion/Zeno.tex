\documentclass[]{article}
\usepackage{amssymb,amsmath}
\usepackage{ifxetex,ifluatex}
\usepackage{fixltx2e} % provides \textsubscript
\ifnum 0\ifxetex 1\fi\ifluatex 1\fi=0 % if pdftex
  \usepackage[T1]{fontenc}
  \usepackage[utf8]{inputenc}
\else % if luatex or xelatex
  \ifxetex
    \usepackage{mathspec}
    \usepackage{xltxtra,xunicode}
  \else
    \usepackage{fontspec}
  \fi
  \defaultfontfeatures{Mapping=tex-text,Scale=MatchLowercase}
  \newcommand{\euro}{€}
\fi
% use upquote if available, for straight quotes in verbatim environments
\IfFileExists{upquote.sty}{\usepackage{upquote}}{}
% use microtype if available
\IfFileExists{microtype.sty}{%
\usepackage{microtype}
\UseMicrotypeSet[protrusion]{basicmath} % disable protrusion for tt fonts
}{}
\ifxetex
  \usepackage[setpagesize=false, % page size defined by xetex
              unicode=false, % unicode breaks when used with xetex
              xetex]{hyperref}
\else
  \usepackage[unicode=true]{hyperref}
\fi
\hypersetup{breaklinks=true,
            bookmarks=true,
            pdfauthor={},
            pdftitle={Zeno 1},
            colorlinks=true,
            citecolor=blue,
            urlcolor=blue,
            linkcolor=magenta,
            pdfborder={0 0 0}}
\urlstyle{same}  % don't use monospace font for urls
\setlength{\parindent}{0pt}
\setlength{\parskip}{6pt plus 2pt minus 1pt}
\setlength{\emergencystretch}{3em}  % prevent overfull lines
\setcounter{secnumdepth}{0}

\title{Zeno 1}
\author{Scott O'Connor}

\begin{document}
\maketitle

\subsubsection{Motion does not exist}\label{motion-does-not-exist}

\begin{enumerate}
\def\labelenumi{\arabic{enumi}.}
\item
  Space is infinitely divisible or not infinitely divisible.
\item
  If space is infinitely divisible, motion is impossible.
\item
  If space is not infinitely divisible, motion is impossible.
\item
  Motion is impossible (From 1-3).
\end{enumerate}

\subsubsection{Premise 1 - the divisibility of
space}\label{premise-1---the-divisibility-of-space}

\begin{itemize}
\item
  If x is infinitely divisible, x can be divided into ever smaller parts
  \emph{ad infinitum}. In other words, x contains no indivisible parts,
  i.e.~parts that cannot further be divided.

  \begin{itemize}
  \item
    For example, suppose that a line, L, is infinitely divisible. Lines
    are divided into line segments. So every line segment of L can be
    divided into further smaller line segments - there is no smallest
    line segment.
  \item
    Think of this process of dividing something out as merely
    conceptual. Don't worry whether or not we could literally do
    something to x to divide it in this way.
  \end{itemize}
\item
  If x is not infinitely divisible, x can be divided into a finite
  number of \emph{smallest} parts, i.e.~parts that cannot be divided
  into any smaller parts.

  \begin{itemize}
  \itemsep1pt\parskip0pt\parsep0pt
  \item
    For example, suppose that a line, L, is not infinitely divisible.
    Then L contains a finite number of smallest line segments, i.e.,
    line segments with some smallest extent that cannot be divided into
    any further line segments.
  \end{itemize}
\end{itemize}

\subsection{Premise 2}\label{premise-2}

This handout will proceed by discussing Premise 2. See Handout 2 for
discussion of Premise 3. I first outline Zeno's argument for Premise 2.
I then examine 3 different responses.

\emph{Strategy:} Assume that space is infinitely divisible. Then argue
that it is impossible to move from one place to another by showing that
(a) doing so requires completing an infinite number of tasks, and (b) it
is impossible to complete an infinite number of tasks.

Zeno argues for premise 2 by using a number of paradoxes. The first is
called \emph{Racecourse}, which argues that it is impossible to complete
an arbitrary journey from A to B - to start at A, move to B, and then
stop:

\begin{enumerate}
\def\labelenumi{\Alph{enumi}.}
\item
  The distance between A and B is infinitely divisible (assumed).
\item
  A journey from A to B is a series of sub-journeys with no last member:
  from A to \(\frac{1}{2}AB\), from \(\frac{1}{2}AB\) to
  \(\frac{3}{4}AB\), and so on.
\item
  It is impossible to complete a series of sub-journeys with no last
  member.
\item
  Completing a journey from A to B, requires completing the series of
  sub-journeys with no last member: from A to \(\frac{1}{2}AB\), from
  \(\frac{1}{2}AB\) to \(\frac{3}{4}AB\), and so on.
\item
  It is impossible to complete the journey from A to B.
\end{enumerate}

\begin{itemize}
\itemsep1pt\parskip0pt\parsep0pt
\item
  An inverted version of the paradox shows us that our traveler cannot
  begin to move.
\item
  A different paradox, the Achilles paradox, shows us that in a race
  between Achilles and a tortoise, where the tortoise is given a head
  start, Achilles could never catch-up and pass the tortoise.
\end{itemize}

\subsubsection{Response 1: Reject C}\label{response-1-reject-c}

Some deny premise C by claiming that as we divide the distances of the
journey, we should also divide the total time taken, and, further, that
the sum of these infinite series of decreasingly short time intervals is
still equal to a finite period of time. These denials assume that the
argument for C is the following:

\begin{itemize}
\item
  C1. Completing an infinite series of tasks would take an infinite
  amount of time.
\item
  C2. It is not possible to spend an infinite amount of time completing
  some task(s).
\item
  C. Therefore, it is not possible to complete an infinite series of
  tasks.
\end{itemize}

This argument for Premise C is valid, but some deny that it is not sound
because C2 is false. They claim that it relies on the false assumption
that completing an infinite series of tasks would take an infinite
period of time. This seems false. As we divide the distances between the
points we travel, we should also divide the time it takes to travel the
ever smaller distances:

\begin{itemize}
\item
  It takes \(\frac{1}{2}\) the time to run from A to \(\frac{1}{2}AB\)
  as it does to run from A to B.
\item
  It takes \(\frac{1}{4}\) of the time to run from \(\frac{1}{2}AB\) to
  \(\frac{3}{4}AB\), and so on.
\item
  The sum of these decreasing times is finite.\footnote{See `A
    Contemporary Look at Zeno's Paradoxes', by Wesley Salmon}
\item
  Therefore, we can complete an infinite series of sub-journeys in a
  finite period of time.
\end{itemize}

Let us grant that C1-C2 fail to establish Premise C. There is an
alternative way of defending Premise C that is immune to our current
objection:

\begin{itemize}
\itemsep1pt\parskip0pt\parsep0pt
\item
  Even if it takes less time to complete each sub-journey, I still need
  to first complete each sub-journey before completing the journey that
  comes after it. If so, I always have one more sub-journey to complete
  before I can complete the final step.
\end{itemize}

Note that the response assumes that time is infinitely divisible, i.e.
divisible into a infinite number of finite parts. We'll be re-visiting
that assumption in Handout 2.

\subsubsection{Response 2: Reject B}\label{response-2-reject-b}

Aristotle claims that a journey from A to B is a series of
\emph{potential} and not \emph{actual} sub-journeys, i.e.~the full
journey does not consist of actual parts each of which are sub-journeys.
The point here is that Zeno is defining journeys in terms of the space
over which you travel. Aristotle responds by denying that journeys are
individuated in this way.

In order to evaluate this response, we need to investigate the nature of
actions/activities and try to understand what is involved in completing
them. In other words, if journeys are not individuated by the distance
over which they occur, how are they individuated?

\end{document}
