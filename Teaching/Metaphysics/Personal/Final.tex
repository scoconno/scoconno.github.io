\documentclass[10pt]{article}
\usepackage{fancyhdr}
 \pagestyle{fancy}
\rhead{\textsc{Scott O`Connor}}

\usepackage{lmodern}
\usepackage{amssymb,amsmath}
\usepackage{ifxetex,ifluatex}
\usepackage{fixltx2e} % provides \textsubscript
\ifnum 0\ifxetex 1\fi\ifluatex 1\fi=0 % if pdftex
  \usepackage[T1]{fontenc}
  \usepackage[utf8]{inputenc}
\else % if luatex or xelatex
  \ifxetex
    \usepackage{mathspec}
    \usepackage{xltxtra,xunicode}
  \else
    \usepackage{fontspec}
  \fi
  \defaultfontfeatures{Mapping=tex-text,Scale=MatchLowercase}
  \newcommand{\euro}{€}
\fi
% use upquote if available, for straight quotes in verbatim environments
\IfFileExists{upquote.sty}{\usepackage{upquote}}{}
% use microtype if available
\IfFileExists{microtype.sty}{%
\usepackage{microtype}
\UseMicrotypeSet[protrusion]{basicmath} % disable protrusion for tt fonts
}{}
\ifxetex
  \usepackage[setpagesize=false, % page size defined by xetex
              unicode=false, % unicode breaks when used with xetex
              xetex]{hyperref}
\else
  \usepackage[unicode=true]{hyperref}
\fi
\usepackage[usenames,dvipsnames]{color}
\hypersetup{breaklinks=true,
            bookmarks=true,
            pdfauthor={},
            pdftitle={Final Project},
            colorlinks=true,
            citecolor=blue,
            urlcolor=blue,
            linkcolor=magenta,
            pdfborder={0 0 0}}
\urlstyle{same}  % don't use monospace font for urls
\setlength{\parindent}{0pt}
\setlength{\parskip}{6pt plus 2pt minus 1pt}
\setlength{\emergencystretch}{3em}  % prevent overfull lines
\providecommand{\tightlist}{%
  \setlength{\itemsep}{0pt}\setlength{\parskip}{0pt}}
\setcounter{secnumdepth}{0}

\title{Final Project}
\author{Scott O’Connor}


% Redefines (sub)paragraphs to behave more like sections
\ifx\paragraph\undefined\else
\let\oldparagraph\paragraph
\renewcommand{\paragraph}[1]{\oldparagraph{#1}\mbox{}}
\fi
\ifx\subparagraph\undefined\else
\let\oldsubparagraph\subparagraph
\renewcommand{\subparagraph}[1]{\oldsubparagraph{#1}\mbox{}}
\fi

\begin{document}
\maketitle

\subsection{Prompt}\label{prompt}

\begin{quote}
Under what conditions is person A (at t) identical to person B (at t')?
\end{quote}

Pick one of the proposed solutions to this question and defend it in
depth:

\begin{enumerate}
\def\labelenumi{\arabic{enumi}.}
\tightlist
\item
  Psychological Continuity
\item
  Material/Bodily Continuity
\item
  Continuity of Soul (Dualism)
\end{enumerate}

Your paper must have the following sections (not including the
introduction and conclusion):

\begin{enumerate}
\tightlist
\item[a.] An explanation of the puzzle of personal identity.
\item[b.] An explanation of the chosen solution.
\item[c.] An identification of one problem for the proposed solution.
\item[d.] A defense of the solution against this objection.
\end{enumerate}

\subsection{Further Instruction}\label{further-instruction}

\begin{itemize}
\item
  This assignment covers the following 4 papers from the textbook in
  `Part II: What is our Place in the World?':

  \begin{enumerate}
  \def\labelenumi{\arabic{enumi}.}
  \tightlist
  \item
    36: `Personal Identity: a Materialist Account', Shoemaker
  \item
    37: `An Argument for Animalism', Olson
  \item
    38: `Divided Minds and the Nature of Persons', Parfit
  \item
    39: `Personal Identity: the Dualist Theory', Swinburne
  \end{enumerate}
\item
  Read each of these four papers carefully. Do not use outside
  resources.
\item
  You will be completing this assignment in several stages. Find the
  grade breakdown and timeline below:

  \begin{enumerate}
  \def\labelenumi{\arabic{enumi}.}
  \tightlist
  \item
    Group work: In class project to prepare summaries of the articles.
    Bring the book to class as well as a laptop/tablet
  \item
    Peer review: In-class peer feedback on the first draft of sections
    (a) and (b). Bring a hard copy to class.
  \item
    Full first draft of sections (a)--(d)
  \item
    Oral Presentations: prepare a powerpoint preparation of 4-6 slides
    (not including pictures, title page, etc) Your presentation must
    address sections (a)--(d) listed in the prompt.
  \item
    Final draft: I will return comments on your first draft. Your final
    draft will be evaluated, in part, on how well you improved your
    paper based on my comments.
  \end{enumerate}
\end{itemize}

\subsection{Word Count}\label{word-count}

\begin{itemize}
\tightlist
\item
  1500--1750 words. Writing less than 1500 words or more than 1750 words
  will lose you points
\end{itemize}

\subsection{Timeline}\label{timeline}

\begin{itemize}
\tightlist
\item
  12/02/2015, Topic Introduced
\item
  12/04/2015, Group Work
\item
  12/07/2015, \textbf{Dr.~Jonathan Pickle's Talk}
\item
  12/09/2015, In-class peer review of the first draft of sections (a)
  and (b).
\item
  12/11/2015, Full first draft due through Blackboard.
\item
  12/14/2015, Oral presentations
\item
  12/21/2015, Final Draft Due
\end{itemize}

\subsection{Grade Breakdown}\label{grade-breakdown}

This assignment represents 40 points towards you final grade. The
breakdown is as follows:

\begin{itemize}
\tightlist
\item
  First Draft:

  \begin{itemize}
  \tightlist
  \item
    In-class peer review of section (a) and (b): 2 points
  \item
    Full-first draft of (a)--(d) through Blackboard: 8 points
  \end{itemize}
\item
  Oral Presentation: 10 points
\item
  Final Draft: 20 points
\end{itemize}

\end{document}
