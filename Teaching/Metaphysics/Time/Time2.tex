\documentclass[oneside]{article}

\usepackage[T1]{fontenc}
\renewcommand{\rmdefault}{ppl}

\usepackage{fancyhdr}
 \pagestyle{fancy}
 \lhead{\textbf{\textsc{\small Scott O'Connor\\Intro. to Metaphysics }}}
 \chead{}
 \rhead{\LARGE\textbf{\textsc{Time 2}}}
 \lfoot{\footnotesize{\thepage}}
 \cfoot{}
 \rfoot{\footnotesize{\today}}
\tolerance=700


\begin{document}
\thispagestyle{fancy}

\subsection*{McTaggart}
Argues that times does not exist. 

\subsection*{The A-Series and the B-Series}

\begin{itemize}
\item Events are ordered in time. The A-series and the B-series are different theories about the nature of this ordering. 
\item A-properties: \emph{being past, being present, being future}.
\begin{itemize}
\item Events can change their A-properties, e.g., in 1992 the event of Obama winning the election had the property of being a future event; an event that will happen. At that time, it did not have the property of being a present event; an event that was currently happening. Nor did it have the property of being a past event; an event that had occurred. This event briefly acquired the property of being present before losing that property and becoming a past event. 
\end{itemize}
\item B-properties: \emph{earlier than, later than, simultaneous with}
\begin{itemize}
\item B-properties are relations between events, e.g., the event of Obama winning the election was later than the event of George Bush winning the election.
\item Events never change their B-properties, e.g., it is eternally true that Obama won the election after George Bush won the election.
\item Consider a spatial analogy....volunteers please.
\end{itemize}
\item A-Series: both A and B properties exist. 
\item B-Series: A properties cannot exist. Only B-properties exist. 
\end{itemize}


\subsection*{Time Does Not Exist}

\begin{enumerate}
\item If time is real, either the B-theory or the A-theory of time is the correct characterization of time. 
\item The B-theory cannot characterize change. [See below]
\item The A-theory is contradictory. [See below]
\item An adequate characterization of change cannot be contradictory. 
\item The A-theory cannot characterize change. [From 4-5] 
\item Time is unreal. [From 2-6]
\item Change does not exist. [From 1\&7]
\end{enumerate}
Notes
\begin{itemize}
\item  P2 assumes that time exists only if change exists. 
\item  You will normally find the A-theory and B-theory debate presented as merely a debate about the reality of time. I have constructed this argument in this way to show why debate about the reality of time is relevant to our main topic; change .
\end{itemize}

\subsection*{Argument for P3}
Should I be doing this with change? Or should I be doing it with time? 

\begin{enumerate}
\item If change exists, the passage of time must be real....
\item If the B-theory is true, the passage of time is unreal.
\item If change exists, the B-theory cannot be true. [From 1-2]
\end{enumerate} 
Some questions:
\begin{itemize}
\item Why accept (1)? Suggestion: Changes are events that have temporal duration, and they seem to be made up of smaller events with shorter durations. For instance, running a 400m race is an event that has temporal duration. But it is itself made up of shorter lasting events. For example, there is the event of running the first 100m, the event of running the second 100m, and so on. But these events do not seem to `eternally' occur, i.e. the journalist shouts, 'the sprinter is \emph{now} on the second leg'. Later he shouts, 'the springer is \emph{now} on the third leg.'  
\item Can we argue for (2) without assuming the A-theory? That is, can we only argue for (2) as follows:  A-properties are required to explain the passage of time. The B-theory rejects the existence of A-properties. Therefore, the  B-theory cannot explain the passage of time.
\item Does the existence of time require that the passage of time be real? If not, could there be a world in which time exists even though the passage of time, and hence change, does not exist? 
\end{itemize} 

\subsection*{Argument for P4}

\begin{enumerate}
\item A particular event E either (a) co-instantiates the properties of being past, present, and future, or (b) does not co-instantiate these properties. 
\item It is impossible for E to co-instantiate the properties of being past, present, and future.
\item It is impossible for E not to co-instantiate the properties of being past, present, and future. 
\end{enumerate}
Big questions:
\begin{itemize}
\item Do you accept (2)? If so, why? If not, why not? 
\item Why might we accept (3)? Recall that there is a worry about an infinite regress. What is this worry? A hint: it assumes a distinction between moments of time and the events occuring at those moments.
\end{itemize}



\end{document} 

