\documentclass[]{article}

\usepackage{fancyhdr}
 \pagestyle{fancy}
\rhead{\textsc{Scott O`Connor}}

\usepackage{lmodern}
\usepackage{amssymb,amsmath}
\usepackage{ifxetex,ifluatex}
\usepackage{fixltx2e} % provides \textsubscript
\ifnum 0\ifxetex 1\fi\ifluatex 1\fi=0 % if pdftex
  \usepackage[T1]{fontenc}
  \usepackage[utf8]{inputenc}
\else % if luatex or xelatex
  \ifxetex
    \usepackage{mathspec}
    \usepackage{xltxtra,xunicode}
  \else
    \usepackage{fontspec}
  \fi
  \defaultfontfeatures{Mapping=tex-text,Scale=MatchLowercase}
  \newcommand{\euro}{€}
\fi
% use upquote if available, for straight quotes in verbatim environments
\IfFileExists{upquote.sty}{\usepackage{upquote}}{}
% use microtype if available
\IfFileExists{microtype.sty}{%
\usepackage{microtype}
\UseMicrotypeSet[protrusion]{basicmath} % disable protrusion for tt fonts
}{}
\ifxetex
  \usepackage[setpagesize=false, % page size defined by xetex
              unicode=false, % unicode breaks when used with xetex
              xetex]{hyperref}
\else
  \usepackage[unicode=true]{hyperref}
\fi
\usepackage[usenames,dvipsnames]{color}
\hypersetup{breaklinks=true,
            bookmarks=true,
            pdfauthor={},
            pdftitle={Essay},
            colorlinks=true,
            citecolor=blue,
            urlcolor=blue,
            linkcolor=magenta,
            pdfborder={0 0 0}}
\urlstyle{same}  % don't use monospace font for urls
\setlength{\parindent}{0pt}
\setlength{\parskip}{6pt plus 2pt minus 1pt}
\setlength{\emergencystretch}{3em}  % prevent overfull lines
\providecommand{\tightlist}{%
  \setlength{\itemsep}{0pt}\setlength{\parskip}{0pt}}
\setcounter{secnumdepth}{0}

\title{Essay}
\author{Scott O’Connor}


% Redefines (sub)paragraphs to behave more like sections
\ifx\paragraph\undefined\else
\let\oldparagraph\paragraph
\renewcommand{\paragraph}[1]{\oldparagraph{#1}\mbox{}}
\fi
\ifx\subparagraph\undefined\else
\let\oldsubparagraph\subparagraph
\renewcommand{\subparagraph}[1]{\oldsubparagraph{#1}\mbox{}}
\fi

\begin{document}
\maketitle

\subsubsection{Introduction}\label{introduction}

The year is 2042. The President is unhappy. Sam Miller was elected in
2038, but citizens everywhere claim that they did not vote for him.
There has been dramatic changes in the last four years. Many citizens
have upgraded their biological limbs with superior mechanical ones. The
advent of the technicolor appearance modifier has allowed people to
change their skin, eye, and hair color, and to make changes every day.
Still others availed of new bone molding technology that allowed them to
shrink, grow, and change the shape of their bones. The changes are so
drastic that those who voted in 2038 can hardly be strictly identical to
anyone who exists now, or so people claim. These dissenters are lead by
Gretchen Weirob - Miller's former campaign manager.

Drastic times call for drastic measures. Miller challenges Weirob to a
televised discussion. Weirob argues that she is not numerically
identical to any voter from 2038 given that nobody from 2038
sufficiently resembles her. Miller responds by arguing that people can
change dramatically while remaining numerically identical because
personal identity consists in either sameness of soul or psychological
continuity {[}you will pick one of these views{]}.

\subsubsection{Purpose}\label{purpose}

The purpose of this assignment is to help you practice the following
skills that are essential to your success in this course and others.

\begin{enumerate}
\def\labelenumi{\arabic{enumi}.}
\tightlist
\item
  Charitably explaining arguments for views that are not your own.
\item
  Explaining difficult concepts in your own words.
\item
  Understanding the difference between qualitative similarity and
  numerical identity.
\end{enumerate}

\subsubsection{Task}\label{task}

Write a transcript of the debate between Miller and Weirob. The dialog
must contain the following:

\begin{enumerate}
\def\labelenumi{\arabic{enumi}.}
\tightlist
\item
  An explanation of the puzzle of personal identity. Why does Weirob
  deny that she is identical to anyone from four years ago?

  \begin{itemize}
  \tightlist
  \item
    \textbf{NB:} Your paper must demonstrate a proper understanding of
    the difference between qualitative similarity (qualitative identity)
    and numerical identity. Failure to do so will result in loss of
    points.
  \end{itemize}
\item
  An attempt to solve this puzzle with Miller appealing to
  \textbf{either} sameness of soul or psychological continuity.

  \begin{itemize}
  \tightlist
  \item
    \textbf{NB:} Miller must pick one of these views to explain personal
    identity. This assignment is not asking you to discuss each possible
    response to the puzzle. Failure to ignore this instruction will
    result in a loss of points.\\
  \end{itemize}
\item
  An identification of one problem for the proposed solution, i.e.,
  Weirob should raise an objection to Miller's supposed solution.
\end{enumerate}

\textbf{NB:} Failure to include any of these elements in your dialog
will result in a loss of points.

\subsubsection{Word Count}\label{word-count}

Your essay must be 500-750 words long. Essays shorter than 500 words or
longer than 750 words will lose points.

\subsubsection{Plagiarism}\label{plagiarism}

Please review the plagiarism policy on the syllabus. It is critical that
you prepare your assignment by yourself. Use only the textbook and
handouts---it will take you less time to work through these materials
than to find and read other sources. I will be checking for significant
overlaps between submission as well as checking answers against
Wikipedia, internet search results, standard essay sites, etc. If you
include material in your essay without citing it, you will receive 0 for
the assignment. A second violation will result in a 0 for the course, a
report to the Dean, and a petition for a note to be added to your
permanent academic record.

\subsubsection{Due Date}\label{due-date}

Please consult the syllabus for the due date.

\subsubsection{Late Submissions}\label{late-submissions}

Per the policies outlined in the syllabus, late work will not be
accepted. Any request for special treatment will be ignored. If you
foresee difficulties submitting work on time, either because of personal
or work commitments, then you should start this paper early and submit
it early.

\subsubsection{Format}\label{format}

Please submit the file as a docx, rtf, or pdf.

\subsubsection{Grading}\label{grading}

Please find the rubric and explanation of it
\href{/Teaching/Grading/}{here}.

\subsubsection{Resources}\label{resources}

Please find links to writing resources
\href{/Teaching/Resources/}{here}.

\end{document}
