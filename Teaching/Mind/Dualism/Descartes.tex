\documentclass[]{article}
%\usepackage{lmodern}
\usepackage{amssymb,amsmath}
\usepackage{ifxetex,ifluatex}
\usepackage{fixltx2e} % provides \textsubscript
\ifnum 0\ifxetex 1\fi\ifluatex 1\fi=0 % if pdftex
  \usepackage[T1]{fontenc}
  \usepackage[utf8]{inputenc}
\else % if luatex or xelatex
  \ifxetex
    \usepackage{mathspec}
    \usepackage{xltxtra,xunicode}
  \else
    \usepackage{fontspec}
  \fi
  \defaultfontfeatures{Mapping=tex-text,Scale=MatchLowercase}
  \newcommand{\euro}{€}
\fi
% use upquote if available, for straight quotes in verbatim environments
\IfFileExists{upquote.sty}{\usepackage{upquote}}{}
% use microtype if available
\IfFileExists{microtype.sty}{%
\usepackage{microtype}
\UseMicrotypeSet[protrusion]{basicmath} % disable protrusion for tt fonts
}{}
\ifxetex
  \usepackage[setpagesize=false, % page size defined by xetex
              unicode=false, % unicode breaks when used with xetex
              xetex]{hyperref}
\else
  \usepackage[unicode=true]{hyperref}
\fi
\hypersetup{breaklinks=true,
            bookmarks=true,
            pdfauthor={},
            pdftitle={Descartes},
            colorlinks=true,
            citecolor=blue,
            urlcolor=blue,
            linkcolor=magenta,
            pdfborder={0 0 0}}
\urlstyle{same}  % don't use monospace font for urls
\setlength{\parindent}{0pt}
\setlength{\parskip}{6pt plus 2pt minus 1pt}
\setlength{\emergencystretch}{3em}  % prevent overfull lines
\setcounter{secnumdepth}{0}

\title{Descartes}
\date{}

\begin{document}
\maketitle

\subsection{Rene Descartes (1596-1650)}\label{rene-descartes-1596-1650}

\subsubsection{Biography}\label{biography}

\begin{itemize}
\item
  Born in La Haye in the Touraine region of France.
\item
  In 1606 or 1607, entered the newly founded Jesuit College of La
  Fleche, where he remained until 1614 or 1615.

  \begin{itemize}
  \itemsep1pt\parskip0pt\parsep0pt
  \item
    He spent five to six years studying classical Latin, Greek, the
    major poets and orators (e.g.~Cicero).
  \item
    This was followed by three years of philosophical training based on
    the philosophy of Aristotle. He studies logic, morals, physics, and
    metaphysics.
  \item
    Mathematics.
  \end{itemize}
\item
  Studied to be a lawyer, but then joined the army after getting his
  degree.
\item
  Settled in the Netherlands in 1629. He had one daughter, by a servant
  (Helene), named Francine, who died young.
\item
  Moved to Sweden in 1649 to instruct Queen Christina in philosophy.
  They met at five in the morning, for five hours, three days a week.
\item
  Died of respiratory infection on Feb. 11th, 1650.
\end{itemize}

Descartes on his education:

\begin{quote}
From my childhood they fed me books, and because people convinced me
that these could give me clear and certain knowledge of everything
useful in life, I was extremely eager to learn them. But no sooner had I
completed the whole course of study that normally takes one straight
into the ranks of the `learned' than I completely changed my mind about
what this education could do for me. For I found myself tangled in so
many doubts and errors that I came to think that my attempts to become
educated had done me no good except to give me a steadily widening view
of my ignorance!(Discourse, I.2)
\end{quote}

\subsubsection{Contributions}\label{contributions}

Descartes made significant contributions in mathematics, optics, and
physics. One commentator writes:

\begin{quote}
Descartes went far beyond the Copernican hypothesis that our Sun lies at
the center of the universe with the Earth moving about it. He contended
that the Earth is one among many planets, revolving around many
different suns distributed throughout the cosmos. He further proposed
that the whole universe is made of one kind of matter, which follows one
set of laws. He invented the concept of a single universe, filled with
matter having a few describable properties and governed by a few laws of
motion. While others, including ancient atomists and Stoics, had
sketched part of this new picture, Descartes' vision of a unified
physics governed by a few laws of motion was far richer and more
detailed. Its combination of breadth and unity was unprecedented in his
earlier work with Beeckman, or in the works of Copernicus, Galileo, or
Kepler. This unified vision set the framework for Newton's subsequent
unification of mechanics and astronomy.'' (Hatfield (2003), 18)
\end{quote}

\subsubsection{A New Foundation for
Physics}\label{a-new-foundation-for-physics}

Descartes worried about the prevailing sciences of the time. This was a
time when scientists still sought to explain the motion of bodies by the
so-called inherent tendencies of earth to move downward, air to rise
upward, etc. Doctors still explained disease by appeal to various humors
in the body. Descartes worried as to whether the various sciences of the
day were making fundamental errors and suggested that we need to take
stock: which of our sciences could really be trusted?

He undertook this project in several works, most importantly `The
Meditations.' He describes the project in this work to a friend as
follows:

\begin{quote}
I will say to you, just between us, that these six Meditations contain
all the foundations of my Physics. But, please, you must not say so; for
those who favor Aristotle would perhaps have more difficulty in
approving them; and I hope that those who will read them will
unwittingly become accustomed to my principles, and will recognize the
truth, before they notice that my principles destroy those of Aristotle.
(Descartes, letter to Marsenne)
\end{quote}

Descartes subsequently identified four Rules for acquiring Knowledge

\begin{enumerate}
\def\labelenumi{\arabic{enumi}.}
\item
  Accept as true only that for which one has proper evidence, which he
  calls clear and distinct perception.
\item
  Resolve problems into their simplest parts.
\item
  Move from the simple and known to the complex.
\item
  Thoroughly review and check one's work to be sure it is comprehensive
  and complete.
\end{enumerate}

The first meditation begins by asking what can be clearly and distinctly
perceived? Descartes wants to find some means/tool that would allow him
test all of his beliefs. He suggests skepticism and proceeds to identify
which of his beliefs, if any, were immune from doubted given the
strongest skeptical challenge.

His first tries to identify a skeptical challenge that is sufficiently
strong for this purpose, i.e., a challenge that will pertain to as many
of his beliefs as possible. Some candidate challenges can be ruled out.
Skepticism about vision only allows us test our beliefs about
perceptible things. Descartes finally settles on the strongest skeptical
challenge and then proceeds to reject whatever is open to the slightest
doubt. He continues until he has found something certain that can
withstand the skeptical challenge.

\subsubsection{Skeptical Arguments
Rejected}\label{skeptical-arguments-rejected}

\paragraph{Argument 1}\label{argument-1}

\begin{enumerate}
\def\labelenumi{\arabic{enumi}.}
\itemsep1pt\parskip0pt\parsep0pt
\item
  If I am occasionally mistaken in my perceptual beliefs, then it is
  possible that I am totally mistaken in my perceptual beliefs.
\item
  I am occasionally mistaken in my perceptual beliefs.
\item
  Thus it is possible that I am totally mistaken in my perceptual
  beliefs.
\item
  If I don't know that I am totally mistaken in my perceptual beliefs,
  then I don't know that O (where O is a proposition about perceptual
  objects).
\item
  I don't know that O.
\end{enumerate}

This argument is valid, but Descartes thinks there is a problem with
premise 2. While 2 is true about some beliefs, it is not true about all.
I occasionally make mistakes about small and distant objects, but I'm
never mistaken about medium sized objects in front of me in good
lighting. Since Descartes is looking for an argument that could test all
his beliefs, and this argument tests only some beliefs, he searches for
a stronger skeptical argument.

\paragraph{Argument 2}\label{argument-2}

\begin{enumerate}
\def\labelenumi{\arabic{enumi}.}
\itemsep1pt\parskip0pt\parsep0pt
\item
  If I don't know that I'm not dreaming, then I don't know that I have
  hands.
\item
  I don't know that I'm not dreaming.
\item
  I don't know that I have hands.
\end{enumerate}

Descartes thinks that the argument does not generalize to every belief
he holds. For instance, it does not generalize to his belief in the
existence of space, time, magnitude, simple qualities, and so on. So,
Descartes searches for an even stronger argument.

\paragraph{Evil Demon Argument
Argument}\label{evil-demon-argument-argument}

\begin{enumerate}
\def\labelenumi{\arabic{enumi}.}
\itemsep1pt\parskip0pt\parsep0pt
\item
  If I don't know that I'm not being deceived by an Evil Demon, then I
  don't know that p.
\item
  I don't know that I'm not being deceived by an Evil Demon.
\item
  I don't know that p.
\end{enumerate}

This is a suitably strong enough challenge to test all Descartes'
beliefs, including his beliefs about space, time, etc. Remember, his
goal is to throw out all his beliefs and only let back in those that he
knows to be true. He will now proceed to search for those beliefs that
he would know to be true even if this skeptical scenario obtained.

\subsection{I am thinking, therefore I
exist}\label{i-am-thinking-therefore-i-exist}

Descartes writes the following in the second Meditation:

\begin{quote}
I have convinced myself that there is absolutely nothing in the world,
no sky, no earth, no minds, no bodies. Does it now follow that I too do
not exist? No: if I convinced myself of something then I certainly
existed. But there is a deceiver of supreme power and cunning who is
deliberately and constantly deceiving me. In that case I too undoubtedly
exist, if he is deceiving me; and let him deceive me as much as he can,
he will never bring it about that I am nothing so long as I think that I
am something. So after considering everything very thoroughly, I must
finally conclude that this proposition, I am, I exist, is necessarily
true whenever it is put forward by me or conceived in my mind.
\end{quote}

Descartes first claims, `I exist'. There is nothing the Evil Demon could
do to deceive him about this belief, i.e., the very fact that the Evil
Demon would be deceiving him, proves that he, himself, would exist.
Knowing that he exists is a good beginning, but we believe we know more
than that. In particular, we believe we know what kind of things we are.
Descartes agrees. He reviews his former beliefs about what kind of thing
he is with the hope of identifying which of these beliefs can also
withstand the skeptical challenge.

\begin{description}
\itemsep1pt\parskip0pt\parsep0pt
\item[Beliefs about his body]
He has a body, arms, face, and so on.
\item[Beliefs about his functions:]
He takes nourishment, moves, senses, and thinks.
\end{description}

Many of these beliefs can be doubted. An evil demon could trick him into
thinking he has a body. An evil demon could trick him into thinking that
he really moves. But Descartes thinks that some cannot be doubted.

\textbf{Cogito Ergo Sum}

He cannot doubt that he is a being that thinks, i.e., the Evil Demon
could never deceive one into thinking that one doesn't think, or into
thinking that one thinks when one doesn't, in fact, think. But what is
thinking?

Descartes claims that thinking involves doubting, understanding,
affirming, refusing to affirm, desiring, having mental images, being
aware of what comes from senses (even if the senses are being deceived).
Thus, Descartes claims that he know that he exists, he thinks, and that
he affirms, desires, and so on.

\subsubsection{Problems with the Cogito}\label{problems-with-the-cogito}

There are several problems with the Cogito. The most famous is that it
seems to assume what it seeks to prove. Descartes argues, `I am
thinking, therefore I exist'. This argument has as a premise,
\texttt{I am thinking\textquotesingle{} and this premise presupposes that the subject of the thought exists. Since the subject of the thought is}I',
the premise assumes the conclusion, i.e., it assumes that I exist. An
argument that assumes what it is trying to prove is obviously circular.
And recall that Descartes is supposed to be testing whether he can doubt
his own existence. He is begging the question if he argues, `I exist,
therefore I cannot doubt that I exist.'

We might try save Descartes by reformulating the argument in a
non-question begging way:

\begin{enumerate}
\def\labelenumi{\arabic{enumi}.}
\itemsep1pt\parskip0pt\parsep0pt
\item
  There is a thought
\item
  Therefore I exist
\end{enumerate}

The Evil Demon certainly cannot deceive us into thinking that there is
no thought occurring. This argument also avoids begging the question in
the way the argument above does. However, the argument is invalid. We
cannot infer from the fact that there is someone doing P, that we are
that person doing P. Thus, even if we can infer from the fact that a
thought exists that \emph{someone} is thinking, we cannot infer that we
are the ones thinking that thought.

\end{document}
