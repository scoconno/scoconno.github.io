\documentclass[11pt,article,oneside]{memoir}

%%% custom style file with standard settings for xelatex and biblatex. Note that when [minion] is present, we assume you have minion pro installed for use with pdflatex.
%\usepackage[minion]{org-preamble-pdflatex} 

%%% alternatively, use xelatex instead
\usepackage{org-preamble-xelatex} 



\def\myauthor{Author}
\def\mytitle{Title}
\def\mycopyright{\myauthor}
\def\mykeywords{}
\def\mybibliostyle{plain}
\def\mybibliocommand{}
\def\mysubtitle{}
\def\myaffiliation{NJCU}
\def\myaddress{101 Phil}
\def\myemail{soconnor@njcu.edu}
\def\myweb{\href{http://scoconno.github.io/Teaching/ExaminedMAIN}{http://scoconno.github.io/Teaching/ExaminedMAIN}}
\def\myphone{}
\def\myversion{}
\def\myrevision{}
\def\myaffiliation{NJCU}
\def\myauthor{Dr. Scott O'Connor}
\def\mykeywords{}
\def\mysubtitle{Syllabus}
\def\mytitle{{\normalsize Phil 140, 3 credits, Fall 2015, M\&W, 9:55am, K 554\newline} \HUGE The Examined Life}


\begin{document}

%%% If using xelatex and not pdflatex
%%% xelatex font choices
\defaultfontfeatures{}
\defaultfontfeatures{Scale=MatchLowercase}    
% You will need to buy these fonts, change the names to fonts you own, or comment out if not using xelatex.      
\setromanfont[Mapping=tex-text]{Georgia} 
\setsansfont[Mapping=tex-text]{Georgia} 
\setmonofont[Mapping=tex-text,Scale=0.8]{Georgia} 

%% blank label items; hanging bibs for text
%% Custom hanging indent for vita items
\def\ind{\hangindent=1 true cm\hangafter=1 \noindent}
\def\labelitemi{$\cdot$}
%\renewcommand{\labelitemii}{~}

%% RCS info string for version tracking
\chapterstyle{article-3}  % alternative styles are defined in latex-custom-kjh/needs-memoir/
\pagestyle{kjh}

\title{\LARGE \mytitle}     
\author{\Large\myauthor \newline \footnotesize\texttt{\noindent\myweb}}
\date{09/02/2015--12/14/2015}

\published{\,}

\maketitle

%\thispagestyle{kjhgit}

% Copyright Page
%\textcopyright{} \mycopyright


%
% Main Content
%

\section{Copyright}
The materials used in this class, including, but not limited to, lectures, exams, quizzes, and homework assignments are copyright protected works.  Any unauthorized copying of the class materials or recording of lectures is a violation of federal law and may result in disciplinary actions being taken against the student.  Additionally, the sharing of class materials without the specific, express approval of the instructor may be a violation of the University's Student Honor Code and an act of academic dishonesty, which could result in further disciplinary action.  This includes, among other things, uploading class materials to websites for the purpose of sharing those materials with other current or future students. 

\section{Catalog Description}

This course teaches students to identify and evaluate those beliefs that guide their thoughts and actions. Reflecting on different sources, students identify those philosophical beliefs that play a role in their own lives. By developing their critical thinking skills, they learn how to clarify, systematize, and assess these beliefs. 

\section{Course Description}

Does God exist? Are you free? Why live? What should you do with your life?  In this
course, we'll be asking some of these deep philosophical questions. We begin by discussing the meaning of life, especially why some philosophers have connected a meaningful life with God's existence. This raises the question as to whether God exists. We will examine some classic arguments for the existence of God as well as concerns that God's existence is incompatible with the existence of evil. Many respond to the problem of evil by claiming that evil is a by-product of our free-will, a gift endowed by God to use as we see fit. But are we free? We will discuss why some think our actions are completely pre-determined by causal factors outside of our control. If they are right, free-will is a mere illusion. This raises deep questions about the nature of moral responsibility; can you be held responsible for an action that was out of your control? In the final part of the course, we will ask what determines the moral character of our actions. Do the ends justify the means? After studying the main ethical theories, you will get a chance to work in groups to apply them to a current controversy of your choosing, e.g., the death penalty, euthanasia, abortion, etc.  

 \section{First Year Experience Program}
Students in this course are also enrolled in the same section of English Composition I--ALP (ECI--ALP) with Dr. Joshua Fausty.  This section of PHIL 140 coordinates with RWAD, in part, by giving students opportunities to practice and reinforce the reading, writing, argument-making and analytical skills necessary to perform well in all other college courses. 



\section{Learning Objectives}

Upon completing this course, students will be able to (i) read
philosophical texts, (ii) clearly and charitably explain viewpoints that
are not their own, (iii) think critically and philosophically, (iv)
write well-structured prose in which they clearly state a thesis and
persuasively defend it, (v) demonstrate an understanding of several core
philosophical topics, (vi) manage their studies in a timely and efficient manner.

\section{Required Textbook}

By 09/14/2015, the following textbook must be purchased or rented:

\begin{itemize}
\item
  \href{http://www.amazon.com/Philosophy-Here-Now-Powerful-Everyday/dp/0199765227}{Philosophy
  Here and Now: Powerful Ideas in Everyday Life', by Lewis Vaughn}
  (Available in the campus book store and online retailers)
\end{itemize}

\section{Requirements}

\begin{itemize}
\item \textit{Workload:} Successfully completing this course requires a minimum commitment of 6 hours per week. 

\item \textit{Attendance:} Roll call will be taken from Week 1. 1 point per class up to a maximum of 10 points. Excludes Weeks 1 and 2 .

\item \textit{2 Short Writing Assignments} submitted through Blackboard. 

\item \textit{2 Essays} submitted through Blackboard. 1000 and 1200 words respectively. 

\item \textit{Final group project} consisting of oral presentation and written submission. Each member must present at least one slide. Groups will grade their member's participation. 

\item \textit{Grade Distribution:}  Attendance---1 point per class (10 total);  Short Writing Assignments---10 points each (20 total); Essays---20 points each (40 total); Presentation---10 oral, 10 points peer evaluation, 20 points written, (40 points total).

\item \textit{Grade Breakdown:}

 \begin{tabular}{ | l | l | p{2cm} | l | l | }
    \hline 
96--\textbf{110} & A  & &  77--79 &  C+ \\  
90--95 & A- & &  73--76 & C \\
87-89 & B+ &  &  70--72 & C- \\ 
83--86 & B  & &  60--69 & D\\
80--82 & B - & & 0--59 & F\\ \hline
    \end{tabular}


\end{itemize}




\section{Policies}

\begin{itemize}

\item \textit{Student Responsibility:} This syllabus outlines the required text, assignments, requirements, and policies for this course. By taking this course, you agree to read this syllabus and be bound by those requirements and policies. 

\item \textit{Late work \& Make-up Policy:} 
\begin{itemize}
\item All assignments must be submitted through Blackboard by 1:00 pm on the due date (see assignment schedule below).
\item  No make-ups or late work accepted under any circumstances. No exceptions. But note that there are 110 points available with 96+ being required for an A.
\item Blackboard difficulties are rare and automatically reported to instructors. Under no circumstance will a student's report of a Blackboard difficulty be reason for an extension. It is your responsibility to contact blackboard support for help: \href{dlsupport@njcu.edu}{dlsupport@njcu.edu}. 

\end{itemize}

\item \textit{Attendance:} You are considered absent if you are (i) not present during roll call, or (ii) leave early, or (iii) leave without permission, or (iv) leave for an extended period of time.

\item \textit{Electronic devices:} Use of electronic device, including, but not limited, to smartphones, dictaphones, tablets, laptops, is prohibited. Recording a lecture is in violation of Copyright. Penalties include, but are not limited to, a lost of attendance grade for the day of violation. Repeat offenders will be reported to the Dean of Students. 

\item \textit{Conduct:} Distracting and disrespectful behavior, including but not limited to eating, leaving your seat, talking out of turn, aggression is prohibited. Penalties include, but are not limited to, a lost of attendance grade for the day of violation. Repeat offenders will be reported to the Dean of Students. 

\item \textit{Communication:} To comply with Federal Privacy Laws (FERPA) and NJCU policies, all communication will be through Blackboard and/or official NJCU email, e.g., I will not respond to email from gmail, yahoo, hotmail, etc. Check both your NJCU email and Blackboard daily. Messages will be responded to within two days of receiving them. 

\item \textit{Grading:} Grades will be available within 1 week of an assignment being submitted. For grading information see\\ \href{http://scoconno.github.io/Teaching/Rubric.}{http:\\scoconno.github.io/Teaching/Rubric}


\item \textit{Statement for students with disabilities:} If you are a student
with a disability and wish to receive consideration for reasonable
accommodations, please register with the Office of Specialized Services
and Supplemental Instruction (OSS/SI). To begin this process, complete
the registration form available on the OSS/SI website at
\href{http://www.njcu.edu/Specialized_Services.aspx}{www.njcu.edu/Specialized\_Services.aspx}
(listed under Student Resources-Forms). Contact OSS/SI at 201-200-2091
or visit the office in Karnoutsos Hall, Room 102 for additional
information.
\end{itemize}

\section{Plagiarism}

\begin{itemize} 
\item You are bound by \href{http://www.njcu.edu/uploadedFiles/About_NJCU/Governance_and_Organization/University_Senate/Policies/Academic\%20INTEGRITY\%20POLICY\%20FINAL\%202-04.pdf}{NJCU's Academic Integrity Policy}
\item Penalty for plagiarism:
\begin{itemize}
\item 1st infraction: 0 for the assignment. 
\item 2nd infraction: 0 for the entire course \& application for permanent record on student's transcript. (Repeated violations can lead to expulsion from NJCU). 
\end{itemize}
\end{itemize}


\section{Weekly Course Schedule}
Dates refer to the first day of the week. Complete the readings before the first class of the week. Readings marked with a '*' can be found on the course website. Changes to the syllabus will be announced in class and by email to your NJCU email address.  
\begin{enumerate}

\item \textit{08/31/2015,} Introduction 
\begin{enumerate}
\item Ch.1.1--1.2.
\item `The Trial and Death of Socrates', Plato, pp.44-53
\item \href{Animation of Plato's Cave}{http://link}
\end{enumerate}

\item \textit{09/07/2015,} Continued
\begin{enumerate}
\item Ch.1.1--1.2.
\item `The Trial and Death of Socrates', Plato, pp.44-53
\item \href{Animation of Plato's Cave}{http://link}
\end{enumerate}

\item \textit{09/14/2015,} Critical Thinking Tools
\begin{enumerate}
\item Ch.1.3
\item \href{Handout}{http://scoconno.github.io/Teaching/ExaminedMAIN/Worksheet}
\end{enumerate}

\item \textit{09/21/2015,} The Meaning of Life: Pessimism 
\begin{enumerate}
\item 'The Good Brahmin', Voltaire, pp.381--382
\item Ch.8.1--8.2 
\item \href{`A Confession', Leo Tolstoy}{http://scoconno.github.io/Teaching/ExaminedMAIN/Tolstoy.pdf}* 
\end{enumerate}

\item \textit{09/28/2015,} The Meaning of Life: Optimism
\begin{enumerate}
\item Ch.8.3
\item TBD: A reading on some world historic individual who dedicated to their life to a meaningful goal. Do Androids Dream? 
\end{enumerate}

\item \textit{10/05/2015,} God: The Design Argument
\begin{enumerate}
\item Ch.2.1-2.2. 
\item *`Intelligent Design Has No Place in the Science Curriculum', Harold Morowitz, Robert Hazen, and James Trefil
 \item *`Design for Living', Michael J. Behe
 \end{enumerate}
 
\item \textit{10/12/2015,} God: Further Arguments for God's Existence
\begin{enumerate}
\item Ch.2.2
\item `The Star', Arthur C. Clarke, pp.118--120
\end{enumerate}

\item \textit{10/19/2015,} God: The Problem of Evil 
\begin{enumerate}
\item Ch.2.3.
\item ``A Little Omniscience Goes a Long Way'', Thomas Davis, pp.248--250
\item Current Affairs: newspaper report. 
\end{enumerate}

\item \textit{10/26/2015,} Free Will: Causal Determinism
\begin{enumerate}
\item Ch.5.1--5.2
\end{enumerate}

\item \textit{11/03/2015,} Free Will: Compatibilism
\begin{enumerate}
\item Ch.5.3
\item `Science and Free Will', p.234
\item Reading on science of character, TBD (Character as Moral Fiction, by Alfano?)
\end{enumerate}

\item \textit{11/09/2015,} Ethics: Relativism, 
\begin{enumerate}
\item Ch.3.1--3.2.
\item TBD, Karinyaga
\end{enumerate}
\item \textit{11/16/2015,} Ethics: Consequentialism \& Deontology 
\begin{enumerate}
\item Ch.3.3--3.4
\item  `The Ones Who Walk Away from Omelas', Ursula Le Guin, pp.181--184 
\end{enumerate}
\item \textit{11/23/2015,} Ethics: Deontology \& Virtue Ethics
\begin{enumerate}
\item Ch.3.4--3.5
\item *`Billy Budd', Herman Melville (extracts)
\end{enumerate}
\item \textit{11/30/2015,} Applied Ethics: Group Projects
\item \textit{12/07/2015,} Applied Ethics: Group Projects
\item \textit{12/14/2015,} Applied Ethics: Group Projects
\end{enumerate}






\section{Assignment Schedule}
Dates refer to the due date. All assignments must be submitted through Blackboard by 1:00pm. No late work accepted. No make-ups. No exceptions. 

\begin{enumerate}
\item \textit{09/27/2015,} SW1--Meaning of Life Dialog
\item \textit{10/18/2015,} SW2--Letter to school board
\item \textit{11/02/2015,} Essay 1--The Problem of Evil
\item \textit{11/23/2015,} Essay 2--Free will
\item \textit{11/25/2015,} Group Projects--final day to have selected topic.
\item \textit{11/30/2015,} Group Projects--Literature Review (in class?)
\item \textit{12/02/2015,} Group Projects--Literature Review (in class?) 
\item \textit{11/07/2015,} Group Projects--Presentations
\item \textit{12/09/2015,} Group Projects--Presentations
\item \textit{12/14/2015,} Group Projects--Presentations
\item \textit{TBD,} Group Projects--Written Submission
\end{enumerate}




%% Uncomment if you want a printed bibliography.
%\printbibliography 

\end{document}