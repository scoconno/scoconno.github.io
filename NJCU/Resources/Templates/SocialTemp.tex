\documentclass[article,oneside]{memoir}
%%% custom style file with standard settings for xelatex and biblatex. Note that when [minion] is present, we assume you have minion pro installed for use with pdflatex.
%\usepackage[minion]{org-preamble-pdflatex} 

%%% alternatively, use xelatex instead
\usepackage{org-preamble-xelatex} 
\def\myauthor{Author}
\def\mytitle{Title}
\def\mycopyright{\myauthor}
\def\mykeywords{}
\def\mybibliostyle{plain}
\def\mybibliocommand{}
\def\mysubtitle{}
\def\myaffiliation{Location:}
\def\myaddress{Time: }
\def\myemail{e-mail:}
\def\myweb{e-mail: }
\def\myphone{}
\def\myversion{}
\def\myrevision{}
\def\myauthor{Instructor: }
\def\mykeywords{}
\def\mysubtitle{Syllabus}
\def\mytitle{{\normalsize Phil 245 ( \ \  \  \  \  \  \   \  \  \  \ ), 3 credits,  Semester: \newline} \HUGE Social Justice }


\begin{document}

%%% If using xelatex and not pdflatex
%%% xelatex font choices
\defaultfontfeatures{}
\defaultfontfeatures{Scale=MatchLowercase}    
% You will need to buy these fonts, change the names to fonts you own, or comment out if not using xelatex.      
\setromanfont[Mapping=tex-text]{Georgia} 
\setsansfont[Mapping=tex-text]{Georgia} 
\setmonofont[Mapping=tex-text,Scale=0.8]{Georgia} 

%% blank label items; hanging bibs for text
%% Custom hanging indent for vita items
\def\ind{\hangindent=1 true cm\hangafter=1 \noindent}
\def\labelitemi{$\cdot$}
%\renewcommand{\labelitemii}{~}

%% RCS info string for version tracking
\chapterstyle{article-3}  % alternative styles are defined in latex-custom-kjh/needs-memoir/
\pagestyle{kjh}

\title{\mytitle}     
\author{{\noindent\myauthor} \newline{ }   \newline {\noindent\myemail}  \newline{ }    \newline {\noindent\myaddress}  \newline{ }    \newline {\noindent\myaffiliation}  }
\date{ }

\maketitle

%\thispagestyle{kjhgit}

% Copyright Page
%\textcopyright{} \mycopyright


%
% Main Content
%



\section{Catalog Description}

This course will provide an overview of social justice topics including: poverty, unemployment, the welfare state, racial discrimination, gender discrimination, and income inequality. The readings for the course will include contemporary philosophic, sociological, and economic writings. Contemporary data sources will also be utilized.

\section{Discipline Specific Learning Outcomes}

Upon completing this course, students will be able to (i) describe different philosophy schools of thought concerning distribution and justice, (ii) identify the major stages in social and economic developments of the recent past, (iii) compare the origin, history, and results of different kinds of welfare state interventions, (iv) use national and international public data sources effectively, (v) compare the social and economic position of people of different national, ethnic, racial, class, and gender backgrounds, (vi) analyze possible interventions into social and economic situations to alleviate suffering and inequality, (vii) explain their own reactions to the ideas presented, and (viii) evaluate other people’s reactions to and commentaries on these ideas.


\section{General Education Information} 
Successfully completing this course satisfies one Tier 2 Social and Historical Perspectives requirement. It teaches the following two University-wide Learning Goals: (1) Information and Technological Literacy, (2) Written Communication. For further information about the General Education Program see \href{http://www.njcu.edu/cas/general-education/}{http://www.njcu.edu/cas/general-education/}.









\section{NJCU Policies}

\begin{itemize}

 \item \textit{Academic Integrity:} All the work you turn in (including papers, drafts, and discussion board posts) must be written by you specifically for this course. It must originate with you in form and content with all contributory sources fully and specifically acknowledged. You are required to read and follow follow NJCU's Academic Integrity Policy available here: \href{http://www.njcu.edu/senate/policies/}{http://www.njcu.edu/senate/policies/} 

\item \textit{Communication:} To comply with Federal Privacy Laws (FERPA) and NJCU policies, all communication will be through Blackboard and/or official NJCU e-mail. 

\item \textit{General Education Program Assessment:} General Education courses participate in programmatic assessment of the six University-wide student learning goals. They include instruction in, and assessment of, at least two of these learning goals. Signature assignments, which may include document, picture, sound, or video files, are uploaded to a secure server for anonymous distribution to the NJCU assessment team, which scores them using approved program rubrics. While instructors also grade their own students’ signature assignments, which count toward the course grade, assessment team results are aggregated to provide information about the Gen Ed program as a whole. Your name will not be included in any programmatic assessment data.

\item \textit{Statement for students with disabilities:} If you are a student with a disability and wish to receive consideration for reasonable accommodations, please register with the Office of Specialized Services and Supplemental Instruction (OSS/SI). To begin this process, complete the registration form available on the OSS/SI website at
\href{http://www.njcu.edu/Specialized_Services.aspx}{www.njcu.edu/Specialized\_Services.aspx}
(listed under Student Resources-Forms). Contact OSS/SI at 201-200-2091
or visit the office in Karnoutsos Hall, Room 102 for additional
information.

\item \textit{TURNITIN:} Students agree by taking this course that all assignments are subject to submission for textual similarity review to Turnitin.com. Assignments submitted to Turnitin.com will be included as source documents in Turnitin.com’s restricted access database solely for the purpose of detecting plagiarism in such documents. The terms that apply to the University’s use of the Turnitin.com service are described on the Turnitin.com web site. For further information about Turnitin, please visit: http://www.turnitin.com.

\end{itemize}






%% Uncomment if you want a printed bibliography.
%\printbibliography 

\end{document}