\documentclass[]{article}
%\usepackage{lmodern}
\usepackage{amssymb,amsmath}
\usepackage{ifxetex,ifluatex}
\usepackage{fixltx2e} % provides \textsubscript
\ifnum 0\ifxetex 1\fi\ifluatex 1\fi=0 % if pdftex
  \usepackage[T1]{fontenc}
  \usepackage[utf8]{inputenc}
\else % if luatex or xelatex
  \ifxetex
    \usepackage{mathspec}
    \usepackage{xltxtra,xunicode}
  \else
    \usepackage{fontspec}
  \fi
  \defaultfontfeatures{Mapping=tex-text,Scale=MatchLowercase}
  \newcommand{\euro}{€}
\fi
% use upquote if available, for straight quotes in verbatim environments
\IfFileExists{upquote.sty}{\usepackage{upquote}}{}
% use microtype if available
\IfFileExists{microtype.sty}{%
\usepackage{microtype}
\UseMicrotypeSet[protrusion]{basicmath} % disable protrusion for tt fonts
}{}
\ifxetex
  \usepackage[setpagesize=false, % page size defined by xetex
              unicode=false, % unicode breaks when used with xetex
              xetex]{hyperref}
\else
  \usepackage[unicode=true]{hyperref}
\fi
\hypersetup{breaklinks=true,
            bookmarks=true,
            pdfauthor={},
            pdftitle={},
            colorlinks=true,
            citecolor=blue,
            urlcolor=blue,
            linkcolor=magenta,
            pdfborder={0 0 0}}
\urlstyle{same}  % don't use monospace font for urls
\setlength{\parindent}{0pt}
\setlength{\parskip}{6pt plus 2pt minus 1pt}
\setlength{\emergencystretch}{3em}  % prevent overfull lines
\setcounter{secnumdepth}{0}

\date{}

\begin{document}

Mind as Immaterial Substance

What is it to have a mind? If we asked Plato or Aristotle, two Greek
philosophers living c.2500 years ago, they would answer that to have a
mind is to have a soul. The English word `soul' translates the Greek
work `psuche' and the latin word `anima'. You likely recognize `psuche'
if you are a psychology major; `psychology' roughly translates as `the
science of soul'. Psychologists today do not think that what it is to
have a mind is to have a soul, but it is worthwhile thinking about why
in years gone past soul and mind have been so closely associated.

The 17th century philosopher, Descartes, is the one most closely
associated with the view. You can read a little about Descartes and his
project \href{Descartes}{here}. You can listen to some podcasts about
him here.

Reality comprises different kinds of entities. There are chairs, trees,
atoms, and molecules. Many think that whatever exists is similar in one
important respect: they are all physical. Philosophers use the word
`substance' to categorize such a general kind of thing, e.g., clouds and
molecules are physical substances. `Substance' is an interesting word
that has a variety of uses in English. Compare `the substance of the
argument' vs. `the substance in this bucket'. For our and Descartes'
purposes, `substance' is closely connected with one of its Greek
origins, `upokeimenon', which means roughly `what underlies', or
`subject'.

Here's the idea: reality is fundamentally divided between entities which
are the bearers of qualities, quantities, relations, and activities and
these various qualities, quantities, relations, and activities they
bear. There is an important difference the height, weight, and swimming
of the dolphin Flipper and Flipper himself. Flipper is a substance, in
the sense that Flipper is a subject in which other things inhere, e.g.,
the swimming belongs to Flipper, the shine and weight inhere in Flipper,
etc. Flipper, though, does not inhere in anything else. He is not a
property of any other substance.

This contrast between substances and what inhere in them obviously
mirrors a linguistic contrast between grammatical subjects, on the one
hand, and predicates on the other, between `Flipper' and `swims'. But
the contrast is more fundamental than the linguistic one; `blue' can be
used as the subject of a predicate, e.g. `blue is nice'. Substances can
never inhere in anything else.

Many think that the world contains only physical substances. This means
that they only things which exist in which properties inhere are
physical things. We will never find, on this view, an existing substance
that does not have physical attributes.

This is not Descartes' view. He argues that reality comprises two
radically different kinds of substances, physical (or bodily)
substances, and souls (mental substance). These substances have very
different features inhering in them, and, according to Descartes have
interesting relations to one another.

According to Descartes, each person is made up of two types of
substance, a physical and a mental one.

Minds are subjects of their own special properties like thinking,
sensing, judging, and willing. These properties inhere in mind. They do
not inhere in bodies. Bodies, on the other hand, are extended in space,
they have weight, undergo nourishment, etc. It is important to recognize
just how different these two types of substances are. A soul is simple,
divine, and immutable. Our bodies are composite and perishable.

This allows Descartes to claim that a soul survives bodily death and
decay. How could a soul decay if it is not made of matter? In part, this
also comes from the notion of a substance. That which inheres in a
substance cannot exist without substances. Color cannot exist without
some object that has color. Substance, on the other hand, does not
depend for its existence on anything to inhere in. Socrates does not
need to be predicated of anything for him to exist. It is normally taken
to follow that substances can exist independently of anything else.
Descartes himself wrote:

\begin{quote}
The notion of a substance is just this---that it can exist by itself,
that is without the aid of any other substance.
\end{quote}

So just as these chairs could exist without one another, so too a mind
and body should exist without one another.

\subsection{Descartes' fundamental
claims}\label{descartes-fundamental-claims}

\begin{enumerate}
\def\labelenumi{\arabic{enumi}.}
\itemsep1pt\parskip0pt\parsep0pt
\item
  There are substances of two fundamentally different kinds in the
  world, mental substances and material substances---or minds and
  bodies. The essential nature of a mind is to think, be conscious, and
  engage in other mental activities; the essence of a body is to have
  spatial extensions (a bulk) and be located in space.
\item
  A human person is a composite being (a ``union,'' as Descartes called
  it) of a mind and a body.
\item
  Minds are diverse from bodies; no mind is identical with a body.
\item
  Minds and bodies causally influence each other. Some mental phenomena
  are causes of physical phenomena and vice versa.
\end{enumerate}

The fourth claim is important. Descartes thinks that there is causal
interaction between mind and body. Mind causes the body to move. The
body can cause the mind to have various features. Here are some
examples. We'll ultimately worry whether 1 and 3 are compatible with 4,
but let's review why Descartes believes in substance dualism.

\subsection{Arguments for Substance
Dualism}\label{arguments-for-substance-dualism}

Descartes offers his famous ``cogito'' argument: ``I think, therefore I
exist.'' This inference convinces him that he can be absolutely certain
about his own existence; his existence is one perfectly indubitable bit
of knowledge he has. Now that he knows he exists, he wonders what kind
of thing he is, asking, ``But what then am I?'' He answers: ``A thinking
thing''. And a thinking thing is ``a thing that doubts, understands,
affirms, denies, is willing, is unwilling, and also imagines and has
sensory perceptions.''

Descartes cannot deny that he is a thinking thing. He therefore knows
that he exists and he knows what kind of thing he is. He proceeds to
inquire into whether the thinking thing is the same as his body. All of
the following arguments turn on identifying some feature that the soul
has but the body does not. In general: To show that X ≠ Y, all we need
do is find a single property P such that X has P but Y lacks it, or Y
has P but X lacks it. Such a property P can be called a differential
property for X and Y.

\paragraph{Argument 1}\label{argument-1}

\begin{enumerate}
\def\labelenumi{\arabic{enumi}.}
\itemsep1pt\parskip0pt\parsep0pt
\item
  I am such that my existence cannot be doubted.
\item
  My body is not such that its existence cannot be doubted.
\item
  Therefore, I am not identical with my body.
\item
  Therefore, the thinking thing that I am, that is, my mind, is not
  identical with my body.
\end{enumerate}

\paragraph{Argument 2}\label{argument-2}

\begin{enumerate}
\def\labelenumi{\arabic{enumi}.}
\itemsep1pt\parskip0pt\parsep0pt
\item
  My mind is transparent to me---that is, nothing can be in my mind
  without my knowing that it is there.
\item
  My body is not transparent to me in the same way.
\item
  Therefore, my mind is not identical with my body.
\end{enumerate}

\paragraph{Argument 3}\label{argument-3}

\begin{enumerate}
\def\labelenumi{\arabic{enumi}.}
\itemsep1pt\parskip0pt\parsep0pt
\item
  Each mind is such that there is a unique subject who has direct access
  to its contents.
\item
  No material body has a specially privileged knower---knowledge of
  material things is in principle public and intersubjective.
\item
  Therefore, minds are not identical with material bodies.
\end{enumerate}

We are said to know something ``directly'' when the knowledge is not
based on evidence, or inferred from other things we know. When knowledge
is direct, like my knowledge of my toothache, it makes no sense to ask,
``How do you know?''

\paragraph{Argument 4}\label{argument-4}

\begin{enumerate}
\def\labelenumi{\arabic{enumi}.}
\itemsep1pt\parskip0pt\parsep0pt
\item
  My essential nature is to be a thinking thing.
\item
  My body's essential nature is to be an extended thing in space.
\item
  My essential nature does not include being an extended thing in space.
\item
  Therefore, I am not identical with my body. And since I am a thinking
  thing (namely a mind), my mind is not identical with my body
\end{enumerate}

\paragraph{Argument 5}\label{argument-5}

\begin{enumerate}
\def\labelenumi{\arabic{enumi}.}
\itemsep1pt\parskip0pt\parsep0pt
\item
  If anything is material, it is essentially material.
\item
  However, I am possibly immaterial---that is, there is a possible world
  in which I exist without a body.
\item
  Hence, I am not essentially material.
\item
  Hence, it follows (with the first premise) that I am not material.
\end{enumerate}

\paragraph{Argument 6}\label{argument-6}

\begin{enumerate}
\def\labelenumi{\arabic{enumi}.}
\itemsep1pt\parskip0pt\parsep0pt
\item
  Suppose I am identical with this body of mine.
\item
  In 2001 this body did not exist.
\item
  Hence, from the first premise, it follows that I did not exist in
  2001.
\item
  But I existed in 2001.
\item
  Hence, a contradiction, and the supposition must be false.
\item
  Hence, I am not identical with my body.''
\end{enumerate}

``Should the soul of a prince, carrying with it the consciousness of the
prince's past life, enter and inform the body of a cobbler, as soon as
deserted by his own soul, everyone sees he would be the same person with
the prince, accountable only for the prince's action\ldots{}. Had I the
same consciousness ``that I saw the ark and Noah's flood, as that I saw
an overflowing of the Thames last winter, or as that I write now, I
could no more doubt that I who write this now, that saw the Thames
overflowed last winter, that viewed the flood at the general deluge, was
the same self \ldots{} than that I who write this am the same myself now
whilst I write \ldots{} that I was yesterday''

\subsection{Problems with Substance
Dualism}\label{problems-with-substance-dualism}

Princess Elisabeth of Bohemia against Descartes

Descarets claimed that the mind and body causally influence one another.
Our decision to throw the ball causes our limbs to move. The cat biting
our finger causes our mind to feel pain. But how does this mental
causation work? Think of an easy case of causation, say the movement of
a pool ball as it is struck by another. One pool ball brings about an
effect in another pool ball by, in part, touching it. Does all causation
involve contact? Contact requires surfaces, so if causation involves
contact, how could the mind, which doesn't have a surface, touch the
body which does have a surface? Descarets explains in his 6th Meditation
as follows:

\begin{quote}
The mind is not immediately affected by all parts of the body, but only
by the brain, or perhaps just by one small part of the brain\ldots{}.
Every time this part of the brain is in a given state, it presents the
same signals to the mind, even though the other parts of the body maybe
in a different condition at the time\ldots{}. For example, when the
nerves in the foot are set in motion in a violent and unusual manner,
this motion, by way of the spinal cord, reaches the inner parts of the
brain, and there gives the mind its signal for having a certain
sensation, namely the sensation of a pain as occurring in the foot. This
stimulates the mind to do its best to get rid of the cause of the pain,
which it takes to be harmful to the foot.
\end{quote}

Descartes identifies the pineal gland as the seat of the soul, as that
place where the soul and mind directly interact. He thinks there are
bodily fluids in the nerves. He thinks that the gland causes the fluid
to move in an appropriate way to transmit the required motion to the
rest of the body.

Elisabeth challenged Descartes to explain

\begin{quote}
how the mind of a human being, being only a thinking substance, can
determine the bodily spirits in producing bodily actions. For it appears
that all determination of movement is produced by the pushing of the
thing being moved, by the manner in which it is pushed by that which
moves it, or else by the qualification and figure of the surface of the
latter. Contact is required for the first two conditions, and extension
for the third. {[}But{]} you entirely exclude the latter from the notion
you have of the soul, and the former seems incompatible with an
immaterial thing.
\end{quote}

\begin{itemize}
\itemsep1pt\parskip0pt\parsep0pt
\item
  P1. For anything to cause a physical object to move, or cause any
  change in one, there must be a transfer of momentum from the cause to
  the physical object.
\item
  P2. If an object imparts momentum to another, it must have mass and
  velocity.
\item
  P3. An unextended mind outside physical space has neither mass nor
  velocity.
\item
  C. An unextended mind cannot cause a physical body to move.
\end{itemize}

\subsection{Varieties of Dualism}\label{varieties-of-dualism}

Recall that Descartes thinks that the mind and body are discrete
substances, they are each the subjects of various features and are not
themselves features of anything else. On this view, we really do have
minds. It is, if you like, a separate thing in itself with its own
nature.

While substance dualism received little current support, a different
form of dualism has received support, namely, property dualism has
received support. Consider the properties of being a bachelor and being
an unmarried man. These are one and the same properties. How about the
property of having a shape and having a size? These are obviously
distinct properties that co-occur together. Anything which has a shape
has a size and vice versa.

Descartes believed that the mind is something that has properties, but
many think that the mind is better characterized as properties had by
some substance, in particular, properties had by the physical body we
possess. Consider the mental property of `being in a state of pain', or
the activity like `thinking about Descartes'. These are properties,
understood broadly, that are instantiated some substance, e.g., the
brain. Whenever these properties occur in the brain, there is some other
feature/property that occurs in the brain, i.e., some c-fibre fires when
you experience pain.

What is the relationship between these physical and mental properties?
Are they identical just like how the properties bachelor and being a
unmarried man are identical? Or are they distinct like the properties
shape and size are distinct?

Many of those who deny that the mind is a real thing still argue that
mental properties and physical properties are distinct, that a C-fibre
firing and the feeling of pain are numerically distinct (even if they
always occur together). Others deny this. They deny that there are any
non-physical properties at all. This debate is the mind-body problem.

\end{document}
