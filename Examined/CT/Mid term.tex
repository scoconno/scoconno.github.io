\documentclass[11pt]{exam}
\usepackage[margin=1in]{geometry}
\newcommand{\class}{The Examined Life}
\newcommand{\examnum}{Critical Thinking Test}
\newcommand{\term}{Date: }
\parindent 0ex
\setlength\answerlinelength{3in}
\begin{document}
\pagestyle{head}
\firstpageheader{}{}{}
\runningheader{\class}{\examnum\ - Page \thepage\ of \numpages}
\runningheadrule

\begin{flushright}
\begin{tabular}{p{2.8in} r l}
\textbf{\class} & \textbf{Name (Print):} & \makebox[2in]{\hrulefill}\\
\textbf{\term} &&\\
\textbf{\examnum} &&\\
\end{tabular}\\
\end{flushright}
\rule[1ex]{\textwidth}{.1pt}

\begin{minipage}[t]{3.7in}
\vspace{0pt}
\begin{itemize}
\item Enter your name and date on the top of this page.
\item You may \textit{not} use your books, notes, etc.
\item Do not write in the table to the right.
\item Write clearly. Poor handwriting may lead to loss in points.
\item 100 points max. out of 110 available.
\end{itemize}
 \rule[1ex]{\textwidth}{.1pt}

\end{minipage}
\hfill
\begin{minipage}[t]{2.3in}
\vspace{0pt}
%\cellwidth{3em}
\gradetablestretch{2}
\vqword{Problem}
\addpoints % required here by exam.cls, even though questions haven't started yet.	
\gradetable[v]%[pages]  % Use [pages] to have grading table by page instead of question

\end{minipage}


\begin{questions}
\addpoints
\section*{Multiple Choice Questions}

\question Circle the correct answer. 
\begin{parts}
\part[5]  A group of statements in which some of them (the premises) are intended to support another of them (the conclusion) is known as a(n)
\begin{choices}
\choice Chain argument		
\choice Claim	
\correctchoice Argument 
\choice Reason
\end{choices}

\part[5] The statements (reasons) given in support of another statement are called
\begin{choices}
\choice An argument
\choice The conclusion	
\correctchoice The premises 	
\choice The complement
\end{choices}

\part[5] These two statements ``The Wall Street Journal says that people should invest heavily in stocks. Therefore, investing in stocks is a smart move'' constitute
\begin{choices}
\choice No argument	
\choice An explanation	
\correctchoice An argument 	
\choice  Two conclusions	
\end{choices}

\part[5] A deductive argument is intended to provide 
\begin{choices}
\choice Probable support for its conclusion	
\choice Persuasive support for its conclusion	
\correctchoice Logically conclusive support for its conclusion 	
\choice Tentative support for its conclusion	
\end{choices} 

\part[5] An inductive argument is intended to provide
\begin{choices}
\choice Valid support for its conclusion	
\correctchoice Probable support for its conclusion 	
\choice Weak support for its conclusion	
\choice Truth-preserving support for its conclusion	
\end{choices}


\part[5] This classic argument ``All men are mortal. Socrates is a man. Therefore, Socrates is mortal'' is
\begin{choices}
\choice  Inductively strong
\choice  Deductively cogent	
\choice Deductively invalid	
\correctchoice Deductively valid 	
\end{choices}


\end{parts}

\section*{Argument Analysis}

\question For each of the following passages, 1) indicate whether it contains an argument. If it does, 2) specify the conclusion, 3) specify the premises, and 4) indicate whether it is valid or invalid. Use the space provided. 

\begin{parts}

\part[10] NJCU must raise tuition. Every university must provide ping pong tables, roller coasters, and free lunches to its students and the only way for NJCU to do this is to raise its prices. 

\vspace{1.5in}

\part[10] Andrea denies that she is an atheist, so she must be an atheist. 

\vspace{1in}

\begin{solution}
\begin{itemize}
\item This is an argument.
\item It is deductive. 
\item The premise(s): Andrea denies that she is an atheist. 
\item Conclusion: She is an atheist. 
\item The argument is invalid. 
\item It does not contain a fallacy.  
\end{itemize}
\end{solution}

\part[10] Senator Jill: ``We'll have to cut education funding this year.'' \\
Senator Bill: ``Why?'' \\
Senator Jill: ``Well, either we cut the social programs or we live with a huge deficit and we can't live with the deficit.''

\vspace{1.5in}

\part[10] Lynch says that we should spend more state revenue on education. But Lynch is a professor who wants a better salary --- so you know that his opinion is worthless. 
\vspace{1in}

\part[10] Hillary Clinton supports gun-control legislation. As you know, only fascists support gun-control legislation. We are forced to conclude that Hillary Clinton is a fascist. 

\vspace{1.5in}

\begin{solution}
\begin{itemize}
\item This is an argument.
\item It is deductive. 
\item The premise(s): 1) Hillary Clinton supports gun-control legislation. 2) All fascist regimes of the twentieth century has passed gun-control legislation. 
\item Conclusion: Hillary Clinton is a fascist. 
\item The argument is vvalid. 
\end{itemize}
\end{solution}

\part[10]  Senate Democrats delivered a major victory to President Obama on Thursday when they blocked a Republican resolution to reject a six-nation nuclear accord with Iran, ensuring that the landmark deal will take effect without a veto showdown between Congress and the White House.

\vspace{1.5in}

\part[10] Senator Jones says that we should not fund the attack submarine program. I disagree entirely. I can't understand why he wants to leave us defenseless like that.

\vspace{1in}

\part[10] There are those on campus who would defend a student's right to display a Confederate flag on his or her dorm room. But there is no such right. Slavery was wrong, is wrong, and always will be wrong.
\begin{solution}
\begin{itemize}
\item This is an argument. 
\item It is deductive. 
\item Premise(s): Slavery was wrong, is wrong, and always will be wrong. 
\item Conclusion: Students do not have the right to display a Confederate Flag on their dorm room. 
\item It is invalid. 
\item It contains a fallacy---Red Herring. 
\end{itemize}
\end{solution}






\end{parts}





\end{questions}



\end{document}