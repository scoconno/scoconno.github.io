\documentclass[10pt]{article}

\usepackage{fancyhdr}
 \pagestyle{fancy}
\rhead{\textsc{Scott O`Connor}}
\lhead{Final Projects}
\usepackage{amssymb,amsmath}
\usepackage{ifxetex,ifluatex}
\usepackage{fixltx2e} % provides \textsubscript
\ifnum 0\ifxetex 1\fi\ifluatex 1\fi=0 % if pdftex
  \usepackage[T1]{fontenc}
  \usepackage[utf8]{inputenc}
\else % if luatex or xelatex
  \ifxetex
    \usepackage{mathspec}
    \usepackage{xltxtra,xunicode}
  \else
    \usepackage{fontspec}
  \fi
  \defaultfontfeatures{Mapping=tex-text,Scale=MatchLowercase}
  \newcommand{\euro}{€}
\fi
% use upquote if available, for straight quotes in verbatim environments
\IfFileExists{upquote.sty}{\usepackage{upquote}}{}
% use microtype if available
\IfFileExists{microtype.sty}{%
\usepackage{microtype}
\UseMicrotypeSet[protrusion]{basicmath} % disable protrusion for tt fonts
}{}
\ifxetex
  \usepackage[setpagesize=false, % page size defined by xetex
              unicode=false, % unicode breaks when used with xetex
              xetex]{hyperref}
\else
  \usepackage[unicode=true]{hyperref}
\fi
\usepackage[usenames,dvipsnames]{color}
\hypersetup{breaklinks=true,
            bookmarks=true,
            pdfauthor={},
            pdftitle={Signature Assignment The Examined Life (PHIL 140)},
            colorlinks=true,
            citecolor=blue,
            urlcolor=blue,
            linkcolor=magenta,
            pdfborder={0 0 0}}
\urlstyle{same}  % don't use monospace font for urls
\setlength{\parindent}{0pt}
\setlength{\parskip}{6pt plus 2pt minus 1pt}
\setlength{\emergencystretch}{3em}  % prevent overfull lines
\providecommand{\tightlist}{%
  \setlength{\itemsep}{0pt}\setlength{\parskip}{0pt}}
\setcounter{secnumdepth}{0}

\title{Signature Assignment \\ The Examined Life (PHIL 140)}
\author{Scott O’Connor}


% Redefines (sub)paragraphs to behave more like sections
\ifx\paragraph\undefined\else
\let\oldparagraph\paragraph
\renewcommand{\paragraph}[1]{\oldparagraph{#1}\mbox{}}
\fi
\ifx\subparagraph\undefined\else
\let\oldsubparagraph\subparagraph
\renewcommand{\subparagraph}[1]{\oldsubparagraph{#1}\mbox{}}
\fi

\begin{document}
\maketitle

\subsection{Final Projects}\label{final-projects}

This assignment will be completed in several stages. You will work in groups of 3--4.
Your group will pick a topic from the list below. You will research the
current status of the topic in question and then use the ethical
theories introduced in Module 6 to address it.

\subsection{Pick a Topic (Class
Assignment)}\label{pick-a-topic-class-assignment}

Each group picks one of the questions from the list below. Note that
each group must pick a different topic:

\begin{enumerate}
\def\labelenumi{\arabic{enumi}.}
\tightlist
\item
  Is it permissible to abort a fetus?
\item
  Ought a doctor help her terminally ill patient die if the patient
  requests it?
\item
  Should we ever execute people for their crimes?
\item
  Should we allow people to reproduce by cloning?
\item
  Should prostitution be legal?


\item
  Should recreational drug use be illegal?
\item
  Is it permissible to eat meat?
\item
  On whom are we allowed to perform medical trials?
\item
  Should corporations be considered persons?
\item
  Should we curb climate change for the sake of future generations?
\end{enumerate}

\subsection{Draft 1: Background
Research}\label{draft-1-background-research}

In the first part of your project, you will be collecting and thinking
through the information needed to write your final paper. Decide between
yourselves how to fairly proportion the work, but clearly indicate in
your submission how you divided up the tasks. Note that you will be
grading each others' level of participation.

Applied Ethicists are more concerned with particular, practical cases
than with more abstract theoretical questions. They want to know how, if
at all, a hospital should distribute donated organs, whether it is
permissible to bribe officials in foreign states to do business, etc.
The questions listed above are all hot topics in the United States. Your
job will be to answer them as an Applied Ethicist. In this first draft,
write three short documents:

\begin{enumerate}
\def\labelenumi{\arabic{enumi}.}
\tightlist
\item
  After researching articles at www.nj.com from the last 5 years,
  outline the major downsides and benefits of the current policy governing your chosen issue for society as a whole. What are the likely major downsides and benefits for society as a whole if the policy is changed? Provide full citations to the newspaper articles. 
  (600--800 words)
\item
 After researching articles at www.nj.com, identify some specific individuals who have been affected by your chosen issue.  Detail the real life downsides and benefits to everyone involved. What would the benefits and downsides have been if the event had not occurred? Back up your answer with evidence from the newspaper articles.    (600--800 words)

\item
  Explain both act utilitarianism and rule utilitarianism. How are they
  different? (600--800 words) Use the textbook and include proper citations.

\end{enumerate}

\subsection{Oral Presentation}\label{oral-presentation}

Each group will prepare a power point presentation on their chosen
topic. Each presentation should be 10--15 minutes long with a minimum of
6 slides and a maximum of 12.

\begin{itemize}
\tightlist
\item
  The presentation should \textbf{summarize} the findings of the threes
  documents prepared in the research phase.
\item
  Each member of the group must present at least one slide.
\end{itemize}

\subsection{Final Submission}\label{final-submission}

You must prepare your final submission by yourself, not in groups. You
can refer to the documents prepared by your group as background
research, but you must cite those group members who were responsible for
those documents.

\textbf{Prompt:} Compare and contrast how the act and rule utilitarian
would answer your chosen question. Which theory has the most unsettling
consequences? Give reasons for your answer. (1250--1500 words)

\subsection{Grade Breakdown}\label{grade-breakdown}

\begin{enumerate}
\def\labelenumi{\arabic{enumi}.}
\tightlist
\item
  Background Research: 10 pts.
\item
  Oral Presentation: 10 pts.
\item
  Final Submission: 10 pts.
\item
  Peer Evaluation of Work Contribution: 5 pts.
\item
  \textbf{Total Points: 35pts}
\end{enumerate}

The first two grades will be awarded to your group contribution, the
third to your individual final paper, and the fourth by averaging each
group member's evaluation of your level of participation.

\begin{itemize}

 \item \textit{12/07/2015,} Research documents in-class peer review. Bring hard copies to class.  

\item \textit{12/09/2015,} Research documents due through Blackboard by 9:55. 

\item \textit{12/14/2015,} In-class presentations.


\item \textit{12/18/2015,} Final submission due through Blackboard by 1:00pm. 

\end{itemize}

\subsection{General Education Program Assessment} General Education courses participate in programmatic assessment of the six University-wide student learning goals. They include instruction in, and assessment of, at least two of these learning goals. Signature assignments, which may include document, picture, sound, or video files, are uploaded to a secure server for anonymous distribution to the NJCU assessment team, which scores them using approved program rubrics. While instructors also grade their own students’ signature assignments, which count toward the course grade, assessment team results are aggregated to provide information about the Gen Ed program as a whole. Your name will not be included in any programmatic assessment data.

\subsection{Academic Integrity} All the work you turn in (including papers, drafts, and discussion board posts) must be written by you specifically for this course. It must originate with you in form and content with all contributory sources fully and specifically acknowledged. Being a student at NJCU requires you to follow \href{http://www.njcu.edu/uploadedFiles/About_NJCU/Governance_and_Organization/University_Senate/Policies/Academic\%20INTEGRITY\%20POLICY\%20FINAL\%202-04.pdf}{NJCU's Academic Integrity Policy.} Penalties for violations are as follows: 1st infraction will result in a 0 for the assignment.  2nd infraction will result in a 0 for the entire course \& application for permanent record on student's transcript. (Repeated violations can lead to expulsion from NJCU). 

\begin{itemize}
\item If a document is plagiarized during the research phase, the penalty will apply to the group as a whole. Make sure you are happy with each document before it is submitted.
\end{itemize}

\end{document}
