\documentclass[]{article}

\usepackage{fancyhdr}
 \pagestyle{fancy}
\rhead{\textsc{Scott O`Connor}}

\usepackage{lmodern}
\usepackage{amssymb,amsmath}
\usepackage{ifxetex,ifluatex}
\usepackage{fixltx2e} % provides \textsubscript
\ifnum 0\ifxetex 1\fi\ifluatex 1\fi=0 % if pdftex
  \usepackage[T1]{fontenc}
  \usepackage[utf8]{inputenc}
\else % if luatex or xelatex
  \ifxetex
    \usepackage{mathspec}
    \usepackage{xltxtra,xunicode}
  \else
    \usepackage{fontspec}
  \fi
  \defaultfontfeatures{Mapping=tex-text,Scale=MatchLowercase}
  \newcommand{\euro}{€}
\fi
% use upquote if available, for straight quotes in verbatim environments
\IfFileExists{upquote.sty}{\usepackage{upquote}}{}
% use microtype if available
\IfFileExists{microtype.sty}{%
\usepackage{microtype}
\UseMicrotypeSet[protrusion]{basicmath} % disable protrusion for tt fonts
}{}
\ifxetex
  \usepackage[setpagesize=false, % page size defined by xetex
              unicode=false, % unicode breaks when used with xetex
              xetex]{hyperref}
\else
  \usepackage[unicode=true]{hyperref}
\fi
\usepackage[usenames,dvipsnames]{color}
\hypersetup{breaklinks=true,
            bookmarks=true,
            pdfauthor={},
            pdftitle={Project},
            colorlinks=true,
            citecolor=blue,
            urlcolor=blue,
            linkcolor=magenta,
            pdfborder={0 0 0}}
\urlstyle{same}  % don't use monospace font for urls
\setlength{\parindent}{0pt}
\setlength{\parskip}{6pt plus 2pt minus 1pt}
\setlength{\emergencystretch}{3em}  % prevent overfull lines
\providecommand{\tightlist}{%
  \setlength{\itemsep}{0pt}\setlength{\parskip}{0pt}}
\setcounter{secnumdepth}{0}

\title{Project}
\author{Scott O’Connor}


% Redefines (sub)paragraphs to behave more like sections
\ifx\paragraph\undefined\else
\let\oldparagraph\paragraph
\renewcommand{\paragraph}[1]{\oldparagraph{#1}\mbox{}}
\fi
\ifx\subparagraph\undefined\else
\let\oldsubparagraph\subparagraph
\renewcommand{\subparagraph}[1]{\oldsubparagraph{#1}\mbox{}}
\fi

\begin{document}

\subsection{Signature Assignment (Final
Project)}\label{signature-assignment-final-project}

\subsubsection{Special Submission
Instruction}\label{special-submission-instruction}

\begin{itemize}
\tightlist
\item
  Presentation slides must be submitted through \textbf{Blackboard}.
\item
  The full paper must be submitted through NJCU's Tk20 site to earn a
  final grade. Details are here:
  \url{http://www.njcu.edu/general-education/signature-assignment-information-students}
\end{itemize}

\subsubsection{Introduction}\label{introduction}

Our course began by investigating the meaning of life. We asked whether
our lives have any external value that is not robbed by our inevitable
death. Tolstoy was initially depressed at the apparent futility of our
lives and considered suicide. The meaning of life then is intimately
connected to questions about mortality and death. In this final part of
the course, you will be investigating one of the following ethical
questions about death (your choice!)

\begin{enumerate}
\def\labelenumi{\arabic{enumi}.}
\item
  Is suicide immoral?
\item
  Should a doctor help her ill patient commit suicide?
\item
  Should we ever execute people for their crimes?
\item
  Should we allow coma patients to remain on life support indefinitely?
\item
  Should it be illegal for a person to risk their life by engaging in
  dangerous activities, e.g., hard drug use, extreme sports, etc?
\item
  Should people with serious heritable diseases be allowed to have
  children?
\item
  Should it be illegal for pregnant women to engage in activities that
  risk the life of their fetus, e.g., drinking and taking drugs while
  pregnant?
\item
  Should people have children if they learn that the Earth will be
  unable to support life in the very near future, either due to a
  changing climate, or a meteor, etc?
\item
  Alternative student question approved by the instructor over
  Blackboard by April 24th.
\end{enumerate}

\subsubsection{Task}\label{task}

The questions listed above are all hot topics in the United States. To
answer your question, you need to think both about the effects your
issue has on society as a whole as well as the effect it has on
individuals within that society. To complete the assignment, you will do
the following three things:

\begin{description}
\tightlist
\item[Task 1:]
After researching articles at www.nj.com, identify some specific
individual who has been affected by your chosen issue. Detail the real
life costs and benefits to that person, their friends, and their
families. What would the benefits and costs have been if the event had
not occurred? Back up your answer with evidence from the newspaper
articles. (600--800 words)
\item[Task 2:]
After researching articles at www.nj.com from the last 5 years, outline
the major costs and benefits of the current policy governing your chosen
issue for society as a whole. What are the likely major costs and
benefits for society as a whole if the policy is changed? Provide full
citations to the newspaper articles. (600--800 words)
\item[Task 3:]
Given the information that you presented in the first two parts of your
paper, state your answer to your chosen question and explain how the
information from the first two parts of the paper influenced your
answer. (400--600 words)
\end{description}

\subsubsection{Presentation}\label{presentation}

Each of you will give an in class presentation on your chosen topic
before the final submission of the paper. Each presentation must include
four slides as follows: 

\begin{description} 
\item[Slide 1:] introduce your question. Define any
relevant terms in the question and say a little about why it is
interest.
\item[Slide 2:] summary of task 1. 
\item[Slide 3:] summary of task 2. 
\item[Slide 4:] summary of task 3.
\end{description}

\textbf{NB:} Don't fill your slides with text. Pictures, graphs, a few
bullet points are much easier to follow. The slides will then be used to
structure your presentation.

\subsubsection{Grade Breakdown}\label{grade-breakdown}

This paper is worth 20 points towards your final grade. 5 points will be
awarded towards the presentation and 15 points toward the final draft.

\subsubsection{Plagiarism}\label{plagiarism}

Please review the plagiarism policy on the syllabus. It is critical that
you prepare your assignment by yourself. Use only the textbook and
handouts---it will take you less time to work through these materials
than to find and read other sources. I will be checking for significant
overlaps between submission as well as checking answers against
Wikipedia, internet search results, standard essay sites, etc. If you
include material in your essay without citing it, you will receive 0 for
the assignment. A second violation will result in a 0 for the course, a
report to the Dean, and a petition for a note to be added to your
permanent academic record.

\subsubsection{Late Submissions}\label{late-submissions}

Per the policies outlined in the syllabus, late work will not be
accepted. As the policies also state, there are no make-ups or extra
credit opportunities. Any request for special treatment will be ignored.
If you foresee difficulties submitting work on time, either because of
personal or commitments, then you should start this paper early and
submit it early.

\subsubsection{Format}\label{format}

Please submit the file as either a word file or simple .rtf file will
also suffice.

\subsubsection{Grading}\label{grading}

Please find the rubric and explanation of it
\href{/Teaching/Grading/}{here}.

\subsubsection{Resources}\label{resources}

Please find links to writing resources \href{/Teaching/Resources/}{here}

\end{document}
