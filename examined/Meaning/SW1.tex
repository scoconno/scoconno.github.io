\documentclass[]{article}
\usepackage{lmodern}
\usepackage{amssymb,amsmath}
\usepackage{ifxetex,ifluatex}
\usepackage{fixltx2e} % provides \textsubscript
\ifnum 0\ifxetex 1\fi\ifluatex 1\fi=0 % if pdftex
  \usepackage[T1]{fontenc}
  \usepackage[utf8]{inputenc}
\else % if luatex or xelatex
  \ifxetex
    \usepackage{mathspec}
    \usepackage{xltxtra,xunicode}
  \else
    \usepackage{fontspec}
  \fi
  \defaultfontfeatures{Mapping=tex-text,Scale=MatchLowercase}
  \newcommand{\euro}{€}
\fi
% use upquote if available, for straight quotes in verbatim environments
\IfFileExists{upquote.sty}{\usepackage{upquote}}{}
% use microtype if available
\IfFileExists{microtype.sty}{%
\usepackage{microtype}
\UseMicrotypeSet[protrusion]{basicmath} % disable protrusion for tt fonts
}{}
\ifxetex
  \usepackage[setpagesize=false, % page size defined by xetex
              unicode=false, % unicode breaks when used with xetex
              xetex]{hyperref}
\else
  \usepackage[unicode=true]{hyperref}
\fi
\hypersetup{breaklinks=true,
            bookmarks=true,
            pdfauthor={},
            pdftitle={Essay},
            colorlinks=true,
            citecolor=blue,
            urlcolor=blue,
            linkcolor=magenta,
            pdfborder={0 0 0}}
\urlstyle{same}  % don't use monospace font for urls
\setlength{\parindent}{0pt}
\setlength{\parskip}{6pt plus 2pt minus 1pt}
\setlength{\emergencystretch}{3em}  % prevent overfull lines
\setcounter{secnumdepth}{0}

\title{Essay}
\date{}

\begin{document}
\maketitle

\subsection{Dialog on the Meaning of
Life}\label{dialog-on-the-meaning-of-life}

\subsubsection{Introduction}\label{introduction}

Flex your imagination. You are a disembodied soul that has yet to see
this world. You know nothing about what life you may live. You know
nothing about what talents you might be born with, about whether you
will be able or disabled, about whether you will be rich or poor. You do
not know what country you will be born in and you know nothing about the
parents you will be born to. In short, you have no idea whether you will
be born to a long life of comfort and ease or a short life of pain and
trouble. You could be born to a wealthy family in New York or an
impoverished family in a war torn country. You don't know.

The only thing you are aware of are two angels nearby. They offer you a
choice. You can choose a body, to be born, live a life and die. Or you
can choose to fade away into the peaceful nothingness from which you
emerged. The angels disagree about what you should do. One tries to
convince you that life is meaningless, that it is better to be nothing
at all than to be a being who is born, lives a while, and dies. The
other disagrees. They claim that regardless of the circumstances of your
life, God's existence guarantees that it will be something full of
meaning and purpose.

\subsubsection{Purpose}\label{purpose}

The purpose of this assignment is to help you practice the following
skills that are essential to your success in this course and others.

\begin{enumerate}
\def\labelenumi{\arabic{enumi}.}
\itemsep1pt\parskip0pt\parsep0pt
\item
  Charitably explaining arguments for opposing claim.
\item
  Explaining why you think one argument succeeds where another fails.
\item
  Explaining difficult concepts in your own words.
\end{enumerate}

\subsubsection{Task}\label{task}

Write a dialogue between you and the two angels. Your dialogue must
contain the following three elements:

\begin{itemize}
\itemsep1pt\parskip0pt\parsep0pt
\item
  The first angel must explain Tolstoy's initial argument that life is
  meaningless. This angel must discuss internal and external value.
\item
  The second angel must explain Tolstoy's subsequent religious
  conversion and argument that life is meaningful. This angel must
  explain in what way God's existence would address Tolstoy's initial
  worry.
\item
  You must choose between life and nothing and explain that choice. Tell
  them why you found one argument convincing, the other unconvincing.
\end{itemize}

\textbf{NB:} Failure to include any of these elements in your dialog
will lose you points.

\subsubsection{Word Count}\label{word-count}

Your dialog must be 500-750 words long. Essays shorter than 500 words or
longer than 750 words will lose points.

\subsubsection{Further Instruction}\label{further-instruction}

\begin{itemize}
\itemsep1pt\parskip0pt\parsep0pt
\item
  This assignment covers material contained in Ch.9.1--9.3.
\item
  Note that this is a dialog. Imagine the way dialogs go. There is some
  back and forth. There is humor. There are jibes. Characters have
  names. Write your dialog accordingly. See any play for some ideas on
  format.
\item
  Allow your characters to question each other, rebut each other, etc.
\item
  Your choice between the two positions should be guided by your
  evaluation of both arguments. Don't merely say which conclusion you
  accept. This part of the dialog should focus on assessing the logic
  and premises of the respective arguments.
\end{itemize}

\subsubsection{Plagiarism}\label{plagiarism}

Please review the plagiarism policy on the syllabus. It is critical that
you prepare your assignment by yourself. Use only the textbook and
handouts---it will take you less time to work through these materials
than to find and read other sources. I will be checking for significant
overlaps between submission as well as checking answers against
Wikipedia, internet search results, standard essay sites, etc. If you
include material in your essay without citing it, you will receive 0 for
the assignment. A second violation will result in a 0 for the course, a
report to the Dean, and a petition for a note to be added to your
permanent academic record.

\subsubsection{Due Date}\label{due-date}

Please consult the syllabus for the due date.

\subsubsection{Late Submissions}\label{late-submissions}

Per the policies outlined in the syllabus, late work will not be
accepted. Any request for special treatment will be ignored. If you
foresee difficulties submitting work on time, either because of personal
or work commitments, then you should start this paper early and submit
it early.

\subsubsection{Format}\label{format}

Please submit the file as a word file.

\subsubsection{Grading}\label{grading}

Please find the rubric and explanation of it
\href{/Teaching/Grading/}{here}.

\subsubsection{Resources}\label{resources}

Please find links to writing resources \href{/Teaching/Resources/}{here}

\end{document}
