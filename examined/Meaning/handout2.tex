\documentclass[]{article}

\usepackage{amssymb,amsmath}
\usepackage{ifxetex,ifluatex}
\usepackage{fixltx2e} % provides \textsubscript
\ifnum 0\ifxetex 1\fi\ifluatex 1\fi=0 % if pdftex
  \usepackage[T1]{fontenc}
  \usepackage[utf8]{inputenc}
\else % if luatex or xelatex
  \ifxetex
    \usepackage{mathspec}
    \usepackage{xltxtra,xunicode}
  \else
    \usepackage{fontspec}
  \fi
  \defaultfontfeatures{Mapping=tex-text,Scale=MatchLowercase}
  \newcommand{\euro}{€}
\fi
% use upquote if available, for straight quotes in verbatim environments
\IfFileExists{upquote.sty}{\usepackage{upquote}}{}
% use microtype if available
\IfFileExists{microtype.sty}{%
\usepackage{microtype}
\UseMicrotypeSet[protrusion]{basicmath} % disable protrusion for tt fonts
}{}
\ifxetex
  \usepackage[setpagesize=false, % page size defined by xetex
              unicode=false, % unicode breaks when used with xetex
              xetex]{hyperref}
\else
  \usepackage[unicode=true]{hyperref}
\fi
\hypersetup{breaklinks=true,
            bookmarks=true,
            pdfauthor={},
            pdftitle={Optimism with God},
            colorlinks=true,
            citecolor=blue,
            urlcolor=blue,
            linkcolor=magenta,
            pdfborder={0 0 0}}
\urlstyle{same}  % don't use monospace font for urls
\setlength{\parindent}{0pt}
\setlength{\parskip}{6pt plus 2pt minus 1pt}
\setlength{\emergencystretch}{3em}  % prevent overfull lines
\setcounter{secnumdepth}{0}

\title{Optimism with God}
\date{}

\begin{document}
\maketitle

\section{Optimism about the meaning of life:
God}\label{optimism-about-the-meaning-of-life-god}

\subsection{Introduction}\label{introduction}

Recall Tolstoy's question:

\begin{quote}
\ldots{} My question - that which at the age of fifty brought me to the
verge of suicide - was the simplest of questions, lying in the soul of
every man from the foolish child to the wisest elder: it was a question
without an answer to which one cannot live, as I had found by
experience. It was: ``What will come of what I am doing today or shall
do tomorrow? What will come of my whole life?'' (Tolstoy,
p.14)\footnote{Tolstoy, Leo, `A Confession', 1882}
\end{quote}

\begin{quote}
Differently expressed, the question is: ``Why should I live, why wish
for anything, or do anything?'' It can also be expressed thus: ``Is
there any meaning in my life that the inevitable death awaiting me does
not destroy? (Tolstoy, p.14)
\end{quote}

Life has meaning only if it has significant value or purpose over time,
where this value makes life choice worthy. There are two different ways
of understanding this value:

\begin{itemize}
\item
  \textbf{Internal Value:} the value or purpose that comes when people
  see their goals or purposes as inherently valuable or worthwhile.
\item
  \textbf{External Value:} Meaning or purpose that comes from outside of
  ourselves in relationship to something that we may or may not be aware
  of.
\end{itemize}

When we ask about the meaning of life, we are asking about internal
value. We are asking why we should feel that there is something in our
lives that makes them worthwhile. Is there any project or goal that
could shape our psychology so dramatically that we are motivated to get
up in the morning, keep going, and find all the trials and tribulations
of life worthwhile? Pessimists, recall, claim no. Their argument: 1.
Life is choice worthy only if it has internal value. 2. Life has
internal value only if life has external value. 3. Life has no external
value. 4. Life has no internal value (from 1--3). 5. Life is not choice
worthy (from 1 \& 4). This argument is valid; the conclusion follows
form the premises. Is it sound, i.e., are the premises true? The most
important Premises are 2 and 3, which we saw Tolstoy arguing for via a
fable. We can summarize his argument for Premise 2 as follows: + a. I
will find some project/goal valuable over a long period of time, only if
I believe that project/goal is externally valuable. + b. None of my
projects/goals are externally valuable. + c. I will inevitably discover
that my projects/goals have no external value. + d. I will inevitably
cease to find internal value in my life (from a--c). + e. I will
inevitably cease to find life choice worthy (from d) Why did Tolstoy
accept Premise 3? I suggested that Tolstoy assumes that a goal or
project has external value only if that goal contributes towards some
eternal enterprise. \#\# OptimismOptimists claim the Tolstoy's arguments
for Premises 2 and 3 fail. There are two versions of Optimism. The first
version accepts Premises 1 and 2, but rejects Premise 3. They find
external value in religion. The second type of Optimist accepts Premise
3, that life has no external value, but denies that internal value
depends on there being external value, i.e., they deny Premise 2. The
first type of Optimism is associated with Theism, the second with
Atheism. This note discusses Theism. A subsequent note will discuss
Atheism.

\subsection{Theism}\label{theism}

Tolstoy did not remain depressed. He reports meeting and talking with
rural farmers at a time in Russia when farmers lived a menial existence.
They had nothing. Yet Tolstoy sees in these rural farmers an acceptance
of life's vicissitudes. Reclaiming his faith, he realized that he had
found value in the wrong things. His art, his family, etc., could never
provide the meaning he had sought. None of these could live forever. But
since God is an eternally perfect being, Tolstoy thought He could
provide that value. These two claims summarize Tolstoy's new optimistic
views:

\begin{enumerate}
\def\labelenumi{\arabic{enumi}.}
\item
  A human's life has external meaning only because it is part of God's
  plan, a grand cosmic order that encompasses every entity in the
  universe.
\item
  A human's life can have internal meaning if they align their
  life---their goals, projects, ambitions---with God's plan.
\end{enumerate}

On this view, life will be choice worthy if you can identify God's plan
for you and set about realizing that plan.

\subsection{Meaning \& Christianity}\label{meaning-christianity}

Tolstoy doesn't detail how Christianity construes the meaning of life.
He merely says that if you believe that God exists, then you can see
that your life has some external value. But can we say something more
about what this external value consists in?

At this point, Christians point to one person, Jesus, who they claim had
a meaningful life. Reflecting on the details of his life, we might
construe the internal and external value of his life as follows:

\begin{itemize}
\itemsep1pt\parskip0pt\parsep0pt
\item
  \textbf{Prime Example:} His life had a purpose. All of Jesus' life
  involves suffering for the sake of mankind. Saving mankind is the
  external value his life had. All the aspects of his life were
  organized around this one overarching purpose; he thought his life had
  God given external purpose and spent his time and energy trying to
  fulfill that purpose.
\end{itemize}

Christians also emphasizes other characters with God given purposes.
Noah, the Saints, the Apostles, are all individuals who were supposedly
given important jobs by God. These tasks, these jobs, give their life
external value according to Christianity. Their lives also had internal
value because they identified their God given purposes, saw them as
worthwhile, and devoted their time and energy to realizing them.

\subsection{Objection}\label{objection}

Here I briefly raise a problem for this account of the meaning of life.
Suppose we grant that the lives of Jesus, Noah, the other Saints had
external value because each of them had a God given purpose. It does not
thereby follow that each human has a God given purpose, i.e., it does
not follow that you have a God given purpose. And the candidates for
this God given purpose seem unsatisfactory:

\begin{itemize}
\itemsep1pt\parskip0pt\parsep0pt
\item
  \textbf{Suggestion 1:} Our purpose is to serve God.
\end{itemize}

This is a very natural suggestion. A Theist might claim that we were
created by God to do his will. That would seem to give our lives the
significance we desired. The problem, though, is that being in service
to someone is not obviously a thing we would always choose. Granted, if
God created us to serve him, then our lives would be significant to God.
But notice that bees are significant to the bee keeper, yet that hardly
shows us why a bee should find its own life significant.

To motivate this objection, consider the very far fetched idea that God
created us to perform a very specific role. Once our species has grown
large enough, he will signal to an alien race to move to Earth where
they will find a new rich food source. Us! If this were the case, God
would have crated us to be the food in some alien's hamburger. We would
have a role in his grand design. We would even know what it is. I doubt,
though, that anyone would be happy to find out that they were created as
food for some superior being. A menial role in a stage designed for
another does not make life choice worthy.

There is a second worry with the claim that God created us to serve him.
God is all loving, all knowing, and all powerful. If he is all loving,
he would never have created us merely to serve him, especially since our
lives involve so much suffering and pain. Suppose that we had the
ability to create a a new fully conscious species. It is only an evil
creator that would create such beings to suffer and toil in servitude to
them.

Let us consider another way that the Theist might respond to the
objection.

\begin{itemize}
\itemsep1pt\parskip0pt\parsep0pt
\item
  \textbf{Suggestion 2:} The existence of God shows that we have a
  purpose, but we do not know what that purpose is.
\end{itemize}

God is all loving, therefore he would never have created us without a
purpose. However, since he is all loving, he would never have created us
merely to serve him. Nevertheless, we do not know why he created us. We
do not know to what end he intended our lives to serve.

This also seems a natural suggestion. The idea is that we do have a
purpose, but we just do not know what it is. The problem is that the
suggestion avoids answering the question at hand. How, if at all, would
the existence of God provide life with external meaning?

By themselves, these objections do not completely undermine the Theist's
account of the meaning of life. What they do, however, is show that
belief in God should not in itself be comforting. For God's existence to
be comforting, we need to know why he created us, to what end our lives
serve. Unless those details are forthcoming, the unsettling possibility
is left open that he created us for reasons that none of us should be
happy to live with.

\end{document}
