\documentclass[]{article}

\usepackage{amssymb,amsmath}
\usepackage{ifxetex,ifluatex}
\usepackage{fixltx2e} % provides \textsubscript
\ifnum 0\ifxetex 1\fi\ifluatex 1\fi=0 % if pdftex
  \usepackage[T1]{fontenc}
  \usepackage[utf8]{inputenc}
\else % if luatex or xelatex
  \ifxetex
    \usepackage{mathspec}
    \usepackage{xltxtra,xunicode}
  \else
    \usepackage{fontspec}
  \fi
  \defaultfontfeatures{Mapping=tex-text,Scale=MatchLowercase}
  \newcommand{\euro}{€}
\fi
% use upquote if available, for straight quotes in verbatim environments
\IfFileExists{upquote.sty}{\usepackage{upquote}}{}
% use microtype if available
\IfFileExists{microtype.sty}{%
\usepackage{microtype}
\UseMicrotypeSet[protrusion]{basicmath} % disable protrusion for tt fonts
}{}
\ifxetex
  \usepackage[setpagesize=false, % page size defined by xetex
              unicode=false, % unicode breaks when used with xetex
              xetex]{hyperref}
\else
  \usepackage[unicode=true]{hyperref}
\fi
\hypersetup{breaklinks=true,
            bookmarks=true,
            pdfauthor={},
            pdftitle={The Design Argument},
            colorlinks=true,
            citecolor=blue,
            urlcolor=blue,
            linkcolor=magenta,
            pdfborder={0 0 0}}
\urlstyle{same}  % don't use monospace font for urls
\setlength{\parindent}{0pt}
\setlength{\parskip}{6pt plus 2pt minus 1pt}
\setlength{\emergencystretch}{3em}  % prevent overfull lines
\setcounter{secnumdepth}{0}

\title{The Design Argument}
\date{}

\begin{document}
\maketitle

\subsection{Introduction}\label{introduction}

Non-philosophers often claim that their belief in God is a matter of
faith. God's existence, they claim, is something that cannot be proved
or disproved. One rather must take a `leap of faith' and accept that he
does, in fact, exist.

Philosophers disagree. Since Aristotle in the 4th Century BC,
philosophers have offered a number of arguments both for and against his
existence. We are concerned in this handout with one argument for his
existence, the Design Argument. The second handout in the module
introduces further arguments for his existence while the third discusses
arguments against his existence.

\subsection{The Design Argument}\label{the-design-argument}

One of our arguments for God's existence, the Design Argument, is a
topic of public controversy. Several states in the U.S. require high
school biology teachers to teach, or at least mention, the possibility
that an intelligent designer, as opposed to natural selection (see
below), is responsible for the existence of life. These states, whether
they realize it or not, are taking a side in a contentious philosophical
argument. I'll first outline the argument before discussing some
objections to it.

The Design Argument begins by noting that the world and its inhabitants
are beautifully complex entities. Consider the eye. Our ability to see
requires incredibly fine calibrations of the light cones inside the eye.
Some have argued that this complexity is evidence for God, an
intelligent designer. These philosophers argue that the only way such
complexity could exist is if God exists and created it.

One influential version of the Design Argument comes from William Paley,
an English Philosopher who died in 1805. Paley asks us to imagine
encountering a stone in a field. There is nothing surprising about the
stone being in that field. It was likely there for an indefinite period
of time. Now suppose you walk a little further in the field and
encounter a watch. It's very likely that you will stop and ask yourself
where the watch came from. It seems impossible that it could just have
been in the field for an indefinite period of time.

Why? What distinguishes the watch from the stone? Upon inspecting the
watch, you will discover that it has several parts which are put
together for a purpose, e.g., the parts are so formed and adjusted as to
produce motion, and that motion so regulated as to tell time. You also
discover that if the different parts had been differently shaped, or of
a different size, that the device would never have told time.

Discovering the complexity of the watch, you likely will conclude that
the watch must have had a maker; that there must have existed, at
sometime, and at some place or other, a maker or makers, who formed the
watch for the purpose of telling time. The stone, in contrast, has no
similar complexity. It does not have parts put together for some
purpose. We have no reason, then, to think that the stone, as opposed to
the watch, was put together by some maker.

\subsection{The Structure of the
Argument}\label{the-structure-of-the-argument}

Paley's interest is God, not watches. He discusses watches to illustrate
a general point about complexity, design, and creators. He thinks that
these general points will apply to organisms just as much as to watches,
and thus show that organisms too must have creators:

\begin{enumerate}
\def\labelenumi{\arabic{enumi}.}
\itemsep1pt\parskip0pt\parsep0pt
\item
  If an object contains various parts that are arranged to achieve some
  purpose, then that object appears to have a design.
\item
  If an object appears to have a design, then that object was created by
  some creator.
\item
  Organisms have parts that are arranged to achieve some purpose.
\item
  Therefore, organisms appear to have a design. (from 1 \& 3).
\item
  Therefore, organisms were created by some creator (from 2 \& 4).
\item
  Therefore, God exists (from 5).
\end{enumerate}

Paley argued for Premise 1 \emph{via} the analogy of the stone and
watch. The reason why we think the watch was designed, but the stone was
not, is precisely the fact that the parts of the watch are perfectly
arranged for the purpose of telling time. The parts of a stone, on the
other hand, are not arranged to achieve any obvious purpose.

We will have more to say about Premise 2 below. For the moment, notice
that we normally would conclude that an object exhibiting some design,
like a watch, was created by some designer.

Support for Premise 3 comes from our detailed studies of organisms and
their parts. Any physiology textbook will tell you the purpose of each
bodily system. Our heart pumps oxygenated blood to the extremities. Our
eyes and hears collect information about our environment. Each of these
bodily parts has a purpose that is achieved only because of the
intricate make up of those parts; if the cones in the eye are not
situated correctly, vision will not be produced.

Premises 4 and 5 are inferences from earlier premises. And 6 states our
conclusion. Paley thinks that if organisms have a creator, then God is
that creator, and, so, God exists.

\subsection{Objections \& Responses}\label{objections-responses}

I have presented Paley's argument as a deductive one. Evaluating a
deductive argument requires us to determine whether the argument is
valid and sound. (If you don't remember these distinctions, please study
Ch.1.3 again. These notions will be used in each module.)

It is also important in evaluating philosophical arguments to
distinguish between objections that are easy to respond to \emph{vs.}
those that are not easy to respond to, perhaps because those objections
are fatal.

Our first three objections are easy to respond to, the fourth is not. In
what follows, I will first list each objection, then offer some notes on
how the objection speaks to the original argument. I will also outline
Paley's response to the first three objections.

\subsection{Objection 1}\label{objection-1}

\textbf{Objection:} since we have never seen a watch made or know how to
make one, we cannot conclude that someone made the watch.

\textbf{Notes:} Let's think about how this objection is meant to work.
Is it attacking the validity or soundness of Paley's argument? The
complaint, again, is that unless we had seen someone making the watch,
we could not conclude that someone made the watch. Looking again at 1-6,
it's clear that this objection is really trying to deny Premise 2. It's
claiming that while something may appear designed for some purpose, it
does not thereby follow that we know it was designed by a creator for
that purpose. Since the objection is rejecting a premise, the objection
is claiming that the argument is not sound.

\textbf{Response:} Paley denies that we need to know how a watch is made
in order to conclude that the watch has a maker. He points out that we
extoll the artist who can do something we cannot do, e.g., we praise the
software engineer for their creation even if we have no idea how they
created it. I know, say, that MS Word was created by someone even though
I have no idea who created it or how they did it.

\textbf{Notes:} Paley responds to the objection by pointing out that it
relies on a false claim, namely, that we can only know that X has a
maker if we know who made X and how X was made.

\subsection{Objection 2}\label{objection-2}

\textbf{Objection:} There are some parts of organisms that seem to serve
no function, e.g., the appendix.

\textbf{Notes:} Again, ask yourself is this objection attacking the
validity of the argument or is it attacking a premise? Looking again at
1-6, you'll see that the objection is attacking Premise 3, which says
that organisms have parts so arranged to achieve some purpose. This
objection points out that there are some parts of an organism that serve
no purpose at all.

\textbf{Response:} Paley responds by clarifying Premise 3. Premise 3
does not require that every part of the organism serves some purpose.
His argument works as long as there are some parts that do; there is
still an explanation that is required of how those parts serve the
purpose they do. Paley also observes that the objection concedes to him
that organisms have parts that are purposive; the objector cannot
distinguish parts that do not have purposes from those that do unless
she concedes that there are parts of the latter kind. In addition, we
may note that while organisms may have some parts that serve no purpose,
these are few in number compared to the vast majority of parts that do.

\subsection{Objection 3}\label{objection-3}

\textbf{Objection:} Organisms are produced from organisms through normal
sexual and asexual reproduction. A baby chipmunk is not produced by God,
but by his or her parents.

\textbf{Notes:} Again, our first task is to see whether the objection is
attacking the validity or soundness of the argument. In this case, it is
the former. The objection does not deny any premise. It denies that the
conclusion follows from the premises, i.e., it denies that the argument
is valid. How? Look again at 6, which states the conclusion. 6 claims
that God exists. This is supposed to follow from Premise 5, the claim
that organisms have a creator (which itself follows from previous
premises). But, the objector presses, just because organisms have
creators, it does not follow that the creator must be God. Parent
chipmunks create babies. A scientist might create an organism in a lab.
In each case, the organism has a creator who is someone other than God.
It seems, then, that 5 could be true and 6 false (because we can
conceive of something other than God which creates organisms.) Since the
premises could be true and the conclusion false, the argument, the
objection claims, is invalid.

\textbf{Response:} Paley responds by denying that the objector has
provided examples of creation that do not involve God. He first observes
that organisms can only replicate because they contain within themselves
a reproductive system, a system of parts that precisely serve the role
of reproduction. Since these parts have a purpose, they have a design,
and hence, Paley, claims a creator. He asks us to imagine an inventor
who creates a robot who can collect material in the workshop and create
new robots by itself. This first robot creates new robots, robots that
are not directly created by the inventor, but the first robot can only
make new robots because it was created with the right parts and
instructions to allow it do so. So, Paley accepts that organisms create
other organisms, however, he claims that the only feasible explanation
for the ability of organisms to reproduce is that there exists a God who
created organisms with that ability.

\subsection{A Fatal Objection?}\label{a-fatal-objection}

The most powerful objection to the Design Argument comes from
evolutionary biology. For our purpose, we can take evolutionary biology
to attack Premise 2. It claims that organisms came to be how they are,
came to display the purposive behaviors that they do, by completely
random forces. This is a philosophy course, so I won't assume that you
have studied biology. Instead, I provide here two of the simplest and
shortest YouTube videos that will help you understand the very basics of
evolutionary biology:

\begin{itemize}
\itemsep1pt\parskip0pt\parsep0pt
\item
  \href{https://www.youtube.com/watch?v=GhHOjC4oxh8}{Random Mutation:}
  small changes in inheritable traits passed down through generations
  over long periods of time.
\item
  \href{https://www.youtube.com/watch?v=0SCjhI86grU}{Natural Selection:}
  the phenomenon that determines which traits survive and are passed
  down through generations.
\end{itemize}

After watching these two videos, you should see how evolutionary biology
attacks Premise 2. It claims that the appearance of design does not
entail the existence of a creator. Instead, the existence of design is
explained by two chance processes. 1) There are random mutations in the
genes that are transmitted from parents to offspring, mutations that
result in slight changes of appearance, function, and behavior. 2) By
pure chance, some of these changes provide to offspring an advantage in
their environment; they find it easier to get food, or escape predators,
or escape disease, etc. These advantages allow those offspring to breed.
Thus the genes responsible for the advantages are passed on.

What's key to note here is that this is all by chance, not by some
creator's intent. It is pure chance that those changes were advantageous
in that environment. In a different environment, those changes would
provide no advantage at all. Indeed, they are as likely to be a
disadvantage: offspring in that environment with those changes might be
more prone to an illness, easier prey, less likely to find food, and so
less likely to survive and reproduce.

\end{document}
