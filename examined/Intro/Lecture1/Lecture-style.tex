% Copyright 2007 by Till Tantau
% Modified by Roman Makarov
%
% This file may be distributed and/or modified
%
% 1. under the LaTeX Project Public License and/or
% 2. under the GNU Public License.
%
% See the file doc/licenses/LICENSE for more details.


\def\coursename{Introdocution to Philosophy}
\def\lecturername{Scott O'Connor }
\def\universityname{University of Maryland, Baltimore County}
\def\departmentname{Philosophy Department}


% Common packages

\usepackage[latin1]{inputenc}
\usepackage{times}
\mode<article>
{
  \usepackage{times}

%  \usepackage{mathptmx}
  \usepackage{bookman}

  \usepackage[left=1.5cm,right=6cm,top=1.5cm,bottom=3cm]{geometry}
}

\usepackage{hyperref}
\usepackage[T1]{fontenc}
\usepackage{tikz}
\usepackage{yfonts}
\usepackage{colortbl}
\usepackage{translator} % comment this, if not available
\usepackage{dsfont,mathbbol}
\usepackage{array}
\usepackage{amscd}

\def\E{{\mathbf{E}}}
\def\P{{\mathbf{P}}}
\def\R{{\mathds{R}}}
\def\C{{\mathds{C}}}
\def\mes{{\rm mes}\:\!}
\def\Var{{\rm Var}\:\!}
\def\Cov{{\rm Cov}\:\!}
\usepackage{graphicx}
\graphicspath{{Images/}}
\usepackage{amsmath,amssymb}
\usepackage{xxcolor}

\newcommand{\solutionwithspace}[1]{ \noindent\parbox[h][#1][t]{\linewidth}{\begin{flushleft}\textbf{Solution.}\end{flushleft}} }
\newcommand{\proofwithspace}[1]{ \noindent\parbox[h][#1][t]{\linewidth}{\begin{flushleft}\textbf{Proof.}\end{flushleft}} }
\newcommand{\emptyspace}[1]{ \parbox[h][#1][t]{\linewidth}{ } }

% Common settings for all lectures in this course

\title{\insertlecture}

\author{\lecturername}

\institute{\universityname}

\subject{Course: \coursename}

% Beamer version theme settings

\useoutertheme[height=0pt,width=2cm,right]{sidebar}
\usecolortheme{seagull,sidebartab}

%\useinnertheme{circles}
%\useinnertheme{rectangles}
\useinnertheme[shadow]{rounded}

\usefonttheme[only large]{structurebold}

\setbeamercolor{sidebar right}{bg=black!15}
\setbeamercolor{structure}{fg=red}
\setbeamercolor{author}{parent=structure}

\setbeamerfont{title}{series=\normalfont,size=\LARGE}
\setbeamerfont{title in sidebar}{series=\bfseries}
\setbeamerfont{author in sidebar}{series=\bfseries}
\setbeamerfont*{item}{series=}
\setbeamerfont{frametitle}{size=}
\setbeamerfont{block title}{size=\small}
\setbeamerfont{subtitle}{size=\normalsize,series=\normalfont}

\setbeamertemplate{navigation symbols}{}
\setbeamertemplate{bibliography item}[book]
\setbeamertemplate{sidebar right}
{
  {\usebeamerfont{title in sidebar}%
    \vskip1.5em%
    \hskip3pt%
    \usebeamercolor[fg]{title in sidebar}%
    \insertshorttitle[width=1.8cm,center,respectlinebreaks]\par%
    \vskip1.25em%
  }%
%  {%
%    \hskip3pt%
%    \usebeamercolor[fg]{author in sidebar}%
%    \usebeamerfont{author in sidebar}%
%    \insertshortauthor[width=2cm-2pt,center,respectlinebreaks]\par%
%    \vskip1.25em%
%  }%
  \hbox to2cm{\hss\insertlogo\hss}
  \vskip1.25em%
  \insertverticalnavigation{2cm}%
  \vfill
  \hbox to 2cm{\hfill\usebeamerfont{subsection in
      sidebar}\strut\usebeamercolor[fg]{subsection in
      sidebar}\insertshortlecture.\insertframenumber\hskip5pt}%
  \vskip3pt%
}%

\setbeamertemplate{title page}
{
  \vbox{}
  \vskip1em
  {\huge Lecture \insertshortlecture\par}
  {\usebeamercolor[fg]{title}\usebeamerfont{title}\inserttitle\par}%
  \ifx\insertsubtitle\@empty%
  \else%
    \vskip0.25em%
    {\usebeamerfont{subtitle}\usebeamercolor[fg]{subtitle}\insertsubtitle\par}%
  \fi%
  \vskip1em\par
  \emph{\coursename}\\  \insertdate\par
  \vskip0pt plus1filll
  \leftskip=0pt plus1fill\insertauthor\par
  \insertinstitute\vskip1em
}

\logo{\includegraphics[width=2cm]{UMBC.png}}



% Article version layout settings

\mode<article>

\makeatletter
\def\@listI{\leftmargin\leftmargini
  \parsep 0pt
  \topsep 5\p@   \@plus3\p@ \@minus5\p@
  \itemsep0pt}
\let\@listi=\@listI


\setbeamertemplate{frametitle}{\paragraph*{\insertframetitle\
    \ \small\insertframesubtitle}\ \par
}
\setbeamertemplate{frame end}{%
  \marginpar{\scriptsize\hbox to 1cm{\sffamily%
      \hfill\strut\insertshortlecture.\insertframenumber}\hrule height .2pt}}
\setlength{\marginparwidth}{1cm}
\setlength{\marginparsep}{4.5cm}

\def\@maketitle{\makechapter}

\def\makechapter{
  \newpage
  \null
  \vskip 2em%
  {%
    \parindent=0pt
    \raggedright
    \sffamily
    \vskip8pt
    {\fontsize{36pt}{36pt}\selectfont Lecture \insertshortlecture \par\vskip2pt}
    {\fontsize{24pt}{28pt}\selectfont \color{blue!50!black} \insertlecture\par\vskip4pt}
    {\Large\selectfont \color{blue!50!black} \insertsubtitle\par}
    \vskip10pt

    \normalsize\selectfont  Print version of the lecture in \emph{\coursename} \par\vskip1em
    presented on \@date\par\vskip1em
    by \lecturername from \departmentname at \universityname
  }
  \par
  \vskip 1.5em%
}

\let\origstartsection=\@startsection
\def\@startsection#1#2#3#4#5#6{%
  \origstartsection{#1}{#2}{#3}{#4}{#5}{#6\normalfont\sffamily\color{blue!50!black}\selectfont}}

\makeatother

\mode
<all>


% Typesetting Listings

\usepackage{listings}
\lstset{language=Java}

\alt<presentation>
{\lstset{%
  basicstyle=\footnotesize\ttfamily,
  commentstyle=\slshape\color{green!50!black},
  keywordstyle=\bfseries\color{blue!50!black},
  identifierstyle=\color{blue},
  stringstyle=\color{orange},
  escapechar=\#,
  emphstyle=\color{red}}
}
{
  \lstset{%
    basicstyle=\ttfamily,
    keywordstyle=\bfseries,
    commentstyle=\itshape,
    escapechar=\#,
    emphstyle=\bfseries\color{red}
  }
}




% Common theorem-like environments

\theoremstyle{definition}
\newtheorem{exercise}[theorem]{\translate{Exercise}}

% New useful definitions:

\newbox\mytempbox
\newdimen\mytempdimen

\newcommand\includegraphicscopyright[3][]{%
  \leavevmode\vbox{\vskip3pt\raggedright\setbox\mytempbox=\hbox{\includegraphics[#1]{#2}}%
    \mytempdimen=\wd\mytempbox\box\mytempbox\par\vskip1pt%
    \fontsize{3}{3.5}\selectfont{\color{black!25}{\vbox{\hsize=\mytempdimen#3}}}\vskip3pt%
}}

\newenvironment{colortabular}[1]{\medskip\rowcolors[]{1}{blue!20}{blue!10}\tabular{#1}\rowcolor{blue!40}}{\endtabular\medskip}

\def\equad{\leavevmode\hbox{}\quad}

\newenvironment{greencolortabular}[1]
{\medskip\rowcolors[]{1}{green!50!black!20}{green!50!black!10}%
  \tabular{#1}\rowcolor{green!50!black!40}}%
{\endtabular\medskip}


