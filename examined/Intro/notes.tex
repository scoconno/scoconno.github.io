\documentclass[report,oneside]{memoir}
\usepackage{unicode-math}
\usepackage{graphicx,url}
\usepackage{rotating} 
\usepackage{memoir-article-styles} 
\usepackage{fontspec} 
\usepackage{xunicode} 
\usepackage[polutonikogreek,english]{babel}
\newcommand{\greek}[1]{{\selectlanguage{polutonikogreek}#1}}
\def\myaffiliation{New Jersey City University}
\def\myauthor{Scott O'Connor}
\def\myemail{\small{\texttt{\href{mailto:soconnor@njcu.edu}{soconnor@njcu.edu}}}}
\def\mytitle{Critical Thinking Handout}
\def\mykeywords{2015, 2016}
\usepackage[usenames,dvipsnames]{color}                     
\usepackage[xetex, 
	colorlinks=true,
	urlcolor=BlueViolet, % external links
       citecolor=BlueViolet, % citations
        filecolor=BlueViolet, % local files
	plainpages=false,
  	pdfpagelabels,
  	bookmarksnumbered,
  	pdftitle={\mytitle},
 	pdfauthor={\myauthor},
  	pdfkeywords={\mykeywords}
  	]{hyperref}   


\usepackage[numbers, curly]{natbib}

\usepackage{setspace}



\begin{document}
\setromanfont[Mapping=tex-text]{Minion Pro} 
\setsansfont[Mapping=tex-text]{Myriad Pro}  
\setmonofont[Mapping=tex-text,Scale=MatchLowercase]{Minion Pro} 
\setkeys{Gin}{width=1\textwidth} 
\chapterstyle{article-2}  
\pagestyle{kjh}

\title{\bigskip \bigskip {{\mytitle}} }

\author{\myauthor \\
 \emph{\myaffiliation \\ 
  \myemail }}
%\thanks{OMITTED}
%\published{\footnotesize Draft. Please do not cite without permission.}
\maketitle

%\section{Work Sheet}\label{work-sheet}

%If you haven't read Ch.1.3 of the textbook, stop. Please read that first before turning to this handout, which is a summary\slash supplement to the textbook. You will find some \textbf{self-assessment} exercises below. The answer key is \href{/Teaching/Examined/CT/Answers}{here}\footnote{\href{/Teaching/Examined/CT/Answers}{\slash Teaching\slash Examined\slash CT\slash Answers}}. \textbf{You do not submit answers to self-assessment exercises to me.}

\section{Introduction}
\label{introduction}

Don't be fooled by every day uses of the words `philosophy', `philosophical', and `philosopher'. Being philosophical is not the same as merely stating your belief about some deep issue. Philosophers have little patience for those who just state their views on some issue. They care about \textbf{the reasons} for accepting such views. These reasons come in the form of \textbf{arguments} which prove those views true or false. So, don't expect us philosophers merely to respect your opinion. We will respect your good arguments and will expect you to provide them. Philosophy, then, is concerned with critical thinking, which we will define as follow:

\begin{description}

\item[\textbf{Critical Thinking:} ]

The systematic evaluation and formulation of beliefs and statements by rational standards.
\end{description}
Critical thinking is systematic because it involves distinct methods or procedures. We'll be learning some of those methods in this course. It concerns evaluation because it concerns itself with assessing existing beliefs, either your own or others. And it concerns formulation because it helps you formulate new beliefs. Finally, critical thinking assesses beliefs by a particular standard, namely, by how well that belief is supported by reasons. This contrasts to evaluating beliefs based on, for instance, how beautiful or psychologically comforting they might be. 

Philosophy, then, is concerned with arguments. You will encounter arguments in two ways. First, you will be learning how to assess the arguments of others. Such arguments are intended to convince you to believe something. If someone argues that, say, corporation tax should be raised, they want you, the listener, to agree with them and form the same belief. You need some tool to decide whether to do that; that's how learning to assess arguments helps. It gives you the ability to determine whether the reason someone has given for raising corporation tax supports that conclusion, and so helps you determine whether you should accept their proposal. 

Second, you will be learning how to construct your own good arguments. If you want me or anyone else to accept any of your beliefs, say about taxation or religion or music, you need to provide me reasons for thinking your belief is true. You do that by providing me good arguments for that belief. 

Since philosophy is concerned with good arguments, you need to learn what arguments are and when an argument is a good one. The branch of philosophy that concerns itself with these questions is called logic. You could dedicate your life to the study of logic, but, for our purposes, we only need a brief introduction to it. You will be introduced to some jargon that you need to learn. It will help you characterize arguments as well as evaluate them. This handout proceeds in four rough parts: 

\begin{enumerate}
\item The elements of an argument are introduced.

\item Arguments are distinguished from non-arguments.

\item The difference between deductive and inductive arguments is introduced.

\item The criteria for distinguishing good from bad arguments (first deductive, then inductive) are outlined.

\end{enumerate}
I encourage you to read a section or two. Try the self-assessment exercises and check the answer key to make sure you understood the material. 

\section{Arguments}
\label{arguments}

What is an argument? (You might find \href{http://www.wi-phi.com/video/intro-critical-thinking}{this video helpful}\footnote{\href{http://www.wi-phi.com/video/intro-critical-thinking}{http:/\slash www.wi-phi.com\slash video\slash intro-critical-thinking}}.)

\begin{description}

\item[\textbf{Argument:} ]

A group of statements in which some of them (\textbf{the premises}) are intended to support another of them (\textbf{the conclusion}).
\end{description}
Arguments are embedded in the conversations we have with one another, the various things we read, and the various papers we write. Philosophers try to extract arguments from all this noise to see exactly what is being claimed and what evidence is being provided. They will regularly present the arguments they hear or read about in bullet form: 

\begin{itemize}
\item P1: All humans are mortal.

\item P2: Socrates is human.

\item C: Socrates is mortal.

\end{itemize}
P1 and P2 are the premises of the argument. C is the conclusion. Not every argument has two premises. In principle, there are arguments with indefinitely many. For our purposes, we'll be focusing on relatively short arguments. But before we can learn to extract arguments we need to look closer at their elements. 

\section{Statements}
\label{statements}

Your beliefs and the reasons that support them are expressed by statements. Think of the conclusion of an argument as the statement which expresses the belief you or someone else holds. Think of the premises as statements which express the reasons that support those beliefs. If we are to properly state our beliefs and identify the reasons for them, we need then to understand what statements are. 

\begin{description}

\item[\textbf{Statements:}]

An assertion that something is or is not the case. They are either true or false.
\end{description}
\begin{itemize}
\item A triangle has three sides.

\item I see blue spots before my eyes.

\item 7+5=12

\item There are black holes in space.

\end{itemize}
Not all sentences express statements, i.e., not all sentences express claims that are true or false. The following are \textbf{not} statements:

\begin{itemize}
\item Does a triangle have three sides?

\item Stop telling lies.

\item Hey, dude.

\item Yay for Penny!

\end{itemize}
Since reasons must be expressed by statements, a non-statement can never serve as a reason for a belief, a premise in an argument. Consider the following: 

\begin{enumerate}
\item Yay for Penny!

\item Penny will make the best President.

\end{enumerate}
Suppose that 2 expresses someone's sincerest belief. You know now that 1, since it is not a statement, could never be used to support that belief. So, you know that anyone who tries to prove 2 by claiming 1 is offering no real argument at all. 

\subsection{Exercise 1:}
\label{exercise1:}
Which of the following sentences are statements?

\begin{itemize}
\item You are not a liar.

\item Is God all-powerful?

\item Turn that music off.

\item You should never hit your mother with a shovel.

\item The best explanation for his behavior is that he was in a trance.

\item Rap music is better than punk rock.

\item Great balls of fire!

\end{itemize}
\subsection{Exercise 2:}
\label{exercise2:}

List 4 sentences that are statements. List 4 sentences that are not statements. 

\section{Identifying arguments}
\label{identifyingarguments}
You know now that an argument is a group of statements in which some of them (\textbf{the premises}) are intended to support another of them (\textbf{the conclusion}). One important skill you need to learn as a philosopher is the ability to identify which statement in an argument is the conclusion and which are the premises. Try the exercises below before checking the answer key. 

\subsection{Exercise 3:}
\label{exercise3:}

Identify and underline the conclusions and premises in the following three arguments. 

\begin{quote}

If Arme is in town, then she's staying at the Barbary Hotel. She's in town. Therefore, she's staying at the Barbary Hotel

Because banning assault rifles violates a constitutional right, the U.S. government should not ban assault rifles.

Listen, any movie with clowns in it cannot be a good movie. Last night's movie had at least a dozen clowns in it. Consequently it was awful.
\end{quote}
\section{Non-Arguments}
\label{non-arguments}

Not every group of statements contains an argument. Another important skill you will need to acquire is the ability to figure out when and when a passage contains a genuine argument. Consider this example: 

\begin{quote}

The cost of the new XJ fighter plane is \$650 million. The cost of three AR21 fighter bombers is \$1.2 billion. The administration intends to fund such projects.
\end{quote}
This passage contains three statements. But none of these statements is clearly a conclusion that is being argued for. In other words, this passage contains claims about the world, but it doesn't contain any reasons (premises) for accepting those claims. Consider now this second passage:

\begin{quote}

Alisha went to the bank to get a more recent bank statement of her checking account. The teller told her that the balance was \$1725. Alisha was stunned that it was so low. She called her brother to see if he had been playing one of his twisted pranks. He wasn't. Finally, she concluded that she had been a victim of bank fraud.
\end{quote}
You might read a passage like this in a novel or a newspaper article. Don't be fooled by the apparently argumentative tone. It does not contain any argument at all. To see this, notice that the author is not trying to tell us why we too should accept that Alisha was the victim of bank fraud. In other words, telling us that Alisha came to have a belief is not the same as providing us reasons for thinking her belief is true. 

\subsection{Exercise 4:}
\label{exercise4:}

Which of the following passages contain an argument? In the passages containing an argument, identify and underline the conclusions and premises. 

\begin{quote}

The GAO says that any weapon that costs more than \$50 million apiece will actually impair our military readiness. The cost of the new XJ fighter plane is \$650 million dollars. The cost of three AR21 fighter bombers is \$1.2 billion. We should never impair our readiness. Therefore, the administration should cancel both these projects.

Attributing alcohol abuse by children too young to buy a drink to lack of parental discipline, intense pressure to succeed, and affluence incorrectly draws attention to proximate causes while ignoring the ultimate cause: a culture that tolerates overt and covert marketing of alcohol tobacco and sex to these easily manipulated, voracious consumers. [Letter to the editor, New York Times]

I don't understand what is happening to this country. The citizens of this country are trying to destroy the beliefs of our forefathers with their liberal views. This country was founded on Christian beliefs. This has been and I believe still is the greatest country in the world. But the issue that we cannot have prayer in public places and on public property because there has to be separation of church and state is a farce. [Letter to the editor, Douglas County Sentinel]
\end{quote}

\section{Complex Passages}
\label{complexpassages}

It is often difficult to identify the core argument in a complex passage. You need to recognize which statements are parts of that argument, but also which statements are not parts of that argument. These other statements may serve some other role such as providing important background information, or explaining a premise, or explaining the consequences of accepting the conclusion, and so on. Try the following exercise before checking the answer key: 

\subsection{Exercise 5:}
\label{exercise5:}

In the following passages, identify those statements which are part of the argument. Identify those which are. Also identify the conclusions and premises. 

\begin{quote}

[1] You have already said that you love me and that you can't imagine spending the rest of your life without me. [2] Once, you even tried to propose to me. [3] And now you claim that you need time to think about whether we should be married. [4] Well, everything that you've told me regarding our relationship has been a lie. [5] In some of your letters to a friend you admitted that you were misleading me. [6] You've been telling everyone that we are just friends, not lovers. [7] And worst of all, you've been secretly dating someone else. [8] Why are you doing this? [9] It's all been a farce, and I'm outta here.

[1] A. L. Jones used flawed reasoning in his letter yesterday praising this newspaper's decision to publish announcements of same-sex unions. [2] Mr. Jones asserts that same-sex unions are a fact of life and therefore should be acknowledged by the news media as a legitimate variation on social partner ships. [3] But the news media are not in the business of endorsing or validating lifestyles. [4] They're supposed to report on lifestyles, not bless them. [5] In addition, by validating same-sex unions or any other lifestyle, the media abandon their objectivity and become political partisans, which would destroy whatever respect people have for news outlets. [6] All of this shows that the news media, including this newspaper, should never (explicitly or implicitly) endorse lifestyles by announcing those lifestyles to the world.
\end{quote}

\section{Evaluating Arguments}
\label{evaluatingarguments}

You can now distinguish statements from non-statements, identify which statements serve as premises and which as the conclusion, and extract the core argument from a piece of text. Our next job as philosophers is to decide whether an argument is a good one, i.e., do the reasons given by the premises prove that the claim stated in the conclusion is true? Good and bad arguments are distinguished as follows:
\begin{itemize}
\item A \textbf{good argument} has (1) solid logic (an appropriate connection between the premises and conclusion), and (2) true premises. 

\item A \textbf{bad argument} (1) fails to have solid logic (fails to have an appropriate connection between the premises and conclusion), or (2) has false premises, or (3) both. 
\end{itemize}
What makes the connection appropriate will vary by the type of argument under consideration. There are two main types of arguments, deductive and inductive. 

\section{Deductive Arguments}
\label{deductivearguments}

A \textbf{deductive argument} is an argument intended to give logically conclusive support to its conclusion. (You find the \href{http://www.wi-phi.com/video/deductive-arguments}{following video helpful}\footnote{\href{http://www.wi-phi.com/video/deductive-arguments}{http:/\slash www.wi-phi.com\slash video\slash deductive-arguments}}.)

If a deductive argument has good logic, then the argument is \textbf{valid}. An argument is valid when the following situation obtains: if the premises are true, then the conclusion must be true. See \href{http://www.wi-phi.com/video/validity}{here}\footnote{\href{http://www.wi-phi.com/video/validity}{http:/\slash www.wi-phi.com\slash video\slash validity}} and \href{http://www.wi-phi.com/video/truth-and-validity}{here}\footnote{\href{http://www.wi-phi.com/video/truth-and-validity}{http:/\slash www.wi-phi.com\slash video\slash truth-and-validity}} for more information on validity. 

Note that the premises do not need to be true for an argument to be valid. Assessing an argument for validity asks us to suppose\slash imagine\slash hypothesize that the premises are true (even if we know they are false), and then determine whether the conclusion must also be true given that supposition. Or, to put it differently, we want to know if there is any possible way the premises could be true and the conclusion false. If there is not, the argument is valid. If there is, then the argument is invalid. 

A deductive argument is \textbf{sound} when (a) the argument is valid, and (b) the premises are true. See \href{http://www.wi-phi.com/video/soundness}{here}\footnote{\href{http://www.wi-phi.com/video/soundness}{http:/\slash www.wi-phi.com\slash video\slash soundness}} for more information on soundness. Note that a deductive argument can be valid without being sound. However, it cannot be both sound and invalid. Examples:
%\noindent \textbf{Examples of valid arguments:}

1) 

\begin{itemize}
\item P1. It's wrong to take the life of an innocent person.

\item P2. Abortion takes the life of an innocent person.

\item C1. Therefore, abortion is wrong.

\end{itemize}

2)

\begin{itemize}
\item P1. All dogs are from Mars.

\item P2. Rex is a dog.

\item C1. Therefore, Rex is from Mars.

\end{itemize}
This second argument is valid, but it is not sound because the first premise is false. The first argument is trickier. It is valid, but there is much debate as to whether the two premises are true. Some deny P1. They claim that mistakingly taking the life of an innocent person is not immoral. Others deny that the fetus is a person and so claim that abortion does not involve taking the life of a person. The point is that determining validity requires looking at the structure of what's given. Determining soundness is a different matter. It often takes serious research and argumentation to determine whether the premises of an argument are true or false.

\subsection{Exercise 6:}
\label{exercise6:}

For each argument, indicate whether it is valid and sound. 

\begin{quote}

`Anyone who wins an Oscar is famous. Halle Berry won an Oscar. Hence, Halle Berry is famous.'

`Bill Clinton must be dishonest. After all, he's a politician and hardly any politicians are honest.'

`Tom Cruise the actor is famous. After all, every actor who has won an Oscar is famous, but he's never won one.'

`Claire must live on the same street as Laura, since she lives on the same street as Max and he and Laura live on the same street. And each of them lives on only one street.'
\end{quote}

\subsection{Exercise 7:}
\label{exercise7:}

Construct two arguments that are (a) sound; (b) valid, but not sound; (c) invalid.

\section{Inductive Arguments}
\label{inductivearguments}

\textbf{Inductive Arguments} are supposed to give probable support to their conclusions, i.e., the premises are supposed to show that the conclusion is probably true. This falls short of showing that the conclusion is decisively true. An inductive argument is \textbf{strong} if its premises show that the conclusion is probably true. An inductive argument is \textbf{weak} if it fails to be strong.


1)

\begin{itemize}
\item P1. 85\% of the students at NJCU are Democrats.

\item P2. Mickey Mouse is a student at NJCU

\item C1. Therefore, Mickey Mouse is probably a Democrat.

\end{itemize}

2)

\begin{itemize}
\item P1. 10\% of the students at NJCU smoke marijuana regularly.

\item P2. Sally is a student at NJCU.

\item C1. Therefore, Sally probably smokes marijuana regularly.

\end{itemize}
The first argument is strong. If the premises are true, then the conclusion is more likely to be true than not. The second argument is weak. Just because a small portion of the student body smoke marijuana, it does not follow that Sally is more likely than not to be a smoker.

\section{Exercise 8:}
\label{exercise8:}

Construct 2 inductive arguments that are (a) strong, and (b) weak. 
\end{document}