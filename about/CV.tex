\documentclass[letterpaper, 11pt]{moderncv}        % possible options include font size ('10pt', '11pt' and '12pt'), paper size ('a4paper', 'letterpaper', 'a5paper', 'legalpaper', 'executivepaper' and 'landscape') and font family ('sans' and 'roman')

\usepackage[utf8x]{inputenx}
\renewcommand{\rmdefault}{ppl}




% moderncv themes
\moderncvstyle{classic}                            % style options are 'casual' (default), 'classic', 'oldstyle' and 'banking'
\moderncvcolor{red}                                % color options 'blue' (default), 'orange', 'green', 'red', 'purple', 'grey' and 'black'

%\definecolor{color1}{RGB}{32,92,170}

%\renewcommand{\familydefault}{\sfdefault}         % to set the default font; use '\sfdefault' for the default sans serif font, '\rmdefault' for the default roman one, or any tex font name
%\nopagenumbers{}                                  % uncomment to suppress automatic page numbering for CVs longer than one page

\usepackage[scale=0.75]{geometry}
%\setlength{\hintscolumnwidth}{1.8cm}                % if you want to change the width of the column with the dates
%\setlength{\makecvtitlenamewidth}{10cm}           % for the 'classic' style, if you want to force the width allocated to your name and avoid line breaks. be careful though, the length is normally calculated to avoid any overlap with your personal info; use this at your own typographical risks...



\firstname{\textsc{Scott}}
\familyname{\textsc{O`Connor}}
\title{Curriculum Vitae}
\address{New Jersey City University}{606 Karnoutsos Hall}{Jersey City, NJ 07305} 


\email{soconnor@njcu.edu}
\homepage{scottoconnor.org}


%\renewcommand*\sectionfont{\fontsize{11}{11}\selectfont}
%\renewcommand*\subsectionfont{\fontsize{12}{18}\selectfont}






%\makeatletter
%\patchcmd{\makelettertitle}% <cmd>
 % {\raggedright \@opening}% <search>
 % {\@opening}% <replace>
 % {}{}% <success><failure>
%\makeatother



%\renewcommand*{\bibliographyitemlabel}{\@biblabel{\arabic{enumiv}}}
%\makeatother
%\renewcommand*{\bibliographyitemlabel}{[\arabic{enumiv}]}% CONSIDER REPLACING THE ABOVE BY THIS

% bibliography with mutiple entries
%\usepackage{multibib}
%\newcites{book,misc}{{Books},{Others}}
%----------------------------------------------------------------------------------
%            content
%----------------------------------------------------------------------------------

\begin{document}


\makecvtitle
\section{Professional Experience}
%\cventry{2013--present}{Visiting Scholar}{}{Cornell University}{}{}{}
%\cventry{2014}{Instructor}{}{University of Maryland, Baltimore County}{}{}
\cventry{2014}{Assistant Professor}{}{New Jersey City University}{Philosophy Department}{}

\section{Education}
\cventry{2013}{Ph.D.}{}{Cornell University}{(MA 2010)}{
\begin{description}
 \item[] Dissertation:  `On the Principles of Nature: An Interpretation of Aristotle's \emph{Physics}\ I.5--8'
\item[] Committee:  Gail Fine (chair), Terence Irwin, Theodore Brennan
\end{description}}
%\cventry{2009}{Visiting Student}{}{University of Oxford}{}{}
%\cventry{2007}{Visiting Student}{}{City University of New York, Greek Institute}{}{}
\cventry{2006}{B.Phil.}{}{Christ Church, University of Oxford}{}{} %\begin{description}  
% \item[] Thesis Title: `The Sortal Dependency of Composition'
%\item[]  Advisor: Christopher Shields
%\item[] B.Phil. Subjects: Plato, Aristotle, Metaphysics \&  Epistemology
%\end{description}}



\cventry{2004}{B.A.}{}{Trinity College, University of Dublin}{(1\textsuperscript{st} Class)}{}
%\begin{description}
% \item[] Dissertation Title: `De-Fineing Essence: The Concept of Essence %and the Relation Between Essence and Existence: A Critique of Kit Fine'
%\item[]  Advisor: Vasilis Politis
%\end{description}}





\section{Areas of Specialization \hfill Areas of Competence}
\begin{cvcolumns}
  \cvcolumn{}{Ancient Philosophy\\Metaphysics}

\cvcolumn{}{\begin{itemize}
\item[]\hfill Philosophy of Mind
\item[]\hfill Early Modern Philosophy
\end{itemize}}
\end{cvcolumns}

\section{Publications}
\cventry{2015}{`The Subjects of Natural Generations in Aristotle's \emph{Physics}\ I.7'}{}{\emph{Apeiron: A Journal for Ancient Philosophy \& Science}}{vol. 48 (1), pp. 45--75}{}

\section{Work in Progress}
\cventry{}{`The Eleatic Challenge in Aristotle's \emph{Physics}\ I.8'}{(Under Review)}{}{}{}
\cventry{}{`Persistence in Aristotle's \emph{Physics}\ I--II'}{}{}{}{}


\cventry{}{`Aristotle's Response to Plato in \emph{Physics}\ I.9'}{}{}{}{}

\section{Presentations}

\cventry{2015}{`The Generation \& Destruction of Matter in Aristotle's \emph{Physics}\ I.9'}{}{Ancient Philosophy Society, Lexington, KY}{April}{}
\cventry{2015}{`The Eleatic Challenge in Aristotle's \emph{Physics}\ I.8'}{}{Ancient Philosophy Workshop, UT Austin}{February}{}
\cventry{2014}{`Aristotle's Criticism of Plato's First Principles in \emph{Physics}\ I.9'}{}{Society for Ancient Greek Philosophy, New York}{October}{}
\cventry{2014}{`Essence and Persistence in Aristotle'}{}{Marquette, Aristotle and Aristotelianism Conference}{June}{}
\cventry{2014}{Comments on Pavle Stojanovic's `Non-unified Objects As Proper Individuals in Stoicism' }{}{Pacific APA, San Diego}{April}{}

\cventry{2013}{`How Aristotle Explains Persistence'}{}{Cornell University}{December}{}

\cventry{2013}{`Explaining Persistence in Aristotle's \emph{Physics}\ I'}{}{Louisiana State University}{April}{}

\cventry{2013}{`The Subjects of Natural Generations in Aristotle's \emph{Physics}\ I'}{}{The College of St. Benedict \& Saint John's University}{January}{}

\cventry{2012}{`The Subjects of Natural Generations in Aristotle's \emph{Physics}\ I'}{}{Cornell Philosophy Department Workshop}{November}{}{}

\cventry{2011}{`The Persisting Simple in Aristotle's \emph{Physics}\ I.7'}{}{Marquette, Aristotle and Aristotelianism Conference}{June}{}


\cventry{2011}{Comments on Evan Keeling's `Unity in Aristotle's \emph{Metaphysics}\ H.6'}{}{Pacific APA}{San Diego, April}{}

\cventry{2011}{`The Persisting Simple in Aristotle's \emph{Physics}\ I.7'}{}{Cornell Philosophy Department Workshop}{May}{}

\cventry{2010}{`Aristotle, Gunk and Change'}{}{Cornell Philosophy Department Workshop}{October}{} 

\cventry{2010}{`Aristotle, \emph{Eudemian Ethics:} 1215a20-1215b30'}{}{Cornell University \emph{Eudemian Ethics} Workshop}{August}{}

\cventry{2010}{`The Possibility of Change in Aristotle's \emph{Physics} I.8'}{}{Cornell Philosophy Department Workshop}{April}{}
 
\cventry{2010}{Comments on Philip Corkum's `Aristotle on Reference and Generality'}{}{Pacific APA}{San Francisco, April}{}

\cventry{2010}{`The Persisting Simple in Aristotle's \emph{Physics}\ I.7'}{}{Princeton Ancient Philosophy Graduate Conference}{April}{}

\cventry{2009}{Comments on Ursula Coope's `Aristotle on the Infinite'}{}{Cornell University Ancient Philosophy Speaker Series}{September}{} 

\cventry{2007}{`Heraclitus' Essentialism'}{}{Cornell Philosophy Department Workshop}{October}{}

\cventry{2004}{`Essence and Existence'}{}{Dublin University Metaphysical Society}{Trinity College, Dublin. April}{}

\section{Selected Honours and Awards}
\cventry{2011}{Pedagogy Group Grant}{}{The Knight Institute, Cornell University}{}{}
\cventry{2009}{Sage Fellowship}{}{Cornell University}{}{}
\cventry{2009}{Humanities Dissertation Group Grant}{}{ Cornell University}{}{}
\cventry{2007}{Tuition Scholarship}{}{CUNY,  Summer Latin/Greek Institute}{}{}
\cventry{2006}{Sage Fellowship}{}{Cornell University}{}{}
\cventry{2005}{The Hugh Pilkington Scholarship}{}{ Christ Church, Oxford University}{}{}
\cventry{2005}{Prendergast Bequest}{}{Oxford University}{}{}
\cventry{2004}{Tuition Scholarship}{}{Oxford University, Department of Philosophy}{}{}
\cventry{2004}{AHRC Research Preparation Master's Scheme}{}{ U.K. National Award}{}{}
\cventry{2004}{Wray Prize (dissertation prize, shared)}{}{  Trinity College, Dublin}{}{}
\cventry{2003}{1st Class Book Award}{}{Trinity College, Dublin}{}{}
\cventry{2002}{John Henry Bernard Prize}{}{Trinity College, Dublin}{}{}


\section{Teaching}


\cventry{}{A Survey of Ancient Philosophy}{}{NJCU}{Fall 2014}{}

\cventry{}{Persons \& Problems}{}{NJCU}{Fall 2014, Spring 2015}{}

\cventry{}{Critical Thinking}{}{NJCU}{Spring 2015}{}

\cventry{}{Introduction to Philosophy}{}{UMBC}{Spring 2014}{}

\cventry{}{The Philosophy of Time}{}{Cornell}{Fall 2012}{}

\cventry{}{Introduction to Metaphysics}{}{Cornell}{Spring 2012, Fall 2011, Fall 2010}{}
\cventry{}{Aristotle's Natural Philosophy}{}{Cornell}{Spring 2011}{}



\section{Other Experience}

\cventry{2010--2013}{Graduate Resident Fellow}{}{Alice Cook House, Cornell University}{}{\normalsize{ A GRF lives in a Residential Hall and supports the House Professor-Dean, Assistant Dean, and House Fellows in creating a positive, vibrant and academically engaging residential community; serves as mentor and role model to undergraduate residents; and serves in a leadership role for academic enrichment by promoting a variety of programs and activities.}}

\section{Academic Service}
\cventry{2014}{A\&S Honors Committee}{}{NJCU}{}{}
\cventry{2014}{Chair, Philosophy Dept. Curriculum Committee}{}{NJCU}{}{}
\cventry{2007--2010}{Organizer}{}{Cornell Graduate Speaker Series}{}{}
\cventry{2009--2010}{Search Committee}{}{Cornell University}{}{}
\cventry{2006--2007}{Graduate Student Association Representative}{}{Cornell University}{}{}
\cventry{2005, 2004}{Referee}{}{Oxford Graduate Conference}{}{}



\section{Language Proficiency}
\cvitem{}{Classical Greek (advanced reading), Latin (basic), German (basic)}



%\section{\textsc{Graduate Study (* = audit)}}
%\cventry{Ancient}{Aristotle's Psychological Theories}{}{D. Charles, Oxford, TT 2009*}{}{}
%\cventry{Philosophy}{Aristotle on Being}{}{C. Shields, Oxford, TT 2009*}{}{}
%\cventry{}{Parts of the Soul}{}{U. Coope \& T. Johannson, Oxford, MT 2009*}{}{}
%\cventry{}{Ancient Logic}{}{P. Crivelli, B. Morison, Oxford, MT \& TT 2009*}{}{}
%\cventry{}{Plato's \emph{Meno} \& \emph{Theaetetus}}{}{G. Fine, Cornell, Fall 2008*}{}{}
%\cventry{}{Aristotle's \emph{Posterior Analytics}}{}{T. Brennan, Cornell, Spring 2008*}{}{}
%\cventry{}{Aristotle's \emph{De Anima}}{}{C. Shields, Cornell, Fall 2007}{}{}
%\cventry{}{Medieval Philosophy}{}{S. MacDonald, Cornell, Fall 2007}{}{}
%\cventry{}{The Pre-Socratics}{}{G. Betegh, Cornell, Spring 2007} {}{}
%\cventry{}{Aristotle and Stoicism}{}{T. Irwin, Cornell, Fall 2006}{}{}
%\cventry{}{Plato}{}{G. Fine, Cornell, Fall 2006}{}{}
%\cventry{}{Substance}{}{C. Shields, Oxford, HT 2006*}{}{}
%\cventry{}{Aristotle}{B.Phil supervision}{B. Morison, Oxford, TT 2005}{}{}
%\cventry{}{Plato}{B.Phil supervision}{M. Frede, Oxford, MT 2004}{}{}
%\cventry{Metaphysics}{Properties}{}{T. Sider, Cornell, Fall 2011*}{}{}
%\cventry{}{Verticality}{}{K. Bennett, Cornell, Spring 2010*}{}{}
%\cventry{}{Topics in Metaphysics}{}{ C. Dorr, J. Hawthorne, Oxford, TT 2009*}{}{}
%\cventry{}{Natural Kinds}{}{R. Boyd, Cornell, Spring 2007}{}{}
%\cventry{}{Metaphysics}{B.Phil supervision}{T. Williamson,  Oxford, MT 2005}{}{}
%\cventry{}{Epistemology \& Metaphysics}{B.Phil seminar}{ G. Pereya, Oxford, MT 2005}{}{}
%\cventry{Other}{Writing 7100}{}{ D. Faulkner, Cornell, Fall 2009}{}{}
%\cventry{}{Virtue Ethics}{Guided Study}{T. Brennan, Cornell, Spring 2008}{}{}
%\cventry{}{  Kant}{Guided Study}{D. Pereboom, Cornell, Spring 2008}{}{}
%\cventry{}{Concepts}{}{A. McGonigal, Cornell, Spring 2007}{}{}
%\cventry{}{Deductive Logic}{}{H. Hodes, Cornell, Fall 2006}{}{}
%\cventry{}{Philosophy of Mind}{B.Phil seminar}{E. Fricker, Oxford, TT 2005*}{}{}
%\cventry{}{   B.Phil Proseminar}{}{B. Child, J. Broome, Oxford, HT 2005}{}{}
%\cventry{}{B.Phil Proseminar}{}{B. Child, I. Rumfitt, Oxford, MT 2004}{}{}
%\cventry{Language}{Intensive Summer Upper-Level Greek}{}{Cornell, Summer 2010}{}{}
%\cventry{Courses}{Herodotus}{}{M. Fontaine, Cornell, Fall 2008}{}{}
%\cventry{}{Introduction to Latin}{}{Cornell, Summer 2008}{}{}
%\cventry{}{Plato's \emph{Apology}}{}{H. Pelliccia, Cornell, Fall 2007*}{}{}
%\cventry{}{Intensive Summer Introduction to Greek}{}{LGI,  CUNY, Summer 2007}{}{}
%\cventry{}{Greek Texts}{11 semesters}{C. Shields, T. Brennan, C. Brittain, %Cornell*}{}{}





\section{References}
\cventry{}{Gail Fine}{}{Cornell}{gjf1@cornell.edu}{}
\cventry{}{Terence Irwin}{}{Oxford}{terry.irwin@philosophy.ox.ac.uk}{}
\cventry{}{Theodore Brennan}{}{Cornell}{tad.brennan@cornell.edu}{}
\cventry{}{Christopher Shields}{}{Oxford}{christopher.shields@lmh.ox.ac.uk}{}
\cventry{}{Andrew Chignell}{}{Cornell}{chignell@cornell.edu}{}


%\section{\textsc{Dissertation Abstract}}


%\cvitem{}{Interpreters have assumed that Aristotle in \emph{Physics}\ I makes substantive claims about what the identity through time of persisting beings consists in, e.g., about what makes it the case that the Socrates who defends himself in the \emph{Apology} is identical to the Socrates who drinks hemlock in the \emph{Phaedo}.  I think Aristotle shows no such interest in this metaphysical question, but rather addresses a different set of questions about persistence that have largely gone unnoticed. According to my view, Aristotle in \emph{Physics}\ I is interested in the scientific question of how, causally, natural beings are not destroyed, and so survive, as they are being changed. Once we recognize that Aristotle's focus on persistence is a causal one and not a metaphysical one, a number of problems that confront Aristotle interpreters dissolve. Most importantly, we can make sense of why Aristotle can claim that (i) matter persists when substances are generated from it, without, at the same time, attributing to him the claim that  (ii)  there are informative, non-circular diachronic criteria of identity for matter; criteria that are difficult to identify given Aristotle's understanding of matter.}

\end{document}

% Publications from a BibTeX file without multibib
%  for numerical labels: \renewcommand{\bibliographyitemlabel}{\@biblabel{\arabic{enumiv}}}% CONSIDER MERGING WITH PREAMBLE PART
%  to redefine the heading string ("Publications"): \renewcommand{\refname}{Articles}
\nocite{*}
\bibliographystyle{plain}
\bibliography{publications}                        % 'publications' is the name of a BibTeX file

% Publications from a BibTeX file using the multibib package
%\section{Publications}
%\nocitebook{book1,book2}
%\bibliographystylebook{plain}
%\bibliographybook{publications}                   % 'publications' is the name of a BibTeX file
%\nocitemisc{misc1,misc2,misc3}
%\bibliographystylemisc{plain}
%\bibliographymisc{publications}                   % 'publications' is the name of a BibTeX file

%\section{Interests}
%\cvitem{hobby 1}{Description}
%\cvitem{hobby 2}{Description}
%\cvitem{hobby 3}{Description}

%\section{Extra 1}
%\cvlistitem{Item 1}
%\cvlistitem{Item 2}
%\cvlistitem{Item 3. This item is particularly long and therefore normally spans over several lines. Did you notice the indentation when the line wraps?}

%\section{Extra 2}
%\cvlistdoubleitem{Item 1}{Item 4}
%\cvlistdoubleitem{Item 2}{Item 5\cite{book1}}
%\cvlistdoubleitem{Item 3}{Item 6. Like item 3 in the single column list before, this item is particularly long to wrap over several lines.}



%\section{References}
%\begin{cvcolumns}
  %\cvcolumn{Category 1}{\begin{itemize}\item Person 1\item %Person 2\item Person 3\end{itemize}}
  %\cvcolumn{Category 2}{Amongst others:\begin{itemize}\item %Person 1, and\item Person 2\end{itemize}(more upon request)}
  %\cvcolumn[0.5]{All the rest \& some more}{\textit{That} %person, and \textbf{those} also (all available upon request).}
%\end{cvcolumns}

%\cventry{year--year}{Degree}{Institution}{City}{\textit{Grade}}{Description}

%\section{Master thesis}
%\cvitem{title}{\emph{Title}}
%\cvitem{supervisors}{Supervisors}
%\cvitem{description}{Short thesis abstract}

%\section{Experience}
%\subsection{Vocational}
%\cventry{year--year}{Job title}{Employer}{City}{}{General %description no longer than 1--2 lines.\newline{}%
%Detailed achievements:%
%\begin{itemize}%
%\item Achievement 1;
%\item Achievement 2, with sub-achievements:
  %begin{itemize}%
  %\item Sub-achievement (a);
 % \item Sub-achievement (b), with sub-sub-achievements %(don't do this!);
    %\begin{itemize}
    %\item Sub-sub-achievement i;
    %\item Sub-sub-achievement ii;
    %\item Sub-sub-achievement iii;
   % \end{itemize}
  %\item Sub-achievement (c);
 % \end{itemize}
%\item Achievement 3.
%\end{itemize}}
%\cventry{year--year}{Job title}{Employer}{City}{}{Description line %1\newline{}Description line 2}
%\subsection{Miscellaneous}
%\cventry{year--year}{Job title}{Employer}{City}{}{Description}

%\section{\textsc{Languages}}
%\cvitemwithcomment{Language 1}{Skill level}{Comment}
%\cvitemwithcomment{Language 2}{Skill level}{Comment}
%\cvitemwithcomment{Language 3}{Skill level}{Comment}

%\section{Computer skills}
%\cvdoubleitem{category 1}{XXX, YYY, ZZZ}{category 4}{XXX, YYY, ZZZ}
%\cvdoubleitem{category 2}{XXX, YYY, ZZZ}{category 5}{XXX, YYY, ZZZ}
%\cvdoubleitem{category 3}{XXX, YYY, ZZZ}{category 6}{XXX, YYY, ZZZ}

