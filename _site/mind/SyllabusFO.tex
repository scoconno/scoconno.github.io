\documentclass[article,oneside]{memoir}
\usepackage{long table}
%%% custom style file with standard settings for xelatex and biblatex. Note that when [minion] is present, we assume you have minion pro installed for use with pdflatex.
%\usepackage[minion]{org-preamble-pdflatex} 

%%% alternatively, use xelatex instead
\usepackage{org-preamble-xelatex} 



\def\myauthor{Author}
\def\mytitle{Title}
\def\mycopyright{\myauthor}
\def\mykeywords{}
\def\mybibliostyle{plain}
\def\mybibliocommand{}
\def\mysubtitle{}
\def\myaffiliation{NJCU}
\def\myaddress{Phil 1}
\def\myemail{soconnor@njcu.edu}
\def\myweb{\href{http://scottoconnor.org/mind}{http://scottoconnor.org/mind}}
\def\myphone{}
\def\myversion{}
\def\myrevision{}
\def\myaffiliation{NJCU}
\def\myauthor{Dr. Scott O'Connor}
\def\mykeywords{}
\def\mysubtitle{Syllabus}
\def\mytitle{{\normalsize \myweb \newline} \HUGE Self: Mind}


\begin{document}

%%% If using xelatex and not pdflatex
%%% xelatex font choices
\defaultfontfeatures{}
\defaultfontfeatures{Scale=MatchLowercase}    
% You will need to buy these fonts, change the names to fonts you own, or comment out if not using xelatex.      
\setromanfont[Mapping=tex-text]{Minion Pro} 
\setsansfont[Mapping=tex-text]{Myriad Pro} 
\setmonofont[Mapping=tex-text,Scale=0.8]{Georgia} 

%% blank label items; hanging bibs for text
%% Custom hanging indent for vita items
\def\ind{\hangindent=1 true cm\hangafter=1 \noindent}
\def\labelitemi{$\cdot$}
%\renewcommand{\labelitemii}{~}

%% RCS info string for version tracking
\chapterstyle{article-3}  % alternative styles are defined in latex-custom-kjh/needs-memoir/
%\pagestyle{kjh}

\title{\LARGE \mytitle}     
\author{\Large\myauthor \newline \footnotesize\texttt{\noindent Office hours: \href{http://scottoconnor.org/contact/office/}{http://scottoconnor.org/contact/office/}}}
\date{1/16/2018--5/14/2018}

\published{\textbf{Phil 236 (2316), 3 credits, Spring 2018, Online}}

\maketitle

%\thispagestyle{kjhgit}

% Copyright Page
%\textcopyright{} \mycopyright


%
% Main Content
%


\section{Disclaimer}
 This syllabus is subject to change at the discretion of the faculty. Students will be notified of such changes ahead of time via Blackboard. 


\section{Copyright}
The materials used in this class, including, but not limited to, lectures, exams, quizzes, and homework assignments are copyright protected works.  Any unauthorized copying of the class materials or recording of lectures is a violation of federal law and may result in disciplinary actions being taken against the student.  Additionally, the sharing of class materials without the specific, express approval of the instructor may be a violation of the University's Student Honor Code and an act of academic dishonesty, which could result in further disciplinary action.  This includes, among other things, uploading class materials to websites for the purpose of sharing those materials with other current or future students. 

\section{Catalog Description}

This course introduces students to the philosophical study of mind. Students will learn how their unique psychologies play a role in distinguishing themselves from others, as well as consider how their psychologies are shaped by their environment and biology. The course also focuses on the relationship between mind and body.

\section{Course Description}

The primary goal of the course is to make students aware of their mental lives and the significance their mental lives have on questions about personal identity and naturalism. It will serve majors and minors in Philosophy and Religion as well as the wider student population. It introduces the former to the philosophical study of mind, which will serve as a spring-board to higher-level courses in mind, the philosophy of psychology, and related issues in epistemology and metaphysics. For the latter, the course will offer the opportunity to analyze the ways in which their unique psychologies shape who and what they are, and, at the same time, offer the opportunity to analyze the ways in which their psychologies are affected by their biology and environment. 

The course also promises to teach students to be tolerant in three distinct ways. First, questions about whether the mind can be explained by the natural sciences is key to deciding the scope of natural science, and, by extension, the nature of the reality we inhabit. The study of mind thus serves as a useful bulwark against an over-confidence in current scientific paradigms. Second, questions about whether, and to what extent, animals have mental states has been taken as key to deciding what if any moral status they enjoy. The study of mind thus serves as a bulwark against speciesism. Third, questions about the nature of psychological traits are key to deciding whether, and how, our environment determines how we act. Thus the study of mind makes students aware of how the actions of their fellow citizens are affected by their environment, not merely their intrinsic features. 

\section{Learning Objectives}

Upon completing this course, students will be able to  (i) identify their own psychological characteristics, (ii) interpret readings ranging from newspaper articles to literary and philosophical texts, as well as artworks and movies, (iii) define fundamental philosophical ideas and concepts, (iv) compare different theories of the mind, (v) apply these theories to settles questions of survival and mentality,  and  (vi) manage their studies in a responsible and timely manner. 


\section{General Education Information} 
Successfully completing this course satisfies one Tier 2 Language, Literary, and Cultural Studies requirement. It teaches the following two University-wide Learning Goals: (1) Critical Thinking and Problem Solving, (2) Written Communication. For further information about the General Education Program see \href{http://njcu.edu/department/general-education}{http://njcu.edu/department/general-education}.


\section{Required Textbooks}
Available in the campus book store and online retailers.


\begin{itemize}
\item \href{https://www.amazon.com/Philosophy-Mind-Jaegwon-Kim/dp/0813344581/ref=sr_1_1?ie=UTF8&qid=1484684648&sr=8-1&keywords=kim+philosophy+of+mind}{Kim, Jaegwon. \emph{The Philosophy of Mind.} 3rd ed. Boulder, CO: Westview, 2010} 
\end{itemize}


\section{Course Website}
There is both a Blackboard site and website for this course (link on first page). Clicking the first link on the left panel within the Blackboard site will bring you to the course website. All assignments will be submitted through Blackboard. Readings, notes, etc. will be posted on the course website. Note that Blackboard difficulties are rare and automatically reported to instructors. Under no circumstance will a student's report of a Blackboard difficulty be reason for an extension. It is your responsibility to contact Blackboard support for help.


\section{Requirements}

\begin{itemize}
\item \textit{Workload:} Expect to spend an average of 8 hours per week completing the readings and assignments. NJCU abides by the Federal and State definitions of a credit hour and adopts a policy consistent with the Carnegie Unit. A three-credit class represents 112.5 hours total of work. See \href{http://scottoconnor.org/resources/Credit.pdf}{here} for more details.

%\item \textit{Attendance:} Roll call will be taken. 0.5 point will be awarded per class up to a maximum of 10 points. Points will not be awarded during weeks 1 \& 2. 

\item \textit{Reading quizzes} administered through Blackboard. 7  will be assigned. You must complete 5. If you complete more than 5, the lowest grades will be dropped. 

\item \textit{Essays (500--750 words)} submitted through Blackboard. 6 will be assigned. You must complete 4. If you complete more than 4, the lowest grades will be dropped. 

\item \textit{Signature Assignment (1000-1250 words)} submitted through Blackboard. 

\item \textit{Course evaluations} completed online. 3 points extra credit for successful completion and screenshot of completed page sent through Blackboard. 
  
\item \textit{Grade distribution:} Reading quizzes---10 points each (50 total);  Essays---10 points each (40 total); Signature Assignment--10 points


\item \textit{Grade Breakdown:}

 \begin{tabular}{ | l | l | p{2cm} | l | l | }
    \hline 
96--100 & A  & &  77--79 &  C+ \\  
90--95 & A- & &  73--76 & C \\
87-89 & B+ &  &  70--72 & C- \\ 
83--86 & B  & &  60--69 & D\\
80--82 & B - & & 0--59 & F\\ \hline
    \end{tabular}


\end{itemize}


\section{Policies}

\begin{itemize}

\item \textbf{Student Responsibility:} This syllabus outlines the required text, assignments, requirements, and policies for this course. By taking this course, you agree to read this syllabus and be bound by those requirements and policies. 

 \item \textit{Academic Integrity:} All the work you turn in (including papers, drafts, and discussion board posts) must be written by you specifically for this course. It must originate with you in form and content with all contributory sources fully and specifically acknowledged. Being a student at NJCU requires you to follow \href{http://scottoconnor.org/resources/Plagiarism.pdf}{NJCU's Academic Integrity Policy.} Penalties for violations are as follows: 1st infraction will result in a 0 for the assignment.  2nd infraction will result in a 0 for the entire course \& application for permanent record on student's transcript. (Repeated violations can lead to expulsion from NJCU). 


%\item \textit{Attendance:} You are considered absent if you are (i) not present during roll call, (ii) leave early, (iii) leave without permission, or (iv) leave for an extended period of time. No excuses. No exceptions.



\item \textit{Communication:} To comply with Federal Privacy Laws (FERPA) and NJCU policies, all communication will be through Blackboard and/or official NJCU e-mail. Check Blackboard daily. For further information see \href{http://scottoconnor.org/contact/}{http://scottoconnor.org/contact/}.

%\item \textit{Conduct:} Distracting and disrespectful behaviors, including but not limited to eating, leaving your seat, talking out of turn, and aggression are prohibited. Penalties include, but are not limited to, a loss of attendance points for the day of violation. Repeat offenders will be reported to the Dean of Students. 

%\item \textit{Electronic devices:} Use of electronic device, including, but not limited, to smartphones, dictaphones, tablets, and laptops, is prohibited. Recording a lecture is in violation of Copyright. Penalties include, but are not limited to, a loss of attendance points for the day of violation. Repeat offenders will be reported to the Dean of Students.


\item \textit{Format for Written Work:} Submit work to Blackboard as either a pdf, rtf, or doc file. Blackboard will not allow any other format. All work must be typed and neatly presented. 


\item \textit{General Education Program Assessment:} General Education courses participate in programmatic assessment of the six University-wide student learning goals. They include instruction in, and assessment of, at least two of these learning goals. Signature assignments, which may include document, picture, sound, or video files, are uploaded to a secure server for anonymous distribution to the NJCU assessment team, which scores them using approved program rubrics. While instructors also grade their own students’ signature assignments, which count toward the course grade, assessment team results are aggregated to provide information about the Gen Ed program as a whole. Your name will not be included in any programmatic assessment data.

\item \textit{Grading:} Grades will be available within 1--2 weeks of an assignment being submitted. See: \href{http://scottoconnor.org/resources/grading}{http://scottoconnor.org/resources/grading} for further information.


\item \textit{Late work \& Make-up Policy:} See the assignment schedule below. No make-ups or late work accepted under any circumstances. No exceptions under any imaginable circumstances.

\item \textit{Statement for students with disabilities:} If you are a student
with a disability and wish to receive consideration for reasonable
accommodations, please register with the Office of Specialized Services
and Supplemental Instruction (OSS/SI). To begin this process, complete
the registration form available on the OSS/SI website at
\href{http://www.njcu.edu/oss}{http://www.njcu.edu/oss}
(listed under Student Resources-Forms). Contact OSS/SI at 201-200-2091
or visit the office in Karnoutsos Hall, Room 102 for additional
information.

\item \textit{Turnitin:} Students agree that by taking this course all assignments are subject to submission for textual similarity review to Turnitin.com. Assignments submitted to Turnitin.com will be included as source documents in Turnitin.com's restricted access database solely for the purpose of detecting plagiarism in such documents.  The terms that apply to the University’s use of the Turnitin.com service are described on the Turnitin.com web site.  For further information about Turnitin, please visit: http://www.turnitin.com 

\item \textit{SafeAssign:} Students agree that by taking this course all assignments are subject to submission for textual similarity review through Blackboard SafeAssign. Assignments submitted to SafeAssign will be included as source documents in SafeAssign's restricted access database solely for the purpose of detecting plagiarism in such documents.  


\end{itemize}



\section{Weekly Course Schedule}
Readings marked with a `**' can be found on the course website. All other listed readings can be found in the required textbook. Changes to the syllabus will be announced through Blackboard and \emph{via} your NJCU email address.  All assignments must be submitted through Blackboard by Monday, 11:59pm of the following week, e.g., quiz 1 is due on 1/29/18 at 11:159pm. No late work accepted. No exceptions.   \newline

\newpage

\begin{center}
\begin{longtable}{p{4.5cm}p{2cm}p{6cm}}
 
  \caption{Course Schedule} \\
  \toprule
  \textbf{Week} &\textbf{Assignments} & \textbf{Reading} \\
  \midrule

  

[1] Introduction		  		& 	 			& --Kim, ch.1  \\
(1/16/18)					&		  		&    \\ [1.8\baselineskip]

[2.] Personal Identity			& Quiz 1				& --\emph{**A Dialogue on Personal Identity and Immortality}, J. Perry   \\
(1/22/18)			        		&			  	&   \\  [1.8\baselineskip]

[3.] Continued			    	& Essay 1    		&  --Continued  \\
(1/29/18)				         &		  	    	&   \\  [1.8\baselineskip]
	
[4.] Dualism			      	& Essay 2			& --Kim, ch.2 \\
(2/5/18)				        	& 			    	&  \\ [1.8\baselineskip]

[5.] Behaviorism		    	& Quiz 2   		& --Kim, ch.3 \\
(2/12/18)			        		&		    	  	&  \\ [1.8\baselineskip]
  
[6.] Mind as Brain 		   	& Essay 3	   		& --Kim, ch.4\\
(2/19/18)				         & 			    	& \\  [1.8\baselineskip]	

[7.] Mental Causation 		& Quiz 3			& --Kim, ch.7, pp.193--209; 214--217 \\
(2/26/18)				      	&	      			& --**``Mental Causation'', sections 1-4, 7-8, S. Yablo  \\  [1.8\baselineskip]

Spring Break	&  & \\
(3/5/18)	&  &   \\ [1.8\baselineskip]	

[8.] Mind as Computer		& Quiz 4			& --Kim, ch.5, pp.129--138; 156--159\\
(3/12/18)		            		&		      		& --**``Computing Machinery and Intelligence'', A.M. Turing  \\  [1.8\baselineskip]

[9.] Continued			    	& Essay 4			& --Kim, ch.5, pp.156--165 \\
(3/19/18)				        &		    		& --**``Minds, Brains, and Programs'', J.  Searle \\ [1.8\baselineskip]


[10.] Physicalism		     	& Quiz 5			& --Kim, ch.10 \\ 
(3/26/18)				      	&			      	& --**``Epiphenomenal Qualia'', F. Jackson \\ [1.8\baselineskip]

						
[11.] Continued 	      			& Essay 5			&  --Continued \\
(4/2/18)				      	&			      	&  \\  [1.8\baselineskip]

						
[12.] Consciousness		 	& Quiz 6			& --Kim, ch.9, pp.263--280 \\
(4/9/18)				      	&			      	& --**``What is it like to be a bat?'', T. Nagel \\ [1.8\baselineskip]
 
[13.] Continued 	    		& Essay 6			& --Kim, ch.9, pp.280--289\\
(4/16/18)			      		&			      	& --**``Knowing Our Minds'', A. Byrne  \\ [1.8\baselineskip]

[14.] Self Consciousness	  	& Quiz 7			& --TBD \\
(4/23/18)				      	&		      		&  \\ [1.8\baselineskip]

[15.] Continued			    	& No assign.		& --TBD \\ 
(4/30/18)				      	&			       	& \\ [1.8\baselineskip]

[16.] Continued				& Sig. Assign.		&  \\ 
(5/7/18)				      	&			       	& \\



\end{longtable}
\end{center}






%% Uncomment if you want a printed bibliography.
%\printbibliography 

\end{document}