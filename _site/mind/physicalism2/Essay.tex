\documentclass[]{article}

\usepackage{fancyhdr}
 \pagestyle{fancy}
\rhead{\textsc{Scott O`Connor}}

\usepackage{lmodern}
\usepackage{amssymb,amsmath}
\usepackage{ifxetex,ifluatex}
\usepackage{fixltx2e} % provides \textsubscript
\ifnum 0\ifxetex 1\fi\ifluatex 1\fi=0 % if pdftex
  \usepackage[T1]{fontenc}
  \usepackage[utf8]{inputenc}
\else % if luatex or xelatex
  \ifxetex
    \usepackage{mathspec}
    \usepackage{xltxtra,xunicode}
  \else
    \usepackage{fontspec}
  \fi
  \defaultfontfeatures{Mapping=tex-text,Scale=MatchLowercase}
  \newcommand{\euro}{€}
\fi
% use upquote if available, for straight quotes in verbatim environments
\IfFileExists{upquote.sty}{\usepackage{upquote}}{}
% use microtype if available
\IfFileExists{microtype.sty}{%
\usepackage{microtype}
\UseMicrotypeSet[protrusion]{basicmath} % disable protrusion for tt fonts
}{}
\ifxetex
  \usepackage[setpagesize=false, % page size defined by xetex
              unicode=false, % unicode breaks when used with xetex
              xetex]{hyperref}
\else
  \usepackage[unicode=true]{hyperref}
\fi
\usepackage[usenames,dvipsnames]{color}
\hypersetup{breaklinks=true,
            bookmarks=true,
            pdfauthor={},
            pdftitle={Project},
            colorlinks=true,
            citecolor=blue,
            urlcolor=blue,
            linkcolor=magenta,
            pdfborder={0 0 0}}
\urlstyle{same}  % don't use monospace font for urls
\setlength{\parindent}{0pt}
\setlength{\parskip}{6pt plus 2pt minus 1pt}
\setlength{\emergencystretch}{3em}  % prevent overfull lines
\providecommand{\tightlist}{%
  \setlength{\itemsep}{0pt}\setlength{\parskip}{0pt}}
\setcounter{secnumdepth}{0}

\title{Project}
\author{Scott O’Connor}


% Redefines (sub)paragraphs to behave more like sections
\ifx\paragraph\undefined\else
\let\oldparagraph\paragraph
\renewcommand{\paragraph}[1]{\oldparagraph{#1}\mbox{}}
\fi
\ifx\subparagraph\undefined\else
\let\oldsubparagraph\subparagraph
\renewcommand{\subparagraph}[1]{\oldsubparagraph{#1}\mbox{}}
\fi

\begin{document}

\subsection{Signature Assignment (Final
Project)}\label{signature-assignment-final-project}

\subsubsection{Special Submission
Instruction}\label{special-submission-instruction}

\begin{itemize}
\tightlist
\item
  This paper must be submitted through NJCU's Tk20 site to earn a final
  grade. Details are here:
  \url{http://www.njcu.edu/general-education/signature-assignment-information-students}
\end{itemize}

\subsubsection{Introduction}\label{introduction}

Physicalism is the belief that mental events and their properties are
entirely physical and can be described in the same objective physical
way as all other phenomena in the world. If physicalism is true, then a
complete physical description of mental states must contain all the
information there is to know about those states. In `The Knowledge
Argument', Frank Jackson uses his famous Mary thought-experiment to show
that this is not the case: that there are facts about mental states not
entailed by a complete physical description. Several objections have
been raised against Jackson's argument, including:

\begin{enumerate}
\def\labelenumi{\arabic{enumi}.}
\tightlist
\item
  Mary gains ability knowledge, not new propositional knowledge
\item
  Mary gains acquaintance knowledge, not new propositional knowledge
\item
  Mary learns nothing new when leaving the room.
\item
  Mary comes to know the same physical fact but in a different way.
\end{enumerate}

\subsubsection{Task}\label{task}

\begin{quote}
Assess one objection to Jackson's Knowledge Argument.
\end{quote}

This assignment involves three distinct tasks, each with its own
sub-tasks.

\textbf{Task 1: Formulate Jackson's Knowledge Argument:} To do this, you
do the following:

\begin{enumerate}
\def\labelenumi{\arabic{enumi}.}
\tightlist
\item
  Identify the premises of the Knowledge Argument.
\item
  Explain each premise carefully in your words. Use your own examples.
\item
  Explain why the conclusion of the argument shows physicalism to be
  false.
\end{enumerate}

\textbf{Task 2: Explain one of the four objections to Jackson's
Knowledge Argument---your choice!} To do this, do the following:

\begin{enumerate}
\def\labelenumi{\arabic{enumi}.}
\tightlist
\item
  Read the entry on the Knowledge Argument in the Stanford Encyclopedia
  \href{https://plato.stanford.edu/entries/qualia-knowledge}{here} or a
  different encyclopedia entry
  \href{http://www.iep.utm.edu/know-arg/}{here}. You can also read the
  relevant entry in the textbook.
\item
  Pick \textbf{one} academic source that argues for your chosen
  objection.
\item
  Identify whether the source objects to one of the premises of
  Jackson's Knowledge Argument or whether it objects to the reasoning of
  that argument.
\item
  If a premise, identify which premise and explain how the article
  objects to that premise in your own words. Use your own examples.
\item
  If the reasoning, identify how the article shows it to be problematic
  in your own words. Use your own examples.
\end{enumerate}

\textbf{Task 3: Criticize the objection.} To do this, do the following:

\begin{enumerate}
\def\labelenumi{\arabic{enumi}.}
\tightlist
\item
  Identify the weakest claim made by the objection.
\item
  Explain why that claim is weak in your own words. Use your own
  examples.
\end{enumerate}

\subsubsection{Word Count}\label{word-count}

Your essay must be 1000-1500 words long. Essays shorter than 1000 words
or longer than 1500 words will lose points. I recommend the following
structure to your paper:

\begin{enumerate}
\def\labelenumi{\arabic{enumi}.}
\tightlist
\item
  Introduction: 50-100 words words.
\item
  Task 1: 400-550 words (make sure to divide this into smaller distinct
  paragraphs).
\item
  Task 2: 400-550 words (make sure to divide this into smaller distinct
  paragraphs).
\item
  Task 3: 150-300 words.
\end{enumerate}

\subsubsection{Plagiarism}\label{plagiarism}

Please review the plagiarism policy on the syllabus. It is critical that
you prepare your assignment by yourself. Use only the textbook and
handouts---it will take you less time to work through these materials
than to find and read other sources. I will be checking for significant
overlaps between submission as well as checking answers against
Wikipedia, internet search results, standard essay sites, etc. If you
include material in your essay without citing it, you will receive 0 for
the assignment. A second violation will result in a 0 for the course, a
report to the Dean, and a petition for a note to be added to your
permanent academic record.

\subsubsection{Late Submissions}\label{late-submissions}

Per the policies outlined in the syllabus, late work will not be
accepted. As the policies also state, there are no make-ups or extra
credit opportunities. Any request for special treatment will be ignored.
If you foresee difficulties submitting work on time, either because of
personal or commitments, then you should start this paper early and
submit it early.

\subsubsection{Format}\label{format}

Please submit the file as either a word file or simple .rtf file will
also suffice.

\subsubsection{Grading}\label{grading}

Please find the rubric and explanation of it
\href{/Teaching/Grading/}{here}.

\subsubsection{Resources}\label{resources}

Please find links to writing resources \href{/Teaching/Resources/}{here}

\end{document}
