% !TEX encoding = UTF-8 Unicode
% !TEX TS-program = xelatex

\documentclass[11pt]{article}
\usepackage{fontspec}
\defaultfontfeatures{Mapping=tex-text}
\usepackage{xunicode}
\usepackage{xltxtra}
\usepackage{verbatim}
\usepackage[margin= 1 in]{geometry} % see geometry.pdf on how to lay out the page. There's lots.
\geometry{letterpaper} % or letter or a5paper or ... etc
%\usepackage[parfill]{parskip}    % Activate to begin paragraphs with an empty line rather than an indent 
\usepackage{mathrsfs}
\usepackage{bbding}
\usepackage[usenames,dvipsnames]{color}
\usepackage{natbib}
\usepackage{stmaryrd}
%\usepackage{mathpartir}
\usepackage{txfonts}
\usepackage{graphicx}
\usepackage{fullpage}
\usepackage{hyperref}
\usepackage{amssymb}
\usepackage{epstopdf}
\usepackage{fontspec}
%\setmainfont{Hoefler Text}
\setmainfont[BoldFont={Minion Pro Bold}]{Minion Pro}
\usepackage{hyperref}
\usepackage{lastpage, fancyhdr}
%\usepackage{setspace}
\pagestyle{fancy}
\lhead{}
\chead{Overview\space---\space Handout} 
\rhead{}
\lfoot{}
\cfoot{\thepage\space of \pageref{LastPage}} 
\rfoot{}
\footskip=30 pt
\headsep=20pt
\thispagestyle{empty}
\hypersetup{colorlinks=true, linkcolor=Sepia, urlcolor=Sepia, citecolor=BrickRed}
\DeclareGraphicsRule{.tif}{png}{.png}{`convert #1 `dirname #1`/`basename #1 .tif`.png}
\usepackage{polyglossia}
\setdefaultlanguage{english}
\setotherlanguage{greek}
\newfontfamily\greekfont{Gentium Plus}
\newcommand{\gk}[1]{\textgreek{#1}}
\newcommand{\gloss}[1]{(\textgreek{#1})}
%\title{Authority in the Two Worlds \\ \hrulefill}
%\author{\href{http://www.princeton.edu/~edlord/Site/home.html}{Errol Lord}}
%\date{}
\usepackage{covington}
\usepackage{fixltx2e}
\usepackage{graphicx}
\begin{document}

%\maketitle
\thispagestyle{empty}
\begin{center} \LARGE{PHIL 321\\ Lectures 2-3: ``Pre-Philosophical'' Greek Worldview\\ and Presocratics}\\ \vspace*{2mm}
\large{9/3--9/5//2013}\end{center}
\thispagestyle{empty}\vspace*{3mm}

\section*{Transmission of Texts}

\begin{itemize}\item{Texts were usually written on scrolls of papyrus (later (*), objects more closely resembling books, with multiple smaller pages bound together, appeared); papyrus is not particularly strong material}\item{Reading the scroll would ``reverse'' it and you'd have to unwind it to get back to the beginning; WEAR and TEAR; so you can imagine that authors would be hesitant to read more than they needed to; how does that affect later quotations?}\item{Typically, authors put no breaks between words (later, in the Roman period, some authors would employ dots to mark end of words)}\item{Ancient editors would often ``athetize'' material (i.e. claim that a portion of a text is spurious) for dubious reasons (e.g. the content was deemed objectionable, such as displaying the gods in a negative light)}\begin{itemize}\item{Fortunately, most editors would just mark the sections, rather than take them out}\end{itemize}\item{Both the Christian and Islamic traditions played a huge role in transmitting the texts (the latter especially Aristotle, but also some Plato)}\item{Texts were copied by hand, century after century, until printing press was invented; difficulty of copying texts by hand; some of the oldest manuscripts we have date from Plato and Aristotle-mid 9th century CE (although we have scraps of papyrus scrolls from the 2nd and third centuries CE; modern texts don't copy any one manuscript, editors go through all the available material and choose the text they think is most faithful to the original; we stand in this stream, people copying these works by hand century after century, with the overriding sense that it was important}\end{itemize}

\section*{``Pre-Philosophical'' Greek Worldview}
\begin{itemize}\item{Main Sources: Homer (\emph{Iliad}, \emph{Odyssey}), Hesiod (\emph{Theogony}, \emph{Works and Days}); also playwrights (e.g. Aeschylus, Sophocles, Euripides, Aristophanes), poets (e.g. Pindar), and philosophers}

\item{Structure of the world:} \begin{itemize}\item{Sky is a solid hemisphere (like a bowl) that covers the round flat earth (earth's surface is shaped like a disc)}\item{The lower part of the gap between earth and sky is filled with air (\emph{a\^{e}r}); upper part filled with aether (\emph{aith\^{e}r}, conceived of as fiery)}\item{The earth stretches far downwards, past Hades (where souls go upon death) to Tartarus}\item{Surrounding the earth is the river \emph{Okeanos}, from which all water ultimately has its source}\end{itemize}\end{itemize}\textbf{Discuss extent to which various aspects of this are supported by observation; shape seems ok; fire rises so that seems to support the idea that the upper reaches of the heaven would be composed of fire; surrounding river? more difficult to support; maybe reports from sailors?}
\begin{itemize}

\item{Ethical: Homer's \emph{Iliad} and \emph{Odyssey} were often viewed as providing the ideal moral code. The actions and characters of the heroes depicted therein were taken as models to emulate. The philosophers we deal with can all be expected to have been very familiar with Homer.}\begin{itemize}\item{\emph{Aret\^{e}} is emphasized. ``\emph{Aret\^{e}}'' is often translated as ``virtue,'' and there are points of contact. But it means more generally, ``excellence,'' being the best \emph{X} or being the best at \gk{Φ}-ing. So, traditional virtues like courage, justice, temperance, wisdom, were emphasized, but also physical strength, quickness of mind (including deviousness), etc..}\item{Getting what one wants (sometimes irrespective of how it affects other people) is taken as a sign of being a successful person.}\item{Honor and esteem of one's fellow citizens highly valued. Material honors highly valued. Slights were taken seriously and viewed as requiring response.}\item{Reverence for the gods emphasized.}\end{itemize}

\item{Explanatory: Supernatural explanations were offered for many phenomena, from the structure of the world, to natural events (weather, volcanic eruptions etc.), to particular incidents in individuals' lives, to the justification for beliefs.}\end{itemize}

\noindent \textbf{[Text 1: Hesiod \emph{Theogony}, 114-126]}\\Tell me these things, Muses, who dwell on Olympus\\From the beginning, and tell me which of them was born first.\\First of all Chaos came into being. Next came\\broad-breasted Earth (\emph{gaia}), the secure dwelling place forever of all\\the immortals who hold the peak of snowy Olympus.\\And murky Tartaros in a recess of the broad-roaded Earth,\\and Love (\emph{Eros}), who is the most beautiful among the immortal gods, who\\loosens the limbs and overpowers the intentions and sensible plans\\of all the gods and all humans too.\\From Chaos there came into being Darkness (\emph{Erebos}) and black Night.\\From Night, Aether and Day (\emph{h\^{e}mera}) cam into being,\\which she conceived and bore after uniting in love with Erebos.\\

\vspace*{1mm}

\noindent\textbf{[Text 2: Homer \emph{Odyssey}, 5.365-70]}\\While he pondered thus in mind and heart,\\Poseidon, the earth-shaker, made to rise up a great wave,\\dread and grievous, arching over from above, and drove it upon him.\\And as when a strong wind tosses a heap of straw that is dry, \\and some it scatters here, some there,\\even so the wave scattered the long timbers of the raft.\\

\vspace*{1mm}

\noindent\textbf{[Text 3: Homer \emph{Iliad}, 4.124-32]}\\But when he had drawn the great bow into a round,\\the bow twanged and the string sang aloud, and the keen arrow leapt,\\eager to wing its way amid the throng.\\Then, O Menelaus, the blessed gods, the immortals, forgat thee not;\\and before all the daughter of Zeus, she that driveth the spoil,\\who took her stand before thee, and warded off the stinging arrow.\\She swept it just aside from the flesh,\\even as a mother sweepeth a fly from her child when he lieth in sweet slumber;\\

\vspace*{1mm}

\noindent \textbf{[Text 4: Homer \emph{Iliad}, 2.484-87]}\\Tell me now Muses, who have dwellings in Olympus\\for you are goddesses and present and know everything,\\while we hear only rumor and we know nothing;\\Who were the Greek commanders and leaders?

\section*{Presocratics: General}

\begin{itemize}\item{Loosely affiliated group of thinkers: doctors, poets, politicians, etc. in the 6th--5th century BCE}\begin{itemize}\item{\emph{Main justification for title ``Presocratic'' is that, as far as our evidence shows, Socrates had little to no influence on them; unlike everyone else in the tradition afterwards}}\end{itemize}

\item{Sources: No complete manuscript of any original work survives}

\item{H. Diels first compiled the extant sources, later revised by W. Kranz (DK = Diels-Kranz), gave each Presocratic philosopher a number}

\begin{itemize}\item{\emph{Testimonia}: Reports of their views by later authors (A)}\item{Fragments: Allegedly direct quotations by later authors (B)}\item{So, each piece of evidence is categorized: e.g. DK \#A/B \# (DK 22A1 is testimonium 1 of Heraclitus)}\end{itemize}
\begin{itemize}\item{\emph{Discuss the problems this introduces; polemical, purposes, combine with the physical problem above}}\end{itemize}
\item{Engaged in theoretical speculation about a wide range of topics. In many ways they were motivated by the ``Problem of Change'' or ``Problem of Contradictions.''} \begin{itemize}\item{Nature of the world, covering what is now considered physics, chemistry, geology, meteorology, astronomy, embryology, psychology, among other areas}\item{More distinctly ``philosophical'' topics: theology, metaphysics, epistemology, and ethics}\end{itemize}

\end{itemize}

\section*{Presocratics: Particular}

\begin{itemize}\item{\textbf{Milesians} (flourished in Miletus (in Ionia, modern western Turkey)): Most famous for natural philosophy, trying to explain the structure of the world and natural events by identifying the fundamental elements that make up the world and the forces that govern them}

\begin{itemize}

\item{\textbf{Thales}: Posited water as the fundamental principle (\emph{arch\^{e}}) of everything}\begin{itemize}\item{(At least) Two ways to interpret such a claim: (1) Everything comes to be from water or, (2) Everything is water in some form or other}\item{Water, then, is the fundamental stuff, and the changes it undergoes are due to its own nature and not due to, e.g., divine intervention}\end{itemize}\end{itemize}

\noindent\textbf{[Text 5: Aristotle \emph{On the Soul,}}\hspace*{35mm}\textbf{[Aristotle attributes to Thales]}\\\textbf{1.2 405a19-21 (11A22)]}\\From what is related about him, it seems that\hspace*{8.5mm}\textbf{[P1]} The lodestone moves iron\\Thales too held that the soul is something\hspace*{14mm}\textbf{[P2]} \underline{Only things with soul can produce motion} \\productive of motion, if indeed he said that\hspace*{11.5mm}\textbf{[C]}: The lodestone has soul\\the lodestone has soul, because it moves iron.

\begin{itemize}

\item{\textbf{Anaximander} (allegedly drew first map of ``inhabited'' world): Posited the ``indefinite'' (\emph{apeiron}) as the fundamental stuff (most likely meaning that it itself has no definite character)}

\item{\textbf{Anaximenes}: Posited air as the fundamental stuff and explicitly states that rarefaction and condensation are the processes whereby things are produced out of air}\begin{itemize}\item{Rarefied air becomes fire and ultimately aether; through progressive condensing, air becomes wind, cloud, water, earth, stones}\end{itemize}\end{itemize}

\item{\textbf{Heraclitus of Ephesus} (end of 6th Century--5th Century BCE)}

\begin{itemize}\item{Posited a theory of universal ``flux'' or change; the basic idea is that everything is constantly changing in every regard but that therein lies the nature of everything; fire was the physical image of this idea--fire is constantly moving but remains the same fire}\item{Also claimed that things could actually possess opposites at the same time}\end{itemize}\end{itemize}

\noindent{\textbf{[Text 6: Plutarch \emph{On the E at Delphi}, 392b (22B91)]}\\It is not possible to step twice into the same river. ... It scatters and again comes together, and approaches and recedes. (Also cf. B12, B49a)\\

\noindent{\textbf{[Text 7: Hippolytus \emph{Refutation of All Heresies}, 9.10.4-5 (22B60-61)]}}\\The road up and the road down are one and the same. The sea is the purest and most polluted water: to fishes drinkable and bringing safety, to humans undrinkable and destructive. (Also cf. B82, B13, B9, B4, B37, B83, B102, B124, B103, B126, B67, B88, B111, B62)

\begin{itemize}

\item{\textbf{Eleatics} (flourished in Elea, southern Italy)}

\begin{itemize}

\item{\textbf{Parmenides}: Wrote a poem that was very influential in the tradition (particularly on Plato); Strongly rejected Heraclitus' idea that one and the same thing could possess contradictory properties (indeed, may have thought that nothing can be more than one thing)}\begin{itemize}\item{Reflected on what we are doing when we say ``\emph{X} is \emph{F}'' (e.g. ``The grass in my yard is green,'' ``Human beings are mortal,'' ``The sum of two and two is four''}\item{He thought that we are saying, quite simply, that \emph{X} \textbf{is} \emph{F}; we do not allow that it is only ``in some respect,'' ``in relation to something,'' ``at some time (e.g. now, for four years),'' ``but may not be \emph{F} later on,'' and so on.}\item{This led him to conclude that, despite appearances, nothing can \emph{really} ``come to be \emph{F},'' or ``cease being \emph{F}''; Nothing that is really \emph{F} can change in respect of \emph{F}-ness}\begin{itemize}\item{\textbf{Does this seem too strong; It seems counter-intuitive to say that the sentence ``\emph{X} is \emph{F}'' is false if there are just some circumstances in which \emph{X} is not \emph{F}; but, what about state claims; or considering it in the abstract; fundamentality of ``is'' of ``was'' and ``will be''}\item{\textbf{Go through the alleged contradictions: if \emph{X} is \emph{F}, then \emph{X} cannot be not-\emph{F}; so, can \emph{X} also be \emph{G}?; well is G \emph{F}?, no; so \emph{G} is not-\emph{F}? yes, then if \emph{X} is \emph{G} then \emph{X} is not-\emph{F}, but we already said it can't be that}}}\end{itemize}\item{If this is right, then we have to ask what kinds of things can support those kinds of claims (the things in the physical world? some other kind of entity?)}\end{itemize}

\item{\textbf{Zeno}: Known primarily for formulating paradoxes (of which 4 are known)}\end{itemize}\end{itemize}
\noindent{\textbf{[Text 8: Aristotle, \emph{Physics} 6.9 239b9-19 (29A25-26)]}}\\First is the argument that there is no motion because that which is moving must reach the midpoint before the end. . . . It is always necessary to traverse half the distance, but these are infinite, and it is impossible to get through things that are infinite. The second is called the ``Achilles.'' This is to the effect that the slowest as it runs will never be caught by the quickest. For the pursuer must first reach the point from which the pursued departed, so that the slower must always be some distance in front.

\begin{itemize}
\item{\textbf{Atomists}: \textbf{Leucippus} and \textbf{Democritus}}\begin{itemize}\item{Posited that all of physical reality was made up of, on the one hand, an infinite number of solid, uncuttable (\emph{atomon}) units of matter, and, on the other hand, void, or empty space, through which the atoms move. The atoms themselves have no definite characteristics, other than size, shape, and arrangement. Through their interaction (bouncing off each other, sticking to each other, separating), all other objects, including macroscopic object, come to be. (There seems to have been no advancement on this physical theory until the 19th Century CE.)}\end{itemize}\end{itemize}

\noindent\textbf{[Text 9: Sextus Empiricus, \emph{M} 7.135 (68B9)]}\\By convention sweet; by convention, bitter; by convention, hot; by convention, cold; by convention, color: but in reality, atoms and void.

\end{document}
