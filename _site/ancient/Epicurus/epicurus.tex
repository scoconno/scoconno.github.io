\documentclass[]{article}
\usepackage{amssymb,amsmath}
\usepackage{ifxetex,ifluatex}
\usepackage{fixltx2e} % provides \textsubscript
\ifnum 0\ifxetex 1\fi\ifluatex 1\fi=0 % if pdftex
  \usepackage[T1]{fontenc}
  \usepackage[utf8]{inputenc}
\else % if luatex or xelatex
  \ifxetex
    \usepackage{mathspec}
    \usepackage{xltxtra,xunicode}
  \else
    \usepackage{fontspec}
  \fi
  \defaultfontfeatures{Mapping=tex-text,Scale=MatchLowercase}
  \newcommand{\euro}{€}
\fi
% use upquote if available, for straight quotes in verbatim environments
\IfFileExists{upquote.sty}{\usepackage{upquote}}{}
% use microtype if available
\IfFileExists{microtype.sty}{%
\usepackage{microtype}
\UseMicrotypeSet[protrusion]{basicmath} % disable protrusion for tt fonts
}{}
\ifxetex
  \usepackage[setpagesize=false, % page size defined by xetex
              unicode=false, % unicode breaks when used with xetex
              xetex]{hyperref}
\else
  \usepackage[unicode=true]{hyperref}
\fi
\hypersetup{breaklinks=true,
            bookmarks=true,
            pdfauthor={},
            pdftitle={},
            colorlinks=true,
            citecolor=blue,
            urlcolor=blue,
            linkcolor=magenta,
            pdfborder={0 0 0}}
\urlstyle{same}  % don't use monospace font for urls
\setlength{\parindent}{0pt}
\setlength{\parskip}{6pt plus 2pt minus 1pt}
\setlength{\emergencystretch}{3em}  % prevent overfull lines
\setcounter{secnumdepth}{0}

\date{}

\begin{document}

\begin{center}\rule{0.5\linewidth}{\linethickness}\end{center}

\section{Epicurus}

Epicurus founded a community in Athens called ``the Garden''. Men and
women, free persons, and slaves, were all equal.

He claimed that the goal of philosophy was to provide a guide to living
well. In this, he was in agreement with Socrates and Plato. However, he
offered a particular view as to what made a life well lived.

According to Epicurus, one should maximize pleasures and minimize pains.
You should accept pains that lead to greater pleasures. You should
reject pleasures that lead to greater pains.

\subsubsection{Death}\label{death}

\begin{quote}
``For there is nothing terrible in life for the man who has truly
comprehended that there is nothing terrible in not living.'' 1. Nothing
is good or bad for one except sense experience, i.e.~feelings of
pleasure and pain. 2. The dead don't have any sense experiences. 3.
Therefore nothing is good or bad for the one who is dead. 4. Therefore
the state of being dead is not (good or) bad for the one who is dead. 5.
If x is not bad for one when it is present, then there is no rational
ground, before it is present, to fear its future presence. 6. Therefore
no living person has any rational ground to fear his future state of
being dead.
\end{quote}

Problem comes from fearing/anticipating death. Not from death itself.

Death is deprivation of sensation, therefore a deprivation of something
good. Understanding death makes life enjoyable because it takes away the
craving for immortality.

Distinguish the pain of death vs.~the pain of anticipating death.

Letter to Menoeceus. Maybe have someone read aloud the first paragraph.

\subsubsection{Ethics\#\#\#\textbf{Epicurean Hedonism (hêdonê =
pleasure):}1. Eudaimonist framework2. Happiness = ataraxia, freedom from
disturbance:}\label{ethicsepicurean-hedonism-huxeadonuxea-pleasure1.-eudaimonist-framework2.-happiness-ataraxia-freedom-from-disturbance}

\begin{quote}
``The unwavering contemplation of these {[}distinctions among desires{]}
enables one to refer every choice and avoidance to the health of the
body and the freedom of the soul from disturbance {[}ataraxia{]}, since
this is the goal of the blessed life''.
\end{quote}

\subsubsection{Pleasure}\label{pleasure}

\begin{itemize}
\itemsep1pt\parskip0pt\parsep0pt
\item
  Static/katastematic pleasure consists in the absence of pain, want and
  desire, e.g., freedom from hunger, thirst. + Kinetic pleasures: always
  involve a change in one's psychic state, valuable as a means to
  achieving static pleasure (e.g., quenching one's thirst).
\item
  ataraxia consists in static pleasure, understood as the absence of
  pain (33, 60).
\item
  Adaptive conception of happiness: ataraxia achieved by (1) satisfying
  desires, (2) eliminating them. Thus, in order to be happy, one ought
  to change desires so that one only wants things that are easy to get.
  \textbf{Types of Desire}
\end{itemize}

Because of the close connection of pleasure with desire-satisfaction,
Epicurus devotes a considerable part of his ethics to analyzing
different kinds of desires. If pleasure results from getting what you
want (desire-satisfaction) and pain from not getting what you want
(desire-frustration), then there are two strategies you can pursue with
respect to any given desire: you can either strive to fulfill the
desire, or you can try to eliminate the desire. For the most part
Epicurus advocates the second strategy, that of paring your desires down
to a minimum core, which are then easily satisfied.

Epicurus distinguishes between three types of desires:

\begin{itemize}
\itemsep1pt\parskip0pt\parsep0pt
\item
  natural and necessary desires:

  \begin{itemize}
  \itemsep1pt\parskip0pt\parsep0pt
  \item
    Examples of natural and necessary desires include the desires for
    food, shelter, and the like.
  \item
    These desires are easy to satisfy, difficult to eliminate (they are
    `hard-wired' into human beings naturally), and bring great pleasure
    when satisfied.
  \item
    Furthermore, they are necessary for life, and they are naturally
    limited: that is, if one is hungry, it only takes a limited amount
    of food to fill the stomach, after which the desire is satisfied.
  \item
    Epicurus says that one should try to fulfill these desires.
  \end{itemize}
\item
  natural but non-necessary desire:

  \begin{itemize}
  \itemsep1pt\parskip0pt\parsep0pt
  \item
    An example of a natural but non-necessary desire is the desire for
    luxury food. Although food is needed for survival, one does not need
    a particular type of food to survive.
  \item
    Thus, despite his hedonism, Epicurus advocates a surprisingly
    ascetic way of life. Although one shouldn't spurn extravagant foods
    if they happen to be available, becoming dependent on such goods
    ultimately leads to unhappiness.
  \item
    As Epicurus puts it, ``If you wish to make Pythocles wealthy, don't
    give him more money; rather, reduce his desires.''
  \item
    By eliminating the pain caused by unfulfilled desires, and the
    anxiety that occurs because of the fear that one's desires will not
    be fulfilled in the future, the wise Epicurean attains tranquility,
    and thus happiness.
  \end{itemize}
\item
  vain and empty desires:

  \begin{itemize}
  \itemsep1pt\parskip0pt\parsep0pt
  \item
    Vain desires include desires for power, wealth, fame, and the like.
  \item
    They are difficult to satisfy, in part because they have no natural
    limit. If one desires wealth or power, no matter how much one gets,
    it is always possible to get more, and the more one gets, the more
    one wants.
  \item
    These desires are not natural to human beings, but inculcated by
    society and by false beliefs about what we need; e.g., believing
    that having power will bring us security from others.
  \item
    Epicurus thinks that these desires should be eliminated.
  \end{itemize}
\end{itemize}

\end{document}
