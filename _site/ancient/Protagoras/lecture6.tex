% !TEX encoding = UTF-8 Unicode
% !TEX TS-program = xelatex

\documentclass[11pt]{article}
\usepackage{fontspec}
\defaultfontfeatures{Mapping=tex-text}
\usepackage{xunicode}
\usepackage{xltxtra}
\usepackage{verbatim}
\usepackage[margin= 1 in]{geometry} % see geometry.pdf on how to lay out the page. There's lots.
\geometry{letterpaper} % or letter or a5paper or ... etc
%\usepackage[parfill]{parskip}    % Activate to begin paragraphs with an empty line rather than an indent 
\usepackage{mathrsfs}
\usepackage{bbding}
\usepackage[usenames,dvipsnames]{color}
\usepackage{natbib}
\usepackage{stmaryrd}
%\usepackage{mathpartir}
\usepackage{txfonts}
\usepackage{graphicx}
\usepackage{fullpage}
\usepackage{hyperref}
\usepackage{amssymb}
\usepackage{epstopdf}
\usepackage{fontspec}
%\setmainfont{Hoefler Text}
\setmainfont[BoldFont={Minion Pro Bold}]{Minion Pro}
\usepackage{hyperref}
\usepackage{lastpage, fancyhdr}
%\usepackage{setspace}
\pagestyle{fancy}
\lhead{}
\chead{Lecture 6, Plato's \emph{Protagoras}\space---\space Handout} 
\rhead{}
\lfoot{}
\cfoot{\thepage\space of \pageref{LastPage}} 
\rfoot{}
\footskip=30 pt
\headsep=20pt
\thispagestyle{empty}
\hypersetup{colorlinks=true, linkcolor=Sepia, urlcolor=Sepia, citecolor=BrickRed}
\DeclareGraphicsRule{.tif}{png}{.png}{`convert #1 `dirname #1`/`basename #1 .tif`.png}
\usepackage{polyglossia}
\setdefaultlanguage{english}
\setotherlanguage{greek}
\newfontfamily\greekfont{Gentium Plus}
\newcommand{\gk}[1]{\textgreek{#1}}
\newcommand{\gloss}[1]{(\textgreek{#1})}

\usepackage{covington}
\usepackage{fixltx2e}
\usepackage{graphicx}
\begin{document}

%\maketitle
\thispagestyle{empty}
\begin{center} \LARGE{PHIL 321\\ Lecture 6: Plato's \emph{Protagoras} (348c-end)}\\ \vspace*{2mm}
\large{9/17/2013}\end{center}
\thispagestyle{empty}\vspace*{3mm}
\vspace*{-8mm}

\section*{317e-334c}

\noindent Protagoras (a well-known sophist) claims to teach his students ``sound deliberation'' (\emph{euboulia}), which Socrates equates with the ``art of citizenship'' (\emph{politik\^{e} techn\^{e}}) and ``virtue'' (\emph{aret\^{e}})
\vspace*{2mm}

\noindent S wonders whether virtue can be taught (P's claim presupposes that it can)
\vspace*{2mm}

\noindent S asks P, ``Is virtue a single thing, with justice and temperance and piety its parts, or are the things I have just listed all names for a single entity'' (329d) Two options for ``parts'':
\vspace*{2mm}

\begin{itemize}\item{[A] In the sense in which the mouth, nose, eyes, and ears are parts of the face (heterogenous parts)}\item{[B] In the sense in which there are ``parts'' of a block of gold (homogenous parts)}\end{itemize}
\vspace*{2mm}

\noindent S asks further whether someone can have some parts of virtue \emph{without} having all of them
\begin{itemize}
\item{P says yes, and points to (allegedly) courageous but unjust people, just but unwise people, ...}\end{itemize}

\noindent S argues that the virtues do form a unity, each of them being some kind of knowledge/understanding (\emph{epist\^{e}m\^{e}})

\begin{itemize}\item{P (begrudgingly) agrees that all the virtues \emph{except} courage may be kinds of knowledge}\end{itemize}

\section*{Hedonism}

\noindent [A] Pleasant things are good insofar as they are pleasant (pleasure is \emph{a} good)
\vspace*{2mm}

\noindent [B] Pleasant things alone are good (pleasure is \emph{the} good)

\begin{itemize}\item{[B1]: One's present pleasure is the good}\item{[B2]: Maximized pleasure is the good (S attributes this view to the many, 354a-c)}\end{itemize}
\vspace*{2mm}

\noindent S also assumes that all pleasures and pains are ``commensurable''---all pleasures and pains can be weighed against each other; thus, in principle, all pleasures and pains can be ranked

\section*{The experience of ``\emph{akrasia},'' (``being overcome,'' ``lack of self-control,'' ``weakness of will''), as reported by ``the many'' (\emph{hoi polloi})}

\noindent One's knowledge (\emph{epist\^{e}m\^{e}}) that an action is bad can be overcome by desire, pleasure, pain, love, fear, etc. (352b, d, 353c)
\vspace*{2mm}

\noindent Two explicit versions:

\begin{itemize}\item{[1] X does B, i) knowing B is bad, ii) when able not to do B, and iii) overwhelmed by pleasure}
\item{[2] X does not do G, i) knowing G is good, ii) when able to do G, and iii) overwhelmed by pleasure}\end{itemize}

\section*{Socrates' aim}

\noindent The probandum (``thing to be proved''): Knowledge ``rules'' in a person---if X knows that A is good, X will do A, if X can; if X knows that A is bad, X will \emph{not} do X, if X can

\begin{itemize}\item{If S proves this, he will have shown that the akratic situation is impossible or misdescribed}

\item{Given that S suspects virtue is some kind of knowledge, it is clear why he wants to argue for this}\end{itemize}

\section*{Socrates' argument}

\noindent [P1] Pleasure is the good
\vspace*{2mm}

\noindent [P2] All pleasures and pains are commensurable
\vspace*{2mm}

\noindent [P3] The akratic situation, when re-described in accordance with [P1], amounts to:

\begin{itemize}
\item{[1$^{*}$]: X did B, knowing B to be bad and able not to do it, because X was overcome by good =}
\item{X did something painful, knowing it to be painful and able not to do it, because X was overcome by pleasure}\end{itemize}
\vspace*{2mm}
 
 \noindent [P4] In 1$^{*}$, the good/pleasure is less than the overall bad/pain X chose to get
 \vspace*{2mm}
 
 \noindent [P5] So in both cases, X chose to do what X knew was worse/more painful instead of what was less bad or less painful
 \vspace*{2mm}
 
 \noindent \underline{[P6] It is impossible to choose the worse of two alternatives when you know (or believe?) it to be the worst}
 \vspace*{2mm}
 
 \noindent [C] Therefore, it is only ignorance of the relative weights of the alternatives that can explain X's selection of the worse
\vspace*{2mm}
 
 \noindent S re-describes the akratic situation as being a manifestation of ``ignorance'' (\emph{amathia})---the ``power of appearance'' leads a person to miscalculate the relative weight of an immediate pleasure (e.g. just as seeing something in a distance can lead us to judge (mistakenly) that it is smaller than something that is up close)
 \vspace*{2mm}
 
 \noindent S uses this conclusion to argue that even courage is a kind of knowledge/understanding, namely knowledge of ``what is and is not to be feared'' (360d8-9)
 
\section*{Problems}
 
 \noindent Is hedonism plausible in its own right?
 \vspace*{2mm}
 
 \noindent [P6] is only plausible if all desire is for the maximized good or that there is only one kind of desire/faculty of desire. Are there desires that are not aimed at the good?
 \vspace*{2mm}
  
 \noindent Does S illegitimately move between Hedonism B1 and B2 in his argument?
 \vspace*{2mm}
 
 \noindent Even if S's argument succeeds, is there an epistemic problem? If virtue is to guarantee correct action, and virtue is a kind of knowledge/understanding, will this require unrealistic demands of knowing \emph{all} the future consequences of our actions?



\end{document}
