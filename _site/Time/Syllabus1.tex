\documentclass[article,oneside]{memoir}
\usepackage{long table}
%%% custom style file with standard settings for xelatex and biblatex. Note that when [minion] is present, we assume you have minion pro installed for use with pdflatex.
%\usepackage[minion]{org-preamble-pdflatex} 

%%% alternatively, use xelatex instead
\usepackage{org-preamble-xelatex} 



\def\myauthor{Author}
\def\mytitle{Title}
\def\mycopyright{\myauthor}
\def\mykeywords{}
\def\mybibliostyle{plain}
\def\mybibliocommand{}
\def\mysubtitle{}
\def\myaffiliation{NJCU}
\def\myaddress{Phil 1}
\def\myemail{soconnor@njcu.edu}
\def\myweb{\href{http://scottoconnor.org/Time}{http://scottoconnor.org/Time}}
\def\myphone{}
\def\myversion{}
\def\myrevision{}
\def\myaffiliation{NJCU}
\def\myauthor{Dr. Scott O'Connor}
\def\mykeywords{}
\def\mysubtitle{Syllabus}
\def\mytitle{{\normalsize \myweb \newline} \HUGE Time}


\begin{document}

%%% If using xelatex and not pdflatex
%%% xelatex font choices
\defaultfontfeatures{}
\defaultfontfeatures{Scale=MatchLowercase}    
% You will need to buy these fonts, change the names to fonts you own, or comment out if not using xelatex.      
\setromanfont[Mapping=tex-text]{Minion Pro} 
\setsansfont[Mapping=tex-text]{Myriad Pro} 
\setmonofont[Mapping=tex-text,Scale=0.8]{Georgia} 

%% blank label items; hanging bibs for text
%% Custom hanging indent for vita items
\def\ind{\hangindent=1 true cm\hangafter=1 \noindent}
\def\labelitemi{$\cdot$}
%\renewcommand{\labelitemii}{~}

%% RCS info string for version tracking
\chapterstyle{article-3}  % alternative styles are defined in latex-custom-kjh/needs-memoir/
%\pagestyle{kjh}

\title{\LARGE \mytitle}     
\author{\Large\myauthor \newline \footnotesize\texttt{\noindent Office hours: \href{http://scottoconnor.org/contact/office/}{http://scottoconnor.org/contact/office/}}}
\date{1/22/2019--5/14/2019}

\published{\textbf{Phil 313--2 (2675), 3 credits, Spring 2019, M\&W 12:45pm--2:00pm, K 112}}

\maketitle

%\thispagestyle{kjhgit}

% Copyright Page
%\textcopyright{} \mycopyright


%
% Main Content
%

\section{Disclaimer}
 This syllabus is subject to change at the discretion of the faculty. Students will be notified of such changes ahead of time via Blackboard. 


\section{Copyright}
The materials used in this class, including, but not limited to, lectures, exams, quizzes, and homework assignments are copyright protected works.  Any unauthorized copying of the class materials or recording of lectures is a violation of federal law and may result in disciplinary actions being taken against the student.  Additionally, the sharing of class materials without the specific, express approval of the instructor may be a violation of the University's Student Honor Code and an act of academic dishonesty, which could result in further disciplinary action.  This includes, among other things, uploading class materials to websites for the purpose of sharing those materials with other current or future students. 

\section{Catalog Description}
The measuring of time is both essential to us and the societies we live in. But what exactly is it that we are measuring? This course introduces students to the interdisciplinary study of time. Readings will be drawn from historical and contemporary work in the social and natural sciences, philosophy, and literature.

\section{Course Description}
Students will be introduced to the study of time by studying its nature and the role it plays in our lives and societies more generally. Readings will be drawn from a variety of disciplines: history, classics, sociology, literature, philosophy, physics, and psychology. Each student will focus for a significant part of the course on a capstone project in which he or she will undertake study in the modes of inquiry that best reflect his or her academic interests and are most appropriate for the chosen capstone project. 





\section{Learning Objectives}

Upon completing this course, students will be able to (i) find the argument of a text and clearly restate it, (ii) clearly and charitably explain viewpoints that are not their own, (iii) think critically, (iv) write well-structured prose in which they clearly state a thesis and persuasively defend it, (v) demonstrate a substantive grasp of how time is an  important topic of study in several distinct disciplines. 

\section{General Education Information} 
Successfully completing this course satisfies one Tier 3 capstone requirement. For further information about the General Education Program see \href{http://njcu.edu/department/general-education}{http://njcu.edu/department/general-education}.

\section{Required Textbooks}
Available in the campus book store and online retailers. All other readings will be distributed on the course website. 
\begin{itemize}
\item \href{http://www.amazon.com/Travels-Four-Dimensions-Enigmas-Space/dp/0198752555/ref=sr_1_1?ie=UTF8&qid=1452098846&sr=8-1&keywords=robin+le+poidevin+enigmas}{Robin LePoidevin, `Travels in Four-Dimensions: The Enigmas of Space and Time', OUP, 2005 (TRAVELS)}
\end{itemize}


\section{Course Website}
There is both a Blackboard site and website for this course (link on first page). Clicking the first link on the left panel within the Blackboard site will bring you to the course website. All assignments will be submitted through Blackboard. Readings, notes, etc. will be posted on the course website. Note that Blackboard difficulties are rare and automatically reported to instructors. Under no circumstance will a student's report of a Blackboard difficulty be reason for an extension. It is your responsibility to contact Blackboard support for help.



\section{Requirements}

\begin{itemize}
\item \textit{Workload:} Expect to spend an average of 6 hours per week completing the readings and assignments. NJCU abides by the Federal and State definitions of a credit hour and adopts a policy consistent with the Carnegie Unit. A three-credit class represents 112.5 hours total of work. See \href{http://scottoconnor.org/resources/Credit.pdf}{here} for more details.


\item \textit{Participation:}  0.5 point will be awarded per class up to a maximum of 10 points. Points will not be awarded during weeks 1 \& 2. Participation points will be awarded if you attend, stay for the duration of the class, stay alert, leave your electronic devices turned off and out of sight, contribute to our discussions, and are non-disruptive. 

\item \textit{Library assignment:} Requires attendance during the library session and a short assignment submitted through Blackboard.

\item \textit{Textbook questions} submitted through Blackboard. 7 sets will be assigned. You must complete 4 sets. If you complete more than 4 sets, the lowest grades will be dropped. 




 
\item \textit{Capstone project} submitted through Blackboard comprising a proposal, annotated bibliography, literature review, presentation, and final submission (2000--2500 words.)



\item \textit{Course evaluations} completed online. 3 points extra credit for successful completion and screenshot of completed page sent through Blackboard. 

\item \textit{Writing center visit} with documented attendance submitted through Blackboard. 2 points extra credit per visit. Maximum of 4 points.  


\item \textit{Grade Distribution:}  Participation--0.5 point per class (max 10 points total); Library assignment--5 points; Textbook questions--10 points each (40 total); Capstone project--4 points for the proposal, 7 points for the annotated bibliography, 7 points for the literature review, 7 points for the presentation, 20 points for the final submission (45 points total).

\item \textit{Grade Breakdown:}

 \begin{tabular}{ | l | l | p{2cm} | l | l | }
    \hline 
96--100 & A  & &  77--79 &  C+ \\  
90--95 & A- & &  73--76 & C \\
87-89 & B+ &  &  70--72 & C- \\ 
83--86 & B  & &  60--69 & D\\
80--82 & B- & & 0--59 & F\\ \hline
    \end{tabular}


\end{itemize}





\section{Policies}


\begin{itemize}

\item \textbf{Student Responsibility:} This syllabus outlines the required text, assignments, requirements, and policies for this course. By taking this course, you agree to read this syllabus and be bound by those requirements and policies. 

 \item \textit{Academic Integrity:} All the work you turn in (including papers, drafts, and discussion board posts) must be written by you specifically for this course. It must originate with you in form and content with all contributory sources fully and specifically acknowledged. Being a student at NJCU requires you to follow \href{http://scottoconnor.org/resources/Plagiarism.pdf}{NJCU's Academic Integrity Policy.} Penalties for violations are as follows: 1st infraction will result in a 0 for the assignment.  2nd infraction will result in a 0 for the entire course \& application for permanent record on student's transcript. (Repeated violations can lead to expulsion from NJCU). 


\item \textit{Attendance:} You are considered absent if you are (i) not present during roll call, (ii) leave early, (iii) leave without permission, or (iv) leave for an extended period of time. No excuses. No exceptions.



\item \textit{Communication:} To comply with Federal Privacy Laws (FERPA) and NJCU policies, all communication will be through Blackboard, which you should check daily. I respond to messages within 48 hours; do not expect a quicker response. For further information see \href{http://scottoconnor.org/contact/}{http://scottoconnor.org/contact/}.

\item \textit{Conduct:} Distracting and disrespectful behaviors, including but not limited to eating, leaving your seat, talking out of turn, and aggression are prohibited. Penalties include, but are not limited to, a loss of attendance points for the day of violation. Repeat offenders will be reported to the Dean of Students. 

\item \textit{Electronic devices:} Use of electronic device, including, but not limited, to smartphones, dictaphones, tablets, and laptops, is prohibited. Recording a lecture is in violation of Copyright. Penalties include, but are not limited to, a loss of attendance points for the day of violation. Repeat offenders will be reported to the Dean of Students.


\item \textit{Format for Written Work:} Submit work to Blackboard as either a MS Word file or pdf. Blackboard will not allow any other format. All work must be typed and neatly presented. Note that NJCU students have free access to MS Word through Office 365 Online.


\item \textit{General Education Program Assessment:} General Education courses participate in programmatic assessment of the six University-wide student learning goals. They include instruction in, and assessment of, at least two of these learning goals. Signature assignments, which may include document, picture, sound, or video files, are scored using approved program rubrics. Results are aggregated to provide information about the Gen Ed program as a whole. Your name will not be included in any programmatic assessment data.

\item \textit{Grading:} Grades will be available within 1--2 weeks of an assignment being submitted. See: \href{http://scottoconnor.org/resources/grading}{http://scottoconnor.org/resources/grading} for further information.


\item \textit{Late work \& Make-up Policy:} Any work submitted past the due date will be incur the following penalty: assignment grade reduced by 20\% for each day late. The capstone project will not be accepted after May 14th.  No exceptions under any imaginable circumstances.

\item \textit{Statement for students with disabilities:} If you are a student
with a disability and wish to receive consideration for reasonable
accommodations, please register with the Office of Specialized Services
and Supplemental Instruction (OSS/SI). To begin this process, complete
the registration form available on the OSS/SI website at
\href{http://www.njcu.edu/oss}{http://www.njcu.edu/oss}
(listed under Student Resources-Forms). Contact OSS/SI at 201-200-2091
or visit the office in Karnoutsos Hall, Room 102 for additional
information.

\item \textit{Turnitin:} Students agree that by taking this course all assignments are subject to submission for textual similarity review to Turnitin.com. Assignments submitted to Turnitin.com will be included as source documents in Turnitin.com's restricted access database solely for the purpose of detecting plagiarism in such documents.  The terms that apply to the University’s use of the Turnitin.com service are described on the Turnitin.com web site.  For further information about Turnitin, please visit: http://www.turnitin.com 

\item \textit{SafeAssign:} Students agree that by taking this course all assignments are subject to submission for textual similarity review through Blackboard SafeAssign. Assignments submitted to SafeAssign will be included as source documents in SafeAssign's restricted access database solely for the purpose of detecting plagiarism in such documents.  


\end{itemize}




\section{Weekly Course Schedule}
Dates refer to the first day of the week. Readings marked with a `**' can be found on the course website. All other listed readings can be found in the required textbook. Changes to the syllabus will be announced through Blackboard.  All assignments must be submitted through Blackboard on the first day of the week at midnight, e.g., the library assignmentis due on 2/11/19 at 12:00pm. \emph{Midnight is the first moment of the day, not the last.}   \newline


\begin{center}
\begin{longtable}{p{4.5cm}p{2.5cm}p{5cm}}
 
  \caption{Course Schedule} \\
  \toprule
  \textbf{Week} &\textbf{Assignments} & \textbf{Reading} \\
  \midrule


[1] Introduction		  		& 	& --**\emph{Heavenly Clockwork}, Falk	\\
(1/22/19)					&	&    \\ [1.8\baselineskip]

[2.] The Calendar			& 				& --**\emph{Years, Months, Days}, Falk \\
(1/28/19)			        		& 				&   \\
\textbf{Wed: meet in library} &  & \\  [1.8\baselineskip]

% have students research the clock, calendar, or time zone issue somehow

[3.] The Clock   	&   	  	&  		--\emph{**Hours, Minutes, Seconds',} Falk  \\
(2/4/19)		 &						&    \\  [1.8\baselineskip]


% Session 1: Clocks
% Session 2: Time-zones
% Geography of Time
% \item **`Time, Work-Discipline, and Industrial Capitalism,' E.P. Thompson
% \item **`Aristotle on Time,' Tony Roark**, ch.2.1

[4.] What do clocks measure?	& Lib. assign. due		& --TRAVELS, Ch. 1\\
(2/11/19)				        	& 	& --\emph{**`The Mystery of Time',} O.K. Bouwsma, pp.341--347 \\ [1.8\baselineskip]

[5.] Time and Change	 	& Ch.1 qs. due	& --TRAVELS, Ch. 2\\
(2/18/19)				  	 &  	& --\emph{Isaac's Time}, Falk pp. 126--134 \\ 
\textbf{Mon: no class}		& 	&	\\[1.8\baselineskip]

%--\emph{**`Time Without Change}, S. Shoemaker\\

[6.] Time's beginning and end	 &  Ch.2 qs. due 	&--TRAVELS, Ch 5  \\
(2/25/19)			        	&	&  --**\emph{In The Beginning}, Falk\\ [1.8\baselineskip]


[7.] Does Time Pass?		& Ch.5 qs. due		& --TRAVELS, Ch. 8  \\ 
(3/4/19)			        		&		    	  					&  --**`Slaughterhouse-Five,' K. Vonnegut, ch.2--5 \\  [1.8\baselineskip] 

\textbf{Spring Break}		    		& 		  						&  \\
(3/11/19)			        		&		    		  				&  \\ [1.8\baselineskip]

[8.]   	The Cinematic Universe 	 & Ch.8 qs. due	   		&  --TRAVELS, Ch. 9\\
(3/22/19)				         &  Proposal due			& --**`A Survey of Metaphysics', E.J. Lowe, ch.16  \\[1.8\baselineskip]	

[9.] Special Relativity 		& Ch.9 qs. due		& --**Albert's Time, Falk \\
(3/25/19)				      	&		 & --Interstellar (movie)  \\  [1.8\baselineskip]

[10.] Time Travel			& An. Bib. due			& --TRAVELS, Ch. 10  \\
(4/1/19)		            		&		      & --**`All You Zombies,'  R.Heinlein \\ 
						&			& --**`The Paradoxes of Time Travel', David Lewis \\ [1.8\baselineskip]

%\item **`Interfering with History', Robin Le Poidevin 

[11.] Time's Arrows			& Ch.10 qs. due		& --TRAVELS, Ch. 12 \\
(4/8/19)		 		 &		    	&  \\ [1.8\baselineskip]




[12] Continued		 	& Lit. Review due	& --**`The Arrow of Time', B. Dowden\\
(4/15/19)				&  	 & \\
\textbf{Wed: no class}	&	& \\ [1.8\baselineskip]

[13.] Our Perception of Time 	& Ch.12 qs. due	& --TBD\\
(4/22/19)			      		& 				&  \\
\textbf{Wed: GenED Symp.}	&				&		\\[1.8\baselineskip]

%A Geography of Time', ch.2
%**`Confessions' (extracts),  St. Augustine
%\item `The Principles of Psychology' (extracts), William James
% Falk on mental time travel

[14.] Continued 		& Presentations 			&  \\ 
(4/29/19)				   &	Wed	   	&  \\ [1.8\baselineskip]

						
[15.] 	No new material    		& 		Presentations	&   \\
(5/6/19)				      	& 			 &  \\ 
\textbf{Mon: final class}		&			&  \\ [1.8\baselineskip]

[16.] 	No new material	 	      		&  Capstone due				&   \\
(5/13/19)				      	&			      	&  \\  [1.8\baselineskip]



\end{longtable}
\end{center}





%% Uncomment if you want a printed bibliography.
%\printbibliography 

\end{document}
