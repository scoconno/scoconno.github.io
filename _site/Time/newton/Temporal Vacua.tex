\documentclass[oneside]{article}
 \headheight = 25pt
\footskip = 20pt
\usepackage{mdwlist}
\usepackage[T1]{fontenc}
\renewcommand{\rmdefault}{ppl}
\usepackage{fancyhdr}
 \pagestyle{fancy}
 \lhead{\textbf{\textsc{\small Scott O'Connor\\Time}}}
 \chead{}
 \rhead{\large\textbf{\textsc{Time and Change}}}
 \lfoot{\footnotesize{\thepage}}
 \cfoot{}
 \rfoot{}
\tolerance=700


\begin{document}
\thispagestyle{fancy}

\section*{A Spatial Introduction}
We don't see space. We see objects that stand in spatial relations to one another. One object is 10 feet from me. Another is 20 feet from me. We also don't see time. We see water flowing faster than sand falling. We see a person walking slower than a cheetah running. Do time and space exist independently of those things that we can observe? Our focus is time, but I will introduce our topic by first discussing space. 

\begin{description}
\item[\textbf{Absolutism}:]
Space exists as an independent object in its own right over and above
the material content of the universe. Space is a continuous and
pervasive media that extends everywhere.
\item[\textbf{Relationalism:}]
Space does not exist as an independent object. There is only the
material content of the universe and the relations objects stand in to
one another. Space is merely defined through spatial relations among the
material objects in the universe.
\end{description}




\subsection*{Absolutism}
Absolutism about space is the view that space exists independently of anything that exists in space. 
\begin{quote}
Space is eternal in duration and immutable in nature\ldots{} Although
space may be empty of body, nevertheless it is itself not a void: and
something is there, because spaces are there, although nothing more than
that (\emph{De Gravitatione}, Newton, as quoted by Dainton, 133).
\end{quote}

\begin{quote}
Absolute space, in its own nature, without relation to anything
external, remains always similar and immovable (\emph{Principia}, Newton as quoted
by Huggett, 118).
\end{quote}
The Absolutist claims that there are absolute spatial relations like \emph{located at} between objects and places in space, e.g., Socrates has an absolute location by standing in the relation of being located at some region of space. Physical objects stands in absolute spatial relations. Suppose that X and Y are 10 feet apart. X is located in the region R1 of space that contains it. Y is located in the region R2 of space that contains it. Absolutism says that between R1 and R2 there exists another thing, a 10 foot region of space that separates X and Y.

The Absolutist about time claims that time exists independently of anything that happens in or over time. Newton describes his view as follows:  

\begin{quote}
Absolute, true and mathematical time, of itself, and from its own nature flows equably without regard to anything external, and by another name is called duration: relative, apparent and common time, is some sensible and external (whether accurate or unequable) measure of duration by the means of motion, which is commonly used instead of true time. Newton
\end{quote}
Absolutists believe that there are absolute times and durations .Suppose that X and Y are one hour apart. X happens at t1 and Y happens at t2. Absolutism says that between t1 and t2 there exists another thing, a one hour duration of absolute time. 

Absolutism about space entails the possiblity of \textbf{spatial vacua}, i.e., empty spaces. It allows for both absolute motion and relative motion.  Absolutism about time entails the possibility of \textbf{temporal vacua}, i.e., intervals of time during which no change or process occurs. It allows for both absolute and relative durations of time. If spatial vacua are impossible, Absolutism about space is false. If temporal vacua are impossible, Absolutism about time is false. 







\section*{Absolutism}

We will now examine arguments for and against the Absolutism. The first are a series of argument that show that Absolutism is best defended by the claim that there are temporal vacua. We will then argue that there can be no such things. All of these are taken from the textbook.


\subsubsection*{Tme cannot flow at different rates}
\begin{enumerate*}
\item Change can go at different rates, speed up, slow down, etc. 
\item If time is a change, then time can speed up, slow down, etc. 
\item We measure how fast things move against time:  rate of change is variation in some dimension in so many units of time.
\item If time can speed up or slow down, we measure how fast it changes against time. 
\item But we cannot measure how fast time changes against time, e.g. five minutes will always last five minutes.  
\item So time is not a change (from 1-5).
\end{enumerate*}
\subsubsection*{Response}
\begin{enumerate*}
\item Particular changes vs the sum of all change. 
\item The argument shows that time is distinct from each particular change. It does not show that it is distinct from the sum of all changes. 
\begin{enumerate*}

\item If the sum of all changes could go at different rates, the sum could double its speed. 
\item But the sum of all changes cannot double its speed.
\item The sum of all changes cannot go at different rates (from (a) and (b)).  
\end{enumerate*}
\item So the rate of change argument does not show that time is distinct from the sum of all change.
\end{enumerate*}

\subsubsection*{Argument for temporal vacua}

Suppose that all processes were to come to an end. It seems that we can make sense of the idea of time existing even though everything has stopped. For instance, we could imagine a difference between everything having stopped for 1 year as opposed to 10 years. If we can measure how long all changes have stopped, then time can exist independently from the sum of all change. Therefore, it is different from the sum of all change.

\subsection*{Arguments against temporal vacua}

\subsubsection*{Time and experience}
\begin{enumerate*}
\item To notice anything is to undergo a change in mental state.
\item The cessation of all change is the cessation of any experience. 
\item So it is impossible to experience a temporal vacuum (in the sense of experiencing anything as a temporal vacuum) (from 1-2). 
\item So it is impossible to imagine the sum of all change stopping and time continuing to exist (from 3). 
\end{enumerate*}

\subsubsection*{Time and measure}
\begin{enumerate*}
\item Periods of time are measured by changes. 
\item Since, by definition, nothing happens in a temporal vacuum, there is no possible means of determining its length. 
\item If there is no means of determining the length of a temporal interval, it has no specific length. 
\item Every interval of time has a specific length. 
\item There cannot be a temporal vacuum (from 1-4). 

\end{enumerate*}


\subsubsection*{The Principle of Sufficient Reason}
\begin{description}
\item[Principle of Sufficient Reason:] For everything that occurs at a given moment, there is always an explanation of why it occurred at precisely \emph{that} moment and not at some other moment.
\end{description} 

\begin{enumerate*}
\item If there have been temporal vacua in the past, then there have been times when change has resumed after a period of no change. 
\item For every change that occurs at a given moment, there is always an explanation, in terms of an immediately preceding change, of why it occurred at precisely that moment and not at some other moment. 
\item There is no explanation of why a change occuring immediately after a temporal vacuum occurred when it did (since no change immediately preceded it). 
\item There have been no temporal vacua in the past. 
\end{enumerate*}


\section*{Relationalism}
Relationalism about space is the view that spatial relations exist but space itself
does not. Relationalism about time is the view that temporal relations exist but time itself does not. Space first: 

\begin{quote}
{[}T{]}he Mind can fancy to itself an Order made up of Genealogical
Lines, whose Bigness would consist only in the Number of Generations,
wherein every Person would have his Place. (The Leibniz-Clarke
Correspondence, (ed.) Alexander 1956, 70)
\end{quote}

\begin{quote}
and if to this one should add the Fiction of a Metempsychosis, and bring
in the same Human Souls again; the Persons in those Lines might change
place; he who was a Father, or a Grandfather, might become a Son, or a
Grandson etc. (Ibid. 70-1)
\end{quote}

\begin{quote}
And yet those Genealogical Places, Lines, and Spaces, though they should
express real Truths, would only be Ideal Things. (Ibid. 71)
\end{quote}
Some of the most important claims of Relationalism:
\begin{enumerate}
\item  If X and Y are 10 feet apart and there is nothing in between them, then there does not also exist the regions of space that X and Y occupy as well as the region of space separating X and Y.
\item Spatial relations do exist, e.g., X and Y stand in the relation of being 10 feet apart.
\end{enumerate}
How might physical objects stand in spatial relations if space itself does not exist? Leibniz uses families as an example. A family is simply a group of people standing in certain relations to each other: daughter, uncle, cousin, etc. The relations do exist. They are something additional to the collection of individual people in the family. How about the family? Is the family something in addition to these individuals and the relations they stand in to one another? Is it an
object in its own right that contains people, grows over time, etc? Leibniz says no.

Consider the sentence `my family has lived in Wexford for two
centuries'. The expression `the family' does not pick out some person
who has been alive for 200 years. We use the phrase to refer to a group
of people who stand in familial relations. Similarly, Leibniz thinks
that space is not a real object above and beyond the objects that it
supposedly contains. There are just the various spatial relations that
objects stand in to one another.

\subsubsection*{What might a relationalist about time say?}


\end{document}

