\documentclass[10pt, oneside]{article}
 \headheight = 25pt
\usepackage[T1]{fontenc}

\usepackage{fancyhdr}
 \pagestyle{fancy}
 \lhead{\textbf{\textsc{\small Time\\Scott O'Connor}}}
 \chead{}
 \rhead{\large\textbf{\textsc{Editing}}}
 \lfoot{\footnotesize{\thepage}}
 \cfoot{}
 \rfoot{\footnotesize {\today}}
\tolerance=700

\begin{document}
\thispagestyle{fancy}

\section{Actions and Characters}

\subsection{Two Principles of Clarity}
\begin{enumerate}
\item Make main characters subjects.
\item Make important actions verbs.\\
\begin{enumerate}
\item Once upon a time, as a \textbf{\underline{walk}} through the woods \textsc{was taking} place on the part of \emph{Little Red Riding Hood}, \emph{the Wolf's} \textbf{\underline{jump}} out from behind a tree \textsc{occurred}, causing \emph{her} \underline{fright}.\\
\item Once upon a time, \textbf{\emph{Little Red Riding Hood}} \textsc{\underline{was walking}} through the woods, when \textbf{\emph{the Wolf}} \textsc{\underline{jumped}} out from behind a tree and \textsc{\underline{frightened}} \emph{her}.\footnote{\emph{characters.} \textbf{subjects.} \textsc{verbs.} \underline{actions.}} 
\end{enumerate}
\end{enumerate}

\subsection{Nominalization}
\noindent Your writing will be clearer if you avoid using nominalizations as the subjects of your verbs. I include here some examples of verbs and how to nominalize them (nom. appear on the right hand side of each pair).
\begin{center}
\begin{tabular}[t]{|c|c|c|c|c|}
\multicolumn{5}{c}{{}}\\\hline
perceive&perception&{}&careless&carelessness\\
resist&resistance&{}&different&difference\\
react&reaction&{}&proficient&proficiency\\
She flies&her flying&{}&We sang&our singing\\\hline
\end{tabular}
\end{center}
\vspace{1mm}

\subsection{Revise: Clear to Unclear} Turn these clear sentences into unclear sentences by ensure that the characters are not the subjects and that the important actions are not the verbs. Use as many words and clauses as you like, but try to preserve what the unrevised sentences mean. I suggest one way of revising the first sentence (please suggest another).\\
\begin{enumerate}
\item{}
\begin{enumerate}
\item John loves Mary.
\begin{enumerate}
\item The love towards the woman named Mary is occasioned in John.
\vspace{20mm}
\end{enumerate}
\item John questions whether tables exist.
\vspace{20mm}
\item John says that tables do not exist.
\vspace{20mm}
\item Mary compares the way that atoms compose humans with the way that atoms compose tables. \vspace{20mm}
\item Unger concludes that we mistake what John feels for Mary. Neither John or Mary exists, and you cannot love what you cannot hold.
\end{enumerate}

\end{enumerate}

\subsection{Revise: Unclear to clear} Apply our two principles:
\begin{enumerate}
\item{}
\begin{enumerate}
\item To advance discussion, it may be helpful if I give the argument just presented something like a formal shape or presentation.
\vspace{20mm}
\item Further, if an occasion arises where, vary conditions as we may, a single atom cannot be removed without substantial relevant disruption, then we remove as few as possible, balanced against a disruptive effect.\vspace{20mm}
\end{enumerate}
\end{enumerate}

\subsection{Concision} 
\begin{verse}
 I think writing is sheer paring away of oneself leaving always something thinner, barer, more meager.
F. Scott Fitzgerald
\end{verse}


\noindent Wordiness is the enemy of clarity. When re-reading your work, remove unnecessary words. When possible, re-write each sentence so that is has fewer words; using concrete subjects and verbs will help. Here is an example: 

\begin{description} 
\item[Unclear:] In my \underline{personal} \underline{opinion}, it is \textbf{necessary} that we \textbf{should} \emph{not ignore} the opportunity \emph{to think over} \textsc{each and every} \textbf{\underline{suggestion offered}}.
\item[Clear:] We should consider each suggestion.
\end{description}

Here are six principles of concision: 

\begin{enumerate}
\item Delete words that mean little or nothing.
\item Delete words that repeat the meaning of other words. 
\item Delete words implied by other words.
\item  Replace a phrase with a word.
\item Change negatives to affirmatives.
\item Delete useless adjectives and adverbs.
\end{enumerate}

 Revise the following sentences so that the fewest words possible are used to convey exactly the same meaning. The sentences must be grammatical: 

\begin{enumerate}
\item It is my belief that in regard to terrestrial type snakes, an assumption can be made that there are probably none in unmapped areas of the world surpassing the size of those we already have knowledge of. \vspace{20mm}


\item Whether the date an operation intends to close down might be part of management's ``duty to disclose'' during contract bargaining is the issue here, it would appear. The minimization of conflict is the central rationale for the duty that management has to bargain in good faith. In order to allow the union to put forth proposal on behalf of its members, companies are obligated to disclose major changes in an operation during bargaining, though the case law is scanty on this matter. \vspace{20mm}

\end{enumerate}









\end{document}



