\documentclass[]{article}

\usepackage{fancyhdr}
 \pagestyle{fancy}
\rhead{\textsc{Scott O`Connor}}

\usepackage{lmodern}
\usepackage{amssymb,amsmath}
\usepackage{ifxetex,ifluatex}
\usepackage{fixltx2e} % provides \textsubscript
\ifnum 0\ifxetex 1\fi\ifluatex 1\fi=0 % if pdftex
  \usepackage[T1]{fontenc}
  \usepackage[utf8]{inputenc}
\else % if luatex or xelatex
  \ifxetex
    \usepackage{mathspec}
    \usepackage{xltxtra,xunicode}
  \else
    \usepackage{fontspec}
  \fi
  \defaultfontfeatures{Mapping=tex-text,Scale=MatchLowercase}
  \newcommand{\euro}{€}
\fi
% use upquote if available, for straight quotes in verbatim environments
\IfFileExists{upquote.sty}{\usepackage{upquote}}{}
% use microtype if available
\IfFileExists{microtype.sty}{%
\usepackage{microtype}
\UseMicrotypeSet[protrusion]{basicmath} % disable protrusion for tt fonts
}{}
\ifxetex
  \usepackage[setpagesize=false, % page size defined by xetex
              unicode=false, % unicode breaks when used with xetex
              xetex]{hyperref}
\else
  \usepackage[unicode=true]{hyperref}
\fi
\usepackage[usenames,dvipsnames]{color}
\hypersetup{breaklinks=true,
            bookmarks=true,
            pdfauthor={},
            pdftitle={Research Question},
            colorlinks=true,
            citecolor=blue,
            urlcolor=blue,
            linkcolor=magenta,
            pdfborder={0 0 0}}
\urlstyle{same}  % don't use monospace font for urls
\setlength{\parindent}{0pt}
\setlength{\parskip}{6pt plus 2pt minus 1pt}
\setlength{\emergencystretch}{3em}  % prevent overfull lines
\providecommand{\tightlist}{%
  \setlength{\itemsep}{0pt}\setlength{\parskip}{0pt}}
\setcounter{secnumdepth}{0}

\title{Research Question}
\author{Scott O’Connor}


% Redefines (sub)paragraphs to behave more like sections
\ifx\paragraph\undefined\else
\let\oldparagraph\paragraph
\renewcommand{\paragraph}[1]{\oldparagraph{#1}\mbox{}}
\fi
\ifx\subparagraph\undefined\else
\let\oldsubparagraph\subparagraph
\renewcommand{\subparagraph}[1]{\oldsubparagraph{#1}\mbox{}}
\fi

\begin{document}
%\maketitle

%\hypertarget{introduction}{%
%\subsection{Introduction}\label{introduction}}

Our goal is to develop the traits of independent researchers. To that
end, you will complete an independent project over the remainder of the
semester. We will scaffold that project around a few basic skills and
tasks to practice those skills (see syllabus for relevant dates).

\begin{enumerate}
\def\labelenumi{\arabic{enumi}.}
\tightlist
\item
  A manageable and interesting research proposal.
\item
  Annotated bibliography.
\item
  Literature review.
\item
  Oral presentation.
\item
  Final draft.
\end{enumerate}


\hypertarget{research-proposalsubmit}{%
\subsection{Research Proposal--SUBMIT}\label{research-proposalsubmit}}

In this first part of the project, you will develop a research proposal.
To do that, complete the following three tasks (see below for further information):

\begin{enumerate}
\def\labelenumi{\arabic{enumi}.}
\tightlist
\item
  State a manageable research question about time. You must use an
  interrogative. There should be a question mark.
\item
  Write a 150--250 words to provide background to your question.
\item
  Write the references for 3 peer reviewed sources that deals with your
  topic.
\end{enumerate}

\textbf{NB: If I do not approve your proposal, you will need to revise
it before moving on to further parts of the capstone.}


\hypertarget{task-1}{%
\subsubsection{Task 1}\label{task-1}}

You must clearly articulate some \textbf{manageable question} about
\textbf{time} you wish to answer this semester. There are three things I
want to emphasize. First, I mean \textbf{question}. Questions have question marks and are formed with an
interrogatives like the following:

\begin{itemize}
\tightlist
\item
  \emph{which}, \emph{what}, 
  \emph{how},  \emph{why},   \emph{who}, \emph{whom}, \emph{whose}, 
  \emph{when},  \emph{when},  \emph{does}
\end{itemize}

So \emph{what} question about time would you like to answer this
semester?

Second, you must pick a \textbf{manageable} question. Research projects
vary in scope and ambition. Some ask narrowly defined questions that can
be answered in a few months. Others ask large questions that take years
to address. Compare the following:

\begin{enumerate}
\def\labelenumi{\arabic{enumi}.}
\tightlist
\item
  How do humans measure time?\\
\item
  Why is cesium a good element to use in an atomic clock?
\end{enumerate}

Our first question would require you to examine every clock humans have
created for measuring time. It would require you to give, in effect, a
full history of time measurement. That's a huge project, and it would
take an entire book to complete it successfully. The second question is
manageable and could be completed in a couple of months. It requires you
to discuss what it means for a clock to be accurate, explain how atomic
clocks work, and discuss the features of cesium that makes it a good
choice for such a clock. It doesn't require a full discussion of cesium,
but it requires that you say enough to answer the question. The second would be a
manageable capstone project. The first would not.

Finally, it is essential that your question is about
\textbf{time}. Some reference to time in the question does not suffice.
Compare the following:

\begin{enumerate}
\def\labelenumi{\arabic{enumi}.}
\tightlist
\item
  Can everyone run a four minute mile?
\item
  How did changes in time-keeping affect the sport of track and field?
\end{enumerate}

The first question mentions time, but it's hardly a project about time.
The second question is better. It is a question that requires some
investigation into different clocks, into how humans used them in a
sporting competition, and into how and why the development of clocks
shaped that sport. If uncertain if your question is about time, ask
yourself whether your project will require you to research and write
about either the nature and existence of time, or the nature of
time-measurement. Of course, your project will likely involve something
in addition to these things. But if you could complete your project
without writing in anyway about these things, then the project is
clearly not about time.

\hypertarget{task-2}{%
\subsubsection{Task 2}\label{task-2}}

The second task asks you to provide some context and background to your
question, to provide sufficient details to your reader to
help them see why your question is an interesting one. Or to put it
differently, it provides enough detail to explain why your question does
not have an obvious answer and is something that requires further
research and study. Suppose you encounter this question:

\begin{itemize}
\tightlist
\item
  Does Socrates claim that a person can be wise if they know nothing
  whatsoever?
\end{itemize}

Suppose you have never heard of Socrates. You wouldn't then have sufficient background information to understand the general area the question concerned. You definitely wouldn't know enough to care about answering
that question. Now suppose that the author prefaced their question with
the following:

\begin{quote}
Socrates, an ancient Greek philosopher, is famous for claiming that he
is wise because he does not take himself to know what he does not know.
But he doesn't explain what wisdom amounts to. The very little he says
seems to leave open the possibility that a person who knew nothing at
all could be wise. As long as they didn't claim to have knowledge, they
would be wise. But lots of thing satisfy this description. My cat does.
She doesn't claim to have knowledge. Does that make her wise? It does
from a simple reading of Socrates' claim. So, we want to take a closer
look at Socrates' account of wisdom. Perhaps he believes that the wise
person is not merely silent about their lack of knowledge but is also
aware of that lack. That kind of awareness is not something my cat
possesses. But if this isn't the relevant addition, is there anything
else to Socrates' account of wisdom that would avoid the conclusion that
my cat is wise?
\end{quote}

\hypertarget{task-3}{%
\subsubsection{Task 3}\label{task-3}}

The third task asks you to identify three peer reviewed sources for your
paper. Remember, a peer reviewed source is an academic source. You can't
use newspapers, or blogs, etc. This task takes time and effort to complete.
You may need help from a librarian. In future assignments, you will read
these sources and write about them. If you don't use peer reviewed
sources, you will unlikely pass the capstone and this course.

\end{document}
