\documentclass[oneside, 11pt]{article}
 \headheight = 25pt
\footskip = 20pt
\usepackage[T1]{fontenc}
\renewcommand{\rmdefault}{ppl}
\usepackage{fancyhdr}
 \pagestyle{fancy}
 \lhead{\textbf{\textsc{\small Scott O'Connor\\The Philosophy of Time (Phil1111)}}}
 \chead{}
 \rhead{\LARGE\textbf{\textsc{Syllabus}}}
\tolerance=700
\usepackage{mdwlist}
\usepackage{multicol}

\begin{document}
\thispagestyle{fancy}

\begin{multicols}{2}
\begin{description}\addtolength{\itemsep}{-0.5\baselineskip}
\item[Instructor:] Dr. Scott O'Connor 
\item[Office:] FA 523
\columnbreak

\item[Contact:] scott.oconnor@umbc.edu
\item[Office Hours:] Tues., 11.30--12.30
\end{description}
\end{multicols}

\section*{Course Description}
\noindent 
We are temporal creatures. We structure our lives around a sense of our history, of what we were in the past, and what we hope to be in the future. Our societies are also essentially temporal. Political, economic, and cultural institutions value certain units of time over others. Work weeks must not exceed a certain length. Elections must happen on certain fixed dates. The measuring of time is both essential to us and the societies we live in. But what exactly is it that we are measuring?

  In this course, we will examine the nature of time and the role it plays in our lives and societies more generally. We begin by discussing the history of time-measurement and various struggles peoples have encountered in measuring it. From there we move to discuss how these struggles have shaped, often profoundly, the development of different societies and cultures. Next we will  discuss  time and the person specifically focusing on two questions. The first is how we could survive as the same person as we change over time. The second is how we perceive and understand time. If we want to understand the importance of time to us and our societies, we need to know what it is. To discuss this, we'll look at two distinct questions. The first is the nature of temporal passage and arguments that suggest that both it and time may, in fact, not exist at all.  The second is about the direction of time. Why is it that time unfolds in the particular direction that it does? Could it flow backwards? We'll conclude by thinking about the future prospects of time. In particular, we will look at the possibility of time-travel and how its advent might radically re-shape our conceptions of ourselves and the societies we live in. 

Readings for this course will be drawn from a variety of disciplines: History, Classics, Sociology, English Literature, Philosophy, Physics, and Psychology.  



\section*{Learning Objectives}

Upon completing this course, students will be able to (i) find the argument of a text and clearly restate it, (ii) clearly and charitably explain viewpoints that are not their own, (iii) think critically, (iv) write well-structured prose in which they clearly state a thesis and persuasively defend it, (v) demonstrate a substantive grasp of how time is an  important topic of study in several distinct disciplines. 

\section*{Required Texts/Media} The following are available from the Campus Store:
\begin{enumerate*}
\item Robin LePoidevin, `Travels in Four-Dimensions: The Enigmas of Space and Time', OUP, 2005 (TRAVELS)
\item Course Packet.
\item Clickers.
\end{enumerate*}

\section*{Course Requirements}

\begin{description}
\item[Writing Assignments:] 4 papers will be assigned throughout the semester varying from 200--800 words in length. The lowest grade of the four papers shall be dropped. Topics include:
\begin{enumerate*}
\item A journal entry in the diary of an Athenian youth c.300 B.C. describing how she structures his day using a water clock.
\item A transcript of a televised debate about whether humans can survive the passage of time (composed with a peer).
\item A Travel Guide describing the human conception of time to an alien race (the Tralfamadorians) who have a radically different conception of time. The guide will also explain the Tralfamadorians' conception of time to humans.
\item A letter written from a future astronaut explaining the physics of the Twin Paradox to her child who will, bizarrely, be the same age as his mother when she returns to Earth.
\end{enumerate*}

\item[Student Conferences:] You must submit a draft of one paper to discuss at a student-teacher conferences. You can schedule this anytime throughout the semester, but you must give me at least one week notice. 
\item[Pop Quizzes:] There will be regular pop quizzes throughout the semester. The quizzes focus on the key terms and questions for whatever reading  we are covering that lecture. No extra preparation is required. Just complete the assignments diligently.  Results will be posted on the Blackboard site. 
\item [Exams:] There are two exams for this course. Further instruction will be given in class.
\item[Grade:] Mid-term, 15\%, Final, 25\%, Pop-Quizzes 20\%, 4 papers, 30\%, Participation, 10\%. 
\end{description} 

\section*{Policies}

\begin{description}
\item[Conduct:] In this course, you will develop your speaking and debating skills. In part, this will be through interaction with both me and your peers. You will be cold-called and often challenged to both clarify and defend your views. Please be cooperative and friendly.


\item[Technology/Mindfulness:] You will also develop your abilities to attend carefully to new and abstract information. Using technology can be a distraction both to yourself and to your peers. So, unless you are given explicit permission otherwise, please turn off laptops, phones, and so on. 

\item[Communication:] I will do my best to respond to emails within two days of receiving them. This will be from 9am-5pm on Mon. through Fri. 
\item[Academic Integrity:] By enrolling in this course, each student assumes the responsibilities of an active participant in UMBC's scholarly community in which everyone's academic work and behavior are held to the highest standards of honesty. Cheating, fabrication, plagiarism, and helping others to commit these acts are all forms of academic dishonesty, and they are wrong. Academic misconduct could result in disciplinary action that may include, but is not limited to, suspension or expulsion. To read the full Student Academic Conduct Policy, consult UMBC policies, or the Faculty Handbook (Section 14.3).

%\item[Turnitin:] By taking this course, students agree that all required papers may be subject to submission for textual similarity review to Turnitin.com for the detection of plagiarism.  All submitted papers will be included as source documents in the Turnitin.com reference database solely for the purpose of detecting plagiarism of such papers.  Use of the Turnitin.com service is subject to the Usage Policy posted on the Turnitin.com site. These rules do not prohibit collaborative work with your peers. I encourage you to read one another's work, discuss it in detail, respond to comments, and so on.

\item[Statement for students with disabilities:] In compliance with the UMBC policy and equal access laws, I am available to discuss appropriate academic accommodations that may be required for students with disabilities. Requests for academic accommodations are to be made during the first three weeks of the semester, except in unusual circumstances, so that arrangements can be made. Students are encouraged to register with Student Disability Services to verify their eligibility for appropriate accommodations.

\end{description}


\subsection*{Tentative Schedule}
Listed here are the topics and readings for the semester. A more detailed calendar of readings and writing assignments will be given at the beginning of class. 

\subsubsection*{Topic 1: Clocks and Calendars}
\begin{enumerate*}
\item TRAVELS, Ch. 1
\item `Old Time and Ancient Chronometers,' by Tony Roark
\end{enumerate*}

\subsubsection*{Topic 2: Time and Society}
\begin{enumerate*}
\item `Time, Work-Discipline, and Industrial Capitalism,' by E.P. Thompson
\end{enumerate*}

\subsubsection*{Topic 3: Time and the Person}
\begin{enumerate*}
\item  `Identity Through Time,' Roderick Chisholm
\item  `Identity, Ostension, and Hypostasis,' W.V.O. Quine
\end{enumerate*}

%\subsubsection*{Writing Assignment}
%\noindent The year is 2041. Orla was elected in 2038, but citizens throughout the country claim that they did not vote for her. They say that no one who voted in 2038 is strictly identical to anyone who exists now. These dissenters are lead by Roderick - Orla's former campaign manager. Drastic times call for drastic measures. Orla challenges Roderick to a televised debate. She argues that she did persist through time while Roderick argued that persisting through time is impossible. 

%Write a transcript of that debate 

\subsubsection*{Topic 4: Our Perception of Time}
\begin{enumerate*}
\item `Confessions' (extracts),  St. Augustine
\item `The Principles of Psychology' (extracts), William James
\end{enumerate*}

\subsubsection*{Topic 5a: The Nature of Time---Its Passage}

\begin{enumerate*}
\item `Slaughterhouse-Five,' Kurt Vonnegut
\item TRAVELS, Ch. 8
\item  `Time,' J.M.E. McTaggart
\end{enumerate*}

%\subsubsection*{Writing Assignment}
%In ` describes an alien race called the Tralfamadorians who claim that they see all of time as a large landscape; just as you can see the lake in front of the mountain, they claim that the see `all together' every event laid out before and after each other. They struggle to explain this to the main character, Billy Pilgrim, and struggle to understand just how Billy experiences his time. 

%Imagine you are a tour guide introducing Tralfmadorians to Earth and introducing humans to Tralfmadore. You can imagine, if you like, that you do this on the space ship which carries them to their respective destinations. First, explain to the human visitors how the inhabitants understand and experience time. Then  explain to a Tralfmadorian visitor to Earth how the inhabitants there understand and experience time. 

\subsubsection*{Topic 5b: The Nature of Time---Its Direction}

\begin{enumerate*}
\item TRAVELS, Ch. 9
\item \emph{Physics}\ VI, Aristotle
\end{enumerate*}


\subsubsection*{Topic 6: Time-Travelers}

\begin{enumerate*}
\item `All You Zombies,' by Robert Heinlein
\item TRAVELS, Ch. 10 
\item `The Paradoxes of Time Travel,' by David Lewis
\end{enumerate*}


\end{document}