\documentclass[article,oneside]{memoir}
\usepackage{longtable}
%%% custom style file with standard settings for xelatex and biblatex. Note that when [minion] is present, we assume you have minion pro installed for use with pdflatex.
%\usepackage[minion]{org-preamble-pdflatex} 

%%% alternatively, use xelatex instead
\usepackage{org-preamble-xelatex} 



\def\myauthor{Author}
\def\mytitle{Title}
\def\mycopyright{\myauthor}
\def\mykeywords{}
\def\mybibliostyle{plain}
\def\mybibliocommand{}
\def\mysubtitle{}
\def\myaffiliation{NJCU}
\def\myaddress{Phil 140}
\def\myemail{soconnor@njcu.edu}
\def\myweb{\href{http://scottoconnor.org/examined/}{http://scottoconnor.org/examined/}}
\def\myphone{}
\def\myversion{}
\def\myrevision{}
\def\myaffiliation{NJCU}
\def\myauthor{Dr. Scott O'Connor}
\def\mykeywords{}
\def\mysubtitle{Syllabus}
\def\mytitle{{\normalsize \myweb \newline} \HUGE The Examined Life}


\begin{document}

%%% If using xelatex and not pdflatex
%%% xelatex font choices
\defaultfontfeatures{}
\defaultfontfeatures{Scale=MatchLowercase}    
% You will need to buy these fonts, change the names to fonts you own, or comment out if not using xelatex.      
\setromanfont[Mapping=tex-text]{Minion Pro} 
\setsansfont[Mapping=tex-text]{Myriad Pro} 
\setmonofont[Mapping=tex-text,Scale=0.8]{Georgia} 

%% blank label items; hanging bibs for text
%% Custom hanging indent for vita items
\def\ind{\hangindent=1 true cm\hangafter=1 \noindent}
\def\labelitemi{$\cdot$}
%\renewcommand{\labelitemii}{~}

%% RCS info string for version tracking
\chapterstyle{article-3}  % alternative styles are defined in latex-custom-kjh/needs-memoir/
%\pagestyle{kjh}

\title{\LARGE \mytitle}     
\author{\Large\myauthor \newline \footnotesize\texttt{\noindent Office hours: \href{http://scottoconnor.org/contact/office}{http://scottoconnor.org/contact/office}}}
\date{9/4/2017--12/18/2018}

\published{\textbf{Phil 140-2 (2346), 3 credits, Fall 2018, M \& W 9:55am--11:10am, K237  }}

\maketitle

%\thispagestyle{kjhgit}

% Copyright Page
%\textcopyright{} \mycopyright


%
% Main Content
%

\section{Copyright}
The materials used in this class, including, but not limited to, lectures, exams, quizzes, and homework assignments are copyright protected works.  Any unauthorized copying of the class materials or recording of lectures is a violation of federal law and may result in disciplinary actions being taken against the student.  Additionally, the sharing of class materials without the specific, express approval of the instructor may be a violation of the University's Student Honor Code and an act of academic dishonesty, which could result in further disciplinary action.  This includes, among other things, uploading class materials to websites for the purpose of sharing those materials with other current or future students. 

\section{Catalog Description}

This course teaches students to identify and evaluate those beliefs that guide their thoughts and actions. Reflecting on different sources, students identify those philosophical beliefs that play a role in their own lives. By developing their critical thinking skills, they learn how to clarify, systematize, and assess these beliefs. 

\section{Course Description}

Does God exist? Are you free? Why live? What should you do with your life?  In this course, we'll be asking some of these deep philosophical questions. We begin by discussing the meaning of life, especially why some philosophers have connected a meaningful life with God's existence. This raises the question as to whether God exists. We will examine some classic arguments for the existence of God as well as concerns that God's existence is incompatible with the existence of evil. Many respond to the problem of evil by claiming that evil is a by-product of our free-will, a gift endowed by God to use as we see fit. But are we free? We will discuss why some think our actions are completely pre-determined by causal factors outside of our control. If they are right, free-will is a mere illusion. This raises deep questions about the nature of moral responsibility; can you be held responsible for an action that was out of your control? In the final part of the course, we will ask what determines the moral character of our actions. Do the ends justify the means? After studying the main ethical theories, you will get a chance to work in groups to apply them to a current controversy of your choosing, e.g., the death penalty, euthanasia, abortion, etc. 

\section{Learning Objectives}

Upon completing this course, students will be able to (i) read
philosophical texts, (ii) clearly and charitably explain viewpoints that
are not their own, (iii) think critically and philosophically, (iv)
write well-structured prose in which they clearly state a thesis and
persuasively defend it, (v) demonstrate an understanding of several core
philosophical topics, (vi) manage their studies in a responsible and timely manner. 

\section{General Education Information} 
Successfully completing this course satisfies one Tier 1 Language, Literary, and Cultural Studies requirement. It teaches the following two University-wide Learning Goals: (1) Critical Thinking and Problem Solving, (2) Written Communication. For further information about the General Education Program see \href{https://www.njcu.edu/department/general-education}{https://www.njcu.edu/department/general-education/}.


\section{Required Textbook}

\begin{itemize}
\item
  \href{http://www.amazon.com/Philosophy-Here-Now-Powerful-Everyday/dp/0190207035/ref=dp_ob_title_bk}{`Philosophy  Here and Now: Powerful Ideas in Everyday Life', 2nd Edition, by Lewis Vaughn}  (Available in the campus book store and online retailers)
\end{itemize}


\section{Course Website}
There is both a Blackboard site and website for this course (link on first page). Clicking the first link on the left panel within the Blackboard site will bring you to the course website. All assignments will be submitted through Blackboard. Readings, notes, etc. will be posted on the course website. Note that Blackboard difficulties are rare and automatically reported to instructors. Under no circumstance will a student's report of a Blackboard difficulty be reason for an extension. It is your responsibility to contact Blackboard support for help.




\section{Requirements}



\begin{itemize}
\item \textit{Workload:} Expect to spend an average of 6 hours per week completing the readings and assignments. NJCU abides by the Federal and State definitions of a credit hour and adopts a policy consistent with the Carnegie Unit. A three-credit class represents 112.5 hours total of work. See \href{http://scottoconnor.org/resources/Credit.pdf}{here} for more details.

\item \textit{Participation:} Roll call will be taken. 1 point will be awarded per class up to a maximum of 10 points. Points will not be awarded during weeks 1 \& 2. 

\item \textit{Reading quizzes} administered through Blackboard. 7 will be assigned. You must complete 4. If you complete more than 4, the lowest grades will be dropped. 


\item \textit{Essays (500--750 words)} submitted through Blackboard.  7 will be assigned. You must complete 5. If you complete more than 5, the lowest grades will be dropped. Your final submission will count as your signature assignment for the course. 

\item \textit{Course evaluations} completed online. 3 points extra credit for successful completion  and screenshot of completed page sent through Blackboard. 

\item \textit{Writing center visit} with documented attendance submitted through Blackboard. 2 points extra credit per visit. Maximum of 4 points.  
  
\item \textit{Grade distribution:} Participation--0.5 points per class (10 maximum); Quizzes---10 points each (40 total);  Essays---10 points each (50 total).


\item \textit{Grade Breakdown:}

 \begin{tabular}{ | l | l | p{2cm} | l | l | }
    \hline 
96--100 & A  & &  77--79 &  C+ \\  
90--95 & A- & &  73--76 & C \\
87-89 & B+ &  &  70--72 & C- \\ 
83--86 & B  & &  60--69 & D\\
80--82 & B - & & 0--59 & F\\ \hline
    \end{tabular}


\end{itemize}


\section{Policies}

\begin{itemize}

\item \textbf{Student Responsibility:} This syllabus outlines the required text, assignments, requirements, and policies for this course. By taking this course, you agree to read this syllabus and be bound by those requirements and policies. 

 \item \textit{Academic Integrity:} All the work you turn in (including papers, drafts, and discussion board posts) must be written by you specifically for this course. It must originate with you in form and content with all contributory sources fully and specifically acknowledged. Being a student at NJCU requires you to follow \href{http://scottoconnor.org/resources/Plagiarism.pdf}{NJCU's Academic Integrity Policy.} Penalties for violations are as follows: 1st infraction will result in a 0 for the assignment.  2nd infraction will result in a 0 for the entire course \& application for permanent record on student's transcript. (Repeated violations can lead to expulsion from NJCU). 


\item \textit{Attendance:} You are considered absent if you are (i) not present during roll call, (ii) leave early, (iii) leave without permission, or (iv) leave for an extended period of time. No excuses. No exceptions.



\item \textit{Communication:} To comply with Federal Privacy Laws (FERPA) and NJCU policies, all communication will be through Blackboard and/or official NJCU e-mail. Check Blackboard daily. For further information see \href{http://scottoconnor.org/contact/}{http://scottoconnor.org/contact/}.

\item \textit{Conduct:} Distracting and disrespectful behaviors, including but not limited to eating, leaving your seat, talking out of turn, and aggression are prohibited. Penalties include, but are not limited to, a loss of participation points for the day of violation. Repeat offenders will be reported to the Dean of Students. 

\item \textit{Electronic devices:} Use of electronic device, including, but not limited, to smartphones, dictaphones, tablets, and laptops, is prohibited. Recording a lecture is in violation of Copyright. Penalties include, but are not limited to, a loss of participation points for the day of violation. Repeat offenders will be reported to the Dean of Students.


\item \textit{Format for Written Work:} Submit work to Blackboard as either a pdf, rtf, or doc file. Blackboard will not allow any other format. All work must be typed and neatly presented. 


\item \textit{General Education Program Assessment:} General Education courses participate in programmatic assessment of the six University-wide student learning goals. They include instruction in, and assessment of, at least two of these learning goals. Signature assignments, which may include document, picture, sound, or video files, are scored using approved program rubrics. Results are aggregated to provide information about the Gen Ed program as a whole. Your name will not be included in any programmatic assessment data.

\item \textit{Grading:} Grades will be available within 1--2 weeks of an assignment being submitted. See: \href{http://scottoconnor.org/resources/grading}{http://scottoconnor.org/resources/grading} for further information.


\item \textit{Late work \& Make-up Policy:} See the assignment schedule below. No make-ups or late work accepted under any circumstances. No exceptions under any imaginable circumstances.

\item \textit{Statement for students with disabilities:} If you are a student
with a disability and wish to receive consideration for reasonable
accommodations, please register with the Office of Specialized Services
and Supplemental Instruction (OSS/SI). To begin this process, complete
the registration form available on the OSS/SI website at
\href{http://www.njcu.edu/oss}{http://www.njcu.edu/oss}
(listed under Student Resources-Forms). Contact OSS/SI at 201-200-2091
or visit the office in Karnoutsos Hall, Room 102 for additional
information.

\item \textit{Turnitin:} Students agree that by taking this course all assignments are subject to submission for textual similarity review to Turnitin.com. Assignments submitted to Turnitin.com will be included as source documents in Turnitin.com's restricted access database solely for the purpose of detecting plagiarism in such documents.  The terms that apply to the University’s use of the Turnitin.com service are described on the Turnitin.com web site.  For further information about Turnitin, please visit: http://www.turnitin.com 

\item \textit{SafeAssign:} Students agree that by taking this course all assignments are subject to submission for textual similarity review through Blackboard SafeAssign. Assignments submitted to SafeAssign will be included as source documents in SafeAssign's restricted access database solely for the purpose of detecting plagiarism in such documents.  


\end{itemize}


\section{Weekly Course Schedule}
Readings marked with a `**' can be found on the course website. All other listed readings can be found in the required textbook. Changes to the syllabus will be announced through Blackboard and \emph{via} your NJCU email address.  All assignments must be submitted through Blackboard by Monday at 12:00am. \emph{Assignments cover the prior week's work, e.g., quiz 1 is due on 9/17/18 and covers the material from week 2.}

\begin{center}
\begin{longtable}{p{4.5cm}p{2cm}>{-- }p{6cm}}
 
  \caption{Course Schedule} \\
  \toprule
  \textbf{Week} &\textbf{Assignments } & \textbf{Reading} \\
  \midrule

  

[1] What is Philosophy?		& 	 			& Ch.1.1--1.2  \\
(9/3/18)					&				&  `The Trial and Death of Socrates', Plato, pp.44-53   \\  [1.8\baselineskip]

[2.] Critical Thinking		 	& 				&  Ch.1.3 \\
(9/10/18)					&				&  **Worksheet \\  [1.8\baselineskip]

[3.] The Meaning of Life 1		& Quiz 1 due			&  Ch.9.1--9.2 \\
(9/17/18)					&				&  `The Good Brahmin',Voltaire, pp.408--409 \\
						&				& **`A Confession', Tolstoy, pp.13--55 (optional)\\ [1.8\baselineskip]
	
[4.] The Meaning of Life 2		& Essay 1	due		&  Ch.9.3;  Ch.2.6, pp.106--112\\
(9/24/18)					& 				&  **`A Confession', Tolstoy, pp.55--93 (optional) \\  [1.8\baselineskip]

[5.] God---Design Argument 	& Quiz 2 due		& Ch.2.1-2.2.\\ 
(10/1/18)					&				& **`Intelligent Design Has No Place in the Science Curriculum', Morowitz, Hazen, and Trefil\\
						&				& **`Design for Living', Behe\\   [1.8\baselineskip]

[6.] God---Further Arguments	& Essay 2	 due		&   Ch.2.2\\
(10/8/18)					&				& `The Star', Clarke, pp.126--128\\ [1.8\baselineskip]

[7.] The Problem of Evil 		& Quiz 3	due		& Ch.2.3\\
(10/15/18)					& 				& **`Surprise! It's Judgment Day',  Davis\\  [1.8\baselineskip]		

[8.] Free Will 1 				& Essay 3	 due		& Ch.5.1--5.3 \\
(10/22/18)					&				&  `A Little Omniscience Goes a Long Way', Davis, pp.258--261 \\  [1.8\baselineskip]

[9.] Free Will 2 				& Quiz 4	due		&  Ch.5.3--5.4\\
(10/29/18)					&				& **`Please Don't Tell Me How The Story Ends', Davis \\  [1.8\baselineskip]

[10.] Phil. of Mind 1			& Essay 4	due			& Ch.4.1--4.2 \\
(11/5/18)					&					& **`Life after Life', Davis\\ [1.8\baselineskip]
						
[11.] Phil. of Mind 2 			& Quiz 5	due			& Ch.4.3--4.5 \\
(11/12/18)					&					& `They're Made out of Meat', Bisson \\ [1.8\baselineskip]
 

[12.] Cultural Relativism 		& Essay 5	due			& Ch.3.1--3.2\\
(11/19/18)					&					& **`Kirinyaga', Mike Resnik, ch.1 \\ [1.8\baselineskip]


[13.] Consequentialism 		& Quiz 6	due			&  Ch.3.3 \\ 
(11/26/18)					&					& `The Ones Who Walk Away from Omelas', Le Guin, pp.191--194 \\ [1.8\baselineskip]
						
[14.] Deontology, Virtue Ethics 	& Quiz 7	due			&  Ch.3.2;  Ch.3.5--3.6\\
(12/3/18)					&					&  `A Horseman in the Sky', Pierce \\ 
 				 		&					&  **`A Piece of Advice', Singer \\  [1.8\baselineskip]

[15.] Continued				& 			& Ch.3.2;  Ch.3.5--3.6 \\ 
(12/10/18)					&					& **`A Piece of Advice', Singer \\  [1.8\baselineskip]

[16.] Final 		& Essay 7 due		&  \\ 
\textbf{(12/17/18)}			& No exam	& \\  [1.8\baselineskip]




\end{longtable}
\end{center}



%% Uncomment if you want a printed bibliography.
%\printbibliography 

\end{document}