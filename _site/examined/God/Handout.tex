\documentclass[]{article}

\usepackage{amssymb,amsmath}
\usepackage{ifxetex,ifluatex}
\usepackage{fixltx2e} % provides \textsubscript
\ifnum 0\ifxetex 1\fi\ifluatex 1\fi=0 % if pdftex
  \usepackage[T1]{fontenc}
  \usepackage[utf8]{inputenc}
\else % if luatex or xelatex
  \ifxetex
    \usepackage{mathspec}
    \usepackage{xltxtra,xunicode}
  \else
    \usepackage{fontspec}
  \fi
  \defaultfontfeatures{Mapping=tex-text,Scale=MatchLowercase}
  \newcommand{\euro}{€}
\fi
% use upquote if available, for straight quotes in verbatim environments
\IfFileExists{upquote.sty}{\usepackage{upquote}}{}
% use microtype if available
\IfFileExists{microtype.sty}{%
\usepackage{microtype}
\UseMicrotypeSet[protrusion]{basicmath} % disable protrusion for tt fonts
}{}
\ifxetex
  \usepackage[setpagesize=false, % page size defined by xetex
              unicode=false, % unicode breaks when used with xetex
              xetex]{hyperref}
\else
  \usepackage[unicode=true]{hyperref}
\fi
\hypersetup{breaklinks=true,
            bookmarks=true,
            pdfauthor={},
            pdftitle={Further Arguments for God's Existence},
            colorlinks=true,
            citecolor=blue,
            urlcolor=blue,
            linkcolor=magenta,
            pdfborder={0 0 0}}
\urlstyle{same}  % don't use monospace font for urls
\setlength{\parindent}{0pt}
\setlength{\parskip}{6pt plus 2pt minus 1pt}
\setlength{\emergencystretch}{3em}  % prevent overfull lines
\setcounter{secnumdepth}{0}

\title{Further Arguments for God's Existence}
\date{}

\begin{document}
\maketitle

\subsection{The Ontological Argument}\label{the-ontological-argument}

Ontological arguments for God's existence try to deduce God's existence
from the very concept of God. Just as being male is part of the concept
of a bachelor and being a figure is part of the concept of a square,
some philosophers have claimed that existing is part of the concept of
God. If existing is part of the concept of God, then it is impossible to
entertain the concept of God while at the same time denying that he
exists. It would be like trying to entertain the idea of a bachelor that
is not male, or of a square which is not a shape.

We will be studying a version of the argument from St.~Anselm who was
writing in the 11th Century. The argument, then, is nearly 1000 years
old. Its age is noteworthy for three reasons. First, many have found the
argument convincing. They think that Anselm's argument is both valid and
sound. Second, many Atheists have thought the argument powerful and in
need of a strong objection. Third, the argument is short, but
sophisticated and incredibly challenging: many have found the challenge
of studying it rewarding in itself just as they might find working on a
challenging crossword puzzle a reward in itself. Take this third point
as a word of caution. You won't get this argument immediately. It takes
several attempts and significant mental gymnastics.

Let me introduce the core idea by way of analogy. Think about a cube. In
your mind, what features does that cube have? It has specific size,
color, 8 points, and 6 sides. Now holding the image of the cube fixed in
your mind, increase its size. Now decrease its size. Now change its
color. All of this is easy. You can do it. Now try holding the image
fixed in your mind and change the number of sides it has. Try it. Make
the cube have just 3 sides. This, of course, is an absurd request. A
cube must have 6 sides; you can never imagine a cube with less or more.

St.~Anselm claims that if we clearly entertain the idea of God, if we
properly identify the features that imagined figure has, then we will
discover that existence is a feature that we cannot imagine God to fail
to have; just as we cannot imagine a cube without 6 sides, Anselm claims
that we cannot imagine a non-existent God. Anselm will move on from this
claim to show that it is in principle impossible to deny that God
exists: since denying that God exists requires that we entertain the
idea of God, and since we cannot imagine a non-existent God, we cannot
coherently at the same time think about God and deny that he exists.

While the argument is difficult, at the core is the following simple
thought: the Theist's belief in God is a belief both that God is the
greatest being that exists and also that existing is one of the things
that makes something great. When the Atheist denies that God exists,
they cannot be denying that this greatest being exists; in their mind
the image of God they form is of a non-existent being, which is an image
of a less than perfect being. It's as if, on the one hand, the Theist
claims both that unicorns have horns and they exist, and, on the other
hand, the Atheist claims that unicorns have no horns and don't exist.
The image of a unicorn in the mind of the Atheist differs from the image
of a unicorn in the mind of the Theist. So, when one claims that
unicorns exist and the other denies it, they are not in disagreement.
They are talking about different things altogether.

\subsection{Summary of argument}\label{summary-of-argument}

\begin{enumerate}
\def\labelenumi{\arabic{enumi}.}
\itemsep1pt\parskip0pt\parsep0pt
\item
  Assume that God exists only as an idea in the mind.
\item
  God is a being than which none greater can be imagined (that is, the
  greatest possible being that can be imagined).
\item
  A being that exists as an idea in the mind and in reality is, other
  things being equal, greater than a being that exists only as an idea
  in the mind.
\item
  Thus, a being greater than God can be imagined, namely, a being that
  is like God in every way but also exists in reality. (From 1\&3)
\item
  God is and is not the greatest possible being that can be imagined.
  (From 2\&4)
\item
  Therefore, God exists, i.e., our assumption in Premise 1 is false
  (From 5)
\end{enumerate}

\subsection{Notes on the argument}\label{notes-on-the-argument}

This type of argument is called a \emph{reductio ad absurdum}. Recall
that a valid argument cannot have true premises and a false conclusion.
This is useful. If an argument is valid and the conclusion false, we
know that a premise is false. Consider this example:

\begin{itemize}
\itemsep1pt\parskip0pt\parsep0pt
\item
  P1. Socrates is immortal.
\item
  P2. Socrates is human.
\item
  P3. All humans are mortal.
\item
  P4. Therefore, Socrates is mortal. (from P2 and P3)
\item
  P5 Therefore, Socrates is mortal and immortal. (from P4 and P5)
\end{itemize}

We know that P5 is false. It is impossible for something to be both
mortal and immortal. Since P5 is false, and since it follows from
earlier premises, one of those earlier premises must be false. Many will
likely conclude that the culprit was our first assumption:

\begin{itemize}
\itemsep1pt\parskip0pt\parsep0pt
\item
  C. It is not the case that Socrates is immortal, i.e., P1 is false.
\end{itemize}

The Ontological Argument works in a similar way. We first assume that
God does not exist in reality, but only in the mind. We then show that
this ultimately leads to an impossible result, namely, that God is and
is not the greatest possible being that can be imagined. Since that
impossible results follows on from earlier premises, we know that one of
the earlier premises must be false. The Theist claims that the mistake
was Premise 1, which says that God exists only as an idea in the mind.
Premise 1 is the core claim of Atheism. Atheists accept that they can
form an idea of God, but they deny that anything in reality corresponds
to this idea. Since the argument is valid and the conclusion false, at
least one premise must also be false. But why think that the fault is
premise 1? The answer: the other premises seem true.

\begin{itemize}
\item
  Premise 2 is the Theist's stipulation. Their belief in God is a belief
  in the greatest possible being. If the Atheist were to deny this
  premise, they would, in effect, be shifting the terms of the debate.
  Why? By ``God'', the Theist means an absolutely unsurpassable being, a
  being that cannot conceivably be improved upon. If being knowledgeable
  is a good feature to have, then God, the greatest possible being, will
  be perfectly knowledgeable, more knowledgeable than any other being.
  The same goes for being good, being powerful, etc. Belief in God just
  is a belief, then, in a being that cannot be improved upon. If the
  Atheist denies Premise 2, then we are not denying the existence of a
  being that the Theist believes in.
\item
  Premise 3 says that existing is a feature something can have and also
  that existing is a good feature to have. Compare this to the claim
  that being strong is a feature and also a good one to have. When we
  imagine God, then, we should imagine a being that possesses all the
  great making features, existing in reality being one of these
  features. Premise 3 is the most controversial premise. See the
  textbook for further discussion.
\item
  Premise 4 follows from Premises 1\&3. If God does not exist in
  reality, but existing in reality is great making feature, then we can
  easily imagine a being that is greater than God.
\item
  Premise 5 follows from Premises 2\&4. It cannot be true both that God
  is the greatest possible being we can imagine and also that we can
  imagine a greater possible being.
\item
  6 states our conclusion. It may not be obvious how 6 follows from the
  premises. The core idea is that it is impossible for you to imagine
  that God does not exist. If you try, then you will be involved in the
  contradiction stated in 5.
\end{itemize}

\subsection{The Cosmological Argument}\label{the-cosmological-argument}

\begin{itemize}
\itemsep1pt\parskip0pt\parsep0pt
\item
  \href{https://www.youtube.com/watch?v=2zS1HiuWPMA}{Video: Part 1}
\item
  \href{https://www.youtube.com/watch?v=mBMAMIFw9n4}{Video: Part 2}
\end{itemize}

\end{document}
