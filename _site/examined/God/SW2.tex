\documentclass[]{article}

\usepackage{amssymb,amsmath}
\usepackage{ifxetex,ifluatex}
\usepackage{fixltx2e} % provides \textsubscript
\ifnum 0\ifxetex 1\fi\ifluatex 1\fi=0 % if pdftex
  \usepackage[T1]{fontenc}
  \usepackage[utf8]{inputenc}
\else % if luatex or xelatex
  \ifxetex
    \usepackage{mathspec}
    \usepackage{xltxtra,xunicode}
  \else
    \usepackage{fontspec}
  \fi
  \defaultfontfeatures{Mapping=tex-text,Scale=MatchLowercase}
  \newcommand{\euro}{€}
\fi
% use upquote if available, for straight quotes in verbatim environments
\IfFileExists{upquote.sty}{\usepackage{upquote}}{}
% use microtype if available
\IfFileExists{microtype.sty}{%
\usepackage{microtype}
\UseMicrotypeSet[protrusion]{basicmath} % disable protrusion for tt fonts
}{}
\ifxetex
  \usepackage[setpagesize=false, % page size defined by xetex
              unicode=false, % unicode breaks when used with xetex
              xetex]{hyperref}
\else
  \usepackage[unicode=true]{hyperref}
\fi
\hypersetup{breaklinks=true,
            bookmarks=true,
            pdfauthor={},
            pdftitle={Essay},
            colorlinks=true,
            citecolor=blue,
            urlcolor=blue,
            linkcolor=magenta,
            pdfborder={0 0 0}}
\urlstyle{same}  % don't use monospace font for urls
\setlength{\parindent}{0pt}
\setlength{\parskip}{6pt plus 2pt minus 1pt}
\setlength{\emergencystretch}{3em}  % prevent overfull lines
\setcounter{secnumdepth}{0}

\title{Essay}
\date{}

\begin{document}
\maketitle

\subsubsection{Introduction}\label{introduction}

You are a religious leader concerned about the teaching of evolutionary
biology in high schools. You think that only God could explain the
existence of life. The local school board is accepting public
submissions on the issue.

\subsubsection{Purpose}\label{purpose}

The purpose of this assignment is to help you practice the following
skills that are essential to your success in this course and others.

\begin{enumerate}
\item Charitably explaining arguments about issues of public importance. 
\item  Explaining difficult concepts in your own words.
\end{enumerate}

\subsubsection{Task}\label{task}

Write a short letter to the school board using the design argument to
prove that intelligent design should be taught in biology classes. Your
letter must do the following:

\begin{itemize}
\itemsep1pt\parskip0pt\parsep0pt
\item
  Use the design argument (and no other argument) to argue that God
  exists.
\item
  Explain all the premises/steps of the design argument. Use a simple
  example like Paley's watch to discuss these premises. Picking a
  different example is encouraged.
\item
  Avoid attacking the design argument. Your job is to prove God's
  existence regardless of your personal opinion.
\end{itemize}

\subsubsection{Word Count}\label{word-count}

Your submission must be 500-750 words long. Essays shorter than 500
words or longer than 750 words will lose points.

\subsubsection{Further Instruction}\label{further-instruction}

\begin{itemize}
\itemsep1pt\parskip0pt\parsep0pt
\item
  This assignment covers material contained in Ch.2.
\item
  Note that this is a letter. Include all appropriate letter headings,
  salutations, etc.
\item
  The members of the school board are not philosophers. Your job here is
  to explain a piece of complex philosophy to smart lay people. Use
  examples, simple vocabulary, etc.
\item
  Use simple examples when explaining the premises/steps of the Design
  Argument. Paley's use of the watch example is key. You are free to use
  other examples of artifacts, but you must explain to the reader how he
  convinces us that an artifact has a purpose before you can explain how
  he applies this result to nature.\\
\item
  Set aside your personal beliefs in completing this assignment. You
  might be a committed atheist. But you are not being asked to defend
  your own views. You are being asked to explain the Design Argument
  regardless of whether you accept its conclusion.
\end{itemize}

\subsubsection{Due Date}\label{due-date}

Please consult the syllabus and course website for the due date.

\subsubsection{Late Submissions}\label{late-submissions}

Per the policies outlined in the syllabus, late work will not be
accepted. Any request for special treatment will be ignored. If you
foresee difficulties submitting work on time, either because of personal
or work commitments, then you should start this paper early and submit
it early.

\subsubsection{Plagiarism}\label{plagiarism}

Please review the plagiarism policy on the syllabus. It is critical that
you prepare your assignment by yourself. Use only the textbook and
handouts---it will take you less time to work through these materials
than to find and read other sources. I will be checking for significant
overlaps between submission as well as checking answers against
Wikipedia, internet search results, standard essay sites, etc. If you
include material in your essay without citing it, you will receive 0 for
the assignment. A second violation will result in a 0 for the course, a
report to the Dean, and a petition for a note to be added to your
permanent academic record.

\subsubsection{Format}\label{format}

Submit your works by first clicking on the relevant assignment in
Blackboard. Click on `browse computer' and upload your document as a
.doc, .rtf or .pdf file. Do not send it to me by e-mail or through
Blackboard's messaging application.

\subsubsection{Grading}\label{grading}

Please find the rubric and explanation of it
\href{/Teaching/Grading/}{here}.

\subsubsection{Resources}\label{resources}

Please find links to writing resources \href{/Teaching/Resources/}{here}

\end{document}
