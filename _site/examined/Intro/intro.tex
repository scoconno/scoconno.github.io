\documentclass[]{article}
\usepackage{amssymb,amsmath}
\usepackage{ifxetex,ifluatex}
\usepackage{fixltx2e} % provides \textsubscript
\ifnum 0\ifxetex 1\fi\ifluatex 1\fi=0 % if pdftex
  \usepackage[T1]{fontenc}
  \usepackage[utf8]{inputenc}
\else % if luatex or xelatex
  \ifxetex
    \usepackage{mathspec}
    \usepackage{xltxtra,xunicode}
  \else
    \usepackage{fontspec}
  \fi
  \defaultfontfeatures{Mapping=tex-text,Scale=MatchLowercase}
  \newcommand{\euro}{€}
\fi
% use upquote if available, for straight quotes in verbatim environments
\IfFileExists{upquote.sty}{\usepackage{upquote}}{}
% use microtype if available
\IfFileExists{microtype.sty}{%
\usepackage{microtype}
\UseMicrotypeSet[protrusion]{basicmath} % disable protrusion for tt fonts
}{}
\ifxetex
  \usepackage[setpagesize=false, % page size defined by xetex
              unicode=false, % unicode breaks when used with xetex
              xetex]{hyperref}
\else
  \usepackage[unicode=true]{hyperref}
\fi
\hypersetup{breaklinks=true,
            bookmarks=true,
            pdfauthor={},
            pdftitle={Introduction to Philosophy},
            colorlinks=true,
            citecolor=blue,
            urlcolor=blue,
            linkcolor=magenta,
            pdfborder={0 0 0}}
\urlstyle{same}  % don't use monospace font for urls
\setlength{\parindent}{0pt}
\setlength{\parskip}{6pt plus 2pt minus 1pt}
\setlength{\emergencystretch}{3em}  % prevent overfull lines
\setcounter{secnumdepth}{0}

\title{Introduction to Philosophy}
\date{}

\begin{document}
\maketitle

\section{What is Philosophy?}\label{what-is-philosophy}

\subsection{The Examined Life}\label{the-examined-life}

``Philosophy'' is a Greek word meaning love of wisdom. As a discipline,
philosophy took its distinctive form in Athens, Greece about 2500 years
ago with the activities of one great individual, Socrates. While
Socrates didn't write anything, we know a lot about him primarily from
dialogues written by Plato, one of his inspired students and the second
of the great Greek philosophers.

Plato tells us that Socrates, like most Greeks, didn't think that living
in itself was that important. What was important was living a good life.
Their heroes like Achilles died young, but lived brightly. From an early
age, Greeks cared about how to distinguish themselves in this life.
Socrates was no different. He didn't want to merely live, he wanted to
live well. But unlike his fellow Greeks, Socrates was unsure what it was
to live a good life. Should he devote his life to pleasure, or to
distinguishing himself on the battleground, or to gaining political
power?

Socrates believed that knowledge, in particular, knowledge of how to
live the best possible life was the most important thing to seek for a
human. If we are ignorant of how to live well, then we risk gravely
damaging ourselves and missing out on living a fulfilling, flourishing
life. For this reason, Socrates claimed that the unexamined life was not
worth living. Unwilling to live his life based on a mistake, he sought
out those who claimed expertise about the good life. Socrates spent his
time quizzing these apparent experts, testing them to see whether they
really did know what they claimed to know. The experts failed these
tests. They were often confused or unable to answer simple questions
about the topics they claimed expertise in. For instance, they might
claim to know which people were courageous, but yet be unable to clearly
state what courage is. Questioning the establishment is dangerous and
Socrates was ultimately executed, in part, for corrupting the youth of
Athens by encouraging them to question the received wisdom of the time.

We might think of others who have been persecuted for holding
controversial views. It's not easy. In part, it's not easy because most
of us don't want to entertain the possibility that we are radically
mistaken in how to live our lives, how to organize our societies, and
mistaken in our thoughts about what we and the world we inhabit are
really like. We'll be following in Socrates's footsteps over the next
several weeks by putting under the microscope deeply held beliefs about
the existence of life, the meaning and value of life, as well as beliefs
about how to live an ethical life.

It's important to reflect on what you can expect to learn by taking a
course in philosophy. One of Socrates' greatest legacies was his claim
that he himself had no knowledge of the things he inquired into. This
single claim has shaped the character of philosophy. Philosophy is not a
body of knowledge, a list of facts that one can learn. Philosophy is
first and foremost an activity, a set of skills that will allow you
identify your beliefs, identify those that are wanting, and inquire in a
mature, humble way into the most important facets of human life. Just as
learning Karate, or dancing, or chess playing involves learning a skill,
learning philosophy involves learning a set of skills. So, don't expect
merely to learn some facts and theories in this course. Expect to learn
how to reason well, clearly, and unemotionally about your most deep
seated beliefs.

\subsection{The Allegory of the Cave}\label{the-allegory-of-the-cave}

Learning philosophy is not easy. Many shy away from examining their deep
convictions. For those who do not, the process is difficult, but
rewarding. The effects of this education were famously described in `The
Republic', by Plato in his famous \emph{Allegory of the Cave}. You will
find an animated version of that allegory
\href{https://www.youtube.com/watch?v=h55X9LJTAg4}{here}.

Plato describes a situation where a group of people have been born and
raised in a cave. They live their lives shackled to the floor, their
heads locked facing in one position. All they can see are shadows cast
on the wall in front of them. Since they can't turn their heads, they
cannot see the things the shadows are of. Imagine a baby that had only
ever seen holograms. They would never think the holograms were anything
but real.

Plato compares the state the cave dwellers are in to the position most
of us are in our ordinary lives. You already have beliefs. You have
beliefs about whether God does or doesn't exist. You have beliefs about
what's moral, for instance, whether it's morally permissible to cheat on
your partner. You have beliefs about what counts as a life well lived
and about lives that fall short of that. For instance, you already have
a belief as to whether a life devoted to taking drugs is a life well
lived. Plato thinks that these beliefs have been formed passively,
formed because we have been brought up in ways that rarely gave us pause
to question their veracity.

Plato also tells us that some of the cave dwellers are honored and given
prizes for the beliefs they have about the shadows. The person who picks
out a shadow of a cat reliably and can distinguish it from shadows of a
dog might be honored for their great perceptual powers, the honor givers
thinking that this ability is good and one we should all possess. In our
lives too, we honor and reward people. We award someone a Nobel Peace
Prize for work we consider valuable, or award a prize for artwork, or a
prize for athleticism at the Olympics. In each case, we make assumptions
about what is valuable, i.e., that art, or athleticism, or charity is
valuable and to be prized.

Imagine what it would be like for a cave dweller to realize that the
shadows were not real. Plato describes the process of being turned away
from the shadow as a violent one, violent in two ways. First, being
turned from the shadows will not happen by itself. It requires some
external force or person to free the cave dweller's head and allow it
turn. Second, it will be a painful experience. After all, a cave dweller
who is proud of their ability to distinguish shadows of a cat from
shadows of a dog will unlikely be happy to realize that their ability
rests on a deep mistake.

The experience of learning philosophy is violent in both these ways. You
will only stop to examine your beliefs, to be suspicious of your
convictions, if someone like Socrates starts challenging you, or
perhaps, if some life event forces you to take stock. Many people
dislike this experience. If you have won prizes based on the belief
that, say, athleticism is a good thing, then it can be disconcerting to
acknowledge that, perhaps, athleticism is worthless.

Plato tells us that philosophy is difficult in a second way too. If you
do wake up, if you are dragged from the cave, then you will initially be
blinded by the sun. You will not immediately see the real things outside
the cave which made the shadows. You will feel dizzy and grapple around
worried by this new experience. Philosophy too has that effect on
students. As you start questioning your beliefs, you will not
immediately see the truth behind those beliefs (if there is any).
Socrates spent over 50 years in this state, trying to identify the truth
about the good life and being killed before he could complete his
search. But just as the real things outside the cave start to slowly
come into view as your eyes adjust, you will slowly make progress in
philosophy, which you might compare to the process of escaping the cave
and allowing yourself adjust to the sunlight.

Finally, Plato describes the process of descending back into the cave
after your eyes adjusted and you have seen the real things outside.
Returning, you will once again see the shadows, but you will now see
them as shadows. Things in the cave will be very different for you now.
You will look at what was prized and honored and find them wanting.
Plato tells us that you will try help others to see the light, but that
you will most likely experience much rejection and perhaps violence by
the other cave dwellers.

As you develop as a philosopher, you will no longer unreflectively live
your life. You will no longer just accept as true the claims of your
family and society. You will see the things you formerly prized as less
prize-worthy. You may try wake up others who have not turned to
philosophy, and, perhaps, you will be met with resistance. Our fist
great philosopher, Socrates, tried to turn the cave dwellers of his
time, the Athenians, away from the shadows towards the light. For that,
he was cruelly executed. But his legacy lives on with each of you who
are deciding to bravely question your beliefs. If he has taught us
anything, it is that philosophy is the process of waking up. It will not
teach you the answers to life's mysteries, but it will teach you how to
search out those answers for yourselves.

\end{document}
