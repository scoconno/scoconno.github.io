\documentclass[10]{article}

\usepackage{fancyhdr}
 \pagestyle{fancy}
\rhead{\textsc{Scott O`Connor}}
\lhead{\textsc{E-mailing your younger self}}
\usepackage{lmodern}
\usepackage{amssymb,amsmath}
\usepackage{ifxetex,ifluatex}
\usepackage{fixltx2e} % provides \textsubscript
\ifnum 0\ifxetex 1\fi\ifluatex 1\fi=0 % if pdftex
  \usepackage[T1]{fontenc}
  \usepackage[utf8]{inputenc}
\else % if luatex or xelatex
  \ifxetex
    \usepackage{mathspec}
    \usepackage{xltxtra,xunicode}
  \else
    \usepackage{fontspec}
  \fi
  \defaultfontfeatures{Mapping=tex-text,Scale=MatchLowercase}
  \newcommand{\euro}{€}
\fi
% use upquote if available, for straight quotes in verbatim environments
\IfFileExists{upquote.sty}{\usepackage{upquote}}{}
% use microtype if available
\IfFileExists{microtype.sty}{%
\usepackage{microtype}
\UseMicrotypeSet[protrusion]{basicmath} % disable protrusion for tt fonts
}{}
\ifxetex
  \usepackage[setpagesize=false, % page size defined by xetex
              unicode=false, % unicode breaks when used with xetex
              xetex]{hyperref}
\else
  \usepackage[unicode=true]{hyperref}
\fi
\usepackage[usenames,dvipsnames]{color}
\hypersetup{breaklinks=true,
            bookmarks=true,
            pdfauthor={},
            pdftitle={Essay},
            colorlinks=true,
            citecolor=blue,
            urlcolor=blue,
            linkcolor=magenta,
            pdfborder={0 0 0}}
\urlstyle{same}  % don't use monospace font for urls
\setlength{\parindent}{0pt}
\setlength{\parskip}{6pt plus 2pt minus 1pt}
\setlength{\emergencystretch}{3em}  % prevent overfull lines
\providecommand{\tightlist}{%
  \setlength{\itemsep}{0pt}\setlength{\parskip}{0pt}}
\setcounter{secnumdepth}{0}

\title{Essay}
\author{Scott O’Connor}


% Redefines (sub)paragraphs to behave more like sections
\ifx\paragraph\undefined\else
\let\oldparagraph\paragraph
\renewcommand{\paragraph}[1]{\oldparagraph{#1}\mbox{}}
\fi
\ifx\subparagraph\undefined\else
\let\oldsubparagraph\subparagraph
\renewcommand{\subparagraph}[1]{\oldsubparagraph{#1}\mbox{}}
\fi

\begin{document}



\subsubsection{Introduction}\label{introduction}

Our beliefs and values change over time. Some think that marriage is
essential to a flourishing life and then find themselves thinking a
solitary life the best. Some think that money is the only thing of
value, but come to realize that, for them, family trumps wealth. Some
are atheists when they are young, but find God later in life. Still
others follow the religion of their parents and abandon it when they
become adults. You too have changed over time. There is some belief your
younger self held that you no longer share. It may concern politics, or
the nature of a meaningful life, or the nature of friendship, or God, or
your ethical requirements, or the nature of the mind, etc. You will have
changed your mind about at least one thing!

\subsubsection{Purpose}\label{purpose}

The purpose of this assignment is to help you practice the following
skills that are essential to your success in this course and others.

\begin{enumerate}
\def\labelenumi{\arabic{enumi}.}
\item
  Identifying your own beliefs about important philosophical topics.
\item
  Questioning your beliefs as Socrates questioned his own and others.
  This requires identifying reasons for your beliefs, being honest if
  you find none, and being courageous enough to change your mind.
\end{enumerate}

\subsubsection{Task}\label{task}

Write an e-mail to your younger self. This email must include the
following:

\begin{enumerate}
\def\labelenumi{\arabic{enumi}.}
\tightlist
\item
  Identify the belief your younger self holds that you no longer do.
\item
  Identify how that belief shaped your younger self. You can do this by,
  for instance, identifying how this belief affected their
  relationships, education, career choices, etc.
\item
  Tell your younger self why they likely hold this belief, i.e.,
  identify the reasons they have for thinking it true.
\item
  Explain to your younger self why you now think that belief is false.
\end{enumerate}

\subsubsection{Word Count}\label{word-count}

 500-750 words long. Essays shorter than 500
words or longer than 750 words will lose points.

\subsubsection{Further Instruction}\label{further-instruction}

\begin{itemize}
\tightlist
\item
  This assignment covers material contained in Ch.1.
\item
  This is an e-mail. Include all appropriate headings,
  salutations, etc.
\item
  You are writing to your younger self. Talk directly to them, not to
  me. Imagine what you would say to your younger self and how you might
  convince them to change their minds.
\item
  Use simple examples.
\end{itemize}

\subsubsection{Due Date}\label{due-date}

Please consult the syllabus and course website for the due date.

\subsubsection{Late Submissions}\label{late-submissions}

Per the policies outlined in the syllabus, late work will not be
accepted. Any request for special treatment will be ignored. If you
foresee difficulties submitting work on time, either because of personal
or work commitments, then you should start this paper early and submit
it early.

\subsubsection{Plagiarism}\label{plagiarism}

Please review the plagiarism policy on the syllabus. It is critical that
you prepare your assignment by yourself. Use only the textbook and
handouts---it will take you less time to work through these materials
than to find and read other sources. I will be checking for significant
overlaps between submission as well as checking answers against
Wikipedia, internet search results, standard essay sites, etc. If you
include material in your essay without citing it, you will receive 0 for
the assignment. A second violation will result in a 0 for the course, a
report to the Dean, and a petition for a note to be added to your
permanent academic record.

\subsubsection{Format}\label{format}

Submit your works by first clicking on the relevant assignment in
Blackboard. Click on `browse computer' and upload your document as a
.doc, .rtf or .pdf file. Please use `attach document'. Do not send it to
me by e-mail or through Blackboard's messaging application.



\end{document}
