\documentclass[oneside]{article}
 \headheight = 25pt
\footskip = 20pt
\usepackage{mdwlist}
\usepackage[T1]{fontenc}
\renewcommand{\rmdefault}{ppl}
\usepackage{fancyhdr}
 \pagestyle{fancy}
 \lhead{\textbf{\textsc{\small Scott O'Connor\\Metaphysics}}}
 \chead{}
 \rhead{\large\textbf{\textsc{Zeno 1}}}
 \lfoot{\footnotesize{\thepage}}
 \cfoot{}
 \rfoot{\footnotesize{\today}}
 \usepackage{longtable,booktabs}
\tolerance=700


\begin{document}
\thispagestyle{fancy}

\section{Introduction}

\begin{enumerate}
\def\labelenumi{\arabic{enumi}.}
\item
  Space is infinitely divisible or not infinitely divisible.
\item
  If space is infinitely divisible, motion is impossible.
\item
  If space is not infinitely divisible, motion is impossible.
\item[C.]
  Motion is impossible (from 1-3).
\end{enumerate}

\section{Premise 1 - the divisibility of
space}\label{premise-1---the-divisibility-of-space}

\begin{itemize}
\item
  If x is infinitely divisible, x can be divided into ever smaller parts
  \emph{ad infinitum}.This entails that x contains no indivisible parts,
  i.e., parts that cannot themselves be divided.

  \begin{itemize}
  \item
    For example, suppose that a line, L, is infinitely divisible. Lines
    are divided into line segments. Since L is infinitely divisible, every line segment that is part of L can be
    divided into further smaller line segments---there is no smallest
    line segment.
  \item
    Think of this process of dividing something out as merely
    conceptual; we do not need to able to physically make the division.
  \end{itemize}
\item
  If x is not infinitely divisible, x is divisible into a finite
  number of \emph{smallest} parts. This entails that some of x's parts cannot be divided into anything smaller. 

  \begin{itemize}
  \itemsep1pt\parskip0pt\parsep0pt
  \item
    For example, suppose that a line, L, is not infinitely divisible.
    If L is not infinitely divisible, then L contains a finite number of smallest line segments, i.e., L contains
    line segments with some smallest extent that cannot be divided into
    any further line segments.
  \end{itemize}
\end{itemize}

\section{Premise 2}\label{premise-2}


\emph{Strategy:} Assume that space is infinitely divisible. We will then argue
that it is impossible to move from one place to another by (i) showing that doing so requires completing an infinite number of tasks, and (ii) arguing that it is
is impossible to complete an infinite number of tasks.

Zeno argues for (i) and (ii) by using a number of paradoxes. The first is
called the \emph{Race Course}, which argues that it is impossible to complete
an arbitrary journey from A to B, i.e., to start at A and move to B. Aristotle describes the paradox as follows:

\begin{quote}
The first asserts the non-existence of motion on the ground that that
which is in locomotion must arrive at the half-way stage before it
arrives at the goal (Aristotle, \emph{Physics,} 239b11).
\end{quote}
Here is a simple presentation of the argument: 
\begin{enumerate}
\item[A.]  The distance between A and B is infinitely divisible (assumed).
\item[B.]  A journey from A to B is a series of sub-journeys with no last member:
  from A to \(\frac{1}{2}AB\), from \(\frac{1}{2}AB\) to
  \(\frac{3}{4}AB\), and so on.
\item[C.]  It is impossible to complete a series of sub-journeys with no last
  member.
\item[D.]  Completing a journey from A to B requires completing the series of
  sub-journeys with no last member: from A to \(\frac{1}{2}AB\), from
  \(\frac{1}{2}AB\) to \(\frac{3}{4}AB\), and so on.
\item[E.]  It is impossible to complete the journey from A to B.
\end{enumerate}
This argument proves that a traveller cannot complete their journey if they are traveling over an infinitely divisible distance. But, the fact that we cannot complete our movements does not entail  that we cannot move. Zeno completes his attack on premise 2 with two further paradoxes. 

\begin{itemize}
\item
  An inverted version of the paradox shows us that our traveler cannot
  begin to move...group project!
\item
  A different paradox, the Achilles paradox, shows us that in a race
  between Achilles and a tortoise, where the tortoise is given a head
  start, Achilles could never catch-up and pass the tortoise.
\end{itemize}
Simplicius reports the very famous Achilles paradox as follows: 
\begin{quote}
The {[}second{]} argument was called ``Achilles,'' accordingly, from the
fact that Achilles was taken {[}as a character{]} in it, and the
argument says that it is impossible for him to overtake the tortoise
when pursuing it. For in fact it is necessary that what is to overtake
{[}something{]}, before overtaking {[}it{]}, first reach the limit from
which what is fleeing set forth. In {[}the time in{]} which what is
pursuing arrives at this, what is fleeing will advance a certain
interval, even if it is less than that which what is pursuing advanced
\ldots{} . And in the time again in which what is pursuing will traverse
this {[}interval{]} which what is fleeing advanced, in this time again
what is fleeing will traverse some amount \ldots{} . And thus in every
time in which what is pursuing will traverse the {[}interval{]} which
what is fleeing, being slower, has already advanced, what is fleeing
will also advance some amount (Simplicius, \emph{On Aristotle's Physics},
1014.10).
\end{quote}

\section{Response 1: Reject C}\label{response-1-reject-c}

Some deny premise C by claiming that as we divide the distances of the
journey, we should also divide the total time taken, and, further, that
the sum of these infinite series of decreasingly short time intervals is
still equal to a finite period of time. These denials assume that the
Zeno's argument for C is the following:

\begin{enumerate}
\item[C1.] Completing an infinite series of tasks would take an infinite
  amount of time.
\item[C2.] It is not possible to spend an infinite amount of time completing
 a series of task(s).
\item[C3.] It is not possible to complete an infinite series of
  tasks (from C1--C2).
 \item[C4.] Sub-journeys are tasks.
 \item[C.] It is impossible to complete a series of sub-journeys with no last
  member, i.e., an infinite series of sub-journeys (from C4--C5).
\end{enumerate}
This argument is valid.  But some who think that this is Zeno's argument for C, deny that the argument is sound because C1 is false. They deny that completing an infinite series of tasks would take an infinite period of time. The idea is that we could complete an infinite series of tasks in a finite period of time if the amount of time to complete each task decreases. 

For instance,  as we divide the distances between the
points we travel, we should also divide the time it takes to travel the
ever smaller distances; it may take me one hour to travel between A and B, but it won't take me one hour to travel halfway between A and B:

\begin{enumerate}
\item
  It takes \(\frac{1}{2}\) the time to run from A to \(\frac{1}{2}AB\)
  as it does to run from A to B.
\item
  It takes \(\frac{1}{4}\) of the time to run from \(\frac{1}{2}AB\) to
  \(\frac{3}{4}AB\), and so on.
\item
  The sum of these decreasing times is finite.\footnote{See `A
    Contemporary Look at Zeno's Paradoxes', by Wesley Salmon}
\item
  Therefore, we can complete an infinite series of sub-journeys in a
  finite period of time.
\end{enumerate}
Let us grant that C1--C4 fail to establish premise C.  Why think that Zeno proves C in this way? If he does, his argument fails. But there is an
alternative way of defending premise C that is immune to our current objection:

\begin{itemize}
\itemsep1pt\parskip0pt\parsep0pt
\item
  Even if it takes less time to complete each sub-journey, I still need
  to first complete each sub-journey before completing the journey that
  comes after it. If so, I always have one more sub-journey to complete
  before I can complete the final step.
\end{itemize}
This argument makes no claim about how much time is required to complete an infinite series of tasks. It relies on the simple observation that completing a series of tasks required completing each task one after the other; I cannot start the second sub-journey until I have completed the first sub-journey. So, in order to complete the journey from A--B, I must complete the very final part of that journey, the very final sub-journey. But there is never a moment at which I have only one more sub-journey to complete. 

\section{Response 2: Reject B}

Recall premise B: 
\begin{enumerate}
\item[B.]  A journey from A to B is a series of sub-journeys with no last member:
  from A to \(\frac{1}{2}AB\), from \(\frac{1}{2}AB\) to
  \(\frac{3}{4}AB\), and so on.
  \end{enumerate}
Aristotle distinguishes between \emph{potential} and \emph{actual} sub-journeys. He agrees that a journey from A to B is a series of potential sub-journeys. So, he agrees with premise B when we add the `potential' qualification. However, Aristotle thinks the premise is false when it is read as claiming that the journey from A to B is a series of actual sub-journeys. On his view, the full journey does not consist of actual parts each of which are sub-journeys. There is only one actual journey, but there are many potential sub-journeys, journeys we could have taken instead of traveling between A and  B. 

How to evaluate this response? Zeno is defining journeys in terms of the distances over which they occur, i.e., he is assuming that whenever a person travels between two points, the journey he takes is defined merely in terms of these two points. Aristotle responds by denying that journeys are individuated in this way (because if they were the problem Zeno raises would be acute). But if journeys are not individuated by the distances over which they occur, how are they individuated? Why is the journey between A and B distinct from the journey between A and C? Aristotle cannot just define the journeys in terms of the relevant points because that is what leads to the paradox. Aristotle owes us, then, an account of actions/activities and a way of individuating them from one another. 

\end{document}
