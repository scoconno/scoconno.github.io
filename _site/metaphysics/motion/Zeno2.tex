\documentclass[]{article}
\usepackage{amssymb,amsmath}
\usepackage{ifxetex,ifluatex}
\usepackage{fixltx2e} % provides \textsubscript
\ifnum 0\ifxetex 1\fi\ifluatex 1\fi=0 % if pdftex
  \usepackage[T1]{fontenc}
  \usepackage[utf8]{inputenc}
\else % if luatex or xelatex
  \ifxetex
    \usepackage{mathspec}
    \usepackage{xltxtra,xunicode}
  \else
    \usepackage{fontspec}
  \fi
  \defaultfontfeatures{Mapping=tex-text,Scale=MatchLowercase}
  \newcommand{\euro}{€}
\fi
% use upquote if available, for straight quotes in verbatim environments
\IfFileExists{upquote.sty}{\usepackage{upquote}}{}
% use microtype if available
\IfFileExists{microtype.sty}{%
\usepackage{microtype}
\UseMicrotypeSet[protrusion]{basicmath} % disable protrusion for tt fonts
}{}
\usepackage{longtable,booktabs}
\ifxetex
  \usepackage[setpagesize=false, % page size defined by xetex
              unicode=false, % unicode breaks when used with xetex
              xetex]{hyperref}
\else
  \usepackage[unicode=true]{hyperref}
\fi
\hypersetup{breaklinks=true,
            bookmarks=true,
            pdfauthor={},
            pdftitle={Zeno 2},
            colorlinks=true,
            citecolor=blue,
            urlcolor=blue,
            linkcolor=magenta,
            pdfborder={0 0 0}}
\urlstyle{same}  % don't use monospace font for urls
\setlength{\parindent}{0pt}
\setlength{\parskip}{6pt plus 2pt minus 1pt}
\setlength{\emergencystretch}{3em}  % prevent overfull lines
\setcounter{secnumdepth}{0}

\title{Zeno 2}

\author{Scott O'Connor}
\date{ }


\begin{document}

\maketitle

\subsubsection{Motion does not exist}\label{motion-does-not-exist}

\begin{enumerate}
\def\labelenumi{\arabic{enumi}.}
\item
  Space is infinitely divisible or not infinitely divisible.
\item
  If space is infinitely divisible, motion is impossible.
\item
  If space is not infinitely divisible, motion is impossible.
\item
  Motion is impossible (From 1-3).
\end{enumerate}

\subsubsection{Premise 1 - the divisibility of
space}\label{premise-1---the-divisibility-of-space}

\begin{itemize}
\item
  If x is infinitely divisible, x can be divided into ever smaller parts
  \emph{ad infinitum}. In other words, x contains no indivisible parts,
  i.e.~parts that cannot further be divided.

  \begin{itemize}
  \item
    For example, suppose that a line, L, is infinitely divisible. Lines
    are divided into line segments. So every line segment of L can be
    divided into further smaller line segments - there is no smallest
    line segment.
  \item
    Think of this process of dividing something out as merely
    conceptual. Don't worry whether or not we could literally do
    something to x to divide it in this way.
  \end{itemize}
\item
  If x is not infinitely divisible, x can be divided into a finite
  number of \emph{smallest} parts, i.e.~parts that cannot be divided
  into any smaller parts.

  \begin{itemize}
  \itemsep1pt\parskip0pt\parsep0pt
  \item
    For example, suppose that a line, L, is not infinitely divisible.
    Then L contains a finite number of smallest line segments, i.e.,
    line segments with some smallest extent that cannot be divided into
    any further line segments.
  \end{itemize}
\end{itemize}

\subsubsection{Premise 3}\label{premise-3}

This handout will proceed by discussing Premise 3. See Handout 1 for
discussion of Premise 2. Zeno offers two distinct arguments for Premise
3 that again come in the form of paradoxes. The strategy for each is
similar. We will fist assume that space is not infinitely divisible,
then prove that certain absurdities follow. If an assumption leads to an
absurdity, we know the assumption is false.

\subsubsection{Stadium Paradox}\label{stadium-paradox}

\begin{quote}
The fourth argument is that concerning equal bodies which move alongside
equal bodies in the stadium from opposite directions---the ones from the
end of the stadium, the others from the middle---at equal speeds, in
which he thinks it follows that half the time is equal to its
double\ldots{}. (Aristotle Physics, 239b33)
\end{quote}

Suppose these rows of blocks represent some chariots in a stadium. The
B's are stationary. The A's are moving from left to right. The last
block in that row is called D. The Cs are moving towards from right to
left. The middle block in that row is called E. 

\begin{longtable}[c]{@{}lll@{}}
\toprule
& T1 &\tabularnewline
\midrule
\endhead
& DAA & --\textgreater{}\tabularnewline
& BBB &\tabularnewline
\textless{}-- & CEC &\tabularnewline
\bottomrule
\end{longtable}

\begin{longtable}[c]{@{}rll@{}}
\toprule
& T3 &\tabularnewline
\midrule
\endhead
--\textgreater{} & ~ DA & A\tabularnewline
& BBB &\tabularnewline
~ ~ C & EC~ & \textless{}--\tabularnewline
\bottomrule
\end{longtable}

Compare Times 1 and 3. Suppose they are separated by a one minute
interval. In this interval, D has passed one B block and two C blocks.
Zeno thinks this is paradoxical. It's unclear why. For our purposes, let
us assume the following:

\begin{enumerate}
\def\labelenumi{\arabic{enumi}.}
\itemsep1pt\parskip0pt\parsep0pt
\item
  There is smallest possible length, \emph{S}
\item
  The length of each block is S.
\item
  There are no gaps between the blocks.
\item
  The blocks move with constant velocity.
\end{enumerate}

It took 1 minute for D to pass two C blocks. It should take 30 seconds to
pass one C block and become level with E. Suppose D passes one C block after 30 seconds. How
many B blocks has it passed? Try filling out the diagram below to answer
that question.

\begin{longtable}[c]{@{}rll@{}}
\toprule
& T2 &\tabularnewline
\midrule
\endhead
--\textgreater{} & DAA &\tabularnewline
& ? &\tabularnewline
~ ~ C & EC~ & \textless{}--\tabularnewline
\bottomrule
\end{longtable}

T3 describes the moment that D and E are level. How does D relate to the B's at this moment?

We are stuck! Suppose that someone claims that D has passed \emph{half
of one B block.} Let this half be called \emph{H}. What is H's length?
You cannot, on pain of contradiction, claim that H has a length less
than S. We have assumed that S is the smallest possible length, so H
cannot be shorter than S.

This way of stating the paradox assumes that the length of time between
T1 and T3 can be divided in two, i.e., 1 minute is divided into two 30
second intervals. Suppose that time is also atomic, that there is a
smallest interval of time, a single quantum of time. Suppose also that
the motion between T1 and T3 takes a single quantum of time. If this is
correct, there is no T2 (which was half the interval between T1 and T3.)

Paradox still threatens. During a single quantum of time, D and E will
have passed each other (as is seen in T3), but there is no moment at
which they are level as is described in T2: since T1 \& T3 are
separated by the smallest possible time, there can be no instant between
them---it would be a time smaller than the smallest time from the two
moments we considered. Conversely, if one insisted that if they pass
then there must be a moment when they are level, then it shows that
cannot be a shortest finite interval.




\end{document}
