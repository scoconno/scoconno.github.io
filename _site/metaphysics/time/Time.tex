\documentclass[oneside, 10pt]{article}
 \headheight = 25pt
\footskip = 20pt
\usepackage[T1]{fontenc}
\renewcommand{\rmdefault}{ppl}
\usepackage{fancyhdr}
 \pagestyle{fancy}
 \lhead{\LARGE\bf{\textsc{Scott O'Connor}}}
 \chead{}
 \rhead{\textsc{The Philosophy of Time}}
 \lfoot{\footnotesize{\thepage}}
 \cfoot{}
 \rfoot{\footnotesize{\today}}
\tolerance=700
\addtocounter{section}{1}

\begin{document}
\thispagestyle{fancy}
\subsection*{Description}
Could time stop? Could time flow backwards? Is time travel possible? Why is time important? What exactly is time? These are a few of the questions that we will discuss in this course. We will cover five modules:
\begin{description}
\item[Telling Time] Time seems mundane, regular, and perfectly unmysterious. We look at our watch. It reads 10pm. And it reads exactly 10pm at the same time every day. But what would life be like if we had difficulty measuring time? We will begin the course by thinking about classical Athens and the water clock. After building a simple version in class, we will ask how we would regulate our own lives and the societies we live in if we only had these sorts of clocks available to us. For instance, how could we write a bus schedule or measure the speed of our favourite race horse?  

\item[Time and change] Could change stop but time continue to flow? Did time exist before the Big Bang? In this module, we will discuss some ancient Greek views about the relationship between time and change. First, we will discuss why Plato thinks that time can exist without change while Aristotle argues that time cannot. Then we will discuss Zeno's famous paradoxes of motion. Zeno argued that these paradoxes show that motion is impossible. These arguments have had a profound effect on the development of philosophy, physics, and mathematics.  Some of the paradoxes, in particular, the arrow paradox, turn on the relationship between time and motion.  We will examine some ancient attempts to solve these paradoxes before turning to discuss more recent attempts. 

\item[The passage of time] What is the past, present, and future? Why does time always flow towards the future and away from the past? What exactly are positions in time?  In this module, we will examine the work of the Cambridge philosopher J.M.E McTaggart. McTaggart suggested that there are two different ways of thinking about positions in time, the A-series and B-series. He argues that neither can adequately describe time and ultimately denied the reality of time. The A-series and B-series have been wonderfully depicted by Kurt Vonnegut in his novel `Slaughterhouse-Five'. We will read the novel to get a sense of both series before examining McTaggart's arguments and various responses to those arguments.

\item[Time travel] Would you like to visit the Hanging Gardens of Babylon? If time travel were possible, you might just be able to do so. In this module, we will discuss whether time travel is possible. We will begin with a quick overview of special and general relativity. Our goal here is to understand the kind of limited future time travel that the best physics allows. We will then discuss philosophical puzzles about backward time travel. For instance, could a time traveller travel into the past and kill their great great grandfather?  

\item[Being and Time] Why is time important ? In this fifth and final module, we will discuss the work of the German philosopher Martin Heidegger. Heidegger argued that we can only understand time by considering our own mortality. He subsequently argues for a close connection between our experience of time and what is required to live an authentic life. We will examine these arguments and assess whether, and if so how, our experience of time relates to our ability to live a fulfilling and rich life. 

\end{description}


\subsection*{Writing}
First-years can write. They can write clear directions to a party, a clear letter to their parents, a clear personal statement for college. What many can't do is apply what they already know to academic writing. They struggle to apply what they know because they struggle to understand how academic writing is also merely a means of communicating information to some audience; of telling \emph{someone} about \emph{something}. 

I design the writing assignments to teach them about their audience, about what their readers must be told if they are to understand a paper. My strategy is twofold. First, I use Joseph Williams, `Style: Lessons in Clarity and Grace'. This book is excellent. Among other things, it teaches students how to construct sentences about abstract topics, e.g. it teaches them how to pick appropriate subjects  and verbs for those sentences. Second, I use make belief. I ask students to place themselves in concrete imaginary scenarios where they must explain some theoretical material to specific people. I then ask them to imagine that how they explain the material will have practical implications. Here are a few examples:

\begin{description}
\item [Business plan] Suppose that water clocks are the only clocks availabe.  Recall their inaccuracies and limitations. I want  you to propose a business plan for TCAT outlining how to both use and overcome the limitations of a water clock. 

\item [Legal brief] Zeno argued that motion does not exist. If motion does not exist, you cannot crash your car. Imagine that you receive a court summons. It says that you are being sued for rear ending another car. You submit a legal brief outlining your defence. You will argue that motion does not exist by using Zeno's famous paradoxes. In class, you will present this argument orally. Several of your peers will serve as judges and cross examine you.  

\item [Travel Guide] In `Slaughterhouse-Five', Kurt Vonnegut describes an alien race called the Tralfamadorians. The Tralfamadorians experience time in a radically different way from how humans experience time. They claim that they see all of time as a large landscape; just as you can see the lake in front of the mountain, they claim that the see `all together' every event laid out before and after each other. They struggle to explain this to the main character, Billy Pilgrim, and struggle to understand just how Billy experiences his time. The differences between how humans and Tralfmadorians experience time corresponds perfectly to the difference between the A-series and B-series. For this assignment, I want you to write a section in two travel guides. In the first, you will try explain to human visitors to Tralfamorde how the inhabitants understand and experience time. In the second, you will try to explain to a Tralfmadorian visitor to Earth how the inhabitants understand and experience time. 

\item [A letter from a time traveller] We talked about one surprising consequence of general and special relativity: an astronaut who left earth and travelled at immense speeds before returning would have aged much slower than those who had remained on Earth. Imagine that you are a 29 year old astronaut. You have a five year old daughter. You are asked to travel deep into space. It is a tough decision, but you decide to take the job. When you return, your daughter will be 40. You will be 30. How could you prepare your child for this bizarre phenomenon? For this assignment, you will write a letter to her that she will open on her 12th birthday. In this letter, you will explain as simply as you can why she will be older than you when you return to Earth. 

\end{description}

\subsection*{Further Information} 
My interests are in ancient and contemporary metaphysics. I am currently writing on Aristotle's \emph{Physics}, and I have taught writing seminars on both his natural philosophy and on related issues in contemporary metaphysics.  I enjoy teaching and thinking about philosophical issues that were first raised by the Greeks and are hotly debated today. Time is a great example. Early Greek philosophers like Parmenides and Zeno denied that change is possible. They think that when we see a horse running, we are merely seeing an illusion. Many of their arguments turn on the relationship they see between time and change. Responding to these arguments requires identifying what exactly time is. But it is unclear that anyone has done so yet. Contemporary philosophers have offered sophisticated theories about what time is in the hope of salvaging change, but an equal number of philosophers have argued that these theories are inadequate. 

Students love these debates. They know that measuring-tapes measure distance, a weighing-scale measure weight, and a clock measures time. They can use each of these instruments. Yet while they can try explain what distances and weights are, they  struggle to explain what it is that they measure with their clocks. In my `Generation and Destruction' course, students asked and discussed why time keeps flowing towards the future and away from the past. These were our best discussions. Students have such intimate experience with time that they are amazed at how difficult it is to explain what it is. I subsequently regretted that I only allocated 3 weeks to time, and I would relish the opportunity of teaching an entire writing seminar on it.

This course would also fit perfectly with the ethos of the Knight Institute. The Institute exists to teach students how to write. It does not exist merely to teach them how to write papers on philosophy, or sociology, or literature, etc. It exists to teach students how to write full stop. First year writing seminars succeed only if they are designed for students with diverse interests and diverse academic career paths. The philosophy of time is a perfect opportunity to introduce students to different disciplines and different forms of writing. Students will read and learn about classical Athens, ancient philosophy, fiction, physics, and metaphysics.  Along the way, I will teach them to write a business plan and a legal brief. The course is inter-disciplinary because time is interdisciplinary.  

The course also promises to offer something that other courses do not. First-years from outside the Humanities  often undervalue the importance of writing. They assume that they will be writing computer code, lab reports, and designing bridges. Many think that they would prefer to work on these projects than working on how to write. But the philosophy of time, and metaphysics in general, appeals to many of these students. In the past, my courses have attracted many physicists, mathematicians, engineers, as well as many from the Humanities. For instance, 14 out of 18 of my students next semester are engineers.   A course on the philosophy of time will cater to these students. It will serve as both an entry point to different disciplines, but also inspire them to learn how to write well. For instance, I will assign science fiction that relates to the metaphysics of time. In the past, science majors loved these readings. They frequently are furious if the fiction is scientifically inconsistent. So I challenge them to write a more consistent short story. To do this, they need to write ordinary English prose. This course on the philosophy of time will both inspire students to write such prose, and this course will give them the means of doing so. 

\end{document}