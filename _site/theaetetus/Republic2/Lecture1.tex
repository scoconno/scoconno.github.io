\documentclass[oneside]{article}
\usepackage{graphicx}
 \headheight = 25pt
\footskip = 20pt
\usepackage{mdwlist}
\usepackage[T1]{fontenc}
\renewcommand{\rmdefault}{ppl}
\usepackage{fancyhdr}
 \pagestyle{fancy}
 \lhead{\textbf{\textsc{\small Scott O'Connor\\Ancient Philosophy}}}
 \chead{}
 \rhead{\large\textbf{\textsc{Republic 4}}}
 \lfoot{\footnotesize{\thepage}}
 \cfoot{}
 \rfoot{\footnotesize{\today}}
 \usepackage{longtable,booktabs}
\tolerance=700


\begin{document}
\thispagestyle{fancy}

\subsection*{Defense of Justice}
We have defined justice as a certain harmony between the parts of the soul that exists when each parts does its own work. But we have yet to show that justice is preferable to injustice, that being just is beneficial in itself. We will return to the topic in greater detail in Book 8, but Socrates does offer a quick argument for the claim in Book 4. Let us call it the \emph{Psychological Health Argument:}

\begin{enumerate}
\item Justice is harmony between the three parts of the soul where each does its appropriate task.
\item Injustice is strife between the three parts.
\item	Justice is like health, since health too involves the proper ordering of parts.  
\begin{quote}
To produce health is to establish the components of the body in a natural relation of control and being controlled, one by another, while to produce disease is to establish a relation of ruling and being ruled contrary to nature...Then...isn`t to produce justice to establish the parts of the soul in a natural relation of control, one by another, while to produce injustice is to establish a relation of ruling and being ruled contrary to nature? (444d)
\end{quote}
\item  Health is preferable to disease.  (Recall that health is good in itself and for its consequences.)
\item[C.]	Therefore, justice is preferable to injustice.
\end{enumerate}

\subsection*{Reason}
In the soul of a just person, the rational part does its own job of ruling over the other parts of the soul. Such a person is wise by virtue of their rational part ruling in accordance with its knowledge of what is best for the whole soul (442c).   Socrates' account assumes that there are facts about what is good for the various parts of the soul and the whole person, and it also assumes that reason can acquire knowledge of such facts. We might agree with Socrates that reason loves wisdom and learning (581c). We might also agree that it reasons about the better and the worse. And we might even agree that it wants to look out for the good of the whole soul. But why should we agree that it can succeed in any of this? I may want to be an astronaut. It doesn't follow that I can be one. Similarly, why assume that there are facts about what's good for the soul that reason can acquire? Socrates' answer has two parts. 

\subsubsection*{What is the proper object of such knowledge?}

Socrates draws a sharp ontological distinction between two kinds of entities, perceptible objects and intelligible objects:

\begin{description}
\item[Perceptible objects:] Entities about which we can gain information \emph{directly} through the five senses. It also includes groups of such objects.
\begin{itemize}\item{Examples: Socrates, Socrates' dog Fido, this building, each person in this room}\item{Also: The people in this room (we do not see this entity directly, rather we gain information about it by seeing (hearing, touching, etc.) each individual person in this room}\end{itemize}

\item[Intelligible objects:] Entities about which we gain information \emph{solely} through the activity of thought (\textbf{NB}: The ``solely'' is crucial here, since we can think about and, hence, gain additional information about perceptible objects. The point is that, in addition to being able to think about perceptible objects, we can \emph{also} perceive them through the senses.)

\begin{itemize}\item{Examples: Mathematical objects (squares, triangles, the number 2), Natures or Essences, which Socrates also calls ``Forms'' (\emph{eid\^{e}}) or ``Ideas'' (\emph{ideai}) (e.g. the nature of Piety, the nature of Justice, the Nature of Goodness)}\item{This is a development from the conception of Natures or Essences we found in the Socratic dialogues (e.g. \emph{Euthyphro}, \emph{Protagoras}, \emph{Meno}). In those dialogues there was no indication that Natures or Essences existed separately from the perceptible world.}\end{itemize}

\end{description}
Perceptible objects do not perfectly instantiate intelligible objects but, rather, ``approximate,'' ``resemble,'' or ``participate in'' them. S also says that perceptible objects are ``images'' or ``imitations'' of intelligible objects. Consider the following two figures:

\begin{figure}[h!]
\hspace*{35mm}
\includegraphics[scale=0.6]{figure1}
\end{figure}
Figure 1 is not a perfect square, but it more closely approximates the nature of Square than Figure 2 does. The person who knows what a perfect square is can reliably determine which perceptible figure most approximates a perfect square.

Similarly, S claims that, while nothing in the perceptible world perfectly instantiates the nature of (for example) justice, certain things in the perceptible world (e.g. distributions of resources, social institutions, etc.) can more closely approximate the nature of justice than others (and, hence, can be ``more just'' than others). Knowledge of justice would allow us to reliably decide which policies or actions most approximate justice.



\subsubsection*{Learning the Forms}

At 471, Glaucon questions Socrates whether a city with the constitution they laid out in Books 2--4 can be realized on earth. 

\begin{itemize}
 \item S maintains that the way to bring about a city \emph{closest to} the ideal city is to vest political power in the hands of philosophers.
\item Philosophers are distinguished from non-philosophers insofar as only philosophers can attain understanding (\emph{epist\^{e}m\^{e}}) and knowledge (\emph{gn\^{o}sis}) of intelligible objects, which makes them epistemic authorities about matters in the perceptible world (think of a doctor's expert ``medical opinion'' that a particular patient should undergo a particular course of treatment). 
\item The intelligible objects are Forms. Philosopher rulers know the various Forms and use that knowledge as a model for deciding which policies are best for the city as a whole. These include those that promote harmony in the city and avoid strife. Likewise, the philosopher uses their knowledge of the Forms to decide which appetitive desires to have and act upon, and also which emotions to have and act upon. These include those that promote harmony in the soul and avoid strife. 
\item The Cave is an allegory of how education effects a radical change in people and can ultimately lead them to understanding of Forms, the most fundamental of which is the Form of the Good.
\end{itemize}
\end{document}
