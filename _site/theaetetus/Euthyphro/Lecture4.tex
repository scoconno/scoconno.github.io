\documentclass[oneside]{article}
 \headheight = 25pt
\footskip = 20pt
\usepackage{mdwlist}
\usepackage[T1]{fontenc}
\renewcommand{\rmdefault}{ppl}
\usepackage{fancyhdr}
 \pagestyle{fancy}
 \lhead{\textbf{\textsc{\small Scott O'Connor\\Ancient Philosophy}}}
 \chead{}
 \rhead{\large\textbf{\textsc{Euthyphro 1}}}
 \lfoot{\footnotesize{\thepage}}
 \cfoot{}
 \rfoot{\footnotesize{\today}}
 \usepackage{longtable,booktabs}
\tolerance=700


\begin{document}





\section*{Overview}

Murder was a religious offense because it entailed `pollution', which, if not ritually purified, was displeasing to the gods. But a son prosecuting his father would also be seen by many Greeks as `impious' (or `unholy', `ungodly'.) However, Euthyphro claims that he must prosecute his father for murder; he claims that the pious thing is, in fact, to prosecute his parent in this case. Socrates is unsure. He thinks that knowledge of whether such a prosecution is pious or not would require us to know the nature of piety. Similarly, to know what actions are impious we must know the nature of impiety. Euthyphro agrees and claims that he does, in fact, know the nature of piety. Socrates tests this claim. Euthyphro  is found wanting and so his claim to know that prosecuting his father is not impious is undermined.\\ 


%The basic structure of the dialog is as follows: 
%\begin{itemize}\item{2a--4a: Socrates (S) encounters Euthyphro (E), who is about to prosecute his (E's) father for murder.}\item{4a--5d: S claims that only someone with expertise about piety, crucially including knowledge of its nature, should be confident that he were not acting impious in prosecuting a relative.}\item{5d--15d: E offers various candidates for the definition of piety, all of which S rejects.}\item{15d--end: S requests to start from the beginning, E refuses, leaves.}\end{itemize}

\noindent Our learning goal for this dialog:  Identify and understand Socrates' views about an adequate answer to a question of the form, `what is X', i.e., what it is you have to know in order to properly know what X is. 







%\section*{Ethical action without knowledge}
%\begin{itemize}
%\item{[1] E's prosecution of his father is based on his belief that he has expertise concerning piety}

%\item{[2] S seems to show that E does \emph{not} have that expertise}\begin{itemize}\item{What conception of expertise is at work here? Why does such expertise require knowledge of the nature of piety? Is it reasonable to demand such expertise? (for all actions? for some?) How does S test whether E has such expertise?}\end{itemize}

%\item{[3] If E becomes aware of [2], how would it be rational for him to act?}

%\begin{itemize}\item{By the end, S seems to think that E shouldn't prosecute his father. Why is this?}\end{itemize}
%\end{itemize}
%

\section*{The ``What is \emph{X}?'' Question}

Socrates asks a simple question, ``what is X?', about moral and aesthetic qualities. In the \emph{Euthyphro}, he asks `what is piety?', but other dialogs find him asking `what is justice?', `what is courage?', `what is wisdom?', `what is love?', `what is beauty?', etc.\\

\noindent A \textbf{Socratic definition} is an answer to a ``what is X?'' question. These definitions are not of words, but of things. S wants to know the nature of piety itself and not just what the word `piety' means. In a similar way, physicists investigating the nature of matter are not interested in what the word `matter' means. If they were, they could just consult a dictionary. They are interested in the nature of that stuff in the world, the nature of matter itself.  A Socratic definition, then, is a true description of the nature of the thing to be defined. For instance, `H2O' is true description of the nature of water and not just our thoughts about water. \\

%Students often resist Socrates at this point. Many will respond that piety, or beauty, or courage, etc., exist and are what they are because of how we think and speak about them. Call such a view \emph{conventionalism}.  If conventionalism is true about, say, beauty, we shouldn't be surprised that different cultures treat different things and people as beautiful. Conventionalism about beauty denies that beauty exists and is what it is independently of how we think and speak about it. 

%Is conventionalism true? It's a substantive position that requires a substantive argument. For our purposes, we can point out that Socrates does not yet hold any substantive position. He searches for the Socratic definition of piety precisely because he does not know what it is. If, after searching with him, we discover that there is no answer, then we might conclude that conventionalism is true. But conventionalism cannot be the starting point. Our starting point, reasonably, is simply the question, `what is piety?'

\noindent What would S consider a satisfactory answer to the question `what is piety'? More generally, what are the criteria that must be satisfied for an answer to a `what is X' question to qualify as a Socratic definition? S doesn't tell us explicitly. Instead,  S ask E to define piety. E offers a candidate, C, as a definition of piety. Then S elicits further claims that seem to entail that C is not, in fact, the definition of piety. S asks E to try again. E offers D as a definition of piety. S elicits further claims that seem to entail that D is not, in fact, the definition of piety. The process repeats. 
By focusing on how S argues against various candidate definitions of piety, we will identify what he thinks is required of a satisfactory answer. 

\section*{Failed Answers}
\begin{enumerate}
\item Prosecuting the wrongdoer regardless of personal relationship to the wrongdoer (5e).

\begin{itemize}\item{This is an \emph{example} of a (kind of) pious action.}
\item S didn`t ask for just one or two examples of pious  actions, but that one form (eidos, idea) in virtue of which all pious actions are pious, so that he can use that form as a model to say of any action whether it is pious.

\item{Moral: S thinks an adequate answer must specify that in virtue of which \textbf{every} pious action is pious.}\end{itemize}

\item{What is dear to the gods (7a).}

\begin{itemize}\item{If the gods disagree on important ethical matters (as E agrees they do at 6c and 7b--8a), then one and the same thing could be both pious (because dear to some god(s)) and impious (because hated by some other god(s)), but that is impossible.}
\item{Moral: S thinks that whatever makes some pious actions pious cannot at the same time make some impious actions impious; if X makes some actions pious, then everything which is X is pious.}\end{itemize}

\item{What is dear to \emph{all} the gods (9e).}
\begin{itemize}
\item{...\emph{next class}...}
\end{itemize}
\end{enumerate}


\section*{Preliminary Results}

Socrates' rejections of these various proposal shows us that a satisfactory answer to the question `what is piety?', and more generally to all `what is X? questions must satisfy certain criteria. X is defined as AB if and only if: 
\begin{description}
\item[General:] everything which is X is also AB. If piety is defined as being God loved, then everything which is pious must also be God loved. If, then, sacrificing is pious and praying is pious, then both sacrificing and praying must be God loved. 
\item[Univocal:] everything which is AB is also X. If piety is defined as being God loved, then everything which is God loved must also be pious. If, then, killing your enemy is battle is God loved, killing your enemy must also be pious. 
\end{description}
Think of these criteria as providing tests for candidate definitions. E proposes definitions of piety. S examines whether those definitions are general and univocal. 

\end{document}




%\begin{itemize}\item{This only gives us a quality or affection of piety, it does not tell us what piety is.}\end{itemize}

%\item{The part of justice concerned with the care of the gods (12e)}

%\begin{itemize}\item{Care for \emph{X} aims at benefitting \emph{X} or making \emph{X} better; But gods cannot be made better}\end{itemize}

%\item{The part of justice concerned with service to the gods (13d)}

%\begin{itemize}\item{Service aims at some goal (e.g. service to generals aims to help them win wars, service to house builders aims to help them build houses) but E. can't specify what ``fine thing'' gods achieve such that service could aim to help them achieve that goal}\end{itemize} 

%\item{Knowledge of how to sacrifice and pray (14d)}

%\begin{itemize}\item{This definition reduces to [3] and the claim that sacrifices and prayers are dear to all the gods}\end{itemize}

%\end{enumerate}




\section*{Socrates' rejection of [3]}
Recall E's third attempt at defining piety. Piety = what is dear to all the gods (10a-11b). Socrates asks, ``is the pious loved by the gods because it's pious? Or is it pious because it's loved?'' (10a). 

This is as a question about the order of explanation. As an example, assume that there is a strict correlation between the crowing of a rooster and the rising of the sun. Assume that when one happens the other also happens (you need to assume there has been roosters for as long as the sun existed.). Do you think that the sun rises because the rooster crows? Or does the rooster crow because the sun rises? This is not a trick question. There are many correlations where one correlate explains the other, e.g., death and the cessation of brain activity---which is the cause of the other? How about death and rigor mortis?\\


\noindent \textbf{``Euthyphro Dilemma''}: If certain actions are pious (right, wrong, obligatory, impermissible, etc.) \emph{because} they are dear to the gods (or commanded but God/the gods, etc.), why do the gods love what they love? Either it is [A] \emph{arbitrary} what the gods love, or [B] the gods love what they do for \emph{reasons}. Either way, trouble looms:\begin{itemize}\item{[A] seems absurd. [A] entails that if the gods love rape, murder, etc. then rape, murder, etc. would be pious. This is ridiculous.}\item{If [B], it is the features of the actions in virtue of which the gods love them that explains why they are pious. The attitude of the gods is not what \emph{explains} or \emph{makes it the case} that they are pious.\\} \end{itemize} 


\noindent Socrates ultimately claims that the fact that a certain action is pious explains why (all) the gods love it, and not the other way round. This is why S says that E has identified (at best) a quality of piety, not the nature of piety. So, here again are the criteria for an adequate answer to a 'What is X?' question. 

%\begin{itemize}
%\item{Socrates asks ``Is the pious loved by the gods because it's pious? Or is it pious because it's loved?''}
%\item{Distinction between something's being \emph{X}ed and something's \emph{X}ing (e.g. something's being carried and something's carrying; being led and leading; being seen and seeing)}

%\begin{itemize}\item{S claims that something \emph{X}ing \emph{Y} is prior to \emph{Y}'s being \emph{X}ed}
%\begin{itemize}\item{E.g. If we ask ``Why is this piece of chalk being carried (led, seen)?'' the answer is ``Because Professor O'Connor is carrying (leading, seeing) it''}\item{If we ask ``Why is Professor  O'Connor carrying this piece of chalk?'' the answer is \emph{not} ``Because the piece of chalk is being carried by Professor O'Connor''}\end{itemize}\end{itemize}
%\item{So, by analogy, something's being loved by the gods is \emph{posterior} to the gods' loving it (i.e. the fact that the gods love it explains why it is being loved, not the other way around)}

%\item{If we ask, ``Why do the gods love it?'', the answer is, ``Because it is pious'' and \emph{not} ``Because it is loved by the gods'' (remember the carrying of the chalk: the answer to ``Why is Professor O'Connor carrying the chalk'' was not ``Because the chalk is being carried by Professor O'Connor'')}

%\item{But, as we saw above, the answer to ``Why is it god-loved?'' is ``Because the Gods love it''} \item{But if ``To be pious'' and ``To be God-loved'' were the same thing (as E claims), then:}\begin{itemize}\item{If the pious were loved because it's pious, [by substituting identicals], the god-loved would be loved because it's god-loved}\item{But, we've shown that to be impossible}\end{itemize}\item{Therefore: To be pious $\neq$ To be dear to the gods}

%\end{itemize}



\section*{Final Result}
 X is defined as AB if and only if: 
\begin{description}
\item[General:] everything which is X is also AB. If piety is defined as being God loved, then everything which is pious must also be God loved. If, then, sacrificing is pious and praying is pious, then both sacrificing and praying must be God loved. 
\item[Univocal:] everything which is AB is also X. If piety is defined as being God loved, then everything which is God loved must also be pious. If, then, killing your enemy is battle is God loved, killing your enemy must also be pious. 
\item[Explanatory:]  every instance of X is so because it has characteristics AB. If piety is defined as being God loved, then a pious act, say, praying to the Gods, must be pious precisely because it is something loved by the Gods. 
\end{description}

\end{document}



