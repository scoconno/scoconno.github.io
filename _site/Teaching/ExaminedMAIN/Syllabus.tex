\documentclass[11pt,article,oneside]{memoir}

%%% custom style file with standard settings for xelatex and biblatex. Note that when [minion] is present, we assume you have minion pro installed for use with pdflatex.
%\usepackage[minion]{org-preamble-pdflatex} 

%%% alternatively, use xelatex instead
\usepackage{org-preamble-xelatex} 



\def\myauthor{Author}
\def\mytitle{Title}
\def\mycopyright{\myauthor}
\def\mykeywords{}
\def\mybibliostyle{plain}
\def\mybibliocommand{}
\def\mysubtitle{}
\def\myaffiliation{NJCU}
\def\myaddress{101 Phil}
\def\myemail{soconnor@njcu.edu}
\def\myweb{scottoconnor.org}
\def\myphone{}
\def\myversion{}
\def\myrevision{}
\def\myaffiliation{NJCU}
\def\myauthor{Dr. Scott O'Connor}
\def\mykeywords{}
\def\mysubtitle{Syllabus}
\def\mytitle{{\normalsize Phil 140, 3 credits, Fall 2015, NJCU Harborside. \newline} \HUGE The Examined Life}


\begin{document}

%%% If using xelatex and not pdflatex
%%% xelatex font choices
\defaultfontfeatures{}
\defaultfontfeatures{Scale=MatchLowercase}    
% You will need to buy these fonts, change the names to fonts you own, or comment out if not using xelatex.      
\setromanfont[Mapping=tex-text]{Georgia} 
\setsansfont[Mapping=tex-text]{Georgia} 
\setmonofont[Mapping=tex-text,Scale=0.8]{Georgia} 

%% blank label items; hanging bibs for text
%% Custom hanging indent for vita items
\def\ind{\hangindent=1 true cm\hangafter=1 \noindent}
\def\labelitemi{$\cdot$}
%\renewcommand{\labelitemii}{~}

%% RCS info string for version tracking
\chapterstyle{article-3}  % alternative styles are defined in latex-custom-kjh/needs-memoir/
\pagestyle{kjh}

\title{\LARGE \mytitle}     
\author{\Large\myauthor \newline \footnotesize\texttt{\noindent\myemail}}
\date{09/01/2015--12/08/2015}

\published{\,}

\maketitle

%\thispagestyle{kjhgit}

% Copyright Page
%\textcopyright{} \mycopyright


%
% Main Content
%

\section{Copyright}
The materials used in this class, including, but not limited to, lectures, exams, quizzes, and homework assignments are copyright protected works.  Any unauthorized copying of the class materials or recording of lectures is a violation of federal law and may result in disciplinary actions being taken against the student.  Additionally, the sharing of class materials without the specific, express approval of the instructor may be a violation of the University's Student Honor Code and an act of academic dishonesty, which could result in further disciplinary action.  This includes, among other things, uploading class materials to websites for the purpose of sharing those materials with other current or future students. 

\section{Catalog Description}

This course teaches students to identify and evaluate those beliefs that guide their thoughts and actions. Reflecting on different sources, students identify those philosophical beliefs that play a role in their own lives. By developing their critical thinking skills, they learn how to clarify, systematize, and assess these beliefs. 

\section{Course Description}

Does God exist? Are you free? Why live? What should you do with your life?  In this
course, we'll be asking some of these deep philosophical questions. We
will begin by examining two classic arguments for the existence of God
as well as concerns that God's existence is incompatible with the
existence of evil. In the second part of the course, we will ask what determines the moral character of our actions. Do the ends justify the means? From there, we will discuss whether we are in control of our own destinies or whether our actions are completely pre-determined by causal factors outside of our control. In the final part of the course, we will discuss the meaning of life, especially why some philosophers have connected a meaningful life with God's existence.

\section{Learning Objectives}

Upon completing this course, students will be able to (i) read
philosophical texts, (ii) clearly and charitably explain viewpoints that
are not their own, (iii) think critically and philosophically, (iv)
write well-structured prose in which they clearly state a thesis and
persuasively defend it, (v) demonstrate an understanding of several core
philosophical topics.

\section{Required Textbook}

By 09/08/2015, the following textbook must be purchased or rented:

\begin{itemize}
\item
  \href{http://www.amazon.com/Philosophy-Here-Now-Powerful-Everyday/dp/0199765227}{Philosophy
  Here and Now: Powerful Ideas in Everyday Life', by Lewis Vaughn}
  (Available in the campus book store and online retailers)
\end{itemize}

\section{Requirements}

\begin{itemize}
\item \textit{Workload:} Successfully completing this course requires a minimum commitment of 6 hours per week. 

\item \textit{Attendance:} Roll call will be taken from Week 2. 1 point per class up to a maximum of 10 points. Excludes Week 1 and Week in which you present.

\item \textit{8 Reading Quizzes} administered through Blackboard on weeks when no other written assignment is due. Once you open a quiz, you will have 30 minutes to complete it. This is an automatic function performed by Blackboard. 

\item \textit{2 Short Writing Assignments} submitted through Blackboard. 

\item \textit{2 Essays} submitted through Blackboard. 1000 and 1500 words respectively.

\item \textit{Final Presentation:} One group oral presentation and written submission. Each member must present at least one slide. Groups will grade their member's participation. 

\item \textit{Grade Distribution:}  Attendance---1 point per class (10 total), Quizzes---2 points each (16 total);  Short Writing Assignments---7 points each (14 total); Essays---20 points each (40 total); Presentation---10 oral, 10 points written, 10 points peer evaluation (30 points total).

\item \textit{Grade Breakdown:}

 \begin{tabular}{ | l | l | p{2cm} | l | l | }
    \hline 
96--110 & A  & &  77--79 &  C+ \\  
90--95 & A- & &  73--76 & C \\
87-89 & B+ &  &  70--72 & C- \\ 
83--86 & B  & &  60--69 & D\\
80--82 & B - & & 0--59 & F\\ \hline
    \end{tabular}


\end{itemize}




\section{Policies}

\begin{itemize}

\item \textit{Student Responsibility:} This syllabus outlines the required text, assignments, requirements, and policies for this course. By taking this course, you agree to read this syllabus and be bound by those requirements and policies. 

\item \textit{Late work \& Make-up Policy:} 
\begin{itemize}
\item All assignments must be submitted through Blackboard by 1:00 pm on the due date (see assignment schedule below).
\item  No make-ups or late work accepted under any circumstances. No exceptions. 
\item Blackboard difficulties are rare and automatically reported to instructors. Under no circumstance will a student's report of a Blackboard difficulty be reason for an extension. It is your responsibility to contact blackboard support for help: \href{dlsupport@njcu.edu}{dlsupport@njcu.edu}. 

\end{itemize}

\item \textit{Attendance:} You are considered absent if you are (i) not present during roll call, or (ii) leave early, or (iii) leave without permission, or (iv) leave for an extended period of time.

\item \textit{Electronic devices:} Use of electronic device, including, but not limited, to smartphones, dictaphones, tablets, laptops, is prohibited. Recording a lecture is in violation of Copyright. Penalties include, but are not limited to, a lost of attendance grade for the day of violation. Repeat offenders will be reported to the Dean of Students. 

\item \textit{Conduct:} Distracting and disrespectful behavior, including but not limited to eating, leaving your seat, talking out of turn, aggression is prohibited. Penalties include, but are not limited to, a lost of attendance grade for the day of violation. Repeat offenders will be reported to the Dean of Students. 

\item \textit{Communication:} All communication will be through Blackboard. Messages will be responded to within two days of receiving them. 

\item \textit{Grading Schedule:} Grades will be available within 1 week of an assignment being submitted.

\item \textit{Statement for students with disabilities:} If you are a student
with a disability and wish to receive consideration for reasonable
accommodations, please register with the Office of Specialized Services
and Supplemental Instruction (OSS/SI). To begin this process, complete
the registration form available on the OSS/SI website at
\href{http://www.njcu.edu/Specialized_Services.aspx}{www.njcu.edu/Specialized\_Services.aspx}
(listed under Student Resources-Forms). Contact OSS/SI at 201-200-2091
or visit the office in Karnoutsos Hall, Room 102 for additional
information.
\end{itemize}

\section{Plagiarism}

\begin{itemize} 
\item You are bound by \href{http://www.njcu.edu/uploadedFiles/About_NJCU/Governance_and_Organization/University_Senate/Policies/Academic\%20INTEGRITY\%20POLICY\%20FINAL\%202-04.pdf}{NJCU's Academic Integrity Policy}
\item Penalty for plagiarism:
\begin{itemize}
\item 1st infraction: 0 for the assignment. 
\item 2nd infraction: 0 for the entire course \& application for permanent record on student's transcript. (Repeated violations can lead to expulsion from NJCU). 
\end{itemize}
\end{itemize}


\section{Weekly Course Schedule}
Dates refer to the first day of the week: 
\begin{enumerate}

\item \textit{08/31/2015,} Introduction \textbf{Class begins Wed 2nd)
\begin{enumerate}
\item Ch.1.1--1.2.
\end{enumerate}

\item \textit{09/07/2015,} Critical Thinking: Barriers to Reflection (no class on Mon)
\begin{enumerate}
\item `Orchestrating Impartiality: The Impact of `'Blind'' Auditions on Female Musicians',  Claudia Goldin and Cecilia Rouse (online)
\item `In a Different Voice',  Carol Gilligan (online)
\end{enumerate}

Move everything up a week

\item \textit{09/14/2015,} Critical Thinking: Logical Tools 
\begin{enumerate}
\item Ch.1.3
\item Handout
\end{enumerate}

\item \textit{09/21/2015,} The Meaning of Life: Pessimism 
\begin{enumerate}
\item Ch. 8. 
\item `Crimes and Misdemeanors' (movie)?
\item `A Confession', Leo Tolstoy (online)**
\end{enumerate}

\item \textit{09/28/2015,} The Meaning of Life: Optimism
\begin{enumerate}
\item Ch. 8. 
\item Epicurus, selections (online)
\end{enumerate}

\item \textit{10/05/2015,} God: The Design Argument
\begin{enumerate}
\item Ch.2.1-2.2. 
\item Short YouTube videos on Natural Selection and Evolution.
\item `Intelligent Design Has No Place in the Science Curriculum', Harold Morowitz, Robert Hazen, and James Trefil (online)
 \item `Design for Living', Michael J. Behe (online)
 \end{enumerate}
 
\item \textit{10/12/2015,} God: Further Arguments for God's Existence
\begin{enumerate}
\item Ch.2.2
\item Perhaps something about how we currently understand the beginning of the Cosmos
\end{enumerate}
\item \textit{10/19/2015,} God: The Problem of Evil 
\begin{enumerate}
\item Ch.2.3.
\item Picasso's Guernica (painting) [or some other depiction of evil]
\item Some movie or short fiction that illustrates the problem
\end{enumerate}
\item \textit{10/26/2015,} Free Will: Psychological Problems
\begin{enumerate}
\item Ch.6
\item Reading from psychology literature, maybe Stanford Prison Experiments
\item Movie, fiction, about environments shaping our action
\end{enumerate}
\item \textit{11/02/2015,} Free Will: Causal Determinism
\begin{enumerate}
\item Ch.6
\item Fiction, movie about fate, destiny.
\end{enumerate}
\item \textit{11/09/2015,} Ethics: Relativism, 
\begin{enumerate}
\item Ch.3.1--3.2.
\item `Anthropology and the Abnormal', Ruth Benedict (online)
\end{enumerate}
\item \textit{11/16/2015,} Ethics: Consequentialism. 
\begin{enumerate}
\item Ch.3.3
\item Sections from the Patriot Act (online)
\item  `The Ones Who Walk Away from Omelas', Ursula LeGuin, (textbook)
\end{enumerate}
\item \textit{11/23/2015,} Ethics: Deontology and Virtue Ethics
\begin{enumerate}
\item Ch.3.4--3.5
\item The Walking Dead, (movie clip)
\end{enumerate}
\item \textit{11/30/2015,} Applied Ethics: Student Presentations
\item \textit{12/07/2015,} Applied Ethics: Student Presentations
\item \textif{12/14/2015,} Applied Ethics: Student Presentations
\end{enumerate}







\section{Writing Assignment Schedule}
Dates refer to the due date. All assignments must be submitted through Blackboard by 1:00pm. No late work accepted. No make-ups. No exceptions. 

\begin{enumerate}
\item \textit{09/07/2015,} No Assignment
\item \textit{09/14/2015,} Quiz 1 
\item \textit{09/21/2015,} Quiz 2
\item \textit{09/28/2015,} SW1--Meaning of Life 
\item \textit{10/05/2015,} Quiz 3
\item \textit{10/12/2015,} SW2--God's Existence.
\item \textit{10/19/2015,} Quiz 4
\item \textit{10/26/2015,} Essay 1--The Problem of Evil
\item \textit{11/02/2015,} Quiz 5
\item \textit{11/09/2015,} Quiz 7 
\item \textit{11/16/2015,} Quiz 8
\item \textit{11/23/2015,} Essay 2--Free will
\item \textit{11/22/2015,} Presentations--Applied Ethics
\item \textit{12/07/2015,} Presentations--Applied Ethics
\item \textit{12/14/2015,} Presentations--Applied Ethics
\end{enumerate}

\newpage
\section{Grading Rubric}

\begin{center}


\resizebox{14cm}{!} {
    \begin{tabular}{ | l | l | l | l | l | l | l | l | l |    }
    \hline
    Rubric &  V. Good & Good & OK & Weak & Poor & V. Poor & No Attempt \\  
	 & (10) & (9) & (8) & (7) & (6) & (3)  & (0) \\    \hline
    Grasp of Material (X2) & & & & & & &\\ \hline   
    Application of logical tools &  & & & & & &\\ \hline
	Independent thought &  & & & & & &\\ \hline
 Mechanics & & & & & &  &\\ \hline \hline
 & &  & & &  & \textbf{Total}  & \\ \hline
    \end{tabular}
}


\end{center}


% \subsection*{Explanation}


%% Uncomment if you want a printed bibliography.
%\printbibliography 

\end{document}