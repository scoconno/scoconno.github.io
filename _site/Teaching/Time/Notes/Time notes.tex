\documentclass[11pt]{article}

%\usepackage{fontspec}
%\defaultfontfeatures{Mapping=tex-text}
%\setmainfont[BoldFont={Minion Pro Bold}]{Minion Pro}
\usepackage{pdfpages}
%\usepackage[usenames,dvipsnames]{color}
\usepackage{graphicx}
\usepackage{fullpage}


\begin{document}

\section{My notes}
\begin{itemize}
\item The Sumarians, c. 4000 BC, used the the duodecimal (based on 12) and sexagesimal (based on 60) for counting.
\item Babylonians divided a circle into 360 degrees. Note the 60.  
\item Egyptians divided daylight into 12 units. We would call each a perior, or what is now an hour.
\item That led to a problem with night time. They eventually noticed that certain stars appear first in the horizon and came to divide the night into 12 hours. 
\item Greeks are strugglin. Athenians need to tell time. How to coordinate an empire. How to coordinate night hours and day hours.
\item Alexander the great around 320 BC begins conquest of known world. Leads to mass communication.
\item Hiparchus 190-120BC. He relied on Babylonian information.
\item An equinox occurs when night and day are of equal lenght. The division in night time and day time were equal.
\item Eratosthenes in 3rd century came up with longitude and lattitude. But Hipparchus used it to identify any place on the map.
\item Latitude is measured from equator to 90o North and 90o South Pople. 
\item Longitude is measured in 180degrees east or west from Greenwich. 
\item It provides a grid that you can then find anywhere on Earth. 
\item Obviously the degrees are too big. Ptolemy invented cuts of these degrees.  Claudius Ptolemy 200 years after Hipparchus wrote The Almagest and Geographia/ So he divided each degree into 60. The  pars minuta prima--each minute into 60, pars minute secunda. He also divided 3rd and 4th. 
\item Rogert Bacon in 1267 defined the time between full moons in terms of hours, minutes, second, thirds and forths after noon on specific days. But this was abstract. No way of measuring how many minutes and seconds have passed.
\item But we still can't use this for time. Fast forward to clocks. From the Celtic clocca, bell. Or Cloc in German, bell. 
\item Mechanical clocks that ring the bell 5 times to day to tell you when to pray. In towers. No clock face. Any other clocks only have hours.
\item Pendulum. Late 1500s Gaillielo. The time it takes to swing is not dependent on its weight, but on its length. 1656 Christian Hyuges invest pendulum Holland. Towards 1690, we see first clock face with hands and minutes. 
\item There is no accurate timekeeping, but standaridization. Problems with weather forecasting and trains. Each city and town would be on own time. You set your clock to their standard.
\item 1860 about 70 time zones in America. 
\item 1883 Rial roads establish 4 time zones. 1918 fed puts the 4 into law. 
\end{itemize} 
\end{document}
