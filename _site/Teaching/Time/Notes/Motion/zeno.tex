\documentclass[]{article}
\usepackage{amssymb,amsmath}
\usepackage{ifxetex,ifluatex}
\usepackage{fixltx2e} % provides \textsubscript
\ifnum 0\ifxetex 1\fi\ifluatex 1\fi=0 % if pdftex
  \usepackage[T1]{fontenc}
  \usepackage[utf8]{inputenc}
\else % if luatex or xelatex
  \ifxetex
    \usepackage{mathspec}
    \usepackage{xltxtra,xunicode}
  \else
    \usepackage{fontspec}
  \fi
  \defaultfontfeatures{Mapping=tex-text,Scale=MatchLowercase}
  \newcommand{\euro}{€}
\fi
% use upquote if available, for straight quotes in verbatim environments
\IfFileExists{upquote.sty}{\usepackage{upquote}}{}
% use microtype if available
\IfFileExists{microtype.sty}{%
\usepackage{microtype}
\UseMicrotypeSet[protrusion]{basicmath} % disable protrusion for tt fonts
}{}
\ifxetex
  \usepackage[setpagesize=false, % page size defined by xetex
              unicode=false, % unicode breaks when used with xetex
              xetex]{hyperref}
\else
  \usepackage[unicode=true]{hyperref}
\fi
\hypersetup{breaklinks=true,
            bookmarks=true,
            pdfauthor={Scott O'Connor},
            pdftitle={Zeno 1},
            colorlinks=true,
            citecolor=blue,
            urlcolor=blue,
            linkcolor=magenta,
            pdfborder={0 0 0}}
\urlstyle{same}  % don't use monospace font for urls
\setlength{\parindent}{0pt}
\setlength{\parskip}{6pt plus 2pt minus 1pt}
\setlength{\emergencystretch}{3em}  % prevent overfull lines
\setcounter{secnumdepth}{0}

\title{Zeno 1}
\author{Scott O'Connor}
\date{}

\begin{document}
\maketitle

\subsubsection{Motion does not exist}\label{motion-does-not-exist}

\begin{enumerate}
\def\labelenumi{\arabic{enumi}.}
\item
  Space is infinitely divisible or not infinitely divisible.
\item
  If space is infinitely divisible, motion is impossible.
\item
  If space is not infinitely divisible, motion is impossible.
\item
  Motion is impossible (From 1-3).
\end{enumerate}

\section{Premise 1 - the divisibility of
space}\label{premise-1---the-divisibility-of-space}

\begin{itemize}
\item
  If x is infinitely divisible, x can be divided into ever smaller parts
  \emph{ad infinitum}. In other words, x contains no indivisible parts,
  i.e.~parts that cannot further be divided.

  \begin{itemize}
  \item
    For example, suppose that a line, L, is infinitely divisible. Lines
    are divided into line segments. So every line segment of L can be
    divided into further smaller line segments - there is no smallest
    line segment.
  \item
    Think of this process of dividing something out as merely
    conceptual. Don't worry whether or not we could literally do
    something to x to divide it in this way.
  \end{itemize}
\item
  If x is not infinitely divisible, x can be divided into a finite
  number of \emph{smallest} parts, i.e.~parts that cannot be divided
  into any smaller parts.

  \begin{itemize}
  \itemsep1pt\parskip0pt\parsep0pt
  \item
    For example, suppose that a line, L, is not infinitely divisible.
    Then L contains a finite number of smallest line segments, i.e.,
    line segments with some smallest extent that cannot be divided into
    any further line segments.
  \end{itemize}
\end{itemize}

\section{Premise 2}\label{premise-2}


\emph{Strategy:} Assume that space is infinitely divisible. Then argue
that it is impossible to move from one place to another by showing that
(a) doing so requires completing an infinite number of tasks, and (b) it
is impossible to complete an infinite number of tasks.

Zeno argues for premise 2 by using a number of paradoxes. The first is
called \emph{Racecourse}, which argues that it is impossible to complete
an arbitrary journey from A to B - to start at A, move to B, and then
stop.

\begin{quote}
The first asserts the non-existence of motion on the ground that that
which is in locomotion must arrive at the half-way stage before it
arrives at the goal. (Aristotle Physics, 239b11)
\end{quote}

\begin{enumerate}
\def\labelenumi{\Alph{enumi}.}
\item
  The distance between A and B is infinitely divisible (assumed).
\item
  A journey from A to B is a series of sub-journeys with no last member:
  from A to \(\frac{1}{2}AB\), from \(\frac{1}{2}AB\) to
  \(\frac{3}{4}AB\), and so on.
\item
  It is impossible to complete a series of sub-journeys with no last
  member.
\item
  Completing a journey from A to B, requires completing the series of
  sub-journeys with no last member: from A to \(\frac{1}{2}AB\), from
  \(\frac{1}{2}AB\) to \(\frac{3}{4}AB\), and so on.
\item
  It is impossible to complete the journey from A to B.
\end{enumerate}

\begin{itemize}
\itemsep1pt\parskip0pt\parsep0pt
\item
  An inverted version of the paradox shows us that our traveler cannot
  begin to move.
\item
  A different paradox, the Achilles paradox, shows us that in a race
  between Achilles and a tortoise, where the tortoise is given a head
  start, Achilles could never catch-up and pass the tortoise.
\end{itemize}

\begin{quote}
The {[}second{]} argument was called ``Achilles,'' accordingly, from the
fact that Achilles was taken {[}as a character{]} in it, and the
argument says that it is impossible for him to overtake the tortoise
when pursuing it. For in fact it is necessary that what is to overtake
{[}something{]}, before overtaking {[}it{]}, first reach the limit from
which what is fleeing set forth. In {[}the time in{]} which what is
pursuing arrives at this, what is fleeing will advance a certain
interval, even if it is less than that which what is pursuing advanced
\ldots{} . And in the time again in which what is pursuing will traverse
this {[}interval{]} which what is fleeing advanced, in this time again
what is fleeing will traverse some amount \ldots{} . And thus in every
time in which what is pursuing will traverse the {[}interval{]} which
what is fleeing, being slower, has already advanced, what is fleeing
will also advance some amount. (Simplicius(b) On Aristotle's Physics,
1014.10)
\end{quote}

\subsubsection{Response 1: Reject C}\label{response-1-reject-c}

Some deny premise C by claiming that as we divide the distances of the
journey, we should also divide the total time taken, and, further, that
the sum of these infinite series of decreasingly short time intervals is
still equal to a finite period of time. These denials assume that the
argument for C is the following:

\begin{itemize}
\item
  C1. Completing an infinite series of tasks would take an infinite
  amount of time.
\item
  C2. It is not possible to spend an infinite amount of time completing
  some task(s).
\item
  C. Therefore, it is not possible to complete an infinite series of
  tasks.
\end{itemize}

This argument for Premise C is valid, but some deny that it is not sound
because C1 is false. They claim that it relies on the false assumption
that completing an infinite series of tasks would take an infinite
period of time. This seems false. As we divide the distances between the
points we travel, we should also divide the time it takes to travel the
ever smaller distances:

\begin{itemize}
\item
  It takes \(\frac{1}{2}\) the time to run from A to \(\frac{1}{2}AB\)
  as it does to run from A to B.
\item
  It takes \(\frac{1}{4}\) of the time to run from \(\frac{1}{2}AB\) to
  \(\frac{3}{4}AB\), and so on.
\item
  The sum of these decreasing times is finite.\footnote{See `A
    Contemporary Look at Zeno's Paradoxes', by Wesley Salmon}
\item
  Therefore, we can complete an infinite series of sub-journeys in a
  finite period of time.
\end{itemize}

Let us grant that C1-C2 fail to establish Premise C. There is an
alternative way of defending Premise C that is immune to our current
objection:

\begin{itemize}
\itemsep1pt\parskip0pt\parsep0pt
\item
  Even if it takes less time to complete each sub-journey, I still need
  to first complete each sub-journey before completing the journey that
  comes after it. If so, I always have one more sub-journey to complete
  before I can complete the final step.
\end{itemize}

Note that the response assumes that time is infinitely divisible, i.e.
divisible into a infinite number of finite parts. 

\subsubsection{Response 2: Reject B}\label{response-2-reject-b}

Aristotle claims that a journey from A to B is a series of
\emph{potential} and not \emph{actual} sub-journeys, i.e.~the full
journey does not consist of actual parts each of which are sub-journeys.
The point here is that Zeno is defining journeys in terms of the space
over which you travel. Aristotle responds by denying that journeys are
individuated in this way.

In order to evaluate this response, we need to investigate the nature of
actions/activities and try to understand what is involved in completing
them. In other words, if journeys are not individuated by the distance
over which they occur, how are they individuated?





\section{Premise 3}\label{premise-3}

Recall that premise 3 states:   If space is not infinitely divisible, motion is impossible. Zeno offers two distinct arguments for Premise
3 that again come in the form of paradoxes. The strategy for each is
similar. We will fist assume that space is not infinitely divisible,
then prove that certain absurdities follow. If an assumption leads to an
absurdity, we know the assumption is false.

\subsubsection{Stadium Paradox}\label{stadium-paradox}

\begin{quote}
The fourth argument is that concerning equal bodies which move alongside
equal bodies in the stadium from opposite directions---the ones from the
end of the stadium, the others from the middle---at equal speeds, in
which he thinks it follows that half the time is equal to its
double\ldots{}. (Aristotle Physics, 239b33)
\end{quote}

Suppose these rows of blocks represent some chariots in a stadium. The
B's are stationary. The A's are moving from left to right. The last
block in that row is called D. The Cs are moving towards from right to
left. The middle block in that row is called E.

{[}c{]}{@lll@} \& T1 \&\& DAA \& --\textgreater{}\& BBB \&\textless{}--
\& CEC \&

{[}c{]}{@rll@} \& T3 \&--\textgreater{} \& ~ DA \& A\& BBB \&~ ~ C \&
EC~ \& \textless{}--

Compare Times 1 and 3. Suppose they are separated by a one minute
interval. In this interval, D has passed one B block and two C blocks.
Zeno thinks this is paradoxical. It's unclear why. For our purposes, let
us assume the following:

\begin{enumerate}
\def\labelenumi{\arabic{enumi}.}
\item
  There is smallest possible length, \emph{S}
\item
  The length of each block is S.
\item
  There are no gaps between the blocks.
\item
  The blocks move with constant velocity.
\end{enumerate}

It took 1 minute for D to pass two C blocks. It should take 30 seconds
to pass one C block and become level with E. Suppose D passes one C
block after 30 seconds. How many B blocks has it passed? Try filling out
the diagram below to answer that question.

{[}c{]}{@rll@} \& T2 \&--\textgreater{} \& DAA \&\& ? \&~ ~ C \& EC~ \&
\textless{}--

T3 This describes the moment that D and E are level. How does D relate
to the B's at this moment?

We are stuck! Suppose that someone claims that D has passed \emph{half
of one B block.} Let this half be called \emph{H}. What is H's length?
You cannot, on pain of contradiction, claim that H has a length less
than S. We have assumed that S is the smallest possible length, so H
cannot be shorter than S.

This way of stating the paradox assumes that the length of time between
T1 and T3 can be divided in two, i.e., 1 minute is divided into two 30
second intervals. Suppose that time is also atomic, that there is a
smallest interval of time, a single quantum of time. Suppose also that
the motion between T1 and T3 takes a single quantum of time. If this is
correct, there is no T2 (which was half the interval between T1 and T3.)

Paradox still threatens. During a single quantum of time, D and E will
have passed each other (as is seen in T3), but there is no moment at
which they are level as is described in T2: since T1 \& T3 are separated
by the smallest possible time, there can be no instant between them---it
would be a time smaller than the smallest time from the two moments we
considered. Conversely, if one insisted that if they pass then there
must be a moment when they are level, then it shows that cannot be a
shortest finite interval.

\subsubsection{The Arrow Paradox}\label{the-arrow-paradox}

\begin{quote}
The third is \ldots{} that the flying arrow is at rest, which result
follows from the assumption that time is composed of moments \ldots{} .
he says that if everything when it occupies an equal space is at rest,
and if that which is in locomotion is always in a now, the flying arrow
is therefore motionless. (Aristotle Physics, 239b.30)
\end{quote}

\begin{quote}
Zeno abolishes motion, saying ``What is in motion moves neither in the
place it is nor in one in which it is not''. (Diogenes Laertius Lives of
Famous Philosophers, ix.72)
\end{quote}

\subsubsection{Outline of the Paradox}\label{outline-of-the-paradox}

Assume the following claims:

\begin{enumerate}
\def\labelenumi{\arabic{enumi}.}
\itemsep1pt\parskip0pt\parsep0pt
\item
  Space is finitely divisible.
\item
  Time is composed of moment.
\end{enumerate}

\begin{itemize}
\itemsep1pt\parskip0pt\parsep0pt
\item
  P1. An arrow must occupy a space equal to itself at each moment that
  it exists.
\item
  P2. If an arrow moves for, say, 1 minute, the arrow will occupy a
  space equal to itself at each moment that is it moving.
\item
  P3. An arrow that occupies a space equal to itself at a specific
  moment is not moving in that moment in that space.
\item
  P4. If an arrow is not moving
\end{itemize}

Consider an arrow, apparently in motion, at any instant. First, Zeno
assumes that it travels no distance during that moment---`it occupies an
equal space' for the whole instant. But the entire period of its motion
contains only instants, all of which contain an arrow at rest, and so,
Zeno concludes, the arrow cannot be moving.

'' a moving arrow must occupy a space equal to itself during any moment.
That is, during any moment it is at the place where it is. But places do
not move. So, if in each moment, the arrow is occupying a space equal to
itself, then the arrow is not moving in that moment because it has no
time in which to move; it is simply there at the place. The same holds
for any other moment during the so-called ``flight'' of the arrow. So,
the arrow is never moving. Similarly, nothing else moves. The source for
Zeno's argument is Aristotle (Physics, Book VI, chapter 5, 239b5-32).

\subsubsection{The Standard Solution}\label{the-standard-solution}

The standard solution to the paradox uses the ``at-at'' theory of
motion:

\begin{itemize}
\item
  \begin{enumerate}
  \def\labelenumi{(\roman{enumi})}
  \itemsep1pt\parskip0pt\parsep0pt
  \item
    being in motion involve being at different places at different
    times, and (ii) being at rest involves being motionless at a
    particular point at a particular time.
  \end{enumerate}
\end{itemize}

This theory asks us to distinguish two things:

\begin{enumerate}
\def\labelenumi{\alph{enumi})}
\itemsep1pt\parskip0pt\parsep0pt
\item
  being in motion in or during an instant.
\item
  being in motion at an instant.
\end{enumerate}

The at-at theory accepts that the arrow cannot move during an instant,
but claims that the arrow can still move at an instant. It does so by
occupying different locations before and after that instant.

If this is correct, the difference between rest and motion has to do
with what is happening at nearby moments and has nothing to do with what
is happening during a moment. The arrow counts as moving at an instant
because it occupies different locations before and after that instant.
The arrow counts as being at rest at an instant because it is not
located at different locations before and after that instant.

\subsubsection{Calculus}\label{calculus}

The issue is acceleration!!!

The instant must be part of a period in which the arrow is continuously
in motion. instantaneous motion from instantaneous rest.

The Arrow Paradox seems especially strong to someone who would say that
motion is an intrinsic property of an instant, being some propensity or
disposition to be elsewhere.

Calculus: speed of an object at an instant (instantaneous velocity) is
the time derivative of the object's position. This means the object's speed is the limit of its speeds during
arbitrarily small intervals of time containing the instant.

The object's speed is the limit of its speed over an interval as the
length of the interval tends to zero.

The derivative of position x with respect to time t, namely dx/dt, is
the arrow's speed, and it has non-zero values at specific places at
specific instants during the flight. The speed during an instant or in an instant, which is what Zeno is
calling for, would be 0/0 and so be undefined.

Using these modern concepts, Zeno cannot successfully argue that at each
moment the arrow is at rest or that the speed of the arrow is zero at
every instant. Therefore, advocates of the Standard Solution conclude
that Zeno's Arrow Paradox has a false, but crucial, assumption and so is
unsound.

\subsubsection{The Future?}\label{the-future}

\begin{itemize}
\item
  The only way to rescue motion requires that we assumes that facts
  about an objects' location in the future are already fixed.
\item
  Suppose God were to wipe the arrow out of existence completely an
  instant after it moved. This would mean that it was not moving. If the
  arrow, though, is moving this instant, then that means it is not
  destroyed in the next instant. It is true now that the arrow is not
  being destroyed in the future!
\end{itemize}

\end{document}
