\documentclass[]{article}

\usepackage{amssymb,amsmath}
\usepackage{ifxetex,ifluatex}
\usepackage{fixltx2e} % provides \textsubscript
\ifnum 0\ifxetex 1\fi\ifluatex 1\fi=0 % if pdftex
  \usepackage[T1]{fontenc}
  \usepackage[utf8]{inputenc}
\else % if luatex or xelatex
  \ifxetex
    \usepackage{mathspec}
    \usepackage{xltxtra,xunicode}
  \else
    \usepackage{fontspec}
  \fi
  \defaultfontfeatures{Mapping=tex-text,Scale=MatchLowercase}
  \newcommand{\euro}{€}
\fi
% use upquote if available, for straight quotes in verbatim environments
\IfFileExists{upquote.sty}{\usepackage{upquote}}{}
% use microtype if available
\IfFileExists{microtype.sty}{%
\usepackage{microtype}
\UseMicrotypeSet[protrusion]{basicmath} % disable protrusion for tt fonts
}{}
\ifxetex
  \usepackage[setpagesize=false, % page size defined by xetex
              unicode=false, % unicode breaks when used with xetex
              xetex]{hyperref}
\else
  \usepackage[unicode=true]{hyperref}
\fi
\hypersetup{breaklinks=true,
            bookmarks=true,
            pdfauthor={},
            pdftitle={Short Essay 1},
            colorlinks=true,
            citecolor=blue,
            urlcolor=blue,
            linkcolor=magenta,
            pdfborder={0 0 0}}
\urlstyle{same}  % don't use monospace font for urls
\setlength{\parindent}{0pt}
\setlength{\parskip}{6pt plus 2pt minus 1pt}
\setlength{\emergencystretch}{3em}  % prevent overfull lines
\setcounter{secnumdepth}{0}



\begin{document}


\subsection{Short Essay 1}\label{short-essay-1}

\subsubsection{Prompt}\label{prompt}

Stop the clock! News alert! The prominent scientist Edwin Hubble has
claimed to discover conclusive evidence that the Universe has a
beginning. It is being called in some quarters evidence that the
Universe began with a `Big Bang'. Others reject this wholeheartedly.
They claim that the evidence for the Big Bang shows only that the
Universe once experienced a `Big Crunch' before expanding once again, a
cycle of crunch, bang, crunch and so on that has and will go on for
eternity.

You are a journalist for the NY Times. Your editor sends you the
following assignment:

\begin{quote}
All this bang/crunch talk is getting those science folk in a tizzy. I
can't make heads or tails of it, but our readers want an explanation.
Please write a very accessible piece explaining the supposed evidence
for the Big Bang and the debate over whether it shows that the Universe
has a beginning. Make sure to explain the alternative of
bang/crunch/bang/crunch and so on .

And remember, we need to sell newspapers! When I say `accessible', I
mean that it better be understandable by your average joe, someone with
only a high school diploma.

If you make this simple enough, we might even give you a regular column!
\end{quote}

\subsubsection{Further Instruction}\label{further-instruction}

\begin{itemize}
\itemsep1pt\parskip0pt\parsep0pt
\item
  Write in very simple English prose.\\
\item
  Use vivid examples.
\item
  This essay covers material in pp.73-76 in `Travels in Four
  Dimensions'.
\item
  We will not have discussed these pages before you complete the
  assignment. This is by design. Read only the allocated pages. While
  they are difficulty, reading it carefully will suffice for completing
  the assignment.
\item
  Write 250-500 words. No more. No less.
\item
  Submit your responses by Mon. @ 1pm.
\end{itemize}

\end{document}
