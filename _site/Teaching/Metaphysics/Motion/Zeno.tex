\documentclass[]{article}

\usepackage{fancyhdr}
 \pagestyle{fancy}
\rhead{\textsc{Scott O`Connor}}

\usepackage{lmodern}
\usepackage{amssymb,amsmath}
\usepackage{ifxetex,ifluatex}
\usepackage{fixltx2e} % provides \textsubscript
\ifnum 0\ifxetex 1\fi\ifluatex 1\fi=0 % if pdftex
  \usepackage[T1]{fontenc}
  \usepackage[utf8]{inputenc}
\else % if luatex or xelatex
  \ifxetex
    \usepackage{mathspec}
    \usepackage{xltxtra,xunicode}
  \else
    \usepackage{fontspec}
  \fi
  \defaultfontfeatures{Mapping=tex-text,Scale=MatchLowercase}
  \newcommand{\euro}{€}
\fi
% use upquote if available, for straight quotes in verbatim environments
\IfFileExists{upquote.sty}{\usepackage{upquote}}{}
% use microtype if available
\IfFileExists{microtype.sty}{%
\usepackage{microtype}
\UseMicrotypeSet[protrusion]{basicmath} % disable protrusion for tt fonts
}{}
\ifxetex
  \usepackage[setpagesize=false, % page size defined by xetex
              unicode=false, % unicode breaks when used with xetex
              xetex]{hyperref}
\else
  \usepackage[unicode=true]{hyperref}
\fi
\usepackage[usenames,dvipsnames]{color}
\hypersetup{breaklinks=true,
            bookmarks=true,
            pdfauthor={},
            pdftitle={Zeno 2},
            colorlinks=true,
            citecolor=blue,
            urlcolor=blue,
            linkcolor=magenta,
            pdfborder={0 0 0}}
\urlstyle{same}  % don't use monospace font for urls
\usepackage{longtable,booktabs}
\setlength{\parindent}{0pt}
\setlength{\parskip}{6pt plus 2pt minus 1pt}
\setlength{\emergencystretch}{3em}  % prevent overfull lines
\providecommand{\tightlist}{%
  \setlength{\itemsep}{0pt}\setlength{\parskip}{0pt}}
\setcounter{secnumdepth}{0}

\title{Zeno 2}
\author{Scott O’Connor}


% Redefines (sub)paragraphs to behave more like sections
\ifx\paragraph\undefined\else
\let\oldparagraph\paragraph
\renewcommand{\paragraph}[1]{\oldparagraph{#1}\mbox{}}
\fi
\ifx\subparagraph\undefined\else
\let\oldsubparagraph\subparagraph
\renewcommand{\subparagraph}[1]{\oldsubparagraph{#1}\mbox{}}
\fi

\begin{document}
\maketitle

\subsubsection{Motion does not exist}\label{motion-does-not-exist}

\begin{enumerate}
\def\labelenumi{\arabic{enumi}.}
\item
  Space is infinitely divisible or not infinitely divisible.
\item
  If space is infinitely divisible, motion is impossible.
\item
  If space is not infinitely divisible, motion is impossible.
\item
  Motion is impossible (From 1-3).
\end{enumerate}

\subsubsection{Premise 1 - the divisibility of
space}\label{premise-1---the-divisibility-of-space}

\begin{itemize}
\item
  If x is infinitely divisible, x can be divided into ever smaller parts
  \emph{ad infinitum}. In other words, x contains no indivisible parts,
  i.e.~parts that cannot further be divided.

  \begin{itemize}
  \item
    For example, suppose that a line, L, is infinitely divisible. Lines
    are divided into line segments. So every line segment of L can be
    divided into further smaller line segments - there is no smallest
    line segment.
  \item
    Think of this process of dividing something out as merely
    conceptual. Don't worry whether or not we could literally do
    something to x to divide it in this way.
  \end{itemize}
\item
  If x is not infinitely divisible, x can be divided into a finite
  number of \emph{smallest} parts, i.e.~parts that cannot be divided
  into any smaller parts.

  \begin{itemize}
  \tightlist
  \item
    For example, suppose that a line, L, is not infinitely divisible.
    Then L contains a finite number of smallest line segments, i.e.,
    line segments with some smallest extent that cannot be divided into
    any further line segments.
  \end{itemize}
\end{itemize}

\subsubsection{Premise 3}\label{premise-3}

This handout will proceed by discussing Premise 3. See Handout 1 for
discussion of Premise 2. Zeno offers two distinct arguments for Premise
3 that again come in the form of paradoxes. The strategy for each is
similar. We will fist assume that space is not infinitely divisible,
then prove that certain absurdities follow. If an assumption leads to an
absurdity, we know the absurdity false.

\subsubsection{Stadium Paradox}\label{stadium-paradox}

\begin{quote}
The fourth argument is that concerning equal bodies which move alongside
equal bodies in the stadium from opposite directions---the ones from the
end of the stadium, the others from the middle---at equal speeds, in
which he thinks it follows that half the time is equal to its
double\ldots{}. (Aristotle Physics, 239b33)
\end{quote}

Suppose these rows of blocks represent some chariots in a stadium. The
B's are stationary. The Bs are stationary. The A's are moving towards
the Bs from left to right. The Cs are moving towards the Bs from right
to left. Suppose also that the As and Cs are traveling at the same
speed.

\begin{longtable}[c]{@{}lll@{}}
\toprule
& T1 &\tabularnewline
\midrule
\endhead
& DAA & --\textgreater{}\tabularnewline
& BBB &\tabularnewline
\textless{}-- & CEC &\tabularnewline
\bottomrule
\end{longtable}

\begin{longtable}[c]{@{}rll@{}}
\toprule
& T2 &\tabularnewline
\midrule
\endhead
--\textgreater{} & ~ DA & A\tabularnewline
& BBB &\tabularnewline
C & EC~ & \textless{}--\tabularnewline
\bottomrule
\end{longtable}

Compare times 1 and 2. Suppose it took 1 minute. How many blocks has D
passed between these two times? It has passed one B block and two C
blocks. Zeno thinks this is paradoxical. It's unclear why. For our
purposes, let us assume the following:

\begin{enumerate}
\def\labelenumi{\arabic{enumi}.}
\tightlist
\item
  There is smallest possible length, \emph{S}
\item
  The length of each block is S.
\item
  There are no gaps between the blocks.
\item
  The blocks move with constant velocity.
\end{enumerate}

It took 1 minute for D to pass two C blocks. It should take 30 second to
pass one C block. Suppose D passes one C block after 30 seconds. How
many B blocks has it passed? Try filling out the diagram below to answer
that question.

\begin{longtable}[c]{@{}rll@{}}
\toprule
& T3 &\tabularnewline
\midrule
\endhead
--\textgreater{} & DAA &\tabularnewline
& ? &\tabularnewline
C & EC~ & \textless{}--\tabularnewline
\bottomrule
\end{longtable}

This describes the time that D and E are level.

We are stuck! Suppose that someone claims that D has passed \emph{half
of one B block.} Let this half be called \emph{H}. What is H's length?
You cannot, on pain of contradiction, claim that H has a length less
than S. We have assumed that S is the smallest possible length, so H
cannot be shorter than S.

This way of stating the paradox assumes that the length of time between
T1 and T2 can be divided in two, i.e., 1 minute is divided into two 30
second intervals. Suppose that time is also atomic, that there is a
smallest interval of time, a single quantum of time. Suppose also that
the motion between T1 and T2 takes a single quantum of time. If this is
correct, there is no T2 (which was half the interval between T1 and T2.
)Paradox still threatens. During a single quantum of time, D and E will
have passed each other (as is seen in T2), but there is no moment at
which they are level as is described in T3: since the two moments are
separated by the smallest possible time, there can be no instant between
them---it would be a time smaller than the smallest time from the two
moments we considered. Conversely, if one insisted that if they pass
then there must be a moment when they are level, then it shows that
cannot be a shortest finite interval---whatever it is, just run this
argument against it.

\subsubsection{The Arrow Paradox}\label{the-arrow-paradox}

\begin{quote}
The third is \ldots{} that the flying arrow is at rest, which result
follows from the assumption that time is composed of moments \ldots{} .
he says that if everything when it occupies an equal space is at rest,
and if that which is in locomotion is always in a now, the flying arrow
is therefore motionless. (Aristotle Physics, 239b.30)
\end{quote}

\begin{quote}
Zeno abolishes motion, saying ``What is in motion moves neither in the
place it is nor in one in which it is not''. (Diogenes Laertius Lives of
Famous Philosophers, ix.72)
\end{quote}

\begin{enumerate}
\def\labelenumi{\arabic{enumi}.}
\tightlist
\item
  Suppose that space is not infinitely divisible.
\item
  If space is not infinitely divisible, then there is a finite number of
  smallest regions of spaces between two points, A and B.
\item
  There is a finite number of smallest regions of space between A and B.
\item
  If an arrow moves from A to B from t1 to tn, at some time between t1
  to tn, it must occupy each region of space between A and B.
\item
  If an arrow occupies a region of space at a time, then it is at rest
  in that moment in that region.
\item
  An arrow at rest in different regions of space at different times,
  does not move.
\item
  The arrow does not move.
\end{enumerate}

This argument is valid and straightforward. It relies on the intuitive
notion that a whole, w, cannot have a property P, when it's parts have
some contradictory property, Q. If each part of the table is blue, the
whole table cannot be red. Similarly, if the parts of the arrows flight
are parts where the arrow is at rest, then the whole flight cannot be a
motion.

, but some deny that it is not sound because C1 is false. They claim
that it relies on the false assumption that completing an infinite
series of tasks would take an infinite period of time. This seems false.
As we divide the distances between the points we travel, we should also
divide the time it takes to travel the ever smaller distances:

\subsubsection{Response}\label{response}

While the argument is valid, many will deny the claim that the arrow is
at rest at each moment. Again, this is about calculus.{[}\^{}1{]}

\subsubsection{Objection}\label{objection}

On this response, motion at a time t1, depends on what is happening
before and after t1. A main concern is that what's happening in this
moment cannot be dependent on what's happening in the future. The future
does not yet exist.

Let us imagine that an arrow is moving. Let us assume that God exists.
Let us assume that God will annihilate all things and time immediately
after t1. Well, it looks like the arrow was not moving at t1.

\end{document}
