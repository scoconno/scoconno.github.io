\documentclass[]{article}
\usepackage{fancyhdr}
 \pagestyle{fancy}
\rhead{\textsc{Scott O`Connor}}
\usepackage{amssymb,amsmath}
\usepackage{ifxetex,ifluatex}
\usepackage{fixltx2e} % provides \textsubscript
\ifnum 0\ifxetex 1\fi\ifluatex 1\fi=0 % if pdftex
  \usepackage[T1]{fontenc}
  \usepackage[utf8]{inputenc}
\else % if luatex or xelatex
  \ifxetex
    \usepackage{mathspec}
    \usepackage{xltxtra,xunicode}
  \else
    \usepackage{fontspec}
  \fi
  \defaultfontfeatures{Mapping=tex-text,Scale=MatchLowercase}
  \newcommand{\euro}{€}
\fi
% use upquote if available, for straight quotes in verbatim environments
\IfFileExists{upquote.sty}{\usepackage{upquote}}{}
% use microtype if available
\IfFileExists{microtype.sty}{%
\usepackage{microtype}
\UseMicrotypeSet[protrusion]{basicmath} % disable protrusion for tt fonts
}{}
\ifxetex
  \usepackage[setpagesize=false, % page size defined by xetex
              unicode=false, % unicode breaks when used with xetex
              xetex]{hyperref}
\else
  \usepackage[unicode=true]{hyperref}
\fi
\usepackage[usenames,dvipsnames]{color}
\hypersetup{breaklinks=true,
            bookmarks=true,
            pdfauthor={},
            pdftitle={Zeno 2},
            colorlinks=true,
            citecolor=blue,
            urlcolor=blue,
            linkcolor=magenta,
            pdfborder={0 0 0}}
\urlstyle{same}  % don't use monospace font for urls
\usepackage{longtable,booktabs}
\setlength{\parindent}{0pt}
\setlength{\parskip}{6pt plus 2pt minus 1pt}
\setlength{\emergencystretch}{3em}  % prevent overfull lines
\providecommand{\tightlist}{%
  \setlength{\itemsep}{0pt}\setlength{\parskip}{0pt}}
\setcounter{secnumdepth}{0}

\title{Time 2}
\author{Scott O’Connor}


% Redefines (sub)paragraphs to behave more like sections
\ifx\paragraph\undefined\else
\let\oldparagraph\paragraph
\renewcommand{\paragraph}[1]{\oldparagraph{#1}\mbox{}}
\fi
\ifx\subparagraph\undefined\else
\let\oldsubparagraph\subparagraph
\renewcommand{\subparagraph}[1]{\oldsubparagraph{#1}\mbox{}}
\fi



\begin{document}
\maketitle

\subsection*{The A-Series and the B-Series}

\begin{itemize} 
\item A-properties: \emph{being past, being present, being future}.
\item B-properties: \emph{earlier than, later than, simultaneous with}
\item A-Series: both A and B properties exist. 
\item B-Series: A properties cannot exist. Only B-properties exist. 
\end{itemize}


\subsection*{Time Does Not Exist}

\begin{enumerate}
\item If time is real, either the B-theory or the A-theory of time is the correct characterization of time. 
\item The correct characterization of time must allow for change. 
\item The B-theory cannot allow for change. [See below]
\item The B-theory cannot characterize time. [From 2--3]
\item An adequate characterization of time cannot be contradictory. 
\item The A-theory is contradictory. [See below]
\item The A-theory cannot characterize time. [From 5--6] 
\item Time is unreal. [From 4\&7]
\end{enumerate}



\subsection*{Argument for Premise 3}
Most assume that the existence of time requires the existence of change; if change is impossible, then so is time. This assumption places a constraint on an adequate theory of time, namely, an adequate theory of time must be compatible with the existence of change. This presents a problem for the B-Theory: 
\begin{enumerate}
\item If change exists, the passage of time must be real.
\item If the B-theory is true, the passage of time is unreal.
\item If change exists, the B-theory cannot be true. [From 1-2]
\end{enumerate} 

Why accept (1)?  Changes are events that have temporal duration. They also seem to be made up of smaller events with shorter durations. For instance, running a 400m race is an event that has temporal duration, but it is itself made up of shorter lasting events. There is the event of running the first 100m, the event of running the second 100m, and so on. But these events do not seem to `eternally' occur, i.e., the journalist shouts, 'the sprinter is \emph{now} on the second leg'. Later he shouts, 'the sprinter is \emph{now} on the third leg.'  The B-Theory cannot seem to accommodate this obvious feature of the sprinter's run. 
 

\subsection*{Argument for P6}

\begin{enumerate}
\item If the A-Theory is true, then a particular event E either (a) co-instantiates the properties of being past, present, and future, or (b) does not co-instantiate these properties. 
\item It is impossible for E to co-instantiate the properties of being past, present, and future.
\item It is impossible for E not to co-instantiate the properties of being past, present, and future. 
\item The A-Theory is not true. 
\end{enumerate}
Big questions:
\begin{itemize}
\item Do you accept (2)? If so, why? If not, why not? 
\item Why might we accept (3)? Recall that there is a worry about an infinite regress. What is this worry? A hint: it assumes a distinction between moments of time and the events occuring at those moments.
\end{itemize}



\end{document} 

