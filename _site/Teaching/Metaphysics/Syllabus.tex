\documentclass[11pt,article,oneside]{memoir}

%%% custom style file with standard settings for xelatex and biblatex. Note that when [minion] is present, we assume you have minion pro installed for use with pdflatex.
%\usepackage[minion]{org-preamble-pdflatex} 

%%% alternatively, use xelatex instead
\usepackage{org-preamble-xelatex} 



\def\myauthor{Author}
\def\mytitle{Title}
\def\mycopyright{\myauthor}
\def\mykeywords{}
\def\mybibliostyle{plain}
\def\mybibliocommand{}
\def\mysubtitle{}
\def\myaffiliation{NJCU}
\def\myaddress{}
\def\myemail{soconnor@njcu.edu}
\def\myweb{\href{http://scottoconnor.org/Teaching/Metaphysics}{http://scottoconnor.org/Teaching/Metaphysics}}
\def\myphone{}
\def\myversion{}
\def\myrevision{}
\def\myaffiliation{NJCU}
\def\myauthor{Dr. Scott O'Connor}
\def\mykeywords{}
\def\mysubtitle{Syllabus}
\def\mytitle{{\normalsize Phil Met, Fall 2015, K358. \newline} \HUGE Metaphysics}


\begin{document}

%%% If using xelatex and not pdflatex
%%% xelatex font choices
\defaultfontfeatures{}
\defaultfontfeatures{Scale=MatchLowercase}    
% You will need to buy these fonts, change the names to fonts you own, or comment out if not using xelatex.      
\setromanfont[Mapping=tex-text]{Georgia} 
\setsansfont[Mapping=tex-text]{Georgia} 
\setmonofont[Mapping=tex-text,Scale=0.8]{Georgia} 

%% blank label items; hanging bibs for text
%% Custom hanging indent for vita items
\def\ind{\hangindent=1 true cm\hangafter=1 \noindent}
\def\labelitemi{$\cdot$}
%\renewcommand{\labelitemii}{~}

%% RCS info string for version tracking
\chapterstyle{article-3}  % alternative styles are defined in latex-custom-kjh/needs-memoir/
\pagestyle{kjh}

\title{\LARGE\mytitle}     
\author{\Large\myauthor \newline \footnotesize\texttt{\noindent\myweb}}
\date{9/01/2015--12/21/2015}

\published{\,}

\maketitle

% \thispagestyle{kjhgit}

% Copyright Page
%\textcopyright{} \mycopyright


%
% Main Content
%

\section{Course Description and Objectives}

Most of us believe that change exists; that babies are born, that trees grow, and that planes fly in the sky. However, the existence of change has long been doubted by certain metaphysicians. Some deny that any kind of change exists. Other deny that certain kinds of change like motion exist. These denials seem radical; surely babies are born, trees do grow, and planes really do fly in the sky. Nevertheless, the arguments that change does not exist are powerful. Our goal in this course is to understand and assess these arguments. 


\section{Learning Objectives}

Upon completing this course, students will be able to (i) read
philosophical texts, (ii) clearly and charitably explain viewpoints that
are not their own, (iii) think critically and philosophically, (iv)
write well-structured prose in which they clearly state a thesis and
persuasively defend it, (v) demonstrate an understanding of several core
metaphysical issues.

\section{Required Textbook}

The following textbook must purchased or rented by 9/14/2015:

\begin{itemize}
\item
\href{http://www.amazon.com/Metaphysics-Questions-Peter-van-Inwagen/dp/1405125861/ref=sr_1_1?ie=UTF8&qid=1440685163&sr=8-1&keywords=metaphysics+big+questions}{`Metaphysics: The Big Questions', ed. Van Inwagen, 2nd edition.} (Available from the campus book store and online retailers)
\end{itemize}

\section{Requirements}

\begin{itemize}
\item \textit{Workload:} Successfully completing this course requires a minimum commitment of 6 hours per week. 


\item \textit{3 Discussion Questions} submitted through Blackboard. Sample answers will be made public to the entire class after the due date. 

\item \textit{1 Essay} submitted through Blackboard. 
 
\item \textit{1 Group project} comprising a proposal, literature review, short presentation (each group member must present), and final submission. 

\item \textit{Attendance:} Roll call will be taken from Week 1. 0.5 point per class up to a maximum of 10 points. Excludes Weeks 1 \& 2.  

\item \textit{Grade Distribution:} Attendance--0.5 point per class (10 total); 3 Discussion Questions---10 points each (30 total); Essay--30 points; Group Project--10 for literature review, 10 for presentation, 20 for written submission (40 points total)

\item \textit{Grade Breakdown:}

 \begin{tabular}{ | l | l | p{2cm} | l | l | }
    \hline 
96--110 & A  & &  77--79 &  C+ \\  
90--95 & A- & &  73--76 & C \\
87-89 & B+ &  &  70--72 & C- \\ 
83--86 & B  & &  60--69 & D\\
80--82 & B - & & 0--59 & F\\ \hline
    \end{tabular}


\end{itemize}




\section{Policies}

\begin{itemize}
\item \textit{Late work \& Make-up Policy:} 
\begin{itemize}
\item All assignments must be submitted through Blackboard by 1:00 pm on the due date (see assignment schedule below).
\item  No make-ups or late work accepted under any circumstances. No exceptions. 
\item Blackboard difficulties are rare and automatically reported to instructors. Under no circumstance will a student's report of a Blackboard difficulty be reason for an extension. It is your responsibility to contact blackboard support for help: \href{dlsupport@njcu.edu}{dlsupport@njcu.edu}. 

\end{itemize}



\item \textit{Communication:} All communication will be through Blackboard. Messages will be responded to within two
days of receiving them. 

\item \textit{Grading Schedule:} Grades will be available within 1 week of an assignment being submitted.



\item \textit{Statement for students with disabilities:} If you are a student
with a disability and wish to receive consideration for reasonable
accommodations, please register with the Office of Specialized Services
and Supplemental Instruction (OSS/SI). To begin this process, complete
the registration form available on the OSS/SI website at
\href{http://www.njcu.edu/Specialized_Services.aspx}{www.njcu.edu/Specialized\_Services.aspx}
(listed under Student Resources-Forms). Contact OSS/SI at 201-200-2091
or visit the office in Karnoutsos Hall, Room 102 for additional
information.
\end{itemize}

\section{Plagiarism}

\begin{itemize} 
\item You are bound by \href{http://www.njcu.edu/uploadedFiles/About_NJCU/Governance_and_Organization/University_Senate/Policies/Academic\%20INTEGRITY\%20POLICY\%20FINAL\%202-04.pdf}{NJCU's Academic Integrity Policy}
\item Penalty for plagiarism:
\begin{itemize}
\item 1st infraction: 0 for the assignment. 
\item 2nd infraction: 0 for the entire course \& application for permanent record on student's transcript. (Repeated violations can lead to expulsion from NJCU). 
\end{itemize}
\end{itemize}


\section{Weekly Course Schedule}
Dates refer to the first day of the week. Complete the readings before the first class of the week. Readings marked with a '**' can be found on the course website. Changes to the syllabus will be announced in class and by email to your NJCU email address.
\begin{enumerate}
\item \textit{08/31} Introduction
\item \textit{09/07} Space
\begin{enumerate}
\item **`Flatland', Edwin A. Abbott
\end{enumerate}
\item \textit{09/14} Space: Absolute vs. Relative Theories of Space
\begin{enumerate}
\item **`Flatland', continued.
\item `The Fourth Dimension: an Excerpt from The Ambidextrous Universe', Martin Gardner
\item `Incongruent Counterparts and Higher Dimensions', James Van Cleve
\end{enumerate}
\item \textit{09/21} Space: The Fourth Dimension
\begin{enumerate}
\item Continued
\end{enumerate}
\item \textit{09/28} Motion: Zeno's Paradoxes
\begin{enumerate}
\item `A Contemporary Look at Zeno’s Paradoxes: an Excerpt from Space, Time, and Motion.', Wesley C. Salmon. 
\item **`The Paradoxes of Motion and the Possibility of Change', E.J. Lowe, pp. 288-297
\end{enumerate}
\item \textit{10/05} Motion: Zeno's Paradoxes
\begin{enumerate}
\item **`The Paradoxes of Motion and the Possibility of Change', E.J. Lowe, pp. 300-306
\end{enumerate}
\item \textit{10/12} Time: Two Theories
\begin{enumerate}
\item **`SlaughterHouse Five', by Kurt Vonnegut (online)
\end{enumerate}
\item  \textit{10/19} Time: The Reality of Time
\begin{enumerate}
\item `Time: an Excerpt from The Nature of Existence', J. McT. E. McTaggart.
\item `McTaggart’s Arguments against the Reality of Time: an Excerpt from Examination of McTaggart’s Philosophy', C. D. Broad.
\item `The Myth of Passage', D. C. Williams.
\end{enumerate}

\item \textit{10/26} Time: The Direction of Time
\begin{enumerate}
\item **`The Arrows of Time', Robin Le Poidevin 
\end{enumerate}

\item \textit{11/02} Change: Problems
\begin{enumerate}
\item `Of Confused Subjects Which Are Equivalent to Two Subjects:. An Excerpt from The Port-Royal Logic', Antoine Arnauld and Pierre Nicole. 
\item **`Change and Identity', Michael Rea, pp.102--111 
\item `The Paradox of Increase', Eric T. Olson (optional)
\end{enumerate}
\item \textit{11/09} Change: Solutions
\begin{enumerate}
 \item `Identity, Ostension, and Hypostasis', W. V. O. Quine.
\item `In Defense of Stages: Postscript B to ``Survival and Identity'' ', David Lewis.
\end{enumerate}
\item \textit{11/16} Time Travel
\begin{enumerate}
\item **`All You Zombies', Robert Heinlein 
\item **`Interfering with History', Robin Le Poidevin 
\end{enumerate}
\item \textit{11/23} Time Travel: Continued.
\begin{enumerate}
\item `The Paradoxes of Time Travel', David Lewis.
\end{enumerate}
\item \textit{11/30} Personal Identity: Student Projects
\item \textit{12/7}  Personal Identity: Student Presentations
\item \textit{12/14}Personal Identity: Student Presentations (no class on Wed)
\end{enumerate}

\section{ Assignment Schedule}
Dates refer to the due date. All assignments must be submitted through Blackboard by 1:00pm. No late work accepted. No make-ups. No exceptions. 

\begin{enumerate}
\item \textit{09/27/2015,} SW1
\item \textit{10/11/2015,} SW2
\item \textit{10/25/2015,} SW3
\item \textit{11/22/2015,} Essay 
\item \textit{11/25/2015,} Group Projects--final day to select topic.
\item \textit{12/06/2015,} Group Projects--Literature Review
\item \textit{12/09/2015,} Group Projects--Presentations
\item \textit{12/14/2015,} Group Projects--Presentations
\item \textit{12/21/2015,} Group Projects--Written Submission
\end{enumerate}




%% Uncomment if you want a printed bibliography.
%\printbibliography 

\end{document}