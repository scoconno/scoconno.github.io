\documentclass[10 pt]{article}

\usepackage{fontspec}
\defaultfontfeatures{Mapping=tex-text}
\setmainfont{Minion Pro}

%\usepackage{graphicx}
%\usepackage{fullpage}

\usepackage{lastpage, fancyhdr}
\pagestyle{fancy}
\lhead{Scott O'Connor}
\chead{Phil 234} 
\rhead{\emph{Apology}}
\lfoot{}
\cfoot{\thepage\space of \pageref{LastPage}} 
\rfoot{}

\thispagestyle{empty}


\begin{document}
\author{Phil 234}
\title{\emph{Apology}}
\maketitle

\section*{Basic structure of dialogue}

\begin{itemize}
\item 17a--35d: S's main defense; found guilty
\item  35e--38b: S proposes as ``punishment'' that he be given free meals in the Prytaneum; sentenced to death
\item 38c--42a: S's parting words
\end{itemize}

\section*{Socrates' Defense}
Socrates thinks there is an implicit accusation, something about him not being liked. He defends himself by arguing that a) he has no knowledge and b) arguing that he somehow makes people better by relieving them of their ignorance. He thinks this makes them better of. 

\section*{Background}
Aristophanes


\section*{Socratic ignorance}
Oracle
\begin{itemize}
\item 20cd: S puts into the mouths of the jurors: where there’s smoke there’s fire; you wouldn’t have gained this reputation if you weren’t up to something
\item 20d: claims that he acquired this reputation as a result of a certain kind of wisdom, human wisdom (ἀνθρωπίνη σοφία)
\item	21a: story of Chaerephon going to oracle
\item	maybe mention that usually the answers were cryptic; here just a “no”
\item	21c: description of his process: approach people to test if they are wise. He went to politicians, (21dc) poets (natural inspiration) and craftspeople.
\end{itemize}

At 21d5: ``it seems that I’m wiser than he in just this one small way: that what I don’t know, I don’t think I know''
\begin{enumerate}
\item S does not know anything about ``physics''/natural philosophy (19c)
\item S does not know how to make people ``excellent'' (= virtuous) (20c)
\item S has (if anything) ``human wisdom'' = does not think he knows what he does not know (20--21d)
\item  But: S grants the craftsmen know many things about their crafts (22d); how?
\begin{itemize} \item These claims are typically stated with noun ``\emph{epist\^{e}m\^{e}}'' and verb ``\emph{epistasthai}''
\item While he allows that they do have knowledge (of their craft), he complains that this led them to think they had knowledge elsewhere, on the more important matters, which they did not, which rendered the knowledge they had, on the whole, undesirable.
\end{itemize}
\end{enumerate}

\section*{What is human wisdom?}

It seems clear that the scope of [3] covers ``the most important things''--seeming to mean moral or ethical truth; but, the meaning of his denial is controversial:

\begin{enumerate}\item[3a.] S has no (ethical) knowledge \emph{at all}; but, he does have (ethical) beliefs (if so, what does he take to be the status of those beliefs?)

\item[3b.] S has no technical or expert (ethical) knowledge, but does have non-technical knowledge.
\item[3c.] (modification of 3b?) S lacks systematic (ethical) understanding but does have some piece-meal knowledge.
\end{enumerate}
\noindent At 29b, Socrates claims, ``To act unjustly, on the other hand, to disobey someone better than oneself, whether god or man, that I do know (\emph{oida}) to be bad and shameful.''

\section*{Relieving Ignorance}
Socrates believes that he relieves others of their ignorance. He use a method called the ``elenchus'' (= ``test,'' ``cross-examining'' ) to do so.
\noindent X = alleged knower; P, Q, R, etc. = propositions
\begin{itemize}
\item{S finds a claim that X asserts, P}\begin{itemize}\item{E.G. S gets Meletus to assert that ``Socrates corrupts the youth willingly''}\end{itemize}
\item{S gets X to assert other claims, Q, R, ...}
\begin{itemize}\item{Associating with bad people harms oneself}\item{No one harms oneself willingly}\item{Therefore: no one would harm one's associates willingly}\end{itemize}

\item{S gets X to conclude (from Q, R, ...), not-P}\begin{itemize}\item{Therefore: Socrates does not corrupt the youth willingly}\end{itemize}

\item{S gets (or tries to get) X to conclude that he does not know whether P}
\end{itemize}
This method shows, at the least, that X has an inconsistent belief set. What kind of knowledge [1], [2], or [3] would this test for? But it is unclear how this method would allow A acquire knowledge.

The problem is that S seems to think that elenchus is \emph{the} method of philosophical thought \emph{and} that philosophical thought is necessary for happiness, but a consistent set of beliefs can still be false, so what would be the upshot even if someone ``survived'' the elenchus? How, despite this, could S think the elenchus can be employed to make progress?

\begin{itemize}
\item{The condition of aporia results in the rejection of the false belief X had (e.g. P) BUT}
\item{This requires S to know that the other beliefs used to conclude not-P are true AND}
\item{S disclaims knowledge}
\item{Resolutions?:}
\begin{itemize}
\item{\textbf{1) accept 3b or 3c above--if S knows (in some sense) Q, R, etc. S can reasonably reject P. This denies that S disclaims (all) knowledge}}
\item{\textbf{accept 3a above, argue that S has justification for thinking that Q, R, etc. are true, but that this justification doesn't count as knowledge}}
\item{\textbf{deny that the elenchus has positive results (beyond testing consistency), deny that it allows you to reject any belief}}
\end{itemize}
\end{itemize}

\section*{Why is ignorance an evil?}

S believes that the state of one's soul (\emph{psuch\^{e}}) is of the utmost importance (29e, 30b). Is it the \emph{only} thing that matters? Or just the most important? The translation of 30b3-5 is controversial. Here are two options:
\begin{enumerate}
\item ``It's not from wealth that virtue comes, but from virtue comes money, and all the other things that are good for human beings, both in private and in public life.'' [Grube]
\item  ``It's not from wealth that virtue comes, but from virtue money and all the other things become good for human beings, both in private and in public life.'' [Alternative]
\end{enumerate}
Our first option explicitly says that there are things other than virtue which are good to have, both in public and private. These include money, but likely include things like beauty, pleasure, etc. Our first option says that if you are virtuous, then you will likely achieve all the other good things in this life. The second option says that money, pleasure, etc., are not good for human in of themselves. If you are not virtuous, money would not be a good thing to have. Rather, it is only virtue alone that matters. 

An obvious challenge to Socrates' emphasis on virtue is that we might face situations where being virtuous will risk our lives. But Socrates  argues that the fear of death should not lead people to act unjustly, impious, etc. He paints two possible pictures of what death is like, neither of which he thinks we should fear (40c--41c)

\begin{enumerate}\item{A dreamless sleep}\item{Existence in Hades with other deceased people}\end{enumerate}

\end{document}
