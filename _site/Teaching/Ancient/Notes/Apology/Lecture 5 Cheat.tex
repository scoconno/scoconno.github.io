% !TEX encoding = UTF-8 Unicode
% !TEX TS-program = xelatex

\documentclass[11pt]{article}
\usepackage{fontspec}
\defaultfontfeatures{Mapping=tex-text}
\usepackage{xunicode}
\usepackage{xltxtra}
\usepackage{verbatim}
\usepackage[margin= 1 in]{geometry} % see geometry.pdf on how to lay out the page. There's lots.
\geometry{letterpaper} % or letter or a5paper or ... etc
%\usepackage[parfill]{parskip}    % Activate to begin paragraphs with an empty line rather than an indent 
\usepackage{mathrsfs}
\usepackage{bbding}
\usepackage[usenames,dvipsnames]{color}
\usepackage{natbib}
\usepackage{stmaryrd}
%\usepackage{mathpartir}
\usepackage{txfonts}
\usepackage{graphicx}
\usepackage{fullpage}
\usepackage{hyperref}
\usepackage{amssymb}
\usepackage{epstopdf}
\usepackage{fontspec}
%\setmainfont{Hoefler Text}
\setmainfont[BoldFont={Minion Pro Bold}]{Minion Pro}
\usepackage{hyperref}
\usepackage{lastpage, fancyhdr}
%\usepackage{setspace}
\pagestyle{fancy}
\lhead{}
\chead{Lecture 5, Plato's \emph{Apology}\space---\space Handout} 
\rhead{}
\lfoot{}
\cfoot{\thepage\space of \pageref{LastPage}} 
\rfoot{}
\footskip=30 pt
\headsep=20pt
\thispagestyle{empty}
\hypersetup{colorlinks=true, linkcolor=Sepia, urlcolor=Sepia, citecolor=BrickRed}
\DeclareGraphicsRule{.tif}{png}{.png}{`convert #1 `dirname #1`/`basename #1 .tif`.png}
\usepackage{polyglossia}
\setdefaultlanguage{english}
\setotherlanguage{greek}
\newfontfamily\greekfont{Gentium Plus}
\newcommand{\gk}[1]{\textgreek{#1}}
\newcommand{\gloss}[1]{(\textgreek{#1})}

\usepackage{covington}
\usepackage{fixltx2e}
\usepackage{graphicx}
\begin{document}

%\maketitle
\thispagestyle{empty}
\begin{center} \LARGE{PHIL 321\\ Lecture 5: Plato's \emph{Apology}}\\ \vspace*{2mm}
\large{9/12/2013}\end{center}
\thispagestyle{empty}\vspace*{3mm}
\vspace*{-8mm}
\section*{Wrap-up from \emph{Euthyphro}}

\noindent S's question: ``Is the pious loved by the gods because it's pious? Or is it pious because it's loved?'' (10a), is about the ``order of explanation,'' ``priority,'' or ``fundamentality'' of the relevant phenomena.
\vspace*{2mm}

\noindent He ultimately claims that the fact that a certain action is pious explains why (all) the gods love it, and not the other way round. This is why S says that E has identified (at best) a quality of piety, not the nature of piety.
\vspace*{2mm}

\noindent \textbf{``Euthyphro Dilemma''}: If certain actions were pious (right, wrong, obligatory, impermissible, etc.) \emph{because} they are dear to the gods (or commanded but God/the gods, etc.), we would ask why the gods love what they love. Either it is [A] \emph{arbitrary} what the gods love, or [B] the gods love what they do for \emph{reasons}. Either way, trouble looms:\begin{itemize}\item{[A] seems absurd (it is just obviously false that if the gods had happened to love rape, murder, etc. then rape, murder, etc. would be pious)}\item{If [B], it is the features of the actions in virtue of which the gods love them that explains why they are pious. The attitude of the gods is not what \emph{explains} or \emph{makes it the case} that they are pious} \end{itemize}

\vspace*{-3mm}
\section*{Basic structure of dialogue}
17a--35d: S's main defense; found guilty
\vspace*{2mm}

\noindent 35e--38b: S proposes as ``punishment'' that he be given free meals in the Prytaneum; sentenced to death
\vspace*{2mm}

\noindent 38c--42a: S's parting words

\vspace*{-3mm}
\section*{Socratic ignorance: specific claims}

\noindent \textbf{Tell story of oracle}
\vspace*{2mm}

\noindent [1] S does not know anything about ``physics''/natural philosophy (19c)
\vspace*{2mm}

\noindent [2] S does not know how to make people ``excellent'' (= virtuous) (20c)
\vspace*{2mm}

\noindent [3] S has (if anything) ``human wisdom'' = does not think he knows what he does not know (20--21d)
\vspace*{2mm}

\noindent [4] But: S grants the craftsmen know many things about their crafts (22d); how?
\vspace*{2mm}

\noindent These claims are typically stated with noun ``\emph{epist\^{e}m\^{e}}''
\vspace*{-3mm}


\section*{What is human wisdom?}

\noindent It seems clear that the scope of [3] covers ``the most important things''--seeming to mean moral or ethical truth; but, the meaning of his denial is controversial:\begin{itemize}\item{[3a] S has no (ethical) knowledge \emph{at all}; but, he does have (ethical) beliefs (if so, what does he take to be the status of those beliefs?)}\item{[3b] S has no technical or expert (ethical) knowledge, but does have non-technical knowledge}\item{[3c] (modification of 3b?) S lacks systematic (ethical) understanding but does have some piece-meal knowledge}\end{itemize}
\vspace*{2mm}

\noindent At 29b, Socrates claims, ``To act unjustly, on the other hand, to disobey someone better than oneself, whether god or man, that I do know (\emph{oida}) to be bad and shameful.''

\section*{The ``Elenchus'' (= ``test,'' ``cross-examining,'' ``method of (dis)proof'')}
\noindent X = alleged knower; P, Q, R, etc. = propositions
\begin{itemize}
\item{S finds a claim X asserts, P}\begin{itemize}\item{\textbf{E.G. S gets Meletus to assert that ``Socrates corrupts the youth willingly''}}\end{itemize}
\item{S gets X to assert other claims, Q, R, ...}

\item{S gets X to conclude (from Q, R, ...), not-P}\begin{itemize}\item{\textbf{Associating with bad people harms oneself}}\item{\textbf{No one harms oneself willingly}}\item{\textbf{Therefore: no one would harm one's associates willingly}}\item{\textbf{Therefore: Socrates does not corrupt the youth willingly}}\end{itemize}

\item{S gets (or tries to get) X to conclude that he does not know whether P}\end{itemize}

\noindent This method shows, at the least, that X has an inconsistent belief set\\

\noindent What kind of knowledge [1], [2], or [3] would this test for?
\vspace*{-2mm}
\section*{``The'' problem of the elenchus}
\noindent S seems to think that elenchus is \emph{the} method of philosophical thought \emph{and} that philosophical thought is necessary for happiness, BUT
\vspace*{2mm}

\noindent A consistent set of beliefs can still be false, so what would be the upshot even if someone ``survived'' the elenchus?

\begin{itemize}\item{How, despite this, could S think the elenchus can be employed to make progress?}
\item{\textbf{The condition of aporia results in the rejection of the false belief X had (e.g. P) BUT}}\item{\textbf{This requires S to know that the other beliefs used to conclude not-P are true AND}}\item{\textbf{S disclaims knowledge}}\item{\textbf{Resolutions?:}}\begin{itemize}\item{\textbf{1) accept 3b or 3c above--if S knows (in some sense) Q, R, etc. S can reasonably reject P. This denies that S disclaims (all) knowledge}}\item{\textbf{accept 3a above, argue that S has justification for thinking that Q, R, etc. are true, but that this justification doesn't count as knowledge}}\item{\textbf{deny that the elenchus has positive results (beyond testing consistency), deny that it allows you to reject any belief}}
\end{itemize}
\end{itemize}

\section*{What matters in life, according to Socrates}

\noindent The state of one's soul (\emph{psuch\^{e}}) is of the utmost importance (29e, 30b). Is it the \emph{only} thing that matters? Or just the most important?
\vspace*{2mm}

\noindent The translation of 30b3-5 is controversial:
\vspace*{1mm}

\noindent ``It's not from wealth that virtue comes, but from virtue comes money, and all the other things that are good for human beings, both in private and in public life.'' [Grube]
\vspace*{2mm}

\noindent ``It's not from wealth that virtue comes, but from virtue money and all the other things become good for human beings, both in private and in public life.'' [Alternative]
\vspace*{2mm}

\noindent The fear of death should not lead people to act unjustly, impious, etc.

\begin{itemize}\item{Socrates paints two possible pictures of what death is like, neither of which he thinks we should fear (40c--41c)}\begin{itemize}\item{A dreamless sleep}\item{Existence in Hades with other deceased people}\end{itemize}\end{itemize}

\end{document}
