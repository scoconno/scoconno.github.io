% !TEX encoding = UTF-8 Unicode
% !TEX TS-program = xelatex

\documentclass[11pt]{article}
\usepackage{fontspec}
\defaultfontfeatures{Mapping=tex-text}
\usepackage{xunicode}
\usepackage{xltxtra}
\usepackage{verbatim}
\usepackage[margin= 1 in]{geometry} % see geometry.pdf on how to lay out the page. There's lots.
\geometry{letterpaper} % or letter or a5paper or ... etc
%\usepackage[parfill]{parskip}    % Activate to begin paragraphs with an empty line rather than an indent 
\usepackage{mathrsfs}
\usepackage{bbding}
\usepackage[usenames,dvipsnames]{color}
\usepackage{natbib}
\usepackage{stmaryrd}
%\usepackage{mathpartir}
\usepackage{txfonts}
\usepackage{graphicx}
\usepackage{fullpage}
\usepackage{hyperref}
\usepackage{amssymb}
\usepackage{epstopdf}
\usepackage{fontspec}
%\setmainfont{Hoefler Text}
\setmainfont[BoldFont={Minion Pro Bold}]{Minion Pro}
\usepackage{hyperref}
\usepackage{lastpage, fancyhdr}
%\usepackage{setspace}
\pagestyle{fancy}
\lhead{}
\chead{Lecture 11, Plato's Cave\space---\space Handout} 
\rhead{}
\lfoot{}
\cfoot{\thepage\space of \pageref{LastPage}} 
\rfoot{}
\footskip=30 pt
\headsep=20pt
\thispagestyle{empty}
\hypersetup{colorlinks=true, linkcolor=Sepia, urlcolor=Sepia, citecolor=BrickRed}
\DeclareGraphicsRule{.tif}{png}{.png}{`convert #1 `dirname #1`/`basename #1 .tif`.png}
\usepackage{polyglossia}
\setdefaultlanguage{english}
\setotherlanguage{greek}
\newfontfamily\greekfont{Gentium Plus}
\newcommand{\gk}[1]{\textgreek{#1}}
\newcommand{\gloss}[1]{(\textgreek{#1})}

\usepackage{covington}
\usepackage{fixltx2e}
\usepackage{graphicx}
\begin{document}

%\maketitle
\thispagestyle{empty}
\begin{center} \LARGE{PHIL 321\\ Lecture 11: Plato's Cave (\emph{Republic} 7, 514-521b)}\\ \vspace*{2mm}
\large{10/1/2013}\end{center}
\thispagestyle{empty}\vspace*{3mm}
\vspace*{-8mm}

\section*{For board}

\noindent Scenario: After eating two donuts in a row, Bob decides it would be best for him not to eat any more. However, when a new batch is placed in front of him, he eats another donut.
\vspace*{2mm}

\noindent How would the Socrates character of the \emph{Protagoras} explain why Bob eats the third donut?
\vspace*{2mm}

\noindent How would the Socrates character of the \emph{Republic} explain why Bob eats the third donut?
\vspace*{2mm}

\noindent ``If someone said that a person who is standing still but moving his hands and head is moving and standing still at the same time, we wouldn't consider, I think, that he ought to put it like that. What he ought to say is that one part of the person is standing still and another part is moving.'' 

\section*{Background}

\noindent Socrates draws a sharp ontological distinction between two kinds of entities: perceptible objects and intelligible objects:
\vspace*{2mm}

\noindent Perceptible objects: Entities about which we can gain information \emph{directly} through the five senses. It also includes groups of such objects.
\begin{itemize}\item{Examples: Socrates, Socrates' dog Fido, this building, each person in this room}\item{Also: The people in this room (we do not see this entity directly, rather we gain information about it by seeing (hearing, touching, etc.) each individual person in this room}\end{itemize}

\noindent Intelligible objects: Entities about which we gain information \emph{solely} through the activity of thought (\textbf{NB}: The ``solely'' is crucial here, since we can think about and, hence, gain additional information about perceptible objects. The point is that, in addition to being able to think about perceptible objects, we can \emph{also} perceive them through the senses.)

\begin{itemize}\item{Examples: Mathematical objects (squares, triangles, the number 2), Natures or Essences, which Socrates also calls ``Forms'' (\emph{eid\^{e}}) or ``Ideas'' (\emph{ideai}) (e.g. the nature of Piety, the nature of Justice, the Nature of Goodness)}\item{This is a development from the conception of Natures or Essences we found in the Socratic dialogues (e.g. \emph{Euthyphro}, \emph{Protagoras}, \emph{Meno}). In those dialogues there was no indication that Natures or Essences existed separately from the perceptible world.}\end{itemize}

\noindent Perceptible objects do not perfectly instantiate intelligible objects but, rather, ``approximate,'' ``resemble,'' or ``participate in'' them. S also says that perceptible objects are ``images'' or ``imitations'' of intelligible objects.
\vspace*{2mm}

\noindent Consider the following two figures:

\begin{figure}[h!]
\hspace*{35mm}
\includegraphics[scale=0.6]{figure1}
%\caption{}
%\label{}
\end{figure}

\noindent Figure 1 is not a perfect square, but it more closely approximates the nature of Square than Figure 2 does
\vspace*{2mm}

\noindent Similarly, S claims that, while nothing in the perceptible world perfectly instantiates the nature of (for example) Justice, certain things in the perceptible world (e.g. distributions of resources, social institutions, etc.) can more closely approximate the nature of Justice than others (and, hence, can be ``more just'' than others).

\noindent At 471, Glaucon questions Socrates whether a city with the constitution they laid out in Books 2-4 can be realized on earth
\vspace*{2mm}

\noindent S maintains that the way to bring about a city \emph{closest to} the ideal city is to vest political power in the hands of philosophers
\vspace*{2mm}

\noindent Philosophers are distinguished from non-philosophers insofar as only philosophers can attain understanding (\emph{epist\^{e}m\^{e}}) and knowledge (\emph{gn\^{o}sis}) of intelligible objects, which makes them epistemic authorities about matters in the perceptible world (think of a doctor's expert ``medical opinion'' that a particular patient should undergo a particular course of treatment)
\vspace*{2mm}

\noindent The Cave is an allegory of how education effects a radical change in people and can ultimately lead them to understanding of Forms, the most fundamental of which is the Form of the Good.

\end{document}
