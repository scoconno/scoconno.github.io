\documentclass[]{article}

\usepackage{amssymb,amsmath}
\usepackage{ifxetex,ifluatex}
\usepackage{fixltx2e} % provides \textsubscript
\ifnum 0\ifxetex 1\fi\ifluatex 1\fi=0 % if pdftex
  \usepackage[T1]{fontenc}
  \usepackage[utf8]{inputenc}
\else % if luatex or xelatex
  \ifxetex
    \usepackage{mathspec}
    \usepackage{xltxtra,xunicode}
  \else
    \usepackage{fontspec}
  \fi
  \defaultfontfeatures{Mapping=tex-text,Scale=MatchLowercase}
  \newcommand{\euro}{€}
\fi
% use upquote if available, for straight quotes in verbatim environments
\IfFileExists{upquote.sty}{\usepackage{upquote}}{}
% use microtype if available
\IfFileExists{microtype.sty}{%
\usepackage{microtype}
\UseMicrotypeSet[protrusion]{basicmath} % disable protrusion for tt fonts
}{}
\ifxetex
  \usepackage[setpagesize=false, % page size defined by xetex
              unicode=false, % unicode breaks when used with xetex
              xetex]{hyperref}
\else
  \usepackage[unicode=true]{hyperref}
\fi
\hypersetup{breaklinks=true,
            bookmarks=true,
            pdfauthor={},
            pdftitle={Phaedo},
            colorlinks=true,
            citecolor=blue,
            urlcolor=blue,
            linkcolor=magenta,
            pdfborder={0 0 0}}
\urlstyle{same}  % don't use monospace font for urls
\setlength{\parindent}{0pt}
\setlength{\parskip}{6pt plus 2pt minus 1pt}
\setlength{\emergencystretch}{3em}  % prevent overfull lines
\setcounter{secnumdepth}{0}

\title{Phaedo}
\date{}

\begin{document}
\maketitle

\section{Phaedo}\label{phaedo}

\subsection{Introduction}\label{introduction}

Socrates claims that philosophers, above everyone else, should welcome
death. His argument is as follows:

\begin{itemize}
\itemsep1pt\parskip0pt\parsep0pt
\item
  P1: Philosophers want knowledge of forms.
\item
  P2: Philosophers cannot acquire that knowledge while alive.
\item
  C: Philosophers can acquire that knowledge when soul has been
  separated from the body.
\end{itemize}

Socrates claims that death is the separation of body and soul (psuchē).
The Greek word was later translated in Latin as `anima'. This is closer
to the original meaning. It is something that animates a body, i.e., it
is that which gives it alive. On this view, every living creature has a
soul (because it has something which animates it).

Socrates offers two reasons for why a philosopher welcomes the
separation of body and soul.

\begin{enumerate}
\def\labelenumi{\arabic{enumi}.}
\itemsep1pt\parskip0pt\parsep0pt
\item
  The philosopher despises bodily pleasures such as food, drink, and
  sex, so he more than anyone else wants to free himself from his body
  (64d-65a).\\
\item
  The bodily senses are inaccurate and deceptive, the philosopher's
  search for knowledge is most successful when the soul is ``most by
  itself.''
\end{enumerate}

Our main interest is the second claim. Socrates is here claiming that
the objects of philosophical knowledge that Plato later on in the
dialogue (103e) refers to as ``Forms.'' Here Forms are mentioned as the
Just itself, the Beautiful, and the Good; Bigness, Health, and Strength;
and ``in a word, the reality of all other things, that which each of
them essentially is'' (65d). Socrates believes that we cannot gain
knowledge of these entities by perceiving them.

The body is a constant impediment to philosophers in their search for
truth: ``It fills us with wants, desires, fears, all sorts of illusions
and much nonsense, so that, as it is said, in truth and in fact no
thought of any kind ever comes to us from the body'' (66c). To have pure
knowledge, therefore, philosophers must escape from the influence of the
body as much as is possible in this life.

Thus, Socrates concludes, it would be unreasonable for a philosopher to
fear death, since upon dying he is most likely to obtain the wisdom
which he has been seeking his whole life.

\subsection{Aristotle on Plato's Reasons for Positing Forms
\#\#According to Aristotle, P introduced Forms b/c P was influenced by
Heraclitus' and Cratylus' views that everything in the
sensible/observable/material world is somehow changing or unstable. The
worry was that we could only find a satisfying definition of F (and
hence have knowledge of F things) if there are stable Forms. Question:
what kind of change are we talking about here? 1. Succession of
Opposites (SO): Typically, when we think of change we think of SO. To
say that something undergoes SO is to say that it is F at t1, but
becomes not F at some later time t2; in other words, cases of SO are
cases where one and the same thing has opposite properties or
characteristics at different times. Example: I was fat in January (t1),
but I went on a diet and by June (t2) I was no longer fat but rather
thin. 2. Compresence of Opposites (CO): Use ``change'' in a broad sense.
One thing has properties F and not F, but at the same time. Heraclitus'
examples. Plato's own examples in the \emph{Phaedo} suggest that he's
concerned primarily with CO. 1. Simias is both taller and shorter; he's
taller than X and shorter than Y (102b 2. ``By a head'' (material cause)
explains BOTH something's being taller, and its being smaller.
100e5-101b2So, material things suffer from CO. Why does that mean that
we can't appeal to them in finding definitions? One idea: Perceptible
objects and properties don't give us the right answer to our search for
definitions. If we focus on the sorts of properties that are matters of
observation, we won't get the right answer, because we'll only come up
with properties that pick out F things no more than not-F things. I
can't focus on some OBSERVABLE ACT to define justice, b/c any observable
thing that I pick out---e.g.~RETURNING WHAT I HAVE BORROWED---in some
cases will be just, and in other cases will be unjust. So, this tells us
that if we're going to find definitions (and hence get K), we must find
something IMPERCEPTIBLE that can serve as the objects of these
definitions. These insensible things are FORMS. Plato's 1st Argument for
Positing Forms
(Summary):}\label{aristotle-on-platos-reasons-for-positing-forms-according-to-aristotle-p-introduced-forms-bc-p-was-influenced-by-heraclitus-and-cratylus-views-that-everything-in-the-sensibleobservablematerial-world-is-somehow-changing-or-unstable.-the-worry-was-that-we-could-only-find-a-satisfying-definition-of-f-and-hence-have-knowledge-of-f-things-if-there-are-stable-forms.-question-what-kind-of-change-are-we-talking-about-here-1.-succession-of-opposites-so-typically-when-we-think-of-change-we-think-of-so.-to-say-that-something-undergoes-so-is-to-say-that-it-is-f-at-t1-but-becomes-not-f-at-some-later-time-t2-in-other-words-cases-of-so-are-cases-where-one-and-the-same-thing-has-opposite-properties-or-characteristics-at-different-times.-example-i-was-fat-in-january-t1-but-i-went-on-a-diet-and-by-june-t2-i-was-no-longer-fat-but-rather-thin.-2.-compresence-of-opposites-co-use-change-in-a-broad-sense.-one-thing-has-properties-f-and-not-f-but-at-the-same-time.-heraclitus-examples.-platos-own-examples-in-the-phaedo-suggest-that-hes-concerned-primarily-with-co.-1.-simias-is-both-taller-and-shorter-hes-taller-than-x-and-shorter-than-y-102b-2.-by-a-head-material-cause-explains-both-somethings-being-taller-and-its-being-smaller.-100e5-101b2so-material-things-suffer-from-co.-why-does-that-mean-that-we-cant-appeal-to-them-in-finding-definitions-one-idea-perceptible-objects-and-properties-dont-give-us-the-right-answer-to-our-search-for-definitions.-if-we-focus-on-the-sorts-of-properties-that-are-matters-of-observation-we-wont-get-the-right-answer-because-well-only-come-up-with-properties-that-pick-out-f-things-no-more-than-not-f-things.-i-cant-focus-on-some-observable-act-to-define-justice-bc-any-observable-thing-that-i-pick-oute.g.returning-what-i-have-borrowedin-some-cases-will-be-just-and-in-other-cases-will-be-unjust.-so-this-tells-us-that-if-were-going-to-find-definitions-and-hence-get-k-we-must-find-something-imperceptible-that-can-serve-as-the-objects-of-these-definitions.-these-insensible-things-are-forms.-platos-1st-argument-for-positing-forms-summary}

\begin{enumerate}
\def\labelenumi{\arabic{enumi}.}
\item
  To have knowledge about F, one must have a definition of F. (Recall
  PDK, Meno 71b: In order to know whether or not virtue is teachable,
  one must first have a definition of virtue)2. Sensibles are in flux
  (suffer compresence of opposites).3. So, we cannot appeal to any
  sensible/perceptible object or property to get an adequate definition
  of F; any sensible object or property that we pick out will be both F
  and not F.4. Knowledge is possible.5. So, there must be adequate
  definitions that would give us this Knowledge.6. So, there must be
  non-sensible abstract objects (Forms) to which we can appeal when
  defining F. This sort of argument tells us the following about Plato's
  Forms1. Forms are imperceptible2. Forms are the materials of
  definitions. \#\# The ``Imperfection Argument'' (Phaedo 74-76) on
  Forms \#\#How does the argument go? A helpful formulation from:
  http://faculty.washington.edu/smcohen/320/phaedo.htm1. We perceive
  sensible objects to be F.2. But every sensible object is, at best,
  imperfectly F. That is, it is both F and not F. It falls short of
  being perfectly F.
\item
  We are aware of this imperfection in the objects of perception.4. So
  we perceive objects to be imperfectly F.5. To perceive something as
  imperfectly F, one must have in mind something that is perfectly F,
  something that the imperfectly F things fall short of. (E.g., we have
  an idea of equality that all sticks, stones, etc., only imperfectly
  exemplify.)6. So we have in mind something that is perfectly F.7.
  Thus, we must have at one time encountered something that is perfectly
  F (e.g., Equality), that we have in mind in such cases.8. Therefore,
  there is such a thing as the F itself (e.g., the Equal itself), and it
  is distinct from any sensible object (given that we recognize that all
  sensible things are imperfectly F).So, this gives us a second argument
  for the existence of Forms. According to this argument, there must be
  perfect Forms---a Form of Equality, Beauty, etc---from which we
  acquire our concepts/ideas of (perfect) equality, beauty, etc, since
  there's no way that we could have acquired such concepts from
  (imperfect) sensibles. What does the argument tell us about Forms?+
  Forms are what we recollect during recollection.+ Forms are somehow
  ``more perfect'' than sensibles. \#\# Argument for Immortality 1 \#\#
\item
  All things come to be from their opposite states: for example,
  something that comes to be ``larger'' must necessarily have been
  ``smaller'' before (70e-71a).
\item
  Between every pair of opposite states there are two opposite
  processes: for example, between the pair ``smaller'' and ``larger''
  there are the processes ``increase'' and ``decrease'' (71b).
\item
  If the two opposite processes did not balance each other out,
  everything would eventually be in the same state: for example, if
  increase did not balance out decrease, everything would keep becoming
  smaller and smaller (72b).
\item
  Since ``being alive'' and ``being dead'' are opposite states, and
  ``dying'' and ``coming-to-life'' are the two opposite processes
  between these states, coming-to-life must balance out dying (71c-e).
\item
  Therefore, everything that dies must come back to life again (72a).
\end{enumerate}

\subsection{Argument for Immortality
2}\label{argument-for-immortality-2}

\begin{enumerate}
\def\labelenumi{\arabic{enumi}.}
\item
  Things in the world which appear to be equal in measurement are in
  fact deficient in the equality they possess (74b, d-e).
\item
  Therefore, they are not the same as true equality, that is, ``the
  Equal itself'' (74c).
\item
  When we see the deficiency of the examples of equality, it helps us to
  think of, or ``recollect,'' the Equal itself (74c-d).
\item
  In order to do this, we must have had some prior knowledge of the
  Equal itself (74d-e).
\item
  Since this knowledge does not come from sense-perception, we must have
  acquired it before we acquired sense-perception, that is, before we
  were born (75b ff.).
\item
  Therefore, our souls must have existed before we were born. (76d-e)
\end{enumerate}

\subsection{Argument for Immortality
3}\label{argument-for-immortality-3}

\begin{enumerate}
\def\labelenumi{\arabic{enumi}.}
\item
  There are two kinds of existences: (a) the visible world that we
  perceive with our senses, which is human, mortal, composite,
  unintelligible, and always changing, and (b) the invisible world of
  Forms that we can access solely with our minds, which is divine,
  deathless, intelligible, non-composite, and always the same (78c-79a,
  80b).
\item
  The soul is more like world (b), whereas the body is more like world
  (a) (79b-e).
\item
  Therefore, supposing it has been freed of bodily influence through
  philosophical training, the soul is most likely to make its way to
  world (b) when the body dies (80d-81a). (If, however, the soul is
  polluted by bodily influence, it likely will stay bound to world (a)
  upon death (81b-82b).)
\end{enumerate}

\end{document}
