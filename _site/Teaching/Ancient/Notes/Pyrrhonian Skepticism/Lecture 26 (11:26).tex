% !TEX encoding = UTF-8 Unicode
% !TEX TS-program = xelatex

\documentclass[11pt]{article}
\usepackage{fontspec}
\defaultfontfeatures{Mapping=tex-text}
\usepackage{xunicode}
\usepackage{xltxtra}
\usepackage{verbatim}
\usepackage[margin= 1 in]{geometry} % see geometry.pdf on how to lay out the page. There's lots.
\geometry{letterpaper} % or letter or a5paper or ... etc
%\usepackage[parfill]{parskip}    % Activate to begin paragraphs with an empty line rather than an indent 
\usepackage{mathrsfs}
\usepackage{bbding}
\usepackage[usenames,dvipsnames]{color}
\usepackage{natbib}
\usepackage{stmaryrd}
%\usepackage{mathpartir}
\usepackage{txfonts}
\usepackage{graphicx}
\usepackage{fullpage}
\usepackage{hyperref}
\usepackage{amssymb}
\usepackage{epstopdf}
\usepackage{fontspec}
%\setmainfont{Hoefler Text}
\setmainfont[BoldFont={Minion Pro Bold}]{Minion Pro}
\usepackage{hyperref}
\usepackage{lastpage, fancyhdr}
%\usepackage{setspace}
\pagestyle{fancy}
\lhead{}
\chead{Lecture 26, Pyrrhonian Skepticism\space---\space Handout} 
\rhead{}
\lfoot{}
\cfoot{\thepage\space of \pageref{LastPage}} 
\rfoot{}
\footskip=30 pt
\headsep=20pt
\thispagestyle{empty}
\hypersetup{colorlinks=true, linkcolor=Sepia, urlcolor=Sepia, citecolor=BrickRed}
\DeclareGraphicsRule{.tif}{png}{.png}{`convert #1 `dirname #1`/`basename #1 .tif`.png}
\usepackage{polyglossia}
\setdefaultlanguage{english}
\setotherlanguage{greek}
\newfontfamily\greekfont{Gentium Plus}
\newcommand{\gk}[1]{\textgreek{#1}}
\newcommand{\gloss}[1]{(\textgreek{#1})}

\usepackage{covington}
\usepackage{fixltx2e}
\usepackage{graphicx}
\begin{document}

%\maketitle
\thispagestyle{empty}
\begin{center} \LARGE{PHIL 321\\ Lecture 26: Pyrrhonian Skepticism}\\ \vspace*{2mm}
\large{11/26/2013}\end{center}
\thispagestyle{empty}\vspace*{3mm}
\vspace*{-8mm}

\section*{Kinds of philosophers}

\noindent Dogmatists: Think they have found the truth
\vspace*{2mm}

\noindent Academics: Think the truth cannot be found
\vspace*{2mm}

\noindent Skeptics: Are still inquiring into the truth
\vspace*{2mm}

\noindent Two major questions:
\vspace*{2mm}

[1] Can someone have the philosophical experience described by Sextus and still be a ``good faith'' inquirer\\\hspace*{12mm}into the truth?
\vspace*{1mm}

[2] Can a person be a skeptic, in Sextus' sense, and still have beliefs about the world? If yes, what kinds\\\hspace*{12mm}of beliefs can they have?

\section*{What is skepticism?}

\noindent For most other ``philosophies'' you can list their central tenets, and being an \emph{X}-philosopher consists in accepting those central tenets (or, at least, a substantial number of them)
\vspace*{2mm}

\noindent Skepticism, on the other hand, is defined as an ability, specifically ``the ability to set in opposition appearances and ideas in any manner whatsoever, the result of which is first that, because of the equal force of the opposed objects and arguments, final suspension of judgment is achieved, and then freedom from disturbance'' (\emph{PH} 1.8)
\vspace*{2mm}

``Appearances'' = perceptual impressions
\vspace*{1mm}

``Ideas'' = mental impressions
\vspace*{1mm}

``Opposed arguments'' = arguments with inconsistent conclusions
\vspace*{1mm}

``Equal force'' = equality with respect to plausibility and lack of plausibility
\vspace*{1mm}

``Suspension of judgment'' (\emph{epoch\^{e}}) = standstill of the intellect according to which we neither affirm nor\\\hspace*{7mm}deny anything
\vspace*{1mm}

``Freedom from disturbance'' (\emph{ataraxia}, ``tranquility'') = serenity or calmness of the soul
\section*{The path to skepticism (\emph{PH} 1.12)}

\noindent Being disturbed by the inconsistency in things $\rightarrow$ inquire into which is true to get rid of disturbance $\rightarrow$ equipollence in arguments $\rightarrow$ \emph{epoch\^{e}} $\rightarrow$ \emph{ataraxia}

\section*{Skeptical Dogma}

\noindent Sextus says in \emph{PH} 1.12 that, due to coming across equipollent arguments, the skeptic ceases to dogmatize
\vspace*{2mm}

\noindent Then, in 1.13, he immediately qualifies this, and says that there are two senses of ``\emph{dogma}'' and that the skeptic has \emph{dogma} in one sense but not in another:
\vspace*{2mm}

``Broader sense'' = approving of something (allowed)
\vspace*{1mm}

``Narrower sense'' = assent to something non-evident investigated by the sciences (not allowed)
\vspace*{2mm}

\noindent What makes the difference between these two type of dogma?
\vspace*{2mm}

Content: whether a proposition \emph{P} is a non-evident matter depends upon the terms in \emph{P}
\vspace*{1mm}

Mode of acquisition: whether \emph{P} is non-evident depends upon the way in which belief in \emph{P} is acquired
\vspace*{2mm}

\noindent The skeptic suspends judgment whether \emph{criteria of truth exist}, and many philosophers have thought that that renders them incapable of having any beliefs whatsoever

\section*{The ``skeptical criterion''}

\noindent Sextus says that there are four ways in which the skeptic is led to act \emph{without} violating skepticism
\vspace*{2mm}

[1] Guidance given by nature: skeptics perceive and think
\vspace*{1mm}

[2] Compulsion exercised by states: hunger leads to food, thirst to drink
\vspace*{1mm}

[3] Traditional laws and customs: accept pious living as good, improper living as bad
\vspace*{1mm}

[4] Teaching of crafts: Sextus was a doctor!
\vspace*{2mm}

\noindent Is it possible to understand someone being able to do all this \emph{without} beliefs? If not, can the kind of \emph{dogma} allowed a skeptic accommodate it?

\section*{Why Pyrrhonian Skepticism is best}

\noindent Sextus says that skepticism is better than the alternative ``philosophies'' because it allows one to achieve the tranquility that they all seek
\vspace*{2mm}

\noindent Everyone is striving to determine the truth, thinking that the truth shall make them tranquil, but the skeptic found that suspending judgment, and no longer believing that tranquility depends upon reaching the truth, brought that tranquility (like a shadow follows a body; sponge of Apelles)
\vspace*{2mm}

\noindent In fact, skeptics are supposed to be better off than people who were never troubled by the inconsistency in things in the first place. While the skeptic, like those people, will experience hunger, pain and the like, they won't be troubled by the \emph{additional} thought that those things are, in reality, bad 



\end{document}
