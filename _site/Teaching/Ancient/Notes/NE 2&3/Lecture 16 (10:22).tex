% !TEX encoding = UTF-8 Unicode
% !TEX TS-program = xelatex

\documentclass[11pt]{article}
\usepackage{fontspec}
\defaultfontfeatures{Mapping=tex-text}
\usepackage{xunicode}
\usepackage{xltxtra}
\usepackage{verbatim}
\usepackage[margin= 1 in]{geometry} % see geometry.pdf on how to lay out the page. There's lots.
\geometry{letterpaper} % or letter or a5paper or ... etc
%\usepackage[parfill]{parskip}    % Activate to begin paragraphs with an empty line rather than an indent 
\usepackage{mathrsfs}
\usepackage{bbding}
\usepackage[usenames,dvipsnames]{color}
\usepackage{natbib}
\usepackage{stmaryrd}
%\usepackage{mathpartir}
\usepackage{txfonts}
\usepackage{graphicx}
\usepackage{fullpage}
\usepackage{hyperref}
\usepackage{amssymb}
\usepackage{epstopdf}
\usepackage{fontspec}
%\setmainfont{Hoefler Text}
\setmainfont[BoldFont={Minion Pro Bold}]{Minion Pro}
\usepackage{hyperref}
\usepackage{lastpage, fancyhdr}
%\usepackage{setspace}
\pagestyle{fancy}
\lhead{}
\chead{Lecture 16, Aristotle \emph{NE}, Books II \& III\space---\space Handout} 
\rhead{}
\lfoot{}
\cfoot{\thepage\space of \pageref{LastPage}} 
\rfoot{}
\footskip=30 pt
\headsep=20pt
\thispagestyle{empty}
\hypersetup{colorlinks=true, linkcolor=Sepia, urlcolor=Sepia, citecolor=BrickRed}
\DeclareGraphicsRule{.tif}{png}{.png}{`convert #1 `dirname #1`/`basename #1 .tif`.png}
\usepackage{polyglossia}
\setdefaultlanguage{english}
\setotherlanguage{greek}
\newfontfamily\greekfont{Gentium Plus}
\newcommand{\gk}[1]{\textgreek{#1}}
\newcommand{\gloss}[1]{(\textgreek{#1})}

\usepackage{covington}
\usepackage{fixltx2e}
\usepackage{graphicx}
\begin{document}

%\maketitle
\thispagestyle{empty}
\begin{center} \LARGE{PHIL 321\\ Lecture 16: Aristotle's \emph{Nicomachean Ethics}, Books II \& III}\\ \vspace*{2mm}
\large{10/22/2013}\end{center}
\thispagestyle{empty}\vspace*{3mm}
\vspace*{-8mm}

\section*{Virtue, continence (\emph{enkrateia}), incontinence (\emph{akrasia}), vice}

\noindent These four kinds of character are distinguished by three conditions they involve: (a) rational concerns/ends\\(b) non-rational attitudes\hspace*{3mm}(c) actions that result from (a) \& (b)
\vspace*{2mm}

\noindent \underline{Rational aim\hspace*{10mm}Non-rational attitude\hspace*{10mm}Resulting action\hspace*{10mm}Agent's disposition}
\vspace*{1mm}

\noindent Good\hspace*{21mm}Good\hspace*{34mm}Good\hspace*{26mm}Virtue
\vspace*{1mm}

\noindent Good\hspace*{21mm}Bad\hspace*{37mm}Good\hspace*{26mm}Continence (Self-control)
\vspace*{1mm}

\noindent Good\hspace*{21mm}Bad\hspace*{37mm}Bad\hspace*{29mm}Incontinence (akratic)
\vspace*{1mm}

\noindent Bad\hspace*{23mm}``Bad''\hspace*{35mm}Bad\hspace*{29mm}Vice

\section*{Virtue as a mean state}

\noindent In Ch. 5, Aristotle sets out to identify what kind of psychic condition virtue is

\begin{itemize}\item{Argues that it is a state (\emph{hexis}, also often translated as ``disposition''): a settled condition in which we have a tendency to feel and do certain things in certain situations}\end{itemize}

\noindent Then, in Ch. 6, he tries to identify what kind of state it is

\begin{itemize}\item{Here Aristotle is employing his central conception of definition as through genus and species (\emph{per genus et differentiam})}\end{itemize}

\noindent Virtue is concerned with feelings and actions, and feelings and actions can admit of a more and a less, and so can be in excess, deficiency, and intermediacy
\vspace*{2mm}

\noindent Since excess and deficiency are detrimental, and virtue is beneficial, virtue aims at what is intermediate
\begin{itemize}\item{A distinguishes between intermediate ``in the object'' and ``relative to us''}\end{itemize}

\noindent Virtues, then, are trained dispositions to exhibit emotions and actions over distinct ranges of behavior or objects of desire or aversion
\vspace*{2mm}

\noindent The following diagram represents A's overall scheme:
\vspace*{2mm}

\noindent\underline{Range of object\hspace*{10mm}Kind of emotion\hspace*{10mm}Excess\hspace*{17mm}Mean\hspace*{18mm}Deficiency}
\vspace*{1mm}

\noindent Danger\hspace*{22mm}Fear/Confidence\hspace*{10mm}Rash\hspace*{16mm}Courageous\hspace*{12mm}Coward
\vspace*{1mm}

\noindent Physical goods\hspace*{11mm}Lust etc.\hspace*{20mm}Intemperate\hspace*{8mm}Temperate\hspace*{13mm}Insensible\\\hspace*{2mm}(e.g. food, drink, sex)
\vspace*{1mm}

\noindent Wealth\hspace*{22mm}``Heart''\hspace*{22mm}Wasteful\hspace*{14mm}Generous\hspace*{12mm}Ungenerous
\vspace*{1mm}

\noindent Status\hspace*{24mm}Spirit\hspace*{27mm}Vain\hspace*{16mm}Magnanimous\hspace*{7mm}Pusillanimous
\vspace*{1mm}

\noindent Respect\hspace*{21mm}Temper (anger)\hspace*{11mm}Irascible\hspace*{16mm}Mild\hspace*{21mm}Cold
\vspace*{1mm}

\noindent Social intercourse\hspace*{6mm}Sensibility\hspace*{18mm}Buffoon\hspace*{16mm}Witty\hspace*{21mm}Boor

\section*{``Pre-conditions'' for virtue: Book III}

\noindent\underline{Voluntariness/Responsibility (III 1)}
\vspace*{2mm}

\noindent [1] First criterion: praise and blame are attached to voluntary (\emph{hekon}) actions
\vspace*{1mm}

\noindent [2] Involuntary actions are caused by (a) force or (b) ignorance
\vspace*{1mm}

(a) Force = caused by an \emph{external principle} (e.g. wind blows you over)
\vspace*{1mm}

\hspace*{14.5mm}$\neq$ caused by \emph{compulsion} (e.g. throwing cargo overboard in storm, the origin of the motion still\\\hspace*{23mm} lies within the agent)
\vspace*{1mm}

(b) Ignorance of the particulars: i) who is doing it, ii) what he is doing, iii) about what or to what he is\\\hspace*{10mm} doing it, iv) what he is doing it with, v) for what result, vi) in what way
\vspace*{1mm}
\vspace{1mm}

\noindent [3] Voluntary actions are those for which the principle is in the agent, and the particulars of which are known by the agent
\vspace*{1mm}

\noindent [4] A draws two conclusions from this discussion:
\vspace*{1mm}

[4a] Actions caused by appetite and spirit are voluntary
\vspace*{1mm}

[4b] Some actions performed by animals and children are voluntary (since caused by appetite and spirit)
\vspace*{3mm}

\noindent\underline{Decision (III 2-4)}
\vspace*{2mm}

\noindent [1] Deliberation involves: (a) rational thought about means-ends and (b) a grasp of the universal
\vspace*{1mm}

\noindent [2] Deliberation concerns: (c) what is possible and (d) what is up to us
\vspace*{1mm}

\noindent [3] Decision is deliberative desire to perform an action that is up to us
\vspace*{1mm}

\noindent [4] All actions caused by decision are voluntary (but not vice-versa, e.g. actions solely by appetite and spirit)
\vspace*{1mm}

\noindent [5] Virtuous actions are caused by decision (but it is not always necessary to deliberate explicitly before each decision leading to a specific action)
\vspace*{3mm}

\noindent\underline{Responsibility for character (III 5)}
\vspace*{2mm}

\noindent Is vice involuntary? It might seem to be if people are not responsible for what appears good to them, they might not be responsible for ignorance of what should be aimed at.
\vspace*{1mm}

\noindent Since A thinks that knowledge of the end is necessary for voluntary action, blameless ignorance of the end might mean that people do not act viciously voluntarily. (This would be disastrous for A's overall picture)
\vspace*{1mm}

\noindent A tries to respond that while no one can change their character or what appears good to him/her through one decision, characters and conceptions of good are caused by repeated voluntary actions

\end{document}
