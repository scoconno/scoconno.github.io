% !TEX encoding = UTF-8 Unicode
% !TEX TS-program = xelatex

\documentclass[11pt]{article}
\usepackage{fontspec}
\defaultfontfeatures{Mapping=tex-text}
\usepackage{xunicode}
\usepackage{xltxtra}
\usepackage{verbatim}
\usepackage[margin= 1 in]{geometry} % see geometry.pdf on how to lay out the page. There's lots.
\geometry{letterpaper} % or letter or a5paper or ... etc
%\usepackage[parfill]{parskip}    % Activate to begin paragraphs with an empty line rather than an indent 
\usepackage{mathrsfs}
\usepackage{bbding}
\usepackage[usenames,dvipsnames]{color}
\usepackage{natbib}
\usepackage{stmaryrd}
%\usepackage{mathpartir}
\usepackage{txfonts}
\usepackage{graphicx}
\usepackage{fullpage}
\usepackage{hyperref}
\usepackage{amssymb}
\usepackage{epstopdf}
\usepackage{fontspec}
%\setmainfont{Hoefler Text}
\setmainfont[BoldFont={Minion Pro Bold}]{Minion Pro}
\usepackage{hyperref}
\usepackage{lastpage, fancyhdr}
%\usepackage{setspace}
\pagestyle{fancy}
\lhead{}
\chead{Lecture 15, Aristotle \emph{NE}, Book II\space---\space Handout} 
\rhead{}
\lfoot{}
\cfoot{\thepage\space of \pageref{LastPage}} 
\rfoot{}
\footskip=30 pt
\headsep=20pt
\thispagestyle{empty}
\hypersetup{colorlinks=true, linkcolor=Sepia, urlcolor=Sepia, citecolor=BrickRed}
\DeclareGraphicsRule{.tif}{png}{.png}{`convert #1 `dirname #1`/`basename #1 .tif`.png}
\usepackage{polyglossia}
\setdefaultlanguage{english}
\setotherlanguage{greek}
\newfontfamily\greekfont{Gentium Plus}
\newcommand{\gk}[1]{\textgreek{#1}}
\newcommand{\gloss}[1]{(\textgreek{#1})}

\usepackage{covington}
\usepackage{fixltx2e}
\usepackage{graphicx}
\begin{document}

%\maketitle
\thispagestyle{empty}
\begin{center} \LARGE{PHIL 321\\ Lecture 15: Aristotle's \emph{Nicomachean Ethics}, Book II}\\ \vspace*{2mm}
\large{10/17/2013}\end{center}
\thispagestyle{empty}\vspace*{3mm}
\vspace*{-8mm}

\section*{Aristotle's conception of the soul (Bk. I, Ch. 13)}

\noindent A's main work on the soul is \emph{De Anima}, which we will look at later
\vspace*{2mm}

\noindent A takes the soul to have both rational and non-rational aspects, and takes human motivation to come in three main kinds (reason, spirit, appetite), similarly to Plato in the \emph{Republic} (correspondingly, there are three kinds of objects of desire/decision: the pleasant, the fine, and the expedient/good (Bk. II, Ch. 3))

\begin{itemize}\item{However, A thinks it may be wrong to think of these as distinct \emph{parts} of the soul, as opposed to distinct aspects of one and the same thing (as the convex and the concave are two aspects of one and the same curved line)}\end{itemize}

\noindent The non-rational aspect itself has two aspects: one that is wholly non-rational (the part responsible for maintenance of the human body), one that is non-rational but can ``obey'' and be ``trained'' by reason
\vspace*{-2mm}

\section*{Virtue (Bk. II, Chs. 1-4)}

\noindent Corresponding to the two aspects of the soul are two ``kinds'' of virtues: virtues of intellect (the aspect that has reason strictly speaking; discussed in Book 6) and virtues of character (the aspect that can obey reason; discussed in Books II-V)
\vspace*{2mm}

\noindent Virtue of character is acquired primarily through habituation (by repeatedly performing just, temperate, etc. acts, one \emph{becomes} just, temperate, etc.); A compares this to the way in which someone acquires a craft (Ch. 1)
\vspace*{1mm}

\noindent A notes that the states of character we acquire tend to be ``ruined'' by excess and deficiency (Ch. 2)

\begin{itemize}\item{E.g. if people stand firm against nothing, they  become cowardly, but if they fear nothing and rush into every confrontation, they become rash; if they give in to every pleasure, they become intemperate, but if they refrain from all they become ``a kind of insensible person''}\end{itemize}

\noindent To evaluate the kind of character people have, A claims we can't just look at their actions---we also have to look at the pleasures and pains ``in consequence of their actions'': ``Virtues are concerned with actions \emph{and} feelings'' (1104b14) (Ch. 3)
\vspace*{2mm}

\noindent The claim that we become virtuous by performing virtuous acts seems to present a puzzle: if we perform virtuous acts, aren't we \emph{already} virtuous? (Ch. 4)
\vspace*{2mm}

\noindent To solve this, A distinguishes between [1] performing a virtuous act and [2] performing a virtuous act \emph{virtuously}---anyone can do the former (even vicious people)
\vspace*{2mm}

\noindent To perform a virtuous act \emph{virtuously}, people must also:
\begin{itemize}\item{[A] Act knowing (what exactly? A says this ``counts for little,'' suggesting the demand can't be too high)}\item{[B] Decide to perform the action and decide to perform it \emph{for its own sake}}\item{[C] Act from a firm and unchanging state}\end{itemize}

\end{document}
