\documentclass[]{article}
\usepackage{lmodern}
\usepackage{amssymb,amsmath}
\usepackage{ifxetex,ifluatex}
\usepackage{fixltx2e} % provides \textsubscript
\ifnum 0\ifxetex 1\fi\ifluatex 1\fi=0 % if pdftex
  \usepackage[T1]{fontenc}
  \usepackage[utf8]{inputenc}
\else % if luatex or xelatex
  \ifxetex
    \usepackage{mathspec}
    \usepackage{xltxtra,xunicode}
  \else
    \usepackage{fontspec}
  \fi
  \defaultfontfeatures{Mapping=tex-text,Scale=MatchLowercase}
  \newcommand{\euro}{€}
\fi
% use upquote if available, for straight quotes in verbatim environments
\IfFileExists{upquote.sty}{\usepackage{upquote}}{}
% use microtype if available
\IfFileExists{microtype.sty}{%
\usepackage{microtype}
\UseMicrotypeSet[protrusion]{basicmath} % disable protrusion for tt fonts
}{}
\ifxetex
  \usepackage[setpagesize=false, % page size defined by xetex
              unicode=false, % unicode breaks when used with xetex
              xetex]{hyperref}
\else
  \usepackage[unicode=true]{hyperref}
\fi
\hypersetup{breaklinks=true,
            bookmarks=true,
            pdfauthor={},
            pdftitle={The Problem of Evil Essay},
            colorlinks=true,
            citecolor=blue,
            urlcolor=blue,
            linkcolor=magenta,
            pdfborder={0 0 0}}
\urlstyle{same}  % don't use monospace font for urls
\setlength{\parindent}{0pt}
\setlength{\parskip}{6pt plus 2pt minus 1pt}
\setlength{\emergencystretch}{3em}  % prevent overfull lines
\setcounter{secnumdepth}{0}

\title{The Problem of Evil Essay}
\date{}

\begin{document}
\maketitle

\subsubsection{Plagiarism}\label{plagiarism}

Please review the plagiarism policy on the syllabus. It is critical that
you prepare your assignment by yourself. Use only the textbook and
handouts---it will take you less time to work through these materials
than to find and read other sources. I will be checking for significant
overlaps between submission as well as checking answers against
Wikipedia, internet search results, standard essay sites, etc. If you
include material in your essay without citing it, you will receive 0 for
the assignment. A second violation will result in a 0 for the course, a
report to the Dean, and a petition for a note to be added to your
permanent academic record.

\subsubsection{Due Date}\label{due-date}

Please consult the syllabus and course website for the due date.

\subsubsection{Late Submissions}\label{late-submissions}

Per the policies outlined in the syllabus, late work will not be
accepted. As the policies also state, there are no make-ups or extra
credit opportunities. Any request for special treatment will be ignored.
If you foresee difficulties submitting work on time, either because of
personal or commitments, then you should start this paper early and
submit it early.

\subsubsection{Format}\label{format}

Please submit your paper as a word file.

\subsubsection{Grading}\label{grading}

Please find the rubric and explanation of it
\href{/Teaching/Grading/}{here}.

\subsubsection{Resources}\label{resources}

Please find links to writing resources \href{/Teaching/Resources/}{here}

\subsubsection{Word Count}\label{word-count}

1000--1200 words. Writing less than 1000 or more than 1200 words will
lose you points.

\subsubsection{Prompt}\label{prompt}

Some believe in the existence of an all loving, all powerful, all
knowing God. Use the existence of evil to argue that such a being cannot
exist. Outline one way a Theist might respond to your argument and then
counter that response.

\subsubsection{Further Instruction}\label{further-instruction}

\begin{itemize}
\itemsep1pt\parskip0pt\parsep0pt
\item
  Write this paper pretending to be an Atheist.
\item
  This essay has three parts. The first asks you to use the Problem of
  Evil to disprove God's existence. The second asks you to explain one
  way the Theist might respond and the third asks you to rebut that
  response.
\item
  This is an academic paper. It must include a thesis statement, an
  introduction, conclusion, and proper citation. See the resources page
  for guidance on writing philosophy papers.
\item
  Assume a hostile audience. You need to change the minds of the Theist.
  Be clear. Offer full, but concise explanations, etc.
\item
  Coffee Test: I will spend no more than 15 minutes reading each paper.
  Write \emph{clearly}!
\end{itemize}

\end{document}
