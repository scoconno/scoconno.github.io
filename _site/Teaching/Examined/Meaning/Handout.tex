\documentclass[]{article}

\usepackage{amssymb,amsmath}
\usepackage{ifxetex,ifluatex}
\usepackage{fixltx2e} % provides \textsubscript
\ifnum 0\ifxetex 1\fi\ifluatex 1\fi=0 % if pdftex
  \usepackage[T1]{fontenc}
  \usepackage[utf8]{inputenc}
\else % if luatex or xelatex
  \ifxetex
    \usepackage{mathspec}
    \usepackage{xltxtra,xunicode}
  \else
    \usepackage{fontspec}
  \fi
  \defaultfontfeatures{Mapping=tex-text,Scale=MatchLowercase}
  \newcommand{\euro}{€}
\fi
% use upquote if available, for straight quotes in verbatim environments
\IfFileExists{upquote.sty}{\usepackage{upquote}}{}
% use microtype if available
\IfFileExists{microtype.sty}{%
\usepackage{microtype}
\UseMicrotypeSet[protrusion]{basicmath} % disable protrusion for tt fonts
}{}
\ifxetex
  \usepackage[setpagesize=false, % page size defined by xetex
              unicode=false, % unicode breaks when used with xetex
              xetex]{hyperref}
\else
  \usepackage[unicode=true]{hyperref}
\fi
\hypersetup{breaklinks=true,
            bookmarks=true,
            pdfauthor={},
            pdftitle={Optimism without God},
            colorlinks=true,
            citecolor=blue,
            urlcolor=blue,
            linkcolor=magenta,
            pdfborder={0 0 0}}
\urlstyle{same}  % don't use monospace font for urls
\setlength{\parindent}{0pt}
\setlength{\parskip}{6pt plus 2pt minus 1pt}
\setlength{\emergencystretch}{3em}  % prevent overfull lines
\setcounter{secnumdepth}{0}

\title{Optimism without God}
\date{}

\begin{document}
\maketitle

\section{Optimism about the meaning of life without
God}\label{optimism-about-the-meaning-of-life-without-god}

\subsection{Introduction}\label{introduction}

Recall Tolstoy's question:

\begin{quote}
\ldots{} My question - that which at the age of fifty brought me to the
verge of suicide - was the simplest of questions, lying in the soul of
every man from the foolish child to the wisest elder: it was a question
without an answer to which one cannot live, as I had found by
experience. It was: ``What will come of what I am doing today or shall
do tomorrow? What will come of my whole life?'' (Tolstoy,
p.14)\footnote{Tolstoy, Leo, `A Confession', 1882}
\end{quote}

\begin{quote}
Differently expressed, the question is: ``Why should I live, why wish
for anything, or do anything?'' It can also be expressed thus: ``Is
there any meaning in my life that the inevitable death awaiting me does
not destroy? (Tolstoy, p.14)
\end{quote}

Life has meaning only if it has significant value or purpose over time,
where this value makes life choice worthy. There are two different ways
of understanding this value:

\begin{itemize}
\item
  \textbf{Internal Value:} the value or purpose that comes when people
  see their goals or purposes as inherently valuable or worthwhile.
\item
  \textbf{External Value:} Meaning or purpose that comes from outside of
  ourselves in relationship to something that we may or may not be aware
  of.
\end{itemize}

When we ask about the meaning of life, we are asking about internal
value. We are asking why we should feel that there is something in our
lives that makes them worthwhile. Is there any project or goal that
could shape our psychology so dramatically that we are motivated to get
up in the morning, keep going, and find all the trials and tribulations
of life worthwhile? Pessimists, recall, claim no. Their argument: 1.
Life is choice worthy only if it has internal value. 2. Life has
internal value only if life has external value. 3. Life has no external
value. 4. Life has no internal value (from 1--3). 5. Life is not choice
worthy (from 1 \& 4). This argument is valid; the conclusion follows
form the premises. Is it sound, i.e., are the premises true? The most
important Premises are 2 and 3, which we saw Tolstoy arguing for via a
fable. We can summarize his argument for Premise 2 as follows: \#\#
OptimismOptimists claim that Tolstoy's arguments for Premises 2 and 3
fail. There are two versions of Optimism. The first version, which we
discussed last week, accepts Premises 1 and 2, but rejects Premise 3.
They find external value in religion. The second type of Optimist
accepts Premise 3, that life has no external value, but denies that
internal value depends on there being external value, i.e., they deny
Premise 2. The first type of Optimism is associated with Theism, the
second with Atheism. I discussed Theism in Note 2. I discuss Atheism in
this handout.

\subsection{Atheism}\label{atheism}

Our second optimistic approach to the meaning of life rejects the need
for external value altogether. These optimists ask us to consider the
lives of people who clearly lead meaningful lives. If we can identify
why we think those lives valuable, we might be able to decide how, we
ourselves, can live meaningful lives without external value.

Who has lived a meaningful life? M.L.K, Gandhi, Einstein, Leonardo Di
Vinci are clear candidates. Can we see anything similar to their lives?
Here are two candidates:

\begin{itemize}
\itemsep1pt\parskip0pt\parsep0pt
\item
  \textbf{Candidate 1:} A person's life is meaningful if and only if
  their life makes them happy.
\end{itemize}

Candidate 1 is an obvious suggestion. Maybe you can live a meaningful
life by just living a happy life. The difficulty with Candidate 1 is
that there are many obvious cases of people who lived meaningful, but
unhappy lives. Gandhi, Mother Teresa, and Einstein, for example were not
happy, but they lived meaningful lives. Additionally, obvious cases of
meaningless lives are filled with a good amount of subjective happiness
and contentedness. So happiness doesn't make a life meaningful. As an
example, watch this vide about the experience machine:
\url{https://www.youtube.com/watch?v=yJ1dsNauhGE}

Susan Wolf has offered the following alternative account:

\begin{itemize}
\itemsep1pt\parskip0pt\parsep0pt
\item
  \textbf{Candidate 2:} A meaningful life is one that is a) actively and
  at least somewhat b) successfully engaged in a project (or projects)
  of c) positive value.\footnote{`The Meaning of Lives', Susan Wolf}Wolf's
  account has three distinct conditions. She argues for each by
  contrasting meaningful and meaningless lives. Her example of a
  meaningless life is `The Blob:' a person who spends every moment in
  front of a television set, drinking beer and watching situation
  comedies. Compare the Blob to the life of Iran Deckard in Phillip
  Dick's \emph{Do Androids Dream of Electric Sheep}, or the soma
  consuming citizens of Aldous Huxley's, \emph{A Brave New World}, or
  the television watchers of Ray Bradbury's \emph{Fahrenheit 451}, or
  the citizens of The Capitol in Suzanne Collins' \emph{The Hunger
  Games}. These are all examples of people who live lives that are
  subjectively pleasant, but are meaningless.
\end{itemize}

\subsubsection{Active vs.~non-Active
Life}\label{active-vs.non-active-life}

Our first condition says that a meaningful life must be one that is
actively engaged. Consider the cases of people whose lives are useless
due to a lack of activity: the Blob sitting on the sofa day in, day out,
drinking beer and watching terrible television. That life is meaningless
precisely because of the lack of activity.

Wolf asks us to conclude with her that a meaningful life must be one
actively engaged in some project, where these projects are any kind of
ongoing activity or involvement. The projects engage the person, they
see them as constituting part of what their life is about and they
pursue them with zest.

\subsubsection{Success vs.~non-Success}\label{success-vs.non-success}

Consider cases in which the project around which somebody has organized
their life is revealed to be bankrupt, e.g., the inventor who devotes
their life to creating an automated car only to be beaten to the punch
by Google. Or the CEO who spends a life developing a business that is
superseded by new technologies, e.g., the CEO of a nuclear power plant
company that ends up seeing nuclear power replaced by solar power. These
lives are clearly tragic. There's something miserable about devoting
your life to a project that never sees fruition. In contrast, the lives
of M.L.K, Gandhi, Mandela were clearly meaningful. Not only did they
single-mindedly pursue some project, they were successful. So too were
those athletes who spent every waking moment not only training for the
olympics, but securing a place and doing well. (Compare this to, say,
the person who trains all day every day to compete for the Olympic track
team, but has never and will never run a mile in less than 10
minutes)These cases show us that for a life to be meaningful it must be
organized around some project that the person succeeds at (or at least
has a very reasonable chance of succeeding at). \#\#\# Projects of
Positive Value Consider cases in which a life, though actively engaged,
is wasted on a project without any positive value. Someone might decide
to adopt as their life project the task of counting the number of grains
of sand on Sandy Hook beach. Someone might decide to take as their
project the task of everyday digging a hole and re-filling it. These
projects energize the person; they actively pursue them. Furthermore,
the person might succeed at these projects. Nevertheless, a life devoted
to such works seems utterly worthless. Wolf concludes from this that a
meaningful life must be one dedicated to project of some positive value.
\#\# Objection The Atheist claims that life can have internal value
without external value. Wolf's defense of this claim relies on the claim
that a life must be devoted to a project of positive value. That is, it
is not enough that you value a project, the project itself must be of
some genuine positive value. It seems we are back where we started.
Tolstoy worried that since he and everyone else will ultimately die,
none of his goals and accomplishments were of any external value. If a
goal had positive value, it would have external value. So Wolf owes us
an account of external value that is compatible with our mortality.

\subsection{Meaning re-considered}\label{meaning-re-considered}

We are at an impasses. Atheists think we can do without external
meaning, but it is unclear that a life without external meaning would be
satisfying. Theists, on the other hand, are convinced that life has
external meaning, but they struggle to give a satisfying account of what
that meaning is. Is there an alternative?

One option is to re-consider Epicureanism, an option towards death that
Tolstoy rejects.

\begin{itemize}
\itemsep1pt\parskip0pt\parsep0pt
\item
  Epicureanism: while knowing the hopelessness of life, make use
  meanwhile of the advantages one has, disregarding the dragon and the
  mice, and licking the honey in the best way, especially if there is
  much of it within reach. This, though, is an unsustainable attitude.
  Many live in terrible conditions. Many have no honey to taste. It is a
  mere accident, claims Tolstoy, that you have good circumstances rather
  than poor, and `'the accident that has today made me a Solomon may
  tomorrow make me a Solomon's slave.'' Epicureans try but cannot
  ultimately forget that all these pleasures are ephemeral. They are as
  easily lost as gained. Nobody can be confident that life will always
  provide these distractions. (Tolstoy, p.22)
\end{itemize}

Tolstoy assumes that the only pleasures in life are ephemeral, that they
only things we can desire are those that ultimately will causes us pain.
Buddha would agree that desiring the ephemeral is a cause of pain, but
is Tolstoy right that we can \emph{only} desire the ephemeral? Read
\emph{Buddhism} in ch.2 for Buddha's views on how desire can be changed
and its pain lessened.

\end{document}
