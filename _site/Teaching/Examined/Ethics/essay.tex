\documentclass[]{article}
\usepackage{amssymb,amsmath}
\usepackage{ifxetex,ifluatex}
\usepackage{fixltx2e} % provides \textsubscript
\ifnum 0\ifxetex 1\fi\ifluatex 1\fi=0 % if pdftex
  \usepackage[T1]{fontenc}
  \usepackage[utf8]{inputenc}
\else % if luatex or xelatex
  \ifxetex
    \usepackage{mathspec}
    \usepackage{xltxtra,xunicode}
  \else
    \usepackage{fontspec}
  \fi
  \defaultfontfeatures{Mapping=tex-text,Scale=MatchLowercase}
  \newcommand{\euro}{€}
\fi
% use upquote if available, for straight quotes in verbatim environments
\IfFileExists{upquote.sty}{\usepackage{upquote}}{}
% use microtype if available
\IfFileExists{microtype.sty}{%
\usepackage{microtype}
\UseMicrotypeSet[protrusion]{basicmath} % disable protrusion for tt fonts
}{}
\ifxetex
  \usepackage[setpagesize=false, % page size defined by xetex
              unicode=false, % unicode breaks when used with xetex
              xetex]{hyperref}
\else
  \usepackage[unicode=true]{hyperref}
\fi
\hypersetup{breaklinks=true,
            bookmarks=true,
            pdfauthor={},
            pdftitle={Essay 1},
            colorlinks=true,
            citecolor=blue,
            urlcolor=blue,
            linkcolor=magenta,
            pdfborder={0 0 0}}
\urlstyle{same}  % don't use monospace font for urls
\setlength{\parindent}{0pt}
\setlength{\parskip}{6pt plus 2pt minus 1pt}
\setlength{\emergencystretch}{3em}  % prevent overfull lines
\setcounter{secnumdepth}{0}

\title{Essay 2}
\date{}

\begin{document}
\maketitle

\subsubsection{Plagiarism}\label{plagiarism}

Please review the plagiarism policy on the syllabus. It is critical that
you prepare your assignment by yourself. Use only the textbook and
handouts---it will take you less time to work through these materials
than to find and read other sources. I will be checking for significant
overlaps between submission as well as checking answers against
Wikipedia, internet search results, standard essay sites, etc. If you
include material in your essay without citing it, you will receive 0 for
the assignment. A second violation will result in a 0 for the course, a
report to the Dean, and a petition for a note to be added to your
permanent academic record.

\subsubsection{Due Date}\label{due-date}

Please consult the syllabus and course website for the due date.

\subsubsection{Late Submissions}\label{late-submissions}

Per the policies outlined in the syllabus, late work will not be
accepted. As the policies also state, there are no make-ups or extra
credit opportunities. Any request for special treatment will be ignored.
If you foresee difficulties submitting work on time, either because of
personal or commitments, then you should start this paper early and
submit it early.

\subsubsection{Format}\label{format}

Please submit the file as a Microsoft Word file through Blackboard.

\subsubsection{Grading}\label{grading}

Please find the rubric and explanation of it
\href{/Teaching/Grading/}{here}.

\subsubsection{Resources}\label{resources}

Please find links to writing resources \href{/Teaching/Resources/}{here}

\subsubsection{Word Count}\label{word-count}

Your dialog must be 1000-1200 words long. Essays shorter than 1000 words
or longer than 1200 words will lose points.

\subsubsection{Prompt}\label{prompt}

Critically evaluate the Cultural Differences Argument for Cultural
Relativism.

\subsubsection{Further Instruction}\label{further-instruction}

This essay covers material contained on pp.123-141. Your essay should
contain the following three parts.

\begin{enumerate}
\def\labelenumi{\arabic{enumi}.}
\itemsep1pt\parskip0pt\parsep0pt
\item
  An explanation of cultural relativism. In this part, you will explain
  what claim the cultural relativist argues for. Do not at this point
  try explain the argument. Students regularly lose points in this
  respect. I will illustrate the mistake as follows: suppose I ask you
  to describe the city of Atlanta and you respond by telling me about
  the variety of different ways of getting to Atlanta. You've missed the
  mark. Similarly, your job in this part is explain the conclusion of
  the argument for Cultural Relativism. It's not to explain how we get
  to that conclusion.
\item
  An identification and explanation of the premises of the argument.
  Here your job is to explain how we get the conclusion. Don't merely
  state the premises. Your job is to discuss each one. Give
  illustrations. I suggest a short paragraph per premise.
\item
  Your assessment of the argument. Here I will be examining you on your
  grasp of the material from ch.1.3. You must decide whether the
  argument is valid and sound.
\end{enumerate}

\end{document}
