\documentclass[]{article}
\usepackage{amssymb,amsmath}
\usepackage{ifxetex,ifluatex}
\usepackage{fixltx2e} % provides \textsubscript
\ifnum 0\ifxetex 1\fi\ifluatex 1\fi=0 % if pdftex
  \usepackage[T1]{fontenc}
  \usepackage[utf8]{inputenc}
\else % if luatex or xelatex
  \ifxetex
    \usepackage{mathspec}
    \usepackage{xltxtra,xunicode}
  \else
    \usepackage{fontspec}
  \fi
  \defaultfontfeatures{Mapping=tex-text,Scale=MatchLowercase}
  \newcommand{\euro}{€}
\fi
% use upquote if available, for straight quotes in verbatim environments
\IfFileExists{upquote.sty}{\usepackage{upquote}}{}
% use microtype if available
\IfFileExists{microtype.sty}{%
\usepackage{microtype}
\UseMicrotypeSet[protrusion]{basicmath} % disable protrusion for tt fonts
}{}
\usepackage{longtable,booktabs}
\ifxetex
  \usepackage[setpagesize=false, % page size defined by xetex
              unicode=false, % unicode breaks when used with xetex
              xetex]{hyperref}
\else
  \usepackage[unicode=true]{hyperref}
\fi
\hypersetup{breaklinks=true,
            bookmarks=true,
            pdfauthor={},
            pdftitle={Free Will},
            colorlinks=true,
            citecolor=blue,
            urlcolor=blue,
            linkcolor=magenta,
            pdfborder={0 0 0}}
\urlstyle{same}  % don't use monospace font for urls
\setlength{\parindent}{0pt}
\setlength{\parskip}{6pt plus 2pt minus 1pt}
\setlength{\emergencystretch}{3em}  % prevent overfull lines
\setcounter{secnumdepth}{0}

\title{Free Will}
\date{}

\begin{document}
\maketitle

\subsection{Causal Determinism}\label{causal-determinism}

Our argument against free will relies on two claims, one a general claim
that applies to all events in the universe, a second a claim about what
free will would require of us.

Causal determinism: : Given a specified way things are at a time
\emph{t}, everything which happens at a time later than \emph{t} is
fixed as a matter of natural law.

Consider a rock falling down a side of a mountain on Mars this morning.
That rock was set in motion by some preceding event, perhaps a rover hit
and dislodged it. Causal determinism says that the entire state of the
universe this morning, including that rock falling down the mountain,
was completely determined by the state of the universe at any time
before this morning plus the laws governing motion, gravity, etc. This
might seem far-fetched. But that's exactly what our scientific laws are;
they determine how things will be in the future given how things are in
the past. Suppose, then, there was a God like scientist. By knowing the
state of the universe 1000 years ago plus all the rules governing our
universe, our God-like scientist would know for certain that the rock
was going to fall this morning.

The problem is that Causal determinism seems to jar with a fairly
ordinary sense of free-will.

Free Will: : An agent, S, freely chooses to do F at time \emph{t1} if
and only if it was S's power at time \emph{t1} to bring either F or not
F about.

Suppose Mike Tyson force feeds me some cake. No-one would claim that I
ate that cake freely. Since it was not in my power to not eat the cake,
the eating was not in my power at all. Similarly, if you were forced to
take a job, or forced to take a course, and it was not in your power to
do otherwise, you would not be taking that job or pursuing that course
of your own free will. If you don't have the power to do otherwise, then
it was never in your power in the first place. The problem is that
Causal Determinism seems to say that it is never in our power to do
otherwise:

\emph{Argument Against Free-Will}

\begin{enumerate}
\def\labelenumi{\arabic{enumi}.}
\itemsep1pt\parskip0pt\parsep0pt
\item
  If a person acts of her own free will, then she could have done
  otherwise.
\item
  If determinism is true, no one can do otherwise than one actually
  does.
\item
  Therefore, if determinism is true, no one acts of her own free will.
\end{enumerate}

The key premise is 2. Recall that Causal Determinism says that states of
the universe, including what you are doing at any particular time, are
completely determined by how the universe was in the past, even the far
far past, plus the laws of nature. If that's true, then my eating the
cake was causally determined, it is already set in stone. In that case,
I had no power to not eat the cake. Causal Determinism constrains my
actions as much as Mike Tyson can constrain my actions.

Philosophers have been concerned with determining, first, whether Causal
Determinism is really true, and, second, whether free-will can be found
compatible with Causal Determinism:

\begin{longtable}[c]{@{}lcr@{}}
\toprule
& Free Will is Not Possible & Free Will is Possible\tabularnewline
\midrule
\endhead
\textbf{Causal Determinism is true} & Hard Determinism &
Compatibilism\tabularnewline
\textbf{Causal Determinism is false} & Hard Indeterminism &
Libertarianism\tabularnewline
\bottomrule
\end{longtable}

\subsection{Compatibilism}\label{compatibilism}

Compatibilists accept Causal Determinism but they claim that humans are
still morally responsible for their behavior and deserving of blame and
punishment. If a person is morally responsible for their behavior, then
they must have control over their own conduct. Compatibilists offer an
alternative account of Free Will that they hope is a) compatible with
Causal Determinism, and b) sufficient for moral responsibility.

\textbf{Forking Path view of Free Will (rejected by compatibilists)}

\begin{itemize}
\itemsep1pt\parskip0pt\parsep0pt
\item
  An agent has control over her conduct at a moment in time if she has
  the ability to select among, or choose between, alternative courses of
  action, i.e., choose between alternative future paths.
\end{itemize}

This forking path view of freedom jars with causal determinism. The
Compatibilist denies, though, that freedom should be understood along
these lines. They instead offer the following account of freedom:

\textbf{Source of Action}

\begin{itemize}
\itemsep1pt\parskip0pt\parsep0pt
\item
  An agent freely does F if 1) F arises from her internal states and
  character, and 2) are not forced by external conditions or agents.
\end{itemize}

Suppose that I have a very sweet tooth. My having a sweet tooth will
cause me to eat the cake. So that action, my eating the cake, is in some
strong sense comes from me and not the outside; it's because of how I am
that I ate the cake. The compatibilist claims that this is all that is
required for an action to be free---it arises from inside of you, from
your personality, and is not caused by external pressure.

Our personality is made up a variety of different traits like honesty,
courage, greed, etc. Some people are honest and courageous. Some are
honest, but cowardly. Traits are complex dispositions to notice,
construe, think, desire, and act in characteristic ways. To be generous,
for instance, is to be disposed to notice occasions for giving, to
construe ambiguous social cues charitably, to desire to give people
things they want, need, or would appreciate, to deliberate well about
what they want, need, or would appreciate, and to act on the basis of
such deliberation.

Character traits are normally characterized with these three claims:

\begin{enumerate}
\def\labelenumi{\arabic{enumi}.}
\item
  Robustness Claim: an individual with a particular character trait will
  exhibit trait-relevant behavior across a broad spectrum of
  trait-relevant situations. Such traits are said to be ``robust''
  traits, e.g., an honest person will tend to tell the truth to friends,
  family members, co-workers, students, etc.
\item
  Stability Claim: traits are relatively stable over time. A soldier who
  behaves courageously for a significant period of time is courageous. A
  soldier who behaves non-courageously for a significant period of time
  is not courageous. A soldier will not become or cease to be courageous
  overnight.
\item
  Integrity Claim: there is a correlation between having one trait and
  having another, e.g., a person who is temperate with regard to the
  pleasures derived from food is likely to also be temperate with regard
  to the pleasures derived from sexual intercourse. Likewise, an
  individual with a particular vice is likely to possess other vices.
\end{enumerate}

The compatibilist is committed to the claim that we have character
traits that are robust, stable, and integrated with one another. They
believe that an action is our own just when it arises from our unique
set of traits; if my character traits makes me eat the cake, then eating
the cake, according to the compatibilist, is free.

\subsection{Objections}\label{objections}

Here are two puzzling features of the compatibilist's view. I'll raise a
stronger objection below.

\begin{enumerate}
\def\labelenumi{\arabic{enumi}.}
\item
  Note the distinction between freely \emph{acting} a certain way and
  freely \emph{choosing} to act a certain way. Compatibilists are
  claiming you act freely when your actions arise from your character.
  You do not freely choose to act. But can your actions be free if they
  you did not freely choose them?
\item
  Since the Compatibilist accepts Causal Determinism, they accept that
  your character is determined by the past.If your character is shaped
  by circumstances outside your control, are the actions that arise from
  your character free?
\end{enumerate}

\subsection{Situationism}\label{situationism}

The compatibilist is committed to the claim that we have character
traits that are robust, stable, and integrated with one another. But
there is strong evidence that there is no such traits at all.

Psychologists have been concerned to decide when the features of the
situation we are in rather than our character traits cause our behavior.
Some have argued that all our behavior is caused by external factors. On
this view, there are no character traits whatsoever:

\begin{quote}
``\ldots{} modern experimental psychology has discovered that
circumstance has surprisingly more to do with how people behave than
traditional images of character and virtue allow (John Doris 2002,
ix).''
\end{quote}

Gilbert Harmon expresses this idea as follows:

\begin{quote}
In trying to characterize and explain a distinctive action, ordinary
thinking tends to hypothesize a corresponding distinctive characteristic
of the agent and tends to overlook the relevant details of the agent's
perceived situation\ldots{}. Ordinary attributions of character traits
to people are often deeply misguided and it may even be the case that
there\ldots{} {[}are{]} no ordinary traits of the sort people think
there are (Harman 1999, 315f).
\end{quote}

The strongest version of the view is called Situationism, which can be
understood as comprised of three central claims:

\begin{enumerate}
\def\labelenumi{\arabic{enumi}.}
\itemsep1pt\parskip0pt\parsep0pt
\item
  Non-robustness Claim: moral character traits are not robust---that is,
  they are not consistent across a wide spectrum of trait-relevant
  situations. Whatever moral character traits an individual has are
  situation-specific.
\item
  Consistency Claim: while a person's moral character traits are
  relatively stable over time, this should be understood as consistency
  of situation specific traits, rather than robust traits.
\item
  Fragmentation Claim: a person's moral character traits do not have the
  evaluative integrity suggested by the Integrity Claim. There may be
  considerable disunity in a person's moral character among her
  situation-specific character traits.
\end{enumerate}

There is empirical evidence, the situationist claims, for each of their
claims:

\begin{itemize}
\itemsep1pt\parskip0pt\parsep0pt
\item
  Hugh Hartshorne and M. A. May's study of the trait of honesty among
  school children found no cross-situational correlation. A child may be
  consistently honest with his friends, but not with his parents or
  teachers. From this and other studies, Hartshorne and May concluded
  that character traits are not robust but rather ``specific functions
  of life situations'' (Hartshorne and May 1928, 379f).
\item
  By-stander effect:
\end{itemize}

Other studies further call into question the Integrity Claim of the
Traditional View. For example, in one experiment persons who found a
dime in a phone booth were far more likely to help a confederate who
dropped some papers than were those who did not find a dime. Another
experiment involved seminary students who agreed to give a talk on the
importance of helping those in need. On the way to the building where
their talks were to be given, they encountered a confederate slumped
over and groaning. Those who were told they were already late were much
less likely to help than those who were told they had time to spare.
These experiments are taken to show that minor factors without moral
significance (finding a dime, being in a hurry) are strongly correlated
with people's helping behavior.

Perhaps most damning for the robust view of character are the results of
the experiments conducted by Stanley Milgram in the 1960s. In these
experiments the great majority of subjects, when politely though firmly
requested by an experimenter, were willing to administer what they
thought were increasingly severe electric shocks to a screaming
``victim.'' These experiments are taken to show that if subjects did
have compassionate tendencies, these tendencies cannot have been of the
type that robust traits require.

See one such experiment
\href{https://www.youtube.com/watch?v=OSsPfbup0ac\&spfreload=1}{here}

\subsection{Yoga: The Freedom of
Surrender}\label{yoga-the-freedom-of-surrender}

``The Supreme Lord is situated in everyone's heart, O Arjuna, and is
directing the wanderings of all living entities, who are seated as on a
machine, made of the material energy.''

\end{document}
