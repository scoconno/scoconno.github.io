\documentclass[11pt,article,oneside]{memoir}

%%% custom style file with standard settings for xelatex and biblatex. Note that when [minion] is present, we assume you have minion pro installed for use with pdflatex.
%\usepackage[minion]{org-preamble-pdflatex} 

%%% alternatively, use xelatex instead
\usepackage{org-preamble-xelatex} 



\def\myauthor{Author}
\def\mytitle{Title}
\def\mycopyright{\myauthor}
\def\mykeywords{}
\def\mybibliostyle{plain}
\def\mybibliocommand{}
\def\mysubtitle{}
\def\myaffiliation{NJCU}
\def\myaddress{101 Phil}
\def\myemail{soconnor@njcu.edu}
\def\myweb{klk}
\def\myphone{}
\def\myversion{}
\def\myrevision{}
\def\myaffiliation{NJCU}
\def\myauthor{Scott O'Connor}
\def\mykeywords{}
\def\mysubtitle{Syllabus}
\def\mytitle{{\normalsize Phil 101, Summer II 2015, Online. \newline} \HUGE Persons \& Problems}


\begin{document}

%%% If using xelatex and not pdflatex
%%% xelatex font choices
\defaultfontfeatures{}
\defaultfontfeatures{Scale=MatchLowercase}    
% You will need to buy these fonts, change the names to fonts you own, or comment out if not using xelatex.      
\setromanfont[Mapping=tex-text]{Georgia} 
\setsansfont[Mapping=tex-text]{Georgia} 
\setmonofont[Mapping=tex-text,Scale=0.8]{Georgia} 

%% blank label items; hanging bibs for text
%% Custom hanging indent for vita items
\def\ind{\hangindent=1 true cm\hangafter=1 \noindent}
\def\labelitemi{$\cdot$}
%\renewcommand{\labelitemii}{~}

%% RCS info string for version tracking
\chapterstyle{article-3}  % alternative styles are defined in latex-custom-kjh/needs-memoir/
\pagestyle{kjh}

\title{\LARGE \mytitle}     
\author{\Large\myauthor \newline \footnotesize\texttt{\noindent\myemail}}
\date{07/06/2015--08/20/2015}

\published{\,}

\maketitle

% \thispagestyle{kjhgit}

% Copyright Page
%\textcopyright{} \mycopyright


%
% Main Content
%

\section{Course Description and Objectives}

Does God exist? Why live? What should you do with your life?  In this
course, we'll be asking some of these deep philosophical questions. We
will begin by examining two classic arguments for the existence of God
as well as concerns that God's existence is incompatible with the
existence of evil. In the second part of the course, we will ask what determines the moral character of our actions. Do the ends justify the means? We will also ask whether living morally is the same as living a full and meaningful life. In this final part of the course, we will discuss the meaning of life, especially why some philosophers have connected a meaningful life with God's existence.

\section{Learning Objectives}

Upon completing this course, students will be able to (i) read
philosophical texts, (ii) clearly and charitably explain viewpoints that
are not their own, (iii) think critically and philosophically, (iv)
write well-structured prose in which they clearly state a thesis and
persuasively defend it, (v) demonstrate an understanding of several core
philosophical topics.

\section{Reading}

The following are required for this course:

\begin{itemize}
\item
  \href{http://www.amazon.com/Philosophy-Here-Now-Powerful-Everyday/dp/0199765227}{Philosophy
  Here and Now: Powerful Ideas in Everyday Life', by Lewis Vaughn}
  (Available in the campus book store and online retailers)
\end{itemize}

\section{Requirements}

\begin{itemize}
\item \textit{Workload:} Successfully completing this course requires a minimum commitment of 6 hours per week. 

\item \textit{Reading Quizzes:}  6 weekly reading quizzes administered through Blackboard. Once you open a quiz, you will have
1 hour to complete it. This is an automatic function performed
by Blackboard.
\item \textit{Discussion Questions:} 3 discussion questions submitted through Blackboard. Answers will be made public to the entire class after the due date. 

\item \textit{Essays:} Two essays submitted through Blackboard. Essays are graded according to a rubric that can be found at the end of this syllabus. 

\item \textit{Grade Distribution:} 6 Quizzes---5 points each (30 total); 3 Discussion Questions---10 points each (30 total); Essays---25 points each (50 total).

\item \textit{Grade Breakdown:}

 \begin{tabular}{ | l | l | p{2cm} | l | l | }
    \hline 
96--110 & A  & &  77--79 &  C+ \\  
90--95 & A- & &  73--76 & C \\
87-89 & B+ &  &  70--72 & C- \\ 
83--86 & B  & &  60--69 & D\\
80--82 & B - & & 0--59 & F\\ \hline
    \end{tabular}


\end{itemize}




\section{Policies}

\begin{itemize}
\item \textit{Late work \& Make-up Policy:} 
\begin{itemize}
\item All assignments must be submitted through Blackboard by 1:00 pm on the due date (see assignment schedule below).
\item  No make-ups or late work accepted under any circumstances. No exceptions. 
\item Blackboard difficulties are rare and automatically reported to instructors. Under no circumstance will a student's report of a Blackboard difficulty be reason for an extension. It is your responsibility to contact blackboard support for help: \href{dlsupport@njcu.edu}{dlsupport@njcu.edu}. 

\end{itemize}



\item \textit{Communication:} All communication will be through Blackboard. Messages will be responded to within two
days of receiving them. 

\item \textit{Grading Schedule:} Grades will be available within 1 week of an assignment being submitted.



\item \textit{Statement for students with disabilities:} If you are a student
with a disability and wish to receive consideration for reasonable
accommodations, please register with the Office of Specialized Services
and Supplemental Instruction (OSS/SI). To begin this process, complete
the registration form available on the OSS/SI website at
\href{http://www.njcu.edu/Specialized_Services.aspx}{www.njcu.edu/Specialized\_Services.aspx}
(listed under Student Resources-Forms). Contact OSS/SI at 201-200-2091
or visit the office in Karnoutsos Hall, Room 102 for additional
information.
\end{itemize}

\section{Plagiarism}

\begin{itemize} 
\item You are bound by \href{http://www.njcu.edu/uploadedFiles/About_NJCU/Governance_and_Organization/University_Senate/Policies/Academic\%20INTEGRITY\%20POLICY\%20FINAL\%202-04.pdf}{NJCU's Academic Integrity Policy}
\item Penalty for plagiarism:
\begin{itemize}
\item 1st infraction: 0 for the assignment. 
\item 2nd infraction: 0 for the entire course \& application for permanent record on student's transcript. (Repeated violations can lead to expulsion from NJCU). 
\end{itemize}
\end{itemize}


\section{Weekly Course Schedule}
Dates refer to the first day of the week: 
\begin{enumerate}
\item \textit{07/06/2015,} Critical Thinking, Ch. 1.
\item \textit{07/13/2015,} Proofs of God's Existence, Ch. 2.1-2.2. 
\item \textit{07/20/2015,} Proofs of God's Non-Existence , Ch. 2.3.
\item \textit{07/27/2015,} Ethics 1, Ch. 3.1--3.2.
\item \textit{08/03/2015,} Ethics 2, Ch. 3.3--3.4.
\item \textit{08/10/2015,} The Meaning of life, Ch. 8. 
\item \textit{08/17/2015,} Final Essay, no new material.
\end{enumerate}

\section{Assignment Schedule}
Dates refer to the due date. All assignments must be submitted through Blackboard by 1:00pm. No late work accepted. No make-ups. No exceptions. 

\begin{enumerate}
\item \textit{07/13/2015,} Quiz 1, discussion question. 
\item \textit{07/20/2015,} Quiz 2, discussion question 2.
\item \textit{07/27/2015,} Quiz 3, no discussion question. 
\item \textit{08/03/2015,} Quiz 4, essay 1 due.
\item \textit{08/10/2015,} Quiz 5, discussion question 3.
\item \textit{08/17/2015,} Quiz 6, no discussion question.
\item \textit{\textbf{Thur.} 08/20/2015,} essay 2 due.
\end{enumerate}

\newpage
\section{Grading Rubric}

\section*{Essay 1}
\begin{center}


\resizebox{14cm}{!} {
    \begin{tabular}{ | l | l | l | l | l | l | l | l | l |    }
    \hline
    Rubric &  V. Good & Good & OK & Weak & Poor & V. Poor & No Attempt \\  
	 & (10) & (9) & (8) & (7) & (6) & (3)  & (0) \\    \hline
    Grasp of Material (X2) & & & & & & &\\ \hline   
    Application of logical tools &  & & & & & &\\ \hline
	Independent thought &  & & & & & &\\ \hline
 Mechanics & & & & & &  &\\ \hline \hline
 & &  & & &  & \textbf{Total}  & \\ \hline
    \end{tabular}
}


\end{center}


% \subsection*{Explanation}


%% Uncomment if you want a printed bibliography.
%\printbibliography 

\end{document}