% !TEX encoding = UTF-8 Unicode
% !TEX TS-program = xelatex

\documentclass[11pt]{article}
\usepackage{polyglossia}
\usepackage{amssymb}
\usepackage{fullpage}
\usepackage{hyperref}
\setdefaultlanguage{english}
\setotherlanguage{greek}
\newfontfamily\greekfont{Gentium Plus}
\newcommand{\gk}[1]{\textgreek{#1}}

\title{Rosalind Hursthouse: `Virtue Ethics'}
\author{}
\date{}

\begin{document}

\maketitle

\noindent So, briefly recap what one of the upshots of Anscombe's MMP was
\vspace*{2mm}

\noindent Here Hursthouse outlines ways to make good on some of A's charges: virtue, practical wisdom, eudaimonia
\vspace*{2mm}

\noindent So, one way in which Virtue Ethics differed, at least initially, was by focusing on a wider range of topics than was currently discussed
\vspace*{2mm}

\noindent Both Consequentialism and Utilitarianism focused on right/wrong action; but, there is a whole host of important notions needing philosophical reflection and clarification: virtues, motives, character, moral education, happiness, emotions, how should I live, what sort of person should I be
\vspace*{2mm}

\section{Virtue}

\noindent Does what she says about virtue line up with Aristotle?
\vspace*{2mm}

\noindent It is multi-track; isn't just about actions but motives, reactive attitudes and all kinds of things
\vspace*{2mm}

\noindent We have here a fairly interesting description of differing value of continence vs. virtue: `plausibility of this depends on exactly what ``makes it hard'''

\begin{itemize}\item{Circumstances in which she acts: e.g. poor; then it is admirable}\item{If it is an imperfection in her character--temptation to keep what is not hers--then it is not}\end{itemize}

\noindent Is that right? How does, say, fear work at this? Fear + pain?
\vspace*{2mm}

\section{Practical wisdom}

\noindent Interesting to note how vague this discussion is, especially as it relates to Aristotle
\vspace*{2mm}

\noindent ``Quite generally, given that good intentions are intentions to act well or “do the right thing”, we may say that practical wisdom is the knowledge or understanding that enables its possessor, unlike the nice adolescents, to do just that, in any given situation. The detailed specification of what is involved in such knowledge or understanding has not yet appeared in the literature, but some aspects of it are becoming well known''
\vspace*{2mm}

\noindent Here's an interesting idea: she claims that the virtuous person (practically wise person) does not \emph{see} things the same way; they may be aware that they could gain personally from, say, not returning money, but they don't in any way see that as a reason to do it; it isn't that they think it's  reason outweighed by other reason, it just isn't a reason

\section{Eudaimonia}

\noindent `Flourishing'---problematic because even plants can flourish; but, gets the objectivity
\vspace*{2mm}

\noindent `Happiness'---subjectively determined: up for me, not you, to pronounce on whether I am happy, for, barring advanced cases of self-deception, if I think I am happy then I am; it is not something I can be wrong about'
\vspace*{2mm}

\noindent The claim that this is, straightforwardly, a mistaken conception, reveals the point that eudaimonia is, avowedly, a moralized, or “value-laden” concept of happiness, something like “true” or “real” happiness or “the sort of happiness worth seeking or having.” It is thereby the sort of concept about which there can be substantial disagreement between people with different views about human life that cannot be resolved by appeal to some external standard on which, despite their different views, the parties to the disagreement concur.
\vspace*{2mm}

\noindent What makes a character trait a virtue?

\begin{itemize}\item{Eudaimonism: need a previous conception of eudaimonia; character traits are virtues insofar as they promote eudaimonia}\item{Pluralism: Good life=morally meritorious life; morally meritorious life=life responsive to demands of the world); virtues are those character traits in virtue of which their possessor is thus responsive}\item{Naturalism: good life is life characteristically lived by someone who is good qua human being, virtues enable possessor to live such a life because virtues just are those character traits that make their possessor good qua human being}\end{itemize}

\noindent Where does Aristotle fall on this?

\section{The objections}

\begin{itemize}\item{Application problem: an ethical theory should come up with universal rules or principles which would amount to a decision procedure and that non-virtuous people could understand and apply them}\begin{itemize}\item{Basically agreed that this is a bad expectation for an ethical theory}\item{Presumably, Aristotle also isn't in this business, that's not what he was trying to do}\end{itemize}\item{It is not action guiding; it isn't normative in the sense in which kantianism and consequentialism is normative}\begin{itemize}\item{Just a supplement, rather than an alternative, to them}\item{Long list of vice terms: don't do actions that would be irresponsible, feckless, lazy, inconsiderate, and so on}\item{We will look at this in more detail on Tuesday}\end{itemize}\item{Cultural relativity}\begin{itemize}\item{Here's a question: how specific are virtues?}\item{In what way does this affect all the normative theories: happiness or welfare conceptions differ; rules of conduct differ}\item{Can we claim that virtues are not relative to culture}\end{itemize}\item{Conflicts}\begin{itemize}\item{Is Aristotle subject to this objection? Can we come up with cases where it seems clear that there will be conflicts among the virtues?}\item{How to resolve if so?}\item{Charity prompts me to kill the person who would be better off dead, but justice forbids it. Honesty points to telling the hurtful truth, kindness and compassion to remaining silent or even lying.}\item{Even if there are irresolvable conflicts, is that a problem?}\end{itemize}\item{Self-effacing: an ethical theory is self-effacing if what makes it the case that a certain action is what ought to be done should \emph{not} be the agent's motive for doing it}\begin{itemize}\item{Is Aristotle's theory self-effacing?}\item{What is the ultimately justification for doing something, according to A?}\item{Is that a problem}\end{itemize}\item{How do we get morally good behavior in an account of \emph{eudaimonia}}\begin{itemize}\item{Some say that eudaimonia requires Aristotle's teleology; others that it is mere rationalization of personal or cultural values}\item{But, science is showing that we are, indeed, social animals, by nature; so, we don't need to explain why cooperation would be a realization of our nature}\item{Like other social animals, our desires are not purely self-directed}\end{itemize}\item{Egoistic: Is A's theory egoistic}\begin{itemize}\item{What is it to act virtuously for its own sake?}\item{Is it not to act for the reasons that make it the virtuous thing to do?}\item{If so, what's the problem?}\item{You're going to get situations where the virtuous person must give up his or her life}\item{Again, if so, how is this egoistic in some objectionable sense}\item{The basic idea is that the ultimate motivation has to be states out there in the world}\item{but, that's fine}\item{Bogus distinction between self-regarding and other-regarding virtues}\end{itemize}\end{itemize}



\end{document}