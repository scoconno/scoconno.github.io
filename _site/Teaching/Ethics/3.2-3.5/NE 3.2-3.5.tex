% !TEX encoding = UTF-8 Unicode
% !TEX TS-program = xelatex

\documentclass[11pt]{article}
\usepackage{polyglossia}
\usepackage{amssymb}
\usepackage{fullpage}
\usepackage{hyperref}
\setdefaultlanguage{english}
\setotherlanguage{greek}
\newfontfamily\greekfont{Gentium Plus}
\newcommand{\gk}[1]{\textgreek{#1}}

\title{\emph{Nicomachean Ethics} 3.2-3.5}
\author{}
\date{}

\begin{document}

\maketitle

\section*{Recap}

\noindent Last class we looked at Aristotle's account of voluntary action: the ultimate view is that voluntary actions are those that have their originating principle within the agent, who is aware of the six particulars
\vspace*{2mm}

\noindent Aristotle says that there are two particulars in particular that are important: what he is doing and for what end
\vspace*{2mm}

\noindent One question about this is, what is involved in knowing what you are doing? Another: how does this relate to the three conditions on a virtuous actions being done virtuously back in 2.4?
\vspace*{2mm}

\noindent Another issue, from a question from Catrina, is what is involved in `to what end'?
\vspace*{2mm}

\noindent Her worry was that, it is impossible for someone to know all the things that could be said to be consequences of their action, so how do we determine when an action has reached its end? And, how does that affect voluntariness?


\section*{Kant v. Hume v. Aristotle}

\noindent Discuss Kantian vs. Humean conceptions of practical reason; some issues will arise in what follows about where Aristotle fits in:\textbf{THIS IS ABOUT ULTIMATE ENDS}

\begin{itemize}\item{Hume (roughly): The ends are given, they are the things we have for reasons, so they aren't rational or irrational; rationality is a matter of figuring out how best to get to the ends, that are simply given (different stories: some recent stories go evolutionary, some just say brute facts)}\item{Kant (roughly): Ends are determined by reason; we can figure out what ultimate ends we ought to pursue by thinking things through; what, given the nature of reason, is it rational to pursue}\end{itemize}

\section*{3.2}

\noindent One major question: Is decision particular or universal; what is the content of decisions going to be?

\noindent\textbf{READ 1111b5-7} Decision = `prohairesis' (taken before or in preference to); Aristotle claims that it is more proper to virtue and it distinguishes characters from one another better than actions do
\vspace*{2mm}

\noindent\textbf{WHAT DO PEOPLE THINK HE MEANS BY THIS?}
\vspace*{2mm}

\noindent\textbf{READ 111b8-11}: All decided upon actions are voluntary, but not vice-versa; children act voluntarily but not on decision; spur of the moment actions are said to be voluntary but not to accord with decision
\vspace*{2mm}

\noindent Pause to discuss this; does that sound right; we must be working with a fairly high-powered notion of decision if its the kind of thing children can't do
\vspace*{2mm}

\noindent Decision is not appetite or spirit
\vspace*{2mm}

\noindent We then get our first mention of the incontinent and continent person (\textbf{READ 1111b14-16}); we get a full discussion of such people; but it is clear that A takes these to be character types, and so, in line with our discussion of situationism, it would seem that at least A thinks that not only virtuous people have stable character traits
\vspace*{2mm}

\noindent Nor is decision wish, but it's close to it; we wish for the impossible but we can't say we decide on impossible things; but we wish for them
\vspace*{2mm}

\noindent We also wish for results for our friends, those things we can't bring about by our own agency; but we don't decide on them
\vspace*{2mm}

\noindent Rather, we wish for the ends, but decide on things that promote the ends (\textbf{DOES THIS SUGGEST ANYTHING ABOUT KANT V. HUME?})
\vspace*{2mm}

\noindent\textbf{READ 1112a14-end}
\vspace*{2mm}

\noindent Hanging questions: (1) Is decision universal or particular? (2) When does the `previous deliberation' have to have taken place

\section*{3.3}

\noindent\textbf{READ 1112A19}: Does this again suggest anything with Kant v. Hume?
\vspace*{2mm}

\noindent In general, we do not deliberate about results that cannot be achieved through our own agency: e.g. the incommensurability of the sides of the diagonal; the solstices, rising of the stars etc.; nothing that always comes about the same way, either from necessity or nature
\vspace*{2mm}

\noindent\textbf{NB:} But, this does not mean that we do not take such things into account in our deliberation; that is, the question `what do we deliberate about'? $\neq$ `what factors into our deliberation'?
\vspace*{2mm}

\noindent Rather, we deliberate about `what is up to us' (epi hemin)
\vspace*{2mm}

\noindent\textbf{READ 1112b12-16}
\vspace*{2mm}

\noindent The last thing found in the analysis (i.e. deliberation?) is the first that comes into being (DOES THIS REQUIRE DECISION TO BE PARTICULAR?)
\vspace*{2mm}

\noindent We do not deliberate about particulars, questions such as `what is this'? aren't deliberated about
\vspace*{2mm}

\noindent\textbf{READ 1113a}: Again, general or particular?
\vspace*{2mm}

\noindent\textbf{READ 1113a10}

\section*{3.4}

\noindent Aristotle has maintained that wish is for the end; now he takes up a question that dogged a lot of ancient thinkers: do we only wish for the good (or can we also wish for the apparent good)
\vspace*{2mm}

\noindent\textbf{READ 1113a22}: I think that we are supposed to read `wished by nature' to mean `by nature worthy of being wished'; similar to choice worthy (I would take a statement like `chosen by nature' to be, but nature such as to be chosen)
\vspace*{2mm}

\noindent Aristotle says that, in reality and without qualification, what is wish for is the good; but for each person on each occasion, what is wished for is what appears to them to be good; for the good person, the two will line up
\vspace*{2mm}

\noindent\textbf{READ 1113a33}: Don't take `measure or standard' to be metaphysical

\section*{3.5}

\noindent\textbf{READ a lot of the beginning of the chapter}
\vspace*{2mm}

\noindent Aristotle takes supporting evidence from the fact that legislators 
\vspace*{2mm}

\noindent\textbf{READ 1113b30}: Here we do get an indication that Aristotle is concerned with something we have wondered; what about being ignorant but being responsible for the ignorance
\vspace*{2mm}

\noindent So, A is basically here trying to argue that we are responsible for our characters; and the fact that we have a character cannot get us off the hook for the actions we do
\vspace*{2mm}

\noindent And this is because A thinks that we acquire our characters through habituation, but that just means repeated particular actions, and in each case, what we do is up to us, so the character we acquire is up to us
\vspace*{2mm}

\noindent This is incredibly difficult territory: look at \textbf{TEXT 1114a12-13}; But, do people know what actions will make them what way? presumably they have some info, but enough?
\vspace*{2mm}

\noindent A major objection comes at \textbf{1114b}
\vspace*{2mm}

\noindent\textbf{READ 1114b15}: What is Aristotle saying here?

\end{document}

