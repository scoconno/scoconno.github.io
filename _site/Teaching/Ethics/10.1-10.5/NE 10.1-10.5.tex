% !TEX encoding = UTF-8 Unicode
% !TEX TS-program = xelatex

\documentclass[11pt]{article}
\usepackage{polyglossia}
\usepackage{amssymb}
\usepackage{fullpage}
\usepackage{hyperref}
\setdefaultlanguage{english}
\setotherlanguage{greek}
\newfontfamily\greekfont{Gentium Plus}
\newcommand{\gk}[1]{\textgreek{#1}}

\title{\emph{Nicomachean Ethics} 10.1-10.5}
\author{}
\date{}

\begin{document}

\maketitle

\section{10.1}

\noindent `Pleasure' = \emph{h\^{e}don\^{e}}; incredibly difficult to define or give an account of
\vspace*{2mm}

\noindent Plato thought it had something to do with a return to an equilibrium state; or, the awareness of such a return occurring
\vspace*{2mm}

\noindent A identifies two views concerning pleasure, each of which he ultimately rejects

\begin{itemize}\item{Pleasure is \emph{the} good}\begin{itemize}\item{This is an extreme hedonist view; pleasure is not just \emph{a} good, but \emph{the only} good}\end{itemize}\item{Pleasure is not a good at all}\begin{itemize}\item{He notes some people advocate this view as a matter of practical advice, in line with \emph{NE} 2.9}\item{People typically tends toward indulging pleasure, so to bring them to an intermediate state such theorists think it's best to treat pleasure as if it were thoroughly bad}\end{itemize}\end{itemize}

\section{10.2}

\noindent An argument that pleasure is \emph{the} good (attributed to Eudoxus)

\begin{itemize}\item{[P1] Every creature pursues pleasure}\item{[P2] That which all creatures aim at is the good}\item{[C] Pleasure is the good}\end{itemize}

\noindent Interestingly: A thinks this argument was accepted because of the character of Eudoxus and not on its own merits
\vspace*{2mm}

\noindent Since pain is something to be avoided for all, pleasure, its opposite, is something choice worthy for all
\vspace*{2mm}

\noindent \textbf{(READ 1172b20ff} Another argument:

\begin{itemize}\item{[P1] That which is never chosen for the sake of anything else is most choice worthy}\item{[P2] Please is never chosen for the sake of anything else}\item{[C] Pleasure is most choice worthy}\end{itemize}

\noindent Yet another argument:

\begin{itemize}\item{[P1] Whenever pleasure is added to a good, it makes that good more choice worthy}\item{[P2] Only the addition of a good can make something more choice worthy}\item{[C] Pleasure is (a) good}\end{itemize}

\noindent Plato actually used the above as an argument that pleasure is not \emph{the} good; since a life of pleasure is made better if it involves \emph{phron\^{e}sis}; and you can't make \emph{the} good better by adding something distinct from its
\vspace*{2mm}

\noindent Argument against extreme anti-hedonism

\begin{itemize}\item{[P1] Pleasure seems good to all animals}\begin{itemize}\item{Presumably supported by their behavior}\end{itemize}\item{[P2] What seems good to all animals is a good}\item{[C] Pleasure is a good}\end{itemize}

\section{10.3}

\noindent Some people think that the good must be a quality; since we say things like `good' person (similar to red tree, tall banana, long day); but, `good' is different; a good person is a person with certain qualities; but the good for a person isn't a quality
\vspace*{2mm}

\noindent \textbf{READ 1172a20}: Here we get A committing to the idea that virtue comes in degrees: people are more or less virtuous
\vspace*{2mm}

\noindent Pleasure is not a process: processes can be performed slowly/quickly; one can become pleased quickly, but they can't be pleased quickly or slowly
\vspace*{2mm}

\noindent What about pleasure from bad sources
\vspace*{2mm}

\noindent One option: say that they are not pleasant except for people in bad conditions
\vspace*{2mm}

\noindent \textbf{READ1173b25} Another option: such pleasures are good, but are outweighed by other factors and so should not be chosen; so, \textbf{PLEASURE IS WITHOUT QUALIFICATION GOOD; BUT, IN CERTAIN CIRCUMSTANCES IS NOT CHOICEWORTHY}
\vspace*{2mm}

\noindent Third option: find some way to differentiate between the pleasures themselves such that pleasures from good sources differ in kind from pleasures from bad sources and only the former are good
\vspace*{2mm}

\noindent \textbf{READ 1174a5}: There's a common mistake, still made today, that if a person derives pleasure from \emph{X} (perhaps even necessarily), then that entails he or she X-ed for the sake of the pleasure; FALSE; counterfactual test (but, even if the counterfactual test fails, we still can't automatically conclude that they did it \emph{only} for the pleasure)

\section{10.4}

\noindent So, what is pleasure?
\vspace*{2mm}

\noindent Well, it isn't a process; it is more like seeing; it isn't like there is some end-state that has to be reached in order to say that someone has engaged in seeing; similarly for experience pleasure
\vspace*{2mm}

\noindent Contrast with house-building: that is an incomplete process; it is only completed when the house is built
\vspace*{2mm}

\noindent If I am enjoying a conversation, for example, I do not need to wait until it is finished in order to feel pleased; I take pleasure in the activity all along the way.
\vspace*{2mm}

\noindent Although pleasure is an activity, it is unique in being an activity we can only engage in when we engage in some other activity: in particular, some activity of perception or thought; NB, perception here is a wide activity; A thinks that if S perceives that P, S perceives that S perceives that P; this is basically the defining feature of pleasure; what distinguishes pleasure from all other activities is that it accompanies other activities and, in some way, completes or perfects them
\vspace*{2mm}

\noindent It is not enough to say that it is what happens when we are in good condition and are active in unimpeded circumstances; one must add to that point the further idea that pleasure plays a certain role in complementing something other than itself.
\vspace*{2mm}

\noindent\textbf{READ 1174b32}: Pleasure completes the activity, can also be translated as `perfects'; note that this does not mean that perception or understanding are, themselves, incomplete or imperfect in the way that house-building is; but that, without pleasure, there is something lacking in the experience of the agent
\vspace*{2mm}

\noindent When he says that pleasure completes an activity by supervening on it, like the bloom that accompanies those who have achieved the highest point of physical beauty, his point is that the activity complemented by pleasure is already perfect, and the pleasure that accompanies it is a bonus that serves no further purpose.
\vspace*{2mm}

\noindent Pleasure is that which fills that gap

\section{10.5}

\noindent Since pleasure is what completes an activity of thought or perception, pleasures will differ in species insofar as the activities they complete differ in species: so, pleasures of thought and perception differ in kind
\vspace*{2mm}

\noindent The pleasure associated with an activity makes us engage in the activity better; pleasures can impede on other activities: eating nuts in the theater; you look for other stimulation when you are under-stimulated
\vspace*{2mm}

\noindent So, given that pleasures can impede one another, even if they were all good we still need to choose between them
\vspace*{2mm}

\noindent\textbf{READ 1175b25}: A thinks that the value of a pleasure is wholly derived from the value of the activity it completes; Importantly, then, when we are considering what pleasures to pursue, we don't ask about the pleasures themselves, but the activities that they complete; all the decision making goes at the level of the activity
\vspace*{2mm}

\noindent\textbf{NB}: This requires that pleasure is not \emph{the} good; if it were, then we would just look to the pleasures themselves
\vspace*{2mm}

\noindent A seems to opt for the first choice laid out w/r/t bad pleasures: they are only pleasant to people in corrupt conditions; thus they aren't indications of what we should take pleasures in
\vspace*{2mm}

\noindent\textbf{READ 1176a25} Segway to highest good

\end{document}

