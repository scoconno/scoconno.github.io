\documentclass[article,oneside]{memoir}

%%% custom style file with standard settings for xelatex and biblatex. Note that when [minion] is present, we assume you have minion pro installed for use with pdflatex.
%\usepackage[minion]{org-preamble-pdflatex} 

%%% alternatively, use xelatex instead
\usepackage{org-preamble-xelatex} 



\def\myauthor{Author}
\def\mytitle{Title}
\def\mycopyright{\myauthor}
\def\mykeywords{}
\def\mybibliostyle{plain}
\def\mybibliocommand{}
\def\mysubtitle{}
\def\myaffiliation{NJCU}
\def\myaddress{Phil 207}
\def\myemail{soconnor@njcu.edu}
\def\myweb{\href{http://scoconno.github.io/Teaching/Ethics/}{http://scoconno.github.io/Teaching/Ethics/}}
\def\myphone{}
\def\myversion{}
\def\myrevision{}
\def\myaffiliation{NJCU}
\def\myauthor{Dr. Scott O'Connor}
\def\mykeywords{}
\def\mysubtitle{Syllabus}
\def\mytitle{{\normalsize \myweb \newline} \HUGE Ethics: Happiness, Friends, and the Good Life}


\begin{document}

%%% If using xelatex and not pdflatex
%%% xelatex font choices
\defaultfontfeatures{}
\defaultfontfeatures{Scale=MatchLowercase}    
% You will need to buy these fonts, change the names to fonts you own, or comment out if not using xelatex.      
\setromanfont[Mapping=tex-text]{Minion Pro} 
\setsansfont[Mapping=tex-text]{Myriad Pro} 
\setmonofont[Mapping=tex-text,Scale=0.8]{Georgia} 

%% blank label items; hanging bibs for text
%% Custom hanging indent for vita items
\def\ind{\hangindent=1 true cm\hangafter=1 \noindent}
\def\labelitemi{$\cdot$}
%\renewcommand{\labelitemii}{~}

%% RCS info string for version tracking
\chapterstyle{article-3}  % alternative styles are defined in latex-custom-kjh/needs-memoir/
%\pagestyle{kjh}

\title{\LARGE \mytitle}     
\author{\Large\myauthor \newline \footnotesize\texttt{\noindent Office hours: \href{http://scoconno.github.io/Contact/Office/}{http://scoconno.github.io/Contact/Office/}}}
\date{09/06/2016--12/19/2016}

\published{\textbf{Phil 207 (2804), 3 credits, Fall 2016, H220, M\&W 4:00pm--5:15pm}}

\maketitle

%\thispagestyle{kjhgit}

% Copyright Page
%\textcopyright{} \mycopyright


%
% Main Content
%

\section{Copyright}
The materials used in this class, including, but not limited to, lectures, exams, quizzes, and homework assignments are copyright protected works.  Any unauthorized copying of the class materials or recording of lectures is a violation of federal law and may result in disciplinary actions being taken against the student.  Additionally, the sharing of class materials without the specific, express approval of the instructor may be a violation of the University's Student Honor Code and an act of academic dishonesty, which could result in further disciplinary action.  This includes, among other things, uploading class materials to websites for the purpose of sharing those materials with other current or future students. 

\section{Catalog Description}

This course teaches students to identify and evaluate those beliefs that guide their thoughts and actions. Reflecting on different sources, students identify those philosophical beliefs that play a role in their own lives. By developing their critical thinking skills, they learn how to clarify, systematize, and assess these beliefs. 

\section{Course Description}

Aristotle's ethical theory has exerted a profound infuence on the history of Western Philosophy.  In this course we will aim to gain a comprehensive understanding of his ethics, focusing on its presentation in the \emph{Nicomachean Ethics}.  Central notions we will examine include \emph{eudaimonia} (happiness or human  flourishing), \emph{arete} (virtue or excellence), \emph{phronesis} (practical or moral wisdom), akrasia (lack  of  self-control), philia (friendship),  and  (pleasure).   

pleasure, friendship, moral failure, responsibility 


Happiness. 
What is it? The relationship between ethical life and happiness. What is pleasure and how does it fit in with both a happy life and ethical life? What is friendship? What distinguishes true friend from a fake friend? Answer to this too. Do you need friends to be happy. 

\section{Learning Objectives}

Upon completing this course, students will be able to (i) read
philosophical texts, (ii) clearly and charitably explain viewpoints that
are not their own, (iii) think critically and philosophically, (iv)
write well-structured prose in which they clearly state a thesis and
persuasively defend it, (v) demonstrate an understanding of several core
philosophical topics, (vi) manage their studies in a responsible and timely manner. 




\section{Required Textbook}

Both texts are available in the campus book store and online retailers.

\begin{itemize}
\item
\href{https://www.amazon.com/Nicomachean-Ethics-Hackett-Classics-Aristotle/dp/1624661173/ref=sr_1_16?ie=UTF8&qid=1473024493&sr=8-16&keywords=aristotle%27s+nicomachean+ethics}{'Nicomachean Ethics', Aristotle, \emph{trans} C.D.C. Reeve, Hackett Classics}, (\textbf{needed for every class}

  \href{https://www.amazon.com/Aristotles-Nicomachean-Ethics-Introductions-Philosophical/dp/0521520681/ref=sr_1_1?ie=UTF8&qid=1473039378&sr=8-1&keywords=nicomachean+ethics+pakaluk}{Aristotle's Nicomachean Ethics: An Introduction, Michael Pakaluk} 
\end{itemize}


\section{Course Website}
There is both a Blackboard site and website for this course (link on first page). Clicking the first link on the left panel within the Blackboard site will bring you to the course website. All assignments will be submitted through Blackboard. Readings, notes, etc. will be posted on the course website. Note that Blackboard difficulties are rare and automatically reported to instructors. Under no circumstance will a student's report of a Blackboard difficulty be reason for an extension. It is your responsibility to contact Blackboard support for help.




\section{Requirements}


\begin{itemize}
\item \textit{Workload:} Expect to spend an average of 6 hours per week completing the readings and assignments. NJCU abides by the Federal and State definitions of a credit hour and adopts a policy consistent with the Carnegie Unit. A three-credit class represents 112.5 hours total of work. See \href{http://scoconno.github.io/Teaching/Credit.pdf}{here} for more details.

\item \textit{Attendance:} Roll call will be taken. 0.5 point will be awarded per class up to a maximum of 10 points. Points will not be awarded during weeks 1 \& 2. 


\item \textit{Essays (400--600 words)} submitted through Blackboard.  14 will be assigned. You must complete 9. If you complete more than 9, the lowest grades will be dropped. 


\item \textit{Course evaluations} completed online. 5 points extra credit for successful completion.
  
\item \textit{Grade distribution:} Attendance--0.5 points per class (10 maximum);  Essays---10 points each (90 total)

\item \textit{Grade Breakdown:}

 \begin{tabular}{ | l | l | p{2cm} | l | l | }
    \hline 
96--100 & A  & &  77--79 &  C+ \\  
90--95 & A- & &  73--76 & C \\
87-89 & B+ &  &  70--72 & C- \\ 
83--86 & B  & &  60--69 & D\\
80--82 & B - & & 0--59 & F\\ \hline
    \end{tabular}


\end{itemize}


\section{Policies}

\begin{itemize}

\item \textbf{Student Responsibility:} This syllabus outlines the required text, assignments, requirements, and policies for this course. By taking this course, you agree to read this syllabus and be bound by those requirements and policies. 

 \item \textit{Academic Integrity:} All the work you turn in (including papers, drafts, and discussion board posts) must be written by you specifically for this course. It must originate with you in form and content with all contributory sources fully and specifically acknowledged. Being a student at NJCU requires you to follow \href{http://scoconno.github.io/Teaching/Plagiarism.pdf}{NJCU's Academic Integrity Policy.} Penalties for violations are as follows: 1st infraction will result in a 0 for the assignment.  2nd infraction will result in a 0 for the entire course \& application for permanent record on student's transcript. (Repeated violations can lead to expulsion from NJCU). 


\item \textit{Attendance:} You are considered absent if you are (i) not present during roll call, (ii) leave early, (iii) leave without permission, or (iv) leave for an extended period of time. No excuses. No exceptions.



\item \textit{Communication:} To comply with Federal Privacy Laws (FERPA) and NJCU policies, all communication will be through Blackboard and/or official NJCU e-mail. Check both your NJCU e-mail and Blackboard daily. For further information see \href{http://scoconno.github.io/Contact/}{http://scoconno.github.io/Contact/}.

\item \textit{Conduct:} Distracting and disrespectful behaviors, including but not limited to eating, leaving your seat, talking out of turn, and aggression are prohibited. Penalties include, but are not limited to, a loss of attendance points for the day of violation. Repeat offenders will be reported to the Dean of Students. 

\item \textit{Electronic devices:} Use of electronic device, including, but not limited, to smartphones, dictaphones, tablets, and laptops, is prohibited. Recording a lecture is in violation of Copyright. Penalties include, but are not limited to, a loss of attendance points for the day of violation. Repeat offenders will be reported to the Dean of Students.


\item \textit{Format for Written Work:} Submit work to Blackboard either as a rich text or Microsoft Word file. All work must be typed. Your papers should be in 12-point Times New Roman font, double-spaced with margins set to one inch on all sides. If hard copies are requested, please staple or paperclip copies of papers and drafts.

\item \textit{Grading:} Grades will be available within 1--2 weeks of an assignment being submitted. See: \href{http://scoconno.github.io/Teaching/Grading}{http://scoconno.github.io/Teaching/Grading} for further information.


\item \textit{Late work \& Make-up Policy:} See the assignment schedule below. No make-ups or late work accepted under any imaginable circumstances. No exceptions. This policy will be applied equally and fairly to all students. \emph{Requests for special treatment will be ignored.}

\item \textit{Statement for students with disabilities:} If you are a student with a disability and wish to receive consideration for reasonable accommodations, please register with the Office of Specialized Services and Supplemental Instruction (OSS/SI). To begin this process, complete the registration form available on the OSS/SI website at \href{http://www.njcu.edu/oss}{http://www.njcu.edu/oss} (listed under Student Resources-Forms). Contact OSS/SI at 201-200-2091 or visit the office in Karnoutsos Hall, Room 102 for additional information.

\item \textit{Turnitin:} Students agree that by taking this course all assignments are subject to submission for textual similarity review to Turnitin.com. Assignments submitted to Turnitin.com will be included as source documents in Turnitin.com's restricted access database solely for the purpose of detecting plagiarism in such documents.  The terms that apply to the University’s use of the Turnitin.com service are described on the Turnitin.com web site.  For further information about Turnitin, please visit: http://www.turnitin.com 


\end{itemize}



\section{Weekly Course Schedule}
Dates refer to the first day of the week. Complete the readings before the first day of that week.  Although assigned sections are often short, they are difficult, and multiple readings are strongly advised.  All numbers refer to Book and Chapter numbers in Aristotle's \emph{Nicomachean Ethics}, e.g.  `1.1--1.6' means \emph{Nicomachean Ethics} Book 1, Chapter One through Chapter Six inclusive.. Changes to the syllabus will be announced through Blackboard and \emph{via} your NJCU email address.  \newline




\begin{description}

\item[Week 1:] Introduction  (\emph{9/7/2016})

\item[Week 2:]  Basic Overview  (\emph{9/12/2016})
\begin{enumerate}
\item Aristotle's life
\item 1.1--1.6
\end{enumerate}

\item[Week 3:]  The Goal of Human Life (\emph{9/19/2016})
\begin{enumerate}
\item 1.7--1.13
\end{enumerate}

\item[Week 4: ] Virtue of Character (\emph{9/26/2016})
\begin{enumerate}
\item 2.1--2.9
\end{enumerate}


\item[Week 5: ] Mind and action (\emph{10/3/2016})
\begin{enumerate}
\item 3.1--3.5
 \end{enumerate}
 
\item[Week 6:] Particular virtues (\emph{10/10/2016})
\begin{enumerate}
\item 3.6--3.12 (courage and temperance)
\end{enumerate}

\item[Week 7:] Particular virtues (\emph{10/17/2016})
\begin{enumerate}
\item Book 4
\end{enumerate}


\item[Week 8:] Justice  (\emph{10/24/2016})
\begin{enumerate}
\item Book 5
\end{enumerate}


\item[Week 9:] Virtues of thought (\emph{10/31/2016})
\begin{enumerate}
\item Book 6
\end{enumerate}

\item[Week 10:] Lack of control  (\emph{11/7/2016})
\begin{enumerate}
\item 7.1--7.10
\end{enumerate}


\item[Week 11:] Friendship (\emph{11/14/2016})
\begin{enumerate}
\item Book 8
\end{enumerate}


\item[Week 12:] Friendship  (\emph{11/21/2016})
\begin{enumerate}
\item Book 9
\end{enumerate}


\item[Week 13:]  Pleasure (\emph{11/28/2016})
\begin{enumerate}
\item  7.11--14
\item 10.1--10.5
\end{enumerate}

\item[Week 14:]  Happiness (\emph{12/5/2016})
\begin{enumerate}
\item 10.6--10.9

\end{enumerate}

\item[Week 15:] Conclusion (\emph{12/12/2016})
\begin{enumerate}
\item To be determined. 
\end{enumerate}



\end{description}





\section{ Assignment Schedule}
Dates refer to the due date. All assignments must be submitted through Blackboard by 11:59pm. No late work accepted. No exceptions. You must complete 5 quizzes, 3 short essays, and the final independent project. If you complete more than the required number, the lowest grades will be dropped.
\begin{itemize}
\item \textit{9/12/2016,} Essay 
\item \textit{9/19/2016,} Quiz CT 
\item \textit{9/26/2016,} Quiz MOL
\item \textit{10/03/2016,} Essay MOL
\item \textit{10/10/2016,} Essay God 
\item \textit{10/17/2016,} Quiz GOD
\item \textit{10/24/2016,}  Essay GOD
\item \textit{10/31/2016,} Quiz GOD
\item \textit{11/7/2016,}  Essay FW
\item \textit{11/14/2016,} Quiz CR
\item \textit{11/21/2016,} Quiz Conseq
\item \textit{11/28/2016,} Essay walking dead
\item \textit{12/05/2016,} Projects 
\item \textit{12/12/2016,} Project 
\end{itemize}




%% Uncomment if you want a printed bibliography.
%\printbibliography 

\end{document}