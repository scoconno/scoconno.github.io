% !TEX encoding = UTF-8 Unicode
% !TEX TS-program = xelatex

\documentclass[11pt]{article}
\usepackage{amssymb}
\usepackage{fullpage}
\usepackage{hyperref}

\title{\emph{Nicomachean Ethics} 2.5-2.9}
\author{}
\date{}

\begin{document}

\maketitle

\noindent\underline{2.5}
\vspace*{2mm}

\noindent One of Aristotle's methods of division is via \emph{genus et differentiam}
\vspace*{2mm}

\noindent Since virtue is clearly a \emph{something} of the soul, we need to determine what kind of something it is; three options:

\begin{itemize}\item{Feelings, affections (\emph{path\^{e}}; noun connected to verb `to undergo' or `to suffer')}\begin{itemize}\item{E.g.: appetite, anger, fear, confidence, envy, joy, love, hate, longing, jealousy, pity, whatever implies pleasure or pain}\item{So, some things that are going to count as \emph{path\^{e}} are sense-perceptions (e.g. a visual or auditory impression); but, presumably those aren't relevant to the question of what virtue is)}\end{itemize}

\item{Capacities, powers (\emph{dunameis}; noun connected to verb `to be able')}\begin{itemize}\item{That in virtue of which we are capable of having feelings: e.g. the capacity to feel anger, be afraid and so on}\item{So, also here are going to be things like sight, hearing, and so on}\end{itemize}

\item{States, dispositions (\emph{hexeis}; connected to verb `to have', `to hold')}\begin{itemize}\item{`What we have when we are well or badly off in relation to feelings'}\item{This is a little confusing, but the idea seems to be, irritability is a state or disposition of our capacity or power to feel anger in that being irritable means that you are prone to feeling anger too much}\end{itemize}
\end{itemize}

\noindent Virtues and vices are states or dispositions: argument by elimination

\begin{itemize}\item{Not Feelings}\begin{itemize}\item{We don't say that someone is a good or bad person in virtue of the feelings they have}\item{Similarly, we don't praise or blame someone just in virtue of having a feeling; we are interested in why they have it, how it came about, the way they have it and so on}\begin{itemize}\item{That might sound wrong; if someone, say, longs for a young child in a sexual way, we are inclined to blame the person without mitigating factors; but, to some degree we do think that just the feeling itself isn't under someone's control}\end{itemize}\item{We don't decide to feel a certain way; but virtues and vices either are or require decisions}\end{itemize}\item{Not capacities or powers}\begin{itemize}\item{We aren't good or bad, praised or blamed, insofar as we are simply capable or able to feel things}\end{itemize}\end{itemize}

\noindent\underline{2.6}
\vspace*{2mm}

\noindent So, the genus of virtue is state or disposition of the soul; what is it's differentiae (i.e. what makes it different from the other states or dispositions of the soul)?
\vspace*{2mm}

\noindent Well, the conceptual truth about virtue is that virtue \emph{causes} its possessor to be in a good state and perform their \emph{erga} well; so, the virtue of a human being is that state or disposition that makes a human being perform his \emph{erga} well
\vspace*{2mm}

\noindent Here we get one of the more famous, and confusing, elements of Aristotle's \emph{Ethics}: the `mean relative to us'
\vspace*{2mm}

\noindent\textbf{READ 1105A27-1106B7}
\vspace*{2mm}

\noindent So, he clearly wants to distinguish:

\begin{itemize}\item{The mean in the object}\begin{itemize}\item{This seems to be an arithmetical notion; what seems to be distinctive of it is that \emph{all} you take into consideration is the extremes and average them}\end{itemize}\item{The mean relative to us}\begin{itemize}\item{This seems to differ insofar as it takes into consideration facts about the agent}\end{itemize}\end{itemize}

\noindent Some issues

\begin{itemize}\item{Who is the `us' to whom this mean is supposed to be relative?}\begin{itemize}\item{Human beings?}\item{Philosophy students?}\end{itemize}\item{Is there anything concrete to this notion?}\end{itemize}

\noindent So, virtue aims at what is intermediate

\begin{itemize}\item{VOC is a matter of feelings and actions; and F\&A admit of excess, deficiency, and intermediates}\begin{itemize}\item{E.g. we be afraid, confident, get angry, feel pity, have appetites, pleasures, and pains too much and too little}\item{And, being disposed to have such things too much or too little is a bad thing}\end{itemize}\item{So, having such feelings: at right time, about right things, toward right people, for right end, in right way is intermediate}\end{itemize}

\noindent\textbf{READ 1106b28-29}: remember what this means, virtue is a disposition of the capacity to feel; what dispositions, the disposition to feel in the ways just adumbrated (right time, etc.)
\vspace*{2mm}

\noindent\textbf{READ 1107a1-4}
\vspace*{2mm}

\noindent So, while virtue is a mean state, insofar as it is the best state to be in it is an extreme (i.e. is the \emph{best}); this suggests Aristotle is open to allowing for degrees
\vspace*{2mm}

\noindent Some actions, when appropriately described, do not admit of a mean
\vspace*{2mm}

\noindent Distinguish thick vs. thin (descriptive vs. evaluative)

\begin{itemize}\item{Feelings: spite, shamelessness, envy}\item{Actions: adultery, theft, murder}\end{itemize}

\noindent This is why he pointed out that, in a sense, virtues are extremes, because extremes don't have excesses or deficiencies
\vspace*{4mm}

\noindent\underline{2.7}
\vspace*{2mm}

\noindent Need to apply it to particular cases (reference to `this chart' suggests that these are lecture notes
\vspace*{2mm}

\noindent\underline{Range of object\hspace*{10mm}Kind of emotion\hspace*{10mm}Excess\hspace*{17mm}Mean\hspace*{18mm}Deficiency}
\vspace*{1mm}

\noindent Danger\hspace*{22mm}Fear/Confidence\hspace*{10mm}Rash\hspace*{16mm}Courageous\hspace*{12mm}Coward
\vspace*{1mm}

\noindent Physical goods\hspace*{11mm}Lust etc.\hspace*{20mm}Intemperate\hspace*{8mm}Temperate\hspace*{13mm}Insensible\\\hspace*{2mm}(e.g. food, drink, sex)
\vspace*{1mm}

\noindent Wealth\hspace*{22mm}``Heart''\hspace*{22mm}Wasteful\hspace*{14mm}Generous\hspace*{12mm}Ungenerous
\vspace*{1mm}

\noindent Status\hspace*{24mm}Spirit\hspace*{27mm}Vain\hspace*{16mm}Magnanimous\hspace*{7mm}Pusillanimous
\vspace*{1mm}

\noindent Respect\hspace*{21mm}Temper (anger)\hspace*{11mm}Irascible\hspace*{16mm}Mild\hspace*{21mm}Cold
\vspace*{1mm}

\noindent Social intercourse\hspace*{6mm}Sensibility\hspace*{18mm}Buffoon\hspace*{16mm}Witty\hspace*{21mm}Boor
\vspace*{4mm}

\noindent\textbf{READ 1107b34}: This is why A needs to claim that the trio applies in all cases; since there is not a common word for the person who is in an intermediate state w/r/t honor, the people at the extreme claim to be virtuous; but, we can see that they are at an extreme and, thus, not virtuous
\vspace*{4mm}

\noindent\underline{2.8}
\vspace*{2mm}

\noindent Reiterates that mean is, in a sense, an extreme in comparison to excess and deficiency
\vspace*{2mm}

\noindent And, in some cases, one of the extremes seems more like the intermediate

\begin{itemize}\item{Rashness is more like courage than cowardice is like courage}\item{Asceticism is more like temperance than gluttony}\end{itemize}

\noindent One interesting reason for this is that the extreme toward which people `naturally drift' seems more opposed to the intermediate case (e.g. intemperance/gluttony is where people naturally drift, so it seems more opposed to temperance than asceticism does)
\vspace*{4mm}

\noindent\underline{2.9}
\vspace*{2mm}

\noindent It is hard work to be excellent, since there is only one midpoint, and everything else falls short
\vspace*{2mm}

\noindent\textbf{READ 1109a31}: Something like advice; in habituating, it's best to tend towards the contrary extreme
\vspace*{2mm}

\noindent Take stock of yourself; really attend to the pleasure and pain you experience in doing actions and feeling certain ways; for those where you tend not to hit the mean, drag yourself in the other direction, `like in straightening bent wood'
\vspace*{2mm}

\noindent And, above all, \textbf{BEWARE OF PLEASURE}
\vspace*{2mm}



\end{document}

