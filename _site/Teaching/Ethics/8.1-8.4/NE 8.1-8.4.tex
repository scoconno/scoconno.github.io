% !TEX encoding = UTF-8 Unicode
% !TEX TS-program = xelatex

\documentclass[11pt]{article}
\usepackage{polyglossia}
\usepackage{amssymb}
\usepackage{fullpage}
\usepackage{hyperref}
\setdefaultlanguage{english}
\setotherlanguage{greek}
\newfontfamily\greekfont{Gentium Plus}
\newcommand{\gk}[1]{\textgreek{#1}}

\title{\emph{Nicomachean Ethics} 8.1-8.4}
\author{}
\date{}

\begin{document}

\maketitle

\section*{8.1}

\noindent The word translated here as `friendship' is `philia'; this word is connected to the verb `philein'=to love; there is no English verb correlate with `friendship'
\vspace*{2mm}

\noindent It covers a broad range of interpersonal relationships (and, possibly, intrapersonal as well); it was viewed as, in some way, one of the bedrocks of civilized communal living; certainly, the case he is most interested counts as a kind of friendship
\vspace*{2mm}

\noindent It includes the `natural' attitude family members have for each other; but also the attitude that you have for your butcher, baker, candlestick maker; it definitely covers what we now consider friendship, but seems to go beyond it
\vspace*{2mm}

\noindent Also includes relationship between members of other species of animals
\vspace*{2mm}

\noindent So, one difference seems to be that `friendship' is often taken to be a voluntary thing, but `philia' isn't always
\vspace*{2mm}

\noindent\textbf{READ 1155a5-12}

\begin{itemize}\item{Friends seem necessary for a full life}\item{The fact that no one would choose to live w/o friends even if they had all the other goods shows that friendship is not \emph{merely} instrumental}\item{But A seems to have somewhat of an odd take on it---without friends, great wealth isn't as awesome as it is with friends, because beneficence is best displayed towards friends}\end{itemize}

\noindent Friends help us in basically every way: figure out what to do, accomplish more, understand better etc.
\vspace*{2mm}

\noindent Puzzles about friendship

\begin{itemize}\item{Is friendship just a matter of similarity?}\item{Does it require similarity}\item{Or, does it exist mainly between people who are different from each other?}\item{Among what kind of people does friendship arise?}\end{itemize}

\section*{8.2}

\noindent At bottom, we love good, pleasant, and useful things 
\vspace*{2mm}

\noindent\textbf{READ 1155b28ff} A thinks: we don't have friendship for inanimate things (taken as obvious) (though we might love them); the reason why must be because there is no reciprocation and no wishing good for it
\vspace*{2mm}

\noindent Good will is wishing good things to happen for X, for X's own sake
\vspace*{2mm}

\noindent\textbf{WHAT IS IT TO WISH SOMETHING TO HAPPEN TO SOMEONE FOR THEIR OWN SAKE}
\vspace*{2mm}

\noindent \textbf{READ 1155b35: NB, HERE A SUGGESTS THAT SOMEONE CAN HAVE GOODWILL TO SOMEONE HE SUPPOSES TO BE DECENT OR USEFUL}

\noindent Good will towards X is still not enough for friendship with X; must be reciprocated

\noindent At its core, Aristotle thinks that friendship is reciprocated goodwill, of which both parties are aware (i.e. if I have goodwill to someone; and that person happens to have goodwill to me, that is not enough to say we are friends)
\vspace*{2mm}

\noindent \textbf{NB: LAST LINE, `FROM ONE OF THE CAUSES MENTIONED ABOVE' THIS COULD MEAN, ANY ONE OF THE THREE, OR, OF THE THREE, ONE}

\section*{8.3}

\noindent Corresponding to the three objects of love are three species of friendship, each of which has one of the objects: 
\vspace*{2mm}

\noindent Big question

\begin{itemize}\item{In what sense are each of these \emph{species} of friendship}\item{at times he speaks as if only the relationship that exists between two virtuous people counts as genuine friendship; the other kinds bearing the title due to similarity; \textbf{READ FIRST LINE OF 8.1}}\item{On the other hand, that would suggest they aren't actually species of friendship}\end{itemize}

\noindent Corresponding to utility and pleasure are two kinds of friendship
\vspace*{2mm}

\noindent It seems here that, if the basis of your love is the useful thing you get out of it, then you don't really wish good things for that person \emph{for their own sake} because of what they are
\vspace*{2mm}

\noindent Similarly, if the basis of your friendship is the pleasure you derive from them, you don't really love the person for \emph{who} he or she is
\vspace*{2mm}

\noindent \textbf{READ 11156a15} So, one question is how these fit under the definition of friendship at all: it seems that A thinks, you do with good things for the other person, not so that they can produce pleasure in you (that would not be for their own sake), but it's the fact that they produce pleasure in you that causes you to wish good for them for their own sake; this is the sense in which it is coincidental that they wish good for the friend for the friend's own sake
\vspace*{2mm}

\noindent These kinds of friendships are easy to form and dissolve, because what you find pleasant, or what is useful to you, can shift quickly; and so, if the basis of your love for X was that X gave you Y, then when X no longer gives you Y, you will cease to love X
\vspace*{2mm}

\noindent Utility friendships are the most common friendships formed in old age, since they don't really take as much pleasure in things, and need a lot of help; also common among young go-getters
\vspace*{2mm}

\noindent Pleasure friendships are the most common among young people, because they tend to hang out with people who produce pleasure in them when they hang out
\vspace*{2mm}

\noindent Complete friendship is friendship between good people similar in virtue; here the basis for your wishing good for that person is that, given who that person is in and of him or herself, you think they are deserving of good things
\vspace*{2mm}

\noindent This endures because the basis of the friendship is the presence of certain qualities which are stable and enduring\vspace*{2mm}

\noindent The notion of similar in virtue must mean something like, at the same point in their moral development; between a budding virtuous person and a full virtuous person there can be mutual recognition of the other as good, but it will still be imperfect

\section*{8.4}

\noindent\textbf{RETURNING TO THE QUESTION OF WHETHER UTILITY AND PLEASURE FRIENDSHIPS ARE GENUINE FRIENDSHIPS}: Difficult to answer, Aristotle does think that virtue-based friendships are pleasant and useful; so maybe he thinks that, while pure utility and pure pleasure based friendships aren't properly speaking `friendships', we call them that because they share features with what are, in fact, genuine friendships
\vspace*{2mm}

\noindent\textbf{READ 1147a4-6}: With the incomplete friendships, if they get the same thing from one another, they are more stable; the pederastic relationship was typically, the older man got pleasure (often in the form of sexual gratification), the younger got utility--training, connections, protection, literal presents, and so on
\vspace*{2mm}

\noindent\textbf{READ 1147a15-17} A little odd here, but does seem to say they aren't actually friends of each other; might be averring back to the coincidentally/in its own right idea
\vspace*{2mm}

\noindent \textbf{READ 1157a17} Aristotle allows that bad people can be friends of one another: good people of bad people and all sorts, but complete friendship is only open to people of virtuous character because that is the only kind that can be loved for its own sake in virtue of what it is and not coincidentally

\end{document}

