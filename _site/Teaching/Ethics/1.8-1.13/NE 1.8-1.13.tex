% !TEX encoding = UTF-8 Unicode
% !TEX TS-program = xelatex

\documentclass[11pt]{article}
\usepackage{amssymb}
\usepackage{fullpage}
\usepackage{hyperref}

\title{\emph{Nicomachean Ethics} 1.8-1.13}
\author{}
\date{}

\begin{document}

\maketitle

\noindent\underline{1.8}

\begin{itemize}\item{Discuss \emph{endoxa}}

\begin{itemize}\item{Account coheres with the common (and philosophically respected view) that goods of the soul are best (vs. of the body, external goods)}\item{Also agrees with aspects of the most reputable views}

\begin{itemize}\item{That happiness is virtue}\item{That happiness requires action: no prizes for just being a certain way}\item{That the happy life is also a pleasant life: the person who loves virtue finds virtuous activity pleasant}\item{In fact, on his view, they don't need to add pleasure as `decoration', since this life is already pleasant for the one who leads it (virtue is its own reward)}\begin{itemize}\item{Support: we have said that the highest good is virtuous activity, and someone who does not enjoy just actions, although they can do just actions, can't act justly; so, genuinely being just or generous requires taking pleasure in those activities}\end{itemize}

\end{itemize}

\item{\textbf{READ 1099a30ff.}}

\begin{itemize}\item{Need external goods to be happy: use wealth and power to perform fine actions}\item{Can't be happy if ugly? Find some way to read this on which it makes sense}\item{Aren't \emph{eudaimon} if we are solitary, friendless, childless; or if our children or friends are bad or have died}\item{\textbf{CAN THIS BE HAPPINESS WE ARE TALKING ABOUT HERE?}}\item{\textbf{DOES FLOURISHING FARE BETTER?}}\end{itemize}

\end{itemize}

\end{itemize}

\noindent\underline{1.9}

\begin{itemize}\item{A raises a very important question: how do we acquire \emph{eudaimonia}?}\begin{itemize}\item{Learning}\item{Habituation or some other form of cultivation}\item{Fate?}\item{Fortune}\item{Allude to \emph{Meno}}\end{itemize}

\item{He doesn't really have much of an argument here, but he seems to just be saying, happiness better be acquired via habituation and/or learning, because something as important as whether our lives go well or not hopefully isn't left to fortune}\begin{itemize}\item{Connections with teleology}\end{itemize}\end{itemize}

\noindent\underline{1.10}

\begin{itemize}\item{A's view might seem to raise a worry; can a person not be happy during their lifetime?}\item{If \emph{eudaimonia} is an activity, it seems crazy to say you are only happy after you die}\item{So, it has to be that we can only safely declare someone that he was blessed after he has died}\begin{itemize}\item{DISCUSS: perhaps JOB is an example; you might have said, if you knew where it was all heading, that he wasn't actually blessed when he seemed to be doing well}\end{itemize}\item{A further problem: if someone actually has good or evils of which he is unaware, that seems to affect whether he is happy or not}\item{But, then again, it would seem that someone's descendants can actually affect whether a person lived a \emph{eudaimon} life or not}\item{Here's the puzzle: either the fortune's of someone's descendants can affect whether they lived a happy life or not}\begin{itemize}\item{If yes, then a person could have lived a happy life up until 10 years after they died, but the switch after that}\item{If no, then descendants have no influence whatsoever; and one worry is that, if you think that, it must be because they are unaware, but then what about the things of which they were unaware in their life times}\end{itemize}

\item{So, A has been considering genuine puzzles concerning eudaimonia, and he thinks his account does the best to support our idea that eudaimonia is stable; because virtuous activity, constantly being performed and reinforced, is stable}\item{The eudaimon person is one who is best able to play the hand he is dealt}\item{\textbf{READ 1101A15}}

\end{itemize}

\noindent\underline{1.11}

\begin{itemize}

\item{So, what's the deal with our eudaimonia being influence after our death}\item{Major misfortunes differ more than minor misfortunes}\begin{itemize}\item{One thing to bear in mind with all this: A does seem to be setting up a picture on which an individual's eudaimonia does depend on the state of other people's lives (and, presumably, whether they are eudaimon)}\item{That is, that I am not as eudaimon if at least my family and friends are not eudaimon}\item{Does this speak to the self-sufficiency issue raised earlier?}\end{itemize}\item{People to consider: Van Gogh; Sextus Empiricus--they had profound influence as a result of doing the things they did, but all after they died; you have to say that it makes no difference whatsoever to the quality of their life; and that' just not obvious}

\end{itemize}

\noindent\underline{1.12}

\begin{itemize}\item{We don't actually praise someone for being happy}\item{That actually suggests it is the highest good}\end{itemize}

\noindent\underline{1.13}

\begin{itemize}\item{Given the account of eudaimonia we got, we have to discuss the virtues}

\item{This whole investigation, he says, is to be tailored to the needs of the political science; we only study it insofar as is necessary for that end}

\item{The soul}\begin{itemize}\item{Rational and non-rational}\begin{itemize}\item{Two different `parts' in some metaphysically robust sense of `part'? Or like the convex and concave? Allusion to Plato. Doesn't matter for our purposes}\end{itemize}\item{Two aspects to the non-rational part}\item{The nutritive part is not relevant: nothing (basically, except maybe dreams) distinguishes the good and bad person in sleep}\item{The other part is seen in the incontinent and continent person (the person who goes against reason}\item{This latter part can \emph{obey} reason; can be trained by reason}\item{The way in which someone can `follow' reason, even if it is not their own}\item{\textbf{RECALL BACK TO THE BASED ON VS. IN ACCORDANCE WITH REASON BACK AT 1.7}}\item{So, we could just as easily say that this is an aspect of the rational part of the soul}\end{itemize}

\item{So, there are two broad kinds of virtue: one virtue of the part that has reason strictly speaking; the other of the part that can obey reason}\item{But, one issue we will consider is whether the latter simply involves that part, or is solely determined by the state of that part}

\end{itemize}

\end{document}

