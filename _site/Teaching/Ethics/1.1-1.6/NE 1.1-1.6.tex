% !TEX encoding = UTF-8 Unicode
% !TEX TS-program = xelatex

\documentclass[11pt]{article}
\usepackage{amssymb}
\usepackage{fullpage}
\usepackage{hyperref}

\title{\emph{Nicomachean Ethics} 1.1-1.6}
\author{}
\date{}

\begin{document}

\maketitle

\noindent\textbf{Cover weekly questions}
\vspace*{2mm}

\noindent\textbf{Outline of \emph{NE} in Irwin xv-xvi}
\vspace*{2mm}

\noindent\textbf{Bekker numbers}
\vspace*{4mm}

\noindent\underline{1.1}

\begin{itemize}

\item{Read the first sentence together and get examples of each of these and the `goods' each seeks}

\begin{itemize}

\item{Craft (\emph{techn\^{e}})}

\item{Line of inquiry (\emph{methodos})}

\item{Action (\emph{praxis})}

\item{Decision (\emph{prohairesis})}

\end{itemize}

\item{Problem with the claim that `the good' is what everything seeks}\begin{itemize}\item{Every person loves some person does not entail that there is a single person that every one loves}\item{Some possibilities}\begin{itemize}\item{Just define `good' as `what something seeks'}\item{`the good' for X is what every X seeks}\end{itemize}\end{itemize}

\item{Read next sentence: distinction among kinds of ends (\emph{telos}); \textbf{GET EXAMPLES}}

\begin{itemize}\item{Activities}\begin{itemize}\item{Having a conversation with a friend}\item{Dancing}\end{itemize}\item{Products apart from activities}\item{E.g. shoe, house, car}\begin{itemize}\item{In this case, the product is by nature better than the activities}\item{I.e. if the aim of some activity is to produce some end, the end has more value than the activity}\end{itemize}\end{itemize}

\item{Ends can form a hierarchical structure}\item{Some ends are pursued, not for their own sake (or not just for their own sake), but for the sake of some further end they enable}\begin{itemize}\item{Equipment producing $\rightarrow$ Horsemanship $\rightarrow$ Generalship $\rightarrow$ Victory}\end{itemize}

\item{In such cases, the superordinate ends are more choice worthy than the ends subordinate to them}\item{The superordinate ends also set the standards of success for the lower ends}\begin{itemize}\item{E.g. if we are talking about producing equipment that will be used in warfare, the standard for whether some piece of equipment is well made is set by the demands of warfare}\end{itemize}

\end{itemize}

\noindent\underline{1.2}

\begin{itemize}

\item{Aristotle begins by saying if is some single end, which we do not pursue for the sake of anything else, but for which we ultimately pursue everything else we do}\begin{itemize}\item{\textbf{NB: he is not assuming there is such a thing, he's just asking us to suppose there is one so that he can talk about what it would be like}}\end{itemize}

\item{Aristotle has an argument that the ends we pursue must ultimately terminate in an end we do not pursue for the sake of anything else}\begin{itemize}\item{[P1] If there were no things we pursued \emph{solely} for themselves, our desires would be empty and futile}\item{\underline{[P2] But, our desires are not empty and futile}}\item{[C] There are some things we pursue solely for themselves}\end{itemize}

\item{So, this argument gives him that chains of ends have to come to an end, and the assumption is that there is only one of them}\begin{itemize}\item{This single end in which all our ends terminate would be the ``best'' (\emph{aristos}) good}\item{Note `\emph{Aristoteles}'=`best end'}\end{itemize}

\item{Aristotle's claim is that \emph{if there were such a best end}, it would be useful [1] to know that and [2] to know what it is}\begin{itemize}\item{This seems reasonable: if there were some single good at which all our ends ultimately terminated, it would be helpful to know what it was to orient ourselves in life, like an archer who actually knows and can see his target, as opposed to one who cannot}\end{itemize}

\item{Aristotle says that the study of this belongs actually to ``political science'', since politics aims at structuring society so that individuals and communities can realize their highest goods}\begin{itemize}\item{So, one important thing to keep in mind is that Aristotle sees the project he is pursuing here in the \emph{Ethics} to be intimately bound up with politics, and indeed the end of the \emph{Ethics} seems to segue into the work, \emph{The Politics}: \textbf{GO TO VERY END OF THE BOOK, P. 171 IN IRWIN, LAST PARAGRAPH}}\end{itemize}

\end{itemize}

\noindent\underline{1.3}

\begin{itemize}

\item{Aristotle here makes some points about methodology, and the kind of exactness that we can expect in the sort of project he is undertaking}\item{When it comes to the subject matter, human goods and such, we cannot expect demonstrative proof}\item{Aristotle says that the claims they make only hold for the most part}\item{\textbf{READ 1094b24--27}}\begin{itemize}\item{It's not entirely clear what this means, but we should keep it in mind as we read}\item{If, for example, we come across some claim that seems to admit of exceptions, perhaps this is where it is relevant}\item{Perhaps Aristotle means that, given the kind of matters we are discussing, we aren't going to discuss things that hold because of the nature of people}\item{E.g. if there are exceptions to his claims (perhaps even just real-world exceptions and not far-fetched examples), realize that he is talking about the main}\begin{itemize}\item{Tuesday pain indifferent as a foil}\end{itemize}\end{itemize}

\item{Youths have trouble with political science, since they lack experience with a broad range of actions}\item{Importantly, Aristotle says the end of political science is action, not knowledge: we aren't engaged in this study just to find things out about the world, but to have our behavior be modified appropriately}

\end{itemize}

\noindent\underline{1.4}

\begin{itemize}\item{Aristotle begins again seeking the highest good}\item{He claims that everyone agrees about the \emph{name} of this highest good, it is ``\emph{eudaimonia}''}\begin{itemize}\item{So, here is one of our first big questions: \emph{eudaimonia} is obviously going to be important to what follows, how should we understand it?}\item{Well, what we are told is that everyone will agree that the end for which we pursue everything else and which we don't pursue for the sake of anything else is \emph{called} \emph{eudaimonia}}\item{He also says that people equate \emph{eudaimonia} will `living well' and `doing well'}\item{He also says that it is a substantive question to ask: what is \emph{eudaimonia}}\item{Not only that, but people give different answers: pleasure, wealth, honor}\item{\textbf{SO, GIVEN THESE FACTS, HOW SHOULD WE SITUATE \emph{EUDAIMONIA} IN OUR CONCEPTUAL SCHEME; HOW SHOULD WE TRANSLATE IT?}}\item{Aristotle in fact notes that there is such disagreement on what \emph{eudaimonia} is that it would be impossible to catalogue them all}\item{He says we can help ourselves, just by considering the most plausible answers: those offered by well brought up people}\begin{itemize}\item{So, this is a pretty big issue, Aristotle isn't going to take just anyone's opinion seriously. Is that a limitation}\end{itemize}\end{itemize}

\noindent\underline{1.5}

\begin{itemize}\item{Aristotle says that, despite variations, there are three dominant views of what \emph{eudaimonia} is (Cf. Plato's \emph{Republic})}\begin{itemize}\item{Pleasure}\begin{itemize}\item{A seems not to take this too seriously, since he says this is the life for grazing animals; but, he does think that some weight is lended to it by the fact that people in positions of serious power have often pursued unbridled pleasure; and, the thought might be, if people who can do whatever they want want that, then maybe it is what is naturally best to pursue}\end{itemize}\item{Honor}\begin{itemize}\item{This is what politicians pursue; but it seems superficial to him}\item{Honor depends too much on others and can be taken away; but we want to think that the end we are looking for doesn't depend too much on the behavior of others (some is ok, but not so fickle)}\end{itemize}\item{Virtue}\begin{itemize}\item{At first seems an alternative to the end of the political life (i.e. not honor); but, it also can't be it, since someone can be virtuous yet have horrible misfortunes befall him}\item{\textbf{READ: 1096a, ALLUSION TO PLATO'S \emph{Republic}}}\end{itemize}\item{Money: purely instrumental}\item{Wisdom: only one left standing}\end{itemize}\end{itemize}

\end{itemize}

\noindent\underline{1.6}

\begin{itemize}\item{A here argues against the notion of the Platonic Form of the Good}\item{I.e. a good that is good in and of its self; not good for something, or good as a kind of thing; just good}\item{Aristotle does have some objections to the very idea that there is such a thing, but he is primarily concerned to argue that, even if there were such a thing, it wouldn't be any help in action}\item{Just as a carpenter can perfectly well get around in their carpentry business just by considering what is good in that domain (not helped at all in knowing what the good in itself is), so too the person pursuing the human good can do just fine without knowing what the good in itself would be}\end{itemize}


\end{document}

