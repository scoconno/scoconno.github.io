% !TEX encoding = UTF-8 Unicode
% !TEX TS-program = xelatex

\documentclass[11pt]{article}
\usepackage{polyglossia}
\usepackage{amssymb}
\usepackage{fullpage}
\usepackage{hyperref}
\setdefaultlanguage{english}
\setotherlanguage{greek}
\newfontfamily\greekfont{Gentium Plus}
\newcommand{\gk}[1]{\textgreek{#1}}

\title{G. E. M. Anscombe: `Modern Moral Philosophy}
\author{}
\date{}

\begin{document}

\maketitle

\noindent The aim of the next week and a half: consider the articles we read, both in their own right, and how they relate to Aristotle; is what they say about Aristotle, right? do the objections they raise actually bite against him?
\vspace*{2mm}

\noindent MMP is a badass paper; it crystallized a growing dissatisfaction with the going consequentialist and deontological ethical theories on offer at the time; it was incredibly influential, both immediately and to this day
\vspace*{2mm}

\noindent Perhaps the central claim of Anscombe's paper is that the notions of `moral obligation' `moral ought' `morally required', and really just `moral' generally, can not work in a secular framework
\vspace*{2mm}

\noindent Important to get the three main `modern moral philosophies' on the table

\begin{itemize}\item{Consequentalism: An action is good/right/morally required if and only if and because it maximizes value}\item{Deontology: An action is good/right/morally required iff and because it is in accord with a universal law}\item{Contractualism: ...iff and because it is in accord with some agreement}\end{itemize}

\noindent There is also serious debate as to what the ultimate upshot of Anscombe's article is supposed to be

\begin{itemize}\item{Straightforward rejection of these moral theories, in favor of one that also dispenses with a law-giver, and also gets rid of those notions which, outside of the legalistic framework, have no content}\item{As a \emph{reductio} in \emph{favor} of a religiously based ethical theory}\begin{itemize}\item{The idea would be the MMP is a carefully crafted argued aimed to show what would be needed to give a viable alternative to a religiously based ethical theory, and that such a thing would be impossible}\item{In particular, Anscombe, here and elsewhere, raises serious doubts about getting actually clear on a notion of human flourishing}\item{And, we know from elsewhere that she was a believer and did believe in absolute prohibitions}\end{itemize}\end{itemize}

\noindent First main claim she makes about Aristotle: the notion of `moral', in the sense of `moral' responsibility, or `morally' blameworthy, just doesn't seem to fit
\vspace*{2mm}

\noindent She says, `we have the term moral by direct inheritance from Aristotle', but that doesn't seem to be right

\begin{itemize}\item{From Latin: \emph{moralis}, related to \emph{mos}, \emph{mor}-; plural \emph{mores}: custom, habits, mores}\item{Cicero coined the term in \emph{On Fate} to translate `\emph{ethik\^{e}}'}\end{itemize}

\noindent `We cannot, then, look to Arisottle for any elucidation of the modern way of talking about `moral' goodness, obligation, etc.'
\vspace*{2mm}

\textbf{\noindent ***The notion of `brute fact' is somewhat difficult to glean, but it seems to be something like: that just is what it is to owe X amount of money; it's not an analysis of the notion}
\vspace*{2mm}

\noindent\textbf{xyz (carted potatoes to house; left there) brute relative to A (supplied me with potatoes); there can be other facts, abc, that are also brute relative to A (e.g. put potatoes on plane, mailed them, and so on)***}
\vspace*{2mm}

\noindent So, what we should talk about: the terms `should', `ought' `needs' relate to good and bad; but not in a moral sense; but, now they have a technical sense: `i.e. a sense in which they imply some absolute verdict (like one of guilty/not guilty on a man)
\vspace*{2mm}

\noindent These terms are part of a `law conception of ethics'; they are equated with ideas like `is obliged', `is bound', `is required to', in the sense in which someone can be required by law
\vspace*{2mm}

\noindent How: VIA CHRISTIANITY and it's `law conception of ethics': to think that failure to act in conformity with the virtues makes you bad \emph{qua} man requires the idea that it is \textbf{REQUIRED BY DIVINE LAW; REQUIRES GOD AS LAWGIVER}
\vspace*{2mm}

\noindent And, in many contemporary moral theories (cons. and deont.), don't invoke God to derive obligation, but still use the term; and it becomes somewhat sppoky
\vspace*{2mm}

\noindent Here we have: `the survival of a concept outside the framework of thought that made it a really intelligible one'
\vspace*{2mm}

\noindent Going from is to `morally ought', is, indeed problematic; it is as if a verdict is being reached; but for that very notion to be intelligible, it has to beagainst the backdrop of a law and a judge
\vspace*{2mm}

\noindent So, she thinks that the central defect of MMP is that it invokes a notion: morally ought; that only makes sense within a framework that they reject (or, at least, that they don't invoke in their moral theory)
\vspace*{2mm}

\noindent But, we don't need such a thing to do ethics; instead of `morally wrong' we should just appeal to virtue and vice genera
\vspace*{2mm}

\noindent Indeed, just do away with talk of right and wrong all together: just ask, was it just/unjust, chaste/unchaste and so on
\vspace*{2mm}

\noindent The crucial similarity between all academic moral philosophers is a hard consequentialist line; inability to say that certain things are wrong or right, irrespective of the consequences
\vspace*{2mm}

\noindent There are, in fact, no prohibitions or doing-orders; it's all just relative value
\vspace*{2mm}

\noindent So, there is no sense of a law conception of ethics on consequentialism: the kinds of things that would have been fawned upon the ethical agent by `wives and flatterers' become moral principles
\vspace*{2mm}

\noindent Is there a way to retain the law conception of ethics in a secular world view?

\begin{itemize}\item{Maybe appeal to self-legislator? I ought to phi because that conforms with the law I set myself}\begin{itemize}\item{But, the idea is that you are looking for the law following which guarantees right/good action; but, as is clear, self-legislation will only lead to good action based on its content}\end{itemize}\item{Perhaps contractualism?}\begin{itemize}\item{But, the awareness for being bound to a contract rises higher than the level of being bound by a law}\end{itemize}\end{itemize}

\noindent Aristotelian style: look for `norms' in human nature; perhaps, given human nature, humans have certain virtues `in them'; but, this isn't really `law'
\vspace*{2mm}

\noindent What we should just do is talk of action, intention, pleasure, wanting, virtue; start ethics by studying this notion
\vspace*{2mm}

\noindent She then closes by examining some upshots of abandoning taught of morally right and morally wrong
\vspace*{2mm}

\noindent One thing: we don't get confused in certain cases; there are cases where something is obviously unjust, no matter what, but we can always ask whether it might be `morally right'
\vspace*{2mm}

\noindent P 17: `But if someone really thinks, \emph{in advance}, that it is open to question whether such an action as procuring the judicial execution of the innocent should be quite excluded from consideration---I do not want to argue with him; he shows a corrupt mind' 
\vspace*{2mm}

\noindent Question: doesn't this just raise the question `well, ought I do the unjust thing'? if `unjust' applies, strictly in virtue of the facts, without bringing wrong in; can't it be a reasonable question to ask whether it can be right to do the unjust thing?
\vspace*{2mm}

\noindent What is this sense of `ought'; `would be morally right/wrong to?' outside of a divine law-giver notion of ethics; within it, we can make sense of saying that it is wrong to do injustice; it is divine law that obliges; but, without that; what are we actually asking?
\vspace*{2mm}

\noindent Anscombe: it has no \emph{content} just spcyhological potency
\vspace*{2mm}

\noindent So, sum up the two understandings of this article:

\begin{itemize}\item{Modus Ponens}\begin{itemize}\item{(1) If religiously based ethics is false, then virtue ethics is the way moral philosophy ought to be developed.}\item{(2a) Religious based ethics is false (at least for her interlocutors)}\item{(3a) Therefore, virtue ethics is the way moral philosophy should be developed.}\end{itemize}\item{Modus Tollens}\begin{itemize}\item{(1) If religiously based ethics is false, then virtue ethics is the way moral philosophy ought to be developed.}\item{(2b)	It is not the case that virtue ethics is the way to develop moral philosophy}\item{(3b) Therefore, it is not the case that religiously based ethics is false.}\end{itemize}\end{itemize}



\end{document}