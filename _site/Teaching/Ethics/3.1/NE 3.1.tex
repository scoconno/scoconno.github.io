% !TEX encoding = UTF-8 Unicode
% !TEX TS-program = xelatex

\documentclass[11pt]{article}
\usepackage{polyglossia}
\usepackage{amssymb}
\usepackage{fullpage}
\usepackage{hyperref}
\setdefaultlanguage{english}
\setotherlanguage{greek}
\newfontfamily\greekfont{Gentium Plus}
\newcommand{\gk}[1]{\textgreek{#1}}

\title{\emph{Nicomachean Ethics} 3.1}
\author{}
\date{}

\begin{document}

\maketitle

\noindent So, we got the general account of virtue down, and a bit of a sketch of the particulars; now Aristotle backs up to talk about some preconditions of virtuous action
\vspace*{2mm}

\noindent It's been clear that A thinks that virtue and virtuous action receives praise; in fact, a test for whether or not you've identified correctly what virtue is is whether what you've specified is the kind of thing that receives praise
\vspace*{2mm}

\noindent Since we only praise or blame voluntary actions, and pardon or pity involuntary ones, we have to discuss the voluntary and the involuntary (\emph{hekousios}, \emph{akousios})
\vspace*{2mm}

\noindent Distinguish voluntarily \gk{φ}-ing from being responsible for \gk{φ}-ing

\begin{itemize}\item{Three main kinds of responsibility}\begin{itemize}\item{(1) Duties, obligations}\item{(2) Storm responsible for knocking out the power}\item{(3) Praiseworthy or blameworthy}\end{itemize}\end{itemize}

\noindent It's hard to determine, because Aristotle says that praise and blame are bestowed only on voluntary actions; but he doesn't seem to endorse the reverse, that if it is a voluntary action, then it is appropriate to bestow praise or blame on it
\vspace*{2mm}

\noindent Aristotle goes about this negatively, citing conditions that make actions involuntary and then negating them to get conditions for voluntary actions
\vspace*{2mm}

\noindent In all of this, we have to identify what the in/voluntary action is supposed to be; in these cases, is it necessary that there is some other action which \emph{is} voluntary which is somehow related to the action that is involuntary?
\vspace*{2mm}

\noindent\textbf{Force}

\begin{itemize}\item{Physical external power: the doer or suffer contributes nothing at all}\begin{itemize}\item{E.g. being carried off under the power of others}\item{So, one thing we have to notice is that, in this discussion, we have to identify an action which, in some sense, the agent `does' and then ask whether the agent does that action voluntarily or not}\end{itemize}\item{Mental, somewhat internal, power: when someone threatens you with a greater evil if you don't \gk{φ}}\begin{itemize}\item{E.g. tyrant holds your family hostage and says, if you \gk{φ}, they will live, if you don't \gk{φ}, they will die}\item{`Mixed' but `more like voluntary'}\item{When they are performed, they are choice worthy, and whether the action is voluntary or involuntary should depend on its particular features}\item{And, at the time of action, the moving principle is in the agent himself}\item{Involuntary \emph{in its own right}, but, in the particular situation, voluntary}\end{itemize}\item{A thinks that, since we do praise or blame such actions, they are voluntary}\item{\textbf{READ 1110b3-9}}\item{Worth noting the asymmetry between the praise and blame and good and bad actions}\item{So, forced actions are just those by external physical sources: \textbf{READ 1110b17-18}}\end{itemize}


\noindent\textbf{Ignorance}: everything `caused by' ignorance is non-voluntary (\emph{ouk hekousios}), but what is involuntary also involves pain and regret (\textbf{DOES THIS APPLY TO FORCE AS WELL?})
\vspace*{2mm}

\noindent I do suspect there is an important difference here: \textbf{DISCUSS}
\vspace*{2mm}

\noindent Two main ways in which we speak of someone being ignorant of X (is only one in play here?)

\begin{itemize}\item{Having a false belief about X}\item{Having no beliefs concerning X}\end{itemize}

\noindent Distinction between:

\begin{itemize}\item{Action caused by ignorance}\begin{itemize}\item{Here, the explanation why you \gk{φ}-ed is that you were ignorant (of something or other)}\end{itemize}\item{Action done in ignorance}\begin{itemize}\item{Here, while you were, in fact, ignorant of something, the fact that you were ignorant is not the explanation why you \gk{φ}-ed}\item{The example here is drunk or angry people; they are ignorant, but their action is not caused by their ignorance}\end{itemize}\end{itemize}

\noindent\textbf{READ 1110b31-1111a4} Importantly, he doesn't think that the ignorance is at the level of decision; that is, deciding to \gk{φ} because you think \gk{φ} is or will achieve something good, when it fact it doesn't, isn't the relevant kind of ignorance
\vspace*{2mm}

\noindent Rather, it is ignorance of certain particulars relevant to the action (does this suggest that decisions are universal-ish?)
\vspace*{2mm}

\noindent The particulars are:

\begin{itemize}\item{Who is doing it}\begin{itemize}\item{A says no one could be ignorant of this, unless he were mad}\end{itemize}\item{What he is doing}\begin{itemize}\item{E.g. when someone says that [the secret] slipped out while he was speaking; Aeschylus said about the mysteries, that he did not know if was forbidden; letting a catapult go when he wanted to demonstrate it}\end{itemize}\item{About what or to what he is doing it}\begin{itemize}\item{I take the example of not knowing his son is an enemy; the idea seems to be that, you killing your enemy is also killing your son, you involuntarily kill your son because you were ignorant that the recipient of your action was your son}\end{itemize}\item{What he is doing it with--with what instrument}\begin{itemize}\item{Thinking spear has button on it (for practice); being ignorant that the stone is pumice stone}\end{itemize}\item{For what result (e.g. for safety)}\begin{itemize}\item{Thinking the drink will have the result of safety, it kills}\end{itemize}\item{In what way (e.g. gently vs. hard)}\begin{itemize}\item{Wanting to merely tap someone in sparring, we swing too hard}\end{itemize}\end{itemize}

\noindent\textbf{READ 1111a17} A little odd, `especially if'
\vspace*{2mm}

\noindent So, voluntary action is what has its principle in the agent himself, knowing the particulars that constitute the action
\vspace*{2mm}

\noindent Importantly for Aristotle, actions caused by appetitive or spirited desires are nevertheless voluntary (this is against a common view that, if you don't act on a rational desire (generated, we will learn, by deliberation)) then the action is involuntary

\end{document}

