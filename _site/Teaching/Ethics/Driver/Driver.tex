% !TEX encoding = UTF-8 Unicode
% !TEX TS-program = xelatex

\documentclass[11pt]{article}
\usepackage{polyglossia}
\usepackage{amssymb}
\usepackage{fullpage}
\usepackage{hyperref}
\setdefaultlanguage{english}
\setotherlanguage{greek}
\newfontfamily\greekfont{Gentium Plus}
\newcommand{\gk}[1]{\textgreek{#1}}

\title{Julia Driver: `A Consequentialist Theory of Virtue'}
\author{}
\date{}

\begin{document}

\maketitle

\noindent So, the issue Driver is concerned with is: what makes a character trait a virtue
\vspace*{2mm}

\noindent So, remember Hursthouse on this last time: many virtue ethicists nowadays give some independent specification of \emph{eudaimonia} and say that virtues are those character traits necessary to achieving that
\vspace*{2mm}

\noindent How do we think that Aristotle would answer this question in the abstract?
\vspace*{2mm}

\noindent Basic view: value of all `virtue' traits resides `in their tendency to produce good consequences'
\vspace*{2mm}

\noindent She is first interested in: what makes a trait a virtue; Hume seems to have been of the view that a virtue is just a trait which, when we think about it, pleases us
\vspace*{2mm}

\noindent And, she thinks Hume is moving towards a consequentialist theory of virtue because, for the most part, he thinks we find certain traits pleasing \emph{because} we think their consequences are beneficial
\vspace*{2mm}

\noindent So, in the abstract, the question is: we view generosity as a virtue? Why? Because we think generous acts and generous people lead to better outcomes?
\vspace*{2mm}

\noindent Now, here's the question: if we learned that actions we typically labeled as `generous' actually had bad outcomes, what would our reaction be?

\begin{itemize}\item{Think the generosity is no longer a virtue}\item{Think that we were wrong as to whether those actions were generous?}\item{What?}\end{itemize}

\noindent `As long as the trait generally produces good, it is a virtue'
\vspace*{2mm}

\noindent Important distinctions:

\begin{itemize}\item{Objective vs. subjective consequentialism}\begin{itemize}\item{MVs=character traits that systematically produce more actual good than not}\item{SC=whenever one faces a choice of actions, one should attempt to determine which act of those available would most promote the good, and should then try to act accordingly; basically, criterion of right action is \emph{expected} utility}\end{itemize}\item{Evaluational internalism vs. externalism}\begin{itemize}\item{Internalism: Moral worth determined by features \emph{internal} to agency: will, motive etc.}\begin{itemize}\item{Kant is arguably an evaluational internalist: for him, moral worth attaches only to the quality of the will; the person is insulated from the capricious intervention of the external world; less subject to luck}\end{itemize}\item{Externalism: external to agency (but can be internal \emph{to the agent}}\end{itemize}\item{Direct vs. Indirect consequentialism}\begin{itemize}\item{Direct: thing to be evaluated is evaluated in terms of its own consequences}\item{Indirect forms hold that the thing to be evaluated is evaluated in terms of the consequences of some related item.}\begin{itemize}\item{Moral quality of action is determined by consequences of the character trait that produces the action}\item{So, for example, if someone acts from a malicious character, but just so happens to save a bunch of people due to external mishap, the action is still deemed bad, since it came from a character whose consequences tend to be bad}\end{itemize}\end{itemize}\end{itemize}

\noindent So, on thing that's important is the Driver thinks her account better captures intuitions about hard moral cases: her view is Direct Externalist Objective Consequentialist
\vspace*{2mm}

\noindent The Sheriff: has to choose between saving one innocent man or allowing twenty innocent people to die in ensuing riot

\begin{itemize}\item{Indirect would say: action is wrong, because it would stem from a vicious character train (willful disregard for human life)}\item{Direct: right, but we could have questions about the character of the agent who could do it--Direct captures the ambivalence}\end{itemize}

\noindent She thinks this captures our often split evaluation of situations: we can recognize that good people can do bad things; and, in fact, can do bad things \emph{because} they are good
\vspace*{2mm}

\noindent Also respects agent/character distinction: act is right but, since it is typically wrong, the agent feels qualms
\vspace*{2mm}

\noindent Her view is not maximizing: for a trait to be a virtue it doesn't have to maximize the good
\vspace*{2mm}

\noindent She thinks that this captures the idea that virtues tend to promote good consequences but sometimes have bad outcomes: seat belts save lives (that's their virtue), but not always, sometimes they lead to death (e.g. exploding on impact)
\vspace*{2mm}

\noindent There is some tension here that relates to the `however circumstances' and `ideal circumstances': she wants to say that the virtues are those character traits that would lead to good outcomes
\vspace*{2mm}

\noindent But, it would seem that we can re-jig circumstances such that what seem to be pretty gnarly character traits do that
\vspace*{2mm}

\noindent \textbf{HOW DOES LUCK FACTOR INTO VIRTUE ASCRIPTIONS AND EVALUATIONS?}
\vspace*{2mm}

\noindent It would seem that Objective Consequentialism is particularly subject to luck objections
\vspace*{2mm}

\noindent But, by distinguishing character from action; and understanding virtues dispositionally, we can get around this problem




\end{document}