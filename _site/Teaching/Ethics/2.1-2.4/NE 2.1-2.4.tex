% !TEX encoding = UTF-8 Unicode
% !TEX TS-program = xelatex

\documentclass[11pt]{article}
\usepackage{amssymb}
\usepackage{fullpage}
\usepackage{hyperref}

\title{\emph{Nicomachean Ethics} 2.1-2.4}
\author{}
\date{}

\begin{document}

\maketitle

\noindent\underline{2.1}
\vspace*{2mm}

\noindent Virtue of thought is acquired via teaching
\vspace*{2mm}

\noindent VOC through habituation (\emph{ethike} related to \emph{ethos}--\emph{endoxic} method)
\vspace*{2mm}

\noindent Thus, VOC doesn't arise naturally

\begin{itemize}\item{By that, A means that, just let it develop, and it will acquire VOC}\begin{itemize}\item{Can't make a stone move against its nature}\end{itemize}\item{But, it's neither against nature}\item{Rather, by nature able to acquire, completed through habit}\end{itemize}

\noindent When something does arise in us by nature, we first have the capacity and then the activity (e.g. perception)
\vspace*{2mm}

\noindent But, virtues, like crafts, we first act and then acquire the capacity

\begin{itemize}\item{So, for Aristotle, phi-ing doesn't always entail having the capacity to phi--i.e. making a chair doesn't entail that you have the relevant craft}\item{This differs from how most people think of possibility: phi-ing entails possibly phi-ing; so, for Aristotle, capacities are genuinely explanatory in a way that possibly isn't (i.e. it might be explanatory to say `because he's a craftsmen (i.e. has the relevant capacity, in a way `because he can' isn't)}\item{So, just as we become builders by building and cobblers by cobbling, we become virtuous by performing virtuous actions}\item{\textbf{READ 1103b14-25}}\end{itemize}

\noindent\textbf{DISCUSS, IN GENERAL, THE ANALOGY OF VIRTUE AND A CRAFT}

\begin{itemize}\item{So far this is just about acquisition}\item{Does A think that, in the same way there is an external product in crafts, there will be for virtue?}\end{itemize}

\noindent\underline{2.2}
\vspace*{2mm}

\noindent\textbf{READ 1103b33-5: CORRECT REASON, GONNA BE A BIG ISSUE}
\vspace*{2mm}

\noindent\textbf{ALSO READ 1104a1-10: LET'S PAY ATTENTION TO THE WAY IN WHICH THIS WORKS GOING FORWARD; THIS REALLY DOES SEEM TO BE HOW HE'S ROLLING; HE IS NOT TRYING TO PROVIDE AN ALGORITHMIC ACCOUNT, SUCH THAT YOU CAN PLUG IN VALUES FOR VARIABLES AND GET AN ANSWER WHETHER X IS VIRTUOUS OR WHAT WOULD BE THE VIRTUOUS ACTION}

\begin{itemize}\item{This seems, at least initially, to conflict with both UTILITARIANISM AND DEONTOLOGY, which are roughly algorithmic}\item{Rather, A seems to be saying, we give the outline of happiness, and the virtues, so that the agent knows the features to be sensitive too}\item{In a sense, the idea is that the virtuous person will just get it right}\end{itemize}

\noindent\textbf{1104a13-19} General observation: excess and deficiency tend to ruin `these sorts of states' (presumably, states acquired via habituation)

\begin{itemize}\item{Courage: if a person stands firm against everything and is afraid of nothing, he becomes/is cowardly; if the person rushes into everything, becomes rash}\item{Temperance: If he indulges in every pleasure, becomes intemperate; if he abstains from all he becomes `insensible' (an ascetic?)}\begin{itemize}\item{But not only are these actions that promote the virtues, virtuous activity consists in them}\end{itemize}\item{So, generally, virtues are ruined by excess or deficiency, and preserved by the mean}\end{itemize}

\noindent\underline{2.3}
\vspace*{2mm}

\noindent FEELS MATTER TOO; it isn't just a matter of what you do, begin virtuous requires certain feelings attendant on the actions
\vspace*{2mm}

\noindent\textbf{READ 1104B5-10}: DISCUSS
\vspace*{2mm}

\noindent So, one reason why pleasure and pain are important to this study is that they are often the things that lead us away from the virtuous action (rarely do we do what is not-courageous because doing so is \emph{more painful} than doing the courageous action)
\vspace*{2mm}

\noindent Indeed, pleasure and pain so often have this tendency, that many people define virtues as various ways of being unaffected and undisturbed by pleasures and pains (this is very Stoic?)

\begin{itemize}\item{Three objects of choice}

\begin{itemize}\item{Fine (\emph{kalon})}\item{Expedient (\emph{sumpheros})}\item{Pleasant (\emph{hEdeos})}\end{itemize}\item{Three objects of avoidance}\begin{itemize}\item{Shameful (\emph{aischron})}\item{Harmful (\emph{blaberos})}\item{Painful (\emph{lupEros})}\end{itemize}\end{itemize}

\noindent Good person is correct, bad person base: this is going to be a running theme throughout the work: what makes things the way they are (is it because the good person likes them that they are good, or something else?)
\vspace*{2mm}

\noindent\textbf{READ 1105a15: TO SUM UP}
\vspace*{4mm}

\noindent\textbf{BASICALLY READ THE WHOLE CHAPTER}

\noindent\underline{2.4}
\vspace*{2mm}

\noindent Puzzle raised by claim that we become virtuous by doing virtuous actions: if we perform virtuous actions, aren't we virtuous?
\vspace*{2mm}

\noindent In other cases, it would seem we can do so: if we play a musical instrument, mustn't we be musical
\vspace*{2mm}

\noindent But, actions aren't enough: we can perform an action by chance and not actually possess the relevant disposition: we must not only do the action, but do it in a certain way
\vspace*{2mm}

\noindent Contrast between virtues and crafts

\begin{itemize}\item{With crafts, it is the quality of the product that determines whether it was performed well or not}\item{But, for actions done in accordance with virtues to be done temperately or justly, it's not sufficient for the actions to have certain qualities; \emph{agent} must be in the right state}\begin{itemize}\item{[1] Must know: \emph{what}? (Irwin puts in `that he is doing virtuous actions'}\item{[2] Must decide on them, and decide on them for themselves}\item{[3] Must do them from a firm and unchanging state (\emph{echOn}}\end{itemize}\item{When it comes to crafts, only [1] matters: \textbf{WHY?}}\item{As conditions for virtue [2] and [3] are all important, [1] counts only for very little}\begin{itemize}\item{And, we acquire [2] and [3] by frequent doing of just and temperate actions}\end{itemize}\end{itemize}

\noindent Thus, he says that `actions are called just or temperate, when they are the sort that a just or temperate person would do
\vspace*{2mm}

\noindent\textbf{BIG INTERPRETATIVE QUESTION}

\begin{itemize}\item{Is this a statement of what makes it the case that a virtuous action is virtuous (i.e. it is such that a virtuous person would do it (in the relevant circumstances)}\item{Or, is it not (perhaps it's just a coextensive claim)}\end{itemize}

\noindent Just an temperate person does it in the right kind of way (i.e. in accord with [1], [2], and [3])



\end{document}

