% !TEX encoding = UTF-8 Unicode
% !TEX TS-program = xelatex

\documentclass[11pt]{article}
\usepackage{polyglossia}
\usepackage{amssymb}
\usepackage{fullpage}
\usepackage{hyperref}
\setdefaultlanguage{english}
\setotherlanguage{greek}
\newfontfamily\greekfont{Gentium Plus}
\newcommand{\gk}[1]{\textgreek{#1}}

\title{\emph{Nicomachean Ethics} 7.1-7.3}
\author{}
\date{}

\begin{document}

\maketitle

\noindent\underline{6.13}
\vspace*{4mm}

\noindent Just as prudence relates to cleverness; genuine virtue relates to natural virtue
\vspace*{2mm}

\noindent Natural virtue has caused a lot of problems: given that, since the very beginning of book 2, we have been learning that we are not virtuous by nature, it's hard to see what `natural virtue' could be; but, it seems to be something like, having an unreflective, `natural' disposition to do virtuous actions
\vspace*{2mm}

\noindent\textbf{READ 1144b4-7}: NB, should take `prone' to cover all examples
\vspace*{2mm}

\noindent Full virtue cannot be acquired without prudence
\vspace*{2mm}

\noindent Soc. was wrong to think that all virtues were prudence (or instances of prudence), but he was right to think that all virtues require prudence
\vspace*{2mm}

\noindent\textbf{READ 144b23}: Here we get the answer to `what is the correct reason'
\vspace*{2mm}

\noindent Now, raise the issue: what is the word `logos' supposed to mean?
\vspace*{2mm}

\noindent We cannot be fully good without prudence, or prudent without virtue of character
\vspace*{2mm}

\noindent So, while people think: yes, you can be courageous without the other virtues, they are mistaken; you can have natural tendencies in line with one virtue and not others, but can't have full virtue (i.e. real virtue)
\vspace*{2mm}

\noindent So, here the basic idea is that, the agent must actually understand why what they are doing is the virtuous thing, to be virtuous
\vspace*{2mm}

\noindent So, even if prudence were useless in action (\textbf= WE COULD GET THE SAME RESULTS EVEN IF WE WEREN'T PRUDENT), it would still be valuable

\section{7.1}

\noindent Three character kinds to be avoided; three opposites

\begin{itemize}\item{Vice / Virtue}\item{Incontinence / Continence}\item{Bestiality / Super human goodness}\end{itemize}
\newpage
\underline{\hspace*{10mm} Vicious\hspace*{10mm}Akratic\hspace*{10mm}Enkratic\hspace*{10mm}Virtuous}

\noindent Action\hspace*{8mm}Bad\hspace*{14mm}Bad\hspace*{15mm}Good\hspace*{17mm}Good
\vspace*{2mm}

\noindent Decision \hspace*{5mm}Bad\hspace*{13mm}Good\hspace*{14mm}Good\hspace*{16mm}Good
\vspace*{2mm}

\noindent NR Desires\hspace*{2mm}Bad\hspace*{12mm}Bad\hspace*{16mm}Bad\hspace*{17mm}Good
\vspace*{4mm}

\section{7.2}

\noindent\textbf{READ start of 7.2}
\vspace*{2mm}

\noindent Some issues: it seems obvious that people can act both against knowledge and belief
\vspace*{2mm}

\noindent But, Soc says it isn't, and Soc was pretty smart guy
\vspace*{2mm}

\noindent One thing that seems clear: \emph{before} acting, the akratic person doesn't think he should do what he ends up doing
\vspace*{2mm}

\noindent Then runs through some issues surrounding questions surrounding standard views: some say you can act against what you merely \emph{believe} to be best, but not against what you \emph{know} to be best
\vspace*{2mm}

\noindent Enkrateia seems like a detrimental thing if it makes someone stand by every decision, including bad ones; it might be better for some people to act against their decisions to do the wrong thing
\vspace*{2mm}

\noindent \textbf{READ 1146a23ff} Interesting argument: foolishness + incontinence = virtue
\vspace*{2mm}

\noindent And, he also thinks it might be the case that someone who decides to do the wrong thing and acts on the decision is actually better off than someone who decides to do the right thing and acts against it: `if water chokes us, what must we drink to wash it down?'

\section{7.3}

\noindent\textbf{READ first graph}: Lays out quite nicely the agenda
\vspace*{2mm}

\noindent Simple or unqualified incontinence is not about everything, but concerns the same range of things that intemperance ranges over (roughly, bodily pleasures)
\vspace*{2mm}

\noindent Answers one question quickly: solution won't rest in distinguishing belief from knowledge/understanding: since people with belief can be as determined as people with knowledge
\vspace*{2mm}

\noindent\textbf{FIRST WAY}: We now get various distinctions in the sense in which a person `has' knowledge

\begin{itemize}\item{`Has' in sense of possesses it vs. `Has' in sense of attends to it (the latter presupposes the former)}\begin{itemize}\item{It's not that odd to think that someone could understand in the former sense that something is the wrong action, but fail to attend to it}\item{E.g. someone knows that it's wrong to drink and drive; decides to be DD at a party; is caught up in a conversation during which people are passing a bottle around, and just unthinkingly drinks}\item{Here, in a sense, they knew/understood/decided that they should not drink, but did it anyways}\end{itemize}\end{itemize}

\noindent Important distinction between two propositions operative in action:

\begin{itemize}\item{Universal proposition: represents the agent's decisions (and, so, their beliefs and/or knowledge about how best to act}\begin{itemize}\item{E.g. Don't eat dark meat}\item{Don't drink alcohol}\end{itemize}\item{Particular proposition: represents agent's awareness or knowledge of particular facts relevant to one's action}\begin{itemize}\item{This is an alcoholic drink}\end{itemize}\end{itemize}

\noindent\textbf{SECOND WAY}: One can act against a universal proposition, which they may attend to, if they fail to use some relevant particular proposition

\begin{itemize}\item{E.g. At the party the person is discussing drinking and driving, vehemently arguing against it; so, she is attending to her universal knowledge; but, given her focus, is unaware that the drink she is passed is alcoholic, or forgets that she herself must drive later}\item{So, although one has the knowledge, and attends to it, she is unaware that it is applicable to her situation in the relevant way}\end{itemize}

\noindent In both these, it seems that, in an important sense, they aren't instances of what we are really looking for: in each case, if the person were aware, we could imagine he or she would have acted differently
\vspace*{2mm}

\noindent \textbf{READ 1147a10-19}
\vspace*{2mm}

\noindent Four important points about this passage
\begin{itemize}\item{In this case a person both has (in a way) and does not have (in a way) knowledge}\item{The way of having and not having is characteristic of people who are drunk, mad, and asleep}\item{People in grip of emotions are like this}\item{The state of mind )w/r/t knowledge) of akratic people is the same as such people when they have but don't use knowledge}\end{itemize}

\noindent One important point: Drunk people \emph{cannot} use their knowledge, though they still possess it
\vspace*{2mm}

\noindent Alcohol affects across the board; it's not just specific (it isn't, for example, a case of forgetfulness or self-deception)
\vspace*{2mm}

\noindent In particular, what is suspended is the ability to grasp reasons for action; why this is the action to perform, as opposed to some other action; or why this action is good and good for one
\vspace*{2mm}

\noindent The reason why someone transitions from having made the decision to being unable to stick to the reasons in favor of it, is physiological, that's why A turns to that issue
\vspace*{2mm}

\noindent\textbf{READ 1147a25ff}: Here we get the famous `practical syllogism'
\vspace*{2mm}

\noindent Many people think that the practical syllogism is, in some way, a model or formalization of deliberation

\begin{itemize}\item{I think that makes no sense: how could this be seen as in any way a formalizing of: laying down an end and thinking about how best to achieve that end?}\item{Rather, it is the explanation for how an agent's beliefs and desires combine with her perception of her circumstances to generate action}\end{itemize}

\noindent Important the `nothing prevents it': means, nothing external to the agent qua agent (i.e. nothing external to her beliefs or desires); if it could allow that the agent's desires could prevent it, it would lose it's explanatory force
\vspace*{2mm}

\noindent So, what the two syllogism passage describes is the agent's state of mind just before acting akratically: how the transition happens and how the reversion happens aren't to be explained by the ethicist
\vspace*{2mm}

\noindent So, what happens is, the presence of countervailing desires interferes with the agent's ability to fully appreciate, in the moment, why what she had originally decided to do was the right thing to do; she can say the words, but not as knowledge
\vspace*{2mm}

\noindent So, perhaps somewhat surprisingly, Aristotle ends up agreeing with Socrates: \textbf{READ 1147b14}
\vspace*{2mm}

\noindent Voluntary? Not forced (appetite is an appropriate `internal' principle); not ignorant either, unless the standards are way to high

\end{document}

