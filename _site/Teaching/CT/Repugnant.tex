\documentclass[9pt]{article}

\usepackage{fancyhdr}
 \pagestyle{fancy}
\rhead{\textsc{Torbjorn Tannsjo}}

\usepackage{lmodern}
\usepackage{amssymb,amsmath}
\usepackage{ifxetex,ifluatex}
\usepackage{fixltx2e} % provides \textsubscript
\ifnum 0\ifxetex 1\fi\ifluatex 1\fi=0 % if pdftex
  \usepackage[T1]{fontenc}
  \usepackage[utf8]{inputenc}
\else % if luatex or xelatex
  \ifxetex
    \usepackage{mathspec}
    \usepackage{xltxtra,xunicode}
  \else
    \usepackage{fontspec}
  \fi
  \defaultfontfeatures{Mapping=tex-text,Scale=MatchLowercase}
  \newcommand{\euro}{€}
\fi
% use upquote if available, for straight quotes in verbatim environments
\IfFileExists{upquote.sty}{\usepackage{upquote}}{}
% use microtype if available
\IfFileExists{microtype.sty}{%
\usepackage{microtype}
\UseMicrotypeSet[protrusion]{basicmath} % disable protrusion for tt fonts
}{}
\ifxetex
  \usepackage[setpagesize=false, % page size defined by xetex
              unicode=false, % unicode breaks when used with xetex
              xetex]{hyperref}
\else
  \usepackage[unicode=true]{hyperref}
\fi
\usepackage[usenames,dvipsnames]{color}
\hypersetup{breaklinks=true,
            bookmarks=true,
            pdfauthor={},
            pdftitle={},
            colorlinks=true,
            citecolor=blue,
            urlcolor=blue,
            linkcolor=magenta,
            pdfborder={0 0 0}}
\urlstyle{same}  % don't use monospace font for urls
\setlength{\parindent}{0pt}
\setlength{\parskip}{6pt plus 2pt minus 1pt}
\setlength{\emergencystretch}{3em}  % prevent overfull lines
\providecommand{\tightlist}{%
  \setlength{\itemsep}{0pt}\setlength{\parskip}{0pt}}
\setcounter{secnumdepth}{0}

\author{Torbjorn Tannsjo}
\date{ \ }
\title{The Repugnant Conclusion}
% Redefines (sub)paragraphs to behave more like sections
\ifx\paragraph\undefined\else
\let\oldparagraph\paragraph
\renewcommand{\paragraph}[1]{\oldparagraph{#1}\mbox{}}
\fi
\ifx\subparagraph\undefined\else
\let\oldsubparagraph\subparagraph
\renewcommand{\subparagraph}[1]{\oldsubparagraph{#1}\mbox{}}
\fi

\begin{document}

You should have kids. Not because it's fun, or rewarding, or in your
evolutionary self-interest. You should have kids because it's your moral
duty to do so.

My argument is simple. Most people live lives that are, on net, happy.
For them to never exist, then, would be to deny them that happiness. And
because I think we have a moral duty to maximize the amount of happiness
in the world, that means that we all have an obligation to make the
world as populated as can be.

Of course, we should see to it that we do not overpopulate the planet in
a manner that threatens the future existence of mankind. But we're
nowhere near that point yet, at least not if we also see to it that we
solve pressing problems such as the one with global warming. In the mean
time, we're ethically obligated to make as many people as possible.

This idea, that having children is a moral obligation, is controversial,
so much so that it's known in philosophy as the ``repugnant
conclusion.'' But I don't think it's repugnant at all.

You might be thinking at this point, ``Sure, more happiness sounds good.
But morality is about helping people, and creating more people helps
`people' who don't exist, not yet anyway.'' This view is known as
actualism. Only actual individuals have rights. We have not done
anything wrong, unless there is an actual person who has a legitimate
complaint to make against our action.

This means that, if I do not create a happy individual, even if I can do
so, I do nothing wrong. A merely hypothetical individual has no
legitimate complaint to make. This is the great appeal of actualism: it
means that people have total freedom in choosing whether to reproduce or
not. My view suggests that we have a moral obligation to keep having
children; actualism let's people do as they like.

I can't help finding all this problematic. Imagine for a second that the
Genesis story is actually true. Under the actualist view, Adam and Eve
could have morally refrained from having children, even if, had they
decided differently, billions of billions of happy persons would have
been around!

Here is another consequence of the theory. Suppose I have a choice as to
whether to have a baby at 15 or at 35. If I have the baby at 15, I'll
earn much less money in my career, the baby will go to worse schools and
live in a worse neighborhood, and generally her life will be much
tougher. If I have her at 35, I'll be able to adequately provide for the
baby, pay for college, and so forth. If I have the baby at 15, then, did
I do anything wrong? I did not, by actualist reasoning. There is no one
there to complain about what I did. The baby is, after all, happy to be
around. By creating her, I did not violate her rights. And the
hypothetical baby I would've had at 35 isn't around to complain. But
this cannot be right. If these are the options I have, I ought to wait.
The world where I have a baby at 35 is just happier than the one where I
have a baby at 15.

The idea that people are morally obliged to have as many children as
possible has some radical implications. The biggest is that a world in
which many people---20, 50, even 100 billion---are alive, but each has a
life that's only barely worth living, is preferable to a world where
only, say, 10 billion people are extremely happy. Let's call these Big
Bad World and Small Happy World, respectively.

This conclusion may seem ludicrous. Of course you'd rather live in a
world where everyone's happy than one where people are just scraping by!
But this intuition is wrong.

Imagine that the end of Small Happy World is the end of humankind.
Everyone's as happy as can be, and then they all die. Meanwhile, in Big
Bad World, the human race continues on for billions of years, at a level
where life is worth living, but not spectacular. Would we not then feel
that the Small Happy World people are doing selfish? Rather than going
on with the human race, and accept the sacrifice that this means,
they're living high and not letting anyone succeed them. This is clearly
wrong.

Furthermore, it's difficult to get a grasp of what Big Bad World would
be like. But the way people live there may be similar to the way we
live. There are ups and downs in our lives. Perhaps a typical human life
often ends up with only a little happiness as its net sum. Perhaps many
lives end up with a negative sum. But then, is the Big Bad World so bad
as one may at first have thought? It's quite possible that people in Big
Bad World aren't living in abject poverty and misery, but instead have
lives similar those of many affluent people living in rich, developed
countries today.

Similarly, it's difficult to imagine what it would be like to live an
extremely happy life, containing much more happiness than our lives do
now. It could be that the gap between a barely-worthwhile life and the
happiest life possible is quite small.

We have an obligation to go on with humanity, as long as we can, and as
long as we create future individuals who live lives worth living.
Procreative decisions are moral decisions, and we ought to see to it
that, by our procreative decisions, we maximize the sum total of
happiness. The popular idea that we may do as we see fit when we
conceive children, as long as there is no one there who can make a
legitimate complaint against us, is mistaken.

We ought to take all easy measures to procreate, such as signing up for
sperm banks, having another child when we can take care of it, and so
forth. Of course, we should see to it that we do not by our procreative
choices make existing lives worth not living nor make lives worth not
living. In the individual case, it is hard to know where to draw the
line. But in many cases, having more kids is clearly better.

\end{document}
