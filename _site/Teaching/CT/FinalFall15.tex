\documentclass[answers]{exam}
\newcommand{\class}{Critical Thinking}
\newcommand{\examnum}{Final Exam}
\newcommand{\term}{Fall, 2015 }
\newcommand{\examdate}{Take home. Due 12/18/2015}
\setlength\answerlinelength{3in}
\begin{document}
\pagestyle{head}
\firstpageheader{}{}{}
\runningheader{\class}{\examnum\ - Page \thepage\ of \numpages}{\examdate}
\runningheadrule

\begin{flushright}
\begin{tabular}{p{2.8in} r l}
\textbf{\class} & \textbf{Name (Print):} & \makebox[2in]{\hrulefill}\\
\textbf{\term} &&\\
\textbf{\examnum} &&\\
\textbf{\examdate} &&\\
\end{tabular}\\
\end{flushright}
\rule[1ex]{\textwidth}{.1pt}

\begin{minipage}[t]{3.7in}
\vspace{0pt}
\begin{itemize}
\item Enter all requested information on the top of this page.
\item Return the exam at \textbf{11:00 am} in room G139 on 12/18
\item \textbf{Under no circumstances will late exams be accepted.  If you ask for an extension, the answer will be an automatic no.} 
\item Do not write in the table to the right.
\item Write clearly. Poor handwriting may lead to loss in points.
\item You must complete this exam by yourself. I will check exam scripts for overlap. The penalty for plagiarism is 0 for the assignment. 
\end{itemize}

\rule[1ex]{\textwidth}{.1pt}

\end{minipage}
\hfill
\begin{minipage}[t]{2.3in}
\vspace{0pt}
%\cellwidth{3em}
\gradetablestretch{2}
\vqword{Problem}
\addpoints % required here by exam.cls, even though questions haven't started yet.	
\gradetable[v]%[pages]  % Use [pages] to have grading table by page instead of question

\end{minipage}


\begin{questions}
\addpoints
\section{Multiple Choice Questions}

\question Circle the correct answer. 
\begin{parts}

\part[1] Which of the following sentence is not a statement? 
\begin{choices}
\choice The guest speakers were lame
\choice Two plus two equals four
\correctchoice  Determine the quality of your beliefs
\choice Sexual harassment should be a crime
\end{choices}

\part[1] This sentence---``Don’t believe anything the Potter says''---is…
\begin{choices}
\choice  A statement
\correctchoice Not a statement
\choice An argument
\choice An explanation
\end{choices}


\part[1] A deductively valid argument cannot have
\begin{choices}
\correctchoice  True premises and a false conclusion 	
\choice  False premises and a false conclusion	
\choice False premises and a true conclusion	
\choice True premises and a true conclusion	
\end{choices}




\part[1] Modus ponens has this argument pattern
\begin{choices}
\choice If p, then q. q. Therefore, p.	
\choice If p, then q. If q, then r. Therefore, if p, then r.	
\choice Either p or q. Not p. Therefore, q.	
\correctchoice If p, then q. p. Therefore, q. 	
\end{choices}

\part[1] The invalid argument form known as affirming the consequent has this pattern:	
\begin{choices}
\choice If p, then q. p. Therefore, q.	
\correctchoice If p, then q. q. Therefore, p.	
\choice If p, then q. Not p. Therefore, not q. 	
\choice If p, then q. If q, then r. Therefore, if p, then r.	
\end{choices}

\part[1] An argument with this form, ``If p, then q. Not q. Therefore, not p'' is known as
\begin{choices}
\correctchoice Modus tollens	
\choice Hypothetical syllogism 	
\choice Modus ponens	
\choice Disjunctive syllogism	
\end{choices}

\part[1] An argument with this form, ``Either p or q. Not p. Therefore, q'' is known as
\begin{choices}
\correctchoice Disjunctive syllogism 	
\choice Hypothetical syllogism	
\choice Modus tollens	
\choice Dual syllogism	
\end{choices}

\part[1] This argument---``If Einstein invented the steam engine, then he's a great scientist. Einstein did not invent the steam engine. Therefore, he is not a great scientist''---is an example of…
\begin{choices}
\choice Affirming the consequent
\choice Affirming the antecedent
\correctchoice Denying the antecedent
\choice Denying the consequent
\end{choices}

\part[1] The truth-table test of validity is based on the fact that it is impossible for a valid argument to have true premises and…
\begin{choices}
\choice A true conclusion	
\choice A negated conclusion	
\choice A conditional	
\correctchoice A false conclusion	
\end{choices}


\part[1] The invalid argument form known as denying the antecedent has this pattern:	
\begin{choices}
\choice If p, then q. p. Therefore, q.	
\choice If p, then q. q. Therefore, p.	
\correctchoice If p, then q. Not p. Therefore, not q. 	
\choice If p, then q. If q, then r. Therefore, if p, then r.	
\end{choices}


\part[1] The fallacy of reasoning that just because B followed A, A must have caused B is known as…
\begin{choices}
\choice Hasty generalization
\choice Faulty analogy
\choice Representative fallacy
\correctchoice Post hoc, ergo prompter hoc
\end{choices}

\part[1] To reason that because two things have some similarities they must be similar in yet another way is to use…
\begin{choices}
\choice Correlative reasoning
\choice Enumerative induction
\correctchoice Analogical induction
\choice Deductive logic
\end{choices} 

\part[1] We're guilty of hasty generalization whenever we draw a conclusion about a target group based on…
\begin{choices}
\choice An irrelevant property
\correctchoice Inadequate sample size
\choice Enumerative induction
\choice An opinion poll
\end{choices}

\part[1] In enumerative induction, the whole collection of individuals being examined is called the…
\begin{choices}
\choice Sample
\correctchoice Target group
\choice Relevant property
\choice Control group
\end{choices} 

\part[1] Mill's method of correlation says that when two events are correlated, they are…
\begin{choices}
\choice Not causally related
\choice Uncaused
\correctchoice Probably causally related
\choice Unrelated
\end{choices}

\part[1] A hypothesis that cannot be verified independently of the phenomenon it is supposed to explain is said to be…
\begin{choices}
\choice Simple
\choice Ad hoc
\choice Fruitful
\choice Incoherent
\end{choices}

\part[1] Scope refers to…
\begin{choices}
\choice The number of assumptions made
\choice The rules of consistency
\choice How well a theory fits with existing knowledge
\correctchoice The amount of diverse phenomena explained
\end{choices}

\part[1] Fruitfulness refers to…
\begin{choices}
\choice How well a theory fits with existing knowledge
\correctchoice The number of novel predictions made
\choice The amount of diverse phenomena explained
\choice The number of assumptions made
\end{choices}

\end{parts}



\section{Fallacies}

\question For each of the following arguments, name the fallacy committed:
 
\begin{parts}

\part[1] ``Either you support the war or you are a traitor to your country. You don’t support the war. So you’re a traitor.''
\answerline

\part[1] ``The Bible says that God exists. The Bible is true because God wrote it. Therefore, God exists.''
\answerline


\part[1] ``The political action committee is very prestigious in Washington; we can expect, then, that each of its members is very prestigious in Washington.''
\answerline


\part[1] ``No one has shown that ghosts aren’t real, so they must be real.''
\answerline

\part[1] ``Sure, the media claims that Senator Bedfellow was taking kickbacks. But we all know about the media's credibility, don't we."  \answerline

\part[1] ``Senator Jones says that we should not fund the attack submarine program. I disagree entirely. I can't understand why he wants to leave us defenseless like that.''
\answerline

\part[1] Interviewer: ``Your resume looks impressive but I need another reference.'' \\
Bill: ``Jill can give me a good reference.'' \\
Interviewer: ``Good. But how do I know that Jill is trustworthy?''\\
Bill: ``Certainly. I can vouch for her.''
\answerline

\part[1] Senator Jill: ``We'll have to cut education funding this year.'' \\
Senator Bill: ``Why?'' \\
Senator Jill: ``Well, either we cut the social programs or we live with a huge deficit and we can't live with the deficit.''
\answerline

\part[1] The new UltraSkinny diet will make you feel great. No longer be troubled by your weight. Enjoy the admiring stares of the opposite sex. Revel in your new freedom from fat. You will know true happiness if you try our diet! \answerline

\part[1] Lynch says that we should spend more state revenue on education. But Lynch is a professor who wants a better salary -- so you know that his opinion is worthless. \answerline

\part[1]``Television entertainer Sam Lockhart argues that religion is just a lot of foolish nonsense. But Lockhart is an arrogant, shameless, self-righteous pig. Obviously his arguments are not worth listening to.''
\answerline

\part[1] ``Mr. Lemon has argued against prayer in the public schools. Obviously Mr. Lemon advocated atheism. But atheism is what they used to have in Russia. Atheism leads to the suppression of all religions and the replacement of God by an omnipotent state. Is that what we want for this country? I hardly think so. Clearly Mr. Lemon's argument is nonsense.''
\answerline

\part[1]``Immediate steps should be taken to outlaw pornography. The continued manufacture and sale of pornography will almost certainly lead to an increase in sex-related crimes such as rape and incest. This in turn will gradually erode the moral fabric of society and result in an increase in crimes of all sorts. Eventually a complete disintegration of law and order will occur leading in the end to the total collapse of civilization.''
\answerline


\part[1] ``Clearly, terminally ill patients have a right to doctor-assisted suicides. After all, many of these people are unable to commit suicide by themselves.''
\answerline

\part[1] ``Either you let me attend the Arcade Fire concert or I'll be miserable for the rest of my life. I know you don't want me to be miserable for the rest of my life, so it follows that you will let me attend the concert.''
\answerline

\end{parts}


\section{Truth-Tables}

\question For each of the following arguments, do the following:
\begin{enumerate}
\item Compose a truth table (2 points).
\item State whether the argument is valid or invalid. (1 point)
\end{enumerate}
\begin{parts}
\part[3] 
\begin{enumerate} 
\item[ ]
\item[P1.] $a \lor (b \& c)$
\item[P2.] $b \& \neg a$
\item[C.] $a \lor c$
\end{enumerate}
\part[3] 
\begin{enumerate}
\item[ ] 
\item[P1.] $a \rightarrow \neg b$
\item[P2.] $a \rightarrow c$
\item[C.] $c \rightarrow \neg b$
\end{enumerate}
\part[3] \begin{enumerate}
\item[ ] 
\item[P1.] $(a \lor c) \rightarrow (a \& b)$
\item[P2.] $\neg (a \& b)$
\item[C.] $ \neg (a \lor c) $
\end{enumerate}
\part[3]
\begin{enumerate}
\item[ ] 
\item[P1.] $a \rightarrow \neg c$
\item[P2.] $b \lor c$
\item[P3.] $\neg c$
\item[C.] $a \& b$
\end{enumerate}
\part[3] \begin{enumerate}
\item[ ] 
\item[P1.] $ a \lor \neg (b \& c)$
\item[P2.] $ \neg(a \& c)$
\item[P3.] $ \neg (\neg b \rightarrow a) $
\item[C.] $ \neg C $
\end{enumerate}
\end{parts}

\section{Deduction: Integrative Exercises}

\question For each of the following arguments, do the following:  
\begin{enumerate}
\item Translate the premises and conclusion into symbols (1 point). 
\item Construct a truth-table for the argument. (1 points)
\item State whether the argument is valid or invalid. (1 point)
\end{enumerate} 

\begin{parts}
\part[3] If numbers were symbols, then there would be just as many numbers as there are symbols. However, there are only a finite number of symbols and an infinite number of numbers. Hence, numbers are not mere symbols.


\part[3] If God were wholly good and he were omnipotent, then he would intervene in human affairs to prevent needless suffering. God never intervenes to prevent such suffering. God is omnipotent. Thus we can conclude that he is not wholly good. 
  
  
\part[3] If students were environmentally aware, they would object if any species of animal were endangered.  The well known Greenwood white squirrel is endangered. Yet, no student objected. Therefore, students are not environmentally aware.   

\end{parts}


\section{Induction}

\question Each of the following passage contains an inductive argument. For each argument, answer the following questions:
\begin{enumerate}
\item What is the conclusion? What are the premises? (1 points)
\item What kind of inductive argument is contained in the passage, e.g, an enumerative induction, an analogical induction, or a causal induction? (1 points)
\item Is the argument strong or weak? Using the criteria for assessing the relevant type of inductive argument, give reasons for your answers. For instance, you might decide that a passage contains an enumerative induction and that the relevant sample size is neither large enough or representative. (2 points)
\end{enumerate}

\begin{parts}
\part[4] For years vehicular accidents at the intersection of Fifth and Main Streets have consistently averaged two to four per month. Since a traffic light was installed there, the rate has been one or two accidents every  three months. That new traffic light has made quite a difference. 
\part[4] Two hundred samples of water taken from many sites all along the Charles River show unsafe concentrations of toxic chemicals. Obviously the water in the Charles River is unsafe.  

\part[4] University fraternities are magnets for all sorts of illegal activity. Last year several frat brothers were arrested at a frat-house party. And this year a fraternity was actually kicked off campus for violating underage drinking laws.  


\part[4] If a single cell, under appropriate conditions, becomes a person in the space of a few years, there can surely be no difficulty in understanding how, under appropriate conditions, a cell may, in the course of untold millions of years, give origin to the human race.

\part[4] Eighty-three percent of the letters to the editor received by this newspaper are adamantly pro-life. And since the `Daily Planet' is the only major newspaper in the city, and it provides the primary forum for discussion of local issues, we must conclude that this town is also overwhelmingly pro-life.

\part[4] Television is destroying morality in this country. As TV violence, sex, and vulgarity have increased, so have violent crimes, sexual assaults, and violations of obscenity laws. 

\part[4] Almost all of the owners of restaurants, bars, and clubs in New York City are opposed to the city's total ban on smoking in indoor public places. The vast majority of New Yorkers simply do not like this law. 

\end{parts}

\section{Inference to the Best Explanation}
\question For each of the following passages, do the following:
\begin{enumerate}
\item State the claim to be evaluated. (1 point)
\item Indicate what phenomenon is being explained. (1 point)
\item Specify one alternative theory. (1 point)
\item Use the criteria of adequacy to assess the two theories and determine which is more plausible. Write a paragraph detailing your reasons for your choice. (2 points)
\end{enumerate}

\begin{parts}

\part[5] A religious sect based in Boston predicts that the end of the world will occur on January 1, 2001. The world, of course, did not end then. The leader of the sect explains that the prophesy failed to come true because members of the sect did not have enough faith in it. 

\part[5] Scientists studied twenty terminal cancer patients, taking note of the overall mental attitudes of the patients. Some of them were characterized as having negative attitudes; they were often angry at their situation and experienced feelings of hopelessness and regret. But other patients were thought to have positive attitudes; they were generally upbeat, optimistic about their treatment, and hopeful. Most of those with positive attitudes lived longer than most of those with negative attitudes. A positive attitude can lengthen cancer patients' lives. 


\part[5] The primary cause of all wars is fear. When people are afraid of others---because of ignorance or perceived threats---they naturally respond with belligerence and acts of violence. If they have no fear of others, they tend to react rationally and calmly and to seek some sort of fair accommodation. 
\end{parts}

\end{questions}



\end{document}