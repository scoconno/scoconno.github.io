\documentclass[article,oneside]{memoir}

%%% custom style file with standard settings for xelatex and biblatex. Note that when [minion] is present, we assume you have minion pro installed for use with pdflatex.
%\usepackage[minion]{org-preamble-pdflatex} 

%%% alternatively, use xelatex instead
\usepackage{org-preamble-xelatex} 



\def\myauthor{Author}
\def\mytitle{Title}
\def\mycopyright{\myauthor}
\def\mykeywords{}
\def\mybibliostyle{plain}
\def\mybibliocommand{}
\def\mysubtitle{}
\def\myaffiliation{NJCU}
\def\myaddress{102 Phil}
\def\myemail{soconnor@njcu.edu}
\def\myweb{\href{http://scoconno.github.io/Teaching/CT/}{http://scoconno.github.io/Teaching/CT/}}
\def\myphone{}
\def\myversion{}
\def\myrevision{}
\def\myaffiliation{NJCU}
\def\myauthor{Dr. Scott O'Connor}
\def\mykeywords{}
\def\mysubtitle{Syllabus}
\def\mytitle{{\normalsize Phil 102 (1725), 3 Credits, Fall 2015, M\&W @ 12:45 pm, G 139 \newline} \HUGE Critical Thinking}


\begin{document}

%%% If using xelatex and not pdflatex
%%% xelatex font choices
\defaultfontfeatures{}
\defaultfontfeatures{Scale=MatchLowercase}    
% You will need to buy these fonts, change the names to fonts you own, or comment out if not using xelatex.      
\setromanfont[Mapping=tex-text]{Georgia} 
\setsansfont[Mapping=tex-text]{Georgia} 
\setmonofont[Mapping=tex-text,Scale=0.8]{Georgia} 

%% blank label items; hanging bibs for text
%% Custom hanging indent for vita items
\def\ind{\hangindent=1 true cm\hangafter=1 \noindent}
\def\labelitemi{$\cdot$}
%\renewcommand{\labelitemii}{~}

%% RCS info string for version tracking
\chapterstyle{article-3}  % alternative styles are defined in latex-custom-kjh/needs-memoir/
\pagestyle{kjh}

\title{\LARGE \mytitle}     
\author{\Large\myauthor \newline \footnotesize\texttt{\noindent\myweb}}
\date{09/02/2015--12/18/2015}

\published{\,}

\maketitle

% \thispagestyle{kjhgit}

% Copyright Page
%\textcopyright{} \mycopyright


%
% Main Content
%

\section{Copyright}
The materials used in this class, including, but not limited to, lectures, exams, quizzes, and homework assignments are copyright protected works.  Any unauthorized copying of the class materials or recording of lectures is a violation of federal law and may result in disciplinary actions being taken against the student.  Additionally, the sharing of class materials without the specific, express approval of the instructor may be a violation of the University's Student Honor Code and an act of academic dishonesty, which could result in further disciplinary action.  This includes, among other things, uploading class materials to websites for the purpose of sharing those materials with other current or future students.  

\section{Course Description}

In this course, we will be learning how how to identify, analyze, and
evaluate arguments. We will begin by discussing what critical thinking
is and why we should improve our ability to do it. Since critical
thinking concerns arguments, the bulk of the course will focus on
reconstructing arguments and learning how to diagnose them. Topics to be covered include argument patterns, argument types, fallacies,
introductory formal logic, probabilistic reasoning, and inference to the best explanation.


\section{Learning Objectives}

Upon completing this course, students will be able to (i) identify the
difference between deductive and inductive arguments, (ii) assess the
former for validity and soundness, the latter for strength, (iii)
identify argument patterns, (iv) diagnose their own thoughts, (v) reason
better, (vi) manage their studies in a responsible and timely manner.

\section{Required Textbook}

From 09/09/2015, students \textbf{must bring the following textbook to class}:

\begin{itemize}
\item
  \href{http://www.amazon.com/Power-Critical-Thinking-Effective-Extraordinary/dp/0199856672/ref=sr_1_1?s=books\&ie=UTF8\&qid=1421936130\&sr=1-1\&keywords=critical+thinking+vaughn}{`The
  Power of Critical Thinking: Effective Reasoning about Ordinary and
  Extraordinary Claims', 4th edition, by Lewis Vaughn} (Available from
  the campus book store as well as online retailers)
\item (Note also the useful companion website for the book: \\  \href{http://global.oup.com/us/companion.websites/9780199856671/student/}{http://global.oup.com/us/companion.websites/9780199856671/student/})
\end{itemize}

\section{Course Website}
There is both a Blackboard site and website for this course (link on first page). Clicking the first link on the left panel within the Blackboard site will bring you to the course website. All assignments will be submitted through Blackboard. Readings, notes, etc. will be posted on the course website. Note that Blackboard difficulties are rare and automatically reported to instructors. Under no circumstance will a student's report of a Blackboard difficulty be reason for an extension. It is your responsibility to contact Blackboard support for help.

\section{Requirements}

\begin{itemize}
\item \textit{Workload:} Expect to spend an average of 5--6 hours per week  completing the readings and assignments.


\item \textit{6 Problem Sets} administered through Blackboard. First problem set due in Week 3. Begin problem sets once they are available and modify/complete over the availability period. 

\item \textit{2 Exams:} There is an open-book in-class mid-term and cumulative take-home final.  

\item \textit{Attendance:} Roll call will be taken. 0.5 point will be awarded per class up to a maximum of 10 points. Points will not be awarded during weeks 1 \& 2. 
 

\item \textit{Grade Distribution:} 6 Problem sets---10 points each (60 total); Mid-term---20; Final---20; Attendance--0.5 point per class (10 total).

\item \textit{Grade Breakdown:}

 \begin{tabular}{ | l | l | p{2cm} | l | l | }
    \hline 
96--\textbf{110} & A  & &  77--79 &  C+ \\  
90--95 & A- & &  73--76 & C \\
87-89 & B+ &  &  70--72 & C- \\ 
83--86 & B  & &  60--69 & D\\
80--82 & B - & & 0--59 & F\\ \hline
    \end{tabular}


\end{itemize}




\section{Policies}

\begin{itemize}

\item \textbf{Student Responsibility:} This syllabus outlines the required text, assignments, requirements, and policies for this course. By taking this course, you agree to read this syllabus and be bound by those requirements and policies. 

 \item \textit{Academic Integrity:} All the work you turn in (including papers, drafts, and discussion board posts) must be written by you specifically for this course. It must originate with you in form and content with all contributory sources fully and specifically acknowledged. Being a student at NJCU requires you to follow \href{http://www.njcu.edu/uploadedFiles/About_NJCU/Governance_and_Organization/University_Senate/Policies/Academic\%20INTEGRITY\%20POLICY\%20FINAL\%202-04.pdf}{NJCU's Academic Integrity Policy.} Penalties for violations are as follows: 1st infraction will result in a 0 for the assignment.  2nd infraction will result in a 0 for the entire course \& application for permanent record on student's transcript. (Repeated violations can lead to expulsion from NJCU). 

\item \textit{Attendance:} You are considered absent if you are (i) not present during roll call, (ii) leave early, (iii) leave without permission, or (iv) leave for an extended period of time. No excuses. No exceptions.







\item \textit{Communication:} To comply with Federal Privacy Laws (FERPA) and NJCU policies, all communication will be through Blackboard and/or official NJCU e-mail. Check both your NJCU e-mail and Blackboard daily. For further information see \href{http://scoconno.github.io/Contact/}{http://scoconno.github.io/Contact/}.

\item \textit{Conduct:} Distracting and disrespectful behaviors, including but not limited to eating, leaving your seat, talking out of turn, and aggression are prohibited. Penalties include, but are not limited to, a loss of attendance points for the day of violation. Repeat offenders will be reported to the Dean of Students. 

\item \textit{Electronic devices:} Use of electronic device, including, but not limited, to smartphones, dictaphones, tablets, and laptops, is prohibited. Recording a lecture is in violation of Copyright. Penalties include, but are not limited to, a loss of attendance points for the day of violation. Repeat offenders will be reported to the Dean of Students.


\item \textit{Grading:} Grades will be available within 1 week of an assignment being submitted. See: \href{http://scoconno.github.io/Teaching/Grading}{http://scoconno.github.io/Teaching/Grading} for further information.


\item \textit{Late work \& Make-up Policy:} See the assignment schedule below. No make-ups or late work accepted under any circumstances. No exceptions. But note that there are 110 points available with 96+ being required for an A.


\item \textit{Statement for students with disabilities:} If you are a student with a disability and wish to receive consideration for reasonable accommodations, please register with the Office of Specialized Services and Supplemental Instruction (OSS/SI). To begin this process, complete the registration form available on the OSS/SI website at
\href{http://www.njcu.edu/Specialized_Services.aspx}{www.njcu.edu/Specialized\_Services.aspx}
(listed under Student Resources-Forms). Contact OSS/SI at 201-200-2091
or visit the office in Karnoutsos Hall, Room 102 for additional
information.

\end{itemize}



\section{Weekly Course Schedule}
Dates refer to the first day of the week. Complete the readings before the first class of the week. Readings marked with a `**' can be found on the course website. Handouts can be found under the relevant modules on the course website. All other listed readings can be found in the required textbook. Changes to the syllabus will be announced in class and \emph{via} your NJCU email address.

\begin{enumerate}
\item \textit{08/31} Introduction,  no class on Mon.
\item \textit{09/07} The Basics,  ch.1., no class on Mon.
\item \textit{09/14} Continued, ch.1 
\item \textit{09/21} Arguments Introduced, ch.3, (especially pp.62--78)
\item \textit{09/28} Judging Arguments, ch.3, (especially pp.78--91)
\item \textit{10/05} Deduction: Translations, ch.6, (especially pp.209--223)
\item \textit{10/12} Deduction: Validity ch.6, (especially pp.223--242)
\item \textit{10/19} Review
\item \textit{10/26} Mid-term (Mon), Induction (Wed) ch.8
\item \textit{11/02} Enumerative Induction, ch.8, (especially pp.276--294)
\item \textit{11/09} Analogical and Causal Arguments, ch.8, (especially pp.294--330), \textbf{no class 11/11}
\item \textit{11/16} Fallacies and Persuaders, ch.5
 \item \textit{11/23} Continued, ch.5
 \item \textit{11/30} Inference to the Best Explanation, ch.9
\item \textit{12/07} Continued,  ch.9
\item \textit{12/14} Exam Review, Final exam (take-home) distributed, (no class on Wed)
\end{enumerate}


\section{Assignment \& Exam Schedule}
Dates refer to the due date. All assignments must be submitted through Blackboard by 1:00pm. Problem sets are designed to be worked on throughout the module they cover. Do not start late. No late work accepted. No make-ups. No exceptions. 

\begin{enumerate}
\item \textit{09/18/2015,} Problem Set 1 (covers ch.1)
\item \textit{10/02/2015,} Problem Set 2 (covers ch.3)
\item \textit{10/16/2015,} Problem Set 3 (covers ch.6)
\item \textit{10/26/2015,} Mid-term 
\item \textit{11/13/2015,} Problem Set 4 (covers ch.8)
\item \textit{11/29/2015,} Problem Set 5 (covers ch.5)
\item \textit{12/11/2015,} Problem Set 6 (covers ch.9)
\item \textit{12/18/2015.} Final exam due, submit hard copy between 11:00--13:00 in G 139 
\end{enumerate}


%% Uncomment if you want a printed bibliography.
%\printbibliography 

\end{document}