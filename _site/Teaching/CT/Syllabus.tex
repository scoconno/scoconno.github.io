\documentclass[]{article}
\usepackage{lmodern}
\usepackage{amssymb,amsmath}
\usepackage{ifxetex,ifluatex}
\usepackage{fixltx2e} % provides \textsubscript
\ifnum 0\ifxetex 1\fi\ifluatex 1\fi=0 % if pdftex
  \usepackage[T1]{fontenc}
  \usepackage[utf8]{inputenc}
\else % if luatex or xelatex
  \ifxetex
    \usepackage{mathspec}
    \usepackage{xltxtra,xunicode}
  \else
    \usepackage{fontspec}
  \fi
  \defaultfontfeatures{Mapping=tex-text,Scale=MatchLowercase}
  \newcommand{\euro}{€}
\fi
% use upquote if available, for straight quotes in verbatim environments
\IfFileExists{upquote.sty}{\usepackage{upquote}}{}
% use microtype if available
\IfFileExists{microtype.sty}{%
\usepackage{microtype}
\UseMicrotypeSet[protrusion]{basicmath} % disable protrusion for tt fonts
}{}
\ifxetex
  \usepackage[setpagesize=false, % page size defined by xetex
              unicode=false, % unicode breaks when used with xetex
              xetex]{hyperref}
\else
  \usepackage[unicode=true]{hyperref}
\fi
\hypersetup{breaklinks=true,
            bookmarks=true,
            pdfauthor={},
            pdftitle={Syllabus},
            colorlinks=true,
            citecolor=blue,
            urlcolor=blue,
            linkcolor=magenta,
            pdfborder={0 0 0}}
\urlstyle{same}  % don't use monospace font for urls
\setlength{\parindent}{0pt}
\setlength{\parskip}{6pt plus 2pt minus 1pt}
\setlength{\emergencystretch}{3em}  % prevent overfull lines
\setcounter{secnumdepth}{0}

\title{Syllabus}
\date{}

\begin{document}
\maketitle

\subsubsection{Course Information}\label{course-information}

\textbf{Title:} Critical Thinking

\textbf{Credits:} 3

\textbf{Prerequisites:} None

\textbf{Corequisites:} None

\subsubsection{Course Description}\label{course-description}

In this course, we will be learning how how to identify, analyze, and
evaluate arguments. We will begin by discussing what critical thinking
is and why we should improve our ability to do it. Since critical
thinking concerns arguments, the bulk of the course will focus on
reconstructing arguments and learning how to diagnose them. Topics to be
covered include argument patterns, argument types, fallacies,
introductory formal logic, probabilistic reasoning, and reasoning in the
law.

\subsubsection{Learning Objectives}\label{learning-objectives}

Upon completing this course, students will be able to (i) identify the
difference between deductive and inductive arguments, (ii) assess the
former for validity and soundness, the latter for strength, (iii)
identify argument patterns, (iv) diagnoze their own thoughts, (v) reason
better.

\subsubsection{Texts/Media}\label{textsmedia}

The following are required for this course:

\begin{itemize}
\itemsep1pt\parskip0pt\parsep0pt
\item
  \href{http://www.amazon.com/Power-Critical-Thinking-Effective-Extraordinary/dp/0199856672/ref=sr_1_1?s=books\&ie=UTF8\&qid=1421936130\&sr=1-1\&keywords=critical+thinking+vaughn}{`The
  Power of Critical Thinking: Effective Reasoning about Ordinary and
  Extraordinary Claims', 4th edition, by Lewis Vaughn} (Available from
  the campus book store as well as online retailers)
\item
  \href{http://global.oup.com/us/companion.websites/9780199856671/student/}{Book's
  Companion website}
\item
  Online problem sets as well as grades are posted on the Blackboard
  site.
\end{itemize}

\subsubsection{Course Requirements}\label{course-requirements}

\textbf{Workload:} Successful students will read materials both before
and again after the classes in which they are discussed, look through
and revise their notes, think about and not just read the materials,
etc.

\textbf{Attendance:} You can earn up to 20 points towards you final
grade by attending regularly. Please note that no points are earned
during an exam day. You are considered absent if any of the following
apply:

\begin{enumerate}
\def\labelenumi{\arabic{enumi}.}
\itemsep1pt\parskip0pt\parsep0pt
\item
  You arrive arrive more than 10 minutes late.
\item
  You leave more than 10 minutes early.
\item
  You leave class for longer than 5 minutes or leave class more than
  once.
\end{enumerate}

\textbf{Problem Sets:} There will be weekly online problem sets
administered through Blackboard starting in week 2. It is your
responsibility to use a charged device on a stable internet connection.
The problem sets will be posted on Mon. night by 11:59 PM. You must
submit your answers by Sun. at 11:59pm. Please note that Blackboard does
not allow me offer to selectively extend the deadline for any student.

\textbf{Exams:} There is both a mid-term and final for this course. If
you have questions, concerns, or would like detailed comments on a
returned exam script, I am happy to give them in person.

\textbf{Late work \& Make-up Policy:} Since the grading scheme allows
you some freedom to choose which assignments to do, I will not allow any
exam make-ups without proper certification. This is a strict policy.
Please do not ask for special treatment.

\textbf{Grade Distribution:}


\begin{itemize}
\item 10 Weekly Problem Sets, 3 points each, 30 points total. 
\item Mid term, 25 points
\item Final, 35 points
\item Attendance, 1 point for each class attended. Begins week 2.  20 total.
\end{itemize}


\textbf{Grading Scheme}


\begin{tabular}{|r|l|}
  \hline
  96+ & A- \\ \hline
  90--95 & A- \\ \hline
  87--89 & B+ \\ \hline
  83--86 & B \\ \hline
  80--82 & B- \\ \hline
  77--79 & C+ \\ \hline
  73--76 & C \\ \hline
  70--72 & C- \\ \hline
  60--69 & D \\ \hline
  0--59 & F \\ \hline
  \end{tabular}


\textbf{Communication:} I will normally respond to emails within two
days of receiving them. This will normally be during business hours on
Mon. through Fri. Please check your NJCU email address.

\textbf{Grading Schedule:} Grades will be available within 2 weeks of an
assignment being submitted.

\textbf{Academic Integrity:} Please familiarize yourself with
\href{http://www.njcu.edu/uploadedFiles/About_NJCU/Governance_and_Organization/University_Senate/Policies/Academic\%20INTEGRITY\%20POLICY\%20FINAL\%202-04.pdf}{NJCU's
Academic Integrity Policy}

\textbf{Attendance:} You are considered absent if any of the following
apply:

\begin{enumerate}
\def\labelenumi{\arabic{enumi}.}
\itemsep1pt\parskip0pt\parsep0pt
\item
  You arrive arrive more than 10 minutes late.
\item
  You leave more than 10 minutes early.
\item
  You leave class for longer than 5 minutes or leave class more than
  once.
\end{enumerate}

\textbf{Conduct:} Please avoid any behavior that would distract your
peers. This includes eating, leaving your seat, talking out of turn,
etc.

\textbf{Demeanor:} You will be developing your speaking and debating
skills through interaction with both me and your peers. You will be
cold-called and often challenged to both clarify and defend your views.
Please be cooperative and friendly.

\textbf{Technology/Mindfulness:} You will also develop your abilities to
attend carefully to new and abstract information. Using technology can
be a distraction both to yourself and to your peers. So, unless you are
given explicit permission otherwise, please turn off laptops, phones,
and so on.

\textbf{Statement for students with disabilities:} If you are a student
with a disability and wish to receive consideration for reasonable
accommodations, please register with the Office of Specialized Services
and Supplemental Instruction (OSS/SI). To begin this process, complete
the registration form available on the OSS/SI website at
\href{http://www.njcu.edu/Specialized_Services.aspx}{www.njcu.edu/Specialized\_Services.aspx}
(listed under Student Resources-Forms). Contact OSS/SI at 201-200-2091
or visit the office in Karnoutsos Hall, Room 102 for additional
information.

\subsubsection{Tentative Schedule (changes will be announced in
class)}\label{tentative-schedule-changes-will-be-announced-in-class}

Listed here are the readings and (non-exhaustive) topics we will be
discussing during each session. Although assigned readings are often
short, they are difficult, and multiple readings are strongly advised.

\begin{enumerate}

\end{enumerate}

\end{document}
