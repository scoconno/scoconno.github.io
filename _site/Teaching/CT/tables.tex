\documentclass[]{article}
\usepackage{amssymb,amsmath}
\usepackage{ifxetex,ifluatex}
\usepackage{fixltx2e} % provides \textsubscript
\ifnum 0\ifxetex 1\fi\ifluatex 1\fi=0 % if pdftex
  \usepackage[T1]{fontenc}
  \usepackage[utf8]{inputenc}
\else % if luatex or xelatex
  \ifxetex
    \usepackage{mathspec}
    \usepackage{xltxtra,xunicode}
  \else
    \usepackage{fontspec}
  \fi
  \defaultfontfeatures{Mapping=tex-text,Scale=MatchLowercase}
  \newcommand{\euro}{€}
\fi
% use upquote if available, for straight quotes in verbatim environments
\IfFileExists{upquote.sty}{\usepackage{upquote}}{}
% use microtype if available
\IfFileExists{microtype.sty}{%
\usepackage{microtype}
\UseMicrotypeSet[protrusion]{basicmath} % disable protrusion for tt fonts
}{}
\ifxetex
  \usepackage[setpagesize=false, % page size defined by xetex
              unicode=false, % unicode breaks when used with xetex
              xetex]{hyperref}
\else
  \usepackage[unicode=true]{hyperref}
\fi
\hypersetup{breaklinks=true,
            bookmarks=true,
            pdfauthor={},
            pdftitle={},
            colorlinks=true,
            citecolor=blue,
            urlcolor=blue,
            linkcolor=magenta,
            pdfborder={0 0 0}}
\urlstyle{same}  % don't use monospace font for urls
\setlength{\parindent}{0pt}
\setlength{\parskip}{6pt plus 2pt minus 1pt}
\setlength{\emergencystretch}{3em}  % prevent overfull lines
\setcounter{secnumdepth}{0}

\date{}

\begin{document}

\subsection{Our goal}\label{our-goal}

We need to determine if an argument like the one below is a good one.
It's deductive. So, it's good if it is both valid and sound. Our focus
at the moment is validity:

\begin{verbatim}
Bill: "God must exist." 
Jill: "How do you know." 
Bill: "Well if the Bible is accurate, then God exists." 
Jill: "Why should I believe the Bible?" 
Bill: "Well, if God exists, then it has got to be accurate!"
\end{verbatim}

\subsection{Validity}\label{validity}

An argument is valid just if it is impossible for the conclusion to be
false and \textbf{all} of the premises to be true.

\subsection{Invalidity}\label{invalidity}

An argument is invalid just if it is possible for the conclusion to be
false and \emph{all} of the premises to be true.

\subsection{Truth Tables}\label{truth-tables}

Truth tables allow us determine if it is possible for the conclusion to
be false and, at the same time, the premises to be true.

Strategy: construct a large truth table with smaller truth tables as
their parts. One part is a truth-table for the conclusion. There is also
one truth-table for each of the premises. After constructing the large
truth-table, we will see if any line has a false conclusion and true
premises.

\subsection{Example}\label{example}

\begin{verbatim}
Bill: "God must exist." 
Jill: "How do you know." 
Bill: "Well if the Bible is accurate, then God exists." 
Jill: "Why should I believe the Bible?" 
Bill: "Well, if God exists, then it has got to be accurate!"
\end{verbatim}

`P': God exists

`Q': The Bible is accurate

\begin{itemize}
\item
  P1: If God exists, then the Bible is accurate. (p--\textgreater{}q)
\item
  P2: If the Bible is accurate, then God exists. (q--\textgreater{}p)
\item
  C: God exists. (p)
\end{itemize}

\end{document}
