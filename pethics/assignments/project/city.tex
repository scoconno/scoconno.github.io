\documentclass[]{article}

\usepackage{fancyhdr}
 \pagestyle{fancy}
\rhead{\textsc{Scott O`Connor}}

\usepackage{lmodern}
\usepackage{amssymb,amsmath}
\usepackage{ifxetex,ifluatex}
\usepackage{fixltx2e} % provides \textsubscript
\ifnum 0\ifxetex 1\fi\ifluatex 1\fi=0 % if pdftex
  \usepackage[T1]{fontenc}
  \usepackage[utf8]{inputenc}
\else % if luatex or xelatex
  \ifxetex
    \usepackage{mathspec}
    \usepackage{xltxtra,xunicode}
  \else
    \usepackage{fontspec}
  \fi
  \defaultfontfeatures{Mapping=tex-text,Scale=MatchLowercase}
  \newcommand{\euro}{€}
\fi
% use upquote if available, for straight quotes in verbatim environments
\IfFileExists{upquote.sty}{\usepackage{upquote}}{}
% use microtype if available
\IfFileExists{microtype.sty}{%
\usepackage{microtype}
\UseMicrotypeSet[protrusion]{basicmath} % disable protrusion for tt fonts
}{}
\ifxetex
  \usepackage[setpagesize=false, % page size defined by xetex
              unicode=false, % unicode breaks when used with xetex
              xetex]{hyperref}
\else
  \usepackage[unicode=true]{hyperref}
\fi
\usepackage[usenames,dvipsnames]{color}
\hypersetup{breaklinks=true,
            bookmarks=true,
            pdfauthor={},
            pdftitle={},
            colorlinks=true,
            citecolor=blue,
            urlcolor=blue,
            linkcolor=magenta,
            pdfborder={0 0 0}}
\urlstyle{same}  % don't use monospace font for urls
\setlength{\parindent}{0pt}
\setlength{\parskip}{6pt plus 2pt minus 1pt}
\setlength{\emergencystretch}{3em}  % prevent overfull lines
\providecommand{\tightlist}{%
  \setlength{\itemsep}{0pt}\setlength{\parskip}{0pt}}
\setcounter{secnumdepth}{0}

\author{Scott O’Connor}


% Redefines (sub)paragraphs to behave more like sections
\ifx\paragraph\undefined\else
\let\oldparagraph\paragraph
\renewcommand{\paragraph}[1]{\oldparagraph{#1}\mbox{}}
\fi
\ifx\subparagraph\undefined\else
\let\oldsubparagraph\subparagraph
\renewcommand{\subparagraph}[1]{\oldsubparagraph{#1}\mbox{}}
\fi

\begin{document}

\subsection{NJCU HONORS 360}\label{njcu-honors-360}

(modified from Dr.~Virginia Ochoa-Winemiller)

\begin{description}
\tightlist
\item[\textbf{Objective:}]
To explore the relationship between education and place. In particular,
how does the landscape represent education, its levels (e.g.~elementary,
middle, college, special), and organization (administration, learning,
food, leisure, movement, reflection, to mention some).
\item[\textbf{Task:}]
Working with another classmate, you will explore the NJCU campus and
nearby areas using the methods of participant observation, interviewing,
and landscape assessment. You and your teammate are encouraged to ask
questions, take notes, and write your observations and insights as soon
as possible after the visit.
\end{description}

Your journal entry must include:

\begin{enumerate}
\def\labelenumi{\arabic{enumi}.}
\tightlist
\item
  Date, time
\item
  Weather conditions
\item
  Name, type, and address of site visited
\item
  Record the following information:

  \begin{enumerate}
  \def\labelenumii{\arabic{enumii}.}
  \tightlist
  \item
    Purpose of educational site. If a building, what is the building's
    function? Is it a library, a lab, etc? If an outside location, or
    some other type of place, what is the purpose of that location
    (educational or other)?
  \item
    Observe and describe the markings that clearly indicate the place's
    function is educational, e.g., signs, statues, types of
    architecture, etc. In other words, what is it about the place that
    tells you it is an educational one and not something else?
  \item
    What view of education do you think the relevant place embodies?
    What values are being communicated? For instance, does it suggest
    that education requires quiet reflection, or perhaps open
    communication?
  \item
    Observe the students, staff, and faculty. Observe their interaction.
    Note their age, race/ethnicity, gender. Is the population diverse?
    Do people from diverse groups interact with one another?
  \end{enumerate}
\end{enumerate}

What do you think about these places for education? Do you agree with
the values communicated? Do you like the way people interacted?

\end{document}
