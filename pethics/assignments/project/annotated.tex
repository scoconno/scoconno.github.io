\documentclass[oneside]{article}
 \headheight = 25pt
\footskip = 20pt
\usepackage{mdwlist}
\usepackage[T1]{fontenc}
\renewcommand{\rmdefault}{ppl}
\usepackage{fancyhdr}
 \pagestyle{fancy}
 \lhead{\textbf{\textsc{\small Scott O'Connor\\Ancient Philosophy}}}
 \chead{}
 \rhead{\large\textbf{\textsc{Annotated Bibliography}}}
 \lfoot{\footnotesize{\thepage}}
 \cfoot{}
 \rfoot{\footnotesize{\today}}
 \usepackage{longtable,booktabs}
\tolerance=700




\begin{document}



\subsection*{What is an annotated bibliography?}\label{header-n11}

An annotated bibliography is a list of citations to books, articles, and
documents. Each citation is followed by a brief descriptive paragraph, the annotation. The purpose of the annotation is to create and collect a small number of documents that you can
 use to write your large research paper. We will focus on three
types of resources:

\begin{description}
\item[The primary text:] this is your chosen work, whether it is a particular
  book of Plato's \emph{Republic}, the \emph{Symposium}, etc. If the
  work is divided into books, like Plato's \emph{Republic}, you will
  likely only focus on a specific book.
\item[Background sources:] these include encyclopedia entries and textbooks.
  Such resources do not argue for any particular view. They explain a
  text and often summarize the main interpretative disputes about a
  text. The Stanford Encyclopedia of Philosophy and the Internet
  Encyclopedia of Philosophy are excellent. They are available online
  and will have suggestions for other readings.
\item[Secondary literature:] these include academic research about your
  chosen text. Very little is agreed upon and the secondary literature
  is where academics argue for one interpretation over the other. These
  come in the form of papers published in peer review journals and books
  published by academic presses. Papers run normally about 30 pages.
  While books are obviously longer, you normally will be able to focus
  on one or two chapters for your project.
\end{description}


\subsection*{The Process}\label{header-n16}

Creating an annotated bibliography calls for the application of a
variety of intellectual skills: concise exposition, succinct analysis,
and informed library research.

\begin{enumerate}
\def\labelenumi{\arabic{enumi}.}
\item
  Locate and record citations to books, periodicals, and documents that
  may contain useful information and ideas on your topic. While you can
  use the library website, you should also go into the library. If you
  locate a book you think is useful, look at the rest of the shelf.
  Often you will find other relevant material.

  \begin{itemize}
  \item
    \textbf{NB:} Use librarians! One of the main jobs of an academic
    librarian is to help users find research material. Tell them what
    you are interested in and they will help you figure out what to
    read.
  \item
    Digital resources are risky. There is as much false information
    online as there is accurate information. If you do use online
    resources, try only to use resources found through the library's
    website. 
  \item
    A good place to start is the list of resources under 'Ancient
    Resources' on the course website.
  \end{itemize}
\item
  Briefly examine and review the actual items. While you only need to
  skim these sources, you need to read them closely enough to decide
  whether they are relevant.
\item
  Choose those works that are relevant and helpful for your topic.
  Discard those that are irrelevant.
\item
  Of those you keep, cite the book, article, or document using the your
  preferred citation style, e.g., APA, Chicago, etc. Make sure to note
  the translator of any Greek texts.
\item
  Write a concise annotation that summarizes the central theme and scope
  of the work. For the primary text,  identify the main
  claim/question that interests you and write a very short summary of
  the work, where your summary picks up on whatever is
  relevant for your topic. Think here of the context from the gobbets.
  For the other types of sources, explain how they relate to your
  question. If they contain an answer, or a variety of answer, clearly
  state those answers. You may also compare or contrast the various works you
  have cited, e.g., work A defends answer X, but work B rejects answer X
  and argues for answer Y. 
\end{enumerate}

\subsection*{Requirements}\label{header-n55}

\begin{itemize}
\item
  Your annotated bibliography must include \textbf{annotations} for your \textbf{one}
  primary text, \textbf{one} background source, and \textbf{two}
  secondary literature sources. Word limits (keep it short!):

  \begin{itemize}
  \item
    Primary text: 250-500 words.
  \item
    Background source: 150-250 words.
  \item
    Secondary literature 1: 150-250 words.
  \item
    Secondary literature 2: 150-250 words.
  \end{itemize}
\item
  Your annotated bibliography must include \textbf{citations} for i)
  your primary text, ii) at least \textbf{two} background sources, i.e.,
  encyclopedias and textbooks, and (iii) at least \textbf{five}
  secondary literature sources, whether they be articles from scholarly
  journals or books published by academic publishers.
\item
  \textbf{NB: You are being asked to collect and cite several sources
  that discuss your topic, but I am asking you to summarize just a few
  of these sources. I encourage you to complete the annotations for the
  rest of the citations before you write your paper, but that is not a
  requirement for this assignment.} 
\item
  Grading: this is worth 7 points towards your final grade in the
  course. (The proposed question was worth 3. The literature review,
  presentation, and final paper will be worth a total of 30.)
 
\end{itemize}

\end{document}
