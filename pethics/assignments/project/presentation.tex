\documentclass[oneside]{article}
 \headheight = 25pt
\footskip = 20pt
\usepackage{mdwlist}
\usepackage[T1]{fontenc}
\renewcommand{\rmdefault}{ppl}
\usepackage{fancyhdr}
 \pagestyle{fancy}
 \lhead{\textbf{\textsc{\small Scott O'Connor}}}
 \chead{}
 \rhead{\large\textbf{\textsc{Presentation}}}
 \lfoot{\footnotesize{\thepage}}
 \cfoot{}
 \rfoot{\footnotesize{\today}}
 \usepackage{longtable,booktabs}
\tolerance=700

\begin{document}


\section*{Introduction}\label{introduction}

Great progress! You have a question. You have identified several
readings that are relevant to that question. You have explained the
background to your questions and sketched out some standard answers in
the literature. 

Now that you have surveyed the literature, it is time for you to form
your own opinions on the question. Do you think one of the answers that
has been defended is correct? If so, why that one as opposed to another?
Do you think there is an ignored alternative? Now is the time for you to
advance the conversation.

This presentation provides you the opportunity to formulate your views
before writing your final paper. So, this presentation is not your final say on the matter; it is your first attempt to formulate your own views. 

\section*{Requirements}\label{requirements}

\begin{itemize}
\item Create a powerpoint presentation. All NJCU students have access to Office 365.  
\item Your presentation should be \textbf{6--8 minutes maximum}.
\item Include 6--7 slides. Each slide should contain a striking picture, or quote,
or graph, or other visual cue. It must also provide in bullet form a
summary of the main point of the slide. You will be using your slides as
cues---\textbf{do not fill your slides with text}. The 6--7 slides must be
written as follows (each number corresponds to a slide):

\begin{enumerate}
\item
  Relevant background on your topic.
\item
  Question raised.
\item
  Answer 1 explained
\item
  Answer 2 explained
\item
  Your novel contribution introduced.
\item
  Evidence/support/argument for your novel contribution...you may include two slides for this.

\end{enumerate}
\end{itemize}

\end{document}
