\documentclass[oneside]{article}
 \headheight = 25pt
\footskip = 20pt
\usepackage{mdwlist}
\usepackage[T1]{fontenc}
\renewcommand{\rmdefault}{ppl}
\usepackage{fancyhdr}
 \pagestyle{fancy}
 \lhead{\textbf{\textsc{\small Scott O'Connor\\Pandemic Ethics}}}
 \chead{}
 \rhead{\large\textbf{\textsc{Virtue Ethics}}}
 \lfoot{\footnotesize{\thepage}}
 \cfoot{}
 \rfoot{\footnotesize}
 \usepackage{longtable,booktabs}
\tolerance=700


\begin{document}
\thispagestyle{fancy}

\section*{Introduction}
In the \emph{Nicomachean Ethics}, Aristotle defends a particular conception of the best life that relies heavily on his work in the \emph{Physics} and \emph{De Anima}.  The basic idea of the \emph{Ethics} might be expressed as follows: 

\begin{enumerate}
\item[A.] There is a final human good. 
\item[B.] Human beings have functions (erga).
\item[C.] The human ergon is an activity of the soul involving reason.
\item[D.] The final good depends on the perfection of this activity
\end{enumerate}

\section*{The highest good (chs. 1 \& 2)}

Aristotle investigations normally follow a pattern: a question is asked, various answers are surveyed, a \emph{test} is developed to decide between the answers, the answers are adjudicated. The \emph{Nicomachean Ethics} follows the same patter. It begins by arguing for what we may take as an obvious point: all of our lives aim at some end, at some goal we wish, even if unconsciously, to realize. Call this \emph{the highest good}. Aristotle asks what this highest good is. What is the single overarching good that you aim for in this life? Aristotle surveys some answers, develops a test to identify the correct answer, and then applies that test. Let us take some of the steps in turn: 

\subsubsection*{Argument that there is a highest good}

\begin{enumerate}
\item  Every craft (\emph{techn\^{e}}), investigation (\emph{methodos}), action (\emph{praxis}), and decision (\emph{prohairesis}) aims at some good (in what follows we will focus on actions)

\item Thus, the good (\emph{to agathon}) is that at which everything aims
\item  Ends (telos) are the goods aimed for in purposeful action. They differ in being. Some  are actions or activities themselves. Some are the products that result from the actions or activities that aim at (producing) them.

\item Ends may be \emph{hierarchically structured}: one end (e.g. bridles) can be pursued for the sake of another (e.g. horse-riding)
\begin{itemize}\item{In such a case, the superordinate end is more \emph{choiceworthy} than the subordinate end}\item{The superordinate end also sets the conditions for when the subordinate end is accomplished well (e.g. the value of a bridle \emph{qua} bridle is determined by the ``needs'' of horse-riding)}\end{itemize}

\item If there is some end we pursue \emph{only} for itself and for which we pursue all our other ends, this end with be the \emph{highest} good (or, simply, \emph{the} good) 

\item If there were no highest good, then our desires would be empty and futile (1094a21-2) 

\item But, our desires are not empty and futile. 
\item[C] There is a highest good. 
\end{enumerate}
What is this highest good that we all seek? Aristotle claims that everyone agrees that ``happiness'' (\emph{eudaimonia}) is the \emph{name} of the highest good, but they disagree about what happiness is. Before we turn to discuss these options, let's focus on the point of agreements. In claiming that happiness is the highest good we seek, we are claiming that happiness is that final end for which we choose to engage in other action. We want to earn money not as en end in itself but because it will lead, we think, to happiness. Likewise, we choose to attend school, eat well, save, avoid excesses because it all somehow will contribute or result in happiness. If we could agree then on what happiness is, on what this final end for which we choose other things, then we could order our lives correctly. So, what is happiness and how can we obtain it? 

While ethics belongs to political science (ch.I.2), Aristotle insists that we should not seek too much exactness in ethics (I.3); we cannot answer our question by appealing to first principles in the way in which we might answer question in arithmetic by appealing to first principles. Instead, Aristotle claims we need to start by focusing on the common beliefs about happiness (I.4). What are the most common and feasible views on the nature of happiness? He identifies four standard answers. The first says that happiness consists in pleasure, the second in wealth, the third in honor and fame, and the fourth in thinking. Each view has profound implications for how we should life. If wealth is the end we should seek, then we should learn the skills needed to succeed in high revenue paying industries. If fame and honor is the end we should seek, then we need to focus all our energies on cultivating the talent, resources, and connections we need to achieve that. 


%\item There must be some limit to the ends we choose for the sake of something else, otherwise our desires would be ``empty and futile'' (1094a21-2)
%\begin{itemize}\item{[P1] If there were no highest good, then our desires would be empty}\item{\underline{[P2] But, our desires are not empty}}\item{[C] There is a highest good}\end{itemize}


%\item If we knew what this highest good was, we could order our lives correctly
% Aristotle maintains that when we are discussing the HG we can't expect the same level of precision as when we are doing, say, mathematics (ch. 3 1094b13-27)



%\noindent [10] People generally develop their conception of the HG from the lives they live, and there are three main kinds of lives: gratification, political activity, study (ch. 5 1095b14-16)

%\section*{Methodological points from Book 1}

%\begin{itemize}

%\item Ethical argument should appeal to common beliefs (endoxa) (I.4).
%\item Forming correct ethical beliefs depends on correct habituation (I.4).
%\item Contra Socrates and Plato, `good' does not denote a single property (I.6).
%\end{itemize}


\subsubsection*{Tests for the HG }

In I.7 Aristotle identifies various formal features of the highest good and argues that happiness meets these criteria (but, again, we don't know what, exactly, happiness is). If happiness meets the criteria for the highest good, concrete proposals for the nature of happiness must also meet these criteria. 

\begin{description}
\item[Most choiceworthy:] The highest good must not merely be good and valuable. It cannot be one good among many.
\item[Complete:]  X is \emph{more complete} than Y if (a) X is pursued for X and (b) Y is pursued for something else, Z. The highest good is is complete without qualification, i.e., the highest good is pursued \emph{only} for itself and all other things are pursued for the sake of the highest good (1097a).
\item[Self-sufficient:] Possessing the highest good all by itself makes a life lacking in nothing (1097b)
\begin{itemize}
\item NB: This does \emph{not} necessarily mean that the value of a life that has the highest good can't be increased. It could mean that a life with the highest good doesn't need anything more to make it choice-worthy as a life.
\end{itemize}
\end{description}

These criteria allow us rule out some conceptions of the best human life. Consider the view that says happiness consists in wealth, that the highest end we should seek is wealth. Is wealth a good that satisfied all of these tests? It's clearly not self-sufficient. We can easily imagine a very rich person whose life lacks in many important things. Perhaps they are unkind and unloveable. Perhaps they are very sick and miserable. It's clearly possible to have lots of money and, nevertheless, lack in many of the things that make life worth living. Is wealth a complete good? We might choose other things in order to gain wealth, e.g., we might study hard in order to acquire the skills needed to succeed at a lucrative job. But wealth is hardly an end in itself; something we pursue for its own sake. Those who pursue wealth normally see it as means to secure something they find even more important, e.g., the security of themselves and their families. Others pursue wealth so they may travel the world, buy luxury items, etc. So wealth is clearly not a complete good. 

\subsubsection*{HG is rational activity of the soul according to virtue}

So, happiness has the marks of the highest good. It is choice-worthy, an end by which we can order our actions, complete, and self-sufficient. But this does not yet tell us what happiness is. Aristotle's proposal comes in his (in)famous ergon argument. The word `ergon' can be translated as `function'. Perhaps a better translation is `work' (or perhaps `activity'). The idea will be that things have characteristic work or functions. An eye's characteristic work is seeing. A knife's characteristic work is cutting. Aristotle will claim that the characteristic activity of human beings has all the marks of the highest good. It passes the test. Perfecting this activity will thereby constitute the end which we should dedicate our lives. Here's the function argument in outline: 

\begin{enumerate}
\item[P1] For any F, where F is a kind with a function, the good of an F = performing the function of Fs well.
\item[P2] The function (ergon) of any living being x is determined by x's unique and  characteristic activity (1097b23)
\item[P3] The parts of a human have a function; so a human as a whole should have a special function
\item[P4] The ``lives'' of nutrition and perception are not unique to humans (recall this from the \emph{De Anima})
\item[P5] The unique and characteristic activity of human beings is reasoning (1098a3) (``Life'' understood as activity (as opposed to capacity) is more properly life).
\item[P6] So the highest good for humans is performing activity of the rational part of the soul well.
\item[P7] An F performs its function well when F acts in accordance with virtue (1098a7-12).
\item[C] Therefore, the the highest good for humans is activity of the soul expressing reason in a virtuous manner.  
\end{enumerate}
Here I note a few additional constraints on happiness.\begin{itemize}
\item It requires activity based on virtue in a \emph{complete} life (ch. 9 1100a)
\item It requires external goods (ch. 8 1099b)
\item It must be the best, finest, and most pleasant (life) (ch. 8 1099a)
\end{itemize}

\section*{Aristotle's conception of the soul}
Aristotle defines happiness as rational activity of the soul in accordance with virtue. But what does that mean?  In the \emph{De Anima}, Aristotle claims that the human soul has both rational and non-rational aspects or parts. Similar to Plato, he takes motivation to come in three main kinds (reason, spirit, appetite).  (Bk. II, Ch. 3)
%\item{However, A thinks it may be wrong to think of these as distinct \emph{parts} of the soul, as opposed to distinct aspects of one and the same thing (as the convex and the concave are two aspects of one and the same curved line)}
The non-rational aspect itself has two aspects. One that is wholly non-rational. This is the part that is responsible for maintenance of the human body. The second is a non-rational part that can ``obey'' and be ``trained'' by reason. Aristotle's developed view of the soul is going to rely heavily on this idea of one part of the soul obeying another. 

\section*{Virtue (Bk. II, Chs. 1-4)}

Corresponding to the two aspects of the soul are two ``kinds'' of virtues. The first are virtues of intellect. These belong to the part of the soul that has reason strictly speaking, i.e., the parts that does the ordering. The second are virtues of character. These belong to the parts of the soul that can obey reason. We have seen that a virtue is that, whatever it is, that allows a thing perform its function well. So the intellectual virtues will be those that allow us reason and ultimately order well. The virtues of character are those that belong to the non-rational soul that allow it follow the dictates of reason well.  Some notes about virtues of character: 
\begin{itemize}
\item Virtues of character are acquired through habituation. By repeatedly performing just, temperate, etc. acts, one \emph{becomes} just, temperate, etc. Aristotle compares this to the way in which someone acquires a craft (Ch. 1)
\item States of character tend to be ``ruined'' by excess and deficiency (Ch. 2).
\begin{enumerate}
\item If people stand firm against nothing, they become cowardly. If they fear nothing and rush into every confrontation, they become rash. 
\item If they give in to every pleasure, they become intemperate. If they refrain from all pleasures, they become ``a kind of insensible person''.
\end{enumerate}
\item A virtue of character is a state of our non-rational soul that ensures we feel and respond rationally in neither of these extreme ways. For instance:  
\begin{enumerate}
\item A rash person feels too much confidence in danger; their rashness causes them to risk their lives. A coward feels too much fear in the face of danger; their cowardice causes them to run away. A courageous person feels the appropriate balance of fear and danger in the face of death; their courage causes them to risk their lives when appropriate.
\item A intemperate person feels too much lust for sex, an insensible feels too little, and a temperate feels an appropriate amount. Each trait causes the relevant emotional response and behavior. 
\end{enumerate}
\end{itemize}

It may seem that Aristotle thinks that a happy life consists only in performing certain kinds of virtuous actions. But Aristotle insists that we can't just look at a person's actions to evaluate the kind of character they have,---we also have to look at the pleasures and pains ``in consequence of their actions''. He claims that ``virtues are concerned with actions \emph{and} feelings'' (1104b14) (Ch. 3)

%\item The claim that we become virtuous by performing virtuous acts presents a puzzle: if we perform virtuous acts, aren't we \emph{already} virtuous? (Ch. 4)
To distinguish between virtuous action performed with and without the appropriate feeling, Aristotle distinguishes between [1] performing a virtuous act and [2] performing a virtuous act \emph{virtuously}---anyone can do the former (even vicious people). To perform a virtuous act \emph{virtuously}, people must also:
\begin{itemize}
\item{[A] Act knowingly
}\item{[B] Decide to perform the action and decide to perform it \emph{for its own sake}}\item{[C] Act from a firm and unchanging state}\end{itemize}
This should hopefully be enough to introduce you to Aristotle's conception of the good life. It is one in which the non-rational part of the soul possesses the kind of virtues that ensures it acts accordance to the dictates of reason. 

\end{document}


\end{document}
