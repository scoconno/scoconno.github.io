% !TEX encoding = UTF-8 Unicode
% !TEX TS-program = xelatex

\documentclass[11pt]{article}
\usepackage{fontspec}
\defaultfontfeatures{Mapping=tex-text}
\usepackage{xunicode}
\usepackage{xltxtra}
\usepackage{verbatim}
\usepackage[margin= 1 in]{geometry} % see geometry.pdf on how to lay out the page. There's lots.
\geometry{letterpaper} % or letter or a5paper or ... etc
%\usepackage[parfill]{parskip}    % Activate to begin paragraphs with an empty line rather than an indent 
\usepackage{mathrsfs}
\usepackage{bbding}
\usepackage[usenames,dvipsnames]{color}
\usepackage{natbib}
\usepackage{stmaryrd}
%\usepackage{mathpartir}
\usepackage{txfonts}
\usepackage{graphicx}
\usepackage{fullpage}
\usepackage{hyperref}
\usepackage{amssymb}
\usepackage{epstopdf}
\usepackage{fontspec}
%\setmainfont{Hoefler Text}
\setmainfont[BoldFont={Minion Pro Bold}]{Minion Pro}
\usepackage{hyperref}
\usepackage{lastpage, fancyhdr}
%\usepackage{setspace}
\pagestyle{fancy}
\lhead{}
\chead{Epicurus} 
\rhead{}
\lfoot{}
\cfoot{\thepage\space of \pageref{LastPage}} 
\rfoot{}
\footskip=30 pt
\headsep=20pt
\thispagestyle{empty}
\hypersetup{colorlinks=true, linkcolor=Sepia, urlcolor=Sepia, citecolor=BrickRed}
\DeclareGraphicsRule{.tif}{png}{.png}{`convert #1 `dirname #1`/`basename #1 .tif`.png}

\usepackage{covington}
\usepackage{fixltx2e}
\usepackage{graphicx}
\begin{document}

%\maketitle
\thispagestyle{empty}
\begin{center} \LARGE{ Epicurus}\\ \vspace*{2mm}
\large{Scott O'Connor}\end{center}
\thispagestyle{empty}\vspace*{3mm}

\section*{Introduction}

Epicurus founded a community in Athens called ``the Garden''. Men and
women, free persons, and slaves, were all equally accepted. He claimed that the goal of philosophy was to provide a guide to living
well. In this, he was in agreement with his famous predecessors, Socrates and Plato. However, he offered a very different view as to what made a life well lived. According to Epicurus, one should maximize pleasures and minimize pains. You should accept pains that lead to greater pleasures. You should reject pleasures that lead to greater pains. Our entry point will be his views about death. 

\section*{Death}\label{death}

\begin{quote}
``For there is nothing terrible in life for the man who has truly
comprehended that there is nothing terrible in not living.''
 \end{quote}
Epicurus believes that most pain in life comes from fearing death. He asks us to distinguish the pain of death  from the pain of anticipating death. He argues that being dead is not itself painful, and he subsequently argues that we should feel no pain in anticipating death.  Understanding death makes life enjoyable because it takes away the craving for immortality.



\begin{enumerate}
\item Nothing is good or bad for one except sense experience, i.e.~feelings of pleasure and pain. 
\item The dead don't have any sense experiences. 
\item Therefore, nothing is good or bad for the one who is dead. 
\item Therefore, the state of being dead is not (good or) bad for the one who is dead. 
\item If X is not bad for one when it is present, then there is no rational
ground, before it is present, to fear its future presence. 
\item Therefore, no living person has any rational ground to fear his future state of being dead.
\end{enumerate}




\section*{Hedonism}
Our first  premise relies on Epicurus' general account of the good life. 

\begin{quote}
Pleasure is the starting point and goal of living blessedly (\emph{LM} 128).\end{quote}

\begin{quote}
 The removal of all feeling of pain is the limit of the magnitude of pleasures. Wherever a pleasurable feeling is present, for as long as it is present, there is neither a feeling of pain nor a feeling of distress, nor both together (\emph{PD} III).\end{quote}
So, according to Epicurus, happiness is what he calls \emph{ataraxia}, which means, roughly, freedom from disturbance. Epicurus' basic idea is that pleasure consists in the absence of the disturbances and discomfort of desires. Because of the close connection of pleasure with desire-satisfaction, Epicurus devotes a considerable part of his ethics to analyzing
different kinds of desires. If pleasure results from getting what you
want (desire-satisfaction) and pain from not getting what you want
(desire-frustration), then there are two strategies you can pursue with
respect to any given desire: you can either strive to fulfill the
desire, or you can try to eliminate the desire. For the most part
Epicurus advocates the second strategy, that of paring your desires down
to a minimum core, which are then easily satisfied. Epicurus distinguishes between three types of desires:

%There are two ways ataraxia can be achieved, either by satisfying
 %them, or by eliminating them. One can eliminate them by changing them altogether. The approach one takes  will depend on the type of desire. 
 
 %Thus, in order to be happy, one ought to change desires so that one only wants things that are easy to get.


%Epicurus investigates the nature of pleasure and draws some important distinction.  
%\begin{description}
%\item[Static pleasure] consists in the absence of pain, want and desire, e.g., freedom from hunger, thirst. 
%\item[ Kinetic pleasures] always  involves a change in one's psychic state, valuable as a means to  achieving static pleasure (e.g., quenching one's thirst).
%\end{description}

%\item ataraxia consists in static pleasure, understood as the absence of pain (33, 60).
%\item Adaptive conception of happiness: ataraxia achieved by (1) satisfying desires, (2) eliminating them. Thus, in order to be happy, one ought  to change desires so that one only wants things that are easy to get.
 





\begin{enumerate}
\item Natural and necessary desires:

  \begin{itemize}
  \item
    Examples of natural and necessary desires include the desires for
    food, shelter, and the like.
  \item
    These desires are easy to satisfy, difficult to eliminate (they are
    `hard-wired' into human beings naturally), and bring great pleasure
    when satisfied.
  \item
    Furthermore, they are necessary for life, and they are naturally
    limited: that is, if one is hungry, it only takes a limited amount
    of food to fill the stomach, after which the desire is satisfied.
  \item
    Epicurus says that one should try to fulfill these desires.
  \end{itemize}
\item Natural but non-necessary desire:

  \begin{itemize}
  \item An example of a natural but non-necessary desire is the desire for
    luxury food. Although food is needed for survival, one does not need
    a particular type of food to survive.
  \item
    Thus, despite his hedonism, Epicurus advocates a surprisingly
    ascetic way of life. Although one shouldn't spurn extravagant foods
    if they happen to be available, becoming dependent on such goods
    ultimately leads to unhappiness.
  \item
    As Epicurus puts it, ``If you wish to make Pythocles wealthy, don't
    give him more money; rather, reduce his desires.''
  \item
    By eliminating the pain caused by unfulfilled desires, and the
    anxiety that occurs because of the fear that one's desires will not
    be fulfilled in the future, the Epicurean attains tranquility,
    and thus happiness.
  \end{itemize}
\item  Vain and empty desires:

  \begin{itemize}
  \item
    Vain desires include desires for power, wealth, fame, and the like.
  \item
    They are difficult to satisfy, in part because they have no natural
    limit. If one desires wealth or power, no matter how much one gets,
    it is always possible to get more, and the more one gets, the more
    one wants.
  \item
    These desires are not natural to human beings, but they are inculcated by  society and by false beliefs about what we need; e.g., believing
    that having power will bring us security from others.
  \item
    Epicurus thinks that these desires should be eliminated.
  \end{itemize}
\end{enumerate}

\section*{Premise 2}
\begin{description}
\item[P2] relies on E's doctrine of the soul. 
\begin{itemize}
\item E is an atomist: all reality consists in atoms (i.e. indivisible, indestructible, units of matter) and void.
\item Even the soul, according to E, is composed of atoms (i.e. the soul is a body---quite different from Plato and Aristotle), which ``dissipate'' upon death
\end{itemize}
\end{description}

\section*{Notes}
If 1 and 2 are too controversial, maybe we can substitute less controversial variants:

\begin{enumerate}
\item[1*] Good and bad depend on there being a subject who could experience them
\item[2*] Death is the extinction of the `self' or `person' --- i.e. of the subject capable of experience
\end{enumerate}
Even if we still doubt these premises, Epicurus has raised three pressing problems: 
\begin{enumerate}
\item[A] How can something be bad for \emph{S} if \emph{S} does not or \emph{cannot} mind or care one way or the other, since \emph{S} is non-existent?
\item[B] Who could be the possessor or subject of this bad once \emph{S} is non-existent?
\item[C] When could the subject suffer this bad?
\end{enumerate}

\section*{Responses}

\noindent \textbf{Nagel}
\vspace*{2mm}

\noindent [1] Death is bad because it involves the deprivation of goods---e.g. perception, thought, emotion
\vspace*{1mm}

\noindent [2] Goods and bads for someone do not depend on that person's awareness of them (cf. \emph{EN} 1.10)
\vspace*{1mm}

\noindent E.g.: Suppose we all have significant others who, while we are here, get together for swinging affairs; suppose that part of our well-being stems from the (perhaps unconscious) faith in our SO's fidelity; suppose further that none of us ever find out about it and that, if anything, the only consequences we experience are in a sense beneficial (e.g. our SO's are nicer, kinder, etc. to us as a result); it still seems like this is bad \emph{for us}
\vspace*{2mm}

\noindent [3] The person who is deprived of goods by death is a ``possible person''---i.e. the person who was alive, but so understood as to include the (unrealised) possibilities of her continued life
\vspace*{1mm}

\noindent E.g.: An accident victim suffers head-injuries. Her IQ drops to 20, but she is ``happy'' or, at least, ``cheerful.'' We tend to think that the ``person'' is unfortunate; but the current person is quite ``happy.'' So we must be ascribing the misfortune to the person-she-could-have-been = a ``possible person''
\vspace*{2mm}

\noindent So, Nagel rejects both [P1] and [P1*]
\vspace*{5mm}

\noindent \textbf{Furley}
\vspace*{2mm}

\noindent [1] Death is bad because it involves the frustration of our previous plans, hopes and desires
\vspace*{1mm}

\noindent [2] The frustration of our current plans etc. would make our present actions pointless
\vspace*{1mm}

\noindent [3] Hence it is rational to fear death, since it is rational to fear that our current actions are pointless
\vspace*{1mm}

\noindent E.g.: a terminally-ill person is deceived about her condition; her concern with her plans for a holiday next spring is pointless---and she would think so too if she knew about her condition
\vspace*{2mm}

\noindent Furley rejects [P5]
\vspace*{5mm}

\noindent \textbf{Epicurus' response to Furley}
\vspace*{2mm}

\noindent \emph{PD}III
\vspace*{1mm}

\noindent [1] Happiness requires only the satisfaction of our natural and necessary desires
\vspace*{1mm}

\noindent [2] These desires can be satisfied by a self-sufficient life---i.e. one which does not involve long-term projects
\vspace*{1mm}

\noindent [3] So you don't understand what happiness is if you think that it involves long-term projects, etc.
\vspace*{1mm}

\noindent [4] Hence your fear of death is ``empty''---i.e. rests on a false belief---and thus irrational




\end{document}
