\documentclass[oneside]{article}
 \headheight = 25pt
\footskip = 20pt
\usepackage{mdwlist}
\usepackage[T1]{fontenc}
\renewcommand{\rmdefault}{ppl}
\usepackage{fancyhdr}
 \pagestyle{fancy}
 \lhead{\textbf{\textsc{\small Scott O'Connor\\Metaphysics}}}
 \chead{}
 \rhead{\large\textbf{\textsc{Zeno 2}}}
 \lfoot{\footnotesize{\thepage}}
 \cfoot{}
 \rfoot{\footnotesize{\today}}
 \usepackage{longtable,booktabs}
\tolerance=700


\begin{document}
\thispagestyle{fancy}



\section{Introduction}

Recall again Zeno's overall argument against the existence of motion. 

\begin{enumerate}
\item Space is infinitely divisible or not infinitely divisible.
\item  If space is infinitely divisible, motion is impossible.
\item  If space is not infinitely divisible, motion is impossible.
\item  Therefore, motion is impossible (from 1--3).
\end{enumerate}

\section{Premise 3}\label{premise-3}

This handout will proceed by discussing premise 3. See Handout 1 for discussion of premises 1 and 2. Zeno offers two distinct arguments for premise 3, the Stadium Paradox and the Arrow Paradox. We will discuss the Arrow Paradox next week.  The strategy in each case is similar. We will fist assume that space is not infinitely divisible, then prove that certain absurdities follow. If an assumption leads to an absurdity, we know the assumption is false. As an example, suppose someone claimed that vaccines cause autism. We could show that assumption is false if we proved that it entailed something false: 

\begin{enumerate}
\item  Childhood vaccines always cause autism (assumed). 
\item If childhood vaccines always cause autism, then every child who gets a vaccine will develop autism. 
\item The vast majority of children get vaccinated. 
\item Therefore, the vast majority of children have, or will develop, autism (from 1--3). 
\end{enumerate}
Our conclusion, 4, is false. Since 4 is false, we know that one of the premises that lead to this conclusion is false. The obvious culprit here is 1, which is the premise that we are testing. So, the general point is that we can show that a claim is false if we show that it entails something that is  false. 


\section{The Stadium Paradox}\label{stadium-paradox}

Zeno asks us to assume that motion occurs over distances that are finitely divisible into smallest parts. This means that there are a finite number of distances between any two points; when you walk between A and B, you have traversed a finite number of some smallest distances. He argues that this assumption leads to an absurd conclusion. Aristotle presents this as follows: 



\begin{quote}
The fourth argument is that concerning equal bodies which move alongside equal bodies in the stadium from opposite directions---the ones from the end of the stadium, the others from the middle---at equal speeds, in which he thinks it follows that half the time is equal to its
double\ldots{}. (Aristotle, \emph{Physics}, 239b33)
\end{quote}
Suppose these rows of blocks represent some chariots in a stadium. The
Bs are stationary. The As are moving from left to right. Use `D' for the last block in the row of As. The Cs are moving towards the Bs from right to left. Use `E' to name the middle block in the Cs. Suppose also that the As and Cs are traveling at the same speed. (The letters in each row should be beside each other in T2 and T3.)

\begin{longtable}[c]{@{}lll@{}}
\toprule
& T1 &\tabularnewline
\midrule
\endhead
& DAA & --\textgreater{}\tabularnewline
& BBB &\tabularnewline
\textless{}-- & CEC &\tabularnewline
\end{longtable}

\begin{longtable}[c]{@{}rll@{}}
\toprule
& T2 &\tabularnewline
\midrule
\endhead
--\textgreater{} & ~ DA & A\tabularnewline
& BBB &\tabularnewline
~ ~ C & EC~ & \textless{}--\tabularnewline
\end{longtable}
Compare times 1 and 2. Suppose they are separated by a one minute
interval of time. In this interval, D has passed one B block and two C blocks. Zeno thinks this is paradoxical. Why? Let us assume the following:

\begin{enumerate}
\item
  There is a smallest possible length, \emph{S}
\item
  The length of each block is S.
\item
  There are no gaps between the blocks.
\item
  The blocks move with constant velocity.
\end{enumerate}
It took 1 minute for D to pass two C blocks. It should take 30 seconds to
pass one C block and become level with E. Suppose D this is true. How many B blocks does D pass after 30 seconds?  Try filling out the diagram below to answer that question.

\begin{longtable}[c]{@{}rll@{}}
\toprule
& T3 &\tabularnewline
\midrule
\endhead
--\textgreater{} & DAA &\tabularnewline
& ? &\tabularnewline
~ ~ C & EC~ & \textless{}--\tabularnewline
\end{longtable}

T3 describes the moment that D and E are level. The paradox arises because of the relationship that D stands in to the Bs at the moment D and E are level.

Suppose that someone claims that D has passed \emph{half
of one B block.} Let this half be called \emph{H}. What is H's length?
You cannot, on pain of contradiction, claim that H has a length less
than S. We have assumed that S is the smallest possible length, so H
cannot be shorter than S. Here is another way of presenting the paradox:

\begin{enumerate}
\item D travels S distance in 30 seconds. 
\item S is the shortest possible distance. 
\item Each block has length-S.
\item Thus, each B block has length S.
\item D passes one B block in 1 minute. 
\item Therefore, D travels S distance and twice S distance in 1 minute. 
\end{enumerate} 
Our assumption that the blocks move over a distance that is divisible into a finite number of smallest distances leads to a contradiction. So, Zeno conclude that that the assumption is false. 

One response to this paradox is to deny that there really is a T3. The paradox assumes that the length of time between T1 and T2 can be divided in two, i.e., 1 minute is divided into two 30
second intervals. Suppose that time is also atomic, that there is a
smallest interval of time, a single quantum of time. Suppose also that
the motion between T1 and T2 takes a single quantum of time. If this is
correct, there is no T3 (which was half the interval between T1 and T2.) 

But, paradox still threatens. During a single quantum of time, D and E will
have passed each other (as is seen in T2), but there is no moment at
which they are level as is described in T3: since T1 \& T2 are
separated by the smallest possible interval of time, there can be no moment of time between them---it would be a interval of time smaller than the smallest interval of time. Conversely, if one insisted that there is some moment when they are level, then this shows that our supposed quantum of time was not the shortest finite interval. We could then run the argument again taking into account this supposed new smallest interval.




\end{document}
