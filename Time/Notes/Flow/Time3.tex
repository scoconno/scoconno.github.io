\documentclass[]{article}
\usepackage{fancyhdr}
 \pagestyle{fancy}
\rhead{\textsc{Scott O`Connor}}
\usepackage{amssymb,amsmath}
\usepackage{ifxetex,ifluatex}
\usepackage{fixltx2e} % provides \textsubscript
\ifnum 0\ifxetex 1\fi\ifluatex 1\fi=0 % if pdftex
  \usepackage[T1]{fontenc}
  \usepackage[utf8]{inputenc}
\else % if luatex or xelatex
  \ifxetex
    \usepackage{mathspec}
    \usepackage{xltxtra,xunicode}
  \else
    \usepackage{fontspec}
  \fi
  \defaultfontfeatures{Mapping=tex-text,Scale=MatchLowercase}
  \newcommand{\euro}{€}
\fi
% use upquote if available, for straight quotes in verbatim environments
\IfFileExists{upquote.sty}{\usepackage{upquote}}{}
% use microtype if available
\IfFileExists{microtype.sty}{%
\usepackage{microtype}
\UseMicrotypeSet[protrusion]{basicmath} % disable protrusion for tt fonts
}{}
\ifxetex
  \usepackage[setpagesize=false, % page size defined by xetex
              unicode=false, % unicode breaks when used with xetex
              xetex]{hyperref}
\else
  \usepackage[unicode=true]{hyperref}
\fi
\usepackage[usenames,dvipsnames]{color}
\hypersetup{breaklinks=true,
            bookmarks=true,
            pdfauthor={},
            pdftitle={Zeno 2},
            colorlinks=true,
            citecolor=blue,
            urlcolor=blue,
            linkcolor=magenta,
            pdfborder={0 0 0}}
\urlstyle{same}  % don't use monospace font for urls
\usepackage{longtable,booktabs}
\setlength{\parindent}{0pt}
\setlength{\parskip}{6pt plus 2pt minus 1pt}
\setlength{\emergencystretch}{3em}  % prevent overfull lines
\providecommand{\tightlist}{%
  \setlength{\itemsep}{0pt}\setlength{\parskip}{0pt}}
\setcounter{secnumdepth}{0}

\title{Time 3}
\author{Scott O’Connor}


% Redefines (sub)paragraphs to behave more like sections
\ifx\paragraph\undefined\else
\let\oldparagraph\paragraph
\renewcommand{\paragraph}[1]{\oldparagraph{#1}\mbox{}}
\fi
\ifx\subparagraph\undefined\else
\let\oldsubparagraph\subparagraph
\renewcommand{\subparagraph}[1]{\oldsubparagraph{#1}\mbox{}}
\fi

\begin{document}
\maketitle


\subsection{The Myth of Passage}

\begin{itemize}
\item The structure of this paper is difficult: there is irrelevant information, no section headings, no sign-posts to guide the reader, etc. However, we can re-structure the content of the paper into three distinct parts. 
\begin{enumerate}
\item An argument that the manifold view cannot explain the passage of time. 
\item An argument that the passage of time does not exist. 
\item An attempt to show that the manifold view can explain the apparent passing of time (even though time does not, in fact, pass). 
\end{enumerate}
\end{itemize}

\subsection{A problem for the manifold view}

\begin{itemize}
\item The manifold view claims that the universe consists in the spread of events in space time, i.e., all events exist irrespective of whether we describe them as being future, present, or past.  
\item These events have temporal relations to one another, i.e., they have B-properties. 
\item If time passes, then events must change in their temporal properties, e.g., the event of you graduating changes from being future to being present to being past. 
\item Alternatively, we can say that if time passes, then the present must move from one event to another event, e.g., the present moves from the event of you studying for the exam to the event of you taking the exam to the event of you receiving the results, and so on. 
\end{itemize}

\subsection{Argument against the passage of time}

\noindent Strategy: identify certain features that each motion must have and then show that what is called the passing of time cannot possess these features. 

\begin{enumerate} 
\item If time flows, then time moves. 
\item For any motion M, M is a motion with respect to time.
\item If time moves, then time is a motion with respect to time. (From 1 and 2) 
\item If Time-A is a motion with respect to time, then Time-A is a motion with respect to itself, Time-A, or with respect to some other type of time, Time-B.  
\item Time-A cannot be a motion with respect to itself, Time-A
\item Time-A cannot be a motion with respect to Time-B. 
\begin{itemize}
\item If it is of the essence of time that time passes, then Time-B will pass as well, requiring some other type of time, Time-C, and so on \emph{ad infinitum}. 
\end{itemize}
\item Time-A is not a motion with respect to time. (From 4-6)
\item Time cannot move. (From 3 and 7) 
\item Time cannot flow. (From 1 and 8)
\end{enumerate}
\section{Flow re-described}

\noindent Strategy: re-describe the phenomenon in way that is compatible with the manifold view and in a way that is immune to the objection in section 3. 

\begin{itemize}
\item Consider sailing down a long lake. You comment to your neighbor, `the river is really changing, it was so broad, now it is so narrow, and it will become narrower still. Also, did you see how dark the river was. But the color too is changing with the depth. It is becoming bluer and bluer as the day goes on'. 
\item These sentences speak of the river changing in both breadth and in color. Taken literally, they mean that one individual entity loses and gains the properties of being broad and narrow. Taken literally, these sentences are then false. But we should not take them literally. What appears to be one object losing and gaining various properties is really different parts of that object having different properties: one stage of the river is broad, a distinct stage of the river is narrow. 
\end{itemize}


\end{document}