\documentclass[oneside, 11]{article}
 \headheight = 25pt
\footskip = 20pt
\usepackage[T1]{fontenc}
\renewcommand{\rmdefault}{ppl}
\usepackage{fancyhdr}
 \usepackage{fancyhdr}
 \pagestyle{fancy}
 \lhead{\textbf{\textsc{\small Scott O'Connor\\Time }}}
 \chead{}
 \rhead{\LARGE\textbf{\textsc{Two Times}}}
 \lfoot{\footnotesize{\thepage}}
 \cfoot{}
 \rfoot{}
\tolerance=700

\begin{document}
\thispagestyle{fancy}
\title{Time 1}
\author{Scott O'Connor}




 

\subsection*{The A-theory and the B-theory}

\begin{itemize}
\item Events are ordered in time. The A-theory and the B-theory are different theories about the nature of this ordering. 
\item A-properties are one-place relations: \emph{being past, being two days past, being present, being future, being 1 year in the future}.
\begin{itemize}
\item Time flows by events changing in their A-properties. 
\item Example: in 1992 the event of Obama winning the election had the property of being a future event, an event that will happen. At that time, it did not have the property of being a present event, an event that was currently happening. Nor did it have the property of being a past event, an event that had occurred. This event briefly acquired the property of being present in 2008 before losing that property and becoming a past event. 
\end{itemize}
\item B-properties are two-place relations: \emph{earlier than, later than, simultaneous with}
\begin{itemize}
\item B-properties are relations between events, e.g., the event of Obama winning the election was later than the event of George Bush winning the election.
\item Events never change their B-properties, e.g., it is eternally true that Obama won the election after George Bush won the election.
\item Consider a spatial analogy....volunteers please.
\end{itemize}
\item A-theory: both A and B properties exist. 
\item B-theory: A properties cannot exist. Only B-properties exist. 
\end{itemize}

\subsection*{Objects at Different Times}

Distinguish two senses of `x exists now'. 1) `x is present'. 2) `x is in the domain of all things that exist.' 

\begin{description}
\item[Presentism] is the view that only present objects exist. If we were to make an accurate list of all the things that exist, there would not be a single non-present object on the list. Thus, you and the Taj Mahal would be on the list, but neither Socrates nor any future Martian outposts would be included. The same goes for any other putative object that lacks the property of being present. All such objects are unreal, according to Presentism.
\item[Eternalism] is the view that objects from both the past and the future exist just as much as present objects. According to Eternalism, non-present objects like Socrates and future Martian outposts exist right now, even though they are not currently present. We may not be able to see them at the moment, on this view, and they may not be in the same space-time vicinity that we find ourselves in right now, but they should nevertheless be on the list of all existing things.
\item[The Growing Universe Theory] is the view that only past and present but not future objects exist. 
\end{description}


\subsection*{McTaggert on Time}
J. M. E. McTaggart argued that time does not exist. He does so by arguing against both the A and B theories. 

\begin{enumerate}
\item If time is real, either the B-theory or the A-theory of time is the correct characterization of time. 
\item The correct characterization of time must allow for change. 
\item The B-theory cannot allow for change. [See below]
\item The B-theory cannot characterize time. [From 2--3]
\item An adequate characterization of time cannot be contradictory. 
\item The A-theory is contradictory. [See below]
\item The A-theory cannot characterize time. [From 5--6] 
\item Time is unreal. [From 4\&7]
\end{enumerate}



\subsection*{Argument  against the B-theory (Premise 3)}
Most assume that the existence of time requires the existence of change; if change is impossible, then so is time. This assumption places a constraint on an adequate theory of time, namely, an adequate theory of time must be compatible with the existence of change. This presents a problem for the B-Theory: 
\begin{enumerate}
\item If change exists, the passage of time must be real.
\item If the B-theory is true, the passage of time is unreal.
\item If change exists, the B-theory cannot be true. [From 1-2]
\end{enumerate} 
Why accept (1)?  Changes are events that have temporal duration. They also seem to be made up of smaller events with shorter durations. For instance, running a 400m race is an event that has temporal duration, but it is itself made up of shorter lasting events. There is the event of running the first 100m, the event of running the second 100m, and so on. But these events do not seem to `eternally' occur, i.e., the journalist shouts, 'the sprinter is \emph{now} on the second leg'. Later he shouts, 'the sprinter is \emph{now} on the third leg.'  The B-Theory cannot seem to accommodate this obvious feature of the sprinter's run. 
 

\subsection*{Argument against the A-theory (Premise 6)}

\begin{enumerate}
\item If the A-Theory is true, then a particular event E either (a) co-instantiates the properties of being past, present, and future, or (b) does not co-instantiate these properties. 
\item It is impossible for E to co-instantiate the properties of being past, present, and future.
\item It is impossible for E not to co-instantiate the properties of being past, present, and future. 
\item The A-Theory is not true. 
\end{enumerate}
Big questions:
\begin{itemize}
\item Do you accept (2)? If so, why? If not, why not? 
\item Why might we accept (3)? Recall that there is a worry about an infinite regress. What is this worry? A hint: it assumes a distinction between moments of time and the events occurring at those moments.
\end{itemize}


\subsection*{An alternative argument for P6}

\noindent Strategy: identify certain features that each motion must have and then show that what is called the passing of time cannot possess these features. 

\begin{enumerate} 
\item If time flows, then time moves. 
\item For any motion M, M is a motion with respect to time.
\item If time moves, then time is a motion with respect to time. (From 1 and 2) 
\item If Time-P is a motion with respect to time, then Time-P is a motion with respect to itself or with respect to some other type of time, Time-Q.  
\item Time-P cannot be a motion with respect to itself, Time-P
\item Time-P cannot be a motion with respect to Time-Q. 
\begin{itemize}
\item If it is of the essence of time that time passes, then Time-Q will pass as well, requiring some other type of time, Time-R, and so on \emph{ad infinitum}. 
\end{itemize}
\item Time-P is not a motion with respect to time. (From 4-6)
\item Time cannot move. (From 3 and 7) 
\item Time cannot flow. (From 1 and 8)
\end{enumerate}

\subsection*{Flow re-described}


\begin{itemize}
\item The B-theory claims that the universe consists in the spread of events in space time, i.e., all events exist irrespective of whether we describe them as being future, present, or past.  
\item These events have temporal relations to one another, i.e., they have B-properties. 
\item Events seen to change in their temporal properties, e.g., the event of you graduating seems to change from being future to being present to being past. 
\item Events cannot change their B-properties, so change cannot consist in change of temporal properties.  
\end{itemize}


\noindent Strategy: re-describe change in a way that is compatible with the B-theory. 

\begin{itemize}
\item Consider sailing down a long lake. You comment to your neighbor, `the river is really changing, it was so broad, now it is so narrow, and it will become narrower still. Also, did you see how dark the river was. But the color too is changing with the depth. It is becoming bluer and bluer as the day goes on'. 
\item These sentences speak of the river changing in both breadth and in color. Taken literally, they mean that one individual entity loses and gains the properties of being broad and narrow. Taken literally, these sentences are then false. But we should not take them literally. What appears to be one object losing and gaining various properties is really different parts of that object having different properties: one stage of the river is broad, a distinct stage of the river is narrow. 
\end{itemize}

\subsection*{Consequences}

What would B-theorists say about the following debates: 

\begin{enumerate}
\item Absolutism vs. Relationalism
\item Does the fact the Universe began entail that the time had a beginning? 
\end{enumerate}




\end{document} 

