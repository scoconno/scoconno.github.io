\documentclass[article,oneside]{memoir}
\usepackage{long table}
%%% custom style file with standard settings for xelatex and biblatex. Note that when [minion] is present, we assume you have minion pro installed for use with pdflatex.
%\usepackage[minion]{org-preamble-pdflatex} 

%%% alternatively, use xelatex instead
\usepackage{org-preamble-xelatex} 
\usepackage{ulem}
\usepackage{lscape}
\usepackage{multirow}
\def\myauthor{Author}
\def\mytitle{Title}
\def\mycopyright{\myauthor}
\def\mykeywords{}
\def\mybibliostyle{plain}
\def\mybibliocommand{}
\def\mysubtitle{}
\def\myaffiliation{NJCU}
\def\myaddress{Phil 1}
\def\myemail{soconnor@njcu.edu}
\def\myweb{\href{http://scottoconnor.org/ethics}{http://scottoconnor.org/ethics}}
\def\myphone{}
\def\myversion{}
\def\myrevision{}
\def\myaffiliation{NJCU}
\def\myauthor{Dr. Scott O'Connor}
\def\mykeywords{}
\def\mysubtitle{Syllabus}
\def\mytitle{{\normalsize \myweb \newline} \HUGE Pandemic Ethics}


\begin{document}

%%% If using xelatex and not pdflatex
%%% xelatex font choices
\defaultfontfeatures{}
\defaultfontfeatures{Scale=MatchLowercase}    
% You will need to buy these fonts, change the names to fonts you own, or comment out if not using xelatex.      
\setromanfont[Mapping=tex-text]{Minion Pro} 
\setsansfont[Mapping=tex-text]{Myriad Pro} 
\setmonofont[Mapping=tex-text,Scale=0.8]{Georgia} 

%% blank label items; hanging bibs for text
%% Custom hanging indent for vita items
\def\ind{\hangindent=1 true cm\hangafter=1 \noindent}
\def\labelitemi{$\cdot$}
%\renewcommand{\labelitemii}{~}

%% RCS info string for version tracking
\chapterstyle{article-3}  % alternative styles are defined in latex-custom-kjh/needs-memoir/
%\pagestyle{kjh}

\title{\LARGE \mytitle}     
\author{\Large\myauthor \newline \footnotesize\texttt{\noindent Office hours: \href{http://scottoconnor.org/contact/office/}{http://scottoconnor.org/contact/office/}}}
\date{1/18/2022--5/11/2022}

\published{\textbf{PHIL390 (2616), 3 credits, Spring 2022, M \& W 9:55am--11:10am, S102}}


\maketitle

%\thispagestyle{kjhgit}

% Copyright Page
%\textcopyright{} \mycopyright


%
% Main Content
%


\section{Disclaimer}
 This syllabus is subject to change at the discretion of the faculty. Students will be notified of such changes ahead of time via Blackboard. 

\section{Covid-19 Policies}
I am very much looking forward to the day when the pandemic is behind us, but we’re not quite there yet. In order to keep each other safe, please review and follow NJCU's Covid-19 related policies here:  
\begin{itemize}
\item \href{https://www.njcu.edu/student-life/campus-services-resources/health-wellness-center/fall-2021-welcome-back-campus}{https://www.njcu.edu/student-life/campus-services-resources/health-wellness-center/fall-2021-welcome-back-campus}
\end{itemize}
If you are unwell or think you have been exposed to Covid-19, do not attend class. If I test positive, must isolate due to exposure, or my children must isolate, I will teach the course on zoom through the length of my isolation. 


\section{Copyright}
The materials used in this class, including, but not limited to, lectures, exams, quizzes, and homework assignments are copyright protected works.  Any unauthorized copying of the class materials or recording of lectures is a violation of federal law and may result in disciplinary actions being taken against the student.  Additionally, the sharing of class materials without the specific, express approval of the instructor may be a violation of the University's Student Honor Code and an act of academic dishonesty, which could result in further disciplinary action.  This includes, among other things, uploading class materials to websites for the purpose of sharing those materials with other current or future students. 

\section{Course Description}

This course examines ethical issues connected with the COVID-19 pandemic, including the fair allocation of scarce medical resources, immunity passports, trade-offs between protecting senior citizens and allowing children to flourish, discrimination against minorities and the disabled, and the myriad issues raised by vaccines.

\section{Learning Objectives}

Upon completing this course students will be able to (i)  determine social, political, economic, philosophical, and other implications of the pandemic, (ii) demonstrate an understanding of the ethical issues raised by the Covid-19 pandemic, and (iii) analyze ways ethical questions about the pandemic arise in their respective disciplines.

\section{Readings}
\section{Required Textbooks}
Available in the campus book store and online retailers. 
\begin{itemize}
\item \href{https://broadviewpress.com/product/pandemic-bioethics/#tab-description}{`Pandemic Bioethics', Pence, Gregory}
\item Posted readings on the website.

\end{itemize}
\subsection{Optional}

\begin{itemize}

\item \href{http://www.amazon.com/Style-Lessons-Clarity-Grace-11th/dp/0321898680/ref=sr_1_1?ie=UTF8&qid=1452356026&sr=8-1&keywords=lessons+in+clarity+and+grace}{`Style: Lessons in Clarity and Grace', Joseph Williams and Joseph Bizup}, a good guide on how to improve your writing. 
\end{itemize}
\section{Course Website}
There is both a Blackboard site and website for this course (link on first page). Clicking the first link on the left panel within the Blackboard site will bring you to the course website. All assignments will be submitted through Blackboard. Readings, notes, etc. will be posted on the course website. Note that Blackboard difficulties are rare and automatically reported to instructors. Under no circumstance will a student's report of a Blackboard difficulty be reason for an extension. It is your responsibility to contact Blackboard support for help.


\section{Requirements}

\begin{itemize}


\item \textit{Workload:} expect to spend an average of 6 hours per week completing the readings and assignments. NJCU abides by the Federal and State definitions of a credit hour and adopts a policy consistent with the Carnegie Unit. A three-credit class represents 112.5 hours total of work. See \href{http://scottoconnor.org/resources/Credit.pdf}{here} for more details.

\item \textit{Participation:} Roll call will be taken. 0.5 point will be awarded per class up to a maximum of 10 points. Points will not be awarded during weeks 1 \& 2. Participation for classes held through zoom will be determined by reactions, chats, and other forms of engagement. Expect to be cold-called. Students should turn their cameras on. There are many more than twenty classes you can attend, so student are able to miss a small number of classes while still earning maximum participation points. In pre-Covid times, this buffer was intended to account for illness, personal emergencies, etc. However, I will accommodate students who have extended absences due to illness or covid-19 isolation. Students must notify me. 


\item \textit{Short essays} submitted through Blackboard. 250--500 words long. 3 will be assigned. You must complete 2. If you complete more than 2, the lowest grade will be dropped.
 
\item \textit{2 Presentations} on topics covered before spring break.   

\item \textit{1 final project} comprising a proposal, 1 progress report (750--1000 words), short presentation, and written submission (1200--1500 words.)

\item \textit{Course evaluations} completed online. 3 points extra credit for successful completion.


\item \textit{Grade Distribution:} Attendance--0.5 point per class (10 total); 2 short essays--10 points each (20 total); 2 short presentations--10 points each (20 total); 1 final project--10 points for the proposal, 10 point for the progress report, 10 points for the presentation, 20 points for the final submission (50 points total).




\item \textit{Grade Breakdown:}

 \begin{tabular}{ | l | l | p{2cm} | l | l | }
    \hline 
96--100 & A  & &  77--79 &  C+ \\  
90--95 & A- & &  73--76 & C \\
87-89 & B+ &  &  70--72 & C- \\ 
83--86 & B  & &  60--69 & D\\
80--82 & B - & & 0--59 & F\\ \hline
    \end{tabular}


\end{itemize}





\section{Policies}

\begin{itemize}

\item \textbf{Student Responsibility:} This syllabus outlines the required text, assignments, requirements, and policies for this course. By taking this course, you agree to read this syllabus and be bound by those requirements and policies. 

 \item \textit{Academic Integrity:} All the work you turn in (including papers, drafts, and discussion board posts) must be written by you specifically for this course. It must originate with you in form and content with all contributory sources fully and specifically acknowledged. Being a student at NJCU requires you to follow \href{http://scottoconnor.org/resources/Plagiarism.pdf}{NJCU's Academic Integrity Policy.} Penalties for violations are as follows: 1st infraction will result in a 0 for the assignment.  2nd infraction will result in a 0 for the entire course \& application for permanent record on student's transcript. (Repeated violations can lead to expulsion from NJCU). 

\item \textit{Attendance:} You are considered absent if you are (i) not present during roll call, (ii) leave early, (iii) leave without permission, or (iv) leave for an extended period of time. (iv) Are non-responsive to prompts during zoom classes when/if your camera is turned off, and I have no means of verifying that you are there. No excuses. No exceptions.



\item \textit{Communication:} To comply with Federal Privacy Laws (FERPA) and NJCU policies, all communication will be through Blackboard and/or official NJCU e-mail. Check Blackboard daily. For further information see \href{http://scottoconnor.org/contact/}{http://scottoconnor.org/contact/}.

\item \textit{Conduct:} Distracting and disrespectful behaviors, including but not limited to eating, leaving your seat, talking out of turn, and aggression are prohibited. Penalties include, but are not limited to, a loss of participation points for the day of violation. Repeat offenders will be reported to the Dean of Students. 

\item \textit{Electronic devices:} Use of electronic device, including, but not limited, to smartphones, dictaphones, tablets, and laptops, is prohibited. Recording a lecture is in violation of Copyright. Penalties include, but are not limited to, a loss of participation points for the day of violation. Repeat offenders will be reported to the Dean of Students.

\item \textit{Format for Written Work:} Submit work to Blackboard as either a pdf, rtf, or doc file. Blackboard will not allow any other format. All work must be typed and neatly presented. 


\item \textit{Grading:} Grades will be available within 1--2 weeks of an assignment being submitted. See: \href{http://scottoconnor.org/resources/grading}{http://scottoconnor.org/resources/grading} for further information.


\item \textit{Late work \& Make-up Policy:} See the assignment schedule below. No make-ups or late work accepted under any circumstances. No exceptions under any imaginable circumstances.

\item \textit{Statement for students with disabilities:} If you are a student with a disability and wish to receive consideration for reasonable accommodations, please register with the Office of Specialized Services and Supplemental Instruction (OSS/SI). To begin this process, complete the registration form available on the OSS/SI website at \href{http://www.njcu.edu/oss}{http://www.njcu.edu/oss} (listed under Student Resources-Forms). Contact OSS/SI at 201-200-2091 or visit the office in Karnoutsos Hall, Room 102 for additional information.
 

\item \textit{SafeAssign:} Students agree that by taking this course all assignments are subject to submission for textual similarity review through Blackboard SafeAssign. Assignments submitted to SafeAssign will be included as source documents in SafeAssign's restricted access database solely for the purpose of detecting plagiarism in such documents.  


\end{itemize}





\section{Weekly Course Schedule}
Readings are either from the required textbook or posted on the course website; these latter readings are marked with an asterisk. Changes to the syllabus will be announced through Blackboard and \emph{via} your NJCU email address.  All assignments must be submitted through Blackboard by Sunday, 11:59pm. No late work accepted. No exceptions.  

\newpage
\begin{landscape}
\begin{center}
\begin{longtable}{p{6cm}p{6cm}p{4cm}}
 
  \caption{Course Schedule} \\
  \toprule
  \textbf{Topic}    \textbf{Assignments} & \textbf{Reading} \\
  \midrule

  
[1.] Introduction (1/17) 						& 				&  --Ch.1 \\ 
\textbf{No class on Monday}					&				&   \\
Historical Epidemics							&				& \\
[1.8\baselineskip] \hline


[2.] Modern Pandemics (1/24)	  				&  				& --Ch.2\\
Library Wed.					 			&				&  \\ 
HIV/AIDS					  	 			& 	 			& \\ [1.8\baselineskip] \hline



[3.] Covid-19/SARS2 (1/31)					& Presentations 1		& --Ch.3   \\ 
Presentations Wed.							& 					& --Ch.8, pp.133--135\\ 
Frontline Experience							& 					& \\[1.8\baselineskip]  \hline 

[4.] Should we fear death? (2/7)  				&     	 				&  Hellenistic Phil., pp.3–31** \\
Epicurus					  				& 					&  \\ [1.8\baselineskip]  \hline % 


[5.] Containment Strategies (2/14)				& Assign. 1 			&  --Ch.4  \\
Lockdowns			    	    	 			&					&  --Ch.8, pp.135--138  \\ 
Utilitarians		    	    			 			&					&  --Ch.5, pp.77--79 \\  [1.8\baselineskip]  \hline 

[6.] Who should live?  (2/21)	   	 			&     					&  --Ch.11 \\
Documentary								&					&	 \\
No class on Mon.			     				&					&  	 \\ [1.8\baselineskip]  \hline 

[7.] Continued (2/28)							& Presentations 2		&  --Ch.5, pp.82--90 \\
Rawls							 		&	  				& \\  [1.8\baselineskip]  \hline

 (3/7) & &  \\
 (3/7) & &  \\ [1.8\baselineskip]  \hline

 							

[8.] Vaccine development (3/14)	 			& Proposal				& --Ch.6 \\ 
Challenge Trials				 	 		&						& --Ch.5, pp.81--82  \\ 
Kant										&						&    	\\ [1.8\baselineskip] \hline		

[9.] Continued	(3/21)			 			&  						&  --Ch.8, pp.138--141\\ 
										&						&    	\\ [1.8\baselineskip] \hline	 

[10.] Vaccine allocation (3/28)		 			& Assign. 2.				& --Ch.7  \\
	    									&      						&  \\  [1.8\baselineskip] \hline


[11.] Individual Liberty (4/4)					& Progress Report			& --Ch.9  \\ 
Libertarianism								&						& 	 \\ [1.8\baselineskip] \hline				


[12.] Immunity passports  (4/11) 	 			&  			    			 & --Ch.10 \\ [1.8\baselineskip] \hline

[13.] Covid Leadership  (4/18) 					&  Assign. 3.    		  		 & --Ch.12   \\ [1.8\baselineskip] \hline


[14.] The Future  (4/25) 			 			& 		     				 & --Ch.13   \\ [1.8\baselineskip] \hline


[15.] Research Presentations	(5/2)			 	&						&  \\ [1.8\baselineskip] \hline

[16.] Finals Week (5/11)						&						 & Final Assignment \\ [1.8\baselineskip] \hline
	

\end{longtable}
\end{center}
\end{landscape}
\end{document}


%% Uncomment if you want a printed bibliography.
%\printbibliography 


