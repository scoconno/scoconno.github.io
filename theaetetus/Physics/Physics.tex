\documentclass[10pt, oneside]{book}
\usepackage[pdftex]{graphicx}
\usepackage[polutonikogreek,english]{babel}
\usepackage[utf8x]{inputenx}
\usepackage[T1]{fontenc}
\usepackage{fancyhdr}
\pagestyle{fancy}
\lhead{{\textsc{The Science of Nature}}}
\chead{}
\rhead{{\textsc{Scott O'Connor}}}
\lfoot{\date}
\cfoot{}
\rfoot{\thepage}
\tolerance=700
\setcounter{secnumdepth}{3}
\newcommand{\greek}[1]{{\selectlanguage{polutonikogreek}#1}}
\font\largefont= cmr17 scaled \magstep5
\def\drop#1#2{{\noindent
 \setbox0\hbox{\largefont #1}\setbox1\hbox{#2}\setbox2\hbox{(}%
 \count0=\ht0\advance\count0 by\dp0\count1\baselineskip
 \advance\count0 by-\ht1\advance\count0by\ht2
 \dimen1=.5ex\advance\count0by\dimen1\divide\count0 by\count1
 \advance\count0 by1\dimen0\wd0
 \advance\dimen0 by.25em\dimen1=\ht0\advance\dimen1 by-\ht1
 \global\hangindent\dimen0\global\hangafter-\count0
 \hskip-\dimen0\setbox0\hbox to\dimen0{\raise-\dimen1\box0\hss}%
 \dp0=0in\ht0=0in\box0}#2}
\newcommand{\versal}[1]{{\noindent
 \setbox0\hbox{\largefont #1}%
 \count0=\ht0                   % height of versal
 \count1=\baselineskip          % baselineskip
 \divide\count0 by \count1      % versal height/baselineskip
 \dimen1 = \count0\baselineskip % distance to drop versal
 \advance\count0 by 1\relax     % no of indented lines
 \dimen0=\wd0                   % width of versal
 \global\hangindent\dimen0      % set indentation distance
 \global\hangafter-\count0      % set no of indented lines
 \hskip-\dimen0\setbox0\hbox to\dimen0{\raise-\dimen1\box0\hss}%
 \dp0=0in\ht0=0in\box0}}
\setcounter{secnumdepth}{1}
\author{Scott O'Connor}
\begin{document}
\chapter{The Science of Nature}





\section{Introduction}
Aristotle begins the first chapter of the first book of the \emph{Physics} as follows:

\begin{quote}
(a) Since knowledge and scientific understanding in every line of inquiry that has principles, or explanations,%
\footnote{Interpreters debate how best to translate  `\greek{αἴτια}'. See Hocutt (1974), Frede (1980), Freeland (1995), Moravcsik (1995), Hankinson (1998). I translate the word as `explanation' and not `cause'. And I use `explanations' for those properties, objects, and states of affairs that we appeal to when answering questions about the world, rather than those sentences and propositions we use to talk about these entities.}
%
 or elements, (b) comes from grasping these,%
%
\footnote{Aristotle uses three different epistemic words here `\greek{εἰδέναι}', `\greek{ἐπίστασθαι}',  and `\greek{γιγνώσκω}'. Ross (1936), and Irwin and Fine (1995) translate `\greek{ἐπίστασθαι}' as `scientific knowledge'. I translate it as `knowledge', `understanding', and `scientific understanding' without marking a difference between these English words.    For discussion. see Burnyeat (1981), Barnes (1975, 1994).} 
%
 - (c) for we think we know a thing when we grasp its first explanations and first principles and grasp as far as its elements - (d) it is clear that also in the scientific understanding of nature, 
(e) that we must first determine the things concerning the principles (184a11-6).%
%
\footnote{\greek{Ἐπειδὴ τὸ εἰδέναι καὶ τὸ ἐπίστασθαι συμβαίνει περὶ πάσας τὰς μεθόδους, ὧν εἰσὶν ἀρχαὶ ἢ αἴτια ἢ στοιχεῖα, ἐκ τοῦ ταῦτα γνωρίζειν (τότε γὰρ οἰόμεθα γιγνώσκειν ἕκαστον, ὅταν τὰ αἴτια γνωρίσωμεν τὰ πρῶτα καὶ τὰς ἀρχὰς τὰς πρώτας καὶ μέχρι τῶν στοιχείων), δῆλον ὅτι καὶ τῆς περὶ φύσεως ἐπιστήμης πειρατέον διορίσασθαι πρῶτον τὰ περὶ τὰς ἀρχάς.}} 
\end{quote}
Aristotle concludes the ninth and last chapter of the first book of the \emph{Physics} as follows:
\begin{quote}
(f) On the one hand, then, that there are principles, what they are, and how many they are has been determined by us in this way, (g) on the other hand, beginning backwards <from these points> let us speak of a different principle  (192b2-4).%
%
\footnote{\greek{ὅτι μὲν οὖν εἰσὶν ἀρχαί, καὶ τίνες, καὶ πόσαι τὸν ἀριθμόν, διωρίσθω ἡμῖν οὕτως· πάλιν δ’ ἄλλην ἀρχὴν ἀρξάμενοι λέγωμεν.} We can translate (g) in different ways. Irwin and Fine translate: 'Let us now continue, after making a fresh start'. See Appendix A for a brief discussion.}  
\end{quote}
%
In (d) Aristotle announces his topic: the science of nature. In (e) he says that he will first determine (\greek{διορίζω}) the things concerning the principles (\greek{ἀρχαί}) of nature.  And in (f) he tells us that this means we must determine whether there are principles, and, if there are principles, we must determine how many they are and what they are. 

Between these opening and closing lines of the first book of the \emph{Physics}, Aristotle systematically answers these three questions. In chapters 2 to 4, Aristotle argues against his predecessors' views of the principles and in chapters 5--9 he develops his own account: there are principles and these principles are matter, form, and privation.\footnote{The English world `principle’ often is used to describe certain propositions, e.g. Newton's third law of motion is a principle of motion. Aristotle uses 'archai' for both propositions and the objects of the propositions. Analogously, we might call force a principle of motion. In this text, `principle' is used for non-propositional entities. } I introduce subject, privation, and form as follows: Aristotle believes that in any process of change some subject acquires a form it previously lacked. And before the change the subject possesses the privation of that form. So form and privation are the two end-points of the change, and they stand to each other as opposites. \footnote{This is true also in cases in which change does not start from one of the extremes or does not reach an extreme, these are intermediates.}



\section{The Science of Nature}

In (a) Aristotle says that there are some lines of inquiry (\greek{μεθόδος}) that have principles, explanations (\greek{αἴτια}), or elements (\greek{στοιχεῖα}). By `line of inquiry', Aristotle means a branch of study. For instance, geometry and biology are each different branches of study. Branches of study each inquire into some unique subject. Geometers study the mathematical limits of bodies. Biologists study living organisms. Living organism and the mathematical limits of bodies are different subjects. So geometry and biology are different branches of study.\footnote{Aristotle claims that each demonstrative science is concerned with three things: a certain kind of thing, the \emph{per se} attributes of that kind of thing, and the axioms by which the science demonstrates that these attributes belong to things of that kind (\emph{A.Po}.76b11-16). }  

In (d) Aristotle says that his focus is the science of nature. Just as the biologist studies some unique subject, so too the scientist of nature must study some unique subject. But what does the science of nature study?\footnote{Elsewhere Aristotle says that we cannot fully explain what a science is unless we already possess that science (\emph{EE}.1216b2-16). For example, we can only explain what the science of geometry is \emph{of} when we possess geometry. Aristotle claims that if we are to know that geometry is the science of limits we must know what limits are. But if we know what limits are, we would thereby be geometers. Similarly, Aristotle does not assume that he knows what the science of nature is \emph{of} because he has yet to establish that there is such a thing as nature, let alone show what nature is. See Stephen Menn (unpublished) I\greek{a}2 for discussion of these points. } Aristotle indicates what the study of nature involves as follows:
 
\begin{quote}
We can assume that some or all natural beings are changeable; this is clear from induction (185a12-14).\footnote{\greek{ἡμῖν δ’ ὑποκείσθω τὰ φύσει ἢ πάντα ἢ ἔνια κινούμενα εἶναι· δῆλον δ’ ἐκ τῆς ἐπαγωγῆς.} I translate the Greek phrase `\greek{τὰ φύσει}' as `natural beings', but it could also be translated as `being by nature' or `beings in nature'.  I assume that beings that are by nature and beings that are in nature just are natural beings.}
 
It has been said, then, how many principles of natural beings concerning change there are, and it has been said in what way they are this many (191a3-4).\footnote{\greek{πόσαι μὲν οὖν αἱ ἀρχαὶ τῶν περὶ γένεσιν φυσικῶν, καὶ πῶς ποσαί, εἴρηται·}}
\end{quote}
%
The first quote comes from the second chapter. There Aristotle discusses Parmenides and Melissus who deny that there is a science of nature altogether.%
%
\footnote{Sextus Empiricus explains Aristotle's inclusion of Parmenides and Melissus by quoting from dialogue that is now lost to us:
\begin{quote}
Its existence [i.e. the existence of change] is denied by Parmenides and Melissus, whom Aristotle has called people who deny change and unnaturalists‚ the former because they maintain that things are unchangeable, unnaturalists because nature is a source of change and in saying that nothing change they abolished nature (dversus mathematicos X 46).
\end{quote}
%
Parmenides and Melissus believe that change does not exist. But Aristotle believes that nature is a principle of change. Since they deny that change exists, Aristotle believes that they must also deny that nature exists. See Appendix A for discussion of nature as a principle of change.} 
 In this chapter, Aristotle tells us that the science of nature is the study of natural beings and  change. He says  that we must assume that some, if not all, natural beings are changeable. And by `changeable' he means that they are either changing or that they are capable of changing.

So Aristotle tells us two things. First, he tell us that the natural scientist studies natural beings. Second, he tells us that some, if not all, natural beings are changeable.  Thus, the natural scientist studies beings that change. This claim could be understood in different ways. First, `beings that change' could be taken merely extensionally: whatever this phrase picks out, the natural scientist studies that. Understood in this way, even the mathematician studies natural beings because the mathematician studies surfaces, solids, lengths and points and all of these are possessed by natural beings, beings that change (193b22-25). 

But in the second quote Aristotle says that he is searching for the principles of natural beings concerning change.\footnote{David Bostock \citep[p.~1]{BostockESSAYS} claims that Aristotle seems to search for `the ultimate constituents of reality' and explain how reality is generated from those constituents. He subsequently denies that Aristotle is interested in the principles of natural beings. Rather, Aristotle is interested in natural processes or changes and in particular generations. This is a false dichotomy. Aristotle believes that all changes involve a natural being that changes. The principles he seeks are the principles of natural beings insofar as they change. But they are also principles of changes, but only because they are the principles of those natural beings that change. Bostock, perhaps, notes this when he says that there is a sense in which the principles are `ingredients' of natural beings (p2).}  This quote comes from the seventh chapter where Aristotle explains what the principles are and how they are related. There he says that the principles are not just any principles of natural beings, they are precisely those principles of natural beings that concern change.

%\footnote{In the second quote, Aristotle seems to say that all natural beings change while in the first quote he said we must assume that some or all natural beings change. So earlier he left open that there might be some natural beings that do not change, while later he does not leave it open. He never argues that there are no non-changeable natural beings, and I am unsure how to understand why he omits this argument. Perhaps we should read the second quote as follows: of those natural beings that change, we must seek the principles of these natural beings concerning change. Aristotle then focuses only on natural beings that change, but leaves open that there are other natural beings that do not change. Aristotle in Met.E 1026a6-22 says that physics differs from maths and theology. Physics treats changeable substance while theology treat changeless substance. iF there were no non-moveable things, physics would be first philosophy. The principles of physics would be the principles of everything. The changes a thing undergoes depend on external factors and inherent characteristics of hte object itself.

%Meteor 1.1: We have already discussed the first causes of nature, and all natural motion, also the stars ordered in the motion of the heavens, and the corporeal elements, and becoming and perishing in general. There remains for consideration a part of this inquiry whihc all our predecessors called meeorology. It is concerned with events that are natural, though their order is less perfect than that of the first element of bodies...When the inquiry into these matters is concluded, let us consider what account we can give, in accordance with the method we have followed, of animals and plants, both generally and in detail. 338a20-339a8}  

So natural beings change. The natural scientist studies these natural beings, and this involves studying the change of these beings. We can express this as follows: the natural scientist studies natural beings \emph{insofar} as they change. In contrast, the mathematician studies natural beings, but does not study these beings \emph{insofar} as they change. 

In \emph{Phy}.1.7, Aristotle also speaks about two kinds of change: qualified change and unqualified change (\greek{γίγνεται ἁπλῶς})(190b1-3).  During a qualified change, a substance alters but no substance comes into being or goes out of being. In contrast, during an unqualified change a substance does come into being or goes out of being. For example, warming, walking, and growing are qualified changes. When Socrates gets warm, walks, or grows, no substance comes into being or goes out of being. Socrates merely becomes warm from being cold (or something in between), moves from one place to another place, or grows tall from being short. In contrast, sculpting, baking, and rotting are unqualified changes. When a tree rots, a substance does go out of being: that very tree goes out of being.

So when a natural scientists studies natural beings, they study these two different kinds of change. And they seek those principles of natural beings that concern these two kinds of change;  both qualified and unqualified changes.

But what does studying natural beings insofar as they qualifiedly and unqualifiedly come to be involve? Aristotle believes that there are many natural sciences. Both biology and meteorology are studies of natural beings. Biologists study living organisms. Meteorologists study non-living organisms, e.g. they study mountains and rain.\footnote{Aristotle includes geography, geology, the study of the atmosphere, and the study of the weather, as parts of meteorology.]} But they both study natural beings and they both study change. So how does the science of nature differ, if at all, from these natural sciences? 

The relevant difference is this: the biologist studies some natural beings, but only living natural beings. They seek the principles of living beings insofar as these beings change. But they do so only insofar as these changes are biological changes, e.g. the changes involved in reproduction and nutrition. In contrast, meteorologists study some natural beings, but only meteorological beings. And they too seek the principles of meteorological beings insofar as they change. But they do so only insofar as these changes are meteorological, e.g.  whirlwinds (\emph{Meteo}. 371a9-15).

So the meteorologist and biologist both seek the principles of natural beings insofar as those beings change. But they both search for different principles. For they study different types of natural beings, and they study different kinds of qualified and unqualified changes.

In contrast, Aristotle searches for the principles that belong to all natural beings, i.e. to natural beings insofar as they are natural beings. And these principles are not principles that concern biological changes. But they are the principles that concern all changes, i.e. the principles that apply generally to all types of qualified and unqualified changes.  

\section{The Principles of Nature}

But why must we grasp principles? In (a) Aristotle says that some branches of study have principles, explanations, or elements. And  in (b) Aristotle says  that we will know about whatever we study in these disciplines by grasping its principles, explanations, or elements. In (c) he adds that we must grasp not just any principles and explanations, but we must grasp the first principles, first explanations, and grasp as far as its elements. So a natural scientist will only gain knowledge about natural beings insofar as they change if they grasp the first principles, first explanations, and elements of natural beings insofar as they change.

Unfortunately, Aristotle explains little about why we must seek principles, explanations, and elements. He says little about what these are and explains little about how grasping them allows us know about whatever subject we are inquiring into. Aristotle seems to believe that his reader will understand what he says in these opening lines without need of him to explain in fuller detail. Fortunately, what Aristotle says here he says and explains elsewhere in greater detail, in particular, in the \emph{Posterior Analytics} (\emph{A.Po}.  And we can understand the project of this first book of the \emph{Physics} easier by highlighting some salient points from that works.  

 Aristotle begins the \emph{A.Po} by claiming that ``all teaching and all learning of an intellectual kind proceed from pre-existing knowledge''(71a1-3)\footnote{\greek{Πᾶσα διδασκαλία καὶ πᾶσα μάθησις διανοητικὴ ἐκ προϋπαρχούσης γίνεται γνώσεως. φανερὸν δὲ τοῦτο θεωροῦσιν ἐπὶ πασῶν·} (I have consulted Irwin and Fine, and Barnes for their translation of \emph{A.Po})}. So when a biologist inquires about living organisms she must do so from prior-knowledge, or prior-grasping (\greek{γνῶσις}).\footnote{I assume that `\greek{γνώσiς}' is more general than and includes `\greek{ἐπιστήμη}'. I use `prior-knowledge' because `prior-grasping' is somewhat awkward. But note that Aristotle does not say that all teaching and all learning require prior \greek{ἐπιστήμη}.}  In particular, Aristotle argues that scientific inquiry requires a special kind of prior-knowledge: he distinguishes between knowing facts about a subject and knowing explanations of those facts, and he believes that scientific inquiry is specially concerned with the latter. He explains as follows:

\begin{quote}We think we understand (\greek{ἐπίστασθαι}) something unqualifiedly, and not in the sophistic coincidental way, whenever we think we know the explanation because of which the thing is <so> that it is the explanation of that thing, and know that it is not possible for it to be otherwise  (\emph{A.Po}. 71b9-16).\footnote{\greek{Ἐπίστασθαι δὲ οἰόμεθ’ ἕκαστον ἁπλῶς, ἀλλὰ μὴ τὸν σοφιστικὸν τρόπον τὸν κατὰ συμβεβηκός, ὅταν τήν τ’ αἰτίαν οἰώμεθα γινώσκειν δι’ ἣν τὸ πρᾶγμά ἐστιν, ὅτι ἐκείνου αἰτία ἐστί, καὶ μὴ ἐνδέχεσθαι τοῦτ’ ἄλλως ἔχειν.}}
\end{quote}
%
So Aristotle thinks that we know in the strictest way that certain things are so when we know why those things are that way.   For example, all humans must possess hearts. Knowing that humans must have hearts differs from knowing why humans must have hearts. And Aristotle believes that biologists must inquire into and come to understand why humans must have hearts. For he thinks that a biologist can only strictly know that humans must have hearts by knowing the explanation for this fact, e.g. humans must have hearts because humans possess the faculty of nutrition and a heart is the principle of that faculty. 
 
This helps us understand why Aristotle says that in any line of inquiry that has first principles, we gain knowledge by grasping these principles. Aristotle claims of first principles:
\begin{quote}
It is through them and from them [first principles] that the other things are known; but they are not known through the things under them (\emph{Met}.982b2-4, \emph{c.f. A.Po}. 76a31-32).\footnote{\greek{διὰ γὰρ ταῦτα καὶ ἐκ τούτων τἆλλα γνωρίζεται ἀλλ’ οὐ ταῦτα διὰ τῶν ὑποκειμένων}}%
\end{quote}
% 
So first principles just are those ultimate and fundamental explanations that the scientist seeks. For instance,  a biologist searches for the first principles of biology because these are the first and ultimate explanations of living organisms. And Aristotle thinks that only when the biologist comes to know these first principles will she strictly know her subject. For it is only then that the she will have grasped the basic fundamental explanations from which she can explain all the facts about her subject. 

Similarly, Aristotle thinks that we can only fully understand natural beings insofar as they change if we grasp those principles of natural beings that explain why they change. For it is only then that we will have grasped the fundamental explanations from which we can explain all the facts about natural beings insofar as they change.

Now  in (a-c) Aristotle speaks about principles, but also about explanations, and elements. And he says that we must grasp both first principles, first explanation and grasp the elements.\footnote{I assume that the `or' is not disjunctive.} We can understand what Aristotle says here in different ways. 

First, Aristotle could use `principles', `explanation', and `elements' interchangeably. On this reading, Aristotle says there is just one thing that we must grasp; something which we can describe as a `principle', `explanation', and `element'. \footnote{Charlton (1970, 1992) accepts this reading.} Alternatively, Aristotle could use these words to describe three different things. On this reading, Aristotle tells us that we must grasp three different things.

To decide between these two readings let us note the following about principles, explanations and elements: Aristotle believes that there are four explanations (or causes), the final, efficient, formal and material (\emph{Phy}.2.3). And Aristotle thinks each of these causes is a principle. For instance, he says in \emph{Met}.D that  ``causes can be spoken of in an equal number of ways; for all causes are origins'' (1013a16-17).\footnote{\greek{ἰσαχῶς δὲ καὶ τὰ αἴτια
λέγεται· πάντα γὰρ τὰ αἴτια ἀρχαί}} But Aristotle does not believe that every explanation (cause) is an element. He says that while `element' can be said in different ways, elements are always  what are first, or primarily, present in each thing of which they are an element.\footnote{\greek{ἁπάντων δὲ κοινὸν τὸ εἶναι στοιχεῖον ἑκάστου τὸ πρῶτον ἐνυπάρχον ἑκάστῳ.}} So, for instance, syllables are the elements of speech, water is an element of blood, and so on. 

So Aristotle believes that some elements are explanations (or causes), but not every explanation (cause) is an element. For instance, water is an element of a cake and is also the material cause of a cake. But the baker too is a cause of the cake. For the baker is the efficient cause of the cake. Nevertheless, the baker is not present in the cake, and is not an element of that cake.   

But this does not mean that Aristotle says we must grasp three different things. I suggest that we read the `or' in (a) `principle, or explanations, or elements' restrictively in the following way: `principles, i.e. explanations, i.e. elements.' Aristotle then tells us that he will inquire into certain principles and explanations, namely elements. We then do not expect Aristotle in Book 1 to speak about each of the four causes, but only those that are present in whatever they are causes of. And while we may not find Aristotle explicitly saying that he sees this relation between principles, explanation, and elements in this book, he clearly makes this point in \emph{Met}.\greek{Λ}:
\begin{quote}
Since not only what is present in each thing [the elements] are causes, but also what is external, for example, the moving cause, it is clear that while principle and element are different, both are causes, and principle is divided into these two kinds, and that which moves a thing or makes it rest is a principle and substance.Therefore, on the one hand ,there are three elements by analogy, and there are four principles and causes (1070b22-26).\footnote{\greek{ἐπεὶ δὲ οὐ μόνον τὰ ἐνυπάρχοντα αἴτια, ἀλλὰ καὶ τῶν ἐκτὸς οἷον τὸ κινοῦν, δῆλον ὅτι ἕτερον ἀρχὴ καὶ στοιχεῖον,
αἴτια δ’ ἄμφω, καὶ εἰς ταῦτα διαιρεῖται ἡ ἀρχή, τὸ δ’ ὡς κινοῦν ἢ ἱστὰν ἀρχή τις καὶ οὐσία, ὥστε στοιχεῖα μὲν κατ’ ἀναλογίαν τρία, αἰτίαι δὲ καὶ ἀρχαὶ τέτταρες·}}
\end{quote}
Here Aristotle repeats his views from \emph{Phy}.1. He argues that form, privation, and matter are the three elements of natural beings insofar as they change. But he says that while they are each principles, and thus causes, they are not the only causes. In particular, the efficient cause is a principle of natural beings insofar as they change but the efficient causes is external to a natural being and thus not an element.

So in (e) Aristotle says that we must determine the things concerning the principles.   In (f) he tells us that he has determined that there are principles, what they are, and how many they are.\footnote{Some interpreters say that in (a)-(e) Aristotle assumes that there are first principles for the scientist of nature to find, e.g.  Simplicius (1882), Philoponus, (1888). But in (f) Aristotle says  that he has shown that there are principles. So Aristotle does not assume that there are principles. He argues that there are principles.} But  in (e) and (f) Aristotle  speaks only about some principles of natural beings insofar as they change, the elements.  And he answers that there are elements and these elements are privation, form, and subject.\footnote{In ch.5-7, I discuss how Aristotle believes these elements are present in natural beings.}

So Aristotle believes that subject, privation, and form are elements and principles of natural beings precisely because they explain why natural beings change. But in what sense are they explanations? Robert Bolton helpfully illustrates Aristotle's claim by comparing it to geometry. In \emph{A.Po} Aristotle says that geometers must assume that two-dimensional magnitudes exist (I. 10, 76a31-6); it is not the job of the geometery to prove that there are these magnitudes. Nevertheless, as Bolton explains, we can still spell out the claim that two-dimensional magnitudes exist with the claim that points and lines exist (76b3-6):

\begin{quote} 
This fills out the content of the principle that two-dimensional magnitude exists because these are the basic objects which make all geomterical magnitude possible. All gemoterical objects are constructible out of points and line, but the latter are not constructible out of, or otherwise reducible to, each other or to any more fundamental entities.\footnote{Bolton p22}
\end{quote}
According to Bolton, Aristotle believes that geometrical objects exist because points and lines exist and these geomterical objects are in someway constructible out of points and lines. So in this sense lines and point explain why geometrical objects are possible. Bolton suggests we should understand subject, form, and privation similarly. Natural beings insofar as they change exist only because subject, form and privation exist. So in a sense subject, form, and privation are explanations for natural beings and change are possible.\footnote{Bolton p23} 

So Aristotle believes that what is for something to be a natural being is to be a being that changes. But he thinks that a necessary condition for any change is that there is some both some subject, privation and form. So since all three are necessary for change, they are required for natural beings to exist. For if subject, privation and form did not exist, then change could not exist. And if change does not exist, then natural beings could not exist. For natural beings are precisely beings that change. 

In this dissertation, I wish to understand just why and how natural beings are constructible out of subject, privation, and form: why does Aristotle think that each must exist if change, and hence natural beings are to exist? Alternatively, we can ask this quesiton as follows: just how does form, privation, and subject make change and hence natural beings possible? 

In order to know how these make change possible, we need to know what must be explained by change? People think that the facts that need explaining are: something persists through a change. For instance, Waterlow thinks he wants to know what is it that explains why the dolphin persists as the very dophin it is. But this won't explain why form and privation are important. You must have one or more principles, they must be changeless or subject to change, if there is more than one, they must be finite or infinite.


\section{The Search for Principles}

Aristotle believes that all learning and inquiry requires prior knowledge. So inquiry and learning first principles too requires prior knowledge. But what is this prior knowledge and how do we come to know first principles? We saw above that Aristotle believes we know thunder only by knowing the explanation for thunder; fire being extinguished in the clouds. But we also saw that Aristotle denies that first principles have first principles (at least first principles of the same science). So the meteorologist cannot come to know the explanation for fire being extinguished in the clouds by grasping some first principle.\footnote{Aristotle believes that all demonstration must come to an end.} So if we do not come to know first principles by grasping their principles, but all learning requires pre-existing knowledge, how can we learn the first principles?  Aristotle answers this question in \emph{Phy}.1.1. After saying that we must inquire into principles, he says:

\begin{quote}
(a) The natural path is to start from what is better known (graspable) and clearest to us, and to (b) advance to what is more clearest and better known by nature; (c) for what is better known to us is not the same as what is better known without qualification. (d) We must advance in this way, then, from what is less clear by nature but more clear to us, to what is more clear and better known by nature (184a16-20).\footnote{\greek{πέφυκε δὲ ἐκ τῶν γνωριμωτέρων ἡμῖν ἡ ὁδὸς καὶ σαφεστέρων ἐπὶ τὰ σαφέστερα τῇ φύσει καὶ γνωριμώτερα· οὐ γὰρ ταὐτὰ ἡμῖν τε γνώριμα καὶ ἁπλῶς. διόπερ ἀνάγκη τὸν τρόπον τοῦτον προάγειν ἐκ τῶν ἀσαφεστέρων μὲν
τῇ φύσει ἡμῖν δὲ σαφεστέρων ἐπὶ τὰ σαφέστερα τῇ φύσει καὶ γνωριμώτερα.}}
\end{quote}
%
Aristotle believes that if we are to inquire into the first principles of natural being concerning change we must do so from some previous knowledge (grasping). In this quote, Aristotle partially describes this prior-knowledge. In (a) he describes what this prior-knowledge is \emph{of}. It is of what is better known and more clear to us. So from grasping these things we can inquire into and come to grasp the first principles. 

Aristotle in (b) describes these first principles as better known and clearest by nature. By `more clear and better known by nature', Aristotle means that these entities are apt for knowing.\footnote{Irwin } For Aristotle believes that principles are essentially explanations and grasping them will involve grasping that they are explanations.\footnote{Armstrong disagrees} For instance, if water is the principle that explains why natural things come to be, then grasping water will involve grasping that water plays this role.And Aristotle in (d) emphasize that the first principles differ from what we know by grasping these first principles. For the first principles of, say, thunder are not thunder. Rather, they are the explanations for thunder.


So Aristotle believes that principles are clear and known by nature but they are not clear and known to us. He does not yet say what these things are that are clear to us but not by nature. But it is clear that Aristotle thinks that if we are to inquire into the principles we must begin from some grasp of something else. While Aristotle does not describe the method that we must use to get from what is clear to us to grasp of the principles, he is clear that there is some appropriate method, or natural road, to gain knowledge from one from an initial grasp of the latter.

Now I said that we can understand the project of the first book in two very different ways. We can as Aristotle as either:
\begin{enumerate}
\item Determining those things which are clearest and more knowable to us.
\item Determining those things which are clearest and more knowable by nature.  
\end{enumerate}

These are different readings. On the first reading,  Aristotle only identifies those starting points from which we can come to learn the first principles.   On the second reading, Aristotle first identifies those starting points \emph{and} investigate and grasps the first principles after identifying these starting points. If Aristotle's project is the latter, then by the end of the first book we expect Aristotle to be in a state of knowledge about natural beings and change for Aristotle would be be reporting that he grasped the first principles which is sufficient for knowledge. But on the former reading, we do not expect Aristotle to be in a state of knowledge about natural beings and change. For Aristotle will not be reporting that he has have explained change. In contrast we expect Aristotle reports  only to have identified those adequate starting points from which we can inquire into and come to learn the first principles of natural beings and change. 

We can see that we must decide between 1 and 2 by examining how Aristotle describes those things clearer and more knowable to us but not in nature in the rest of \emph{Phy}.1.1:

\begin{quote}
The things, that, most of all, are initially clear and perspicuous to us are inarticulate; later, as we articulate them, the elements and principles come to be known from them. We must, then, advance from universals to particulars; for the whole is better known in perception, and the universal is a sort of whole, since it includes many things as parts. The same is true, in a way, of names in relation to their accounts. For a name - for instance, `circle' - signifies a sort of whole and signifies indefinitely, whereas the definition <of a circle> articulate it by stating the particular properties. Again, children begin by calling all men father and all women mother; only later do they distinguish different men and different women.\footnote{\greek{ ἔστι δ’ ἡμῖν τὸ πρῶτον δῆλα καὶ σαφῆ τὰ συγκεχυμένα μᾶλλον· ὕστερον δ’ ἐκ τούτων γίγνεται γνώριμα τὰ στοιχεῖα καὶ αἱ ἀρχαὶ διαιροῦσι ταῦτα. διὸ ἐκ τῶν καθόλου ἐπὶ τὰ καθ’ ἕκαστα δεῖ προϊέναι· τὸ γὰρ ὅλον κατὰ τὴν αἴσθησιν γνωριμώτερον, τὸ δὲ καθόλου ὅλον τί ἐστι· πολλὰ γὰρ περιλαμβάνει ὡς μέρη τὸ καθόλου. πέπονθε δὲ ταὐτὸ τοῦτο τρόπον τινὰ καὶ τὰ ὀνόματα πρὸς τὸν λόγον· ὅλον γάρ τι καὶ ἀδιορίστως σημαίνει, οἷον ὁ κύκλος, ὁ δὲ ὁρισμὸς αὐτοῦ διαιρεῖ εἰς τὰ καθ’ ἕκαστα. καὶ τὰ παιδία τὸ μὲν πρῶτον προσαγορεύει πάντας τοὺς ἄνδρας πατέρας καὶ
μητέρας τὰς γυναῖκας, ὕστερον δὲ διορίζει τούτων ἑκάτερον.}} 
\end{quote}
%
Aristotle tell us what are those things better know to us and better known by nature. He describes the things clearer to us as inarticulate, or put together \greek{συγκεχυμένα} , universal \greek{καθόλου}, a whole \greek{ὅλος} ,and what is better known to perception. And he describes the things better known by nature, the principles, as particulars (\greek{τὰ καθ’ ἕκαστα}). So Aristotle believes that we learn the principles by advancing from something universal, inarticulate, a whole, and closer to perception. 

We may struggle to understand what Aristotle says here because it seems to conflict with what he says in \emph{PA}. There Aristotle says that what is better known to us are particulars. These are things that we perceive. And he says that what is better known in nature, the principles, are universal. So in \emph{PA} he says that we learn the principles by advancing from the particular to the universal, while here in the \emph{Phy} we learn the principles by advancing from the universal to the particular.

We may try to resolve this tension as follows: we may understand the `put together'  as `compounded', and `wholes' as `particulars', and `universal' as a certain kind of particular. Aristotle then would also say in the \emph{Phy} that we learn principles by first perceiving particulars. Nevertheless, we would then struggly to understand why Aristotle calls the principles also particulars.  So this seems a poor way of reconciling both works.

Most interpreters agree that what Aristotle says in both works is consistent, and they seem to agree how to reconcile it, though they disagree about the detail. 

Aristotle contrasts `what is put together' and `universal' on the one hand, and what is particular on the other. Interpreters seem to agree that Aristotle does not distinguish between universals and individuals on the one hand, but between two different levels of generality. Both `what is put together' and what is particular are universal, but universal in some different way. On the one hand, `what is put together' should be understood as an inarticulate universal, while the particular should be understood as a clearer universal. What is common. Aristotle uses this in 3.1 and 1.7.

For instance, Thomas Aquinas suggests that the particulars (and hence principles) are genera, while the universals are more general than genera. So, for instance, Aquinas would say that body is something which belongs to many genera of things, but is not itself a genus. Body is a more general type than a genus. In contrast, animals applies to many things but animal is a genus and not a universal. So a theory of the principles that is general and indiscrimate as regards the peculiar principles of each things - matter as predicate of matter.  

So on Aquinas's reading, Aristotle says that we can only inquire into principles by first grasping some highly general type. We inquire into principles by distinguishing the different genera that fall under this type. For instance, we may first grasp that all natural things have a body. But having a body does not distinguish natural beings into distinct genera. Nevertheless, after first grasping that natural beings have bodies, we can inquire into what kinds of bodies they have, and how different genera of natural beings have different kinds of bodies. 

So Aquinas believes that what is better known to us is universal and common to all kinds of principles.\emph{We may ask how such general types are closer to perception. David Bronsten (unpublished) and Robert Bolton believe that Aristotle in in \emph{Phy}.B.19 gives a genetic account of the physical processes by which we grasp these universals, those that we must grasp before we can go inquire into principles. Other believe that Aristotle in this chapter gives an account of how we reach knowledge of the first principles. For more about learning the principles see Irwin and Owen.}

Charlton refers to 1.7 - natural course to say something general. By 'general': Aristotle will talk about the principles of natural things and artefacts without distinguishing between them. He thinks that we perceive individuals but our perception is of universals.

W.Wieland: Aristotle's Physics and the Problem of Inquiry into Principle. I think he characterizes as looking for the concepts presupposed by the deductive system.  He thinks that 'general' means something indeterminate - yet differentiated into its factors. So he thinks we do find the first principles in this book. Aristotle undertakes "an investigation into what presuppositions we have already made if we speak of natural things and events."p132

Bostock: "So Aristotle is not after all engaging in physical enquiry himself, as it had seemed from the beginning of the book that he was going to, but rather trying to lay down in advance the general from which any physical enquiry must have. Despite appearances he is not - or should not be - engaging in a dispute with the older physicists as to how many oppositions need to be taken as fundamental in physics, but is rather saying that however many principles the physicist needs to invoke some of them must be classificable as 'forms' and others as the corresponding 'privations' (and still others as 'underlying things')." p4. Bostock thinks that Aristotle is engaged in a meta-investigations of the general form which any account of change must take.

In the opening of the politics, Aristotle says that he will go from wholes to parts. 1252a20 also 1256a p 101 You can quote from HA1.6 491a10 and PAI.1 639b8 (and GC I.2 316a5) - describes the method of collecting the most general facts. (DE CAELO)







I think there is much agreement on this reading of the passage. But the question is how does Aristotle end. Do people think he arrives at the first principles by the end of the book, or does he only arrive at the universal at the end of the book?\emph{So Aquinas argues that we only grasp what is closest to us.} 

Interpreters agree that Aristotle believes we search for the first principles after first grasping some general schematic account. But how far does Aristotle get? 


Let us consider an example. Aristotle argues that matter is one of the principles of natural beings. But interpreters debate about what Aristotle believes matter to be.\footnote{Some examples: Philoponus (1888), Simplicius (1882), Zeller (1897),  Joachim (1922), Ross (1936), King (1956), Charlton (1970, 1972), J.L. Ackrill (1972/3), Jones (1974), Robinson (1974), Dancy (1978), Waterlow (1982), Williams (1982), Cohen (1984), Graham (1987), Furth (1988), Irwin (1988),  Gill (1989), Bostock (2001), Charles (2004), Ebrey (2007), Kelsey (2008), Lewis (2008).} Some believe that Aristotle in this book argues that the four elements fire, air, earth, and water, are the most ultimate matter. Others believe that Aristotle argues that something more fundamental than these elements are ultimate matter. This prime matter is something indeterminate in virtue of itself, and can change into anything. So interpreters agree that Aristotle argues that matter is principles but they disagree about what Aristotle argues matter to be in this work. 

But should we expect Aristotle to answer this question in this book? On the one hand, if Aristotle intends to fully grasp the principles, we expect Aristotle to identify what is matter, e.g. just as the meteorlogist tells us that thunder is a certain noise in the clouds, Aristotle tells us only that there is a certain sort of matter that is a principle, but he would not tell us which sort at all. 



\enddocument
 
 