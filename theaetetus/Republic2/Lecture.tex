\documentclass[oneside]{article}
 \headheight = 25pt
\footskip = 20pt
\usepackage{mdwlist}
\usepackage[T1]{fontenc}
\renewcommand{\rmdefault}{ppl}
\usepackage{fancyhdr}
 \pagestyle{fancy}
 \lhead{\textbf{\textsc{\small Scott O'Connor\\Ancient Philosophy}}}
 \chead{}
 \rhead{\large\textbf{\textsc{Republic 3}}}
 \lfoot{\footnotesize{\thepage}}
 \cfoot{}
 \rfoot{\footnotesize{\today}}
 \usepackage{longtable,booktabs}
\tolerance=700


\begin{document}
\thispagestyle{fancy}

\subsection*{Context of the tri-partition argument}

\begin{itemize}
\item The city has three ``kinds'' or ``classes'' of people in it: workers (money-lovers), guardians (honor-lovers), and rulers (wisdom-lovers). They have the distinct functions of: producing, guarding, and ruling respectively.
\item Some of the city's virtues are ``located'' in its parts: \emph{wisdom} in rulers, \emph{courage} in guardians. Some consist in a certain relation ``between'' its parts: \emph{temperance} is a certain concord between workers and rulers. \emph{Justice} consists in \emph{each part }performing its proper function.
\item If the city is a good model for the individual, both should have three ``parts'' with distinct functions.
\end{itemize}

\subsection*{The three kinds of desires or faculties in the \emph{Republic}}

\begin{description}
\item[Appetite:] Desires for food, drink, sex, etc. These are ``physiological'' desires (439)
\item[Spirit:] Emotions of anger, self-disgust, shame and desires for honor or respect (440-41)
\item[Reason:] Rational desires are for the overall good or good of the whole (441e)
\end{description}

\noindent Either we always ``go for'' things with part of the soul \emph{or} the whole soul---e.g. we learn with one part of it, get angry with another, and desire pleasures of food with a third (436a). Socrates proves to argue that it must be the former, i.e., that each of these faculties must reside in its own separate part of the sou. He now will argue that there are, in fact, three parts of the soul. 

\subsection*{Argument for one non-rational part (i.e. the appetitive part)}

\begin{enumerate}
\item [P1] \textbf{Principle of Opposites}: A thing cannot undergo opposites in the same part of itself, in relation to the same thing, at the same time (436b-37a)
\item[P2] Going (assent, wishing) for X and rejecting (dissent, not wishing for) X, are opposites
\item[P3] The desire for a drink is an unqualified desire $\neq$ the qualified desire for a good drink
\begin{enumerate}
\item[P3.i] Thirst \emph{as such} is a desire for drink \emph{as such} (437b-39a)
\item[P3.ii] Unqualified desires are for unqualified objects
\end{enumerate}
\item[P4] Sometimes we have a desire for drink, but choose not to drink
\item[P5] Having this desire to drink and not wanting to drink are opposites
\item[C1] So there are two ``things'' in the person involved in this event---one thing is the proper subject of the desire to drink, a \emph{distinct} thing is the proper subject of the rejection of drink. The former is a motivation generated by appetite, the latter by reason.
\item[C2] So there are at least two parts of the soul, the rational part and the appetitive part. 
\end{enumerate}

\subsection*{Arguments for a second non-rational part (i.e. the spirited part)}

\begin{enumerate}
\item[1] Spirit is distinct from appetite (439e-40)
\begin{itemize}
\item Evidence: Leontius---wants to look at corpses and does, but is angry with himself
\end{itemize}
\item[2] Spirit is distinct from reason (440e-41)
\begin{itemize}
\item Evidence 1: Children and animals do not have reason but do act contrary to their appetites
\item Evidence 2: Odysseus---wants to take vengeance on his unfaithful servants, but is restrained by reason
\end{itemize}
\item[3] Spirit is ``allied'' with reason (440)
\begin{itemize}
\item Evidence: The cases of a good person undergoing just and unjust punishment
\end{itemize}
\item[C1] Spirit is distinct from both appetite and reason
\item[C2] There are at least three parts of the soul
\end{enumerate}

\subsection*{\emph{Akrasia} (lack of self-control) on this theory}

S here seems to allow that lack of self-control is possible. Such a phenomenon occurs when  X does B, i) \emph{thinking} B is bad, ii) when able not to do B, and iii) overwhelmed by pleasure. S maintains that in such a case the judgment of the rational part of the soul (i.e. a rational desire) is overcome by the non-rational appetitive desire of the soul.

\noindent Question: On the \emph{Republic}'s theory, is \emph{akrasia} possible when X knows that B is bad, or only when X \emph{merely believes} that B is bad?

\subsection*{Justice}

Given the tri-partition of the soul, and the city-soul analogy, S maintains that justice is the harmonious functioning of the three parts of the soul, with each part fulfilling its proper ``function'' or ``work'': 

\begin{itemize}
\item{Reason rules, making judgments about the overall good for the person}
\item{Spirit and appetite ``obey'' reason in the sense that they only desire things that, in fact, accord with reason's determinations about what is good overall}
\begin{itemize}
\item{Spirit and appetite, however, \emph{do not} do this because they can judge what is best, or because they can judge that reason ``knows best,'' or because they can judge anything at all (they are \emph{non-rational}---they cannot grasp reasons)}
\item{They do it because they have been trained or conditioned only to generate such desires}
\end{itemize}\end{itemize}

\noindent So, only the just person's soul is genuinely unified, with no internal conflicts. This is why the participants agree that it is always better to be just than unjust (although S himself says that the argument is not done)

\section*{Problems}
\begin{enumerate}
\item [P3] makes a logical point about qualified and unqualified things: how could this show that we actually have non-rational desires?
\item It may be possible to re-describe [P4] as conflict between non-simultaneous rational beliefs
\item Does [P1] entail indefinite partition of the soul if two desires of a single soul part can conflict with each other?
\item What does justice, in the sense described above, have to do with justice as G \& A were discussing it at the beginning of Book II? Has S changed the subject?
\end{enumerate}

\end{document}
