\documentclass[oneside]{article}
\usepackage{graphicx}
\usepackage[final]{pdfpages}

 \headheight = 25pt
\footskip = 20pt
\usepackage{mdwlist}
\usepackage[T1]{fontenc}
\renewcommand{\rmdefault}{ppl}
\usepackage{fancyhdr}
 \pagestyle{fancy}
 \lhead{\textbf{\textsc{\small Scott O'Connor\\Ancient Philosophy}}}
 \chead{}
 \rhead{\large\textbf{\textsc{Overview}}}
 \lfoot{\footnotesize{\thepage}}
 \cfoot{}
 \rfoot{\footnotesize{\today}}
 \usepackage{longtable,booktabs}
\tolerance=700


\begin{document}






\section*{Chronology of Major Historical Events}
\begin{itemize}
\item{c. 12th Century BCE: Trojan War, North Turkey,  semi-mythical, the last year is recounted in Homer's \emph{Iliad}}
\item{8th Century BCE: Greek colonies populate southern Italy, Ionia (western Turkey), and elsewhere, Greek religion described by Hesiod holds sway}
\item 585 BCE: Thales' alleged prediction of an eclipse, the ``beginning'' of ancient philosophy
\item{6th--5th Century BCE: Presocratic Philosophy, centered around Asia Minor (Turkey), Greece, and Southern Italy}
\item{490 BCE: First Persian Invasion of Greece, under Darius 1}
\item{480 BCE: Second Persian Invasion of Greece, under Xerxes (famous Battle of the 300 at Thermopylae), Athens is becoming center of Greece}
\item{470/69 BCE: Socrates is born in Athens}
\item{431--404 BCE: Peloponnesian War between Athens and Sparta, ends with Athens' defeat}
\item{428/27 BCE: Plato is born, studies with Socrates}
\item{399 BCE: Socrates is executed for impiety and corrupting the youth of Athens}
\item{387 BCE: Plato establishes the Academy}

\item{384 BCE: Aristotle is born in Stagira, Macedonia}
\item{c. 363 BCE: Aristotle goes to study at Plato's Academy}
\item{356 BCE: Alexander the Great, son of King Phillip of Macedonia, is born}
\item{348 BCE: Plato dies. Aristotle leaves the Academy and Athens, moves to the island of Lesbos and establishes the studies of botany and zoology}
\item{343--340 BCE: Aristotle tutors Alexander in Macedonia}
\item{334/35 BCE: Aristotle establishes the Lyceum in Athens}
\item{336--323 BCE: Alexander the Great conquers ``all'', initiating the Hellenistic Age (\emph{H\^{e}llas} = Greece)}
\item{323 BCE: Aristotle flees Athens }
\item{322 BCE: Aristotle dies}

\item{196--86 BCE: Gradual takeover of Greece (and ``Asia'') by Rome}
\item{31 BCE: Battle of Actium ends Roman civil wars; Octavian becomes \emph{de facto} emperor (recognized more formally in 27 BCE); transition from Roman Republic to Empire}
\item{1st--5th Century CE: Spread of Christianity (towards the end of this period, it is  adopted by many Roman elite, replacing dominance of polytheistic, ``pagan'' religion)}
\item{476 CE: Rome finally conquered, Western Roman Empire falls (but lives on in the East, centered around Constantinople until 1453 CE (when Conquered by Ottoman Empire and renamed ``Istanbul''))}
\item{529 CE: Justinian, emperor of Eastern Roman Empire, declares all pagan schools closed, i.e., the Academy and the Lyceum, marking the end of ``ancient'' philosophy (but much of it lived on, for example in Islamic and Christian philosophy)}
\end{itemize}




\section*{Athenian Democracy}

In the fifth and fourth centuries, during the lifetimes of Socrates, Plato, and Aristotle, Athens was an independent state consisting of the urban area and the surrounding rural areas of Attica. The total population is difficult to estimate. It may have been as high as 300,000, but only 20,000 to 30,000 of these were citizens. Citizenship was granted only to men who were male, adult, native-born, and free. The rest of the population was comprised of foreign residents, women, children, and slaves. Other Greek cities were allies of or subordinate to Athens, which was the dominant naval power in the area. The Athenian alliance was opposed to a roughly similar alliance of other Greek cities led by Sparta.  

Athens, like some other Greek cities, had a democratic constitution. The Greek `demokratia'  is derived from `kratos' (power) and `demos' (used both for `the people’ as a whole, and in a more specific sense for the lower classes); hence it indicates that power belonged to the people as a whole, rather than to a particular class. 

Greek democracy was significantly different from the representative system that we are used to calling `democracy'. The Greeks would have regarded the present US constitution as a form of `oligarchy’---literally ‘rule by the few’.)  The Greeks only considered a state democratic  if its citizens are directly responsible for governing it, i.e, they advocated for direct, and not representative democracy. Power was distributed as follows:

\begin{enumerate}
\item The sovereign Assembly of all the citizens. 
\item The Council had 500 members, 50 chosen by lot from each of the 10 tribes into which the citizen body was divided. The members from each tribe were appointed 'presidents' (prutaneis) for one-tenth of the year; they were a standing committee, preparing business for the Assembly and Council. One member each day was chosen (by lot) as 'foreman', who presided over any meeting of the Council or Assembly on his day of office.  
\item A board of nine archons (archontes, 'rulers') was chosen by lot for one year of office, to which they were not re-eligible. 
\item A board of ten generals (stratêgoi;  civil as well as military and naval officials) was  chosen by election for one year of office; each general was re-eligible. Since generals were elected, not chosen by lot, and since they were re-eligible, they were powerful and influential. Pericles and other prominent Athenians (including Sophocles) exercised their influence by serving as generals. 
\item  The jury-courts (often large, perhaps as many as 1500 in some cases) were chosen  by lot from the citizens. 
\end{enumerate}
 



\section*{Greek Alphabet}

\includegraphics{apha.png}

\includepdf{text.pdf}


\section*{Major Philosophical Periods and Philosophers}
\begin{itemize}
\item{Presocratic: 6th--5th Century BCE (centered around Asia Minor (Turkey), Greece, and Southern Italy) (\textbf{NB}: Several of these thinkers were contemporary with or even younger than Socrates)}
\begin{itemize}\item{Thales' alleged prediction of an eclipse in 585 BCE traditionally marks the ``beginning'' of Ancient Philosophy} (so, Ancient Philosophy spans from 585 BCE--529 CE)\end{itemize}
\begin{itemize}
\item{Major philosophers: Thales, Anaximander, Anaximenes, Xenophon, Heraclitus, Parmenides, Zeno, Anaxagoras, Empedocles, Leucippus, Democritus}
\item{Sophists: Mix of rhetoricians, politicians, and itinerant teachers who taught ``success'' at political life for a fee (much maligned by Plato)--Gorgias, Protagoras, Melissus}
\end{itemize}
\item{Classical: 5th--4th Century BCE (centered around Athens)}
\begin{itemize}
\item{Socrates: 470/69--399 BCE (executed for impiety and corrupting the youth)}\item{Plato: 428/27--348 BCE (``student'' of Socrates; founded the Academy appx. 387 BCE)}\item{Aristotle: 384--322 BCE (went to study at Plato's Academy at age 17, left after Plato's death (aged 37); founded the Lyceum 335 BCE)}
\end{itemize}
\item{Hellenistic: 3rd Century BCE--2nd Century CE (centered around Athens, (later) Alexandria, and (even later) Rome)}
\begin{itemize}
\item{Major philosophers and ``schools'': Epicureanism (Epicurus, Lucretius), Stoicism (Zeno, Cleanthes, Chrysippus, Posidonius, Seneca, Marcus Aurelius), Skepticism (Plato's Academy beginning under Arcesilaus, Cicero, Sextus Empiricus)}
\begin{itemize}\item{Some people distinguish an ``Imperial'' period (corresponding to the time of the Roman Empire), containing Cicero, Seneca, Marcus Aurelius, and others. But, these thinkers mainly just advanced earlier Hellenistic philosophy.}\end{itemize}
\end{itemize}
\item{``Late Antiquity'': 2nd--6th Century CE}
\begin{itemize}
\item{Neoplatonism (Plotinus); Rise of Christianity and Christian Philosophy; Many commentaries on Aristotle (Alexander of Aphrodisias, Simplicius, Philoponous)}
\end{itemize}
\end{itemize}

\includepdf{map.pdf}

%\section*{The Central Task of the Historian of Philosophy}

%\begin{itemize}
%\item{Determine what view the historical figure is presenting}
%\item{Determine what reasons the historical figure offers to commend that view}
%\item{Determine why those reasons might have seemed to that figure to be good reasons to accept that view}
%\item{Determine whether and why you agree or disagree with the view}
%\end{itemize}

%\section*{Methodological Principles}

%\begin{itemize} 
%\item{Ancient $\neq$ Dumb}
%\item{Principle of Charity: If two interpretations are equally consistent with the text, attribute the more philosophically interesting view to the author.}
%\item{Principle of Humility: Be humble in your approach to these texts. If you have an interpretation that attributes an absurd view to a thinker, you should suspect that you haven't exactly figured out what the view is. These views (like all views) can be challenged, but must be challenged \emph{respectfully}.}
%\end{itemize}








\end{document}
