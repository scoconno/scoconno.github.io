\documentclass[11pt]{article}

\usepackage{fontspec}
\defaultfontfeatures{Mapping=tex-text}
\setmainfont[BoldFont={Minion Pro Bold}]{Minion Pro}

\usepackage[usenames,dvipsnames]{color}
\usepackage{graphicx}
\usepackage{fullpage}

\usepackage{hyperref}

\usepackage{lastpage, fancyhdr}
\pagestyle{fancy}
\lhead{}
\chead{Handout} 
\rhead{}
\lfoot{}
\cfoot{\thepage\space of \pageref{LastPage}} 
\rfoot{}
\footskip=30 pt
\headsep=20pt

\thispagestyle{empty}

\hypersetup{colorlinks=true, linkcolor=Sepia, urlcolor=Sepia, citecolor=BrickRed}

\usepackage{polyglossia}
\setdefaultlanguage{english}
\setotherlanguage{greek}
\newcommand{\gk}[1]{\textgreek{#1}}
\newcommand{\gloss}[1]{(\textgreek{#1})}

\begin{document}
\author{Phil 234}
\title{Overview}
\maketitle

\section*{Introductions}
\begin{itemize}\item{\emph{Little bit about me}}\item{\emph{ask for breakdown of students: year, College of Arts, Humanities and Social Sciences; College of Engineering and Information Technology; College of Natural and Mathematical Sciences}}\end{itemize}

\section*{Chronology of Major Historical Events}
\begin{itemize}
\item{Trojan War (circa 12th Century BCE), semi-mythical, the last year of which is recounted in Homer's \emph{Iliad}}
\item{By 8th Century BCE, Greek colonies populate southern Italy, Ionia (western Turkey), and elsewhere}
\item{490 BCE: First Persian Invasion of Greece, under Darius 1}
\item{480 BCE: Second Persian Invasion of Greece, under Xerxes (famous Battle of the 300 at Thermopylae)}
\item{431--404 BCE: Peloponnesian War between Athens and Sparta, ends with Athens' defeat}
\item{336--323 BCE: Alexander the Great conquers ``all'', initiating the Hellenistic Age (\emph{H\^{e}llas} = Greece)}
\item{196--86 BCE: Gradual takeover of Greece (and ``Asia'') by Rome}
\item{31 BCE: Battle of Actium ends Roman civil wars; Octavian becomes \emph{de facto} emperor (recognized more formally in 27 BCE); transition from Roman Republic to Empire}
\item{1st--5th Century CE: Spread of Christianity (towards the end of this period, it is adopted by many Roman elite, replacing dominance of polytheistic, ``pagan'' religion)}
\item{410 CE: Sack of Rome by Visigoths}
\item{476 CE: Rome finally conquered, Western Roman Empire falls (but lives on in the East, centered around Constantinople until 1453 CE (when Conquered by Ottoman Empire and renamed ``Istanbul''))}
\item{529 CE: Justinian, emperor of Eastern Roman Empire, declares all pagan schools closed, marking the end of ``ancient'' philosophy (but much of it lived on, for example in Islamic and Christian philosophy)}
\end{itemize}

\section*{Major Philosophical Periods and Philosophers}
\begin{itemize}
\item{Presocratic: 6th--5th Century BCE (centered around Asia Minor (Turkey), Greece, and Southern Italy) (\textbf{NB}: Several of these thinkers were contemporary with or even younger than Socrates)}
\begin{itemize}\item{Thales' alleged prediction of an eclipse in 585 BCE traditionally marks the ``beginning'' of Ancient Philosophy} (so, Ancient Philosophy spans from 585 BCE--529 CE)\end{itemize}
\begin{itemize}
\item{Major philosophers: Thales, Anaximander, Anaximenes, Xenophon, Heraclitus, Parmenides, Zeno, Anaxagoras, Empedocles, Leucippus, Democritus}
\item{Sophists: Mix of rhetoricians, politicians, and itinerant teachers who taught ``success'' at political life for a fee (much maligned by Plato)---e.g. Gorgias, Protagoras, Melissus}
\end{itemize}
\item{Classical: 5th--4th Century BCE (centered around Athens)}
\begin{itemize}
\item{Socrates: 470/69--399 BCE (executed for impiety and corrupting the youth)}\item{Plato: 428/27--348 BCE (``student'' of Socrates; founded the Academy appx. 387 BCE)}\item{Aristotle: 384--322 BCE (went to study at Plato's Academy at age 17, left after Plato's death (aged 37); founded the Lyceum 335 BCE)}
\end{itemize}
\item{Hellenistic: 3rd Century BCE--2nd Century CE (centered around Athens, (later) Alexandria, and (even later) Rome)}
\begin{itemize}
\item{Major philosophers and ``schools'': Epicureanism (Epicurus, Lucretius), Stoicism (Zeno, Cleanthes, Chrysippus, Posidonius, Seneca, Marcus Aurelius), Skepticism (Plato's Academy beginning under Arcesilaus, Cicero, Sextus Empiricus)}
\begin{itemize}\item{Some people distinguish an ``Imperial'' period (corresponding to the time of the Roman Empire), containing Cicero, Seneca, Marcus Aurelius, and others. But, these thinkers mainly just advanced earlier Hellenistic philosophy.}\end{itemize}
\end{itemize}
\item{``Late Antiquity'': 2nd--6th Century CE}
\begin{itemize}
\item{Neoplatonism (Plotinus); Rise of Christianity and Christian Philosophy; Many commentaries on Aristotle (Alexander of Aphrodisias, Simplicius, Philoponous)}
\end{itemize}
\end{itemize}

\section*{Syllabus} 
\begin{itemize}
\item{\emph{structure of class sessions: 5--10 minutes question: 50-60 minutes lecture; remainder questions/discussion}}
\item{\emph{Other course policies: ask questions whenever you have one; especially if there is a term being used that you don't understand; even if the term has been used several times---often a word comes up in philosophical discussion that you don't understand and you don't realize it will be used a lot in the ensuing discussion so you don't ask, but then it turns out that the whole discussion is about it; we are approaching difficult material and we shouldn't let our language make it even more difficult}}\end{itemize}
\section*{The Central Task of the Historian of Philosophy}

\begin{itemize}
\item{Determine what view the historical figure is presenting}\begin{itemize}\item{\emph{as best you can, reconstruct those views in the terms the historical figure used, using as plain english as possible}}\end{itemize}
\item{Determine what reasons the historical figure offers to commend that view}
\item{Determine why those reasons might have seemed to that figure to be good reasons to accept that view}\begin{itemize}\item{\emph{this will often require taking account of the quite different period the figure lived in; one question we must always ask is whether the person has certain biases (for example, is the fact that the person is male or wealthy, or so on) that makes a reason which is not, in fact a good reason to hold that view, \emph{seem} to that person to be a good reason; for example, when we consider Aristotle's account of virtue, we may wonder whether the fact that he is a man shapes the virtues he talks about and what he says about them; if we do identify such biases, we must then ask how to respond to them}}
\item{\emph{these first three tasks affect each other; if we can't find any reason to commend the view, we can either think (a) the figure had no reasons for the view or (b) we haven't yet figured out exactly what the view is; likewise, if we can't see why certain reasons would seem to someone like good reasons, we may suspect that we haven't figured out exactly what the reasons are that the figure had for holding the view or we haven't taken enough about that person into consideration;}}\item{\emph{analogy with trying to figure out why someone did something: you see your friend walking into the commons; you see your friend walking into The Commons, you ask someone, ``why is she going in there?'' that person responds, ``to escape invading aliens'' you know your friend pretty well and know that she does not believe in aliens. this should lead you to suspect that that is not, in fact, the reason why she is entering the commons. now imagine someone says, ``because there's a special on burritos at salsarita's''; you know your friend loves burritos and is economical; knowing that, you can see why the fact that there is a special at salsarita's would be a good reason for her to enter the commons; this gives you a reason for thinking you have identified the reason why your friend went into the commons (of course, it's incomplete; but it gives you a better reason for thinking you've identified the reason than the initial idea about aliens}}\end{itemize}
\item{Determine whether and why you agree or disagree with the view}\begin{itemize}\item{\emph{these views can be challenged; they are not sacrosanct; and we will challenge them in this course; many of the thinkers we will read, for one thing, disagreed with each other}}\end{itemize}
\end{itemize}

\section*{Methodological Principles}

\begin{itemize} 
\item{Ancient $\neq$ Dumb}
\item{Principle of Charity: If two interpretations are equally consistent with the text, attribute the more philosophically interesting view to the author}\begin{itemize}\item{\emph{For example, you ask your friend ``Where's the canoe?'' and she says ``at the bank'', in English there are two readings of this sentence ``at the (river) bank'' and ``at the (money) bank'' given that the first is a sensible answer to the question and the second is not, charity demands that you interpret the answer as meaning ``at the river bank''; this is a simple example, but it gets more demanding when we consider complex philosophical thoughts}}\end{itemize}
\item{Principle of Humility: Be humble in your approach to these texts. If you have an interpretation that attributes an absurd view to a thinker, you should suspect that you haven't exactly figured out what the view is. These views (like all views) can be challenged, but must be challenged \emph{respectfully}}
\end{itemize}


\end{document}
