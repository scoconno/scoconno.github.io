\documentclass[article,oneside]{memoir}
\usepackage{long table}
%%% custom style file with standard settings for xelatex and biblatex. Note that when [minion] is present, we assume you have minion pro installed for use with pdflatex.
%\usepackage[minion]{org-preamble-pdflatex} 

%%% alternatively, use xelatex instead
\usepackage{org-preamble-xelatex} 



\def\myauthor{Author}
\def\mytitle{Title}
\def\mycopyright{\myauthor}
\def\mykeywords{}
\def\mybibliostyle{plain}
\def\mybibliocommand{}
\def\mysubtitle{}
\def\myaffiliation{NJCU}
\def\myaddress{Phil 1}
\def\myemail{soconnor@njcu.edu}
\def\myweb{\href{http://scottoconnor.org/theaetetus}{http://scottoconnor.org/theaetetus}}
\def\myphone{}
\def\myversion{}
\def\myrevision{}
\def\myaffiliation{NJCU}
\def\myauthor{Dr. Scott O'Connor}
\def\mykeywords{}
\def\mysubtitle{Syllabus}
\def\mytitle{{\normalsize \myweb \newline} \HUGE Great Philosophers}


\begin{document}

%%% If using xelatex and not pdflatex
%%% xelatex font choices
\defaultfontfeatures{}
\defaultfontfeatures{Scale=MatchLowercase}    
% You will need to buy these fonts, change the names to fonts you own, or comment out if not using xelatex.      
\setromanfont[Mapping=tex-text]{Minion Pro} 
\setsansfont[Mapping=tex-text]{Myriad Pro} 
\setmonofont[Mapping=tex-text,Scale=0.8]{Georgia} 

%% blank label items; hanging bibs for text
%% Custom hanging indent for vita items
\def\ind{\hangindent=1 true cm\hangafter=1 \noindent}
\def\labelitemi{$\cdot$}
%\renewcommand{\labelitemii}{~}

%% RCS info string for version tracking
\chapterstyle{article-3}  % alternative styles are defined in latex-custom-kjh/needs-memoir/
%\pagestyle{kjh}

\title{\LARGE \mytitle}     
\author{\Large\myauthor \newline \footnotesize\texttt{\noindent Office hours: \href{http://scottoconnor.org/contact/office/}{http://scottoconnor.org/contact/office/}}}
\date{9/4/2018--12/18/2018}

\published{\textbf{PHIL 236-3 (2349), 3 credits, Fall 2018, M \& W 12:45pm--2:00pm, K429}}

\maketitle

%\thispagestyle{kjhgit}

% Copyright Page
%\textcopyright{} \mycopyright


%
% Main Content
%


\section{Disclaimer}
 This syllabus is subject to change at the discretion of the faculty. Students will be notified of such changes ahead of time via Blackboard. 

\section{Copyright}
The materials used in this class, including, but not limited to, lectures, exams, quizzes, and homework assignments are copyright protected works.  Any unauthorized copying of the class materials or recording of lectures is a violation of federal law and may result in disciplinary actions being taken against the student.  Additionally, the sharing of class materials without the specific, express approval of the instructor may be a violation of the University's Student Honor Code and an act of academic dishonesty, which could result in further disciplinary action.  This includes, among other things, uploading class materials to websites for the purpose of sharing those materials with other current or future students. 

\section{Catalog Description}
The class will teach you the reading skills necessary to understand and analyze philosophical texts, and teach you the skills required to adjudicate philosophical and interpretative problems about the central topics raised by those texts.
 

\section{Learning Objectives}

Upon completing this course students will be able to (i) read
philosophical texts, (ii) clearly and charitably explain viewpoints that
are not their own, (iii) think critically and philosophically, and (iv)
write well-structured prose in which they clearly state a thesis and
persuasively defend it.


\section{Readings}
\section{Required Textbooks}

Available in the campus book store and online retailers. You must bring a physical copy to class; we will read the text together. Failure to bring a text will result in loss of participation points. 

\begin{itemize}
\item \href{https://www.amazon.com/Theaetetus-Plato-Hackett-Classics/dp/0915144816/ref=sr_1_2?s=books&ie=UTF8&qid=1533665731&sr=1-2}{`The Theaetetus of Plato', tr. Myles Burnyeat and M.J. Levett, Hackett Classics, 1990}
\end{itemize}
%\begin{itemize}
%\item \href{https://www.amazon.com/Readings-Ancient-Greek-Philosophy-Aristotle/dp/1624665322/ref=dp_ob_title_bk}{`Readings in Ancient Greek Philosophy: From Thales to Aristotle', 5th edition, Cohen, Curd, Reeve (RAG)}
%\item \href{http://www.amazon.com/Hellenistic-Philosophy-Hackett-Classics-Inwood/dp/0872203786/ref=sr_1_1?ie=UTF8&qid=1452099186&sr=8-1&keywords=hellenistic+philosophy}{`Hellenistic Philosophy', Hackett Classics, 2nd Edition (HP)}


%\end{itemize}
\subsection{Optional}

\begin{itemize}
\item \href{https://www.amazon.com/Classical-Thought-History-Western-Philosophy/dp/0192891774/ref=sr_1_1?s=books&ie=UTF8&qid=1515009994&sr=1-1&keywords=classical+thought}{`Classical Thought', Terence Irwin}, a general introduction to ancient philosophy.

\item \href{http://www.amazon.com/Style-Lessons-Clarity-Grace-11th/dp/0321898680/ref=sr_1_1?ie=UTF8&qid=1452356026&sr=8-1&keywords=lessons+in+clarity+and+grace}{`Style: Lessons in Clarity and Grace', Joseph Williams and Joseph Bizup}, a good guide on how to improve your writing. 
\end{itemize}
\section{Course Website}
There is both a Blackboard site and website for this course (link on first page). Clicking the first link on the left panel within the Blackboard site will bring you to the course website. All assignments will be submitted through Blackboard. Readings, notes, etc. will be posted on the course website. Note that Blackboard difficulties are rare and automatically reported to instructors. Under no circumstance will a student's report of a Blackboard difficulty be reason for an extension. It is your responsibility to contact Blackboard support for help.


\section{Requirements}

\begin{itemize}
\item \textit{Workload:} Expect to spend an average of 6 hours per week completing the readings and assignments. NJCU abides by the Federal and State definitions of a credit hour and adopts a policy consistent with the Carnegie Unit. A three-credit class represents 112.5 hours total of work. See \href{http://scottoconnor.org/resources/Credit.pdf}{here} for more details.

\item \textit{Participation:} 0.5 point will be awarded per class up to a maximum of 10 points. Points will not be awarded during weeks 1 \& 2. Participation points will be awarded if you attend, bring your text, stay alert, stay for the duration of the class, leave your electronic devices turned off and out of sight, try to contribute to our discussions, and are cordial and non-disruptive. 


\item \textit{Weekly textual reflection (150--250 words)} submitted through Blackboard beginning week 3. Each reflection must articulate two different ways of reading some sentence, passage, or claim in the assigned portion of the text. It should outline brief support for both readings; reflections are not mere summaries of the text. Your scores on the highest ten submissions will be counted towards your final grades. 



\item \textit{Final paper  (1250--1750 words) } submitted through Blackboard. 

\item \textit{Course evaluations} completed online. 3 points extra credit for successful completion.

\item \textit{Grade Distribution:} Participation--0.5 point per class (10 total); Reflections--4 points each (40 total); Final paper--20 points. 

\item \textit{Grade Breakdown:}

 \begin{tabular}{ | l | l | p{2cm} | l | l | }
    \hline 
96--102 & A  & &  77--79 &  C+ \\  
90--95 & A- & &  73--76 & C \\
87-89 & B+ &  &  70--72 & C- \\ 
83--86 & B  & &  60--69 & D\\
80--82 & B - & & 0--59 & F\\ \hline
    \end{tabular}


\end{itemize}





\section{Policies}

\begin{itemize}

\item \textbf{Student Responsibility:} This syllabus outlines the required text, assignments, requirements, and policies for this course. By taking this course, you agree to read this syllabus and be bound by those requirements and policies. 

 \item \textit{Academic Integrity:} All the work you turn in (including papers, drafts, and discussion board posts) must be written by you specifically for this course. It must originate with you in form and content with all contributory sources fully and specifically acknowledged. Being a student at NJCU requires you to follow \href{http://scottoconnor.org/resources/Plagiarism.pdf}{NJCU's Academic Integrity Policy.} Penalties for violations are as follows: 1st infraction will result in a 0 for the assignment.  2nd infraction will result in a 0 for the entire course \& application for permanent record on student's transcript. (Repeated violations can lead to expulsion from NJCU). 

\item \textit{Attendance:} You are considered absent if you are (i) not present during roll call, (ii) leave early, (iii) leave without permission, or (iv) leave for an extended period of time. No excuses. No exceptions.



\item \textit{Communication:} To comply with Federal Privacy Laws (FERPA) and NJCU policies, all communication will be through Blackboard and/or official NJCU e-mail. Check Blackboard daily. For further information see \href{http://scottoconnor.org/contact/}{http://scottoconnor.org/contact/}.

\item \textit{Conduct:} Distracting and disrespectful behaviors, including but not limited to eating, leaving your seat, talking out of turn, and aggression are prohibited. Penalties include, but are not limited to, a loss of participation points for the day of violation. Repeat offenders will be reported to the Dean of Students. 

\item \textit{Electronic devices:} Use of electronic device, including, but not limited, to smartphones, dictaphones, tablets, and laptops, is prohibited. Recording a lecture is in violation of Copyright. Penalties include, but are not limited to, a loss of participation points for the day of violation. Repeat offenders will be reported to the Dean of Students.

\item \textit{Format for Written Work:} Submit work to Blackboard as either a pdf, rtf, or doc file. Blackboard will not allow any other format. All work must be typed and neatly presented. 


\item \textit{Grading:} Grades will be available within 1--2 weeks of an assignment being submitted. See: \href{http://scottoconnor.org/resources/grading}{http://scottoconnor.org/resources/grading} for further information.


\item \textit{Late work \& Make-up Policy:} See the assignment schedule below. The final paper will not be accepted late under any imaginable circumstances. The Gobbets will be accepted late with a penalty of 2 points per calendar day. If, for example, you submit on Tuesday a gobbet due the day before, the maximum grade you can receive for that assignment is 6 (instead of 8). If you submit on Thursday, the maximum grade you can receive is 2. No exceptions under any imaginable circumstances.

\item \textit{Statement for students with disabilities:} If you are a student
with a disability and wish to receive consideration for reasonable
accommodations, please register with the Office of Specialized Services
and Supplemental Instruction (OSS/SI). To begin this process, complete
the registration form available on the OSS/SI website at
\href{http://www.njcu.edu/oss}{http://www.njcu.edu/oss}
(listed under Student Resources-Forms). Contact OSS/SI at 201-200-2091
or visit the office in Karnoutsos Hall, Room 102 for additional
information.

\item \textit{Turnitin:} Students agree that by taking this course all assignments are subject to submission for textual similarity review to Turnitin.com. Assignments submitted to Turnitin.com will be included as source documents in Turnitin.com's restricted access database solely for the purpose of detecting plagiarism in such documents.  The terms that apply to the University’s use of the Turnitin.com service are described on the Turnitin.com web site.  For further information about Turnitin, please visit: http://www.turnitin.com 

\item \textit{SafeAssign:} Students agree that by taking this course all assignments are subject to submission for textual similarity review through Blackboard SafeAssign. Assignments submitted to SafeAssign will be included as source documents in SafeAssign's restricted access database solely for the purpose of detecting plagiarism in such documents.  


\end{itemize}



\section{Weekly Course Schedule}
All listed readings can be found in the required textbooks. See above for the abbreviations. Some references are given by their Stephanus numbers. For a good explanation, see \href{http://www.columbia.edu/itc/lithum/wong/stephanus.html}{http://www.columbia.edu/itc/lithum/wong/stephanus.html.
} 
 Changes to the syllabus will be announced through Blackboard and \emph{via} your NJCU email address.  All assignments must be submitted through Blackboard by Monday, 11:59pm. No late work accepted. No exceptions.   \newline
						


\newpage
\begin{center}
\begin{longtable}{p{6cm}p{3cm}p{5cm}} 
  \toprule
  \textbf{Week} &\textbf{Assignments} & \textbf{Reading} \\
  \midrule
\end{longtable}
\end{center} 
\vspace{-1.5cm}

\begin{center}
\begin{longtable}{p{6cm}p{3cm}p{5cm}}
 
\caption*{PART I: EXAMPLES OF KNOWLEDGE	} \\

[1.] Introduction	(9/4)	  			& 	 			&   \\


[2.] Examples (9/10)				& 	Reflection 1			& --142a-151d \\
% Day 1: definition by example. 
%Day 2: midwifery. Does Socrates know anything. 
\end{longtable}
\end{center} 
\vspace{-1.5cm}

								
\begin{center}
\begin{longtable}{p{6cm}p{3cm}p{5cm}}
  \caption*{PART 2: KNOWLEDGE IS PERCEPTION} \\

		
[3.] Metaphysical assumptions  (9/17)		& Reflection 2		&  --151d--155c \\ 

% The Definition of Knowledge as Perception: 151d-e
% Protagoras man is the measure of all things: 152b-c
% Introduction of Flux: 152c8–152e1
% Arguments for the flux theory: 152e1–153d
% Consequences of flux theory: 153d6-155c6

[4.] Flux theory of perception  (9/24)				& Reflection 3		& --155c--160d \\ 
% Flux theory of perception: 155c-157c
% Refinement after objection: 157c-160c
% Summary: 160b-d


[5.] Objections 1  (10/1)				& Reflection 4		& --160e--168c \\ 

% Day 1 Absurd implications of Protagoras' view (160e--163a) 
% Day 2 Counter-examples  to the alleged equivalence of knowledge and perception (163a-168c) 

[6.] Objections 2 (10/8)				& Reflection 5		 & --168c--179b \\

% Day1: Digression 172c1–177b7
% Day 2: last objection to Protagoras 177c6–179b5

[7.] Objections 3 (10/15)	   			& Reflection 6		 & --179c--186e    \\
% Day 1: last objection to Heraclitus 179c-182b 
% Day 2 final objection to everyone: 183c-187 

\end{longtable}
\end{center}
\vspace{-1.5cm}
\begin{center}
\begin{longtable}{p{6cm}p{3cm}p{5cm}}
 
  \caption*{PART 3: KNOWLEDGE IS TRUE BELIEF	} \\
  					
  
[8.] False belief (10/22)			 	& Reflection 7		& --187a--190e  \\

% The Puzzle of Misidentification: 187e5–188c8

[9.] The Wax Tablet	(10/29)			& Reflection 8		& --190e--196c\\
%Maybe do Hume this week
[10.] The Aviary and the Jury	(11/5)	& Reflection 9		& --196c--201c \\ 

\emph{No class on Wednesday (11/7)}		    					& 				& \\

[11.] Continued (11/12)				& Reflection 10		& Continued \\ 

\end{longtable}
\end{center}
\vspace{-1.5cm}
\begin{center}
\begin{longtable}{p{6cm}p{3cm}p{5cm}}
  \caption*{PART 4: KNOWLEDGE IS TRUE BELIEF AND AN ACCOUNT} \\
  			

[12.] Socrates' Dream (11/19)			& Reflection 11		& --201c-206b \\
						
[13.] What is an account? (11/26)	      	& Reflection 12		&  --206b-210d \\ 

[14.] TBD  (12/3)	    				& Reflection 13 	& -- TBD\\ 
			      				
[15.] TBD 	(12/10)			  		& Reflection 14		&-- TBD \\ 
\emph{No class on Wednesday (12/12)}   &	 			&--\\ 

[16.] No classes (12/17)		    		& Essay 3			& \\ 



\end{longtable}
\end{center}



%% Uncomment if you want a printed bibliography.
%\printbibliography 

\end{document}
