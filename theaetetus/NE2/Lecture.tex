\documentclass[oneside]{article}
 \headheight = 25pt
\footskip = 20pt
\usepackage{mdwlist}
\usepackage[T1]{fontenc}
\renewcommand{\rmdefault}{ppl}
\usepackage{fancyhdr}
 \pagestyle{fancy}
 \lhead{\textbf{\textsc{\small Scott O'Connor\\Ancient Philosophy}}}
 \chead{}
 \rhead{\large\textbf{\textsc{Nicomachean Ethics 2}}}
 \lfoot{\footnotesize{\thepage}}
 \cfoot{}
 \rfoot{\footnotesize{\today}}
 \usepackage{longtable,booktabs}
\tolerance=700


\begin{document}
\thispagestyle{fancy}

\section*{Aristotle's conception of the soul}

Aristotle's main work on the soul is \emph{De Anima}, which we will look at in detail later. Here let us observe the following: 

\begin{itemize}
\item A says the soul has both rational and non-rational aspects, and takes human motivation to come in three main kinds (reason, spirit, appetite), similarly to Plato in the \emph{Republic}  (Bk. II, Ch. 3)
%\item{However, A thinks it may be wrong to think of these as distinct \emph{parts} of the soul, as opposed to distinct aspects of one and the same thing (as the convex and the concave are two aspects of one and the same curved line)}

\item The non-rational aspect itself has two aspects: one that is wholly non-rational (the part responsible for maintenance of the human body), one that is non-rational but can ``obey'' and be ``trained'' by reason
\end{itemize}

\section*{Virtue (Bk. II, Chs. 1-4)}

Corresponding to the two aspects of the soul are two ``kinds'' of virtues: virtues of intellect (the aspect that has reason strictly speaking; discussed in Book 6) and virtues of character (the aspect that can obey reason; discussed in Books II-V). We will focus on virtue of character: 

\begin{itemize}
\item Virtue of character is acquired hrough habituation (by repeatedly performing just, temperate, etc. acts, one \emph{becomes} just, temperate, etc.); A compares this to the way in which someone acquires a craft (Ch. 1)
\item A notes that the states of character we acquire tend to be ``ruined'' by excess and deficiency (Ch. 2). For instance, if people stand firm against nothing, they  become cowardly, but if they fear nothing and rush into every confrontation, they become rash. If they give in to every pleasure, they become intemperate, but if they refrain from all they become ``a kind of insensible person''.

\item To evaluate the kind of character people have, A claims we can't just look at their actions---we also have to look at the pleasures and pains ``in consequence of their actions'': ``Virtues are concerned with actions \emph{and} feelings'' (1104b14) (Ch. 3)

\item The claim that we become virtuous by performing virtuous acts presents a puzzle: if we perform virtuous acts, aren't we \emph{already} virtuous? (Ch. 4)
\item Solution:  A distinguishes between [1] performing a virtuous act and [2] performing a virtuous act \emph{virtuously}---anyone can do the former (even vicious people)

\item To perform a virtuous act \emph{virtuously}, people must also:
\begin{itemize}\item{[A] Act knowingly
}\item{[B] Decide to perform the action and decide to perform it \emph{for its own sake}}\item{[C] Act from a firm and unchanging state}\end{itemize}
\end{itemize}
\end{document}
