\documentclass[oneside]{article}
 \headheight = 25pt
\footskip = 20pt
\usepackage{mdwlist}
\usepackage[T1]{fontenc}
\renewcommand{\rmdefault}{ppl}
\usepackage{fancyhdr}
 \pagestyle{fancy}
 \lhead{\textbf{\textsc{\small Scott O'Connor\\Ancient Philosophy}}}
 \chead{}
 \rhead{\large\textbf{\textsc{Republic 1}}}
 \lfoot{\footnotesize{\thepage}}
 \cfoot{}
 \rfoot{\footnotesize}
 \usepackage{longtable,booktabs}
\tolerance=700


\begin{document}
\thispagestyle{fancy}

\section*{Plato: Life and Works}

Plato was Born in 428 BCE to an aristocratic and politically powerful Athenian family. In his youth, he became a student of Socrates, who he uses as the main character in many of his dialogues. Plato left Athens upon the execution of Socrates in 399 BCE, but he returned and founded his Academy in around 387 BCE. At the academy he both wrote his own philosophical dialogues and oversaw various research programs in philosophy, mathematics, and natural science. He also seems to have taken trips to Sicily to try and bring his vision of the Philosopher-King into fruition, but that was ultimately a failure. He died in 348 or 347 BCE in Athens.

While he would have been expected to go into politics, in one letter he says he was tempted to partake in public life, but the treatment of Socrates led him to believe that you could not better Athenian society (or any actually existing society) ``from the inside''. So, he embarked on a philosophical life, with the hope of his philosophy making society better.

%\item{His work is conventionally divided into three or four ``groups'', often chronological; but we must note that it is difficult to be certain about the chronology:}\begin{itemize}\item{``Socratic'' (sometimes called ``early'')--Socrates is the main character; some moral concept is investigated (e.g. piety, courage, temperance, friendship; often ends in \emph{aporia} or failure to come to a decisive conclusion}\item{``Middle'': the \emph{Republic}--here we get more positive theses stated and argued for, loses some of the aporetic nature of the Socratic dialogues}\item{Some people identify a distinct group of ``transitional'' (e.g. the \emph{Meno}) which they don't want to classify as Socratic nor as Middle}\item{Late}\end{itemize}
%\item{\textbf{Stephanus}}


\section*{\emph{Republic}: Introduction}

While parts of the \emph{Republic} are studied in courses ranging from aesthetics to the philosophy of education to political science, the work is structured around answering one central question: is the just person happier than the unjust person? Or, to put it slightly differently, is the best life for a human to live a just one, or might an unjust life be a better life to live? In order to answer this question, Socrates says we must investigate what justice is. This is an upshot of the priority of the definition we saw in the \emph{Meno}; in order to know whether X is F, we must know the nature of X. So, if we are to know whether justice is beneficial, Socrates thinks we must know the nature of justice. It is helpful to emphasize these two guiding questions: 

\begin{description}
\item[Q1:] What is justice? 
\item[Q2:] Is it in my best interest to be just? 
\end{description}
The word `justice' translations a Greek word, dikaios. It can be used to describe people, actions, institutions, laws, and states. Participatory democracy might be a just system. Socrates might be a just person. Returning money you find might be a just action. The \emph{Republic} is primary focused on what it is for a human being to be just and whether being just is always in our best interest. It may seem obvious that it is not always in our best interest. If I could avoid paying taxes without being caught, I commit an unjust act while still benefiting myself. But in saying that this unjust act benefited me, I am obviously trading on some undefined notion of justice and benefit. Socrates, in one way or other, spends the whole dialog trying to find out what is truly beneficial to us and what is truly just. 

 We will see in the next note that Socrates investigates Q1 and Q2 by focusing on the just city, which is called Kallipolis. If we can identify the just city and decide whether a just city is always better than an unjust city, we can use that information in our investigation of justice in the soul. Socrates will identify justice in the city by imagining that we watch that city come into being, i.e., by discussing how we could create a just city. In this context, he discusses how citizens should be educated, resources should be allocated, art regulated, etc. But all of this is subservient to his main goal. He intends to figure out the nature of justice and ultimately prove that the happiest life is the just life. 

\section*{Book I}
The topic of the work is introduced in the first book (note there are 10 books in the \emph{Republic}), which treats various conventional views of justice:
\begin{enumerate}
\item Cephalus: Justice is telling the truth and returning what one owes (331)
\item Polemarchus: Justice is doing good to one's friends and harm to one's enemies (332)
\item Thrasymachus: Justice is the advantage of the stronger (338c)
\begin{itemize}
\item{[A] ``the stronger'' = the established power in a community, or ruler}
\item{[B] Being just = serving the good of another person or class, as defined in [A]}
\item{[C] Being unjust = serving one's self, \emph{either} as a citizen by not performing [B] actions, \emph{or} as a ruler or tyrant by setting up the standards of justice for others, but ignoring them oneself (343-44)}
\end{itemize}
\end{enumerate}
The first two suggestions are easily dismissed. Against the first, we shouldn't return a sword that we borrowed to a person who is drunk. Against the second, a just person wouldn't do unjustifiable harm to anyone (regardless of whether they are an enemy). The third suggestion is not as easily dismissed and will be picked up again in the second Book. According to this suggestion, it is unprofitable to the agent to be just. The only reason we are just is that we are forced to serve the interests of the ruler. On this view, injustice is, in fact, a virtue---i.e., it is profitable to the agent to be unjust (348). The perfectly unjust person would, if he/she were clever, courageous, etc. be happy. 

\begin{itemize}\item{Remember the background notion that happiness (\emph{eudaimonia}) is the goal in life and that virtue is supposed to contribute to/ensure the attainment of that goal}\end{itemize}

\section*{Book II: Division of goods (357)}
The reaction of S and the other interlocutors indicates that T's claims go against the generally received view. Most don't think we are always better being unjust if we can get away with it. But the interlocutors do see in T's claim a challenge that they develop and put to S. The rest of the work is dedicated to answering the challenge. Glaucon, one of the interlocutors, first tries to get clear about the type of benefits that justice and injustice might bring to us: 

\begin{enumerate}
\item[1] Good for own sake only, e.g., joy, harmless pleasures
\item[2] Good for own sake and consequences, e.g., knowing, seeing, being healthy.
\item[3] Good for consequences only, e.g., physical training, medical treatment, medicine, making money.
\end{enumerate}
Even if justice is a good, the question is what kind of good it is. Do we benefit by being just merely because of the consequences that being just brings, e.g., we are treated well by our peers, are able to access mortgages, etc? Or would we benefit by being just even if we received no good consequences? 

\begin{itemize}
\item{None of these senses = T's view in Book I. He thinks being just brings no benefit to the just person.}
\item{Sense [3] = As Glaucon argues in Book II (although note that G says he himself isn't persuaded by the argument but wants to hear S's response to it (358c6))}
\item{Sense [2] = Socrates (but note that in the ensuing discussion he focuses on arguing that justice is good for its own sake, irrespective of its consequences)}
\end{itemize}

\section*{G: justice is good only for its consequences}

Glaucon develops an account of the origin of justice: a contract was developed between individuals that encoded various laws regulating their behavior; following these laws is meant to be mutually beneficial (358e-59b)

\begin{itemize}
\item{Doing injustice is naturally good \& suffering injustice is naturally bad.}
\item Justification of contract: assume there a limited number of cookies. We can either co-operate and ensure we get an even share. Or we can try go alone and get as many for ourselves as we can. This includes stealing from each other, etc. \begin{center}
\begin{tabular}{ |c|c|c| } 
 \hline
 \emph{Cookie yield}	 & A co-operates   & A does not co-operate  \\ \hline 
 B co-operates & A:2, B:2 & A:4, B:0 \\ 
 B does not co-operate & A:0, B:4 & A:1, B:1 \\ 
 \hline
\end{tabular}
\end{center}

\begin{itemize}
\item If we co-operate and agree not to rob one another, we can ensure we each get 2 cookies. This is not ideal. If I could break the law while you respected it, I would get all the cookies. But it is better than the situation where we both rob one another, undermine one another, etc. In that case, we would minimize the total yield of cookies between us (as well as our individual share).
\item{The situation in which we both act unjustly is worse than when we both act justly because the cost of suffering injustice outweighs the benefits of acting unjustly oneself.}
\end{itemize}
\item Justice is practiced unwillingly---i.e., justice is only good in sense [3] (or, at least, people \emph{think} it's good in sense [3]).
\begin{itemize}\item{Proof: Gyges' ring---supposed to show that if the consequences of justice were attainable for an agent without actually being just, and if the outcome of this were better than the outcome when the agent is just, the agent would have better justification for acting unjustly than justly.}
\item {Justice without its consequences has no value---i.e., it doesn't contribute to happiness and so is not a good in sense [1] or [2].}
\end{itemize}
\item So, justice is a good only in sense [3]. 
\end{itemize}

%begin{itemize}\item{Two assumptions: i) the consequences of others' acting justly are beneficial\\\hspace*{28mm}ii) happiness is an independent criterion by which we can judge which outcome\\\hspace*{32mm}is better for the agent}





\section*{Socrates' Goals}

\begin{enumerate}
\item Prove that justice is good for its own sake and its consequences, i.e., it is a good in sense [2]. (But he thinks the real work is showing that it is good for its own sake.)
\item Show that justice is better than any other combination of goods without it, i.e., the very best thing for a human is to be just regardless of whether they have any other goods. He believes that a just person being tortured would still be living a better and happier life than the unjust person who is enjoying countless riches, pleasures, honors, etc.
\end{enumerate}

%\noindent \emph{Class discussion:} How might we try to prove 1 and 2? What must justice and happiness be like if being just is the happiest life a human can live? 

%\begin{itemize}\item{Are either of these necessary to support a reasonable theory of justice?}\end{itemize}

%\noindent S sets out to argue that:
%\begin{itemize}\item{[X] Inter-personal justice is strictly analogous to intra-personal justice. So, by examining the nature of political justice, S thinks he can show:}\item{[Y] Intra-personal justice is a state of character which is both intrinsically good and a necessary condition for any kind of ordered life, and hence for happiness, AND}\item{[Z] Intra-personal justice is a dominant constituent of happiness itself, and hence happiness cannot be specified or achieved independently of it, and justice is better for the agent than any combination of goods without it}\end{itemize}
%\noindent [X], [Y], and [Z] are problematic. Is it reasonable to \emph{define} happiness by reference to the virtues? What relation is there between justice in a city and an individual? Why think that ``psychic justice'' would produce recognizably just acts?
\end{document}
