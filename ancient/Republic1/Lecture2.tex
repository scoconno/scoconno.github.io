\documentclass[oneside]{article}
 \headheight = 25pt
\footskip = 20pt
\usepackage{mdwlist}
\usepackage[T1]{fontenc}
\renewcommand{\rmdefault}{ppl}
\usepackage{fancyhdr}
 \pagestyle{fancy}
 \lhead{\textbf{\textsc{\small Scott O'Connor\\Ancient Philosophy}}}
 \chead{}
 \rhead{\large\textbf{\textsc{Republic 2}}}
 \lfoot{\footnotesize{\thepage}}
 \cfoot{}
 \rfoot{}
 \usepackage{longtable,booktabs}
\tolerance=700


\begin{document}
\thispagestyle{fancy}

\section*{Socrates's Response}

Glaucon wants to know what justice and injustice are and what power each itself has when by itself in the soul. His challenge: (i) there are goods we welcome for their own sake and not because we desire what comes from then. There are also goods we desire for their own sake and also for what comes from them. Finally, there are goods we seek not for their own sake but for their consequences. (ii) Justice is only beneficial in this third way; it is not a good we welcome for its own sake. (iii)  Injustice would benefit us more than justice if we could get away with it. 

Socrates must prove that justice is one of the second goods. His approach: 

\begin{enumerate}
\item The investigation requires keen eyesight. If you were asked to read letters in the distance you would be relieved if you noticed the same letters existed elsewhere in a larger size and on a larger surface. Likewise, we need a larger model to investigate justice. 
\item There is the justice of a man and justice of a city, so we will investigate the city first and then find out what justice in the individual is. 
\item We will examine the just city by watching it come to be (in the mind's eay). He says, ``if we could watch a city coming to be in theory, wouldn't we also see its justice coming to be, and its injustice as well?''
\end{enumerate}

\section*{Constructing the Just City}

\begin{itemize}
\item Cities, any city, comes to be, because none of us is self-sufficient. We all need many things, and so we need one another to provide those things. 
\item It is better for people to stick to what they are good at for the benefit of everyone rather than to split their tasks.
\begin{enumerate}
\item  We are all born with different natures.
\item Different natures are suitable for different tasks. 
\item If we either (i) do whatever we are not suited for, or (b) attempt to do more than one job, then this will be less efficient and worse for the city overall...(remind you of any book?)
\end{enumerate}
\item If we take care of the necessities, our city will not be very luxurious. But we need to build a city in which our desires for fine things can be realized. Overstepping the limit of necessities will cause strife and war (we will need to take other lands and others will try take ours). The point here is that to surrender ``to the endless acquisition of money and to overstep the limit of necessities' seems to result in strife and conflict.'' 
\item So our city will need guardians/army. They protect us from internal and external threats. Given our previous requirement that each person dedicate themselves to the one task that they are natural suited to, we need to identify what nature is best suited to being a guardian. 
\begin{enumerate}
\item The guardian will have keen senses; if you are to have a natural aptitude you must have that which is the condition for that aptitude. 
\item The guardians must have certain physical characteristics
 like strength. 
\item The guardians will have a spirited/violent nature. Problem: if they have natures like that, they will be savage to each other and to the rest of the citizens. So they need a soul that is both gentle and spirited; this seems  a contradiction. 
\begin{enumerate}
\item Dogs act gently to those they know and the opposite towards those they don't know. 
\item A dog judges anyone it sees to be either a friend or an enemy on ``no other basis that it knows the one and doesn't know the other." It defines what is its own and what is alien to it in terms of knowledge and ignorance.
\item Similarly, the guardian must be kindly towards fellow citizens and fierce towards non-citizens. 
\end{enumerate}
\end{enumerate}
\item Since a guardian must have such a complex nature, how should they be educated? 
\begin{enumerate}
\item  Music and poetry, including stories, must be carefully censored to ensure children grow up with the right attitudes towards their city. 
\item We must do this before physical training.
\item	All the stories about Gods warring with each other and treating each other (and each others families) unfairly must be kept secret. For if they are told, the children will think it is ok to behave like the Gods behaved.
\item Stories in which citizens love and respect one another are to be told. Stories about the Gods causing goodness are to be told. 
\item In general, disordered and chaotic music and poetry cause chaotic and disordered public laws and contracts, and disordered and intemperate characters (ways of life) of those who hear it. Our guardians must never be exposed to such corrupting influences.
\end{enumerate}
\item Some of our guardians must know who are enemies and who are friends. They must know how to benefit the citizens. So, the very best of the guardians will become rulers. They will be aided by the remainder, who we call auxiliaries.
\end{itemize}



\section*{Is the just city a happy city? Book IV}

Adeimantus objects that Socrates has not  shown the guardians to be very happy. Socrates responds that he is making the whole city happy and not any group of individual happy. His argument for why this city is the best:

\begin{itemize}
\item  Everyone must follow the policy of sticking to their own work and being as good as possible at it. In this way, with the whole city developing and being governed well, we must leave it to nature to provide each group with its own share of happiness.''
\item The city has three ``kinds'' or ``classes'' of people in it: workers (money-lovers), auxiliaries (honor-lovers), and rulers (wisdom-lovers). They have the distinct functions of: producing, guarding, and ruling respectively.
\item The city is provisionally finished, and they now must ``see where the justice and the injustice might be in it, what the difference between them is, and which of the two the person who is to be happy should possess, whether its possession is unnoticed by all the gods and human beings or not.''
\item	The city, if correctly founded is completely good, hence it is wise, courageous, moderate and just. If he can find a few of these then ``what's left over will be the ones we haven't found.''
\begin{description}
\item[Wisdom:] The city is wise because it has good judgment. The rulers have knowledge not of ``any particular matter but about the city as a whole and the maintenance of good relations, both internally and with other cities.'' So we have found wisdom amongst some of the knowledge of the city' it is the knowledge of the guardians. 
\item[Courage:]  the part of the city fights and does battle on behalf of the city. ``The city is courageous, then, because of a part itself that has the power to preserve through everything its belief about what things are to be feared, namely, that they are the things and kinds of things that the lawgiver declared to be such in the course of educating it.'' So courage is a kind of preservation he says. 
\item[Moderation:] A kind of order and mastery of pleasures and desires in accordance with knowledge of what should be desired. So in the hypothetical city, the desires of the inferior are controlled by the desires of the  wise. Moderation resembles a kind of harmony because both the ruled and the ruler share the belief about who should rule; ``moderation spreads throughout the whole.''
\item[Justice:] ``is exactly what we said must be established throughout the city when we were founding it---either that or some form of it. We stated, and often repeated, if you remember, that everyone must practice one of the occupations in the city for which he is naturally best suited.''...``Justice is doing one's own work and not meddling with what isn't one's own.''
\begin{itemize}
\item Further evidence: ``I think that this is what was left over in the city when moderation courage, and wisdom have been found. It is the power that makes it possible for them to grow in the city and that preserves them when they've grown for as long as it remain there itself.'' If we don't stick to our jobs, then we will start, for instance, trying to regulate what is desired. But then we are taking the job of the ruler and we will ultimately cause our city difficulty.
\item This power is the most important aspect of a city's virtue because without it the city would not be moderate or courageous or wise. So this must be what is justice.
\end{itemize}
\end{description}
\end{itemize}

\subsection*{Summary}

 Some of the city's virtues are ``located'' in its parts: \emph{wisdom} in rulers, \emph{courage} in guardians. Some consist in a certain relation ``between'' its parts: \emph{moderation} is a certain concord between workers and rulers. \emph{Justice} consists in \emph{each part }performing its proper function.



\end{document}
