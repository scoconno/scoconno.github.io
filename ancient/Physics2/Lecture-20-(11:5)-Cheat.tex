% !TEX encoding = UTF-8 Unicode
% !TEX TS-program = xelatex

\documentclass[11pt]{article}
\usepackage{fontspec}
\defaultfontfeatures{Mapping=tex-text}
\usepackage{xunicode}
\usepackage{xltxtra}
\usepackage{verbatim}
\usepackage[margin= 1 in]{geometry} % see geometry.pdf on how to lay out the page. There's lots.
\geometry{letterpaper} % or letter or a5paper or ... etc
%\usepackage[parfill]{parskip}    % Activate to begin paragraphs with an empty line rather than an indent 
\usepackage{mathrsfs}
\usepackage{bbding}
\usepackage[usenames,dvipsnames]{color}
\usepackage{natbib}
\usepackage{stmaryrd}
%\usepackage{mathpartir}
\usepackage{txfonts}
\usepackage{graphicx}
\usepackage{fullpage}
\usepackage{hyperref}
\usepackage{amssymb}
\usepackage{epstopdf}
\usepackage{fontspec}
%\setmainfont{Hoefler Text}
\setmainfont[BoldFont={Minion Pro Bold}]{Minion Pro}
\usepackage{hyperref}
\usepackage{lastpage, fancyhdr}
%\usepackage{setspace}
\pagestyle{fancy}
\lhead{}
\chead{Lecture 20, Aristotle's \emph{Physics} Book 2\space---\space Handout} 
\rhead{}
\lfoot{}
\cfoot{\thepage\space of \pageref{LastPage}} 
\rfoot{}
\footskip=30 pt
\headsep=20pt
\thispagestyle{empty}
\hypersetup{colorlinks=true, linkcolor=Sepia, urlcolor=Sepia, citecolor=BrickRed}
\DeclareGraphicsRule{.tif}{png}{.png}{`convert #1 `dirname #1`/`basename #1 .tif`.png}
\usepackage{polyglossia}
\setdefaultlanguage{english}
\setotherlanguage{greek}
\newfontfamily\greekfont{Gentium Plus}
\newcommand{\gk}[1]{\textgreek{#1}}
\newcommand{\gloss}[1]{(\textgreek{#1})}

\usepackage{covington}
\usepackage{fixltx2e}
\usepackage{graphicx}
\begin{document}

%\maketitle
\thispagestyle{empty}
\begin{center} \LARGE{PHIL 321\\ Lecture 20: Aristotle's \emph{Physics}, 2.3-9}\\ \vspace*{2mm}
\large{11/5/2013}\end{center}
\thispagestyle{empty}\vspace*{3mm}
\vspace*{-8mm}

\section*{Aristotle's distinction between natural and artificial things (from 2.1)}

\noindent Natural things (e.g. animals, parts of animals, earth, air, fire, water): have within themselves a principle of motion and stability in place, in growth, in decay, or in alteration
\vspace*{2mm}

\noindent Artificial things (e.g. bed, cloak, artifacts in general) do not have internal principle of motion or stability. It's not insofar as something is a bed that it changes in a certain way. But, they do have such principles coincidentally (i.e. insofar as they are made of simple natural bodies)
\vspace*{2mm}

\noindent In general, A accepts a version of \emph{hylomporhpism}, a metaphysical view on which (almost) all existing things are composites of matter and form
\begin{itemize}\item{Matter: the underlying material (e.g. the bronze of a bronze statue, the flesh and bones of a human being)}\item{Form: the shape or, for more complex things, set of capacities that jointly constitute what it is to be something of a certain kind (e.g. the shape having which makes a piece of bronze a statue; the capacities and dispositions that jointly constitute being a human being)}\end{itemize}

\section*{The four \emph{aitiai} (``causes,'' ``explanations,'' ``becauses'')}

\noindent Since having \emph{epist\^{e}m\^{e}} requires grasping \emph{aitiai}, A sets out (in Ch. 3) to determine how many and what kinds of \emph{aitiai} there are
\vspace*{2mm}

\noindent [1] ``That from which, as present in it, a thing comes to be'' (often called the ``material cause'')
\vspace*{1mm}

E.g.: the bronze and silver, and their genera, are \emph{aitiai} of the statue and the bowl; letters of syllables; matter\\\hspace*{6mm}of artifacts; fire and such things of bodies; parts of wholes; premises of conclusion
\vspace*{2mm}

\noindent [2] ``The form (i.e. the pattern)... The form is the account (and the genera of the account) of the essence, and\\\hspace*{6mm}the parts in that account'' (often called the ``formal cause'')
\vspace*{1mm}

E.g.: the \emph{aitia} of an octave is the ratio two to one, and in general number; the whole, the composition
\vspace*{2mm}

\noindent [3] ``Source of the primary principle of change or stability'' (often called the ``efficient cause'')
\vspace*{1mm}

E.g.: adviser is an \emph{aitia}, father is an \emph{aitia} of his child; the producer is a \emph{aitia} of the product; the initiator\\\hspace*{6mm}of a change is an \emph{aitia} of what is changed; the seed; the doctor
\vspace*{2mm}

\noindent [4] ``Something's end, i.e. what it is for''
\vspace*{1mm}

E.g.: health of walking; health of purging, slimming, drugs, instruments; the good of the thing
\vspace*{2mm}

\noindent In Chs. 8 and 9 we learn that A thinks final \emph{aitiai} are relevant not only in the explanation of human behavior (e.g. Bob runs for the sake of health) but in nature as a whole
\newpage

\noindent The things that occur in nature always or for the most part, for A, happen for the sake of ends (e.g. organisms develop the parts they do in the way that they do for the good of the organism: in plants, leaves grow \emph{for the sake of} protecting the fruit; hearts contract for the sake of pumping blood (which, in turn, is done for the sake of maintaining the organism); rain falls in the patterns it does for the sake of nourishing the soil)
\vspace*{1mm}

\noindent This (highly controversial) view allows A to explain the development of organisms and why the parts of organisms are the way that they are by showing how it is good or best for the parts to be that way

\section*{Other features of \emph{aitiai}}
\vspace*{1mm}

--There can be multiple \emph{aitiai} of the same thing (e.g. the bronze and the sculpting are \emph{aitiai} of the statue)
\vspace*{1mm}

--Things can be \emph{aitiai} of each other (e.g. we work hard to be fit and we are fit so that we can work hard)
\vspace*{1mm}

--The same thing is the \emph{aitia} of contraries, i.e. if a thing's presence accounts for \emph{F}, by its\\\hspace*{6mm}absence it can account for the contrary of \emph{F} (e.g. if a pilot is the \emph{aitia} of safety of a ship, his absence can\\\hspace*{6mm}be the \emph{aitia} of a shipwreck)
\vspace*{1mm}

--There are proper \emph{aitiai} and coincidental \emph{aitiai}---the latter are \emph{aitiai} insofar as they happen to coincide\\\hspace*{6mm}with things that are proper \emph{aitiai} (e.g. the house-builder is a proper \emph{aitia} of the house, the doctor is a\\\hspace*{6mm}coincidental \emph{aitia}, insofar as the house-builder happens to also be a doctor)
\vspace*{1mm}

--There can be more or less precise specifications of an \emph{aitia} (even of the same kind) (e.g. a man is building\\\hspace*{6mm}because he is a builder; he is a builder insofar as he has the building craft; his building craft, then, is the\\\hspace*{6mm}prior \emph{aitia} [of the thing being built])
\vspace*{2mm}

\noindent\textbf{Similarities and Differences between aitia and cause}
\begin{itemize}\item{Differences}\begin{itemize}\item{Contemporaneous}\item{Things}\item{Mathematics}\end{itemize}\end{itemize}

\section*{\emph{Aitia} problem set}

\noindent [1] Why is the statue shiny? (what kind(s) of \emph{aitia(i)} would you cite to answer this question and what is a possible example(s))
\vspace*{1.5mm}

\noindent [2] Why is the bowl falling? (same as [1])
\vspace*{1.5mm}

\noindent [3] Sally's decision to go to the mall is the\hspace*{25mm} \emph{aitia} of Sally's being in her car driving to the mall
\vspace*{1.5mm}

\noindent [4] Why are the organism's front teeth becoming sharp and back teeth becoming flat? (same as [1] and [2]) 
\vspace*{1.5mm}

\noindent [5] What is the \emph{aitia} of the fact that a triangle's internal angles sum to $180\,^{\circ}$
\vspace*{1.5mm}

\noindent [6] What are the 4 \emph{aitiai} of Socrates' birth?
\vspace*{1.5mm}

\section*{Luck and Chance}

\noindent Luck and Chance seem to present a problem: we certainly say that things happen ``by chance'' or ``because of luck'' (etc.), but, for each thing that we describe this way, we can also specify some definite thing that led to the ``chance'' event happening.
\vspace*{2mm}

E.g.: If you go to the marketplace to buy some fruit, and run into someone you did not expect to meet\\\hspace*{6mm}that owes you money, we say it was by luck or by chance that you collected the money. But, we can also\\\hspace*{6mm}answer the question ``how did you collect the money'' by saying ``by going to the marketplace,'' and it\\\hspace*{6mm}wasn't by chance that you went to the marketplace.
\vspace{2mm}

\noindent Negatively: luck and chance are not \emph{aitiai} of things that happen always or usually in a certain way
\vspace*{2mm}

\noindent Positively: the kind of events that can happen because of luck and chance are those that \emph{can}  be result from something done for the sake of it but happen coincidentally
\vspace*{2mm}

So, go back to the marketplace case. You could have gone to the market in order to collect your debt. But,\\\hspace*{6mm}you went in order to buy fruit. Your collecting the debt was brought about coincidentally by your going\\\hspace*{6mm}to the marketplace (i.e. it just so happened that by doing that, you collected your debt).
\vspace*{2mm}

\noindent Chance is actually the wider notion: luck only concerns those things that could result from decision or thought (i.e. roughly, the affairs of adult humans); whereas chance includes those and things including inanimate and non-adult human animals
\end{document}
