\documentclass[oneside]{article}
 \headheight = 25pt
\footskip = 20pt
\usepackage{mdwlist}
\usepackage[T1]{fontenc}
\renewcommand{\rmdefault}{ppl}
\usepackage{fancyhdr}
 \pagestyle{fancy}
 \lhead{\textbf{\textsc{\small Scott O'Connor\\Ancient Philosophy}}}
 \chead{}
 \rhead{\large\textbf{\textsc{Posterior Analytics}}}
 \lfoot{\footnotesize{\thepage}}
 \cfoot{}
 \rfoot{\footnotesize{\today}}
 \usepackage{longtable,booktabs}
\tolerance=700


\begin{document}
\thispagestyle{fancy}



\section*{Aristotle and Meno's Paradox}

\noindent A begins the \emph{Posterior Analytics} by claiming that all teaching and learning result form previous knowledge (in the translation, it says ``cognition,'' but the word is ``\emph{gn\^{o}sis},'' which is cognate also with ``knowledge'')

\begin{itemize}\item{For deduction: to come to know the conclusion of a deductive argument, you must know the premises}\item{For induction: to come to know the universal claim, you must know the particulars}\end{itemize}

\noindent Recall Meno's paradox: 

\begin{itemize}
\item[P1.] Either you know X or you do not know X
\item[P2.] If you know X, you cannot inquire into X
\item[P3.] If you do not know X, you cannot inquire into X
\item[C.] You cannot inquire into X
\end{itemize}

\noindent To dissolve this paradox, A introduces a distinction between two ways of knowing a given proposition P that is about a particular (kind of) object

\begin{itemize}\item{Knowing P universally: i.e. knowing a more general proposition that P falls under}\item{Knowing P without qualification: knowing the particular proposition P itself}\end{itemize}
\vspace*{2mm}

\noindent So, A dissolves the paradox by denying [P2] and [P3]: you can inquire if you know in one way but don't know in another (question: how do we know the more general proposition?)

\section*{A's account of \emph{epist\^{e}m\^{e}}}

\noindent A states his definition of \emph{epist\^{e}m\^{e}} at the opening of Ch. 2
\vspace*{2mm}

\noindent In \emph{Prior Analytics}, A defines a \emph{syllogism} as: ``A discourse in which, certain things beings stated, something other than what is stated follows of necessity from their being so'' (24b19-20)
\vspace*{2mm}

\noindent In \emph{APo} 1.2, A posits six conditions that the premises of a \emph{sullogismos} must meet for it to count as a demonstration (\emph{apodeixis}):

\begin{itemize}\item{[1] true: can't know what's false}\item{[2] primary: if there were more fundamental facts than the premises, true \emph{epist\^{e}m\^{e}} would require going back to the most fundamental}\item{[3] immediate: this seems to amount to the same as [2]}\item{[4] better known than: you can't ``bootstrap'' your way up to a more secure grasp of the conclusion than the premises}\item{[5] prior to: can't be explanatory without being prior}\item{[6] explanatory of: follows from definition of \emph{epist\^{e}m\^{e}}}\end{itemize}

\noindent [1]--[3] are all properties of the premises themselves, [4]--[6] are relational properties that the premises must bear to the conclusion (i.e. what fills out the ``than,'' ``to,'' and ``of'' is ``the conclusion'')
\vspace*{2mm}

\section*{The seeds of skepticism (Ch. 3)}

\noindent A imagines two different groups, each of whom accept that knowledge requires demonstration, but differ in that one group says this makes knowledge impossible, the other possible

\begin{itemize}\item{Impossible: Either demonstrations must go on \emph{ad infinitum} or stop at some point. Either way, knowledge is impossible because:}\begin{itemize}\item{\emph{Ad infinitum}: Finite mind can't grasp an infinite chain}\item{Stop: There is no demonstration of the principles and, hence, no knowledge of them}\end{itemize}\item{Possible: Allow circular and reciprocal demonstration}\end{itemize}

\noindent A rejects this by saying that not all knowledge is through demonstration. Knowledge of the first principles is indemonstrable. Two possibilities for this:

\begin{itemize}\item{[A] There is a genuine kind of \emph{epist\^{e}m\^{e}} that does not require demonstration}\item{[B] \emph{Epist\^{e}m\^{e}} does require demonstration, and so knowledge of the first principles does not count as \emph{epist\^{e}m\^{e}}, strictly speaking}\end{itemize}

\noindent A opts for [B] and claims that we have \emph{nous} (``intuition,'' ``comprehension'') of first principles

\section*{Knowledge of first principles (2.19)}

\noindent The above discussion clearly shows that A needs some story about how we come to know first principles
\vspace*{2mm}

\noindent He rejects both the idea that knowledge of the principles is innate (sorry Plato, but how could we have knowledge that is ``more exact'' than the knowledge of demonstrable facts without being aware) and the idea that we can come to know them from a complete blank slate. So, he concludes we must have some potentiality or power to acquire them which is activated through the appropriate stimulus
\vspace*{2mm}

\noindent At bottom, A says it is our perceptual faculties (broadly construed) that give us the potentiality: perception $\rightarrow$ memory $\rightarrow$ experience $\rightarrow$ (knowledge of a) universal
\vspace*{2mm}

\noindent A calls this process ``induction''

\end{document}
