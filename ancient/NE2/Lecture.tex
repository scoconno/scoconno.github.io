\documentclass[oneside]{article}
 \headheight = 25pt
\footskip = 20pt
\usepackage{mdwlist}
\usepackage[T1]{fontenc}
\renewcommand{\rmdefault}{ppl}
\usepackage{fancyhdr}
 \pagestyle{fancy}
 \lhead{\textbf{\textsc{\small Scott O'Connor\\Ancient Philosophy}}}
 \chead{}
 \rhead{\large\textbf{\textsc{Nicomachean Ethics 2}}}
 \lfoot{\footnotesize{\thepage}}
 \cfoot{}
 \rfoot{\footnotesize}
 \usepackage{longtable,booktabs}
\tolerance=700


\begin{document}
\thispagestyle{fancy}

\section*{Aristotle's conception of the soul}
We saw that Aristotle defines happiness as rational activity of the soul in accordance with virtue. But what does that mean? Let us first recall Aristotle's view of the soul. In the \emph{De Anima}, we learned that Aristotle believes that the human soul has both rational and non-rational aspects or parts. Similar to Plato, he take motivation to come in three main kinds (reason, spirit, appetite).  (Bk. II, Ch. 3)
%\item{However, A thinks it may be wrong to think of these as distinct \emph{parts} of the soul, as opposed to distinct aspects of one and the same thing (as the convex and the concave are two aspects of one and the same curved line)}
The non-rational aspect itself has two aspects. One that is wholly non-rational. This is the part that is responsible for maintenance of the human body. The second is a non-rational part that can ``obey'' and be ``trained'' by reason. Aristotle's developed view of the soul is going to rely heavily on this idea of one part of the soul obeying another. 

\section*{Virtue (Bk. II, Chs. 1-4)}

Corresponding to the two aspects of the soul are two ``kinds'' of virtues. The first are virtues of intellect. These belong to the part of the soul that has reason strictly speaking, i.e., the parts that does the ordering. The second are virtues of character. These belong to the parts of the soul that can obey reason. We saw in the last note that a virtue is that, whatever it is, that allows a thing perform its function well. So the intellectual virtues will be those allow us reason and ultimately order well. The virtues of character are those that belong to the non-rational soul that allow it follow the dictates of reason well.  We will focus on virtues of character: 

\begin{itemize}
\item Virtues of character are acquired through habituation. By repeatedly performing just, temperate, etc. acts, one \emph{becomes} just, temperate, etc. Aristotle compares this to the way in which someone acquires a craft (Ch. 1)
\item States of character tend to be ``ruined'' by excess and deficiency (Ch. 2).
\begin{enumerate}
\item If people stand firm against nothing, they become cowardly. If they fear nothing and rush into every confrontation, they become rash. 
\item If they give in to every pleasure, they become intemperate. If they refrain from all pleasures, they become ``a kind of insensible person''.
\end{enumerate}
\item A virtue of character is a state of our non-rational soul that ensures we feel and respond rationally in neither of these extreme ways. For instance:  
\begin{enumerate}
\item A rash person feels too much confidence in danger; their rashness causes them to risk their lives. A coward feels too much fear in the face of danger; their cowardice causes them to run away. A courageous person feels the appropriate balance of fear and danger in the face of death; their courage causes them to risk their lives when appropriate.
\item A intemperate person feels too much lust for sex, an insensible feels too little, and a temperate feels an appropriate amount. Each trait causes the relevant emotional response and behavior. 
\end{enumerate}
\end{itemize}

It may seem that Aristotle thinks that a happy life consists only in performing certain kinds of virtuous actions. But Aristotle insists that we can't just look at a person's actions to evaluate the kind of character they have,---we also have to look at the pleasures and pains ``in consequence of their actions''. He claims that ``virtues are concerned with actions \emph{and} feelings'' (1104b14) (Ch. 3)

%\item The claim that we become virtuous by performing virtuous acts presents a puzzle: if we perform virtuous acts, aren't we \emph{already} virtuous? (Ch. 4)
To distinguish between virtuous action performed with and without the appropriate feeling, Aristotle distinguishes between [1] performing a virtuous act and [2] performing a virtuous act \emph{virtuously}---anyone can do the former (even vicious people). To perform a virtuous act \emph{virtuously}, people must also:
\begin{itemize}
\item{[A] Act knowingly
}\item{[B] Decide to perform the action and decide to perform it \emph{for its own sake}}\item{[C] Act from a firm and unchanging state}\end{itemize}
I will let you follow up on some of these remarks in the assigned readings. But this should hopefully be enough to introduce you to Aristotle's conception of the good life. It is one in which the non-rational part of the soul possesses the kind of virtues that ensures it acts accordance to the dictates of reason. As for what else it involves, please see the reading. 

\end{document}


