\documentclass[11pt]{tufte-handout}
%\usepackage[landscape, scale=0.7]{geometry}
\usepackage{graphicx}
\usepackage[T1]{fontenc}
\renewcommand{\rmdefault}{ppl}
\usepackage{fancyhdr}
 \pagestyle{fancy}
 \lhead{\textsc{Civil Disobedience in the \emph{Crito}}}
 \chead{3/5/14}
 \rhead{\textsc{Lesson Plan}}
 \lfoot{}
 \cfoot{}
 \rfoot{}
\usepackage{multicol}
\usepackage{mdwlist}
\title{Civil Disobedience in the `Crito'}
\author{Dr. Scott O'Connor}
\date{3/5/14}

\begin{document}

\maketitle
\begin{fullwidth}


\newthought{\LARGE{Socrates' argument against civil disobedience}}
\begin{multicols}{3}
\raggedcolumns
\subsection*{Instruction}
\begin{description}
\item[Overview:] Socrates argues that one ought never to disobey either a law or a lawful command of the state (50a--52b). His argument relies on the claim that a social contract exists between himself and the state. 
\item[Contract:] Discuss the main elements of a contract. Use a building contract to illustrate these elements.
\item[Apply:] Identify the elements of the social contract in \emph{Crito}. Point to the lines in the text (on a slide) that contain the relevant information. 
\end{description}
\columnbreak

\subsection*{Application}
\begin{itemize}
\item ``Break into groups of three. What would Socrates say about each of the four examples of civil disobedience discussed at the beginning of class? Would he think that they were justified, not justified?''
\item Select representatives from different groups to report their discussions. 
\end{itemize} 

\end{multicols}

\newthought{\LARGE{The \emph{Apology} vs. the \emph{Crito}}}

\begin{multicols}{2}
\raggedcolumns
\subsection*{Instruction}

\begin{description}
\item[Problem:] In \emph{Apology}, Socrates says that he would never stop practicing philosophy even if so ordered by the jury at his trial (29d). He also reports of his disobedience to the Thirty Commissioners when they ordered him to go to Salamis to get Leon (32c). 
\item[Failed solutions:] Explain why it is unlikely (i) that Plato is being inconsistent, or  (ii) that the historical Socrates is being inconsistent, or (iii) that Socrates changed his mind.  
\end{description}
\columnbreak


\subsection*{Application}
\begin{itemize}
\item ``Break into groups of three. Looking at the text of both dialogs, try to reconcile the apparent tension between Socrates' views in both.''
\item Ask representatives to report back to the class. Help students develop their solutions. 
\end{itemize}
\columnbreak
\end{multicols}


\end{document}





