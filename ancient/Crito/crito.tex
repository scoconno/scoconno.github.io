\documentclass[oneside]{article}
 \headheight = 25pt
\footskip = 20pt
\usepackage{mdwlist}
\usepackage[T1]{fontenc}
\renewcommand{\rmdefault}{ppl}
\usepackage{fancyhdr}
 \pagestyle{fancy}
 \lhead{\textbf{\textsc{\small Scott O'Connor\\Ancient Philosophy}}}
 \chead{}
 \rhead{\large\textbf{\textsc{Crito}}}
 \lfoot{\footnotesize{\thepage}}
 \cfoot{}
 \rfoot{\footnotesize{\today}}
 \usepackage{longtable,booktabs}
\tolerance=700


\begin{document}



\subsection*{1. Introduction}

An act of civil disobedience consists in the refusal to obey some laws of the state, legally binding orders of the courts, judiciary, or other official body that has the  legal authority to compel action. Socrates argues that it would be unjust for him to engage in civil disobedience. Since he believes that fleeing prison would violate the rulings of the court, he concludes he must stay and face his court mandated execution.

Acts of civil disobedience are common, some are famous, others infamous. Our goal is to identify why S thinks his acts of civil disobedience would be unjust and identify how we would judge some famous examples of civil disobedience. 

\begin{enumerate}
\item Susan B. Anthony illegally voted in US House of Representative elections in 1872. 
\item Rosa Parks refused an order by a bus driver to give up her seat to a white passenger in 1955.
%  \item Henry Thoreau refused to pay his federal taxes in protest of the Mexican-American War.
\item   In 2013, Edward Snowden, a former NSA contractor, leaked classified NSA documents to news outlets revealing unprecedented surveillance programs of US citizens, foreign nationals, and national governments.
%\item Kim Davis, a county clerk from Kentucky, defied a U.S. federal court order to issue marriage licenses to same sex couples in 2015. 
\item Cambridge pastor Henning Jaccobson refused to be vaccinated against smallpox in 1902 after the Board of Health in Cambridge, Massachusetts, pursuant to state law, required all inhabitants to be vaccinated. The penalty was a fine and eventual imprisonment. Jaccobson argued that the law violated his fourteenth amendment constitutional rights: 
\begin{description}
\item[Fourteenth Amendment]{Section 1. All persons born or naturalized in the United States, and subject to the jurisdiction thereof, are citizens of the United States and of the State wherein they reside. No State shall make or enforce any law which shall abridge the privileges or immunities of citizens of the United States; \textbf{nor shall any State deprive any person of life, liberty, or property, without due process of law; nor deny to any person within its jurisdiction the equal protection of the laws.}}
\end{description}
 SCOTUS ruled in 1905 the following: 
\begin{quote}
 "in every well ordered society charged with the duty of conserving the safety of its members the rights of the individual in respect of his liberty may at times, under the pressure of great dangers, be subjected to such restraint, to be enforced by reasonable regulations, as the safety of the general public may demand" and that "[r]eal liberty for all could not exist under the operation of a principle which recognizes the right of each individual person to use his own [liberty], whether in respect of his person or his property, regardless of the injury that may be done to others."
 \end{quote}
Further rulings established the precedent that vaccine mandates are one of those laws that it is for states themselves to enact: 
\begin{description}
\item[Tenth Amendment]{The powers not delegated to the United States by the Constitution, nor prohibited by it to the states, are reserved to the states respectively, or to the people.}
\end{description}
\end{enumerate}



\textbf{Discuss:} Were these illegal acts morally permissible? If not, why not? If some were, but others weren't, what's the salient difference?


\subsection*{2. Socrates' argument: a social contract}

Elements of a contract:

\begin{enumerate}
\item[ ] 1) Benefit, 2) obligation, 3) penalty, and 4) evidence of assent.
\item[ ]  Analogy with car loan:  1) Benefit = loan amount. 2) Obligation = repayment terms. 3) Penalty = repossession. 4) Evidence of assent = signatures.\end{enumerate}
\textbf{Reading Exercise:} S argues that there exists a (social) contract between himself and the city. Identify the four elements of this contract.


\subsection*{3. S's views about contemporary civil disobedience}
 What would Socrates say about the cases above? Would he claim that each illegal act was unjust? 



%\textbf{\emph{Exegetical Puzzle}}

%\begin{enumerate}
%\def\labelenumi{\arabic{enumi}.}
%\item
%  Socrates in Apology offers says that he did not obey the Thirty
 % Commanders. He will also never stop philosophizing. I want to make
  %this look really inconsistent. I won't examine alternative ways of
  %responding to it.
%\item
 % This is an apparent inconsistency. Ask students to look in text for
 % find ways of alleviating the puzzle.
%\item
 % If/when a student identifies a way of alleviating the inconsistency
  %that has been explored in the text, point it out and develop the view
 % further for her/him. Direct them to that literature.
%\end{enumerate}

\end{document}
