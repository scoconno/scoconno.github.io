\documentclass[]{article}

\usepackage{amssymb,amsmath}
\usepackage{ifxetex,ifluatex}
\usepackage{fixltx2e} % provides \textsubscript
\ifnum 0\ifxetex 1\fi\ifluatex 1\fi=0 % if pdftex
  \usepackage[T1]{fontenc}
  \usepackage[utf8]{inputenc}
\else % if luatex or xelatex
  \ifxetex
    \usepackage{mathspec}
    \usepackage{xltxtra,xunicode}
  \else
    \usepackage{fontspec}
  \fi
  \defaultfontfeatures{Mapping=tex-text,Scale=MatchLowercase}
  \newcommand{\euro}{€}
\fi
% use upquote if available, for straight quotes in verbatim environments
\IfFileExists{upquote.sty}{\usepackage{upquote}}{}
% use microtype if available
\IfFileExists{microtype.sty}{%
\usepackage{microtype}
\UseMicrotypeSet[protrusion]{basicmath} % disable protrusion for tt fonts
}{}
\ifxetex
  \usepackage[setpagesize=false, % page size defined by xetex
              unicode=false, % unicode breaks when used with xetex
              xetex]{hyperref}
\else
  \usepackage[unicode=true]{hyperref}
\fi
\hypersetup{breaklinks=true,
            bookmarks=true,
            pdfauthor={},
            pdftitle={},
            colorlinks=true,
            citecolor=blue,
            urlcolor=blue,
            linkcolor=magenta,
            pdfborder={0 0 0}}
\urlstyle{same}  % don't use monospace font for urls
\setlength{\parindent}{0pt}
\setlength{\parskip}{6pt plus 2pt minus 1pt}
\setlength{\emergencystretch}{3em}  % prevent overfull lines
\setcounter{secnumdepth}{0}



\title{\emph{The Crito}}
\date{}

\begin{document}
\maketitle

\section*{Review from Apology}

\begin{enumerate}
\def\labelenumi{\arabic{enumi}.}
\item
  The charge
\item
  The judgment
\end{enumerate}

\section*{Introduction}

\textbf{\emph{Civil Disobedience: Discuss}}

\begin{enumerate}
\def\labelenumi{\arabic{enumi}.}
\item
  Rosa Parks
\item
  Edward Snowden
\item
  Maybe tax avoidance
\item
  Were these illegal acts morally permissible? If not, why not? If some
  were, but others weren't, what's the salient difference?
\end{enumerate}

\textbf{\emph{Socrates' Argument: Introduced}}

\begin{enumerate}
\def\labelenumi{\arabic{enumi}.}
\item
  Social Contract.
\end{enumerate}

\textbf{\emph{Elements of a Contract}}

\begin{enumerate}
\def\labelenumi{\arabic{enumi}.}
\item
Benefit, obligation, penalty, and assent.
\item
  Analogy with course policies. 1) Set of benefits. 2) Set of
  obligations. 3) Penalty 3) Tacit assent ( they could have left)
\end{enumerate}

\textbf{\emph{Return to Socrates Argument to identify these three
distinct elements. }}

\begin{enumerate}
\def\labelenumi{\arabic{enumi}.}
\item
  Benefits:
\item
  Obligations: Obey or persuade them otherwise.
\item
  Penalty:
\item
  Assent: he could have gone elsewhere
\end{enumerate}

\textbf{\emph{Return to the four examples above. What would Socrates say
about these cases? Can they think of problematic cases? }}

\emph{Goal is to make sure they understand the view. Gives them a way of
thinking with and beyond the material. }

\textbf{\emph{Exegetical Puzzle}}

\begin{enumerate}
\def\labelenumi{\arabic{enumi}.}
\item
  Socrates in Apology offers says that he did not obey the Thirty
  Commanders. He will also never stop philosophizing. I want to make
  this look really inconsistent. I won't examine alternative ways of
  responding to it.
\item
  This is an apparent inconsistency. Ask students to look in text for
  find ways of alleviating the puzzle.
\item
  If/when a student identifies a way of alleviating the inconsistency
  that has been explored in the text, point it out and develop the view
  further for her/him. Direct them to that literature.
\end{enumerate}

\end{document}
