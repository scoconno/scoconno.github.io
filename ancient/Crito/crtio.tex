\documentclass[oneside]{article}
 \headheight = 25pt
\footskip = 20pt
\usepackage{mdwlist}
\usepackage[T1]{fontenc}
\renewcommand{\rmdefault}{ppl}
\usepackage{fancyhdr}
 \pagestyle{fancy}
 \lhead{\textbf{\textsc{\small Scott O'Connor\\Ancient Philosophy}}}
 \chead{}
 \rhead{\large\textbf{\textsc{Crito}}}
 \lfoot{\footnotesize{\thepage}}
 \cfoot{}
 \rfoot{\footnotesize{\today}}
 \usepackage{longtable,booktabs}
\tolerance=700


\begin{document}



\subsection*{Introduction}

An act of civil disobedience consists in the refusal to obey some laws of the state, legally binding orders of the courts, judiciary, or other official body that has the  legal authority to compel action. Socrates argues that it would be unjust for him to engage in civil disobedience. Since he believes that fleeing prison would violate the rulings of the court, he concludes he must stay and face his court mandated execution.

Acts of civil disobedience are common, some are famous, others infamous. Our goal is to identify why S thinks his acts of civil disobedience would be unjust and identify how we would judge some famous examples of civil disobedience. 

\begin{enumerate}
\item Susan B. Anthony illegally voted in US House of Representative elections in 1972. 
\item Rosa Parks refused an order by a bus driver to give up her seat to a white passenger in 1955.
  \item Henry Thoreau refused to pay his federal taxes in protest of the Mexican-American War.
\item   In 2013, Edward Snowden, a former NSA contractor, leaked classified NSA documents to news outlets revealing unprecedented surveillance programs of US citizens, foreign nationals, and national governments.
\item Kim Davis, a county clerk from Kentucky, defied a U.S. federal court order to issue marriage licenses to same sex couples in 2015. 

\end{enumerate}
Were these illegal acts morally permissible? If not, why not? If some
 were, but others weren't, what's the salient difference?


\subsection*{Socrates' argument: a social contract}

Elements of a contract:

\begin{enumerate}
\item[ ] 1) Benefit, 2) obligation, 3) penalty, and 4) evidence of assent.
\item[ ]  Analogy with car loan:  1) Benefit = loan amount. 2) Obligation = repayment terms. 3) Penalty = repossession. 4) Evidence of assent = signatures.\end{enumerate}
S argues that there exists a (social) contract between himself and the city. Identify the four elements of this contract.


\subsection*{Socrates views about contemporary civil disobedience}
 What would Socrates say about the cases above? Would he claim that each illegal act was unjust? 



%\textbf{\emph{Exegetical Puzzle}}

%\begin{enumerate}
%\def\labelenumi{\arabic{enumi}.}
%\item
%  Socrates in Apology offers says that he did not obey the Thirty
 % Commanders. He will also never stop philosophizing. I want to make
  %this look really inconsistent. I won't examine alternative ways of
  %responding to it.
%\item
 % This is an apparent inconsistency. Ask students to look in text for
 % find ways of alleviating the puzzle.
%\item
 % If/when a student identifies a way of alleviating the inconsistency
  %that has been explored in the text, point it out and develop the view
 % further for her/him. Direct them to that literature.
%\end{enumerate}

\end{document}
