\documentclass[oneside]{article}
\usepackage{graphicx}
 \headheight = 25pt
\footskip = 20pt
\usepackage{mdwlist}
\usepackage[T1]{fontenc}
\renewcommand{\rmdefault}{ppl}
\usepackage{fancyhdr}
 \pagestyle{fancy}
 \lhead{\textbf{\textsc{\small Scott O'Connor\\Ancient Philosophy}}}
 \chead{}
 \rhead{\large\textbf{\textsc{Republic 5}}}
 \lfoot{\footnotesize{\thepage}}
 \cfoot{}
 \rfoot{}
 \usepackage{longtable,booktabs}
\tolerance=700


\begin{document}
\subsection*{Introduction}
Glaucon wanted Socrates to show not only that justice was good in
itself but also that being just was always preferable to being unjust. We may agree with Socrates that a just soul consists in each part of the soul doing its own job, and we may agree with him that just acts promote that harmony while unjust acts disrupt it. But we might disagree with him that such a soul is better off than an unjust soul. A just person who was imprisoned would seem worse off than an unjust tyrant who enjoyed unlimited riches and pleasure. So,  Socrates needs to identify that power justice has on the soul itself which makes the just soul better of than the unjust soul in every circumstance. In other words, Socrates needs to show that the just soul is the happiest of all
possible souls. He does this in Books 8--9 by contrasting justice with
every form of injustice, both in the person and in the city. His goal is
to show that each type of injustice will generate less happiness than
justice. The four main species of injustice (see 445c) are timocracy, oligarchy,
democracy, tyranny. Corresponding to each type of these unjust cities are the
timocratic soul, oligarchic soul, democratic soul, and tyrannical soul
(544a, d--e).

\begin{description}
\item[The aristocratic soul:] ruled by its rational part, is concerned with
  what is best for each part of the soul in relation to the whole.
\item[The timocratic soul (538-9):] ruled by its spirited part, is concerned
  with honor and victory.
\item[The oligarchic soul (553ff):] a lover of money above everything else;
  it is concerned with being able to satisfy its necessary appetites.
\item[The democratic soul (559-61):] ruled by unnecessary appetites.
\item[The tyrannical soul (556-576):] enslaved to appetites. It constantly
  feels the need to satisfy this desire and never manages to fill it
  fully.
\end{description}

\subsection*{Argument that injustice is worse than justice}

Plato argues that any value other than justice will cause the soul and
city to degenerate into an even worse state. He assumes the
following:

\begin{itemize}
\item
  Most constitutions pursues a particular goal: to coordinate the city's
  actions and unify it around some one value. That value can be honor or
  money, pleasure in general or sexual pleasure in particular. The
  leaders agree that they need to unify their people around some goal,
  but they disagree about the goal.
\item
  Only the just city pursues an ideal that produces and enhances
  coordination. Every ideal other than justice engenders instability in
  the city that honors that ideal. This instability then resolves itself
  into a worse constitution.
\item
  If the ideal a constitution values causes the city to degrade further,
  there is something inherently wrong with that ideal.

\item
  The cause of trouble begins when the wrong children enter the ruling
  class (546b--547a). This causes a lack of unity in the rulers (545d).
\end{itemize}
Plato and the participants begin by discussing the regimes and individuals that deviate the least from the just city and individual. They then proceed to discuss the ones that deviate the most (545b-c). We will look at each deterioration after briefly reviewing the best city and soul, the aristocratic city and soul.

\subsection*{Aristocracy}

In Greek, \emph{aristocracy} means rule of the best. In Plato's ideal
city, the just city, the philosophers rule, guardians enforce their
orders, and the workers are content to be ruled. In the just soul, the
rational part rules with a concern for each part of their soul as part
of their overall good (586d-587a; cf.~442c). Plato illustrates this
claim by appealing to an image of the soul as consisting of a man, a
lion, and a many-headed beast (588c-590d). It is fitting for the man to
look after the beast, with the help of the lion; only in this way does
the beast get what's best for it (589ab). The point of the illustration
is to emphasize that the rational part alone has a comprehensive and
impartial concern for the whole soul, including each of its parts. The
rational part does not deny the appetitive and spirited parts but trains
them and satisfies them as is appropriate for them as parts of an
ordered whole (591c-592a). The overall goal is to ensure that they can
be satisfied while preserving psychic harmony. Reason inspects every
motivation, then chooses which ones to permit.

Consider the health analogy: in caring for my whole body, I am concerned
with the health and functioning of my bodily parts as parts of a larger,
functioning whole. I will not eat as much as I can but as much as
contributes to optimal functioning of someone with my frame and build.
Caring for my right arm will not consist in making it as strong as
possible but in strengthening it in proportion to the rest of my body.
Similarly, in pursuing the good of the whole soul, one will gratify
diverse appetites and passions insofar as their gratification
contributes to a well functioning tripartite soul in which reason rules.

\subsection*{From aristocracy to timocracy}
I will discuss the degeneration in cities and souls separately. 
\subsubsection*{Applied to cities}

\begin{itemize}
\item
  The just soul city will eventually change since everything which comes
  into being must decay (546a-b).
\item
  The good city will only exist given human interventions into the
  natural order to breed natures attuned to society's needs.
\item
  Those interventions ultimately fail.The rulers are bound to make
  mistakes in assigning people jobs suited to their natural capacities
  and each of the classes will begin to be mixed with people who are not
  naturally suited for the tasks relevant to each class (546e).
\item
  The next generation will yield a lesser crop of rulers (546a--547a).
\item
  This will lead to class conflicts (547a).
\item
  Timocracy arises when the rational part loses its power over the whole
  (547b, 550a--b). The productive class in the city insist on their
  claims to satisfaction. In a compromise between lowest and highest,
  the spirited part between them comes to rule.
\item
  Sparta is the best illustration of this second-best type of government
  (544c). Although this city enjoys considerable stability, the fact
  that the spirited part comes to power in the midst of conflict shows
  that the timocracy will possess less unity than we found in the best
  city.
\end{itemize}

\subsubsection*{Applied to souls}

\begin{itemize}
\item
  Reason miscalculates. It allows some appetites to be satisfied that
  don't promote harmony.
\item
  Certain desires dominate, e.g., a successful academic might start to
  like the money.
\item
  The desire for honor takes charge; out of all the
  bodily desires, it is the one that most resembles an organizational
  force. Unlike lust and hunger, greed knows the value of discipline
  (however anxious: 554d) and long-term planning (however ignobly aimed:
  554e--555a).
\item
  The rational part loses its power over the whole (547b, 550a--b). The
  appetites in the soul insist on their claims to satisfaction. In a
  compromise between lowest and highest, the spirited part between them
  comes to rule.
\end{itemize}

\subsection*{From timocracy to oligarchy}

\subsubsection*{Applied to cities} 

\begin{itemize}
\item Oligarchy arises out of timocracy when the timocratic city emphasizes
wealth rather than honor (550c-e). 
\item The competitive spirit of the timocracy's citizens prompts them to accumulate private wealth (550e)
and turns them into oligarchs (551a).
\item The productive class takes charge, and money becomes the dominant force in a society; thus it will be
not all members of the producing class who rule but only the richest (551b).
\item  People will pursue wealth. It will essentially be two cities,
a city of wealthy citizens and a city of poor people. The few wealthy
will fear the many poor. The poor people will do various jobs simultaneously. The
city will allow for poor people without means, but it will have a high crime rate. 
\end{itemize}

\subsubsection*{Applied to souls} 

\begin{itemize}
\item The oligarchic individual comes into being by seeing his father lose his
possessions and feeling insecure he begins to greedily pursue wealth
(553a-c). Thus he allows his appetitive part to become a more dominant
part of his soul (553c). The oligarchic individual's soul is at middle
point between the spirited and the appetitive part.
\item In their soul, the desire for money has taken charge; the desire for money is the one that most resembles an
organizational force. Unlike lust and hunger, greed knows the value of
discipline (however anxious: 554d) and long-term planning (however
ignobly aimed: 554e--555a). 
\item This single appetite dominates the oligarchic soul, but that appetite can't unify it. Unlike reason, which inspects
every motivation and chooses which ones to permit, avarice rules by
insisting on its own goals. Avarice knows no way of reining itself in. Not having been born to rule, it lacks the capacity for
self-examination. 
\item Plato would cite billionaires who crave money beyond
anything they could spend as proof of the unfitness of greed to rule
the soul.
\end{itemize}


\subsection*{From oligarchy to democracy}
\subsubsection*{Applied to cities} 

\begin{itemize}
\item Democracy comes about when the rich become too rich and
the poor too poor (555c-d); the oligarchy necessarily carries its greed
too far, and so necessarily impoverishes its solid citizens (555d--e). 
\item Too much luxury makes the oligarchs soft and the poor revolt against
them (556c-e).
\item In democracy most of the political offices are distributed by lot
(557a).
\item The primary goal of the democratic regime is freedom or license
(557b-c).
\item People will come to hold offices without having the necessary
knowledge (557e) and everyone is treated as an equal in ability (equals
and unequals alike, 558c). 
\item No value predominates in the democratic city apart from toleration (557b, 558a). A pact of mutual toleration is
like a code all citizens adhere to, and playing by the rules is the closest
thing to moral principle they know. But the very idea of harmony, or of
a ruler superior to the citizens, has become repugnant to.
\end{itemize}

\subsubsection*{Applied to souls} 

\begin{itemize}
\item The democratic individual comes to pursue all sorts of
bodily desires excessively (558d-559d) and allows his appetitive part to
rule his soul.
\item The democratic soul prefers not to choose among its desires, certainly
not to condemn any objects of desire (561b), but indulges each one as it
arises. Desires may be necessary or unnecessary (558d--559c).
\item  The democratic soul has no principle to guide its steps, not even the crass
and unlovely rule of greed. 
\item He comes about when his bad education
allows him to transition from desiring money to desiring bodily and
material goods (559d-e).
\item The democratic individual has no shame and no self-discipline (560d).
\end{itemize}

\subsection*{From democracy to tyranny}

\subsubsection*{Applied to cities} 

\begin{itemize}
\item Tyranny arises out of democracy when the desire for
freedom to do what one wants becomes extreme (562b-c).
\item The freedom or license aimed at in the democracy becomes so extreme
that any limitations on anyone's freedom seem unfair.

\begin{quote}
The father habitually tries to resemble the child and is afraid of his
sons, and the son likens himself to the father and feels no awe or fear
of his parents, so that he may be forsooth a free man. And the resident
alien feels himself equal to the citizen and the citizen to him, and the
foreigner likewise
\end{quote}

\begin{quote}
The teacher in such case fears and fawns upon the pupils, and the pupils
pay no heed to the teacher or to their overseers either. And in general
the young ape their elders and vie with them in speech and action, while
the old, accommodating themselves to the young, are full of pleasantry
and graciousness, imitating the young for fear they may be thought
disagreeable and authoritative.
\end{quote}

\item When freedom is taken to such an extreme it produces its opposite,
  slavery (563e-564a).
\item The tyrant comes about by presenting himself as a champion of the
  people against the class of the few people who are wealthy
  (565d-566a).
\item The tyrant is forced to commit a number of acts to gain and retain
  power: accuse people falsely, attack his kinsmen, bring people to
  trial under false pretenses, kill many people, exile many people, and
  purport to cancel the debts of the poor to gain their support
  (565e-566a). The tyrant eliminates the rich, brave, and wise people in
  the city since he perceives them as threats to his power (567c).
\item  The tyrant faces the dilemma to either live with worthless people or
  with good people who may eventually depose him and chooses to live
  with worthless people (567d). The tyrant ends up using mercenaries as
  his guards since he cannot trust any of the citizens (567d-e). The
  tyrant also needs a very large army and will spend the city's money
  (568d-e), and will not hesitate to kill members of his own family if
  they resist his ways (569b-c).
\end{itemize}

\subsubsection*{Applied to souls} 

\begin{itemize}
\item The greatest dictatorship in the city arises out of the
greatest anarchy (564a). In the soul, the democratic person's refusal to
judge among desires brings one of those desires, lust (erôs), to outgrow
all the rest (572e--573a). 
\item Plato separates unnecessary desires into
the law-abiding and the lawless (571b). The worst of the lawless desires
is lust, especially monstrous lust for forbidden persons, foods, and
deeds (574e--575a). 
\item Unlike the oligarch's greed, this transgressive
lewdness has nothing to do with self-control. It rules the soul wildly
-- indeed, it emerges as the dominant commitment of the tyrannical soul
not by virtue of any deliberation on the person's part but because it
has out-shouted every other desire. It comes to dominate by being the
most uncontrollable desire and not because it is suited to controlling;
thus its rule is of all states the least recognizable as rule.
\end{itemize}
\end{document}
