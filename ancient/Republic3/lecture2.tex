\documentclass[oneside]{article}
\usepackage{graphicx}
 \headheight = 25pt
\footskip = 20pt
\usepackage{mdwlist}
\usepackage[T1]{fontenc}
\renewcommand{\rmdefault}{ppl}
\usepackage{fancyhdr}
 \pagestyle{fancy}
 \lhead{\textbf{\textsc{\small Scott O'Connor\\Ancient Philosophy}}}
 \chead{}
 \rhead{\large\textbf{\textsc{Republic 6}}}
 \lfoot{\footnotesize{\thepage}}
 \cfoot{}
 \rfoot{}
 \usepackage{longtable,booktabs}
\tolerance=700


\begin{document}


\subsection*{Description of the Tyrant}\label{description-of-the-tyrant}

S has distinguished between necessary and unnecessary desires and
pleasures.

\begin{description}
\item[Necessary desires:]
desires that cannot train ourselves to overcome. These are desires for
objects that we need to survive, e.g., the desire for food.
\item[Unnecessary desires:]
desires which we can train ourselves to overcome, e.g., desire for
luxurious items and a decadent lifestyle.
\end{description}
S begins his description of the tyrant by saying that some of the
unnecessary desires are particularly abhorrent. These are what he calls
the lawless desires:

\begin{itemize}
\item
  Desires awakened in sleep when the rest of the soul slumbers. During sleep, it is free  from the control of shame and reason.
\item
  S lists 1) sexual desires for anyone and anything, e.g., family
  members, animals, etc. 2) Desires for violence, e.g., desires to
  commit murders. 3) Desires for all and every food. 4) Generally, these are the  desires that we would normally be ashamed to admit having while awake.
\item
  They are present in most of us, but they are held in check by
  the better desires in alliance with reason. In a few people, they have
  been eliminated entirely or only a few weak ones remain, while in
  others they are stronger and more numerous.
\end{itemize}

\subsection*{How the tyrant comes to be}

Recall that the oligarch produces democratic children. These parents are driven by thrift and greed, and they hate those unnecessary desires that cause them to spend more than needed. Their children are rich, but they hate their parents for their thriftiness. They associate with people who desire the fine and unnecessary things. Pulled between two types of desires, greed and opulence, they indulge the latter more than their parents. But they do not become lawless since they don't completely abandon their parents' values. \\

\noindent  The child of the democrat does not have the values of an oligarchic parent that can check their lawless desires. They satisfy them more and more, and so they develop an erotic love of these lawless desires. The tyrant arises when the lawless desires take hold.


\begin{quote}
And when the other desires-filled with incense, myrrh, wreaths, wine,
and the other pleasures found in their company-buzz around the drone,
nurturing it and making it grow as large as possible, they plant the
sting of longing in it. Then this leader of the soul adopts madness as
its bodyguard and becomes frenzied. If it finds any beliefs or desires
in the man that are thought to be good or that still have some shame, it
destroys them and throws them out, until it has purged him of moderation
and filled him with imported madness.
\end{quote}

\subsection*{How the tyrant lives}

General point: the tyrant's waking life is like the nightmares in which lawless desires roam free.

\begin{itemize}
\item
  Since erotic love rules, the tyrant directs all their energies to feasts, revelries, luxuries, sex, etc. Unleashed, many desires demand satisfaction and the tyrant is  singularly driven to acquire whatever is needed to satisfy them. They spend all their income, borrow wildly, sell all their goods.
\item But desires need satisfaction, so when all this has gone, the tyrant
  will acquire whatever desire demands by any source. They take and
  spends their parent's wealth without shame. They rob, bribe, loot
  temples, etc. They do this without shame.  
  \item Tyrant associates with flatterers, makes merely instrumental friends,
  etc. They are driven to power to ensure that they have the means to
  satisfy their lawless desires without rebuke.
  \item Beliefs and opinions about   what is fine and noble have been replaced with various beliefs about  what they deserve, and any other belief that lets these desires to be  pursued without guilt and shame.
 \end{itemize}

\begin{quote}
Now, however, under the tyranny of erotic love, he has permanently
become while awake what he used to become occasionally while asleep, and he won't hold back from any terrible murder or from any kind of food or
act. But, rather, erotic love lives like a tyrant within him, in
complete anarchy and lawlessness as his sole ruler, and drives him, as
if he were a city, to dare anything that will provide sustenance for
itself.
\end{quote}


%\subsection{Three Arguments that the Just person is the
%happiest}\label{three-arguments-that-the-just-person-is-the-happiest}

%\subsubsection{First argument 576b-578b}\label{first-argument-576b-578b}

%Just as a dictatorship is the least happy of cities, so the dictatorial person is the least happiest individual.

%--city: best citizens are slaves and can least do what they want.

%--person: best parts of the soul are enslaved and ruled by a small part which is the worst.

%You're right. But what if some god were to lift one of these men, his fifty e or more slaves, and his wife and children out of the city and deposit him with his slaves and other property in a deserted place, where no free person could come to his assistance? How frightened would he be that he himself and his wife and children would be killed by the slaves?

\subsection*{An argument that the philosophical life is preferable to the tyrant's (580c-583b)}\label{second-argument-580c-583b}

\begin{enumerate}
\item
  Each part of the soul has a distinct kind of pleasure.
  \item
  Each part of the soul is motivated by a desire for the associated pleasure.

  \begin{enumerate}
  \item
    Reason desires knowledge and truth.
  \item
    Spirit desires honor, power, and fame.
  \item
    Appetite desires (i) food, drink, sex, rest, and (ii) wealth and
    money as a means to these.
  \end{enumerate}
\item
  Different type of people are dominated by different parts.

  \begin{enumerate}
  \item
    The philosopher has (i) fully experienced the pleasure of learning
    and understands the genuine value of wisdom, (ii) has fully
    experienced the pleasures of honor and rightly judges its limited
    value, (ii) has fully experienced the pleasures of wealth and
    rightly judges its very limited value.
  \item
    The honor lover (i) does not really understand the value of learning
     and says it is valueless since it does not bring distinction,
    (ii) has experienced pleasure from honor, victory, strife, but
    mistakenly thinks honor is the highest value, (iii) thinks that
    pleasure of money is simply vulgar and grasps that wealth is of
    limited value.
  \item
    The wealth lover (i) lacks experience with value of learning or
    wisdom and says it is valueless since it does not bring wealth, (ii)
    lacks experience with value of honor or position and says it is
    valueless since it does not bring wealth, (iii) has experienced
    pleasure from wealth and gain, and mistakenly thinks wealth is
    highest value
  \end{enumerate}
\item
  Each type of person believes their life is most pleasant.
\item
  Only the philosopher, the one ruled by reason, has experience of all
  three types of pleasure. 
  \item[C1.] Therefore, the philosopher is the most
  reliable judge of which life is most pleasant. 
  \item[C2.] Therefore, the
  philosopher's life is the most pleasant.
\end{enumerate}

\end{document}

\subsubsection{Third argument 583b-585a}\label{third-argument-583b-585a}

\begin{enumerate}
\def\labelenumi{\arabic{enumi}.}
\item
  There are 3 sentient states: pleasure, pain, intermediate
  {[}neutrality?{]}
\item
  Confusion can arise when people think that pleasure = absence of pain.
  Most bodily pleasures and anticipations of pleasure are of this kind.
  People are generally not aware of genuine pleasure. The greater
  pleasures come through the exercise of reason.
\item
  Just as hunger is an emptiness of the body, so ignorance is an
  emptiness of the soul. But the things that fill the latter are more
  real than the things that fill the former.
\item
  In proportion as sustenance and the thing sustained by it are more
  real, the resulting satisfaction is more real.
\item
  The body and the objects required to sustain are less real than the
  soul and the objects required to sustain it.
\item
  If the satisfaction is more real, then it produces more pleasure.
\item
  The just and philosophical life better satisfies the desires of the
  soul than does the unjust life.
\item
  So, the just and philosophical life is more pleasant than the unjust
  life.
\end{enumerate}

\subsubsection{588b-590c: The three part
Creature:}\label{b-590c-the-three-part-creature}

A. Many-headed beast. B. Lion C. Man {[}sic: human-being{]} --To praise
injustice is to favor the enslavement of the man by the many headed
beast. --To praise justice is to favor the man taming the beast, making
the lion his ally, and caring for all in common. 590c-591a: Those with
insufficient reason in their souls must be ruled by reason from without,
thus producing a harmonious situation in which all citizens are ruled by
the same thing. 591a-592b: The just individual is happiest and has the
harmonious soul, whereas the unjust person is least happy even his/her
injustices are not discovered.


