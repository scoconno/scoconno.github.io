\documentclass[oneside]{article}
 \headheight = 25pt
\footskip = 20pt
\usepackage{mdwlist}
\usepackage[T1]{fontenc}
\renewcommand{\rmdefault}{ppl}
\usepackage{fancyhdr}
 \pagestyle{fancy}
 \lhead{\textbf{\textsc{\small Scott O'Connor\\Ancient Philosophy}}}
 \chead{}
 \rhead{\large\textbf{\textsc{Euthyphro}}}
 \lfoot{\footnotesize{\thepage}}
 \cfoot{}
 \rfoot{}
 \usepackage{longtable,booktabs}
\tolerance=700


\begin{document}





\section*{Overview}

Murder was a religious offense because it entailed `pollution', which, if not ritually purified, was displeasing to the gods. But a son prosecuting his father would also be seen by many Greeks as `impious' (or `unholy', `ungodly'.) However, Euthyphro claims that he must prosecute his father for murder; he claims that the pious thing is, in fact, to prosecute his parent in this case. Socrates is unsure. He thinks that knowledge of whether such a prosecution is pious or not would require us to know the nature of piety. Similarly, in order to know what actions are impious, we must know the nature of impiety. Euthyphro agrees and claims that he does, in fact, know the nature of piety. Socrates tests this claim. Euthyphro  is found wanting and so his claim to know that prosecuting his father is not impious is undermined.\\ 


%The basic structure of the dialog is as follows: 
%\begin{itemize}\item{2a--4a: Socrates (S) encounters Euthyphro (E), who is about to prosecute his (E's) father for murder.}\item{4a--5d: S claims that only someone with expertise about piety, crucially including knowledge of its nature, should be confident that he were not acting impious in prosecuting a relative.}\item{5d--15d: E offers various candidates for the definition of piety, all of which S rejects.}\item{15d--end: S requests to start from the beginning, E refuses, leaves.}\end{itemize}

\noindent Our learning goal for this dialog:  Identify and understand Socrates' views about an adequate answer to a question of the form, `what is X', i.e., what it is you have to know in order to properly know what X is. 







%\section*{Ethical action without knowledge}
%\begin{itemize}
%\item{[1] E's prosecution of his father is based on his belief that he has expertise concerning piety}

%\item{[2] S seems to show that E does \emph{not} have that expertise}\begin{itemize}\item{What conception of expertise is at work here? Why does such expertise require knowledge of the nature of piety? Is it reasonable to demand such expertise? (for all actions? for some?) How does S test whether E has such expertise?}\end{itemize}

%\item{[3] If E becomes aware of [2], how would it be rational for him to act?}

%\begin{itemize}\item{By the end, S seems to think that E shouldn't prosecute his father. Why is this?}\end{itemize}
%\end{itemize}
%

\section*{The ``What is \emph{X}?'' Question}

Socrates asks a simple question, ``what is X?', about moral and aesthetic qualities. In the \emph{Euthyphro}, he asks `what is piety?', but other dialogs find him asking `what is justice?', `what is courage?', `what is wisdom?', `what is love?', `what is beauty?', etc.\\

\noindent A \textbf{Socratic definition} is an answer to a ``what is X?'' question. These definitions are not of words, but of things. S wants to know the nature of piety itself and not just what the word `piety' means. In a similar way, physicists investigating the nature of matter are not interested in what the word `matter' means. If they were, they could just consult a dictionary. They are interested in the nature of that stuff in the world, the nature of matter itself.  A Socratic definition, then, is a true description of the nature of the thing to be defined. For instance, `H2O' is true description of the nature of water and not just our thoughts about water. \\

%Students often resist Socrates at this point. Many will respond that piety, or beauty, or courage, etc., exist and are what they are because of how we think and speak about them. Call such a view \emph{conventionalism}.  If conventionalism is true about, say, beauty, we shouldn't be surprised that different cultures treat different things and people as beautiful. Conventionalism about beauty denies that beauty exists and is what it is independently of how we think and speak about it. 

%Is conventionalism true? It's a substantive position that requires a substantive argument. For our purposes, we can point out that Socrates does not yet hold any substantive position. He searches for the Socratic definition of piety precisely because he does not know what it is. If, after searching with him, we discover that there is no answer, then we might conclude that conventionalism is true. But conventionalism cannot be the starting point. Our starting point, reasonably, is simply the question, `what is piety?'

\noindent What would S consider a satisfactory answer to the question `what is piety'? More generally, what are the criteria that must be satisfied for an answer to a `what is X' question to qualify as a Socratic definition? S doesn't tell us explicitly. Instead,  S asks E to define piety. E offers a candidate, C, as a definition of piety. Then S elicits further claims that seem to entail that C is not, in fact, the definition of piety. S asks E to try again. E offers D as a definition of piety. S elicits further claims that seem to entail that D is not, in fact, the definition of piety. The process repeats.\\ 

\noindent By focusing on how S argues against various candidate definitions of piety, we will identify what he thinks is required of a satisfactory answer. 

\section*{Failed Answers}
\begin{enumerate}
\item Prosecuting the wrongdoer regardless of personal relationship to the wrongdoer (5e).

\begin{itemize}\item{This is an \emph{example} of a (kind of) pious action.}
\item S didn`t ask for just one or two examples of pious  actions, but that one form (eidos, idea) in virtue of which all pious actions are pious, so that he can use that form as a model to say of any action whether it is pious.

\item{Moral: S thinks an adequate answer must specify that in virtue of which \textbf{every} pious action is pious.}\end{itemize}

\item{What is dear to the gods (7a).}

\begin{itemize}\item{If the gods disagree on important ethical matters (as E agrees they do at 6c and 7b--8a), then one and the same thing could be both pious (because dear to some god(s)) and impious (because hated by some other god(s)), but that is impossible.}
\item{Moral: S thinks that whatever makes some pious actions pious cannot at the same time make some impious actions impious; if X makes some actions pious, then everything which is X is pious.}\end{itemize}

\item What is dear to all the gods (9e).
\begin{itemize}
\item...below...
\end{itemize}
\end{enumerate}

\section*{Preliminary Results}

Socrates' rejections of these various proposal shows us that a satisfactory answer to the question `what is piety?', and more generally to all `what is X? questions must satisfy certain criteria. An adequate answer to a `what is X?' question must identify some P such that the following conditions obtain:
\begin{description}
\item[ X is defined as P if and only if:]\
\begin{description}
\item[General:] everything which is X is also P. If piety is defined as P, then everything which is pious must also be P. If sacrificing is pious and praying is pious, then both sacrificing and praying must also be P. 
\item[Univocal:] everything which is P is also X. If piety is defined as P, then everything which is P must also be pious. If killing your enemy in battle is P, killing your enemy must also be pious. 
\end{description}
\end{description}
%\begin{description}
%\item[General:] everything which is X is also AB. If piety is defined as being God loved, then everything which is pious must also be God loved. If, then, sacrificing is pious and praying is pious, then both sacrificing and praying must be God loved. 
%\item[Univocal:] everything which is AB is also X. If piety is defined as being God loved, then everything which is God loved must also be pious. If, then, killing your enemy is battle is God loved, killing your enemy must also be pious. 

\noindent Think of these criteria as providing tests for candidate definitions. E proposes definitions of piety. S examines whether those definitions are general and univocal. 






%\begin{itemize}\item{This only gives us a quality or affection of piety, it does not tell us what piety is.}\end{itemize}

%\item{The part of justice concerned with the care of the gods (12e)}

%\begin{itemize}\item{Care for \emph{X} aims at benefitting \emph{X} or making \emph{X} better; But gods cannot be made better}\end{itemize}

%\item{The part of justice concerned with service to the gods (13d)}

%\begin{itemize}\item{Service aims at some goal (e.g. service to generals aims to help them win wars, service to house builders aims to help them build houses) but E. can't specify what ``fine thing'' gods achieve such that service could aim to help them achieve that goal}\end{itemize} 

%\item{Knowledge of how to sacrifice and pray (14d)}

%\begin{itemize}\item{This definition reduces to [3] and the claim that sacrifices and prayers are dear to all the gods}\end{itemize}

%\end{enumerate}



\section*{E's Third Definition}

E's third definition of piety satisfies these two conditions: what is dear to \emph{all} the gods (9e). 
\begin{itemize} 
\item Generality test:  everything which is pious is dear to all the Gods. 
\item Univocity test: everything which is dear to all the Gods is also pious
\end{itemize}
While S agrees that E's third definition is an improvement, he thinks it fails a third test, which he introduces by asking, ``is the pious loved by the gods because it's pious? Or is it pious because it's loved?'' (10a). Don't get lost in the details. Our interest is how S's criticism of this third definition reveals a third criterion for a Socratic definition: 
\begin{description} 
\item[Explanatory:]  every instance of X is so because it is P. If piety is defined as P, then a pious act, say, praying to the Gods, must be pious precisely because it is P. 
\end{description}
In examining S's complicated argument that E's third definition fails this criterion, remember that our interest is primarily the criterion and not the success or failure of this particular view of piety. 


\section*{Euthyphro Dilemma}

%So, consider prosecuting the wrongdoer. Only one of the following can be true:
%\begin{enumerate}
%\item Prosecuting the wrong-doer is pious \emph{because} such prosecution is loved by all the gods.
%\item All the gods love prosecution of wrongdoers \emph{because} such prosecution is pious.
%\end{enumerate}
%S has observed that even if A and B are strictly correlated, A might still explain B, or vice versa. For instance, assume that there is a strict correlation between the crowing of a rooster and the rising of the sun (you need to assume there has been roosters for as long as the sun existed). Do you think that the sun rises because the rooster crows? Or does the rooster crow because the sun rises? This is not a trick question. There are many correlations where one correlate explains the other, e.g., death and the cessation of brain activity---which is the cause of the other? How about death and rigor mortis?

Assume being loved by all the gods is unique to, and shared by, all pious acts, i.e., everything which is pious is god-loved and nothing which is not pious is god loved. Does being god loved explain why something is pious? Or is the fact that something is pious explain why the gods love it? This question leads to what is called the \emph{Euthyphro dilemma}, a dilemma that has had continued interest for its apparent refutation of the divine command theory, which says, roughly, that things are morally good or bad, or morally obligatory, permissible, or prohibited, solely because of God’s will or commands. The dilemma consists in arguments against both lemmas (both horns of the dilemma).\\

\noindent \emph{First Lemma:}

\begin{enumerate}
\item[P1.] Assume that certain actions are pious (right,  obligatory, etc.) \emph{because} they are dear to the gods (or commanded by God/the gods, etc.)
\item[P2.] If P1 is true, then, if the gods (or God) love theft, murder, etc. then theft, murder, etc. would be pious (or right, obligatory, etc). 
\item[P3.] Theft, murder, etc., could never be pious (right, obligatory, etc.) 
\item[P4.] Thus, being god loved (or commanded) does not explain why actions are pious (right, obligatory, etc.)
\end{enumerate}

\noindent \emph{Second Lemma:}
\begin{enumerate}
\item[P5.] Assume that certain actions are dear to the gods (or God) because those actions are pious (right, obligatory, etc.)
\item[P6.] If P5 is true, the actions have some feature, other than being god loved, that explain why they are pious. 
\item[P7.] Thus,  the attitude of the gods (or God) is not what \emph{explains} or \emph{makes it the case} that they are pious (right, obligatory, etc).
\item[P8.] Thus, being god loved (or commanded) does not explain why actions are pious (right, obligatory, etc.)
\end{enumerate}

\section*{Back to the dialog}
\noindent Socrates argues that the fact that a certain action is pious explains why (all) the gods love it, and not the other way round. Let's reconstruct the argument from 10a--11b. If S can show that X's being pious explains why all the gods love X, then he has shown that piety is not identical to being loved by all the gods. (We can drop the 'all' qualifier below). 

\begin{description}
\item[Step 1:] S distinguishes between something's being \emph{X}-ed and something's \emph{X}-ing. For instance, there is a difference being being carried and carrying something; being led and leading; being seen and seeing. Similarly, there is a distinction between being loved and loving. The former is about the state of being the recipient of love from another. The latter is about the state in which one is actively loving another.
\item[Step 2:] S then claims that something \emph{X}-ing \emph{Y} is prior to \emph{Y}'s being \emph{X}-ed:
\begin{itemize}
\item If we ask "Why is this piece of chalk being carried (led, seen)?'' the answer is "Because Dr. O'Connor is carrying (leading, seeing) it''.
\item If we ask "Why is Dr.  O'Connor carrying this piece of chalk?'' the answer is \textbf{not} "Because the piece of chalk is being carried by Dr. O'Connor''.
\end{itemize}
\item[Step 3:] Similarly, the gods loving X is prior to X being loved by the gods, i.e., the fact that the gods love X explains why X is being loved by the gods, not the other way around. Or to put it slightly differently, the fact that X is receiving love from the gods does not explain why the gods are actively loving X. We should really say that X is receiving love from the gods because the gods are actively loving X. This may help: 
\begin{itemize}
\item X loves Y.
\item Y is loved by X. 
\item X does not love Y because Y is loved by X.
\item Y is loved by X because X loves Y.
\end{itemize}
\item[Step 4:] Suppose, now, we ask, "Why do the gods love X?''. There are two possible answers:
\begin{description}
\item[Option 1:] Because X is loved by the gods. 
\begin{itemize}
\item Absurd! Remember the carrying of the chalk: the answer to "Why is Dr. O'Connor carrying the chalk'' was not "Because the chalk is being carried by Dr. O'Connor''.
\item We have argued above that X is  (being) loved by the Gods because the Gods (actively) love X.
\end{itemize}
\item[Option 2:] Because X is pious. What, then, is the relationship between piety and being loved by the gods? Either (1) piety = being loved by the gods, or (2) piety \# being loved by the Gods. If (1), S argues that we commit the same error as option 1, so (2) must be true. Argument against (1):
   \begin{itemize}
   \item Assume that 'to be pious' = 'to be loved by the Gods', i.e., 'X is loved by the Gods' means that 'X is pious'.
   \item If X, a pious thing, is loved by the Gods because it is pious, then X is loved by the Gods because X is loved by the Gods (by substituting identicals). 
  \item But, we've shown that to be impossible.
  \item Therefore: to be pious \# to be loved by the gods.
\end{itemize}
\end{description}
\end{description}

\section*{Conclusion}

S has shown that pious \# to be loved by the gods. At best, these two strictly correlate with one another. But don't get too bogged down in this specific argument; perhaps you are unsure how exactly S argues that pious \# to be loved by the gods. The general point you want to take away from this week is that S believes an adequate definition of X must satisfy three conditions: 

\begin{description}
\item[ X is defined as P if and only if:]\
\begin{description}
\item[General:] everything which is X is also P. If piety is defined as P, then everything which is pious must also be P. If sacrificing is pious and praying is pious, then both sacrificing and praying must also be P. 
\item[Univocal:] everything which is P is also X. If piety is defined as P, then everything which is P must also be pious. If killing your enemy in battle is P, killing your enemy must also be pious. 
\item[Explanatory:]  every instance of X is so because it is P. If piety is defined as P, then a pious act, say, praying to the Gods, must be pious precisely because it is P. 
\end{description}
\end{description}
\noindent As you read the dialogs, you will encounter S asking supposed experts to tell him the nature of various things. When he argues with someone that they fail to adequately define something, you should ask which of these conditions the relevant answer does not fulfill. 



%\begin{itemize}
%\item{Socrates asks ``Is the pious loved by the gods because it's pious? Or is it pious because it's loved?''}
%\item{Distinction between something's being \emph{X}ed and something's \emph{X}ing (e.g. something's being carried and something's carrying; being led and leading; being seen and seeing)}

%\begin{itemize}\item{S claims that something \emph{X}ing \emph{Y} is prior to \emph{Y}'s being \emph{X}ed}
%\begin{itemize}\item{E.g. If we ask ``Why is this piece of chalk being carried (led, seen)?'' the answer is ``Because Professor O'Connor is carrying (leading, seeing) it''}\item{If we ask ``Why is Professor  O'Connor carrying this piece of chalk?'' the answer is \emph{not} ``Because the piece of chalk is being carried by Professor O'Connor''}\end{itemize}\end{itemize}
%\item{So, by analogy, something's being loved by the gods is \emph{posterior} to the gods' loving it (i.e. the fact that the gods love it explains why it is being loved, not the other way around)}

%\item{If we ask, ``Why do the gods love it?'', the answer is, ``Because it is pious'' and \emph{not} ``Because it is loved by the gods'' (remember the carrying of the chalk: the answer to ``Why is Professor O'Connor carrying the chalk'' was not ``Because the chalk is being carried by Professor O'Connor'')}

%\item{But, as we saw above, the answer to ``Why is it god-loved?'' is ``Because the Gods love it''} \item{But if ``To be pious'' and ``To be God-loved'' were the same thing (as E claims), then:}\begin{itemize}\item{If the pious were loved because it's pious, [by substituting identicals], the god-loved would be loved because it's god-loved}\item{But, we've shown that to be impossible}\end{itemize}\item{Therefore: To be pious $\neq$ To be dear to the gods}

%\end{itemize}










\end{document}



