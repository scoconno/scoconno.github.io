\documentclass[oneside]{article}
 \headheight = 25pt
\footskip = 20pt
\usepackage{mdwlist}
\usepackage[T1]{fontenc}
\renewcommand{\rmdefault}{ppl}
\usepackage{fancyhdr}
 \pagestyle{fancy}
 \lhead{\textbf{\textsc{\small Scott O'Connor\\Ancient Philosophy}}}
 \chead{}
 \rhead{\large\textbf{\textsc{Euthyphro 2}}}
 \lfoot{\footnotesize{\thepage}}
 \cfoot{}
 \rfoot{\footnotesize{\today}}
 \usepackage{longtable,booktabs}
\tolerance=700


\begin{document}





\section*{Re-cap}


A \textbf{Socratic definition} is an answer to a ``what is X?'' question. These definitions are not of words, but of things. S wants to know the nature of piety itself and not just what the word `piety' means. S asks E to define piety. E offers a candidate, P, as a definition of piety. S elicits further claims that seem to entail that P is not, in fact, the definition of piety. S asks E to try again. The process repeats. The first two rejected answers were as follows:

\begin{enumerate}
\item Prosecuting the wrongdoer regardless of personal relationship to the wrongdoer (5e).
\item What is dear to the gods (7a).
\end{enumerate}
Socrates' rejection of these two proposal showed us that a satisfactory answer to the question `what is piety?', and more generally to all `what is X? questions, must satisfy certain conditions.
\begin{description}
\item[ X is defined as P if and only if:]\
\begin{description}
\item[General:] everything which is X is also P. If piety is defined as P, then everything which is pious must also be P. If sacrificing is pious and praying is pious, then both sacrificing and praying must also be P. 
\item[Univocal:] everything which is P is also X. If piety is defined as P, then everything which is P must also be pious. If killing your enemy in battle is P, killing your enemy must also be pious. 
\end{description}
\end{description}

\section*{E's Third Definition}
E's third definition of piety satisfies these conditions: what is dear to \emph{all} the gods (9e). 
\begin{itemize} 
\item Passes the general test:  everything which is pious is dear to all the Gods. 
\item Passes the univocal test: everything which is dear to all the Gods is also pious
\end{itemize}
While S agrees that E's third definition is an improvement, he thinks it fails a third test, which he instroduces by asking, ``is the pious loved by the gods because it's pious? Or is it pious because it's loved?'' (10a). So, consider prosecuting the wrongdoer. Only one of the following can be true:
\begin{enumerate}
\item Prosecuting the wrong-doer is pious \emph{because} such prosecution is loved by all the gods.
\item All the gods love prosecution of wrongdoers \emph{because} such prosecution is pious.
\end{enumerate}
S has observed that even if A and B are striclty correlated, A might still explain B, or vice versa. For instance, assume that there is a strict correlation between the crowing of a rooster and the rising of the sun (you need to assume there has been roosters for as long as the sun existed). Do you think that the sun rises because the rooster crows? Or does the rooster crow because the sun rises? This is not a trick question. There are many correlations where one correlate explains the other, e.g., death and the cessation of brain activity---which is the cause of the other? How about death and rigor mortis?



\section*{Euthyphro Dilemma}
Assume being loved by all the gods is unique to, and shared by, all pious acts, i.e., everything which is pious is god-loved and nothing which is not pious is god loved. Does being god loved explain why something is pious? Or is the fact that something is pious explain why the gods love it? This question leads to what is called the \emph{Euthyphro dilemma}, a dilemma that has had continued interest for its apparent refutation of the divine command theory, which says, roughly, that things are morally good or bad, or morally obligatory, permissible, or prohibited, solely because of God’s will or commands. The dilemma consists in arguments against both lemmas (both horns of the dilemma).\\

\noindent \emph{First Lemma:}

\begin{enumerate}
\item[P1.] Assume that certain actions are pious (right,  obligatory, etc.) \emph{because} they are dear to the gods (or commanded by God/the gods, etc.)
\item[P2.] If P1 is true, then, if the gods (or God) love rape, murder, etc. then rape, murder, etc. would be pious (or right, obligatory, etc). 
\item[P3.] Rape, murder, etc., could never be pious (right, obligatory, etc.) 
\item[P4.] Thus, being god loved (or commanded) does not explain why actions are pious (right, obligatory, etc.)
\end{enumerate}

\noindent \emph{Second Lemma:}
\begin{enumerate}
\item[P5.] Assume that certain actions are dear to the gods (or God) because those actions are pious (right, obligatory, etc.)
\item[P6.] If P5 is true, the actions have some feature, other than being god loved, that explain why they are pious. 
\item[P7.] Thus,  the attitude of the gods (or God) is not what \emph{explains} or \emph{makes it the case} that they are pious (right, obligatory, etc).
\item[P8.] Thus, the divine command theory is false.
\end{enumerate}
\noindent ...\emph{Class discussion of general argument}...

\section*{Back to the dialog: class project}
\noindent Socrates argues that the fact that a certain action is pious explains why (all) the gods love it, and not the other way round. Let's reconstruct the argument from 10a--11b. 

%\begin{itemize}
%\item{Socrates asks ``Is the pious loved by the gods because it's pious? Or is it pious because it's loved?''}
%\item{Distinction between something's being \emph{X}ed and something's \emph{X}ing (e.g. something's being carried and something's carrying; being led and leading; being seen and seeing)}

%\begin{itemize}\item{S claims that something \emph{X}ing \emph{Y} is prior to \emph{Y}'s being \emph{X}ed}
%\begin{itemize}\item{E.g. If we ask ``Why is this piece of chalk being carried (led, seen)?'' the answer is ``Because Professor O'Connor is carrying (leading, seeing) it''}\item{If we ask ``Why is Professor  O'Connor carrying this piece of chalk?'' the answer is \emph{not} ``Because the piece of chalk is being carried by Professor O'Connor''}\end{itemize}\end{itemize}
%\item{So, by analogy, something's being loved by the gods is \emph{posterior} to the gods' loving it (i.e. the fact that the gods love it explains why it is being loved, not the other way around)}

%\item{If we ask, ``Why do the gods love it?'', the answer is, ``Because it is pious'' and \emph{not} ``Because it is loved by the gods'' (remember the carrying of the chalk: the answer to ``Why is Professor O'Connor carrying the chalk'' was not ``Because the chalk is being carried by Professor O'Connor'')}

%\item{But, as we saw above, the answer to ``Why is it god-loved?'' is ``Because the Gods love it''} \item{But if ``To be pious'' and ``To be God-loved'' were the same thing (as E claims), then:}\begin{itemize}\item{If the pious were loved because it's pious, [by substituting identicals], the god-loved would be loved because it's god-loved}\item{But, we've shown that to be impossible}\end{itemize}\item{Therefore: To be pious $\neq$ To be dear to the gods}

%\end{itemize}



\section*{Final Result}
S believes that an adequate answer to a `what is X?' question must identify some P such that the following conditions obtain:
\begin{description}
\item[ X is defined as P if and only if:]\
\begin{description}
\item[General:] everything which is X is also P. If piety is defined as P, then everything which is pious must also be P. If sacrificing is pious and praying is pious, then both sacrificing and praying must also be P. 
\item[Univocal:] everything which is P is also X. If piety is defined as P, then everything which is P must also be pious. If killing your enemy in battle is P, killing your enemy must also be pious. 

\item[Explanatory:]  every instance of X is so because it is P. If piety is defined as P, then a pious act, say, praying to the Gods, must be pious precisely because it is P. 
\end{description}
\end{description}

\end{document}



