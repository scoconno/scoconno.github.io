% !TEX encoding = UTF-8 Unicode
% !TEX TS-program = xelatex

\documentclass[11pt]{article}
\usepackage{fontspec}
\defaultfontfeatures{Mapping=tex-text}
\usepackage{xunicode}
\usepackage{xltxtra}
\usepackage{verbatim}
\usepackage[margin= 1 in]{geometry} % see geometry.pdf on how to lay out the page. There's lots.
\geometry{letterpaper} % or letter or a5paper or ... etc
%\usepackage[parfill]{parskip}    % Activate to begin paragraphs with an empty line rather than an indent 
\usepackage{mathrsfs}
\usepackage{bbding}
\usepackage[usenames,dvipsnames]{color}
\usepackage{natbib}
\usepackage{stmaryrd}
%\usepackage{mathpartir}
\usepackage{txfonts}
\usepackage{graphicx}
\usepackage{fullpage}
\usepackage{hyperref}
\usepackage{amssymb}
\usepackage{epstopdf}
\usepackage{fontspec}
%\setmainfont{Hoefler Text}
\setmainfont[BoldFont={Minion Pro Bold}]{Minion Pro}
\usepackage{hyperref}
\usepackage{lastpage, fancyhdr}
%\usepackage{setspace}
\pagestyle{fancy}
\lhead{}
\chead{Lecture 8, Plato's \emph{Meno} continued\space---\space Handout} 
\rhead{}
\lfoot{}
\cfoot{\thepage\space of \pageref{LastPage}} 
\rfoot{}
\footskip=30 pt
\headsep=20pt
\thispagestyle{empty}
\hypersetup{colorlinks=true, linkcolor=Sepia, urlcolor=Sepia, citecolor=BrickRed}
\DeclareGraphicsRule{.tif}{png}{.png}{`convert #1 `dirname #1`/`basename #1 .tif`.png}
\usepackage{polyglossia}
\setdefaultlanguage{english}
\setotherlanguage{greek}
\newfontfamily\greekfont{Gentium Plus}
\newcommand{\gk}[1]{\textgreek{#1}}
\newcommand{\gloss}[1]{(\textgreek{#1})}

\usepackage{covington}
\usepackage{fixltx2e}
\usepackage{graphicx}
\begin{document}

%\maketitle
\thispagestyle{empty}
\begin{center} \LARGE{Sample Answers}\end{center}
\thispagestyle{empty}\vspace*{3mm}

\noindent\underline{Short answer: Answer as specifically and fully as you can}
\vspace*{2mm}

\noindent Why does Socrates reject Euthyphro's proposal that the pious is what is loved by the gods?
\vspace*{2mm}

\noindent [1] Because there are many gods, and the gods differ in what they love. [2] This entails that one and the same thing could be both loved and hated by the gods. On Euthyphro's proposal, this would entail that one and the same thing could be both pious and impious. However, since one and the same thing cannot have contradictory properties, any definition of piety that allows one and the same thing to be both pious and impious cannot be correct.
\vspace*{2mm}

\noindent It is important to include both [1], which states the main fact (or alleged fact) that Socrates cites and [2], which explains how that fact entails or leads to the rejection of Euthyphro's proposal.
\vspace*{4mm}

\noindent\underline{Passage identification: Identify the dialogue in which the following passage appears, then briefly describe}\\\underline{\emph{both} its main philosophical point \emph{and} its relevance to the broader argument of the dialogue in which it is found} 
\vspace*{2mm}

\noindent ``It's not from wealth that virtue comes, but from virtue money, and all the other things become good for human beings, both in private and in public life.''
\vspace*{2mm}

\noindent Dialogue: \emph{Apology}
\vspace*{2mm}

\noindent The main philosophical point expressed in this passage is that many things which people typically consider good, such as wealth, are, in fact, only good for a certain type of person, namely the virtuous person. If someone is not virtuous, things such as wealth can, in fact, be bad for him/her. Socrates makes this point in order to support his overall claim that people should care first and foremost about the state of their soul.

\end{document}