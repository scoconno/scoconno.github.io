% !TEX encoding = UTF-8 Unicode
% !TEX TS-program = xelatex

\documentclass[11pt]{article}
\usepackage{fontspec}
\defaultfontfeatures{Mapping=tex-text}
\usepackage{xunicode}
\usepackage{xltxtra}
\usepackage{verbatim}
\usepackage[margin= 1 in]{geometry} % see geometry.pdf on how to lay out the page. There's lots.
\geometry{letterpaper} % or letter or a5paper or ... etc
%\usepackage[parfill]{parskip}    % Activate to begin paragraphs with an empty line rather than an indent 
\usepackage{mathrsfs}
\usepackage{bbding}
\usepackage[usenames,dvipsnames]{color}
\usepackage{natbib}
\usepackage{stmaryrd}
\usepackage{graphicx}
\usepackage{fullpage}
\usepackage{hyperref}
\usepackage{amssymb}
\usepackage{epstopdf}
\usepackage{fontspec}
%\setmainfont{Hoefler Text}
\setmainfont[BoldFont={Minion Pro Bold}]{Minion Pro}
\usepackage{hyperref}
\usepackage{lastpage, fancyhdr}
%\usepackage{setspace}
\pagestyle{fancy}
\lhead{}
\chead{Stoics on Freedom} 
\rhead{}
\lfoot{}
\cfoot{\thepage\space of \pageref{LastPage}} 
\rfoot{}
\footskip=30 pt
\headsep=20pt
\thispagestyle{empty}
\hypersetup{colorlinks=true, linkcolor=Sepia, urlcolor=Sepia, citecolor=BrickRed}
\DeclareGraphicsRule{.tif}{png}{.png}{`convert #1 `dirname #1`/`basename #1 .tif`.png}

\usepackage{covington}
\usepackage{fixltx2e}
\usepackage{graphicx}
\begin{document}

%\maketitle
\thispagestyle{empty}
\begin{center} \LARGE{Ancient Philosophy\\ Stoics on Freedom}\\ \vspace*{2mm}
\large{Scott O'Connor}\end{center}
\thispagestyle{empty}\vspace*{3mm}

\section*{Motivating concern for the Stoics}

\noindent The Stoics become worried about human freedom in a way that just did not seem to be on Plato and Aristotle's radar. They were very concerned with the following argument:
\vspace*{2mm}

[P1] For an action to be ``in our power,'' i.e. for it to be a free action, we must be able to do its opposite\vspace*{1mm}

[P2] Praise and blame (and other responses) are bestowed \emph{only} on free actions
\vspace*{1mm}

[P3] In any situation, the virtuous person cannot act otherwise than she actually does
\vspace*{1mm}

\underline{[P4] In any situation, the virtuous person's action is not free (from [P1] and [P3])}
\vspace*{1mm}

[C] The virtuous person's actions are not praiseworthy (from [P2] and [C1])
\vspace*{2mm}


\vspace*{2mm}

\noindent The argument for [P3:] ``Fate is a sempiternal and unchangeable series and chain of things, rolling and unravelling itself through eternal sequences of cause and effect, of which it is composed and compounded'' (Chrysippus, II-89). Our actions are no different.
\vspace*{2mm}

\noindent This argument is meant to be a \emph{reductio}---[C] is supposed to be obviously false---thus, an actions' being free cannot require that the agent be able to do otherwise (`` `in our power'' is not like that'').




\section*{Stoic Solution}
\noindent The Stoics attempts to reconcile determinism with human freedom by distinguishing different notions of free-will. 
\begin{description}
\item[Libertarian:] An action is free only if it is possible for X to have chosen not to do it \textbf{even though} everything relevant to X's choice to do A \textbf{except} his ``will'' remains the same---e.g. his beliefs, his character, the external world.
\item[Stoic:] An action is free only if it is caused by the agent's beliefs.
\end{description}
\vspace*{2mm}



\noindent Stoics believe our actions are always caused by our beliefs.
\vspace*{2mm}

\noindent Desire/impulse for \emph{X} = belief that \emph{X} is good; fear of \emph{X} = belief that \emph{X} is harmful; and so on
\vspace*{2mm}

\noindent Beliefs arise in the following way: 
\vspace*{2mm}

External object impresses itself upon the soul, which generates an ``impression'' or ``presentation'' that\\\hspace*{6mm}something is the case
\vspace*{1mm}

The soul either assents (i.e. accepts the content of the impression) and a belief is formed, or withholds\\\hspace*{6mm}(i.e. does not accept) and a belief is not formed
\vspace*{2mm}

\noindent Virtue, for the Stoics, is a state of the soul which is causally responsible for ensuring we act on our beliefs (so, the Stoics don't want to reject [P3] above)
\vspace*{2mm}




\noindent Rational mechanisms [agency model]
\vspace*{1mm}

\hspace*{5mm} [1] External world + perceptual nature of person $\rightarrow$ have impression
\vspace*{1mm}

\hspace*{5mm} [2] Have impression + assent to impression $\rightarrow$ believe content of impression
\vspace*{1mm}

\hspace*{5mm} [3] Belief + character (virtue) $\rightarrow$ effect
\vspace*{2mm}

\noindent  On the Stoic view, the paradigm of free action is an action which stems from one's character. A perfectly free action is one which is consistent with every belief and desire the agent has---an action which stems from the ``whole'' person. Hence only the virtuous person is \emph{unconditionally} free, since only the virtuous person has a consistent set of value-related beliefs [and so of desires, which are kinds of beliefs for the Stoics].
\vspace*{2mm}

\section*{Human freedom}

\begin{itemize}
\item Humans differ from things: our individual beliefs and characters are part of the causal process
\item The impressions which we have are causally determined by the state of the external world
\item Our acceptance of rejection of impressions is not causally determined by the external world
\item Free action = action not causally determined by the external world
\item Hence, human action is free
\end{itemize}

\section*{Objection to Stoics and response}

\noindent Either [A] our acts of assent are causally determined by our characters or [B] they are not
\vspace*{1mm}

\noindent If [B], then determinism is false
\vspace*{1mm}

\noindent If [A], there is no human freedom (only a more complex kind of determination)
\vspace*{1mm}

Support: the necessitarian or libertarian opponent can cite the Stoic model of animal action [stimulus-\\\hspace*{6mm}response model]:
\vspace*{1mm}

External world + perceptual nature of animal $\rightarrow$ has impression of thing
\vspace*{1mm}

Impression of thing + appetitive nature of animal $\rightarrow$ effect 
\vspace*{1mm}

\noindent Animals can't act freely. So, main question for Stoics: how do  we differ from animals such that our actions are free?
\vspace*{2mm}

\noindent Stoic response is that we \emph{are} our characters




\end{document}
