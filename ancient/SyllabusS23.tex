\documentclass[article,oneside]{memoir}
\usepackage{long table}
%%% custom style file with standard settings for xelatex and biblatex. Note that when [minion] is present, we assume you have minion pro installed for use with pdflatex.
%\usepackage[minion]{org-preamble-pdflatex} 

%%% alternatively, use xelatex instead
\usepackage{org-preamble-xelatex} 
\usepackage{ulem}
\usepackage{lscape}
\usepackage{multirow}
\def\myauthor{Author}
\def\mytitle{Title}
\def\mycopyright{\myauthor}
\def\mykeywords{}
\def\mybibliostyle{plain}
\def\mybibliocommand{}
\def\mysubtitle{}
\def\myaffiliation{NJCU}
\def\myaddress{Phil 1}
\def\myemail{soconnor@njcu.edu}
\def\myweb{\href{http://scottoconnor.org/ancient}{http://scottoconnor.org/ancient}}
\def\myphone{}
\def\myversion{}
\def\myrevision{}
\def\myaffiliation{NJCU}
\def\myauthor{Dr. Scott O'Connor}
\def\mykeywords{}
\def\mysubtitle{Syllabus}
\def\mytitle{{\normalsize \myweb \newline} \HUGE Ancient Philosophy}


\begin{document}

%%% If using xelatex and not pdflatex
%%% xelatex font choices
\defaultfontfeatures{}
\defaultfontfeatures{Scale=MatchLowercase}    
% You will need to buy these fonts, change the names to fonts you own, or comment out if not using xelatex.      
\setromanfont[Mapping=tex-text]{Minion Pro} 
\setsansfont[Mapping=tex-text]{Myriad Pro} 
\setmonofont[Mapping=tex-text,Scale=0.8]{Georgia} 

%% blank label items; hanging bibs for text
%% Custom hanging indent for vita items
\def\ind{\hangindent=1 true cm\hangafter=1 \noindent}
\def\labelitemi{$\cdot$}
%\renewcommand{\labelitemii}{~}

%% RCS info string for version tracking
\chapterstyle{article-3}  % alternative styles are defined in latex-custom-kjh/needs-memoir/
%\pagestyle{kjh}

\title{\LARGE \mytitle}     
\author{\Large\myauthor \newline \footnotesize\texttt{\noindent Office hours: \href{http://scottoconnor.org/contact/office/}{http://scottoconnor.org/contact/office/}}}
\date{1/17/2023--5/15/2023}
\published{\textbf{HON 360-1 (1786), 3 credits, Spring 2023, M \& W 11:20am--12:35pm, K429}}\\


\maketitle

%\thispagestyle{kjhgit}

% Copyright Page
%\textcopyright{} \mycopyright


%
% Main Content
%

\section{Approved Substitution}

This course substitutes for PHIL 234: Ancient Philosophy. As such, it satisfies a major requirement for the Philosophy, B.A., and an elective in the Philosophy and Religion minor. If you would like to avail of this substitution, you must discuss it with me. It might also be used as an elective by History students. 

\section{Disclaimer}
 This syllabus is subject to change at the discretion of the faculty. Students will be notified of such changes ahead of time via Blackboard. 


\section{Copyright}
The materials used in this class, including, but not limited to, lectures, exams, quizzes, and homework assignments are copyright protected works.  Any unauthorized copying of the class materials or recording of lectures is a violation of federal law and may result in disciplinary actions being taken against the student.  Additionally, the sharing of class materials without the specific, express approval of the instructor may be a violation of the University's Student Honor Code and an act of academic dishonesty, which could result in further disciplinary action.  This includes, among other things, uploading class materials to websites for the purpose of sharing those materials with other current or future students. 

\section{Course Description}

What do you know? Do you know anything? What exists? Are there objective truths about what’s right and wrong for you to do, or is it all a matter of convention? Does being a moral person go against your self-interest? If so, why should you be a moral person? What is happiness? Will being a moral person contribute to your happiness? These questions were raised by philosophers speaking and writing in Greek over two millennia ago. In this course, we will think hard about these questions and try to identify how they were answered by three of the most influential philosophers of all time---Socrates, Plato, Aristotle, and the Stoics. 

\section{Learning Objectives}

Upon completing this course students will be able to (i) read philosophical texts, (ii) clearly and charitably explain viewpoints that
are not their own, (iii) think critically and philosophically, (iv) write well-structured prose in which they clearly state a thesis and persuasively defend it, (v) demonstrate an understanding of the philosophies of Socrates, Plato, and Aristotle.


\section{Readings}
\section{Required Textbooks}

Available in the campus book store and online retailers. Please bring the readings to class. Physical copies required.
\begin{itemize}
\item \href{https://www.amazon.com/Readings-Ancient-Greek-Philosophy-Aristotle/dp/1624665322/ref=dp_ob_title_bk}{`Readings in Ancient Greek Philosophy: From Thales to Aristotle', 5th edition, Cohen, Curd, Reeve}


\end{itemize}
\subsection{Optional}

\begin{itemize}
\item \href{https://www.amazon.com/Classical-Thought-History-Western-Philosophy/dp/0192891774/ref=sr_1_1?s=books&ie=UTF8&qid=1515009994&sr=1-1&keywords=classical+thought}{`Classical Thought', Terence Irwin}, a general introduction to ancient philosophy.

\item \href{http://www.amazon.com/Style-Lessons-Clarity-Grace-11th/dp/0321898680/ref=sr_1_1?ie=UTF8&qid=1452356026&sr=8-1&keywords=lessons+in+clarity+and+grace}{`Style: Lessons in Clarity and Grace', Joseph Williams and Joseph Bizup}, a good guide on how to improve your writing. 
\end{itemize}
\section{Course Website}
There is both a Blackboard site and website for this course (link on first page). Clicking the first link on the left panel within the Blackboard site will bring you to the course website. All assignments will be submitted through Blackboard. Readings, notes, etc. will be posted on the course website. Note that Blackboard difficulties are rare and automatically reported to instructors. Under no circumstance will a student's report of a Blackboard difficulty be reason for an extension. It is your responsibility to contact Blackboard support for help.


\section{Requirements}

\begin{itemize}


\item \textit{Workload:} expect to spend an average of 6 hours per week completing the readings and assignments. NJCU abides by the Federal and State definitions of a credit hour and adopts a policy consistent with the Carnegie Unit. A three-credit class represents 112.5 hours total of work. See \href{http://scottoconnor.org/resources/Credit.pdf}{here} for more details.




\item \textit{Reading questions} submitted through Blackboard. These are multiple choice questions that you must complete before we discuss the material in class. 10 will be assigned. 8 must be completed. If more than 8 are completed, the lowest grades will be dropped. 

%\item \sout{\textit{Experiential learning related assignments:} short reflective on-site assignments (20 points total), or four non-experiential learning related essays (5 points each).}

%\item \textit{Two papers} These wi course project will include a proposed question and exploratory paper (10 points), first draft (10 points), presentation (10 points), and final paper (20 points).

\item \textit{Two papers} submitted through Blackboard. Each has a word limit of 1250-1500 words. 

\item \textit{Mock Trial} assignments, including oral performance, group submission, and individual reflections. 

\item \textit{Stoic Week Journal} submitted through Blackboard.


\item \textit{Final examination.} This will include short answers and longer essays.  

\item \textit{Course evaluations} completed online. 3 points extra credit for successful completion.

\item \textit{Grade Distribution:} Reading questions--3 points each (24 total); papers--15 points each (30 total); mock trial assignments--26 points; Stoic journal final examination--30 points.



\item \textit{Grade Breakdown:}

 \begin{tabular}{ | l | l | p{2cm} | l | l | }
    \hline 
96--100 & A  & &  77--79 &  C+ \\  
90--95 & A- & &  73--76 & C \\
87-89 & B+ &  &  70--72 & C- \\ 
83--86 & B  & &  60--69 & D\\
80--82 & B - & & 0--59 & F\\ \hline
    \end{tabular}


\end{itemize}





\section{Policies}

\begin{itemize}

\item \textbf{Student Responsibility:} This syllabus outlines the required text, assignments, requirements, and policies for this course. By taking this course, you agree to read this syllabus and be bound by those requirements and policies. 

 \item \textit{Academic Integrity:} All the work you turn in (including papers, drafts, and discussion board posts) must be written by you specifically for this course. It must originate with you in form and content with all contributory sources fully and specifically acknowledged. Being a student at NJCU requires you to follow \href{http://scottoconnor.org/resources/Plagiarism.pdf}{NJCU's Academic Integrity Policy.} Penalties for violations are as follows: 1st infraction will result in a 0 for the assignment.  2nd infraction will result in a 0 for the entire course \& application for permanent record on student's transcript. (Repeated violations can lead to expulsion from NJCU). 

\item \textit{Attendance:} You are considered absent if you are (i) not present during roll call, (ii) leave early, (iii) leave without permission, or (iv) leave for an extended period of time. No excuses. No exceptions.



\item \textit{Communication:} To comply with Federal Privacy Laws (FERPA) and NJCU policies, all communication will be through Blackboard and/or official NJCU e-mail. Check Blackboard daily. For further information see \href{http://scottoconnor.org/contact/}{http://scottoconnor.org/contact/}.

\item \textit{Conduct:} Distracting and disrespectful behaviors, including but not limited to eating, leaving your seat, talking out of turn, and aggression are prohibited. Penalties include, but are not limited to, a loss of participation points for the day of violation. Repeat offenders will be reported to the Dean of Students. 

\item \textit{Electronic devices:} Use of electronic device, including, but not limited, to smartphones, dictaphones, tablets, and laptops, is prohibited. Recording a lecture is in violation of Copyright. Penalties include, but are not limited to, a loss of participation points for the day of violation. Repeat offenders will be reported to the Dean of Students.

\item \textit{Format for Written Work:} Submit work to Blackboard as either a pdf, rtf, or doc file. Blackboard will not allow any other format. All work must be typed and neatly presented. 


\item \textit{Grading:} Grades will be available within 1--2 weeks of an assignment being submitted. See: \href{http://scottoconnor.org/resources/grading}{http://scottoconnor.org/resources/grading} for further information.


\item \textit{Late work \& Make-up Policy:} See the assignment schedule below. No make-ups or late work accepted under any circumstances. No exceptions under any imaginable circumstances.

\item \textit{Statement for students with disabilities:} If you are a student
with a disability and wish to receive consideration for reasonable
accommodations, please register with the Office of Specialized Services
and Supplemental Instruction (OSS/SI). To begin this process, complete
the registration form available on the OSS/SI website at
\href{http://www.njcu.edu/oss}{http://www.njcu.edu/oss}
(listed under Student Resources-Forms). Contact OSS/SI at 201-200-2091
or visit the office in Karnoutsos Hall, Room 102 for additional
information.

\item \textit{Turnitin:} Students agree that by taking this course all assignments are subject to submission for textual similarity review to Turnitin.com. Assignments submitted to Turnitin.com will be included as source documents in Turnitin.com's restricted access database solely for the purpose of detecting plagiarism in such documents.  The terms that apply to the University’s use of the Turnitin.com service are described on the Turnitin.com web site.  For further information about Turnitin, please visit: http://www.turnitin.com 

\item \textit{SafeAssign:} Students agree that by taking this course all assignments are subject to submission for textual similarity review through Blackboard SafeAssign. Assignments submitted to SafeAssign will be included as source documents in SafeAssign's restricted access database solely for the purpose of detecting plagiarism in such documents.  


\end{itemize}





\section{Weekly Course Schedule}
Readings can be found in the required textbook. See above for abbreviations. Some references are given by Stephanus numbers.\\ See \href{http://www.columbia.edu/itc/lithum/wong/stephanus.html}{http://www.columbia.edu/itc/lithum/wong/stephanus.html} for an explanation. Readings marked with an $^{*}$ can be found on the course website.

 Changes to the syllabus will be announced through Blackboard and \emph{via} your NJCU email address.  All assignments must be submitted through Blackboard by Sunday, 11:59pm. No late work accepted. No exceptions.  

\newpage
\begin{landscape}
\begin{center}
\begin{longtable}{p{8cm}p{4cm}p{8cm}}
 
  \caption{Course Schedule} \\
  \toprule
  \textbf{Topic}   & \textbf{Assignments} & \textbf{Reading} \\
  \midrule

  
[1.] Introduction (1/16) 						& No assign.		&  \\ 
\textbf{No class on Monday}					&				& \\
Eastern Mediterranean in the 6th Century			&				& \\
Homer and Hellenic Conception of the World		 &				&    \\ [1.8\baselineskip] \hline

[2.] Socrates (1/23)	  			&  No assign.			& --`Socrate', 1st half (in class)'  \\
Socrates and Athens				&					 & --`Euthyphro'  (in RAGP)  \\ [1.8\baselineskip] \hline



[3.] S. Trial (1/30)				& Reading Qs 1		& --`Apology' (in RAGP)  \\ 
							&					& -- Selections from Xenephon$^{*}$\\[1.8\baselineskip]  \hline % (\emph{Meno's slave receives a geometry lesson}) \\ 


[4.] Corruption (2/6)					& Reading Qs 2		&  --`Meno' (in RAGP \\
			    	    				& 					&  `The Clouds', Aristophanes$^{*}$	 \\ [1.8\baselineskip]  \hline 

[5.] Corrupting the young (2/13)			& Reading Qs 3		&  --`Gorgias' (in RAGP) \\
						      		& 					&  --`Symposium' (in RAGP)  \\ [1.8\baselineskip]  \hline 

[6.] Civic Disobedience (2/20)			& Reading Qs 4		&  --`Crito' (in RAGP) \\
No class on Mon					& 					&  \\ [1.8\baselineskip]  \hline 


[7.] Plato's Defense of Socrates (2/27)		& Reading Qs 5				& --`Phaedo', 57a--118a (in RAGP)	 \\
Death, forms, mind-body					                                  & 							&  \\ [1.8\baselineskip]  \hline

	
[8.] Spring Break (3/6)		 			& 				    	 &  \\ [1.8\baselineskip]  \hline  

							
[9.] Plato 1 (3/13)					& Readings Qs	6		& -- `Republic', Bk 2, 357a--377d (in RAGP)  \\ 
Glaucon's Challenge					& 	 				& -- `Republic', Bk 4, 427d--434d (in RAGP) \\ 
Justice in the City 					&					&    	\\ [1.8\baselineskip] \hline	

[10.] Plato 2 (3/20)					& 					& --`Republic', Bk 4, 434d-445e (in RAGP) \\
Justice in the Soul	         			& 					& --`Republic', Bk 8 \& 9 (in RAGP)   \\  
Types of Constitutions                    		& 					&   \\ [1.8\baselineskip] \hline	

[11.] Trial 1 (3/27)					&  				&  \\
Opening Arguments				 	&  				&  \\  
Prosecution Witnesses			 	&  				&  \\  [1.8\baselineskip] \hline
                            

[12.] Trial 2  (4/3)					&             		  	&  \\ 
Defense Witnesses			    		& 				&  \\ 
Closing Argument					&  				&  \\  [1.8\baselineskip] \hline

							
							
[13.] Aristotle (4/10)					& Readings Qs 9 		& -- `Physics',  Bk 1.5-8 (in RAGP)  \\ %forms
Introduction						&					& --`Physics', Bk 2 (in RAGP) \\ 				
The Four Causes					&					& --`De Anima', Bk 1 \& 2 (in RAGP) \\ [1.8\baselineskip] \hline				


[14.] Aristotle's Ethics (4/17)			& 	 				& -- `Nicomachean Ethics', Bk 1 \& 2 (in RAGP) \\
The Function Argument				&					& 	 \\ [1.8\baselineskip] \hline

[15.] Stoic Week (4/24)				& Journal entries 		&  See website \\  [1.8\baselineskip] \hline


[16.] Continued (5/1) 				& Journal entries.			& \\ 
No class on Wed.					& 					& \\ [1.8\baselineskip] \hline
	
[17.] Continued (5/8) no w				& Research Project 	& \\ 
Finals (5/10)						& 				& \\ [1.8\baselineskip] \hline

[18.] Finals (5/10)					& Final 12/15		& \\ [1.8\baselineskip] \hline
	

\end{longtable}
\end{center}
\end{landscape}
\end{document}


%% Uncomment if you want a printed bibliography.
%\printbibliography 


