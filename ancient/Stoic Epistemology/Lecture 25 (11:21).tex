% !TEX encoding = UTF-8 Unicode
% !TEX TS-program = xelatex

\documentclass[11pt]{article}
\usepackage{fontspec}
\defaultfontfeatures{Mapping=tex-text}
\usepackage{xunicode}
\usepackage{xltxtra}
\usepackage{verbatim}
\usepackage[margin= 1 in]{geometry} % see geometry.pdf on how to lay out the page. There's lots.
\geometry{letterpaper} % or letter or a5paper or ... etc
%\usepackage[parfill]{parskip}    % Activate to begin paragraphs with an empty line rather than an indent 
\usepackage{mathrsfs}
\usepackage{bbding}
\usepackage[usenames,dvipsnames]{color}
\usepackage{natbib}
\usepackage{stmaryrd}
%\usepackage{mathpartir}
\usepackage{txfonts}
\usepackage{graphicx}
\usepackage{fullpage}
\usepackage{hyperref}
\usepackage{amssymb}
\usepackage{epstopdf}
\usepackage{fontspec}
%\setmainfont{Hoefler Text}
\setmainfont[BoldFont={Minion Pro Bold}]{Minion Pro}
\usepackage{hyperref}
\usepackage{lastpage, fancyhdr}
%\usepackage{setspace}
\pagestyle{fancy}
\lhead{}
\chead{Lecture 25, Academic Criticisms of Stoic Epistemology\space---\space Handout} 
\rhead{}
\lfoot{}
\cfoot{\thepage\space of \pageref{LastPage}} 
\rfoot{}
\footskip=30 pt
\headsep=20pt
\thispagestyle{empty}
\hypersetup{colorlinks=true, linkcolor=Sepia, urlcolor=Sepia, citecolor=BrickRed}
\DeclareGraphicsRule{.tif}{png}{.png}{`convert #1 `dirname #1`/`basename #1 .tif`.png}
\usepackage{polyglossia}
\setdefaultlanguage{english}
\setotherlanguage{greek}
\newfontfamily\greekfont{Gentium Plus}
\newcommand{\gk}[1]{\textgreek{#1}}
\newcommand{\gloss}[1]{(\textgreek{#1})}

\usepackage{covington}
\usepackage{fixltx2e}
\usepackage{graphicx}
\begin{document}

%\maketitle
\thispagestyle{empty}
\begin{center} \LARGE{PHIL 321\\ Lecture 25: Academic Criticisms of Stoic Epistemology}\\ \vspace*{2mm}
\large{11/21/2013}\end{center}
\thispagestyle{empty}\vspace*{3mm}
\vspace*{-8mm}

\section*{Origins of Academic skepticism}

\noindent When Arcesilaus was scholarch of Plato's Academy (appx. 268--241 BCE) the Academy ``went skeptical'' (\textbf{NB}: they did not call themselves ``skeptics,'' rather they probably called themselves ``academics'')
\vspace*{2mm}

\noindent The story goes that, as scholarch, Arcesilaus encouraged ``arguing both sides of a question'':
\vspace*{2mm}

[1] They would pick some topic for debate, whether \emph{P}? (e.g. does motion exist, is the skin made of pores?)
\vspace*{-4mm}

[2] A pupil would propound an argument for \emph{P}; they would consider it as a group, see if any refinements\\\hspace*{12mm}needed to/could be made
\vspace*{1mm}

[3] Arcesilaus would come back and propound an argument for not-\emph{P}.
\vspace*{1mm}

[4] They would consider both arguments side-by-side, seeing if one was stronger than the other, and\\\hspace*{12mm}ultimately suspend judgment whether \emph{P} on that basis
\vspace*{2mm}

\noindent This kind of skepticism, whereby no affirmative declarations are made about the world one way or another, seems to have changed after Arcesilaus' death to a form of negative dogmatism---i.e. the claim that nothing can be known
\vspace*{2mm}

\noindent The early kind of skepticism was re-invigorated by Aenesidemus in the 1st Century BCE, who claimed ``Pyrrho'' as his figurehead (and the skepticism is called ``Pyrrhonism'')
\vspace*{-2mm}

\section*{The Stoic criterion of truth}

\noindent According to the Stoics, the criterion of truth is the cataleptic impression, i.e. an impression which
\vspace*{2mm}

[1] Comes from what is
\vspace*{1mm}

[2] Is stamped and impressed in accordance with what is
\vspace*{1mm}

[3] Is such as could not come about from what is not
\vspace*{-2mm}

\section*{Academic ``master argument'' (from \emph{HP} 267-70, 276-77)}

\noindent [P1] Some impressions are true, some are false
\vspace*{1mm}

\noindent [P2] False impressions can't lead to knowledge
\vspace*{1mm}

\noindent [P3] True impressions that can be matched by false impressions of the same kind cannot lead to knowledge
\vspace*{1mm}

\noindent\underline{[P4] All true impressions are such that they can be matched by false impressions of the same kind}
\vspace*{1mm}

\noindent [C] Nothing can be known (no impression can lead to knowledge)
\vspace*{2mm}

\noindent In essence, the Academics are willing to grant that it is possible to have impressions that meet conditions [1] and [2] for a cataleptic impression but insist that it is \emph{impossible} to have an impression that meets condition [3] (i.e. this is what [P4] rejects)
\vspace*{2mm}

\noindent For support, the Stoics point to cases such as: twins, eggs, bees, grains of sand etc.
\vspace*{2mm}

\section*{Further Academic moves}

\noindent The Academics actually went further and advanced an argument about the cognitive life of the wise person, using the Stoics' own conception of wisdom
\vspace*{2mm}

[P1] Assent to an impression yields either: knowledge (i.e. \emph{katal\^{e}psis}) if the impression is cataleptic, or\\\hspace*{13mm}opinion, if the impression is non-cataleptic
\vspace*{1mm}

[P2] There are no cataleptic impressions
\vspace*{1mm}

[P3] If the wise person ever assents to an impression, he or she will have and opinion
\vspace*{1mm}

\underline{[P4] The wise person will never have opinions}
\vspace*{1mm}

[C] The wise person will never assent to an impression (and, thus, will have no beliefs)
\vspace*{2mm}

\noindent [P4] is part of the Stoic conception of wisdom; i.e. to be wise is to be such that you hold all your beliefs as a matter of knowledge (\emph{katal\^{e}psis}) and \emph{never} as a matter of opinion (\emph{doxa})
\vspace*{2mm}

\noindent Thus, the Academics claim, \emph{by the Stoics' own lights}, wisdom must consist in holding no beliefs whatsoever

\section*{The Stoic counter: the ``inactivity'' (\emph{apraxia}) argument (\emph{HP} 272-73)}

\noindent The Stoics attempt to argue that the wise person will assent with the following argument:
\vspace*{2mm}

[P1] Actions are caused by agent's impulse
\vspace*{1mm}

\underline{[P2] Impulses are caused by assenting to (a certain kind of) impression}
\vspace*{1mm}

[C] Hence, if one does not assent, one cannot act
\vspace*{2mm}

\noindent This is meant to be a \emph{reductio}: life is supposed to be impossible without action

\section*{Academic responses to the Stoic counter}

\noindent The Academics responded in two main ways:
\vspace*{2mm}

\noindent [1] On the Stoic view, action is possible without assent---recall the mechanisms involved in animal ``action'' from the Freedom discussion
\vspace*{2mm}

So, even on the Stoic view, assent isn't necessary for action
\vspace*{1mm}

The Stoics might respond that \emph{human} action (i.e. in the full-blown Stoic sense of human action) is im-\\\hspace*{6mm}possible, to which the Academics would likely respond, ``so what?''
\vspace*{2mm}

\noindent [2] ``Life without assent'': picked up by the Pyrrhonian Skeptics

\end{document}
