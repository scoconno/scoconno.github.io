\documentclass[oneside]{article}
 \headheight = 25pt
\footskip = 20pt
\usepackage{mdwlist}
\usepackage[T1]{fontenc}
\renewcommand{\rmdefault}{ppl}
\usepackage{fancyhdr}
 \pagestyle{fancy}
 \lhead{\textbf{\textsc{\small Scott O'Connor\\Ancient Philosophy}}}
 \chead{}
 \rhead{\large\textbf{\textsc{Protagoras}}}
 \lfoot{\footnotesize{\thepage}}
 \cfoot{}
 \rfoot{}
 \usepackage{longtable,booktabs}
\tolerance=700

\begin{document}
\section*{Introduction}

\begin{quote}
\emph{Protagoras took it from there and said, ``Young man, this is what you will get if you study with me: The very day you start, you will go home a better man, and the same thing will happen the day after. Every day, day after day, you will get better and better.''}

\emph{but if he comes to me he will learn only what he has come for. What I teach is sound deliberation, both in domestic matters--how best to manage one`s household, and in public affairs---how to realize one's maximum potential for success in political debate and action}
\end{quote}

\noindent The dialogue starts with Socrates and a friend, Hippocrates, in conversation. Hippocrates very clearly wants to succeed in life and he has heard that Protagoras is a teacher of the skills and qualities necessary for such success. So he asks Socrates to entreat Protagoras on his behalf. Socrates, however, asks Hippocrates whether he knows what it is that Protagoras is supposed to teach people, that is supposed to make them better. Hippocrates is shown to be unclear on the matter. So, when they finally achieve admittance, Socrates sets about trying to figure out what exactly it is that Protagoras teaches. Protagoras (a well-known sophist) claims to teach his students ``sound deliberation'' (\emph{euboulia}), which Socrates equates with the ``art of citizenship'' (\emph{politik\^{e} techn\^{e}}) and ``virtue'' (\emph{aret\^{e}}).  Socrates offers two grounds for skepticism about whether virtue can be taught:
\begin{enumerate}
\item The Athenians let everyone speak about morality and politics and do not appeal to experts (319b4-e1).
\item Good citizens often produce offspring who are not virtuous (319e1-320b4).
\end{enumerate}
\vspace*{2mm}
Protagoars gives various replies: 
\begin{enumerate} 
\item All Athenians are knowledgeable.
\item Justice (virtue) is a necessary condition of political community (323a1-4, 324d8-325a3).
\item The Athenians evidently believe that virtue can be taught as evidenced by their practices of
punishment (323d6-324d1) and their concern with private training in virtue (325ac5- 326e5).
\end{enumerate}


\noindent Let's take a step back. Why the focus on the teachability of virtue? This was a much discussed issue. Could teaching bring about those characteristics that ensure good living and happiness? It is connected to the idea Socrates goes on to explore, whether virtue is knowledge/understanding? But it is easy to get lost in the discussion of what appears three distinction topics, \textbf{virtue}, \textbf{knowledge}, and \textbf{teaching}. The dialog ends in some puzzlement about how these three are related. But perhaps we can at least say the following. Protagoras believes that teachers can only teach knowledge to their students. So, if virtue is teachable, then virtue needs to be knowledge. Socrates will force him to accept that virtue is not knowledge, which entails by Protagoras' own light that it cannot be taught. What is Socrates' own view? 

For Socrates, it seems that virtue cannot be taught, or at least, it cannot be learned from a teacher. But by the end of the dialog, he will argue against Protagoras that virtue is knowledge/understanding. While this suggests that he should revise his earlier view agains the teachability of virtue, perhaps Socrates will insist that some knowledge just cannot be taught at all. And while it is easy to be side tracked by this question, there are some clearer commitments that Socrates makes in this dialog, which is our primary concern. Our focus today is over the nature and unity of virtues. 

\section*{What is a virtue?}

Socrates lists five virtues: justice, piety, self-control, courage and wisdom. We can summarize the core options for the general structure of a virtue as follows: 

\begin{description}
\item[Externalist:] A virtue is defined merely in terms of their some behavior, e.g., courage is \textbf{standing} one`s ground in battle.
\item[Emotivist:] A virtue is defined in terms of a psychological but non-rational mental states, e.g.,  \textbf{feeling of endurance} in battle.
\item[Intellectualist:] Courage is \textbf{knowledge} of what is to be dreaded or dared.
\end{description}




\vspace*{2mm}



\section*{The Unity of the Virtues}
\begin{quote} \emph{You've made this point yourself, and with good reason, I might add. But when the debate involves political excellence, which must proceed entirely from justice and temperance, they accept advice from anyone, and with good reason, for they think that this particular virtue, political or civic virtue, is shared by all, or there wouldn't be any cities. This must be the explanation for it, Socrates.}
\end{quote}




\noindent S asks P, ``Is virtue a single thing, with justice and temperance and piety its parts, or are the things I have just listed all names for a single entity'' (329d). Answering this requires just how the individual virtues might be part of the virtue. Socrates identifies two options for ``parts'':
\begin{itemize}
\item[A] In the sense in which the mouth, nose, eyes, and ears are parts of the face (heterogenous parts). 

\item[B] In the sense in which there are ``parts'' of a block of gold (homogenous parts).
\end{itemize}
\vspace*{2mm}
How are the notions of `part' in A and B different? The most relevant difference concerns separability. In A, each part is clearly delineated and can be removed while leaving the whole intact. For instance, if a person loses an eye, it is clear what part of the face is been lost. It is also clear how the remainder of the face differs from the eye that was lost. In other words, some parts of the face can be had without others. In B, it is difficult to discern clear parts to a perfectly homogenous chunk of gold. If you remove some randomly delineated part, what remains will be qualitatively identical to what was taken away. It seems that there is some sense in which the parts of the chunk of gold can't be separated out in the kind of way the parts of a face can. So, how do the parts of virtue stand to virtue? Can they exist independently, fully unique and clearly distinguishable from one another? This is Protagoras's view. And he holds it primarily because it is the sense of parts that allows for the separability of the virtues. Protagoras thinks that this is obvious and he points to (allegedly) courageous but unjust people, just but unwise people, etc. 

Socrates will challenge this position by really focusing on two virtues, wisdom and courage. He will attempt to show it impossible that a person can be wise but a coward, and impossible that a person could be courageous but ignorant. In other words, he will attempt to show that wisdom cannot be had without courage, and \emph{vice versa}. It's an interesting question how this helps him establish that all virtues are unified similarly, e.g., that you cannot be temperate without being courageous. But it is perhaps useful to observe that Protagoras concedes the following: 
\begin{quote}
\emph{What I am saying to you, Socrates, is that all these are parts of virtue, and that while four of them are reasonably close to each other, courage is completely different from all the rest. The proof that what I am saying is true is that you will find many people who are extremely unjust, impious, intemperate, and ignorant, and yet exceptionally courageous.}
\end{quote}
So, if Protagoras thinks that piety, wisdom, justice, and temperance are inseparable, and he is forced to accept that wisdom and courage are inseparable, then it would seem that having any of these virtues would entail having the others. Anyway, the focus will be courage and wisdom. 

\section*{Knowledge is a virtue}
Recall that a virtue is what guarantees correct action; it is what guarantees that you get what is needed to make a life worth living. Socrates has a long argument that knowledge is a virtue, and that argument contains his reason that courage cannot be separated from knowledge/wisdom. 


\subsection*{Hedonism}

\noindent [A] Pleasant things are good insofar as they are pleasant (pleasure is \emph{a} good)
\vspace*{2mm}

\noindent [B] Pleasant things alone are good (pleasure is \emph{the} good)

\begin{itemize}
\item[B1]  One`s present pleasure is the good
\item[B2]  Maximized pleasure is the good
\begin{itemize}
\item S attributes this view to the many, 354a-c
\item An action is good just to the degree that it partakes in or produces pleasure; it is good or bad overall just to the degree that it produces maximized pleasure or pain--i.e. taking the subsequent effects of the action into account.
\end{itemize}
\end{itemize}

\vspace*{2mm}

\noindent S also assumes that all pleasures and pains are ``commensurable''---all pleasures and pains can be weighed against each other; thus, in principle, all pleasures and pains can be ranked

\section*{The experience of ``\emph{akrasia}''}
\begin{quote} Now, no one goes willingly toward the bad or what he believes to be bad; neither is it in human nature, so it seems, to want to go toward what one believes to be bad instead of to the good. And when he is forced to choose between one of two bad things, no one will choose the greater if he is able to choose the lesser. \end{quote}

Akrasia means something like ``being overcome,'' ``lack of self-control,'' ``weakness of will'', as reported by ``the many'' (\emph{hoi polloi}). One's knowledge (\emph{epist\^{e}m\^{e}}) that an action is bad can be overcome by desire, pleasure, pain, love, fear, etc. (352b, d, 353c). In other words, my knowledge that I should avoid eating ice-cream does not always prevent me eating it. My desire for the pleasure that ice-cream brings often overwhelms my knowledge that I shouldn't eat any more. 

\vspace*{2mm}

\noindent Two explicit versions:

\begin{itemize}\item{[1] X does B, i) knowing B is bad, ii) when able not to do B, and iii) overwhelmed by pleasure}
\item{[2] X does not do G, i) knowing G is good, ii) when able to do G, and iii) overwhelmed by pleasure}\end{itemize}

\section*{Socrates' aim}

\noindent The probandum (``thing to be proved''): Knowledge ``rules'' in a person---if X knows that A is good, X will do A, if X can; if X knows that A is bad, X will \emph{not} do X, if X can

\begin{itemize}\item{If S proves this, he will have shown that the akratic situation is impossible or misdescribed. That is, knowledge can never be overwhelmed by pleasure. }



\item{Given that S suspects virtue is some kind of knowledge, it is clear why he wants to argue for this}\end{itemize}

\section*{Socrates' argument}

\noindent [P1] Pleasure is the good
\vspace*{2mm}

\noindent [P2] All pleasures and pains are commensurable
\vspace*{2mm}

\noindent [P3] The akratic situation, when re-described in accordance with [P1], amounts to:

\begin{itemize}
\item{[1$^{*}$]: X did B, knowing B to be bad and able not to do it, because X was overcome by good =}
\item{X did something painful, knowing it to be painful and able not to do it, because X was overcome by pleasure}\end{itemize}
\vspace*{2mm}
 
 \noindent [P4] In 1$^{*}$, the good/pleasure is less than the overall bad/pain X chose to get
 \vspace*{2mm}
 
 \noindent [P5] So in both cases, X chose to do what X knew was worse/more painful instead of what was less bad or less painful
 \vspace*{2mm}
 
 \noindent [P6] It is impossible to choose the worse of two alternatives when you know (or believe?) it to be the worst
 
 \vspace*{2mm}
 \begin{itemize}
 \item  358d: S claims that it is not in human nature to go to the bad because everyone wants to be happy.
 \item There can only be desires for the good. And every action, to be an intentional action, must be preceded (and caused by the judgment that the action is good
 \item There is only one kind of desire--rational or ``maximizing' desire. What might appear `irrational desires' can be redescribed as (mistaken) beliefs about value 358d5. 
 \item Note the distinction between intentional action and compelled action
 \end{itemize}

 \noindent [C] Therefore, it is only ignorance of the relative weights of the alternatives that can explain X's selection of the worse
\vspace*{2mm}
 
 \noindent S re-describes the akratic situation as being a manifestation of ``ignorance'' (\emph{amathia})---the ``power of appearance'' leads a person to miscalculate the relative weight of an immediate pleasure (e.g. just as seeing something in a distance can lead us to judge (mistakenly) that it is smaller than something that is up close)
 \begin{itemize}
 \item If the hedonistic thesis is true, and if Socrates' argument is successful, virtue will be knowledge: knowledge of the relative weights of pleasures/pains will be sufficient to produce happiness.
 \end{itemize}
 \vspace*{2mm}
 
 \noindent S uses this conclusion to argue that even courage is a kind of knowledge/understanding, namely knowledge of ``what is and is not to be feared'' (360d8-9)
 
\section*{Problems}
 
 \noindent Is hedonism plausible in its own right?
 \vspace*{2mm}
 
 \noindent [P6] is only plausible if all desire is for the maximized good or that there is only one kind of desire/faculty of desire. Are there desires that are not aimed at the good?
 \vspace*{2mm}
  
 \noindent Does S illegitimately move between Hedonism B1 and B2 in his argument?
 \vspace*{2mm}
 
 \noindent Even if S's argument succeeds, is there an epistemic problem? If virtue is to guarantee correct action, and virtue is a kind of knowledge/understanding, will this require unrealistic demands of knowing \emph{all} the future consequences of our actions?



\end{document}
