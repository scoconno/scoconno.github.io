\documentclass[oneside]{article}
 \headheight = 25pt
\footskip = 20pt
\usepackage{mdwlist}
\usepackage[T1]{fontenc}
\renewcommand{\rmdefault}{ppl}
\usepackage{fancyhdr}
 \pagestyle{fancy}
 \lhead{\textbf{\textsc{\small Scott O'Connor\\Ancient Philosophy}}}
 \chead{}
 \rhead{\large\textbf{\textsc{Theaetetus 1}}}
 \lfoot{\footnotesize{\thepage}}
 \cfoot{}
 \rfoot{}
 \usepackage{longtable,booktabs}
\tolerance=700


\begin{document}





\section*{Overview}

Theaetetus was about sixteen when he met with Socrates, but their conversation is recounted by people who meet him before his death. This sets the dialog in 339BC. We are also told it happened just before Socrates' death, so the original dialog occurred sometime before 399BC. Socrates asks Theaetetus, 'What is knowledge?' His interest, in part, in asking this question is explained by who he is talking to: Theodorus is a great mathematician. Theaetetus is studying geometry, astronomy, music, and arithmetic under Theodorus. To learn a subject is to become wiser (or perhaps more expert) in that subject, for example, to learn geometry involves becomes wiser about geometry (145d). Wisdom is the same as knowledge, assumes Socrates (145e). Thus, learning involves acquiring knowledge about what one does not know. While left unsaid, it seems that Socrates is concerned with whether Theaetetus is acquiring knowledge about geometry, astronomy, etc. And, while he talks primarily with Theaetetus, the student, his likely target is Theodorus, the teacher. Is Theodorus really passing on any knowledge to his students? This is a question that Socrates regularly worries about. Do students really acquire any knowledge from their teachers? Do teachers who claim to possess knowledge about some subject really possess that knowledge? 


Socrates and his interlocutors are interested in the nature of knowledge. What is it? The dialog proceeds by Theaetetus proposing different answers and Socrates refuting him. They agree that whatever knowledge is, it must have these following two marks: 
\begin{enumerate}
\item Knowledge must always be of what is
\item knowledge must always be unerring (152c).
\end{enumerate}

They consider several suggestions. The one we are interested in is the claim that \textbf{knowledge is perception} (KP)(151e). If knowledge is perception, then perception must always be of what is, and perception must always be unerring. What ensues is an argument that perception does satisfy these requirements and then Socrates' rebuttal. 

\section*{Knowledge is Perception}
If this account is successful it must be unique and general, i.e., it must say the following
\begin{itemize}
\item x is a perceiving if and only if x is a knowing.
\end{itemize}
This claim says that every act of knowing is an act of perceiving and every act of perceiving is an act of knowing. If correct, there cannot be an act of knowing that is not an act of perceiving, e.g., you can never know X if you did not perceive X. And there cannot be a case of perceiving that cannot be a case of knowing, e.g., you can never perceive X without knowing X. 

One difficulty understanding this position is an ambiguity over `perception'. Perception can be taken as restricted to the senses. So, on one version of Theaetetus's claim, the only things that we can know are the things that we see, hear, touch, etc. And, more controversially, if we see, hear, touch, etc., something as F, then we know it is F. But we sometimes use `perception' more broadly to mean something like `seems' or `appears'. Is it restricted in this way? 

\section*{Measure Doctrine}
Theatetus's attempt is not yet a fully 'born' idea that can be tested. So, just like a midwife who helps induce labor, Socrates helps Theaetetus develop his idea by suggesting that it involves two further ideas, that man is the measure of all things and that all is flux. Our goal is to first understand what these doctrines are and to try identify how they supposedly relate to the proposal that knowledge is perception. We will look at Socrates's objections to the proposal in our next class. 

Socrates first says that the KP is closely associated with a view defended by his predecessor, Protagoras. 
\begin{description}
\item[Protagoras' Measure Doctrine:](“man is the measure of all things: of the things which are, that they are, and of things which are not, that they are not.” (MM)(152a) 
\end{description}
(Note that 'man' translates the Greek word 'anthropos'. This word is neutral with respect to gender. Newer translations use 'human', but that raises issues for how to translate single neuter personal pronouns from Greek into English, e.g., a human is a measure by him/her/them? being such and such. So, just remember that the doctrines presented in Greek are gender neutral.) 

\begin{description}
\item[ Infallibilism:] ``The wind is cold'' and ``The wind is not cold'' are both absolutely true. The wind, in effect, becomes cold when I believe it to be cold and it becomes not cold when you believe it to be warm. 
\end{description}

There is a position, or groups of position, in philosophy that has continually fascinated philosophers: reality can be one way for me and a different way for you. The idea is that reality, or some portion of reality, exists, or at least is how it is, because of how it appears to me and not in and of itself. So, for instance, moral relativists say that murder is wrong not in and of itself but only because it appears to me to be wrong. If it appears ok to someone, then it is ok for that person. A moral relativist says that those who think stealing to save your children is immoral are not in disagreement with who those believe it is moral; both are right because what it is for this to be moral or not is just dependent on what it appears to one. 

The view seems plausible for certain kinds of features. Consider the property of being tasty. Is brocolli sprouts tasty? Some people enjoy them. Others do not. Whose right? Presumably, both of them are right because taste is not a property that brucolli sprouts have independently of them being tasted. The view we are interested in says that the entire world is somehow like this, that the world is exactly as each of us think it is even though we have different experiences of it. 


\section*{Secret Doctrine}

The Secret Doctrine is introduced to provide support for Protagoras's claim that something is the case for one if and only if it appears so to one.
\begin{enumerate}
\item Qualities have no independent existence in time and space (153d6-e1).
\item Qualities do not exist except in perceptions of them (153e3–154a8). They do not exist in the sense organ or in the object perceived. 
\item (The dice paradox:) changes in a thing's qualities are not so much changes in that thing as in perceptions of that thing (154a9–155c6).
\item  Flux theory is used to develop a Protagorean/Heracleitean account of perception, to replace accounts based on the object/property ontology of common sense (155c-157c.) 
\end{enumerate}
The flux theory says there are two parents, the sense organ and an object that acts upon it. They have various powers to act upon and be acted upon. By virtue of these, when they do interact there are two offspring, a perceptible property and the perceiving

The offspring undergo fast change moving between both parents and do not remain in one place. The perceiving and perceptible properties cannot exist without each other and without the parents that created them. They are generated together in each encounter between perceiver and perceived object, thus one cannot separate them. The quality is essentially tied to one's perception of it. The parents undergo slow change. The eye becomes a seeing eye. So, a perceptual property belongs to a thing only in relation to a perception, and a perception can be attributed to a perceiver only in relation to the perceived object. This is meant to be like the dice example. 




\end{document}







