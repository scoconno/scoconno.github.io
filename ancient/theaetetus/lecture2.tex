\documentclass[oneside]{article}
 \headheight = 25pt
\footskip = 20pt
\usepackage{mdwlist}
\usepackage[T1]{fontenc}
\renewcommand{\rmdefault}{ppl}
\usepackage{fancyhdr}
 \pagestyle{fancy}
 \lhead{\textbf{\textsc{\small Scott O'Connor\\Ancient Philosophy}}}
 \chead{}
 \rhead{\large\textbf{\textsc{Theaetetus 2}}}
 \lfoot{\footnotesize{\thepage}}
 \cfoot{}
 \rfoot{}
 \usepackage{longtable,booktabs}
\tolerance=700


\begin{document}
\section*{Review}
\begin{description}
\item[Protagoras' Measure Doctrine:] ``man is the measure of all things: of the things which are, that they are, and of things which are not, that they are not.'' (MM)(152a) 
\end{description}
Protagoras believes the following claims ('p' stands for a proposition): 
\begin{enumerate}
\item If it seems to S that p, then it true for S that p.
\item If is true for S that p, then it seems to S that p.
\end{enumerate}
The conjunction of 1 and 2 is the official Protagorean doctrine: 
\begin{description}
\item[Protagoreanism:] It seems to S that p  if and only if it is true for S that p 
\end{description}
The scope of this doctrine is unclear, i.e., it is unclear what range of values can be substituted for S and p.
\begin{description}
\item[Narrow Protagoreanism:] a)  S is any individual organism with at least one of the five senses, and b) p is restricted to any proposition about what is perceived with the five senses, e.g., 'the cat is wet', 'the chair is blue', 'the drum beat is quick', etc. 
\item[Broad Protagoreanism:] a) S is any entity whatsoever that can hold a believe, b) p is any proposition whatsoever.
\end{description} 

\begin{description}
\item[Infallibilism:] ``The wind is cold'' and ``The wind is not cold'' are both absolutely true. The wind, in effect, becomes cold when I believe it to be cold and it becomes not cold when you believe it to be warm. 
\end{description}


\section*{Self-Refutation}


Socrates argues that Protagoras refutes himself, where the target of that refutation is Broad Protagoreanism. What does self-refutation mean? In this case, Socrates wants to show that the very formulation of Broad Protagoreanism is, or at least entails by itself, a contradiction. Consider these following examples of contradictory claims: 

\begin{enumerate}
\item  'Every sentence is false.' 
\item 'Every sentence I utter is a lie.'
\item. 'The errors that are found herein are mine alone.'
\item 'I do not exist'.
\item 'Only what can be perceived by the five senses exists.'
\item 'All truth is subjective'.
\item 'A claim about morality is true only if a community agrees with that claim.'
\end{enumerate}

\section*{Socrates's Argument}

An important implication of Protagoreanism is the following claim:
\begin{description}
\item[P1.] If it does not seem to S that p, then it is not true for S that p. \end{description}

But, suppose Lilly doesn't agree with Protagoreanism:

\begin{description}
\item[P2.] Lilly does not believe P1. \end{description}
Thus:
\begin{description}
\item[P3.] P1 is not true for Lilly (from P1 and P2).
\end{description}
Socrates thinks that Protagoras is then forced to accept: 
\begin{description}
\item[P4.] P1 is not true (from P3).
\end{description}
And, so, accept: 
\begin{description}
\item[C.] P1 is and is not true. 
\end{description}


\section*{Assessment}

The refutation of BP does not undermine Narrow Protagoreanism (NP). Socrates assumes that Protagoras might respond by restricting NP to things that seem to individuals warm, dry, sweet, etc., and seem to cities admirable and shameful, just and unjust, pious and impious, and religious and not religious. But he suggests that Protagoras would concede that NP does not apply to what seems to an individual healthy or sickening, and to a city what is or is not in its best interest. 








\end{document}

The dialog takes a sudden and mysterious turn. Socrates says at 172c that these concessions are made even by non-Protagoreans, that is, the concessions that there are expertise in respect of health for the body and soul, and expertise in respect of what is in the best interest of the state. But Socrates does not continue the discussion. Instead, he turns to compare the lawyer/orator with the philosopher. He concludes that discussion at 177c by claiming that the comparison was a digression. Interpreters are at a loss as to why Plato included this discussion. Some claim that it is irrelevant and could be removed without any losses. Others have claimed that it does play an important role in the discussion, but they disagree what role it plays. Our goal is to figure out ways that it might be relevant. They key likely lies in Socrates claim that for Protagoras "whatever view a city takes on these matters and establishes as its law or convention, is truth and fact for the city." (172a) The lawyer and orator are versed in the laws and conventions of a city. What else might we say? 



