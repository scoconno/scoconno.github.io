\documentclass{tufte-handout}
\usepackage{amsmath}
% Set up the images/graphics package
\usepackage{graphicx}
\setkeys{Gin}{width=\linewidth,totalheight=\textheight,keepaspectratio}
\graphicspath{{graphics/}}
\title{\Large{Scott O`Connor  \hfill Inquiry and Recollection in Plato's `Meno'}}
\usepackage[polutonikogreek,english]{babel}
\usepackage[utf8x]{inputenx}
\usepackage[T1]{fontenc}
\newcommand{\greek}[1]{{\selectlanguage{polutonikogreek}#1}}
\usepackage{mdwlist}


\begin{document}
%\maketitle
\thispagestyle{fancy}

\Large \newthought{Introduction:} Is virtue teachable? What's at stake?\footnote{\small{Can you tell me, Socrates, can virtue be taught? Or is it not teachable but the result of practice, or is it neither of these, but men possess it by nature or in some other way? (70a1-4)}}
%\begin{enumerate*}
%\item Virtue is a set of qualities whose possession is required to govern a city well. 
%\item Meno is asking whether this skill can be taught. It's about nature and nurture. Recall here the Protagoras. There are people who think that they can teach you political wisdom. It's an innocuous question but if it cannot be taught then this has serious reprecussions for moral education and sophistry. 
%\item Meno is from Thessaly (sp?). He may not be virtuous. He wants to become a tyrrant, run things. 
%\item Larger epistemological concerns. This dialogue is important!
%\end{enumerate*}
 
\newthought{Socrates' Challenge:} He does not know whether it can be taught because he does not know what it is.\footnote{\small{I myself, Meno, am as poor as my fellow citizens in this matter, and I blame myself for my complete ignorance about virtue. If I do not know what something is, how could I know what qualities it possesses? Or do you think that someone who does not know at all who Meno is could know whether he is good-looking or rich or well-born, or the opposite of these? (71b1-8)}}
\begin{enumerate*}
\item In order for S to know what qualities x possesses, S must know what x is. 
\item In order for S to know whether Meno is good-looking, rich, or well-born, S must know who Meno is.
\begin{enumerate*}
\item If S does not know at all who Meno is, then S cannot know what qualities belong to Meno. 
\end{enumerate*} 
\item In order for S to know whether virtue is teachable, S must know what virtue is.  
\end{enumerate*}
\begin{fullwidth}
\newthought{The Challenge assumes:}
\begin{enumerate*}
\item A distinction between essential and non-essential features, e.g. humans are essentially rational, but they are not essentially capable of laughter.
\item It may also assume that essential features are explanatory of non-essential features: 
\begin{enumerate*}
\item The essence of virtue explains what non-essential features virtue possesses. 
\item If virtue is teachable, then the essence of virtue explains why virtue is teachable. 
\end{enumerate*} 
%\item The Meno case is hard. In a room, can't I know that the person has red hair without knowing who they are. Introduce me to that person with red hair. Is Socrates denying this? Suppose the person is John? You don't know 'John has red hair'. You know 'a person has red hair'. 
\end{enumerate*}
\end{fullwidth}
%\newthought{Knowing what x is}. In order to know what x is, you need to know the definition of x. Socrates illustrates the requirements for definitions in two ways. He first has Meno try define virtue and illustrates them by failure. He then gives a few successful definitions of colour. This tells Meno how he should definine virtue. Features of definitions: 
%\begin{enumerate*}
%\item Unitary - one over many. 
%\item Explanatory - explains why they are virtuous. 
%\end{enumerate*}

\newthought{Meno's Challenge:}\footnote[2][+1cm]{\small{Meno: How will you look for it [e.g. the esssence of virtue], Socrates, when you do not know at all what it is? How will you aim to search for something you do not know at all? If you should meet with it, how will you know that this is the thing that you did not know?

Socrates: Do you realize what a debater's argument you are bringing up,  that a man cannot search either for what he knows or for what he does not know? He cannot search for what he knows---since he knows it, there is no need to search---nor for what he does not know, for he does not know what to look for (80d5--80e5).}}
\begin{enumerate*}
\item[P1] If you know x already, you cannot genuinely inquire into x. 
\begin{enumerate*}
\item[] (Read `inquire into x' as `inquire into the essence of x'.) 
\end{enumerate*}
\item[P2] If you do not know x, you cannot inquire into x because you do not even know what you are inquiring into. 
\item[P3] Either you know x or you do not. (Implicit Premise) 
\item[C] Therefore you cannot inquire into x.
\end{enumerate*}

\newthought{Argument for P2:}
\begin{enumerate*}
\item If you do not know \emph{at all} what x is, then you cannot start to inquire into x. 
\item If you do not know what x is, and if you stumble upon x, you will not know that what you stumbled upon is x. 
\end{enumerate*}
\begin{fullwidth}
\newthought{Impasse:} Socrates claims that we cannot find out whether virtue is teachable until we know what virtue is, i.e. know the essence of virtue.  But Meno argues that we cannot inquire into the essence of virtue if we know nothing at all about it. This is a serious impasses---it seems impossible to find out whether virtue is teachable. 
\end{fullwidth}
\newthought{The Theory of Recollection} says that what we call learning is recollecting things we already know.\footnote{\small{They say that the human soul is immortal; at times it comes to an end, which they call dying, at times it is reborn, but it is never destroyed...As the soul is immortal, has been born often and has seen all things here and in the underworld, there is nothing which it has not learned; so it is in no way surprising that it can recollect the things it knew before, both about virtue and other things (81b3--c9.}} 


\newthought{The Slave Boy Example} is meant to support the Theory of Recollection. The Slave Boy must find out how long the sides are of a 8 sq ft square. By being questioned by Socrates, the Slave Boy \emph{recollects} the correct answer.\footnote{\small{Socrates: And he will know it without having been taught but only questioned, and find the knowledge within himself?-Yes.

Socrates: And is not finding knowledge within oneself recollection?-Certainly.

Socrates: Must he not either have at some time acquired the knowledge he now possesses, or else have always possessed it? (85d2-10}}

\begin{enumerate*}
\item[A] S recovers the knowledge from within himself.
\item[B] Recovering knowledge for oneself that is in oneself is recollection. (Assumed)
\item[C] If S recollects what he knew, he either acquired that knowledge in the past or else always possessed it. 
\end{enumerate*} 

\newthought{Socrates' cross-examination} of the slave boy is meant to prove A. This assumes:
\begin{enumerate*}
\item Since the Slave gives the correct answers to certain questions, he already knew those answers. 
\item The Slave cannot be learning these answers by Socrates questioning him. 
\end{enumerate*}

\begin{fullwidth}
\newthought{Socrates} treats C briefly. Upon being asked, Meno tells Socrates that the slave was never educated in geometry. So Socrates concludes that the Slave boy must have possessed this knowledge when he was not a human being. Socrates then infers that he must have possessed it for all time. 

\newthought{How does the Theory of Recollection meet Meno's Challenge?} Some think that Socrates distinguishes between \emph{latent} and \emph{explicit} knowledge. On this reading, Socrates responds to the Challenge with both a concession and an objection. He thinks:
\begin{enumerate*}
\item We cannot inquire into what we explicitly know, but \emph{we can} inquire into what we only latently know. 
\item We cannot inquire into what we do not even latently know, but \emph{we can} inquire into what we do not explicitly know.
\end{enumerate*}
\end{fullwidth}





\end{document}
