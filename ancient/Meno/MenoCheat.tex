% !TEX encoding = UTF-8 Unicode
% !TEX TS-program = xelatex

\documentclass[11pt]{article}
\usepackage{fontspec}
\defaultfontfeatures{Mapping=tex-text}
\usepackage{xunicode}
\usepackage{xltxtra}
\usepackage{verbatim}
\usepackage[margin= 1 in]{geometry} % see geometry.pdf on how to lay out the page. There's lots.
\geometry{letterpaper} % or letter or a5paper or ... etc
%\usepackage[parfill]{parskip}    % Activate to begin paragraphs with an empty line rather than an indent 
\usepackage{mathrsfs}
\usepackage{bbding}
\usepackage[usenames,dvipsnames]{color}
\usepackage{natbib}
\usepackage{stmaryrd}
%\usepackage{mathpartir}
\usepackage{txfonts}
\usepackage{graphicx}
\usepackage{fullpage}
\usepackage{hyperref}
\usepackage{amssymb}
\usepackage{epstopdf}
\usepackage{fontspec}
%\setmainfont{Hoefler Text}
\setmainfont[BoldFont={Minion Pro Bold}]{Minion Pro}
\usepackage{hyperref}
\usepackage{lastpage, fancyhdr}
%\usepackage{setspace}
\pagestyle{fancy}
\lhead{}
\chead{Lecture 7, Plato's \emph{Meno}\space---\space Handout} 
\rhead{}
\lfoot{}
\cfoot{\thepage\space of \pageref{LastPage}} 
\rfoot{}
\footskip=30 pt
\headsep=20pt
\thispagestyle{empty}
\hypersetup{colorlinks=true, linkcolor=Sepia, urlcolor=Sepia, citecolor=BrickRed}
\DeclareGraphicsRule{.tif}{png}{.png}{`convert #1 `dirname #1`/`basename #1 .tif`.png}
\usepackage{polyglossia}
\setdefaultlanguage{english}
\setotherlanguage{greek}
\newfontfamily\greekfont{Gentium Plus}
\newcommand{\gk}[1]{\textgreek{#1}}
\newcommand{\gloss}[1]{(\textgreek{#1})}

\usepackage{covington}
\usepackage{fixltx2e}
\usepackage{graphicx}
\begin{document}

%\maketitle
\thispagestyle{empty}
\begin{center} \LARGE{PHIL 321\\ Lecture 7: Plato's \emph{Meno} (up to 86a)}\\ \vspace*{2mm}
\large{9/19/2013}\end{center}
\thispagestyle{empty}\vspace*{3mm}

\vspace*{-8mm}
\section*{Clearing up a mistake from last lecture}

\noindent Hedonism: Any view that holds pleasure to be good (i.e. of positive value) (versions differ in holding pleasure to be \emph{a} good, \emph{the} good, and so on) (so, hedonism is a theory of what is of value (an ``axiology''))
\vspace*{2mm} 

\noindent Consequentialism: Any view that accepts the following bi-conditional: \emph{X} is right/obligatory if and only if and because X maximizes value (So, consequentialism is a ``theory of rightness'' or ``theory of obligation'')
\vspace*{2mm}

\noindent Utilitarianism: A consequentialist view that holds ``utility'' to be what is of value (i.e. X is right/obligatory if and only if and because X maximizes utility)
\vspace*{2mm}

\noindent So, hedonic consequentialism is any view that accepts: X is right/obligatory if and only if and because X maximizes pleasure

\begin{itemize}\item{Different hedonic consequentialist theories will differ over what/whose ``pleasure'' is to be maximized (e.g. one's own pleasure; the total amount of pleasure in the world, etc.)}\end{itemize}\section*{The priority of definition}

\noindent If you don't know what something is, you can't know what qualities it has---i.e. you can't know anything else about it
\vspace*{2mm}

\noindent In other words, knowing whether \emph{X} is \emph{F} is epistemically \emph{posterior} to knowing what \emph{X} is

\begin{itemize}\item{So, knowing what virtue is is a necessary condition for knowing whether virtue is teachable. Thus, Socrates and Meno turn their attention to determining what virtue is.}\end{itemize}

\noindent What conception of knowledge might make this plausible?
\vspace*{2mm}

\noindent \textbf{Talk about the example of knowing who Meno is 71b; Ask what it would take to have a belief \emph{about} some object}
\vspace*{2mm}

\noindent \textbf{So far we have looked at dialogues dealing with the questions ``What is piety?,'' ``What is courage?,'' ``What is virtue?''. This is another instance of the latter, but note that this discussion is dialectically subordinate to the question ``Is virtue teachable?''}
\vspace*{2mm}

\noindent \textbf{Socrates then solicits various candidate definitions from Meno and, in his usual fashion rejects them all}

\begin{itemize}\item{\textbf{[1] virtue of a man is X; virtue of a woman is Y; virtue of a child is Z, indeed ``There is virtue for every action and every age, for every task of ours and everyone of us'' (72b)}}\begin{itemize}\item{\textbf{Soc does not deny that X is the virtue of men, Y is the virtue of woman and so on (although he certainly doesn't accept that); rather, he rejects this because it does not state the single Form in virtue of which all virtues are virtue; TIE UP WITH THE ? WHAT KIND OF KNOWLEDGE ARE WE TALKING ABOUT THAT REQUIRES KNOWING WHAT X IS; IF IT'S JUST KNOWING THAT X IS F, E.G. THAT X IS A VIRTUE, THAT SEEMS STRONG; IF IT'S UNDERSTANDING WHY X IS THE VIRTUE OF MAN, THAT SEEMS BETTER}}\end{itemize}\item{\textbf{[2] To be able to rule over people}}\begin{itemize}\item{\textbf{Socrates rejects this because you need to add ``justly'' but there are other virtues than justice, so this again fails}}\end{itemize}\end{itemize}

\noindent \textbf{Interlude on example definition: here we get the point that we must not only state a true answer but use terms that our discussant understands 75cd}

\begin{itemize}\item{\textbf{[3] To desire beautiful things and have the power to attain them (77B ff)}}\begin{itemize}\item{\textbf{Soc rejects these because he thinks that everyone desires only beautiful things, and so that part of the definition doesn't separate off anyone, and we're just back to having power, which he's already rejected}}\end{itemize}\end{itemize}


\section*{The paradox of inquiry}

\noindent Meno's version: He asks, first, ``How will you look for it [the nature of virtue] when you do not know at all what it is?'' And then presents something like a dilemma:
\vspace*{2mm}

\noindent Grube translation of 80d5-7: ``How will you aim to search for something you do not know at all? If you should meet with it, how will you know that this is the thing that you did not know?''
\vspace*{2mm}

\noindent Alternative translation: ``Which among the things you do not know [to be virtue] will you put before you to investigate? Or, if you happened to encounter it [the nature of virtue], how will you know that this is it, that thing which you did not know?''
\vspace*{2mm}

\noindent Socrates' version
\begin{itemize}\item{[1] Successful inquiry is impossible}\item{[2] If X knows, there is no point in searching (since X knows it already)}\item{[3] If X doesn't know, X cannot search (since X doesn't know what to search for)}\begin{itemize}\item{What suppressed premise is necessary to make [2], [3]$\rightarrow$ [1] a valid argument?}\end{itemize}\end{itemize}

\noindent What is the relationship between Meno's version and Socrates' version?

\section*{Socrates' response: Recollection and the Exchange with Meno's slave}

\noindent Plato clearly takes the paradox seriously, but which premise of the dilemma does he reject?
\vspace*{2mm}

\begin{itemize}\item{[R] Learning is recollection---which seems to reject [2]: you \emph{already} knew and \emph{later} recollect}\item{[E] The elenchus of the slave---which seems to reject [3]: the slave didn't know and now does (or will after further questioning)}\item{Perhaps he rejects [2] and [3]: There is a sense in which you can know (i.e. latently) but still have successful inquiry (by recollecting) (i.e. 2 is false) and there is a sense in which you don't know (i.e. explicitly) and can still have successful inquiry (i.e. in an exchange such as the one with Meno's slave)}\end{itemize}

\noindent S claims that the slave:
\begin{itemize}\item{[A] Has not learned geometry before}\item{[B] Was not taught it by him in this episode--so the slave didn't \emph{learn} it then either}\item{[C] And yet \emph{will} know it through further questioning}\end{itemize}

\noindent Re [B]: S claims that his questions merely elicit the slave's opinions. S did use the elenctic method---he asks misleading questions sometimes; he gets the slave to claim knowledge and then to confess \emph{aporia}; thereafter the slave gets it right by applying his own reasoning, not by bowing to S's authority.
\vspace*{2mm}

\noindent Perhaps, then, [R] supports [E]. We need some explanation of why the slave, through his discussion with S, will reliably \emph{reject} false beliefs and \emph{accept} true beliefs. Perhaps S thinks this is possible because the slave (like everyone) once knew (in a disembodied state) but forgot, and hence can recognize the truth, and so reject the false.
\vspace*{2mm}

\noindent Additional question: Even if we grant [A]-[C] do we have to think that the slave has recollected knowledge? What alternative explanations for the success of the slave-boy's discussion might we consider?

\section*{The scope of recollection}

\noindent What kinds of truths does S think we can recollect?

\begin{itemize}\item{He clealy allows geometrical truths (and so mathematical truths generally?). He should allow ethical truths (including truths about value) (otherwise the unity of the dialogue would be in jeopardy). He also suggests nature ``as a whole''?---What could that mean?}\item{What about empirical truths?}\end{itemize}
\end{document}
