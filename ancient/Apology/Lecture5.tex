\documentclass[oneside]{article}
 \headheight = 25pt
\footskip = 20pt
\usepackage{mdwlist}
\usepackage[T1]{fontenc}
\renewcommand{\rmdefault}{ppl}
\usepackage{fancyhdr}
 \pagestyle{fancy}
 \lhead{\textbf{\textsc{\small Scott O'Connor\\Ancient Philosophy}}}
 \chead{}
 \rhead{\large\textbf{\textsc{Apology}}}
 \lfoot{\footnotesize{\thepage}}
 \cfoot{}
 \rfoot{\footnotesize{\today}}
 \usepackage{longtable,booktabs}
\tolerance=700


\begin{document}
\thispagestyle{fancy}

\section*{Introduction}

Socrates faces a number of charges:

\begin{enumerate}
\item He studies things in the heavens and below the earth...implicit charge.
\item He makes the worse argument into the stronger (better) argument....implicit charge.
\item He is guilty of corrupting the young...explicit charge.
\item He does not believe in the gods of the city...explicit charge.
\end{enumerate}
The dialog is structured as follows: 

\begin{itemize}
\item 17a--24b: S's defense against the implicit accusations. 
\item 24b--35d: S's defense against the explicit charges; found guilty

\item  35e--38b: S proposes as ``punishment'' that he be given free meals in the Prytaneum; sentenced to death
\item 38c--42a: S's parting words.
\end{itemize}
Our interest is twofold: his  view of \textbf{wisdom} and his view of the \textbf{good life}. 

\subsection*{Defense Against Implicit Accusation}
In defending himself against the implicit  accusations, S makes some radical claims about knowledge and wisdom. He begins by  offering a diagnosis of the implicit accusations. Against 1, he points out that Aristophanes' popular play \emph{the Clouds} presents Socrates as studying natural science, but this is inaccurate. No real Athenian would testify to S really engaging in such an investigation. They need, he urges, to distinguish the fictional portrayals of him from his real views. Against 2, he claims that Athenians have misunderstood his motivation for examining supposed experts. 

\begin{itemize}
\item 20cd: S puts into the mouths of the jurors: where there's smoke there's fire; you wouldn't have gained this reputation if you weren't up to something
\item 20d: S claims that he acquired this reputation as a result of a certain kind of wisdom, human wisdom.
\item	21a: story of Chaerephon going to oracle.
\item	21c: description of his process: approach people to test if they are wise. He went to politicians, (21dc) poets (natural inspiration) and craftspeople.
\begin{itemize} \item  But: S grants the craftsmen know many things about their crafts (22d); how? 
\item While he allows that they do have knowledge (of their craft), he complains that this led them to think they had knowledge elsewhere, on the more important matters, which they did not, which rendered the knowledge they had, on the whole, undesirable.
\end{itemize}
\item  21d5: ``it seems that I'm wiser than he in just this one small way: that what I don't know, I don't think I know''
\begin{enumerate}
\item S does not know anything about ``physics''/natural philosophy (19c)
\item S does not know how to make people ``excellent'' (= virtuous) (20c)
\item S has (if anything) ``human wisdom'' = does not think he knows what he does not know (20--21d)
\end{enumerate}
\end{itemize}


\subsection*{What is human wisdom?}
S claims that he has human wisdom, but also that he lacks knowledge. This raises the difficult question about what S thinks human wisdom is. It's likely that S is drawing some distinction between knowledge and belief in his characterization of human wisdom. Here are some options:

\begin{description}
\item[Option 1:] S is wise iff S believes he is not wise.
\end{description}
Option 1 includes a claim about what S believes about himself, namely, he believes he is not wise. The idea is that human wisdom involves insight into our own minds. As an analogy, John might love Sue without being aware that he loves Sue. Alternatively, he might believe that he loves Sue, but, in fact, does not; he is mistaken about his emotions. Just as we can be mistaken about our emotions, we can be mistaken about our beliefs and knowledge. I believe that I know the proof to Pythagoras' Theorem, but I could be mistaken that I have this knowledge. So, option 1 is suggesting that you are wise only if you believe that you are not wise. 
Problems for this option:  (i) S believes that he is wise; he believes the oracle, and (ii) option 1 excludes the possibility that a person could have a true belief that they are not, in fact, wise. 
\begin{description}
\item[Option 2:] S is wise iff S believes S does not know anything.
\end{description}
But, (i) S doesn't claim that wisdom is incompatible with knowledge; presumably a person who only believes they know something when they do in fact know it would be a wise person. (ii) At 29b, S claims, ``To act unjustly, on the other hand, to disobey someone better than oneself, whether god or man, that I do know (\emph{oida}) to be bad and shameful.''  
\begin{description}
\item[Option 3:] S is wise iff for all P, S believes S knows P iff S knows P.
\end{description}
Option 3 claims that a wise person only believes they know what they really do know. If they don't know something, they don't believe they know it (which shows why the craftsperson is not wise). And if they do know something, then they believe they know it. Their beliefs about what they do and don't know are fully accurate. But does human wisdom concern all possible truths? It seems clear that ``the most important things''that S recognizes he does not know include moral or ethical truth. So, perhaps this is S's view of human wisdom: 
\begin{description}
\item[Option 4:] S is wise iff for any moral or ethical truth P, S believes S knows P iff S knows P.
\end{description}

%\begin{enumerate}\item[3a.] S has no (ethical) knowledge \emph{at all}; but, he does have (ethical) beliefs (if so, what does he take to be the status of those beliefs?)

%\item[3b.] S has no technical or expert (ethical) knowledge, but does have non-technical knowledge.
%\item[3c.] (modification of 3b?) S lacks systematic (ethical) understanding but does have some piece-meal knowledge.
%\end{enumerate}


%\subsection*{Defense against the Explicit M's Charges}
%Socrates cross-examined Meletus, his prosecutor. His intent is to undermine the explicit charges.  He use a method called the ``elenchus'' (= ``test,'' ``cross-examining'' ) to do so. The elenchus relies on a claim about epistemic closure: 

%\begin{description}
%\item[Epistemic closure:] If X believes P, and P entails Q, then X is committed to Q.
%\end{description}
%This principle says that we should accept whatever is logically entailed by our beliefs. If I believe that it is never healthy to eat ice-cream, then, rationally, I should also accept that it is not healthy to eat ice-cream on Wednesdays. Similarly, if I believe that the murderer knew the victim, and I believe that Moriarty does not know the victim, then I ought to believe that Moriarty is innocent. 

%S's interlocutors first accept some belief, P. S then quizzes his interlocutors, who concede that they hold some other beliefs, Q, R, etc. S then argues that Q and R entail the negation of P. The interlocutor, who believes P, is now forced to accept that they also believe not-P, which violates the following unstated assumption: 
%\begin{description}
%\item[Coherence:] If X believes P, and X believes not-P, then X does not know P (or not-P). 
%\end{description}
%S uses this method on M against his two explicit accusations; the goal is to show that M's accusations are contradictory. The first accusation: 

%\begin{itemize}
%\item{S gets M to assert (P), ``Socrates corrupts the youth willingly''.}
%\item{ S gets M to assert (Q), ``Harming/corrupting someone makes them a bad person.''}
%\item{S gets M to assert (R), ``Associating with bad people harms oneself.''}
%\item{S gets M to assert (T), ``No one harms oneself willingly''.}
%\item{S gets M to accept what Q,  R, and T entails, (U) ``Therefore, no one would harm one's associates willingly''.}
%\item{S gets M to accept what T entails, (not-P), ``Therefore, Socrates does not corrupt the youth willingly''.}
%\item{Since M believes both P and not-P, S gets (or tries to get) M to conclude that he does not know whether P}
%\end{itemize}

%\noindent \emph{Class project:} how does S use this method to disprove M's charge that he does not believe in the Gods of the city? 


%This method shows, at the least, that X has an inconsistent belief set. What kind of knowledge [1], [2], or [3] would this test for? But it is unclear how this method would allow A acquire knowledge.

%The problem is that S seems to think that elenchus is \emph{the} method of philosophical thought \emph{and} that philosophical thought is necessary for happiness, but a consistent set of beliefs can still be false, so what would be the upshot even if someone ``survived'' the elenchus? How, despite this, could S think the elenchus can be employed to make progress?

%\begin{itemize}
%\item{The condition of aporia results in the rejection of the false belief X had (e.g. P) BUT}
%\item{This requires S to know that the other beliefs used to conclude not-P are true AND}
%\item{S disclaims knowledge}
%\item{Resolutions?:}
%\begin{itemize}
%\item{\textbf{1) accept 3b or 3c above--if S knows (in some sense) Q, R, etc. S can reasonably reject P. This denies that S disclaims (all) knowledge}}
%\item{\textbf{accept 3a above, argue that S has justification for thinking that Q, R, etc. are true, but that this justification doesn't count as knowledge}}
%\item{\textbf{deny that the elenchus has positive results (beyond testing consistency), deny that it allows you to reject any belief}}
%\end{itemize}
%\end{itemize}

\subsection*{The Good Life}

How should we live? S and the ancient Greeks understand this question as asking about the priorities that we should pursue in this life. S insists that we should not organize our lives around the pursuit of wealth, power, and pleasure. Aristotle will call the lives devoted to such things the money-making life, the political life, and the hedonistic life. There are different ways such lives could be pursued, e.g., the drug addict and foodie are both living hedonistic lives. Such lives were considered candidates for \emph{eudaimonia}, which roughly translates as 'the life well lived', 'the happy life', 'the excellent life'. But S rejects these candidates for the good life: 

\begin{quote}
You are mistaken my friend, if you think that a man who is worth anything ought to spend his time weighing up the prospects of life and death. He has only one thing to consider in performing any action---that is, whether he is acting right or wrongly, like a good man or a bad one. (28b)
\end{quote}
S here criticizes those who fear death. By 'good person' he means something like 'an excellent person', a person who is living an excellent life: 
\begin{enumerate}
\item If a person is excellent, then they regard only one property of their actions as ultimately decisive (or important), whether they are right or wrong.
\item If a person is concerned with the risk of death, then it is not the case that they regard only one property of their action as ultimately important, whether they are right or wrong.
\item If a person is concerned with the risk of death, then the person is not an excellent person.
\end{enumerate}
So, S believes that the way to live an excellent life concerns doing what is right or wrong. This is why he refuses to give up the practice of philosophy as a condition for his acquittal; he thinks it would be wrong to disobey the gods' command to examine  fellow citizens. This is also why he refuses to leave Athens and live quietly in exile; it would be wrong to disobey the gods' injunction that he test those who claim to know. But what does S mean by claiming that the best life concerns right and wrong actions?

\begin{description}
\item[Care for his own soul:] Throughout his defense (20a-b, 24c-25c, 31b, 32d, 36c, 39d) S repeatedly stresses that a human being must care for their soul more than anything else. 
\item[Care for others' soul:] S argues that the god gave him to the city as a gift and that his mission is to help improve the city.  S characterizes himself as a gadfly and the city as a sluggish horse in need of stirring up (30e).  Just as the gadfly is an irritant to the horse but rouses it to action, so S supposes that his purpose is to persuade his fellow citizens that the most important good for a human being was the health of the soul.
\item[Soul harms:] S thinks that false beliefs harm our souls, and so caring for our souls requires that we avoid false beliefs and only hold true beliefs. 
\end{description}

\subsection*{Good souls}
A good soul is a virtuous soul; virtue is that, whatever it is, that makes a soul a good one. Vice, on the other hand, is that, whatever it is, that makes a soul a bad one. But what exactly are these virtues and vices? 

S thinks that a good soul is analogous to a healthy body. Benefiting a soul is analogous to promoting health in the body and harming the soul is analogous to harming the health of the body.  But this doesn't quite tell us what exactly a healthy soul, a virtuous soul, is meant to be. There are various options: 
\begin{description}
\item[Intellectualism:] what it is to have a healthy soul just is to possess only true beliefs; what it is to have a unhealthy soul just is to have false beliefs. 
\item[Instrumentalism:] true beliefs produce health in the soul and false beliefs undermine health in the soul. Here is one way that might happen:
\begin{itemize}
\item Appetites  for pleasure, aversions to pain, and emotions, such as fear, anger, or love arise from our beliefs about what is good for us, e.g., if I believe that ice-cream is good for me, then I will desire ice-cream. 
\item When we act for the sake of such appetites, aversions, etc., they are strengthened and we become increasingly habituated to believing that we should act accordingly--the  more I satisfy my desire for ice-cream, the more I believe that eating ice-cream is good for me.
%\item The more I am habituated to act according to my aversions, appetites, etc., the less I am able to evaluate the reasons for and against an action; I am become increasingly enslaved to my aversions, appetites, etc. 
\end{itemize}
\end{description}
On both interpretations, S believes that the state of one's soul (\emph{psuch\^{e}}) is of the utmost importance (29e, 30b). Is it the \emph{only} thing that matters? Or just the most important? The translation of 30b3-5 is controversial. Here are two options:
\begin{enumerate}
\item ``It's not from wealth that virtue comes, but from virtue comes money, and all the other things that are good for human beings, both in private and in public life.'' [Grube]
\item  ``It's not from wealth that virtue comes, but from virtue money and all the other things become good for human beings, both in private and in public life.'' [Alternative]
\end{enumerate}
Our first option explicitly says that there are things other than virtue which are good to have, both in public and private. These include money, but likely include things like beauty, pleasure, etc. Our first option says that if you are virtuous, then you will likely achieve all the other good things in this life. The second option says that money, pleasure, etc., are not good for humans in of themselves. If you are not virtuous, money would not be a good thing to have. Rather, it is virtue alone that matters. This second interpretation is more likely the correct one. S is claiming that without virtue, money, pleasure, and power would be harmful to us, by, for instance, posing a distorting effect on our ability to reason. But with virtue, we can discipline our desires for such things; virtue allows us desire such things appropriately without those desires enslaving us. 

\subsection*{The Examined Life}

After the conviction and sentencing, S tells the jury that he could never keep silent, because
\begin{quote}
it's the greatest good for a man to discuss virtue every day, and the other things you've heard me discussing and examining myself and others about, on the grounds that the unexamined life isn't worth living for a human being (38a).
\end{quote}
S claims that ``the unexamined life is not worth living for human beings'', and because of this he must discuss virtue and examine both himself and others. What does S mean by this famous claim? Why does he emphasize the life of human beings? 


% the call to live examined lives follows from our nature as human beings. We are naturally directed by pleasure and pain.  We are drawn to power, wealth and reputation, the sorts of values to which Athenians were drawn as well.  Socrates’ call to live examined lives is not necessarily an insistence to reject all such motivations and inclinations but rather an injunction to appraise their true worth for the human soul.  The purpose of the examined life is to reflect upon our everyday motivations and values and to subsequently inquire into what real worth, if any, they have.  If they have no value or indeed are even harmful, it is upon us to pursue those things that are truly valuable.





%\section*{Socrates on death}

%An obvious challenge to Socrates' emphasis on virtue is that we might face situations where being virtuous will risk our lives. But Socrates  argues that the fear of death should not lead people to act unjustly, impious, etc. He paints two possible pictures of what death is like, neither of which he thinks we should fear (40c--41c)

%\begin{enumerate}
%\item A dreamless sleep: \emph{class project}
%\item Existence in Hades with other deceased people: \emph{class project}
%\end{enumerate}

\end{document}



\section*{The Good Life}

S believes that the state of one's soul (\emph{psuch\^{e}}) is of the utmost importance (29e, 30b). Is it the \emph{only} thing that matters? Or just the most important? The translation of 30b3-5 is controversial. Here are two options:
\begin{enumerate}
\item ``It's not from wealth that virtue comes, but from virtue comes money, and all the other things that are good for human beings, both in private and in public life.'' [Grube]
\item  ``It's not from wealth that virtue comes, but from virtue money and all the other things become good for human beings, both in private and in public life.'' [Alternative]
\end{enumerate}
Our first option explicitly says that there are things other than virtue which are good to have, both in public and private. These include money, but likely include things like beauty, pleasure, etc. Our first option says that if you are virtuous, then you will likely achieve all the other good things in this life. The second option says that money, pleasure, etc., are not good for humans in of themselves. If you are not virtuous, money would not be a good thing to have. Rather, it is virtue alone that matters. This second interpretation is more likely the correct one.

An obvious challenge to Socrates' emphasis on virtue is that we might face situations where being virtuous will risk our lives. But Socrates  argues that the fear of death should not lead people to act unjustly, impious, etc. He paints two possible pictures of what death is like, neither of which he thinks we should fear (40c--41c)

\begin{enumerate}\item{A dreamless sleep}\item{Existence in Hades with other deceased people}\end{enumerate}

