\documentclass[oneside]{article}
 \headheight = 25pt
\footskip = 20pt
\usepackage{mdwlist}
\usepackage[T1]{fontenc}
\renewcommand{\rmdefault}{ppl}
\usepackage{fancyhdr}
 \pagestyle{fancy}
 \lhead{\textbf{\textsc{\small Scott O'Connor\\Ancient Philosophy}}}
 \chead{}
 \rhead{\large\textbf{\textsc{Apology 2}}}
 \lfoot{\footnotesize{\thepage}}
 \cfoot{}
 \rfoot{\footnotesize{\today}}
 \usepackage{longtable,booktabs}
\tolerance=700


\begin{document}
\thispagestyle{fancy}




\section*{The Good Life}

How should we live? Socrates and the ancient Greeks understand this question as asking about the priorities that we should pursue in life. In the \emph{Apology}, S insists that we should not organize our lives around the pursuit of wealth, power, and pleasure. Aristotle will call the lives devoted to such things the money-making life, the political life, and the hedonistic life. There are different ways such lives could be pursued, e.g., the drug addict and foodie are both living hedonistic lifes. Such lives were considered candidates for \emph{eudaimonia}, which roughly translates as 'the life well lived', 'the happy life', 'the excellent life'. But S rejects these candidates for the good life: 

\begin{quote}
You are mistaken my friend, if you think that a man who is worth anything ought to spend his time weighing up the prospects of life and death. He has only one thing to consider in performing any action---that is, whether he is acting right or wrongly, like a good man or a bad one. (28b)
\end{quote}
S here criticizes those who fear death. By 'good person' he means something like 'an excellent person', a person who is living an excellent life: 
\begin{enumerate}
\item If a person is excellent, then they regard only one property of their actions as ultimately decisive (or important), whether they are right or wrong.
\item If a person is concerned with the risk of death, then it is not the case that they regard only one property of their action as ultimately important, whether they are right or wrong.
\item If a person is concerned with the risk of death, then the person is not an excellent person.
\end{enumerate}
So, S believes that the way to live an excellent life concerns doing what is right or wrong. This is why he refuses to give up the practice of philosophy as a condition for his acquittal; he thinks it would be wrong to disobey the gods' command to examine  fellow citizens. This is also why he refuses to leave Athens and live quietly in exile; it would be wrong to disobey the gods' injunction that he test those who claim to know. But what does S mean by claiming that the best life concerns right and wrong actions?

\begin{description}
\item[Care for his own soul:] Throughout his defense (20a-b, 24c-25c, 31b, 32d, 36c, 39d) S repeatedly stresses that a human being must care for their soul more than anything else. 
\item[Care for others' soul:] S argues that the god gave him to the city as a gift and that his mission is to help improve the city.  S characterizes himself as a gadfly and the city as a sluggish horse in need of stirring up (30e).  Just as the gadfly is an irritant to the horse but rouses it to action, so S supposes that his purpose is to persuade his fellow citizens that the most important good for a human being was the health of the soul.
\item[Intellectualism:] S thinks that false beliefs harm our souls, and so caring for our souls requires that we avoid false beliefs and only hold true beliefs. 
\item[What is a good soul?] S thinks that a good soul is analogous to a healthy body, i.e., he believes in psychic health. Benefiting a soul is analogous to promoting health in the body and harming the soul is analogous to harming the health of the body. The health of the soul is a virtue and the opposite is a vice. But this doesn't quite tell us what exactly a healthy soul, a virtuous soul, is meant to be. There are various options here: 
\begin{description}
\item[Pure Intellectualism:] what it is to have a healthy soul is to possess only true beliefs; what it is to have a unhealthy soul is to have false beliefs. 
\item[Complex Intellectualism:] true beliefs produce health in the soul and false beliefs undermine health in the soul. Here is one way that might happen:
\begin{itemize}
\item Appetites  for pleasure, aversions to pain, and emotions, such as fear, anger, or love can shape our beliefs about what is good for us. 
\item When we act for the sake of such appetites, aversions, etc., they are strengthened and we become increasingly habituated to believing that we should act accordingly--the  more I satisfy my desire for ice-cream, the more I believe that eating ice-cream is good for me.
\item The more I am habituated to act according to my aversions, appetites, etc., the less I am able to evaluate the reasons for and against an action; I am become increasingly enslaved to my aversions, appetites, etc. 
\end{itemize}
\end{description}
\end{description} 
S believes that the state of one's soul (\emph{psuch\^{e}}) is of the utmost importance (29e, 30b). Is it the \emph{only} thing that matters? Or just the most important? The translation of 30b3-5 is controversial. Here are two options:
\begin{enumerate}
\item ``It's not from wealth that virtue comes, but from virtue comes money, and all the other things that are good for human beings, both in private and in public life.'' [Grube]
\item  ``It's not from wealth that virtue comes, but from virtue money and all the other things become good for human beings, both in private and in public life.'' [Alternative]
\end{enumerate}
Our first option explicitly says that there are things other than virtue which are good to have, both in public and private. These include money, but likely include things like beauty, pleasure, etc. Our first option says that if you are virtuous, then you will likely achieve all the other good things in this life. The second option says that money, pleasure, etc., are not good for humans in of themselves. If you are not virtuous, money would not be a good thing to have. Rather, it is virtue alone that matters. This second interpretation is more likely the correct one. Socrates is, then, claiming that without virtue, money, pleasure, and powerful would be harmful to us, by, for instance, posing a distorting effect on our ability to reason. But with virtue, we can discipline our desires for such things; virtue allows us desire such things appropriately without those desires enslaving us. 

\section*{The Examined Life}

After the conviction and sentencing, S tells the jury that he could never keep silent, because
\begin{quote}
it’s the greatest good for a man to discuss virtue every day, and the other things you’ve heard me discussing and examining myself and others about, on the grounds that the unexamined life isn’t worth living for a human being (38a).
\end{quote}
S claims that ``the unexamined life is not worth living for human beings'', and because of this he must discuss virtue and examine both himself and others. What does S mean by this famous claim? Why does he emphasize the life of human beings in the claim. 


% the call to live examined lives follows from our nature as human beings. We are naturally directed by pleasure and pain.  We are drawn to power, wealth and reputation, the sorts of values to which Athenians were drawn as well.  Socrates’ call to live examined lives is not necessarily an insistence to reject all such motivations and inclinations but rather an injunction to appraise their true worth for the human soul.  The purpose of the examined life is to reflect upon our everyday motivations and values and to subsequently inquire into what real worth, if any, they have.  If they have no value or indeed are even harmful, it is upon us to pursue those things that are truly valuable.





\section*{Socrates on death}

An obvious challenge to Socrates' emphasis on virtue is that we might face situations where being virtuous will risk our lives. But Socrates  argues that the fear of death should not lead people to act unjustly, impious, etc. He paints two possible pictures of what death is like, neither of which he thinks we should fear (40c--41c)





\begin{enumerate}
\item A dreamless sleep: \emph{class project}
\item Existence in Hades with other deceased people: \emph{class project}
\end{enumerate}

\end{document}