% !TEX encoding = UTF-8 Unicode
% !TEX TS-program = xelatex

\documentclass[11pt]{article}
\usepackage{fontspec}
\defaultfontfeatures{Mapping=tex-text}
\usepackage{xunicode}
\usepackage{xltxtra}
\usepackage{verbatim}
\usepackage[margin= 1 in]{geometry} % see geometry.pdf on how to lay out the page. There's lots.
\geometry{letterpaper} % or letter or a5paper or ... etc
%\usepackage[parfill]{parskip}    % Activate to begin paragraphs with an empty line rather than an indent 
\usepackage{mathrsfs}
\usepackage{bbding}
\usepackage[usenames,dvipsnames]{color}
\usepackage{natbib}
\usepackage{stmaryrd}
%\usepackage{mathpartir}
\usepackage{txfonts}
\usepackage{graphicx}
\usepackage{fullpage}
\usepackage{hyperref}
\usepackage{amssymb}
\usepackage{epstopdf}
\usepackage{fontspec}
%\setmainfont{Hoefler Text}
\setmainfont[BoldFont={Minion Pro Bold}]{Minion Pro}
\usepackage{hyperref}
\usepackage{lastpage, fancyhdr}
%\usepackage{setspace}
\pagestyle{fancy}
\lhead{}
\chead{Scott O'Connor\\Ancient Philosophy} 
\rhead{}
\lfoot{}
\cfoot{\thepage\space of \pageref{LastPage}} 
\rfoot{}
\footskip=30 pt
\headsep=25pt
\thispagestyle{empty}
\hypersetup{colorlinks=true, linkcolor=Sepia, urlcolor=Sepia, citecolor=BrickRed}
\DeclareGraphicsRule{.tif}{png}{.png}{`convert #1 `dirname #1`/`basename #1 .tif`.png}


\usepackage{covington}
\usepackage{fixltx2e}
\usepackage{graphicx}
\begin{document}

%\maketitle
\thispagestyle{empty}
\begin{center} \LARGE{Ancient Philosophy\\ \emph{De Anima}}\\ \vspace*{2mm}
\large{Scott O'Connor}\end{center}
\thispagestyle{empty}\vspace*{3mm}
\vspace*{-8mm}


\section*{Introduction}
In the \emph{Phaedo}, we encountered a radical view of the soul and its relationship to the body. We saw that Plato believed that the soul is immaterial and capable of existing independently of that body in which it is imprisoned. Further, we saw that the Plato thinks that the body is an impediment to the soul's quest for knowledge; only being freed from the body might the soul finally gain knowledge of the forms. On the other extreme, some ancient philosophers believed that the soul was material, that it was composed of the very same stuff that composes our limbs, organs, etc. On such a view, the soul is obviously incapable of existing in an immaterial state divorced from the rest of the body. 

Aristotle attempts to thread a path between these two extremes. According to him, the soul and body are distinct, but he believes the soul is incapable of existing separated from the body. He explains and defends his theory of the soul in \emph{De Anima}. This is a Latin title for the work. No english translation is really suitable. The English `soul' has religious connotations. Many view it as the seat of consciousness which carriers the marks of the moral decisions made in a person's life. But Aristotle's view of the soul is radically different. The soul, for Aristotle, is that which makes a body the kind of living body that it is. The soul of a plant makes a particular body into a living body that can take in nutriment, grow and reproduce. The soul of an animal makes a a particular body into a living body that can move and perceive its environment. So, Aristotle's core notion of soul is deeply related to what makes a body alive, and, especially to what makes it the kind of living thing it is. 

Understanding Aristotle's view of soul, then, requires saying something about how he distinguishes different kinds of living things. Aristotle does this by identifying the kind of activities that living beings can engage in. Some are listed in \emph{DA} II.1, others in \emph{DA} II.2:
\begin{itemize}
\item Self-nourishment
\item Growth
\item Decay
\item Movement and rest (in respect of place)
\item Perception
\item Intellect
\end{itemize}
So anything that nourishes itself, that grows, decays, can move itself, perceives, or thinks is alive. Why can certain bodies engage in these activities while others cannot? Why can a plant grow but not move? This is Aristotle's core innovation. According to him, the soul's presence in a body explains why that body is able to engage in the kind of activities that it can, and, in doing so, it simultaneously explains both why a body is alive and why it is the kind of living thing that it is. Aristotle divides these kinds into three wide groups: 

\newpage

\begin{description}
\item[Plants:] They only have \textbf{nutritive souls}. Such souls explain why plans can engage in growth and nutrition. 
\item[Animals:] They have \textbf{sensitive souls}. Such souls explain why plants can engage in locomotion and perception. 
\item[Human being:] They have \textbf{rational souls}. Such souls explain why humans can thing. 
\end{description}
Aristotle thinks that there is a nested hierarchy between these souls. They are nested in the sense that anything that has a higher degree of soul also has all of the lower degrees. All living things grow, nourish themselves, and reproduce. Animals not only do that, but move and perceive. Humans do all of the above and reason, as well. (There are further subdivisions within the various levels, which we will ignore.) 

While the core notion is simple enough, Aristotle's sophisticated development is anything but simple. He develops an account of how soul explains the activities of a living body by deploying not just the notion of form and matter from the \emph{Physics}, but a new and difficulty distinction between what he calls actuality and potentiality. 

\begin{quote}
It is necessary, then, that the soul is a substance as the form of a natural body which has life in potentiality. But substance is actuality; hence, the soul will be an actuality of a certain sort of body.
(DA 412a 19–22)
\end{quote}

\noindent At the beginning of Ch.~1, Aristotle tells us that the soul is a particular kind of nature, namely the form of a living organism; thus, the relation of body--soul an instance of the more general form--matter relationship. But he adds something new. He says that soul is the actuality of a certain sort of body.\\

\noindent He will ultimately claim that the soul is a second potentiality / first actuality of a certain kind of body. The kind of body at question is ``organic'' (composed of organs), i.e. one's whose parts are capable of functioning in integrated ways. For example, the leaves, roots, stem etc., of plants are capable of functioning in various integrated ways.\\

%\section*{The general account of soul}

%\noindent In Bk. 2, Ch. 1 A offers a general definition that covers all souls. Such a definition, while it will be true of all souls, will not be particularly informative, and so he also proceeds, in Chs. 4 and following, to discuss the various kinds of soul in more detail. He will ultimately argue that there is a nutritive soul, locomotive soul, perceptive soul, and rational soul.  \\




%\noindent A thinks this renders the question ``are the soul and body one'' easily answerable: of course they are different, since form and matter are different; but, since the presence of soul makes the body the kind of body it is, you can't have that kind of body without it being ensouled

%\section*{Homonymy}

%It has now been said in general what the soul is: the soul is a substance corresponding to the account; and this is the essence of such and such a body. It is as if some tool were a natural body, e.g. an axe; in that case being an axe would be its sub- stance, and this would also be its soul. If this were separated, it would no longer be an axe, aside from homonymously. But as things are, it is an axe. For the soul is not the essence and structure of this sort of body, but rather of a certain sort of natural body, one having a source of motion and rest in itself. What has been said must also be considered when applied to parts. For if an eye were an animal, its soul would be sight, since this would be the substance of the eye corresponding to the account. The eye is the matter of sight; if sight is lost, it is no longer an eye, except homonymously, in the way that a stone eye or painted eye is. (DA 412b10–21)

%that is, that a dead body is not properly speaking a body at all

%\noindent A's general account of homonymy: A and B are ``homonyms'' iff the same name ``N'' applies to both A and B but the reason why it applies is different. For example, the eye of a living organism and the eye of a statue are ``homonyms'' because the same name applies to them (i.e. ``eye'') but the reason why that name applies is different for each. For the eye of a living organism, it applies because it has the power of sight; for the eye of the statue it applies because it ``looks like'' or ``resembles'' the eye of a living organism in shape and location.\\

%\noindent So, for A, a body that lacks the ability to perform the activities characteristic of an X (e.g. a dog, cat, human being) can only be the body of an X homonymously



\section*{Aristotle's potentiality / actuality distinction}

\noindent Consider the ingredients for a cake sitting on your kitchen counter. These ingredients, for Aristotle, are not actually a cake yet. They are potentially a cake. This notion between actual and potential is Aristotle's; these English words are translations of Greek words that Aristotle coined. Central to Aristotle's metaphysics is a distinction between potential beings and actual beings. There is a difference between being potentially a house and being actually a house; being potentially a dog and being actually a dog; being potentially seeing and being actually seeing.\\

\noindent What exactly is this difference? Why are the ingredients on my kitchen counter potentially a cake but not potentially a house? Aristotle does not think that facts about potential beings can be reduced to facts solely about actual beings. For example, he does not think that the fact that a collection of bricks constitutes a potential house can be reduced solely to facts about bricks.\\

\noindent Rather, Aristotle thinks that a being (or collection of beings) X is (are) a potential \emph{F} if and only if there is a single process such that, as a result of undergoing that process, X is actually \emph{F}. For example, the reason why those ingredients are potentially a cake is because there is a single process, namely an exercise of the art of baking, that it can undergo to turn them into an actual cake. Likewise, a collection of bricks constitute a potential house because there is a single process, namely an exercise of the art of housebuilding, that it can undergo such that it becomes an actual house.\\

\noindent Aristotle develops his account by distinguishing types of actuality and potentiality. To illustrate this distinction, consider whether you can speak Ancient Greek? Can you do this? This question admits of two different answers depending on what we mean by `can'. It could mean, `do you know how to speak Greek such that you could start speaking it whenever you choose?' Alternatively, it could mean, `do you have the appropriate physical and intellectual capacities to allow you speak Greek?' The answer to the latter is clearly `yes'. Dogs do not have the physique and minds for Greek, but humans do. But while the answer to the second question might be `yes', the answer to the first might be `no'. In Aristotle's words, such a person has the first potential for Greek but not the second potential. Once they acquire the knowledge of Greek, they will become actual Greek speakers. They will also have what is called the `second potential' for Greek, namely, they can now speak Greek whenever the choose. When they do, in fact, exercise their knowledge of Greek and speak it, they will be Greek speakers in a different way, namely, they will be second actuality Greek speakers. 



\noindent Thus, Aristotle's general account of soul: the first actuality of a natural body that is potentially alive. In other words, the soul just is the integrated set of abilities the possession of which makes a living organism the kind of organism it is. The nutritive soul is akin to the knowledge of Greek. Just as the knowledge of Greek, allows for the actual exercise of Greek speaking, so too the nutritive soul allows for the actual exercise of nutrition.\\

\noindent This means that Aristotle does \emph{not} distinguish between the various kinds of activities human beings engage in and attribute a subset of them to the soul and a subset to something else (such as the body). But, it does \emph{not} mean that Aristotle can't make distinctions between various activities human beings can perform when such a distinction is relevant (e.g. he can think that the purely automatic maintenance of the body isn't relevant for Ethics or practical philosophy).\\

\noindent The ability to perform the activities of nutrition can be had without the abilities to perceive and think (e.g. in plants); the ability to perceive can be had without the ability to think but \emph{not} without the ability to nourish; the ability to think requires all the others (at least for earthly organisms).\\

\noindent As noted above, Aristotle thinks that there can be a general definition of soul as such (given above), but that general account is not that informative. In part, this is because the possession of higher-order abilities ``colors'' or ``affects'' the lower-order abilities of the organism. For example, possessing the ability to think means that the human beings ability to nourish itself is quite different from, say, plants or animals ability to nourish itself.\\

\noindent This is part of the reason why Aristotle thinks that virtue (and ethics generally) is something that only human beings can possess; it isn't just that we possess the ability to think; its because our possessing that ability means we can perform the lower-order abilities in a manner not available to non-human animals.

\section*{The soul as \emph{aitia}}

\noindent In Ch. 4 Aristotle says that the soul is the \emph{aitia} in three ways of the organism whose soul it is: as source of motion, what it is for, and substance (i.e. form)
\begin{itemize}
\item Efficient \emph{aitia} of the locomotion, growth, and alteration of the organism

\item Formal \emph{aitia} of the organism (i.e. it makes the organism the kind of organism it is)

\item Final \emph{aitia} of the organism's ``body'' (i.e. its parts are for the sake of performing those abilities)
\end{itemize}

\end{document}
