% !TEX encoding = UTF-8 Unicode
% !TEX TS-program = xelatex

\documentclass[11pt]{article}
\usepackage{fontspec}
\defaultfontfeatures{Mapping=tex-text}
\usepackage{xunicode}
\usepackage{xltxtra}
\usepackage{verbatim}
\usepackage[margin= 1 in]{geometry} % see geometry.pdf on how to lay out the page. There's lots.
\geometry{letterpaper} % or letter or a5paper or ... etc
%\usepackage[parfill]{parskip}    % Activate to begin paragraphs with an empty line rather than an indent 
\usepackage{mathrsfs}
\usepackage{bbding}
\usepackage[usenames,dvipsnames]{color}
\usepackage{natbib}
\usepackage{stmaryrd}
%\usepackage{mathpartir}
\usepackage{txfonts}
\usepackage{graphicx}
\usepackage{fullpage}
\usepackage{hyperref}
\usepackage{amssymb}
\usepackage{epstopdf}
\usepackage{fontspec}
%\setmainfont{Hoefler Text}
\setmainfont[BoldFont={Minion Pro Bold}]{Minion Pro}
\usepackage{hyperref}
\usepackage{lastpage, fancyhdr}
%\usepackage{setspace}
\pagestyle{fancy}
\lhead{}
\chead{Lecture 27, The Modes of Skepticism\space---\space Handout} 
\rhead{}
\lfoot{}
\cfoot{\thepage\space of \pageref{LastPage}} 
\rfoot{}
\footskip=30 pt
\headsep=20pt
\thispagestyle{empty}
\hypersetup{colorlinks=true, linkcolor=Sepia, urlcolor=Sepia, citecolor=BrickRed}
\DeclareGraphicsRule{.tif}{png}{.png}{`convert #1 `dirname #1`/`basename #1 .tif`.png}
\usepackage{polyglossia}
\setdefaultlanguage{english}
\setotherlanguage{greek}
\newfontfamily\greekfont{Gentium Plus}
\newcommand{\gk}[1]{\textgreek{#1}}
\newcommand{\gloss}[1]{(\textgreek{#1})}

\usepackage{covington}
\usepackage{fixltx2e}
\usepackage{graphicx}
\begin{document}

%\maketitle
\thispagestyle{empty}
\begin{center} \LARGE{PHIL 321\\ Lecture 27: The Modes of Skepticism}\\ \vspace*{2mm}
\large{12/3/2013}\end{center}
\thispagestyle{empty}\vspace*{3mm}
\vspace*{-8mm}

\section*{Modes}

\noindent Generally speaking, ``modes'' were codifications of the skeptical ability to oppose any argument with an argument of equal strength
\vspace*{2mm}

\noindent There were several sets of modes, the most famous of which were the ``Ten modes'' (attributed by Sextus to ``the older skeptics'') and the ``Five modes'' (attributed by Sextus to ``the later skeptics'')

\section*{The Five Modes}

\noindent \textbf{[1]} The mode based on ``disagreement''
\vspace*{2mm}

[P1] Epicurus says that void exists\hspace*{30mm}[P1*] Chrysippus says that void does not exist
\vspace*{1mm}

\underline{[P2] Epicurus is always correct}\hspace*{35mm}\underline{[P2*] Chrysippus is always correct}
\vspace*{1mm}

[C] Void exists\hspace*{60mm}[C*] Void does not exist
\vspace*{2mm}

\noindent\hspace*{3mm}Other frequent example: appealing to religious/spiritual/lay authority (both textual and personal)
\vspace*{3mm}

\noindent \textbf{[2]} The mode based on ``relativity''---this actually is a reference to the Ten Modes (so the ten modes are ``contained in'' the five modes) (\emph{HP} 325-337)
\vspace*{2mm}

[1] Variation among animals
\vspace*{1mm}

[2] Differences among humans
\vspace*{1mm}

[3] Different conditions of the sense organs
\vspace*{1mm}

[4] Circumstances
\vspace*{1mm}

[5] Positions, distances, and places
\vspace*{1mm}

[6] Mixtures
\vspace*{1mm}

[7] Quantities and structures of external objects
\vspace*{1mm}

[8] Relativity (Sextus notes that this, in a sense, contains the other nine modes)
\vspace*{1mm}

[9] Constant or rare occurrences
\vspace*{1mm}

[10] Practices, laws, beliefs in myths, and dogmatic suppositions
\vspace*{2mm}

\noindent Example: [1] Variation among animals---olive oil is beneficial to humans but harmful to bees
\vspace*{2mm}

[P1] Olive oil benefits humans\hspace*{44.5mm}[P1*] Olive oil harms bees
\vspace*{1mm}

\underline{[P2] What benefits humans is beneficial}\hspace*{30mm}\underline{[P2*] What harms bees is harmful}
\vspace*{1mm}

[C] Olive oil is beneficial\hspace*{53mm}[C*] Olive oil is harmful
\newpage

\noindent \textbf{[3]} The mode based on ``infinite regress''
\vspace*{2mm}

\hspace*{2.5mm}.\hspace*{40mm}.
\vspace*{1mm}

\hspace*{2.5mm}.\hspace*{40mm}.
\vspace*{1mm}

\hspace*{2.5mm}.\hspace*{40mm}.
\vspace*{1mm}

[Pn]\hspace*{34mm}[Pn*]
\vspace*{1mm}

\underline{[Pn+1]}\hspace*{30mm}\underline{[Pn+1*]}
\vspace*{1mm}

[C]\hspace*{36mm}[C*]
\vspace*{3mm}

\noindent \textbf{[4]} The mode based on ``hypothesis''
\vspace*{2mm}

[P1] Human beings are mortal\hspace*{30mm}[P1*] Human beings are immortal
\vspace*{1mm}

\underline{[P2] Socrates is a human being}\hspace*{29.5mm}\underline{[P2*] Socrates is a human being}
\vspace*{1mm}

[C] Socrates is mortal\hspace*{43.5mm}[C*] Socrates is immortal
\vspace*{2mm}

\noindent This might strike you as bizarre, since the likely response is to say ``But P1* is \emph{false} and P1 is \emph{true}.'' The skeptic will say, ``Ahh, that is a different issue. You have just stated P1 as a hypothesis here, so it has no more force than P1*, also stated as a hypothesis. If you would like to discuss the relative merits of P1 vs. P1*, by all means, let's do that.'' Thus, it's easy to see how these discussions, when carried far enough, will go back to issues concerning criteria of truth.
\vspace*{2mm}

\noindent But, think about more plausible ``hypotheses,'' such as ``something is only bad for X if X can somehow be aware of X or X's consequences''
\vspace*{3mm}

\noindent \textbf{[5]} The mode based on ``circular reasoning''
\vspace*{2mm}

[P1] Socrates is mortal\hspace*{42.5mm}[P1*] Socrates is immortal
\vspace*{1mm}

\underline{[P2] Socrates is a human being}\hspace*{30mm}\underline{[P2*] Socrates is a human being}
\vspace*{1mm}

[C] Socrates is mortal\hspace*{43.5mm}[C*] Socrates is immortal
\vspace*{4mm}

\noindent Think about how these three modes ``work together''---unless someone is going to present an infinite argument, the argument has to stop somewhere. It can either stop at a premise that is different from the conclusion or the same as the conclusion. The worry is that, either way, the skeptic can pounce.
\vspace*{2mm}

\noindent Is there any hope?

\end{document}
