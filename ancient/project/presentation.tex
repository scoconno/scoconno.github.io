\documentclass[oneside]{article}
 \headheight = 25pt
\footskip = 20pt
\usepackage{mdwlist}
\usepackage[T1]{fontenc}
\renewcommand{\rmdefault}{ppl}
\usepackage{fancyhdr}
 \pagestyle{fancy}
 \lhead{\textbf{\textsc{\small Scott O'Connor\\Ancient Philosophy}}}
 \chead{}
 \rhead{\large\textbf{\textsc{Presentation}}}
 \lfoot{\footnotesize{\thepage}}
 \cfoot{}
 \rfoot{\footnotesize{\today}}
 \usepackage{longtable,booktabs}
\tolerance=700

\begin{document}


\section*{Introduction}\label{introduction}

Great progress! You have a question. You have identified several
readings that are relevant to that question. You have explained the
background to your questions and sketched out some standard answers in
the literature. Congratulate yourself on how much research you have now
done.

Now that you have surveyed the literature, it is time for you to form
your own opinions on the question. Do you think one of the answers that
has been defended is correct? If so, why that one as opposed to another?
Do you think there is an ignored alternative? Now is the time for you to
advance the conversation.

I hope that you are beginning to realize that research is a very gradual
process. Forming your own views is also a gradual process. Researchers
don't come up with something new overnight. They take the germ of an
idea, mull over it, talk about it with peers and colleagues, and then slowly
develop it.

This presentation provides you the opportunity to formulate your views
before writing your final paper. Researchers often do present their
findings at conferences before publishing them. So too you will be
presenting to your colleagues your views on your topic.


\section*{Requirements}\label{requirements}

\noindent \textbf{7 minutes maximum!!!}\\

\noindent Write 6 slides. Each slide should contain a striking picture, or quote,
or graph, or other visual cue. It must also provide in bullet form a
summary of the main point of the slide. You will be using your slides as
cues---do not put the entire text on the slide. The 6 slides must be
written as follows (each number corresponds to a slide):

\begin{enumerate}
\item
  Relevant background on your topic.
\item
  Question raised.
\item
  Answer 1.
\item
  Answer 2.
\item
  Your novel contribution introduced.
\item
  Evidence/support/argument for your novel contribution.
\end{enumerate}

\noindent \textbf{If interested in doing an alternative presentation, please talk to me in advance. Some might like to introduce some music, show clips of plays, etc.}

\end{document}
