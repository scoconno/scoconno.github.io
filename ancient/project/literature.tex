\documentclass[oneside]{article}
 \headheight = 25pt
\footskip = 20pt
\usepackage{mdwlist}
\usepackage[T1]{fontenc}
\renewcommand{\rmdefault}{ppl}
\usepackage{fancyhdr}
 \pagestyle{fancy}
 \lhead{\textbf{\textsc{\small Scott O'Connor\\Ancient Philosophy}}}
 \chead{}
 \rhead{\large\textbf{\textsc{Literature Review}}}
 \lfoot{\footnotesize{\thepage}}
 \cfoot{}
 \rfoot{\footnotesize{\today}}
 \usepackage{longtable,booktabs}
\tolerance=700



\begin{document}
\thispagestyle{fancy}

\subsection*{Requirements}

\begin{itemize}
\item
  Literature Review--see below
\item
  Due Date: 04/23/2018 at midnight through Blackboard
\item
  Word Count: 750-1000 words divided fairly equally between the sections
  described below.
\item
  Citations and bibliography required.
\end{itemize}



\subsection*{What is a literature
review?}\label{what-is-a-literature-review}

Let's start with an analogy. You've been asked to design a website to
help renters find homes in Jersey City. To make it easier for renters,
you've been asked to identify some criteria that would allow them narrow
their searches. What should these search criteria be? In order to answer
this question, you need to read several different rental adverts. What
are the common types of information that comes up? Obviously, price,
location, size, square footage, etc. But what about a dishwasher? What
about the color of the walls, carpet type, proximity to cinemas? Users
of the website will be overwhelmed by too many criteria, so you need to
decide what's most useful for them.

Writing a literature review is similar in two ways:

\begin{enumerate}
\def\labelenumi{\arabic{enumi}.}
\itemsep1pt\parskip0pt\parsep0pt
\item
  Your job in writing a literature review is to identify relevant
  similarities and dissimilarities in the various resources that seem to
  address your topic. You will regularly find that different parts of
  the literature take opposing sides on your question. Here we have an
  important trend that you want to mention. You will also find common or
  dominant reasons for various answers. Again, this is something you
  want to note and write about.
\item
  The person developing the website doesn't tell you the best house to
  live in. They group together the most important features that renters
  might look for. Similarly, literature reviews don't take a stand on
  the issues. They try to present the various dominant/common views
  about an issue as neutrally as possible.
\end{enumerate}

What about all the nitty gritty details? Do you have to write in length
about them? Probably not. Suppose you would like to know why students
pick one university over another. You conduct an interview of 1000
students. You may find that cost, reputation, location, etc., are common
responses. Let us also imagine that you collect detailed information
about each student, about how much they can afford, their price
tolerance, etc. Do you need to include these details of the 1000
students into your study? Of course not! Your job is to extrapolate from
the details and report what's common.

Does that mean that no detail is required? No. Literature reviews often
include a discussion of some representative resource. Think of our
previous example. Suppose you discover that students weigh the cost of
their education heavily when deciding what school to attend. Don't tell
me about each of the 1000 students. But it helps to talk in detail about
one or two students who really illustrate the general discovery that you
have made.

Similarly, literature reviews will both explain some of the current,
common views about a topic \emph{and} discuss a couple of resources to
illustrate these common features/trends.

%\subsection*{The Process--Don't submit thisvpart!}\

%Here are some general approaches to writing a literature review. Best
%practice may vary by discipline:

%\begin{enumerate}
%\item
 % Create an annotation for each item in your bibliography. Don't worry about reading the item in full. Often reading abstracts, introductions, and conclusions will suffice. Skim a relevant chapter  in a book, not the whole book. Your goal is to identify the relevance it has to your main question.
%\item Write a couple of sentences stating how your source answers your
 % question. Write another couple of sentences outlining the main reason
%  for its answer. Note some interesting details and examples. One or two
 % nice quotes is also helpful. Keep account of the page numbers.
%\item
%  Look at your annotations. Identify themes and common claims. Group the
%  items into different categories. For instance, you might notice that
%  there are two diverging answers to your question. You might also
%  notice that there are some agreements on why to answer the question
%  one way rather than the other.
%\item
%  Write down the common themes, arguments, you see in your annotations.
 % Try describe these commonalities as clearly as possible.
%\item
%  Choose some representative works, e.g., maybe there is a particularly
 % strong article that argues for one answer and a different one that
 % argues for a different answer.
%\item
%  Read your representative works in detail. Take detailed notes as you
%  do so.
%\end{enumerate}

\subsection*{Writing the Literature Review}\label{writing-the-literature-reviewdo-submit-this-part}

The structure of literature reviews can vary. However, since the main goal of this exercise is to teach you
how to do research, I want you to write the following three sections (further details below).

\begin{enumerate}
\item Establish a shared context and raise your question. 
\item Explain answer 1 using a representative resource. 
\item Explain answer 2 using a representative resource.
\end{enumerate}
 This structure is artificial. Many questions admit of more than two
answers. Many similar answers have subtle differences. That's ok! If you
can identify and explain two different mains answers to your question,
then you will have learned the skill this assignment is designed to
teach. If you really can't fit your project into this structure, then talk to me about an alternative.


%The first raises a question. The second offers one answer. The third
%offers a different answer. Both the second and third sections outline
%the reasons for their respective answers and include some details from a
%representative resource defending that answer.


\subsubsection*{Section 1: Context}\label{section-1-context}

Suppose you pick up a paper and it begins `Students choose college based
on either cost alone or reputation alone.' That's a bit abrupt. The
author does not explain why it's important to identify why students pick one
college over another, nor do they explain why anyone would be interested
in reading about the issue. Now consider this opening:

\begin{quote}
New Jersey's high schools are full and getting fuller. We are graduating
more and more of these students too. Not only that, more and more of our
graduates are going on to attend college. So, New Jersey seems to be
doing well by its young folk. Unfortunately, though, this good news is
tempered by a worrying trend. Fewer and fewer or our students are
attending college in New Jersey itself. Our young people are leaving in
great numbers to attend colleges and universities in other states. This
bodes poorly for the future. It's unclear whether they will return to
benefit the state they left. Why is this happening? Why are young people
choosing to attend college out of state? There are two opposing views.
One group of researchers argue that cost alone is the culprit. They
claim that our young people are leaving because colleges and
universities in other states are cheaper. Not everyone agrees. A
different group of researchers blame the exodus on students' belief in
the respective qualities of in-state and out of state colleges. The
researchers offer very different remedies to the proposal. If the former
is correct, then we should focus on lowering tuition in the Garden
State. If the latter is correct, we need to do a better job at marketing
the strengths and successes of our state institutions.
\end{quote}

Notice that this introduction does three things. First, it establishes a
shared context, i.e., it informs the reader about some phenomenon.
Second, it raises a question or problem about the phenomenon. Third, it
briefly summarizes a few answers and explains those answers' importance.

Your literature review needs to begin in this way too. Provide some
context. You will do this by telling the reader about the primary text you are interested in. What is that text about? What details must a reader absolutely be aware of if they are to understand the question you want to ask about the text? Then explicitly state your question about the text and spell it out; why does the question arise? Why is the question not easy to answer? Motivate your reader to read the paper. 


\subsubsection*{Section 2: Answer 1}\label{section-2-answer-1}

Here you will explain one way your question is answered in the
literature. Your job here is to summarize the answer given and also the
main reasons offered for that answer. You will also write a few
paragraphs summarizing some details from a representative work that
defends this answer.

\subsubsection*{Section 3: Answer 2}\label{section-3-answer-2}

Here you will explain a different way that your question is answered in
the literature. Your job here is to summarize the answer given and also
the main reasons offered for that answer. You will also write a coupe of
paragraphs summarizing some details from a representative work that
defends this answer.

\end{document}
