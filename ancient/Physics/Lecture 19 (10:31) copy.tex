\documentclass[oneside]{article}
 \headheight = 25pt
\footskip = 20pt
\usepackage{mdwlist}
\usepackage[T1]{fontenc}
\renewcommand{\rmdefault}{ppl}
\usepackage{fancyhdr}
 \pagestyle{fancy}
 \lhead{\textbf{\textsc{\small Scott O'Connor\\Ancient Philosophy}}}
 \chead{}
 \rhead{\large\textbf{\textsc{Physics 1}}}
 \lfoot{\footnotesize{\thepage}}
 \cfoot{}
 \rfoot{\footnotesize{\today}}
 \usepackage{longtable,booktabs}
\tolerance=700


\begin{document}
\thispagestyle{fancy}



\section*{Background}

\noindent Aristotle posits a fundamental metaphysical distinction between substantial entities and non-substantial entities (qualities, quantities, etc.). Substance translates the Greek word, ‘hupokeimenon’, which means roughly ‘what underlies’. It is used to describe that which possesses qualities, quantities, relations, and activities, but it is not itself possessed by anything else. We will say that qualities, quantities, relations, and activities inhere in substances, where P inheres in S if S somehow possesses or has P, e.g., the color purple inheres in my water bottle because the bottle has that color, my sitting inheres in me because I am doing that sitting, etc. 

Substances are obviously different from what inheres in them. The weight of a stone is different from the stone itself. The motion of a molecule is different from the molecule itself. Motion and weight belong to these substances, but they are not themselves substances. In contrast, the molecule and the stone are substances because they are subjects in which other things inhere but do not themselves inhere in anything else, e.g., the stone is not a property of anything else, but other things are properties of it.

This contrast between substances and what inheres in them obviously resembles a linguistic contrast between grammatical subjects and predicates, between ‘Flipper’ and ‘swimming’ in the sentence ‘Flipper is swimming’. But the contrast is more fundamental than the linguistic one. The word ‘swimming’ could also serve as a grammatical subject, e.g., ‘swimming is a pleasant activity’. The word ‘Flipper’ could also serve as a predicate, e.g., ‘the toy belongs to Flipper’. But substances can never inhere in anything else; they can never be possessed by anything else. They are the fundamental bedrock for everything that inheres in them.

\vspace*{2mm}

\noindent The criteria for something's being a \emph{primary} substance are: [1] not said of or in anything else, [2] subject for everything else, [3] prior, [4] a ``this'' (individually and ``numerically one''), [5] is able to receive contraries.
\vspace*{2mm}

\noindent In the \emph{Physics} he sets out to show that there are items in the natural world which meet these conditions, and Book 1 focuses primarily on [5]
\vspace*{-3mm}

\section*{A Presocratic problem with change}

\noindent A claims that several Presocratics argued that, despite appearances, coming-to-be is impossible. A presents their argument as follows (Ch. 8):

\begin{enumerate}
\item [P1] Anything that comes-to-be either comes-to-be from what is or from what is not
\item [P2] It cannot come-to-be from what is, for what is already is
\item [P3] It cannot come-to-be from what is not, since there isn't anything for it to come-to-be from
\item  [C] Thus, nothing can come-to-be
\end{enumerate}

\section*{Aristotle's ``quick'' answer}

\noindent A distinguishes between coming-to-be ``unqualifiedly'' and ``coincidentally'' (Ch. 5)
\begin{itemize}\item{Something would come-to-be unqualifiedly from its opposite (e.g. tan from pale; tall from short; musical from non-musical)}\item{Something comes-to-be unqualifiedly from what ``coincides'' with something that it would come-to-be from primarily (e.g. musical comes-to-be coincidentally from pale insofar as what is pale happens to be unmusical; tall comes-to-be coincidentally rom human insofar as what is a human happens to be short)}\end{itemize}

\noindent With that distinction in hand, A says that both ``horns'' (i.e. P2 and P3) are true \emph{in a way} (Ch. 8)
\vspace*{2mm}

\noindent There are three distinct items involved in any change:
\vspace*{1mm}

[1] The old property (the ``privation'') [`G']
\vspace*{1mm}

[2] The new property (the ``form'') [`F']
\vspace*{1mm}

[3] The enduring subject of both properties [`X']
\vspace*{2mm}

\noindent P2 is, in a sense, true: what is, for example, Hot (`F'), can't come-to-be from what is already Hot. But, in another sense, it is false in that Hot can come-to-be coincidentally from Socrates, which already is (something else, such as not-Hot (`G')).   
\vspace*{2mm}
\noindent Similarly P3 is, in a sense, true: it's not wholly accurate to say, for example, that Hot (`F') comes-to-be from not-Hot (i.e. Cold, `G'), because there needs to be some subject, such as Socrates (`X'). And so the Presocratics were right that nothing can come to be from what is not, if that is understood as meaning nothing can come to be unqualifiedly from what is not. But, something can come to be from what is not in another sense, namely coincidentally:
\begin{itemize}
\item For example, when Hot (`F') comes-to-be it comes-to-be coincidentally from what is not, namely Socrates (`X'), who happens to be not-Hot (`G')
\end{itemize}

\section*{General Analysis}

\noindent There are three ways to describe any change

\begin{enumerate}

\item[A.] The man become musical...[the man, the subject, remains]

\item[B.] Not-musical becomes musical...[G, the privation does not remain] 

\item[C.]The not-musical man becomes a musical man...[X, the subject part of the compound remains. The compound, however, does not remain]
\end{enumerate}
\noindent Despite appearances, Aristotle thinks that [A], [B], and [C] really only describe one thing: a case where a man goes from being unmusical to musical. Thus, in all changes, a subject endures through the change and receives the contraries, where one of the contraries is a privation and the other is the form
\vspace*{2mm}

\noindent At each time, the thing-which-can-change is a compound of subject \& form \emph{or} privation. So, in a sense, A thinks that it was the Presocratics' failure to recognize that all items are compounds that led them astray.

\section*{Applied to Substances}

\noindent Changes are changes of place (locomotion), of degree (growth), of quality (alteration), or of substance (generation)
\vspace*{2mm}

\noindent Substantial change (i.e. generation) = unqualified coming-to-be = X comes-to-be (the other kinds are qualified changes = X comes-to-be F)
\vspace*{2mm}

\noindent The above cases are examples of qualified changes. Unqualified changes seem more difficult: what is the enduring subject when a human being comes into existence? Can you say that there is something that was not a human being but comes-to-be a human being?
\vspace*{2mm}

\noindent To locate the enduring subject in this case, we need to decompose the final result:
\vspace*{1mm}
\begin{itemize}
\item When a brazen sphere comes into existence, it comes-to-be from a ``lump of bronze'' and the bronze in the lump endures.
\vspace*{1mm}
\item When a human embryo comes-to-be, it comes to be from ``seed'' and the stuff in the seed endures,
\end{itemize}
\end{document}
