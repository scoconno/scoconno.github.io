\subsubsection*{The ``Imperfection Argument'' (Phaedo 74-76)}
A helpful formulation from  http://faculty.washington.edu/smcohen/320/phaedo.htm
\begin{enumerate}
\item  We perceive sensible objects to be F.
\item But every sensible object is, at best, imperfectly F. That is, it is both F and not F. It falls short of
  being perfectly F.
\item We are aware of this imperfection in the objects of perception.
\item  So we perceive objects to be imperfectly F.
\item  To perceive something as  imperfectly F, one must have in mind something that is perfectly F,  something that the imperfectly F things fall short of. (e.g., we have
  an idea of equality that all sticks, stones, etc., only imperfectly
  exemplify.)
\item So we have in mind something that is perfectly F.
\item  Thus, we must have at one time encountered something that is perfectly
  F (e.g., the form of equality), that we have in mind in such cases.
\item  Therefore, there is such a thing as the F itself (e.g., the equal itself), and it
  is distinct from any sensible object (given that we recognize that all
  sensible things are imperfectly F).
\end{enumerate}
According to this argument, there must be  perfect forms---a form of equality, beauty, etc---from which we  acquire our concepts/ideas of (perfect) equality, beauty, etc, since  there's no way that we could have acquired such concepts from  (imperfect) sensibles.

\subsubsection*{ What do both arguments tell us about forms?}

The first argument tells us, for instance, that forms are imperceptible. What else do we learn about them? 

%+ Forms are the materials of  definitions.

%+ Forms are what we recollect during recollection.
%+ Forms are somehow ``more perfect'' than sensibles.

\subsection*{Death and the forms}
S believes that 1) we cannot gain knowledge of the forms by perceiving them, and 2) the body is a constant impediment to philosophers in their search for knowledge of them. 
\begin{quote}
``It [the body] fills us with wants, desires, fears, all sorts of illusions and much nonsense, so that, as it is said, in truth and in fact no thought of any kind ever comes to us from the body'' (66c). 
\end{quote} 
So, to obtain knowledge of the forms, philosophers must escape from the influence of the
body as much as is possible. The only way to fully to do so is to die. 

