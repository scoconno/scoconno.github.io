\documentclass[oneside]{article}
 \headheight = 25pt
\footskip = 20pt
\usepackage{mdwlist}
\usepackage[T1]{fontenc}
\renewcommand{\rmdefault}{ppl}
\usepackage{fancyhdr}
 \pagestyle{fancy}
 \lhead{\textbf{\textsc{\small Scott O'Connor\\Ancient Philosophy}}}
 \chead{}
 \rhead{\large\textbf{\textsc{Phaedo 1}}}
 \lfoot{\footnotesize{\thepage}}
 \cfoot{}
 \rfoot{\footnotesize}
 \usepackage{longtable,booktabs}
\tolerance=700


\begin{document}
\thispagestyle{fancy}

\subsection*{Introduction}\label{introduction}
In the \emph{Phaedo}, Socrates and his friends gather for the last time. The day of his execution has arrived. Socrates' friends are stunned at how calm he seems, a calmness he maintains even as he drinks the Hemlock.  He explains that not only does he not fear death, he welcomes it. The dialog consists of a long discussion on why he welcomes it. This continues as he drinks the hemlock. 

Socrates' most striking claim is that philosophers, above everyone else,  should welcome
death. Death, according to him, is the separation of body and soul (psuche).
The Greek word psuche was later translated in Latin as `anima', which is closer
to the original meaning. It is something that animates a body, i.e., it
is that which gives life to a body. On this view, every living creature has a
soul (because it has something which animates it). In later philosophical and religious thought, only humans are seen as having a soul. It is an interesting question how, for instance, Descartes came to understand soul differently than Socrates. For our purposes, we only need note that the Greek view of the soul is not the that of the Abrahamic religions. 

If death involves separation of the soul from the body, then the soul and body are obviously different. But the exact nature of that difference is unclear. Even in the \emph{Phaedo} itself we find participants disagreeing over how exactly the soul and body differ. To help characterize these disagreements, it is helpful to ask two distinct questions: 1) which functions do the body and soul perform each by themselves and which, if any, do they perform together? 2) Can soul and body exist independently of one another? Socrates will answer `yes' to this latter question. But, while many, including Aristotle, agree that the body and soul are different, they disagree with Socrates on whether they can exist without one another. 

What does Socrates think the soul and body do? Consider the following: 
\begin{quotation}
When then, he asked, does the soul \textbf{grasp the truth?} For whenever it attempts to \textbf{examine} anything with the body, it is clearly \textbf{deceived} by it.

True.

Is it not in \textbf{reasoning} if anywhere that any reality becomes clear to the soul?

Yes.

And indeed the soul reasons best when none of these senses troubles it, neither hearing nor sight, nor pain nor pleasure, but when it is most by itself, taking leave of the body and as far as possible having no contact or association with it in its search for reality.

\end{quotation}
From this passage, we can easy see Socrates believes that the soul examines, reasons, is deceived, grasps the truth, etc. We also see that the soul does not seem to be thing that hears, sees, or experiences pains or pleasures.  Examine 66a--67e to identify features and functions that Socrates thinks belong to the body. 

\subsection*{Why does Socrates' welcome death?}
Socrates believes that a the body is a constant impediment to philosophers in their search for truth: “It fills us with wants, desires, fears, all sorts of illusions and much nonsense, so that, as it is said, in truth and in fact no thought of any kind ever comes to us from the body” (66c).  To have pure knowledge, therefore, philosophers must escape from the influence of the body as much as is possible in this life. Thus, Socrates concludes, it would be unreasonable for a philosopher to fear death, since upon dying he is most likely to obtain the wisdom which he has been seeking his whole life. More formally, S's argument is as follows:

\begin{itemize}
\item[P1:] Philosophers desire knowledge.
\item[P2:] Philosophers cannot acquire that knowledge while they are alive, i.e., when their soul inhabits their body.
\item[P3:] Either philosophers are mortal and can never acquire knowledge, or they are immortal and can acquire knowledge when they die (when their soul is separated from their body).
\item[P4:] Philosophers are immortal. 
\item[C1:] Philosophers can acquire the knowledge they desire when they die, i.e., when their soul has been separated from their body.
\item[C2:] Philosophers should welcome death, since upon dying they are most likely to obtain the knowledge they have been seeking their whole life.
\end{itemize}
Our focus this class is on premise 2.


\subsection*{Forms}
S believes that the objects of knowledge are what are called ``forms'' later in the dialog (103e). Forms  mentioned are the just itself, the beautiful, and the good; bigness, health, and strength; and ``in a word, the reality of all other things, that which each of them essentially is'' (65d). We will learn the following about forms today, and we will learn more tomorrow and next week: 
\begin{enumerate}
\item Forms are imperceptible
\item Forms are the materials of definitions. 
\end{enumerate} 
The form of beauty cannot be seen, touched, heard, etc. They are not something we can use our five senses to access; we will need something other than our senses to gain knowledge of them. Socrates thinks that these forms play a special role in explaining reality. In general, a form F is predicated of individual sensible F things; individual sensible things are F by participating in the form of F. For instance, an individual person is beautiful by participating in the form of beauty. 

\subsection*{Aristotle on Plato's reasons for positing forms}
The \emph{Phaedo} is a transitional dialog. While it likely reports S's death accurately, and while there was likely some conversation between S and his friends, interpreters generally believe we find in this dialog some of Plato's own views. But that does not mean he radically diverged from S. P developed his teacher's ideas and offered new arguments for them. So, for instance, S is unlikely to have used the word `forms', but P calls S's search for the just itself as a search for the form of justice. According to Aristotle, P introduced forms because P was influenced by Heraclitus' and Cratylus' views that everything in the sensible/observable/material world is somehow changing or unstable. The worry was that we could only find a satisfying definition of F (and hence have knowledge of F things) if there are stable, unchanging, forms. By `change', P seems concerned with two issues: 

\begin{description}
\item[Succession of Opposites (SO):] If X undergoes SO, then X is F at t1, but
becomes not-F at some later time t2; in other words, cases of SO are
cases where one and the same thing has opposite properties or
characteristics at different times. F and G are opposite properties iff an object can be either F or G, but not both F and G at the same time, e.g., I was sick in January (t1), but I was healthy by February (t2).

\item[Compresence of Opposites (CO):] X has  properties F and not-F at the same time, e.g., Simias is both taller and shorter. He's taller than X, shorter than Y.

\end{description}
Since perceptible objects undergo change, P believes we can't appeal to them in finding definition. Why?  The worry seems to be that if we focus on the sorts of properties that are matters of observation we'll come up with properties that pick out F things no more than not-F things. For instance, suppose you see a beautiful person with perfectly equal proportion. You subsequently define beauty as the equal proportions between parts. If this definition is adequate, then every beautiful things would need to be beautiful because of equal proportions between their parts. But Plato believes that we will likely find other objects with such parts that are not beautiful. So, in general, if you try to define X by appeal to a feature you can see, hear, touch, etc., then you cannot find a definition that is sufficiently general and univocal. Here is a summary of the argument: 
\begin{enumerate}
\item  To have knowledge about F, one must have a definition of F. (Recall  Meno 71b: In order to know whether or not virtue is teachable,  one must first have a definition of virtue
\item  Sensibles are in flux  (suffer compresence of opposites).
\item  So, we cannot appeal to any sensible/perceptible object or property to get an adequate definition
  of F; any sensible object or property that we pick out will be both F  and not F.
\begin{itemize}
\item I can't focus on some observable act to define justice, because any observable
thing that I pick out---e.g. returning what I have borrowed---in some
cases will be just, and in other cases will be unjust.
\end{itemize}
\item Knowledge is possible.
\item  So, there must be adequate definitions that would give us this knowledge
\item So, there must be non-sensible abstract objects (forms) to which we can appeal when
  defining F.
\end{enumerate}


\end{document}
