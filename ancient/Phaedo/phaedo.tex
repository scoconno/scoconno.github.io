\documentclass[oneside]{article}
 \headheight = 25pt
\footskip = 20pt
\usepackage{mdwlist}
\usepackage[T1]{fontenc}
\renewcommand{\rmdefault}{ppl}
\usepackage{fancyhdr}
 \pagestyle{fancy}
 \lhead{\textbf{\textsc{\small Scott O'Connor\\Ancient Philosophy}}}
 \chead{}
 \rhead{\large\textbf{\textsc{Phaedo 1}}}
 \lfoot{\footnotesize{\thepage}}
 \cfoot{}
 \rfoot{\footnotesize{\today}}
 \usepackage{longtable,booktabs}
\tolerance=700


\begin{document}
\thispagestyle{fancy}

\subsection*{Introduction}\label{introduction}

Socrates claims that philosophers should welcome
death. Death is defined as the separation of body and soul (psuche).
The Greek word was later translated in Latin as `anima', which is closer
to the original meaning. It is something that animates a body, i.e., it
is that which gives life to a body. On this view, every living creature has a
soul (because it has something which animates it). S's argument that philosophers should welcome death:

\begin{itemize}
\item[P1:] Philosophers desire knowledge.
\item[P2:] Philosophers cannot acquire that knowledge while they are alive, i.e., when their soul inhabits their body.
\item[P3:] Either philosophers are mortal and can never acquire knowledge, or they are immortal and can acquire knowledge when they die (when their soul is separated from their body).
\item[P4:] Philosophers are immortal. 
\item[C1:] Philosophers can acquire the knowledge they desire when they die, i.e., when their soul has been separated from their body.
\item[C2:] Philosophers should welcome death, since upon dying they are most likely to obtain the knowledge they have been seeking their whole life.
\end{itemize}
Our focus this class is on premise 2. S claims that the philosopher despises bodily pleasures such as food, drink, and
  sex, so he more than anyone else wants to free himself from his body. Why does S think that the philosopher despises bodily pleasures? There are at least two reasons given in 66a--67e. 

\subsection*{Forms}
S believes that the objects of knowledge are what are called ``forms'' later in the dialog (103e). Forms  mentioned are the just itself, the beautiful, and the good; bigness, health, and strength; and ``in a word, the reality of all other things, that which each of them essentially is'' (65d). In general, (i) a form F is predicated of individual sensible F things; individual sensible things are F by participating in the form of F, (ii) the form of F is itself F and is never not-F, the form of beauty is beautiful and cannot be ugly, (iii) individual sensible things can be both F and not-F, i.e., e.g., a vase can be both beautiful and ugly. 


\subsection*{Aristotle on Plato's reasons for positing forms}
The \emph{Phaedo} is a transitional dialog. While it likely reports S's death accurately, and while there was likely some conversation between S and his friends, interpreters generally believe we find in this dialog some of Plato's own views. But that does not mean he radically diverged from S. P developed his teacher's ideas and offered new arguments for them. So, for instance, S is unlikely to have used the word `forms', but P calls S's search for the just itself as a search for the form of justice. According to Aristotle, P introduced forms because P was influenced by Heraclitus' and Cratylus' views that everything in the sensible/observable/material world is somehow changing or unstable. The worry was that we could only find a satisfying definition of F (and hence have knowledge of F things) if there are stable, unchanging, forms. By `change', P seems concerned with two issues: 

\begin{description}
\item[Succession of Opposites (SO):] If X undergoes SO, then X is F at t1, but
becomes not-F at some later time t2; in other words, cases of SO are
cases where one and the same thing has opposite properties or
characteristics at different times. F and G are opposite properties iff an object can be either F or G, but not both F and G at the same time, e.g., I was sick in January (t1), but I was healthy by February (t2).

\item[Compresence of Opposites (CO):] X has  properties F and not-F at the same time, e.g., Simias is both taller and shorter. He's taller than X, shorter than Y.

\end{description}
Since perceptible objects undergo change, P believes we can't appeal to them in finding definition. Why?  The worry seems to be that if we focus on the sorts of properties that are matters of observation we'll come up with properties that pick out F things no more than not-F things. For instance, suppose you define beauty as the observable feature of equal proportions between parts. P believes that some entities with equal proportions between their parts will be beautiful, but we will find other objects with such parts that are not beautiful. Here is a summary of the argument: 
\begin{enumerate}
\item  To have knowledge about F, one must have a definition of F. (Recall  Meno 71b: In order to know whether or not virtue is teachable,  one must first have a definition of virtue
\item  Sensibles are in flux  (suffer compresence of opposites).
\item  So, we cannot appeal to any sensible/perceptible object or property to get an adequate definition
  of F; any sensible object or property that we pick out will be both F  and not F.
\begin{itemize}
\item I can't focus on some observable act to define justice, because any observable
thing that I pick out---e.g. returning what I have borrowed---in some
cases will be just, and in other cases will be unjust.
\end{itemize}
\item Knowledge is possible.
\item  So, there must be adequate definitions that would give us this knowledge
\item So, there must be non-sensible abstract objects (forms) to which we can appeal when
  defining F.
\end{enumerate}





\end{document}
