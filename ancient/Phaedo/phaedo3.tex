\documentclass[oneside]{article}
 \headheight = 25pt
\footskip = 20pt
\usepackage{mdwlist}
\usepackage[T1]{fontenc}
\renewcommand{\rmdefault}{ppl}
\usepackage{fancyhdr}
 \pagestyle{fancy}
 \lhead{\textbf{\textsc{\small Scott O'Connor\\Ancient Philosophy}}}
 \chead{}
 \rhead{\large\textbf{\textsc{Phaedo 3}}}
 \lfoot{\footnotesize{\thepage}}
 \cfoot{}
 \rfoot{}
 \usepackage{longtable,booktabs}
\tolerance=700


\begin{document}
\thispagestyle{fancy}
\subsection*{Introduction}
After summarizing Cebes' objection that the soul may outlast the body yet not be immortal, Socrates says that this problem requires ``a thorough investigation of the cause of generation and destruction''. Socrates reports an intellectual biography in which he became disillusioned by traditional scientific explanations. He then asks, "What is an explanation, whether of generation and destruction or anything else?" He will ultimately answer that explanations are forms. And he will use that answer to argue for the immortality of the soul. 

In studying this material, it is important to see Socrates as developing his account of Socratic definitions. We saw that Socrates searches for answers to the Socratic 'what is x' question. The answers are called Socratic definitions. In the \emph{Phaedo}, we find a new question, 'why is x as it is?' or 'why is x F?' Answering the second question requires finding an \emph{aitia}, which can be translated as either cause or explanation. The questions 'what is x?' and 'why is x as it is?' are different. 'What is thunder?' and `why is thunder so loud?' are different questions. So too, `who is Achilles?' and `why is Achilles courageous'? are different questions.  

The \emph{Euthyphro}, we saw, illustrated what is required to adequately answer the `what is x?' question through failure, i.e., by showing us how Euthyphro's definitions of piety fail. The \emph{Phaedo} will do two things. It will show us what is required to answer the second question,  the `what is x as it is' question, through failure. It will also show us that there is, surprisingly, a close connection between answers to the 'what is x?' question and 'why is x as it is?' question. 

It may seem clear that answers to these two questions are related. Someone might respond to the question 'what is ice?' by claiming that it is frozen water, or water that has been frozen. And, this of course, does relate to why ice is as it is, or at least why ice came to be. But not all questions are so obviously related. 'What is Socrates?' and 'Why is Socrates in prison?' do not have clearly related answers at all. The former question has as its answer that Socrates is an ensouled the body. The latter has as its answer that Socrates thinks it better to follow the court's execution order. 

\subsection*{Explanations}
We saw that Socratic definitions must be general and univocal. They must also be explanatory, something we have not yet discussed in detail. Socrates places the following general constraint on an adequate explanation: 

\begin{description}
\item[REQ1:] If same explanandum, then same explanans.
\begin{itemize}
\item it is impossible for numerically different particular things, in so far as they have the same quality, to have different and incompatible explanantia.
\item it is necessary that, if E1 is the explanans of why x is f, and E2 is the explanans of why y is f (where x and y are numerically different particulars), then E1 = E2
\end{itemize}
\item[REQ2:] If same explanans, then same explanandum
\begin{itemize}
\item	it is impossible for numerically different particular things, in so far as they have different and incompatible qualities, to have the same explanantia.
\item	it is necessary that, if E is the explanans of why x is f, and E is the explanans of why y is g (where x and y are numerically different particulars), then f = g
\end{itemize}
\end{description}

%\noindent These are requirements of the \textbf{uniqueness} of explanation because they imply that each explanandum has a single explanans, the one proper to it, and conversely each explanans explains a single explanandum, the one proper to it.\\

%\noindent These are requirements of the \textbf{generality} of explanation because they imply that, if a particular thing, x, has an explanation, it does so in virtue of having a general quality, f. This is a general quality in the sense that numerically different particular things, x and y, can have the same quality, f.


\subsection*{The positive proposal}

\begin{description}
\item[Simple Schema:] The explanation of why a thing, x, has a quality, f, is that x is appropriately related directly to the essence of this quality (100b1–102b3). For instance, a body is hot because of heat,  and a body is  sick because of sickness. 
\item[Complex Schema:]  The explanation of why a thing, x, has a quality, f, is that x is appropriately related to something, e.g. a physical thing, which, though not itself an essence, is in turn appropriately related to the essence of this quality, f (102b3–105c7). For instance, a body is hot because of fire--something which always brings heat---, and a body is sick because of fever---something which always bring sickness. 
\end{description}

\subsection*{Argument for Immortality}
\begin{enumerate}
\item Nothing can become its opposite while still being itself: it either flees away or is destroyed at the approach of its opposite.  (For example, the cold cannot become the hot while still being cold). 
\item This is true in a similar way of things that contain opposites.  (For example, fire and snow are not themselves opposites, but fire always brings the hot with it, and cold always brings the cold with it.  So fire will not become cold without ceasing to be fire, nor will snow become hot without ceasing to be snow.) (103c-105b)
\item The soul always brings life with it. (105c-d)
\item Therefore, the soul will never admit the opposite of life, that is death, without ceasing to be soul. (105d-e)
\item But what does not admit death is also indestructible. (105e-106d)
\item  Therefore, the soul is indestructible. (106e-107a)
\end{enumerate}
\end{document}



\end{document}