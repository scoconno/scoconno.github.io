\documentclass[oneside]{article}
 \headheight = 25pt
\footskip = 20pt
\usepackage{mdwlist}
\usepackage[T1]{fontenc}
\renewcommand{\rmdefault}{ppl}
\usepackage{fancyhdr}
 \pagestyle{fancy}
 \lhead{\textbf{\textsc{\small Scott O'Connor\\Ancient Philosophy}}}
 \chead{}
 \rhead{\large\textbf{\textsc{Phaedo 2}}}
 \lfoot{\footnotesize{\thepage}}
 \cfoot{}
 \rfoot{\footnotesize{\today}}
 \usepackage{longtable,booktabs}
\tolerance=700


\begin{document}
\thispagestyle{fancy}
\subsection*{Introduction}
Cebes challenges Socrates to prove that the soul continues to exist on its own after it is separated from the body. In order to do this, Socrates must prove that that the soul continues to exist after a person's death, and also prove that the soul still possesses intelligence.\footnote{Notes and summary of arguments are taken directly from the Internet Encyclopedia of Philosopher. See this link for the full article:  https://www.iep.utm.edu/phaedo/#SH4c} 
\subsection*{Argument for Immortality 1}

%Socrates mentions an ancient theory holding that just as the souls of the dead in the underworld come from those living in this world, the living souls come back from those of the dead (70c-d
\begin{enumerate}
\item
  All things come to be from their opposite states: for example,
  something that comes to be ``larger'' must necessarily have been
  ``smaller'' before (70e-71a).
\item
  Between every pair of opposite states there are two opposite
  processes: for example, between the pair ``smaller'' and ``larger''
  there are the processes ``increase'' and ``decrease'' (71b).
\item
  If the two opposite processes did not balance each other out,
  everything would eventually be in the same state: for example, if
  increase did not balance out decrease, everything would keep becoming
  smaller and smaller (72b).
\item
  Since ``being alive'' and ``being dead'' are opposite states, and
  ``dying'' and ``coming-to-life'' are the two opposite processes
  between these states, coming-to-life must balance out dying (71c-e).
\item
  Therefore, everything that dies must come back to life again (72a).
\end{enumerate}

\subsection*{Argument for Immortality 2}
The theory of recollection: knowledge is recollecting what we once knew and subsequently forgot. The supposed evidence: you can recollect answers to questions that you initially report not knowing the answers to. 


\begin{enumerate}
\def\labelenumi{\arabic{enumi}.}
\item
  Things in the world which appear to be equal in measurement are in
  fact deficient in the equality they possess (74b, d-e).
\item
  Therefore, they are not the same as true equality, that is, ``the
  Equal itself'' (74c).
\item
  When we see the deficiency of the examples of equality, it helps us to
  think of, or ``recollect,'' the Equal itself (74c-d).
\item
  In order to do this, we must have had some prior knowledge of the
  Equal itself (74d-e).
\item
  Since this knowledge does not come from sense-perception, we must have
  acquired it before we acquired sense-perception, that is, before we
  were born (75b ff.).
\item
  Therefore, our souls must have existed before we were born. (76d-e)
\end{enumerate}

\subsection*{Argument for Immortality
3}\label{argument-for-immortality-3}

\begin{enumerate}
\def\labelenumi{\arabic{enumi}.}
\item
  There are two kinds of existences: (a) the visible world that we
  perceive with our senses, which is human, mortal, composite,
  unintelligible, and always changing, and (b) the invisible world of
  Forms that we can access solely with our minds, which is divine,
  deathless, intelligible, non-composite, and always the same (78c-79a,
  80b).
\item
  The soul is more like world (b), whereas the body is more like world
  (a) (79b-e).
\item
  Therefore, supposing it has been freed of bodily influence through
  philosophical training, the soul is most likely to make its way to
  world (b) when the body dies (80d-81a). (If, however, the soul is
  polluted by bodily influence, it likely will stay bound to world (a)
  upon death (81b-82b).)
\end{enumerate}

\subsection*{Argument for Immortality 4}
\begin{enumerate}
\item Nothing can become its opposite while still being itself: it either flees away or is destroyed at the approach of its opposite.  (For example, the cold cannot become the hot while still being cold). 
\item This is true in a similar way of things that contain opposites.  (For example, fire and snow are not themselves opposites, but fire always brings the hot with it, and cold always brings the cold with it.  So fire will not become cold without ceasing to be fire, nor will snow become hot without ceasing to be snow.) (103c-105b)
\item The soul always brings life with it. (105c-d)
\item Therefore, the soul will never admit the opposite of life, that is death, without ceasing to be soul. (105d-e)
\item But what does not admit death is also indestructible. (105e-106d)
\item  Therefore, the soul is indestructible. (106e-107a)
\end{enumerate}
\end{document}