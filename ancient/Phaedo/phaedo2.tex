\documentclass[oneside]{article}
 \headheight = 25pt
\footskip = 20pt
\usepackage{mdwlist}
\usepackage[T1]{fontenc}
\renewcommand{\rmdefault}{ppl}
\usepackage{fancyhdr}
 \pagestyle{fancy}
 \lhead{\textbf{\textsc{\small Scott O'Connor\\Ancient Philosophy}}}
 \chead{}
 \rhead{\large\textbf{\textsc{Phaedo 2}}}
 \lfoot{\footnotesize{\thepage}}
 \cfoot{}
 \rfoot{\footnotesize{\today}}
 \usepackage{longtable,booktabs}
\tolerance=700


\begin{document}
\thispagestyle{fancy}

\subsection*{Introduction}\label{introduction}

\subsection*{Argument for Immortality 1}


\begin{enumerate}
\item
  All things come to be from their opposite states: for example,
  something that comes to be ``larger'' must necessarily have been
  ``smaller'' before (70e-71a).
\item
  Between every pair of opposite states there are two opposite
  processes: for example, between the pair ``smaller'' and ``larger''
  there are the processes ``increase'' and ``decrease'' (71b).
\item
  If the two opposite processes did not balance each other out,
  everything would eventually be in the same state: for example, if
  increase did not balance out decrease, everything would keep becoming
  smaller and smaller (72b).
\item
  Since ``being alive'' and ``being dead'' are opposite states, and
  ``dying'' and ``coming-to-life'' are the two opposite processes
  between these states, coming-to-life must balance out dying (71c-e).
\item
  Therefore, everything that dies must come back to life again (72a).
\end{enumerate}

\subsection*{Argument for Immortality 2}

\begin{enumerate}
\def\labelenumi{\arabic{enumi}.}
\item
  Things in the world which appear to be equal in measurement are in
  fact deficient in the equality they possess (74b, d-e).
\item
  Therefore, they are not the same as true equality, that is, ``the
  Equal itself'' (74c).
\item
  When we see the deficiency of the examples of equality, it helps us to
  think of, or ``recollect,'' the Equal itself (74c-d).
\item
  In order to do this, we must have had some prior knowledge of the
  Equal itself (74d-e).
\item
  Since this knowledge does not come from sense-perception, we must have
  acquired it before we acquired sense-perception, that is, before we
  were born (75b ff.).
\item
  Therefore, our souls must have existed before we were born. (76d-e)
\end{enumerate}

\subsection*{Argument for Immortality
3}\label{argument-for-immortality-3}

\begin{enumerate}
\def\labelenumi{\arabic{enumi}.}
\item
  There are two kinds of existences: (a) the visible world that we
  perceive with our senses, which is human, mortal, composite,
  unintelligible, and always changing, and (b) the invisible world of
  Forms that we can access solely with our minds, which is divine,
  deathless, intelligible, non-composite, and always the same (78c-79a,
  80b).
\item
  The soul is more like world (b), whereas the body is more like world
  (a) (79b-e).
\item
  Therefore, supposing it has been freed of bodily influence through
  philosophical training, the soul is most likely to make its way to
  world (b) when the body dies (80d-81a). (If, however, the soul is
  polluted by bodily influence, it likely will stay bound to world (a)
  upon death (81b-82b).)
\end{enumerate}

\end{document}