\documentclass[oneside]{article}
 \headheight = 25pt
\footskip = 20pt
\usepackage{mdwlist}
\usepackage[T1]{fontenc}
\renewcommand{\rmdefault}{ppl}
\usepackage{fancyhdr}
 \pagestyle{fancy}
 \lhead{\textbf{\textsc{\small Scott O'Connor\\Ancient Philosophy}}}
 \chead{}
 \rhead{\large\textbf{\textsc{Physics 1}}}
 \lfoot{\footnotesize{\thepage}}
 \cfoot{}
 \rfoot{\footnotesize{\today}}
 \usepackage{longtable,booktabs}
\tolerance=700


\begin{document}
\thispagestyle{fancy}

\section*{Aristotle's \emph{Physics}}

\begin{quote}
We think we understand a thing simpliciter . . . whenever we think we are aware both that the explanation because of which the object is is its explanation, and that it is not possible for this to be otherwise (\emph{Posterior Analytics} 70b9).
\end{quote}



\section*{Aristotle's distinction between natural and artificial things (from 2.1)}

\noindent Natural things (e.g. animals, parts of animals, earth, air, fire, water): have within themselves a principle of motion and stability in place, in growth, in decay, or in alteration
\vspace*{2mm}

\noindent Artificial things (e.g. beds, cloaks, artifacts in general) do not \emph{as such} have an internal principle of motion or stability. It's not insofar as something is a bed that it changes in a certain way. But, they do have such principles coincidentally (i.e. insofar as they are made of simple natural bodies)
\vspace*{2mm}

\noindent In general, A accepts a version of \emph{hylomporhpism}, a metaphysical view on which (almost) all existing things are composites of matter and form
\begin{itemize}\item{Matter: the underlying material (e.g. the bronze of a bronze statue, the flesh and bones of a human being)}\item{Form: the shape or, for more complex things, set of capacities that jointly constitute what it is to be something of a certain kind (e.g. the shape having which makes a piece of bronze a statue; the capacities and dispositions that jointly constitute being a human being)}\end{itemize}

\section*{The four \emph{aitiai} (``causes,'' ``explanations,'' ``becauses'')}

\noindent Since having \emph{epist\^{e}m\^{e}} requires grasping \emph{aitiai}, A sets out (in Ch. 3) to determine how many and what kinds of \emph{aitiai} there are
\vspace*{2mm}

\noindent A indicates in some of his examples that these are 4 kinds of entities that can be cited in response to various questions of the form ``Why is \emph{X} F?''
\vspace*{2mm}

\noindent [1] ``That from which, as present in it, a thing comes to be'' (often called the ``material cause'')
\vspace*{1mm}

E.g.: the bronze and silver, and their genera, are \emph{aitiai} of the statue and the bowl; letters of syllables; matter\\\hspace*{6mm}of artifacts; fire and such things of bodies; parts of wholes; premises of conclusion
\vspace*{2mm}

\noindent [2] ``The form (i.e. the pattern)... The form is the account (and the genera of the account) of the essence, and\\\hspace*{6mm}the parts in that account'' (often called the ``formal cause'')
\vspace*{1mm}

E.g.: the \emph{aitia} of an octave is the ratio two to one, and in general number; the whole, the composition
\vspace*{2mm}

\noindent [3] ``Source of the primary principle of change or stability'' (often called the ``efficient cause'')
\vspace*{1mm}

E.g.: adviser is an \emph{aitia}, father is an \emph{aitia} of his child; the producer is a \emph{aitia} of the product; the initiator\\\hspace*{6mm}of a change is an \emph{aitia} of what is changed; the seed; the doctor
\vspace*{2mm}

\noindent [4] ``Something's end, i.e. what it is for''
\vspace*{1mm}

E.g.: health of walking; health of purging, slimming, drugs, instruments; the good of the thing
\vspace*{2mm}

\noindent In Chs. 8 and 9 we learn that A thinks final \emph{aitiai} are relevant not only in the explanation of human behavior (e.g. Bob runs for the sake of health) but in nature as a whole

\noindent The things that occur in nature always or for the most part happen for the sake of ends (e.g. organisms develop the parts they do in the way that they do for the good of the organism: in plants, leaves grow \emph{for the sake of} protecting the fruit; hearts contract for the sake of pumping blood (which, in turn, is done for the sake of maintaining the organism); rain falls in the patterns it does for the sake of nourishing the soil)
\vspace*{1mm}

\noindent This (highly controversial) view allows A to explain the development of organisms and why the parts of organisms are the way that they are by showing how it is good or best for the parts to be that way

\section*{\emph{Aitia} problem set}

\noindent [1] Why is the statue shiny? (what kind(s) of \emph{aitia(i)} would you cite to answer this question and what is a possible example(s))
\vspace*{3mm}

\noindent [2] Why is the bowl falling? (same as [1])
\vspace*{3mm}

\noindent [3] Sally's decision to go to the mall is the\hspace*{25mm} \emph{aitia} of Sally's being in her car driving to the mall
\vspace*{3mm}

\noindent [4] Why are the organism's front teeth becoming sharp and back teeth becoming flat? (same as [1] and [2]) 
\vspace*{3mm}

\noindent [5] What is the \emph{aitia} of the fact that a triangle's internal angles sum to $180\,^{\circ}$
\vspace*{3mm}

\noindent [6] What are the 4 \emph{aitiai} of Socrates' birth?

\section*{Luck and Chance}

\noindent Luck and Chance seem to present a problem: we certainly say that things happen ``by chance'' or ``because of luck'' (etc.), but, for each thing that we describe this way, we can also specify some definite thing that led to the ``chance'' event happening.
\vspace*{2mm}

E.g.: If you go to the marketplace to buy some fruit, and run into someone you did not expect to meet\\\hspace*{6mm}that owes you money, we say it was by luck or by chance that you collected the money. But, we can also\\\hspace*{6mm}answer the question ``how did you collect the money'' by saying ``by going to the marketplace,'' and it\\\hspace*{6mm}wasn't by chance that you went to the marketplace.
\vspace{2mm}

\noindent Negatively: luck and chance are not \emph{aitiai} of things that happen always or usually in a certain way
\vspace*{2mm}

\noindent Positively: the kind of events that can happen because of luck and chance are those that \emph{can}  be result from something done for the sake of it but happen coincidentally
\vspace*{2mm}

So, go back to the marketplace case. You could have gone to the market in order to collect your debt. But,\\\hspace*{6mm}you went in order to buy fruit. Your collecting the debt was brought about coincidentally by your going\\\hspace*{6mm}to the marketplace (i.e. it just so happened that by doing that, you collected your debt).
\vspace*{2mm}

\noindent Chance is actually the wider notion: luck only concerns those things that could result from decision or thought (i.e. roughly, the affairs of adult humans); whereas chance includes those and things including inanimate and non-adult human animals

\end{document}
