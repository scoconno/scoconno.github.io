\documentclass[oneside]{article}
 \headheight = 25pt
\footskip = 20pt
\usepackage{mdwlist}
\usepackage[T1]{fontenc}
\renewcommand{\rmdefault}{ppl}
\usepackage{fancyhdr}
 \pagestyle{fancy}
 \lhead{\textbf{\textsc{\small Scott O'Connor\\Ancient Philosophy}}}
 \chead{}
 \rhead{\large\textbf{\textsc{Physics 1}}}
 \lfoot{\footnotesize{\thepage}}
 \cfoot{}
 \rfoot{\footnotesize{\today}}
 \usepackage{longtable,booktabs}
\tolerance=700


\begin{document}
\thispagestyle{fancy}



\section*{Natural vs artificial things)}

\noindent Natural things (e.g. animals, parts of animals, earth, air, fire, water): have within themselves a principle of motion and stability in place, in growth, in decay, or in alteration
\vspace*{2mm}

\noindent Artificial things (e.g. beds, cloaks, artifacts in general) do not \emph{as such} have an internal principle of motion or stability. It's not insofar as something is a bed that it changes in a certain way. But, they do have such principles coincidentally (i.e. insofar as they are made of simple natural bodies)
\vspace*{2mm}

\noindent In general, A accepts a version of \emph{hylomporhpism}, a metaphysical view on which (almost) all existing things are composites of matter and form
\begin{itemize}\item{Matter: the underlying material (e.g. the bronze of a bronze statue, the flesh and bones of a human being)}\item{Form: the shape or, for more complex things, set of capacities that jointly constitute what it is to be something of a certain kind (e.g. the shape having which makes a piece of bronze a statue; the capacities and dispositions that jointly constitute being a human being)}\end{itemize}

\section*{The four \emph{aitiai} (``causes,'' ``explanations,'' ``becauses'')}


\begin{quote}
We think we understand a thing simpliciter . . . whenever we think we are aware both that the explanation because of which the object is is its explanation, and that it is not possible for this to be otherwise (\emph{Posterior Analytics} 70b9).
\end{quote}
A repeats this claim that proper knowledge is knowledge of the cause is repeated in the \emph{Physics:} we think we do not have knowledge of a thing until we have grasped its why, that is to say, its cause (194b17--20). Since having knowledge requires grasping causes, A sets out to determine how many and what kinds of causes there are.
He says that there a four kinds of answers that can be cited in response to various questions of the form ``Why is \emph{X} F?'' He ultimately thinks matter provides the first answer and form provides the the remaining three.

\begin{description}
\item[The Material Cause:] ``That from which, as present in it, a thing comes to be''. E.g.: the bronze and silver are material causes of the statue and the bowl. So too are letters of syllables, matterof artifacts, fire and such things of bodies, parts of wholes, and premises of a conclusion.
\item[The Formal Cause:] ``The form (i.e. the pattern)... The form is the account (and the genera of the account) of the essence, and\ parts in that account''. E.g.: the  formal cause of an octave is the ratio two to one, and in general number; the whole, the composition.
\item[The Efficient Cause:] ``Source of the primary principle of change or stability''. E.g.: adviser is a efficient cause. So too is a father of his child (A. believes the mother contributes the material cause), the producer of the product, the initiator of a change of what is changed, the doctor of the healing, etc.

\item [The Final Cause:] ``Something's end, i.e. what it is for.'' E.g.: health of walking; health of purging, slimming, drugs, instruments; the good of the thing.
\begin{itemize}
\item  In Chs. 8 and 9 we learn that A thinks final causes are relevant not only in the explanation of human behavior (e.g. Bob runs for the sake of health) but in nature as a whole.
\item The things that occur in nature always or for the most part happen for the sake of ends (e.g. organisms develop the parts they do in the way that they do for the good of the organism: in plants, leaves grow \emph{for the sake of} protecting the fruit; hearts contract for the sake of pumping blood (which, in turn, is done for the sake of maintaining the organism); rain falls in the patterns it does for the sake of nourishing the soil).
\item This (highly controversial) view allows A to explain the development of organisms and why the parts of organisms are the way that they are by showing how it is good or best for the parts to be that way.
\end{itemize}
\end{description}


\section*{\emph{Aitia} problem set}

\noindent [1] Why is the statue shiny? (what kind(s) of causes would you cite to answer this question and what is a possible example(s))
\vspace*{3mm}

\noindent [2] Why is the bowl falling? (same as [1])
\vspace*{3mm}

\noindent [3] Sally's decision to go to the mall is a cause of Sally's being in her car driving to the mall. What kind of cause? 
\vspace*{3mm}

\noindent [4] Why are the organism's front teeth becoming sharp and back teeth becoming flat? 
\vspace*{3mm}

\noindent [5] What is the  cause of the fact that a triangle's internal angles sum to $180\,^{\circ}$
\vspace*{3mm}

\noindent [6] What are the 4 causes of Socrates' birth?

\section*{Luck and Chance}

\noindent Luck and Chance seem to present a problem: we certainly say that things happen ``by chance'' or ``because of luck'' (etc.), but, for each thing that we describe this way, we can also specify some definite thing that led to the ``chance'' event happening.
\vspace*{2mm}

E.g.: If you go to the marketplace to buy some fruit, and run into someone you did not expect to meet that owes you money, we say it was by luck or by chance that you collected the money. But, we can also answer the question ``how did you collect the money'' by saying ``by going to the marketplace,'' and it wasn't by chance that you went to the marketplace.
\vspace{2mm}

\noindent Negatively: luck and chance are not causes of things that happen always or usually in a certain way
\vspace*{2mm}

\noindent Positively: the kind of events that can happen because of luck and chance are those that \emph{can}  result from something done for the sake of it but happen coincidentally
\vspace*{2mm}

So, go back to the marketplace case. You could have gone to the market in order to collect your debt. But,  you went in order to buy fruit. Your collecting the debt was brought about coincidentally by your going to the marketplace (i.e. it just so happened that by doing that, you collected your debt).

%\noindent Chance is actually the wider notion: luck only concerns those things that could result from decision or thought (i.e. roughly, the affairs of adult humans); whereas chance includes those and things including inanimate and non-adult human animals

\end{document}
