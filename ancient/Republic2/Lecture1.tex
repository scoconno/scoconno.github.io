\documentclass[oneside]{article}
\usepackage{graphicx}
 \headheight = 25pt
\footskip = 20pt
\usepackage{mdwlist}
\usepackage[T1]{fontenc}
\renewcommand{\rmdefault}{ppl}
\usepackage{fancyhdr}
 \pagestyle{fancy}
 \lhead{\textbf{\textsc{\small Scott O'Connor\\Ancient Philosophy}}}
 \chead{}
 \rhead{\large\textbf{\textsc{Republic 4}}}
 \lfoot{}
 \cfoot{}
 \rfoot{}
 \usepackage{longtable,booktabs}
\tolerance=700


\begin{document}
\thispagestyle{fancy}

\subsection*{Why be just?}
We have defined justice as a certain harmony between the parts of the soul that exist when each parts does its own work. But we have yet to show that justice is preferable to injustice, that being just is beneficial in itself. We will return to the topic in greater detail in Book 8, but Socrates does offer a quick argument for the claim in Book 4. Let us call it the \emph{Psychological Health Argument:}

\begin{enumerate}
\item Justice is harmony between the three parts of the soul where each does its appropriate task.
\item Injustice is strife between the three parts.
\item	Justice is like health, since health too involves the proper ordering of parts.  
\begin{quote}
To produce health is to establish the components of the body in a natural relation of control and being controlled, one by another, while to produce disease is to establish a relation of ruling and being ruled contrary to nature...Then...isn`t to produce justice to establish the parts of the soul in a natural relation of control, one by another, while to produce injustice is to establish a relation of ruling and being ruled contrary to nature? (444d)
\end{quote}
\item  Health is preferable to disease.  (Recall that health is good in itself and for its consequences.)
\item[C.]	Therefore, justice is preferable to injustice.
\end{enumerate}

\subsection*{Why does reason rule in the just soul?}
Socrates attempts to compare a healthy body with a just soul. In order to have a healthy body, I need to exercise, eat well, avoid harmful substances, etc. This all takes effort; I must master my body if I am to ensure its health. If my body becomes unwell, it will master me. Likewise, Socrates thinks that in the just soul reason orders and rules over the other parts; in the soul of a just person, the rational part does its own job of ruling over the other parts of the soul. Such a person is wise by virtue of their rational part ruling in accordance with its knowledge of what is best for the whole soul (442c).   But there is risk of the analogy breaking down. I know that kale is healthy for my body and that eating too much sugar is unhealthy for it. For his analogy to work, not only must there be facts about what is good for us to desire, reason must be able to grasp these facts and order the others parts to desire accordingly. 

This means, for instance, that there are objective facts about  what food you should desire, which person you should love, what instances warrant anger. It also requires that the just soul will desire in accordance with what reason judges best. Reason does not merely say, 'I notice that you are attracted to that person, but they are not good for you, so do not ask them out.' The rational part of the just soul will ensure that the appetitive part is only attracted to who reason knows is, in fact, the right person to be attracted to. 

This all creates a problem.  We might agree with Socrates that reason loves wisdom and learning (581c). We might also agree that it reasons about the better and the worse. And we might even agree that it wants to look out for the good of the whole soul. But why should we agree that it can succeed in any of this? I may want to be an astronaut. It doesn't follow that I can be one. Similarly, why assume that there are facts about what's good for the soul that reason can acquire? And why assume that reason can change the desires we have? Socrates' answer has two parts. 

\subsubsection*{The Forms}

Socrates draws a sharp ontological distinction between two kinds of entities, perceptible objects and intelligible objects:

\begin{description}
\item[Perceptible objects:] Entities about which we can gain information \emph{directly} through the five senses. It also includes groups of such objects.
\begin{itemize}\item{Examples: Socrates, Socrates' dog Fido, this building, each person in this room}\item{Also: The people in this room (we do not see this entity directly, rather we gain information about it by seeing (hearing, touching, etc.) each individual person in this room}\end{itemize}

\item[Intelligible objects:] Entities about which we gain information \emph{solely} through the activity of thought (\textbf{NB}: The ``solely'' is crucial here, since we can think about and, hence, gain additional information about perceptible objects. The point is that, in addition to being able to think about perceptible objects, we can \emph{also} perceive them through the senses.)

\begin{itemize}\item{Examples: Mathematical objects (squares, triangles, the number 2), Natures or Essences, which Socrates also calls ``Forms'' (\emph{eid\^{e}}) or ``Ideas'' (\emph{ideai}) (e.g. the nature of Piety, the nature of Justice, the Nature of Goodness)} %\item{This is a development from the conception of Natures or Essences we found in the Socratic dialogues (e.g. \emph{Euthyphro}, \emph{Protagoras}, \emph{Meno}). In those dialogues there was no indication that Natures or Essences existed separately from the perceptible world.}
\end{itemize}

\end{description}
Perceptible objects do not perfectly instantiate intelligible objects but, rather, ``approximate,'' ``resemble,'' or ``participate in'' them. Socrates says that perceptible objects are ``images'' or ``imitations'' of intelligible objects. Consider the following two figures:
\newpage
\begin{figure}[h!]
\hspace*{35mm}
\includegraphics[scale=0.6]{figure1}
\end{figure}

Figure 1 is not a perfect square. A perfect square is two-dimensional and is thus not perceptible.  Nevertheless, Figure 1 more closely approximates the nature of a perfect square than Figure 2 does. The person who knows what a perfect square will reliably judge that perceptible Figure 1 approximates a perfect square more than perceptible Figure 2 does. Similarly, Socrates claims that, while nothing in the perceptible world perfectly instantiates the nature of justice, certain things in the perceptible world like distributions of resources, social institutions, etc., can more closely approximate the nature of justice than others and, hence, can be ``more just'' than others. If we agree with Socrates in the reality of such forms, then we might agree that there are objective facts about how a city should be arranged and about what a person should desire. If the ruler could grasp this form, he or she would reliably judge which policies or actions most approximate justice. Likewise, if reason could grasp this form, it would reliably judge what its appetitive and spiritive correlates should desire. 



\subsubsection*{Knowledge of forms }

At 471 Glaucon questions Socrates whether a city with the constitution they laid out in Books 2--4 can be realized on earth. Socrates maintains that the way to bring about a city \emph{closest to} the ideal city is to vest political power in the hands of philosophers. Philosophers are distinguished from non-philosophers insofar as only philosophers can attain understanding (\emph{epist\^{e}m\^{e}}) and knowledge (\emph{gn\^{o}sis}) of intelligible objects, which makes them epistemic authorities about matters in the perceptible world (think of a doctor's expert ``medical opinion'' that a particular patient should undergo a particular course of treatment). 

So, philosopher rulers know the various Forms and use that knowledge as a model for deciding which policies are best for the city as a whole. These include those that promote harmony in the city and avoid strife. Likewise, the philosopher uses their knowledge of the Forms to decide which appetitive desires to have and act upon, and also which emotions to have and act upon. These include those that promote harmony in the soul and avoid strife. 

In order to decide how reason could grasp the forms, Socrates turns to discussing how it might come about that philosophers rule the city. For instance, what kind of education system must be in place? How should property be distributed? What kind of music should children listen to?  The \emph{Republic} subsequently makes a significant transgression into areas that may seem far removed from the original inquiry into the just soul. Amongst Socrates more shocking claims are that private property should be abolished, a system wide eugenics program should be implemented, children should be held in common, and artists should be expelled from the city. We will not focus on these parts of the \emph{Republic} apart from observing that they all focus on the creation of a just city and philosopher rulers. 

All we need observe here is that Socrates answers the question of how reason might come to know the forms by also focusing on education, a many decades long process that requires intense theoretical and physical training. The nature and effect of that education is illustrated in \emph{Book 7} through the \emph{Allegory of the Cave}. I have included links on the course website to some animated versions of that allegory. As you watch it, ask how education might effect a radical change in people that could ultimately lead them to understanding the forms, the most fundamental of which is the form of the good.


\end{document}
