\documentclass[oneside]{article}
 \headheight = 25pt
\footskip = 20pt
\usepackage{mdwlist}
\usepackage[T1]{fontenc}
\renewcommand{\rmdefault}{ppl}
\usepackage{fancyhdr}
 \pagestyle{fancy}
 \lhead{\textbf{\textsc{\small Scott O'Connor\\Ancient Philosophy}}}
 \chead{}
 \rhead{\large\textbf{\textsc{Republic 3}}}
 \lfoot{\footnotesize{\thepage}}
 \cfoot{}
 \rfoot{\footnotesize{\today}}
 \usepackage{longtable,booktabs}
\tolerance=700


\begin{document}
\thispagestyle{fancy}

\subsection*{Introduction}

Recall that Socrates is building a just city as a model for a just soul. We first identify what a just city is and whether it is better off than an unjust city. We then use that as a model to help identify the just soul and whether it is better off than the unjust soul.  In building the model, we learned that the just city has three ``kinds'' or ``classes'' of people in it: workers (money-lovers), guardians (honor-lovers), and rulers (wisdom-lovers). Each has a distinct function of producing, guarding, and ruling respectively. We also saw that some of the city's virtues are ``located'' in its parts. \emph{Wisdom} is located in its rulers and \emph{courage} in its guardians. Other virtues consist in a certain relation ``between'' its parts.  \emph{Temperance} is a certain concord between workers and rulers. \emph{Justice} consists in \emph{each part }performing its proper function. If the city is a good model for the soul, the soul should have three ``parts'' with distinct functions, and we should be able to locate the virtues in these parts, either by themselves on in relation to one another. This is what Socrates now turns to do. 


\subsection*{The three kinds of soul faculties}
Socrates will first argue that there are three parts of the soul that correspond to the three parts of the city. By `part' it is perhaps best to take him to mean something like psychological faculty. They are as follows:  
\begin{description}
\item[Appetite:] Desires for food, drink, sex, etc. These are ``physiological'' desires (439). The appetitive part corresponds to the workers in the city. Their love of money corresponds to appetite's love of food, drink, and sex. 
\item[Spirit:] Emotions of anger, self-disgust, shame and desires for honor or respect (440-41). This part corresponds to the guardians in the city. The guardians love of honor is closely parallel to spirit's attachment to anger, self-disgust, shame, etc. 
\item[Reason:] Rational desires are for the overall good or good of the whole (441e). This part corresponds to the rulers in the city. Both desire knowledge and to rule with knowledge about the good of who/what they rule. 
\end{description}
Why think that our souls are divided neatly into these three parts? Socrates's strategy focuses on the phenomenon on the divided mind, on the fact that we are often torn over what we want, what to do, and what to think. He thinks that such divisions give evidence of distinct soul parts: 

\begin{quote}
Either we always ``go for'' things with part of the soul \emph{or} the whole soul---e.g. we learn with one part of it, get angry with another, and desire pleasures of food with a third (436a). 
\end{quote}
Socrates argue that it must be the former, i.e., that each of these faculties must reside in its own separate part of the soul. He argues for these various parts in turn. 

\subsection*{Argument for the appetitive part}
Before I outline the argument, let us consider a simple example to illustrate the phenomenon the argument turns on. Many enjoy alcohol. Many also recognize that they shouldn't drink, either at all or on a particular occasion. Susan, let's say, is offered a chilled glass of Sancerre with her sea bass. She really would enjoy that wine. She really wants that wine! But Susan must drive after dinner so also doesn't want to drink that wine. So, she both wants and doesn't want the wine. Socrates thinks that no single part of her soul could simultaneously want and not want the wine. So, he concludes that there must be two parts of the soul, one that wants it and another that doesn't. Here is the argument more formally:

\begin{enumerate}
\item [P1] \textbf{Principle of Opposites}: A thing cannot undergo opposites in the same part of itself, in relation to the same thing, at the same time (436b-37a). For instance, my son cannot grow and shrink in the same part, at the same time, in the same respect, etc. Obviously, his hair could shrink as his torso lengthens, but this involves one part growing and another shrinking (when he gets a haircut).
\item[P2] Going (assent, wishing) for X and rejecting (dissent, not wishing for) X, are opposites. Thus, you cannot both assent and dissent for X with the same part of yourself, in the same way, at the same time, etc. Same too with wishing and not wishing for something. 
\item[P3] The desire for a drink is an unqualified desire. It is not the same as the qualified desire for a good drink. There is a difference between wanting any drink whatsoever and the desire to drink something healthy. 
\begin{enumerate}
\item[P3.i] Thirst \emph{as such} is a desire for drink \emph{as such} (437b-39a). It is not a desire for orange juice as opposed to water. It is just a desire for whatever will quench  thirst. Likewise, hunger \emph{as such} is a desire for food \emph{as such}. It is not a desire for chocolate as opposed to fruit. It is a desire for whatever will alleviate hunger. 
\item[P3.ii] Unqualified desires are for unqualified objects
\end{enumerate}
\item[P4] Sometimes we have a desire for drink, but choose not to drink.
\item[P5] Having this desire to drink and not wanting to drink are opposites.
\item[C1] So there are two ``things'' in the person involved in this event---one thing is the proper subject of the desire to drink, a \emph{distinct} thing is the proper subject of the rejection of drink. The former is a motivation generated by appetite, the latter by reason.
\item[C2] So there are at least two parts of the soul, the rational part and the appetitive part. 
\end{enumerate}

\subsection*{Arguments for the spirited part}
Socrates's first argument shows that there must be at least two parts of the soul, the rational part and the appetitive part. He now needs to show that there is a third part of the soul---the spirited part. This is the hardest part of the argument. 


\begin{enumerate}
\item[1] Spirit is distinct from appetite (439e-40)
\begin{itemize}
\item Evidence: Leontius---wants to look at corpses and does, but is angry with himself.
\end{itemize}
\item[2] Spirit is distinct from reason (440e-41)
\begin{itemize}
\item Evidence 1: Children and animals do not have reason but do act contrary to their appetites.
\item Evidence 2: Odysseus---wants to take vengeance on his unfaithful servants, but is restrained by reason.
\end{itemize}
\item[3] Spirit is ``allied'' with reason (440)
\begin{itemize}
\item Evidence: The cases of a good person undergoing just and unjust punishment.
\end{itemize}
\item[C1] Spirit is distinct from both appetite and reason.
\item[C2] There are at least three parts of the soul.
\end{enumerate}

%\subsection*{\emph{Akrasia} (lack of self-control) on this theory}

%S here seems to allow that lack of self-control is possible. Such a phenomenon occurs when  X does B, i) \emph{thinking} B is bad, ii) when able not to do B, and iii) overwhelmed by pleasure. S maintains that in such a case the judgment of the rational part of the soul (i.e. a rational desire) is overcome by the non-rational appetitive desire of the soul.

%\noindent Question: On the \emph{Republic}'s theory, is \emph{akrasia} possible when X knows that B is bad, or only when X \emph{merely believes} that B is bad?

\subsection*{Justice}
Socrates has now shown that the soul is divided in three parts in an analogous fashion to how the city is so divided. He now returns to the question of what a just soul consists in. Again, his strategy is to use the analogy with the just city: just as justice in the city consists in the harmonious function of the three classes, justice in the soul consists in the harmonious functioning of the three parts of the soul, with each part fulfilling its proper ``function'' or ``work'': 

\begin{itemize}
\item{Reason rules, making judgments about the overall good for the person.}
\item{Spirit and appetite ``obey'' reason in the sense that they only desire things that, in fact, accord with reason's determinations about what is good overall.}
\begin{itemize}
\item{Spirit and appetite, however, \emph{do not} do this because they can judge what is best, or because they can judge that reason ``knows best,'' or because they can judge anything at all (they are \emph{non-rational}---they cannot grasp reasons).}
\item{They do it because they have been trained or conditioned only to generate such desires.}
\end{itemize}\end{itemize}

\noindent So, only the just person's soul is genuinely unified, with no internal conflicts. This suggests that there is some benefit to being just, but Socrates must still show that the having a just soul is always better than having an unjust soul. 

%\section*{Problems}
%\begin{enumerate}
%\item [P3] makes a logical point about qualified and unqualified things: how could this show that we actually have non-rational desires?
%\item It may be possible to re-describe [P4] as conflict between non-simultaneous rational beliefs
%\item Does [P1] entail indefinite partition of the soul if two desires of a single soul part can conflict with each other?
%\item What does justice, in the sense described above, have to do with justice as G \& A were discussing it at the beginning of Book II? Has S changed the subject?
%\end{enumerate}

\end{document}
