% !TEX encoding = UTF-8 Unicode
% !TEX TS-program = xelatex

\documentclass[11pt]{article}
\usepackage{fontspec}
\defaultfontfeatures{Mapping=tex-text}
\usepackage{xunicode}
\usepackage{xltxtra}
\usepackage{verbatim}
\usepackage[margin= 1 in]{geometry} % see geometry.pdf on how to lay out the page. There's lots.
\geometry{letterpaper} % or letter or a5paper or ... etc
%\usepackage[parfill]{parskip}    % Activate to begin paragraphs with an empty line rather than an indent 
\usepackage{mathrsfs}
\usepackage{bbding}
\usepackage[usenames,dvipsnames]{color}
\usepackage{natbib}
\usepackage{stmaryrd}
%\usepackage{mathpartir}
\usepackage{txfonts}
\usepackage{graphicx}
\usepackage{fullpage}
\usepackage{hyperref}
\usepackage{amssymb}
\usepackage{epstopdf}
\usepackage{fontspec}
%\setmainfont{Hoefler Text}
\setmainfont[BoldFont={Minion Pro Bold}]{Minion Pro}
\usepackage{hyperref}
\usepackage{lastpage, fancyhdr}
%\usepackage{setspace}
\pagestyle{fancy}
\lhead{}
\chead{Epicurus 2} 
\rhead{}
\lfoot{}
\cfoot{\thepage\space of \pageref{LastPage}} 
\rfoot{}
\footskip=30 pt
\headsep=20pt
\thispagestyle{empty}
\hypersetup{colorlinks=true, linkcolor=Sepia, urlcolor=Sepia, citecolor=BrickRed}
\DeclareGraphicsRule{.tif}{png}{.png}{`convert #1 `dirname #1`/`basename #1 .tif`.png}

\usepackage{covington}
\usepackage{fixltx2e}
\usepackage{graphicx}
\begin{document}

%\maketitle
\thispagestyle{empty}
\begin{center} \LARGE{Ancient Philosophy\\ Epicurus 2}\\ \vspace*{2mm}
\large{Scott O'Connor}\end{center}
\thispagestyle{empty}\vspace*{3mm}

\section*{Epicurus' main argument}

\begin{enumerate}
\item Nothing is good or bad for one except sense experience, i.e.~feelings of pleasure and pain. 
\item The dead don't have any sense experiences. 
\item Therefore, nothing is good or bad for the one who is dead. 
\item Therefore, the state of being dead is not (good or) bad for the one who is dead. 
\item If X is not bad for one when it is present, then there is no rational
ground, before it is present, to fear its future presence. 
\item Therefore, no living person has any rational ground to fear his future state of being dead.
\end{enumerate}

We have already discussed the first premise:  E claims that ``Pleasure is the starting point and goal of living blessedly'' (\emph{LM} 128)

\section*{Premise 2}
\begin{description}
\item[P2] relies on E's doctrine of the soul. 
\begin{itemize}
\item E is an atomist: all reality consists in atoms (i.e. indivisible, indestructible, units of matter) and void.
\item Even the soul, according to E, is composed of atoms (i.e. the soul is a body---quite different from Plato and Aristotle), which ``dissipate'' upon death
\end{itemize}
\end{description}

\section*{Notes}
If 1 and 2 are too controversial, maybe we can substitute less controversial variants:

\begin{enumerate}
\item[1*] Good and bad depend on there being a subject who could experience them
\item[2*] Death is the extinction of the `self' or `person' --- i.e. of the subject capable of experience
\end{enumerate}
Even if we still doubt these premises, Epicurus has raised three pressing problems: 
\begin{enumerate}
\item[A] How can something be bad for \emph{S} if \emph{S} does not or \emph{cannot} mind or care one way or the other, since \emph{S} is non-existent?
\item[B] Who could be the possessor or subject of this bad once \emph{S} is non-existent?
\item[C] When could the subject suffer this bad?
\end{enumerate}

\section*{Responses}

\noindent \textbf{Nagel}
\vspace*{2mm}

\noindent [1] Death is bad because it involves the deprivation of goods---e.g. perception, thought, emotion
\vspace*{1mm}

\noindent [2] Goods and bads for someone do not depend on that person's awareness of them (cf. \emph{EN} 1.10)
\vspace*{1mm}

\noindent E.g.: Suppose we all have significant others who, while we are here, get together for swinging affairs; suppose that part of our well-being stems from the (perhaps unconscious) faith in our SO's fidelity; suppose further that none of us ever find out about it and that, if anything, the only consequences we experience are in a sense beneficial (e.g. our SO's are nicer, kinder, etc. to us as a result); it still seems like this is bad \emph{for us}
\vspace*{2mm}

\noindent [3] The person who is deprived of goods by death is a ``possible person''---i.e. the person who was alive, but so understood as to include the (unrealised) possibilities of her continued life
\vspace*{1mm}

\noindent E.g.: An accident victim suffers head-injuries. Her IQ drops to 20, but she is ``happy'' or, at least, ``cheerful.'' We tend to think that the ``person'' is unfortunate; but the current person is quite ``happy.'' So we must be ascribing the misfortune to the person-she-could-have-been = a ``possible person''
\vspace*{2mm}

\noindent So, Nagel rejects both [P1] and [P1*]
\vspace*{5mm}

\noindent \textbf{Furley}
\vspace*{2mm}

\noindent [1] Death is bad because it involves the frustration of our previous plans, hopes and desires
\vspace*{1mm}

\noindent [2] The frustration of our current plans etc. would make our present actions pointless
\vspace*{1mm}

\noindent [3] Hence it is rational to fear death, since it is rational to fear that our current actions are pointless
\vspace*{1mm}

\noindent E.g.: a terminally-ill person is deceived about her condition; her concern with her plans for a holiday next spring is pointless---and she would think so too if she knew about her condition
\vspace*{2mm}

\noindent Furley rejects [P5]
\vspace*{5mm}

\noindent \textbf{Epicurus' response to Furley}
\vspace*{2mm}

\noindent \emph{PD}III
\vspace*{1mm}

\noindent [1] Happiness requires only the satisfaction of our natural and necessary desires
\vspace*{1mm}

\noindent [2] These desires can be satisfied by a self-sufficient life---i.e. one which does not involve long-term projects
\vspace*{1mm}

\noindent [3] So you don't understand what happiness is if you think that it involves long-term projects, etc.
\vspace*{1mm}

\noindent [4] Hence your fear of death is ``empty''---i.e. rests on a false belief---and thus irrational


\end{document}
