% !TEX encoding = UTF-8 Unicode
% !TEX TS-program = xelatex

\documentclass[11pt]{article}
\usepackage{fontspec}
\defaultfontfeatures{Mapping=tex-text}
\usepackage{xunicode}
\usepackage{xltxtra}
\usepackage{verbatim}
\usepackage[margin= 1 in]{geometry} % see geometry.pdf on how to lay out the page. There's lots.
\geometry{letterpaper} % or letter or a5paper or ... etc
%\usepackage[parfill]{parskip}    % Activate to begin paragraphs with an empty line rather than an indent 
\usepackage{mathrsfs}
\usepackage{bbding}
\usepackage[usenames,dvipsnames]{color}
\usepackage{natbib}
\usepackage{stmaryrd}
%\usepackage{mathpartir}
\usepackage{txfonts}
\usepackage{graphicx}
\usepackage{fullpage}
\usepackage{hyperref}
\usepackage{amssymb}
\usepackage{epstopdf}
\usepackage{fontspec}
%\setmainfont{Hoefler Text}
\setmainfont[BoldFont={Minion Pro Bold}]{Minion Pro}
\usepackage{hyperref}
\usepackage{lastpage, fancyhdr}
%\usepackage{setspace}
\pagestyle{fancy}
\lhead{}
\chead{Lecture 14, Aristotle \emph{NE}, Book I\space---\space Handout} 
\rhead{}
\lfoot{}
\cfoot{\thepage\space of \pageref{LastPage}} 
\rfoot{}
\footskip=30 pt
\headsep=20pt
\thispagestyle{empty}
\hypersetup{colorlinks=true, linkcolor=Sepia, urlcolor=Sepia, citecolor=BrickRed}
\DeclareGraphicsRule{.tif}{png}{.png}{`convert #1 `dirname #1`/`basename #1 .tif`.png}
\usepackage{polyglossia}
\setdefaultlanguage{english}
\setotherlanguage{greek}
\newfontfamily\greekfont{Gentium Plus}
\newcommand{\gk}[1]{\textgreek{#1}}
\newcommand{\gloss}[1]{(\textgreek{#1})}

\usepackage{covington}
\usepackage{fixltx2e}
\usepackage{graphicx}
\begin{document}

%\maketitle
\thispagestyle{empty}
\begin{center} \LARGE{PHIL 321\\ Lecture 14: Aristotle's \emph{Nicomachean Ethics}, Book I}\\ \vspace*{2mm}
\large{10/15/2013}\end{center}
\thispagestyle{empty}\vspace*{3mm}
\vspace*{-8mm}

\section*{The highest good (chs. 1-5)}

\noindent [1] Every craft (\emph{techn\^{e}}), investigation (\emph{methodos}), action (\emph{praxis}), and decision (\emph{prohairesis}) aims at some good (in what follows we will focus on actions)
\vspace*{2mm}

\noindent [2] Thus, the good (\emph{to agathon}) is that at which everything aims

\begin{itemize}\item{Is this an invalid step?}\end{itemize}

\noindent [3] Ends (``\emph{telos},'' i.e. the goods aimed for in purposeful action) differ in being
\begin{itemize}\item{[2a] actions or activities themselves, or}\item{[2b] products that result from the actions or activities aim at (producing) them}\end{itemize}

\noindent [4] Ends may be \emph{hierarchically structured}: one end (e.g. bridles) can be pursued for the sake of another (e.g. horse-riding)
\begin{itemize}\item{In such a case, the superordinate end is more \emph{choiceworthy} than the subordinate end}\item{The superordinate end also sets the conditions for when the subordinate end is accomplished well (e.g. the value of a bridle \emph{qua} bridle is determined by the ``needs'' of horse-riding)}\end{itemize}

\noindent [5] If there is some end we pursue \emph{only} for itself and for which we pursue all our other ends, this end with be the \emph{highest} good (or, simply, \emph{the} good) 

\vspace*{2mm}

\noindent [6] There must be some limit to the ends we choose for the sake of something else, otherwise our desires would be ``empty and futile'' (1094a21-2)
\begin{itemize}\item{[P1] If there were no highest good, then our desires would be empty}\item{\underline{[P2] But, our desires are not empty}}\item{[C] There is a highest good}\end{itemize}

\noindent [7] If we knew what this highest good was, we could order our lives correctly
\vspace*{2mm}

\noindent [8] \noindent Aristotle claims that almost everyone agrees that ``happiness'' is the \emph{name} for the HG; but, they disagree about the \emph{content} of happiness?

\section*{Chs 3, 4, and 5: Not on Handout}

\noindent Ch. 3: Note that A says, when we're talking about the HG we can't demand the same exactitude as when we are talking about mathematics, for example; he just aims to describe it in outline; what he seems to mean is that, while he will say, for example, that justice is part of the good human life, the particular actions justice requires will vary from person to person so dramatically that you can't be more specific than that
\vspace*{2mm}

\noindent Ch. 5: the three lives again: of gratification, political activity, and study

\section*{Formal criteria for the highest good (ch. 7)}

\noindent [A] \textbf{Choiceworthy}: The HG must be good or valuable in some way
\vspace*{1mm}

\noindent [B] \textbf{End}: The HG must be an end
\vspace*{1mm}

\noindent [C] \textbf{Complete}: X is \emph{more complete} than Y if (a) X is pursued for X and (b) Y is pursued for something else, Z\\\hspace*{24mm}X is complete without qualification if X is pursued \emph{only} for itself (1097a)
\vspace*{1mm}

\noindent [D] \textbf{Self-sufficient}: All by itself makes a life choiceworthy and lacing nothing (1097b)\\\hspace*{26mm}NB: This does \emph{not} necessarily mean that the value of a life that has the HG can't be increased;\\\hspace*{26mm} It could mean that a life with the HG doesn't need anything more to make it choiceworthy\\\hspace*{26mm} as a life
\vspace*{2mm}

\section*{``\emph{Ergon}'' argument (``\emph{ergon}'' often rendered as ``function,'' ``characteristic activity,'' or ``work'')}

\noindent [P1] The good for something that has a work and (characteristic) action depends on its work
\vspace*{1mm}

\noindent [P2] The parts of a human have a work; so a human as a whole should have a special work
\vspace*{1mm}

\noindent [P3] The ``lives'' of nutrition and perception are not unique to humans
\vspace*{1mm}

\noindent [P4] A life expressing reason (of two kinds) is remaining possibility for human special function
\vspace*{1mm}

\noindent \underline{[P5] ``Life'' understood as activity (as opposed to capacity) is more properly life}
\vspace*{1mm}

\noindent [C1] The human work is the soul's activity expressing reason
\vspace*{3mm}

\noindent\underline{[P7] The work of an X is the same in kind as the work of a good F; the latter accomplishes its work \emph{well}}
\vspace*{1mm}

\noindent [C2] Given C1 and P7, the good man's work is doing C1 well
\vspace*{3mm}

\noindent \underline{[P8] Work is completed well when its completion expresses its proper virtue}
\vspace*{1mm}

\noindent [C3] Given C1, C2, and P8, the human good is the souls activity which expresses virtue
\vspace*{2mm}

\noindent Happiness is a life [i.e. the soul's activity] based on [i.e. expressing] virtue (in chs. 8-9, A canvasses how this candidate gels with other ideas about happiness

\section*{Additional constraints on happiness}

\noindent [E] It requires activity based on virtue in a \emph{complete} life (ch. 9 1100a; ch. 10 1101a-b)
\vspace*{2mm}

\noindent [F] It requires external goods (ch. 8 1099b; ch. 10 1101a)
\vspace*{2mm}

\noindent [G] It must be the best, finest, and most pleasant (life) (ch. 8 1099a)

\section*{Problems}

\noindent [1] Has A shows us that there is one highest good (is the argument in section 1.4 sound)?
\vspace*{2mm}

\noindent [2] Why believe that the work of a human being is what is \emph{special} or \emph{unique} to human beings?
\vspace*{2mm}

\noindent [3] Why should we think that accomplishing the human work well is productive of, or identical with, happiness (e.g. does it meet constraint G?)?

\end{document}
