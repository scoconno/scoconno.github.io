\documentclass[oneside]{article}
 \headheight = 25pt
\footskip = 20pt
\usepackage{graphicx}

\usepackage{mdwlist}
\usepackage[T1]{fontenc}
\renewcommand{\rmdefault}{ppl}
\usepackage{fancyhdr}
 \pagestyle{fancy}
 \lhead{\textbf{\textsc{\small Scott O'Connor\\Metaphysics}}}
 \chead{}
 \rhead{\large\textbf{\textsc{Zeno 3}}}
 \lfoot{\footnotesize{\thepage}}
 \cfoot{}
 \rfoot{\footnotesize{\today}}
 \usepackage{longtable,booktabs}
\tolerance=700


\begin{document}
\thispagestyle{fancy}




\section{Motion}\label{motion}

asd

\subsection{Master Argument}\label{master-argument}

What makes an argument right or left handed?

\begin{enumerate}
\def\labelenumi{\arabic{enumi}.}
\item
  A hand is left or right either (a) solely in virtue of the
  \emph{internal} relations among the parts of the hand, or (b) at least
  partly in virtue of the \emph{external} relations of the hand to
  something outside it, either (i) other material objects, or (ii) space
  itself.
\item
  A hand is not left or right solely in virtue of its internal
  relations, since the internal relation are the same for right and
  left, i.e.~the proportion and the position of the parts to one another
  of each hand are the same.
\item
  A hand is neither right nor left even partly in virtue of its
  relations to other material objects, since a hand that was all alone
  in the universe would still be right or left.
\item
  Therefore, a hand is left or right at least party in virtue of its
  relation to absolute space.\footnote{`Incongruent Counterparts and
    Higher Dimensions', by James Van Cleve}
\end{enumerate}

1--3 state the premises of the argument. 4 states the conclusion. The
argument is valid. If it is sound, then 4 must be true. Here are the
arguments for the premises.

\subsection{Internalism}\label{internalism}

Internalists accept 1 and 3, but they rejects 2. They accept, 3, that a
hand that was all alone in the Universe would still be a right or left
hand, but they think that it's being left or right can be explained
without looking outside the hand itself. The features of the hand alone
will make it a right or left one. Hence, one needs no other material
objects or space itself to explain these features.

\begin{itemize}
\item
  Kant assumed that the internal relation were distances between points
  and angles between lines. These are the same for both hands. If these
  are the only internal relations, then Kant is right to deny that
  internal relations can distinguish right from left handedness.
\item
  But might there be other other internal relations that explain the
  difference between the left and right hand?
\item
  Any candidates face a certain killer objection: if we can flip the
  hands in a fourth dimension to make them congruent, then right
  handedness and left handedness cannot be intrinsic properties of the
  hands.
\end{itemize}

\subsection{Externalism}\label{externalism}

Externalists accept 1 and 2, but reject 3. They claim that being a left
or right hand depends on a relationship to other material objects.

Objection: a world in which only a hand exists would either be left or
right handed. (a) Suppose that there is a world in which only a hand, H,
exists. (b) Now suppose that a body possessing no hands pops into
existence. (c) H can fit only the left or the right wrist. (d)
Therefore, H was right or left handed before the body popped into
existence.

How would the existence of a fourth-dimension effect this objection, if
at all? If 4th exists, then it was nor right or left beforehand.

\subsection{Absolutism}\label{absolutism}

Absolutists accept each premises, and since the argument is also valid,
they think the conclusion is true. Space, they claim, is absolute.

If we only accept the existence of 3 spatial dimensions, must we then
accept that absolute space exists? I think the issue is that 4th
dimension undermines the absolutists. The Externalist can appeal to 4th
to say that they are not congruent at all. But then there is an issue of
being incongruent in the 4th dimension.

\end{document}
