\documentclass[oneside]{article}
 \headheight = 25pt
\footskip = 20pt
\usepackage{graphicx}

\usepackage{mdwlist}
\usepackage[T1]{fontenc}
\renewcommand{\rmdefault}{ppl}
\usepackage{fancyhdr}
 \pagestyle{fancy}
 \lhead{\textbf{\textsc{\small Scott O'Connor\\Metaphysics}}}
 \chead{}
 \rhead{\large\textbf{\textsc{Zeno 3}}}
 \lfoot{\footnotesize{\thepage}}
 \cfoot{}
 \rfoot{\footnotesize{\today}}
 \usepackage{longtable,booktabs}
\tolerance=700


\begin{document}
\thispagestyle{fancy}


\subsection*{Background Assumptions}
\begin{description}
\item [Two types of `sameness':] Qualitative similarity \emph{vs.} numerical identity. 
\begin{itemize} 
\item Two ginger bread men made by the same cookie cutter are qualitatively similar in several respects, e.g., they have the same shape, the same weight, the same colour, smell, and so on. But these are \emph{two} numerically distinct entities. 
\item Superman and Clark Kent are numerically identical, i.e., they are one and the same entity. If you want to count the number of entities in the room, you should only count Superman and Clark Kent once: Louis Lane makes a mistake when she counts them as two. 
\end{itemize}
\item [Indiscernibility of Identicals:] If x and y are numerically identical, then for every quality F, if x is F, then y is also F and \emph{vice versa}.  
\begin{itemize}
\item If Superman and Clark Kent are numerically identical, then any quality which belongs to one will also belong to the other, e.g., if the ability to fly belongs to Superman, it also will belong to Clark Kent.
\item Think of this principle as a tool for testing whether two things are numerically distinct, i.e., if x and y have different qualities, then they cannot be numerically identical. 
\end{itemize}
\item [Identity of Indiscernibles:] If x and y have all the same qualities, then x and y are numerically identical. 
\begin{itemize}
\item If Superman and Clark Kent have all the same qualities, then they are identical. 
\end{itemize}
\end{description}

\subsection*{Four Puzzles} 
\begin{enumerate}
\item The Ship of Theseus 
\item Tibbles and Tib 
\item The Statue and The Clay 
\item The Debtor's Paradox 

\end{enumerate}

\subsection*{Diagnosis of these puzzles}

Each puzzles assumes five incompatible assumptions:\footnote{From `Material  Constitution: A  Reader',  by Michael Rea, Intro.}
\begin{description}
\item [The Existence Assumption:] There is an F and there are parts that compose an F, e.g., there is a statue and there are parts that compose the statue. 
\item [Essentialist Assumption:] For any group of parts, if those parts compose an F, then the same parts compose some object \emph{a} such that \emph{a} essentially is composed of those parts, e.g., if some bits of clay compose a statue, then there exists something, \emph{a lump}, that is essentially composed of those bits of clay. 
\begin{itemize}
\item If \emph{a} is essentially composed of some parts, then those parts compose \emph{a} at every time that \emph{a} exists, e.g., the bits of clay that essentially compose a lump, compose that lump throughout its existence. 
\end{itemize}
\item [Principle of Alternative Combinatorial Possibilities (PACP):] For any group of parts, if those parts compose an F, then those parts compose some object \emph{b} such that  \emph{b} is not essentially composed of those parts.
\begin{itemize}
\item If \emph{b} is not essentially composed of some parts, then \emph{b} can be composed of  those parts at some times but not at others, e.g., if my lego house is not essentially composed of some red bricks, then the red bricks could make up the house at one time, but not another. 
\end{itemize}
\item [The Identity Assumption:] For any objects x  and y, if x and y  share all the same parts at the same time, then x is identical with y. 
\begin{itemize}
\item If Superman and Clark Kent have exactly the same parts,  then Superman and Clark Kent are numerically identical. 
\item The Identity Assumption entails that no two material objects can occupy the exact same region of space, a principle which has been endorsed from philosophers as early as Aristotle. 
\end{itemize}

\item[The Necessity Assumption:] For any objects x  and y, if x is identical with y, then it is necessary that x is identical with y.
\begin{itemize}
\item If it is necessary that x is identical with y, then x is identical with y at every moment that y exists. 
\end{itemize}
\end{description}

\subsection*{Why are these assumptions incompatible?} 

Our Strategy: We assign the names `\emph{a}' and `\emph{b}' to arbitrary objects. We assume  that \emph{a} and \emph{b} are composed of the same parts. We then show that under some of these assumptions, \emph{a} and \emph{b} are identical, but under other assumptions \emph{a} and  \emph{b} are not identical. Rea explains: 
\begin{quote}The problem is that for any composite object \emph{a} we can (generally) identify an object \emph{b} that constitutes it and \emph{b} is essentially related to its parts in a way that \emph{a} is not. [However] the  Identity Assumption tells us that \emph{a} is identical with \emph{b}, the Necessity Assumption tells us that \emph{a} and \emph{b} could not have been distinct; hence, \emph{a} and \emph{b} are not essentially related to their parts in different ways. (Rea, pxxviii.) 
\end{quote}
NB: When we speak of the parts that compose an object, we mean all the parts of  that object. We are not speaking about partial composition.

\subsubsection*{Application to Lumpl/Goliath} 
\begin{enumerate}
\item By the existence assumption, there exists a statue and there exists parts of that  statue (the \emph{p}s). 
\item By the essentialist assumption, since the \emph{p}s compose the statue, then there  exists an object \emph{a} such that the parts of \emph{a} essentially compose a statue. 
\item Assume that \emph{a} is Goliath. 
\item By PACP, since the \emph{p}s compose the statue, then there exists an object \emph{b}  such that the parts of \emph{b} do not essentially compose a statue. 
\item Assume that \emph{b} is Lumpl. 
\item By the identity assumption, Goliath and Lumpl are identical. 
\item By the necessity assumption, Goliath and Lumpl are not identical. 
\item Therefore, Goliath and Lumpl are and are not identical. 
\end{enumerate}

\subsubsection*{An abstract version of the argument}

\begin{enumerate}
\item By the existence assumption, there exists an F and there exists parts (\emph{p}s) that compose this object. 
\item By the essentialist assumption, since the \emph{p}s compose something, then there exists an \emph{a} that is essentially composed of the \emph{p}s. 
\item By PACP, since the \emph{p}s compose something, then there exists a \emph{b}  that is not essentially composed of the \emph{p}s. 
\item \emph{a} and \emph{b} are identical. Proof: 
\begin{enumerate}
\item[i] By the identity assumption, if \emph{a} and \emph{b} have the same parts, then \emph{a} and \emph{b} are identical. 
\item[ii] By the essentialist assumption and PACP, \emph{a} and \emph{b} have the same parts. 
\item[iii] \emph{a} and \emph{b} are identical (from i\&ii).
\end{enumerate}
\item \emph{a} and \emph{b} are not identical. Proof: 
\begin{enumerate} 
\item[i] By the necessity assumption, if \emph{a} and \emph{b} are identical then at every time that \emph{a} and \emph{b} exist, \emph{a} and \emph{b} must have the same relationship to the same \emph{p}s. 
\item[ii] By the essentialist assumption, \emph{a} is essentially composed of the \emph{p}s. 
\item[iii] By PACP, \emph{b} is not essentially composed of the \emph{p}s. 
\item[iv] \emph{a} and \emph{b} do not have the same relationship to the same \emph{p}s (from ii\&iii).
\item[v] \emph{a} and \emph{b} are not numerically identical (from i\&iv).
\end{enumerate}
\end{enumerate}



\subsection*{Responses} 
These five assumptions cannot all be true together. At least one assumption must be  false. If we can identify which assumption or assumptions are false, then we can  identify how to solve the puzzles. 





\end{document}

