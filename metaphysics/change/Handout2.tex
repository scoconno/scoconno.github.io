\documentclass[oneside]{article}
 \headheight = 25pt
\footskip = 20pt
\usepackage{graphicx}

\usepackage{mdwlist}
\usepackage[T1]{fontenc}
\renewcommand{\rmdefault}{ppl}
\usepackage{fancyhdr}
 \pagestyle{fancy}
 \lhead{\textbf{\textsc{\small Scott O'Connor\\Metaphysics}}}
 \chead{}
 \rhead{\large\textbf{\textsc{Change 2}}}
 \lfoot{\footnotesize{\thepage}}
 \cfoot{}
 \rfoot{}
 \usepackage{longtable,booktabs}
\tolerance=700


\begin{document}
\thispagestyle{fancy}

\section*{Wholes and Parts}
The puzzles of material constitution partly arise partly because over claims about when there exists a whole composed of parts. For instance, the puzzles of Tibbles and Tib assumes that Tim exists in the first place. Our focus in this handout is on when some entities compose a whole. Peter Van Inwagen has dubbed this \emph{The Special Composition Question}. For some groups of entities, the \emph{x}s, we can phrase the question in two ways: 
\begin{itemize}
\item When is it true that there is something the \emph{x}s compose?...the official question.
\item Suppose one had certain nonoverlapping objects, the \emph{x}s, at one’s disposal; what would one
have to do--what could one do--to get the \emph{x}s to compose something?...the practical question.
\end{itemize}

Three possible general answers: 

\begin{enumerate}

\item The \emph{x}s never compose something. There is no such thing as composition...nihilism
\item The \emph{x}s sometimes compose something, but not always...moderate
\item The \emph{x}s always compose something. EVERY collection of \emph{x}s is an object...universalism
\end{enumerate}



\section*{Mereological Nihilism}
The nihilist makes two claims: 1) No composite object exists, where a composite object is one that has material parts, and 2) material simples exist, where a material simple is an entity with no material parts.


\subsection*{Sorites argument for Nihilism} 

We encounter sorites paradoxes for a predicate `P' when we cannot determine the range of entities that we can apply `P' to, e.g. `heap', `bald', `table', `person', and so on. Here is an example:\\ 

\begin{description}
\item[P1] A man with 10,000 hairs on his head is not bald.
\item[P2] If a man with 10,000 hairs on his head is not bald,  then a man with 9,999 hairs on his head is not bald.
\item [P3] If a man with 9,999 hairs on his head is not bald, then a man with 9,998 hairs on his head is not bald. 
\item [PN] If a man with 1 hair on his head is not bald, then a man with 0 hairs on his head is not bald.
\item[C] A man with 0 hairs on his head is not bald.
\end{description}


\subsubsection*{Tables do not exist}

\begin{description}
\item[P1] There exists at least one table.
\item[P2] For anything there may be, if it is a table, then it consists of many atoms, but only a finite number.
\item[P3] From P1-P2 it follows that at least one table exists that consists of atoms.
\item [P4] For anything there may be, if it is a table (which consists of many atoms, but a finite number) then the net removal of one atom, or only a few, in a way which is most innocuous and favorable, will not mean the difference as to whether there is a table in the situation. 
\item [P5] If P4 is true, then there can exist a table that consists of no atoms. Proof:
\begin{description}
\item[P5.1] There exists a table T{\scriptsize 1}, and T{\scriptsize 1} consists of 10,000 atoms.
\item[P5.2] From P4 and P5.1, if we (net) remove one atom from T{\scriptsize 1}, then there exists a table T{\scriptsize 2} and T{\scriptsize 2} consists of 9,999 atoms.
\item[P5.3] From P4 and P5.1-P5.2,  if we (net) remove one atom from T{\scriptsize 2}, then there exists a table T{\scriptsize 3}, and T{\scriptsize 3} consists of 9,998 atoms.
\item[P5.4] From P4 and P5.1-P5.3,  if we (net) remove one atom from T{\scriptsize 3}, then there exists a table T{\scriptsize 4}, and T{\scriptsize 4} consists of 9,998 atoms.
\item[P5.C] From P4 and P5.1 - P5.N, if we (net) remove one atom from T{\scriptsize N}, then there exists a table T{\scriptsize C}, and T{\scriptsize 0} consists of 0 atoms.
\end{description}
\end{description}

\section*{Moderate Answers}
\subsection*{Example: Contact}
\begin{quote}
To get the \emph{x}s to compose something, one need only bring them into contact; if the \emph{x}s are in contact, they compose something; and if they are not in contact, they do not compose anything.
\end{quote}
What does `contact' mean? Aristotle defines contact in terms of boundaries: y and z are in contact if their boundaries touch. 

%\begin{quote}
%The \emph{x}s are in contact if (1) no two of them overlap spatially, and (2) if y and z are among the \emph{x}s, then y is in contact with z, or y is in contact with w, which is one of the \emph{x}s, and w is in contact with z--and so on.
%\end{quote}


\subsection*{First objection to contact}
\begin{enumerate}
\item Suppose contact is the correct answer to the Special Composition Question.
\item For anything at all, if it is a composite material object, it is composed of quarks and electrons.
\item No quarks and electrons are in contact.
\item Therefore, there is no thing such that quarks and electrons compose it.
\item Therefore, there are no composite material objects.
\item But there are composite material objects--this table, for example.
\item So contact is not the correct answer to the Special Composition Question.
\end{enumerate}

\subsection*{Second objection to contact}
\begin{enumerate}
\item Suppose contact is the correct answer to the Special Composition Question.
\item Then when Bob and Sam shake hands, they compose something.
\item But when Bob and Sam shake hands, they do not compose something.
\item So contact is not the correct answer to the Special Composition Question.
\end{enumerate}

\subsection*{Other Moderate Answers}

\begin{description}
\item[Fastening:] Some xs compose a y if and only if the xs are in contact and fastened together such that they are not easily pulled apart.
\item[Cohesion:] Some xs compose a y if and only if the xs are in contact such that they cannot be pulled apart without breaking
\item[Fusion:] Some xs compose a y if and only if the xs are in contact such that they have fused together and it is no longer clear where the boundaries of the parts are (i.e., there is no obvious place where one part ends and the next part begins).
\end{description}
\subsection*{Van Inwagen’s answer moderate answer}
\begin{quote}
The \emph{x}s compose something iff the activity of the \emph{x}s constitutes a life.
\end{quote}
A life is a certain kind of self-maintaining event, that is reasonably well-individuated. It also never happens that the activity of some things constitutes two different lives. Two consequence: 1) If Tib is a part of Tibble, then Tib's activity must, with the rest of Tibble's parts, constitute a life. But it's unclear that Tib has any activity at all. 2) There are no inanimate objects like tables - such things would have to be composed by some things whose activity did not constitute a life.





\end{document}
