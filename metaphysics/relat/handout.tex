\documentclass[article,oneside]{memoir}
\usepackage{unicode-math}
\usepackage{graphicx,url}
\usepackage{rotating} 
\usepackage{memoir-article-styles} 
\usepackage{fontspec} 
\usepackage{xunicode} 
\usepackage[polutonikogreek,english]{babel}
\newcommand{\greek}[1]{{\selectlanguage{polutonikogreek}#1}}
\def\myaffiliation{Metaphysics: Phil 205}
\def\myauthor{Scott O'Connor}
\def\myemail{\small{\texttt{\href{mailto:soconnor@njcu.edu}{soconnor@njcu.edu}}}}
\def\mytitle{Phenomenalism}
\def\mykeywords{2015, 2016}
\usepackage[usenames,dvipsnames]{color}                     
\usepackage[xetex, 
	colorlinks=true,
	urlcolor=BlueViolet, % external links
       citecolor=BlueViolet, % citations
        filecolor=BlueViolet, % local files
	plainpages=false,
  	pdfpagelabels,
  	bookmarksnumbered,
  	pdftitle={\mytitle},
 	pdfauthor={\myauthor},
  	pdfkeywords={\mykeywords}
  	]{hyperref}   



\usepackage{setspace}



\begin{document}
\setromanfont[Mapping=tex-text]{Minion Pro} 
\setsansfont[Mapping=tex-text]{Myriad Pro}  
\setmonofont[Mapping=tex-text,Scale=MatchLowercase]{Minion Pro} 
\setkeys{Gin}{width=1\textwidth} 
\chapterstyle{article-3}  
\pagestyle{kjh}

\title{\mytitle}

\author{\myauthor \\
 \emph{\myaffiliation \\ }}
%\thanks{OMITTED}
%\published{\footnotesize Draft. Please do not cite without permission.}


\maketitle

\section{Introduction}

Things must be a certain way for us to have any experiences of them. For instance, things must have color and shape for us to see them, they must be spatially separated from us and other things if we are distinguish them from one another and from ourselves, etc. Let us call the features and relations that things must have for us to perceive them \emph{phenomenal properties}. Metaphysicians have been interested in the following question: 
\begin{itemize}
\item Do we experience things  \emph{because} they have phenomenal properties, or do things have phenomenal properties \emph{because} we experience them?
\item Compare: Is piety good \emph{because} God loves it, or does God love piety \emph{because} it is good? 
\end{itemize} 
We shall call \emph{phenomenalists} those who believe that things have phenomenal properties because we experience them.  It is a view that Aristotle attributes to Protagoras: 
\begin{quote}
...all things believed [ta dokounta panta] and [all] appearances [ta phainomena] are true’ (1009a8).
\end{quote}
`The things that appear to us' (ta phainomena) are things as they appear to our senses, i.e. the contents of our sense perceptions. The `things that we believe’ (ta dokounta) are general things as we believe them to be, i.e. the contents of our beliefs in general. We can believe things that we do not sense and \emph{vice versa} 

\begin{description}
\item[Phenomenalism (PHEN):] Anything that appears thus and so to someone, and anything that someone believes to be thus and so, is thus and so (i.e. all appearances and all beliefs are true).
\end{description}
Distinguish the following two claims:


\begin{enumerate}
\item For \emph{some} property F, an object O is F because O appears F to someone. 
\item For \emph{any} property F,  an object O is F because O appears F to someone. 
\end{enumerate}
The first claims says that objects have just some properties because of how they appear to us. The second claims that any property an object has is explained by how it appears to us. PHEN makes the second claim. The schematic argument for PHEN:
\begin{itemize}
\item P1. No object has an essence.  
\item P2. If no object has an essence, then we cannot speak and think about things. 
\item P3. But we can speak and think about things. 
\item P4. We can speak and think about things if they have an essence because of how they appear to us (even if they have no essence independently of how they appear to us.) 
\item C. Therefore, things have an essence because of how they appear to us.  
\end{itemize}

\section{P1. Essence}

The word `essence'  is from the Latin word `essentia', which was coined to translate a phrase invented by Aristotle, `to ti en einai.' This roughly translates as `what it was for something to be'. He introduced it to describe the stable natures that changing being have over time. The phrase was then extended to discuss the natures of non-changing beings. For our purposes, we will define \emph{essentialism} as the conjunction of the following two claims:
\begin{itemize}
\item Each thing belongs to some kind K such that if that thing ceases to be a K, it will cease to exist...e.g., Socrates dies if he ceases to be human, Flipper dies if he ceases to be a dolphin. 
\item For every kind K, there are a unique set of individually necessary conditions that are jointly sufficient for membership of that kind, e.g., being rational and being an animal are each necessary conditions for being human. They are also jointly sufficient for being a human; anything that satisfies both is a human and nothing is a human that does not satisfy both. 
\end{itemize}
Some standard objections to essentialism: 

\begin{enumerate}
\item It is possible for objects to change kinds without ceasing to exist, e.g., Lot's wife, some fish, werewolves, etc.
\item There are no necessary and sufficient conditions for kind membership, e.g., species evolve, amputees are still members of their kind, etc.
\item Belief is essentialism leads to prejudices. 
\item Things change always and in every way, and nothing is constant about them.
\end{enumerate}

\section{P2. Essence and thought/language}
\begin{quotation}
For not to signify one thing [a unitary thing, a determinate thing] amounts to not signifying anything at all. (Aristotle)
\end{quotation}


\begin{enumerate}
\item To think and speak about something is to think and speak about one thing as opposed to another, e.g., to think and speak about Socrates is to think and speak about Socrates as opposed to something else. 
\item  In order to think and speak about one thing as opposed to another, we must be able to indicate what distinguishes the thing that we are thinking and speaking about from other things, e.g., to think and speak about a cloud, we must be able to indicate what distinguishes the cloud from other things.  
\item It is in virtue of a thing having an essence that is the very thing that it is, as opposed to any other thing.
\item So, in order to distinguish the thing that we are thinking and speaking about from other things, we must be able to indiciate its essence. 
\item[C.] If an object does not have an essence, then we cannot think and speak about it. 
\end{enumerate}


\noindent \emph{Moral: thought and language about things is possible only if they have essences, e.g., I can think and speak about x only if there is some essence E such that x is E.}


\section{P3. Language and thought are possible}

Aristotle thinks it is obvious that language and thought is possible. But, proving that it is possible is tricky. He writes the following: 

\begin{quote}
If he [the disputant] says nothing [i.e. if he really accepts that language is impossible], it is ridiculous to try to say something against one who does not have a statement to make about anything, in so far as he is like that; for such a person is to that extent already like a vegetable (1006a13–15).
\end{quote}
This is a self-refutation argument. Aristotle argues that anyone who denies the possibility of thought and language refutes themselves. 

\begin{enumerate}
\item Suppose that S denies that thought and language are possible. 
\item All denials involve thought and language. 
\item So S thinks and speaks when she denies that thought and language are possible. 
\item[C.] Therefore, S cannot consistently deny that thought and language are possible. 
\end{enumerate}

\noindent \emph{NB: this argument does not prove that thought and language are possible. It proves only that you cannot consistently deny that it possible.}

\section{P4. Essences as they appear to us}

Distinguish two claims: 
\begin{itemize}
\item Socrates has an essence in virtue of himself. 
\item Socrates has an essence in virtue of how he appears to us. 
\end{itemize}
Suppose that Socrates is essentially human. Our first claim says that he is human, as opposed to a giraffe, dolphin, or hedgehog, not because of his relation to anything else. He is a human precisely because of himself. Our second claims agrees that Socrates is essentially human. But it states that it is precisely in virtue of how he appears to us that he is human; his being human is explained by how he appears to us and not explained by how he is in virtue of himself. 

Here, then, is a simple way of presenting our argument for Phenomenalism: 

\begin{enumerate}
\item Language and thought about things is possible. 
\item Language and thought about things is possible only if things have an essence. 
\item Things do not have an essence in virtue of themselves. 
\item[C1.] So, things must have an essence in virtue of how they appear to us. 
\end{enumerate}
With a little work, we can also use this conclusion to establish the following claim: 

\begin{itemize}
\item[C2.] We can only think and speak of appearances, i.e., of things as they appear to us and as we conceive them, but we cannot think and speak about things without qualification, i.e., of things as they are in themselves. 
\end{itemize}

\noindent Group project: How might we prove C2? You will find material on this handout to help. 




\section{Objection to Phenomenalism}

If Phen is true, then our sense perceptions and beliefs are only of things as they appear to us and as we conceive them; they are not about the things themselves. For example, if some wine appears sweet to one, and if one goes on on this basis to believe that the wine is sweet, then this belief is not at all about the wine itself, it is only about the wine as it appears to one and to one’s taste. Aristotle objects as follows: 


\begin{quote}
 But in general, if only what is sense-perceptible exists, then nothing would exist if there were no minds; for there would be no sense perception. Now, the view that neither sense-perceptible things nor sense perceptions exist may perhaps be true; for sense perception is an affection of a thing that has sense perception. But the view that the underlying things, which produce the sense perception [in a thing that has sense perception], do not exist without the sense perception---this view is impossible.

For sense perception is not simply directed at itself, rather there is also something other than and distinct from the sense perception, which is necessarily prior to the sense perception. This is because that which causes change is naturally prior to that which is caused to change, and this is so even if these are said to be correlative. (1010b30–1011a2)
\end{quote}
Here is my formulation of the argument: 
\begin{enumerate}
\item Minds \# perceivings, e.g., my mind \# my seeing green.
\item If the objects that we perceive depends on our perceptions of them, then they depend on our minds. If they do not depend on our perception of them, then they do not depend on our minds.
\item The objects that we perceive are not identical to our perceptions of them, e.g., green is distinct from the seeing of green, sweetness is different from the tasting of sweetness, etc. 
\item The objects that we perceive are the causes of our sense perceptions.
\item If X is the cause of Y, then X exists independently of Y.
\item[C1.] Thus, the objects that we perceive exist independently of our perceptions of them.
 \item[C2.] Thus, the objects that we perceive exist independently of our minds. 
\end{enumerate}
Phenomenalists believe that there is a sharp distinction between things as they appear to us and the things themselves. They think that this distinction is so sharp that we can think and speak about appearances even if it is impossible to think and speak about the things themselves. Aristotle challenges the very possibility of drawing such a sharp distinction. If we are to accept that there is a reality that is behind the appearances, a reality populated by entities that fail to have essences in themselves,  then our acceptance of such reality involves thinking about and conceiving a reality that is behind the appearances. But such thinking is possible only if there really is some way that we can think about and conceive reality independently of how it appears to us.  



\section{Conclusion}

Aristotle's argument against Phen proves one of the following: 

\begin{itemize}
\item Things have essences independently of how we think and speak about them, or
\item Thought and language are impossible, or
\item Things in themselves may or may not have essences, but that is something we cannot even think and talk about meaningfully. 
\end{itemize}











\end{document}
