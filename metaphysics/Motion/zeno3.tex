\documentclass[]{article}
%\usepackage{lmodern}
\usepackage{amssymb,amsmath}
\usepackage{ifxetex,ifluatex}
\usepackage{fixltx2e} % provides \textsubscript
\ifnum 0\ifxetex 1\fi\ifluatex 1\fi=0 % if pdftex
  \usepackage[T1]{fontenc}
  \usepackage[utf8]{inputenc}
\else % if luatex or xelatex
  \ifxetex
    \usepackage{mathspec}
    \usepackage{xltxtra,xunicode}
  \else
    \usepackage{fontspec}
  \fi
  \defaultfontfeatures{Mapping=tex-text,Scale=MatchLowercase}
  \newcommand{\euro}{€}
\fi
% use upquote if available, for straight quotes in verbatim environments
\IfFileExists{upquote.sty}{\usepackage{upquote}}{}
% use microtype if available
\IfFileExists{microtype.sty}{%
\usepackage{microtype}
\UseMicrotypeSet[protrusion]{basicmath} % disable protrusion for tt fonts
}{}
\ifxetex
  \usepackage[setpagesize=false, % page size defined by xetex
              unicode=false, % unicode breaks when used with xetex
              xetex]{hyperref}
\else
  \usepackage[unicode=true]{hyperref}
\fi
\hypersetup{breaklinks=true,
            bookmarks=true,
            pdfauthor={},
            pdftitle={Zeno 3},
            colorlinks=true,
            citecolor=blue,
            urlcolor=blue,
            linkcolor=magenta,
            pdfborder={0 0 0}}
\urlstyle{same}  % don't use monospace font for urls
\setlength{\parindent}{0pt}
\setlength{\parskip}{6pt plus 2pt minus 1pt}
\setlength{\emergencystretch}{3em}  % prevent overfull lines
\setcounter{secnumdepth}{0}

\title{Zeno 3}
\date{}

\begin{document}
\maketitle

\subsubsection{The Arrow Paradox}\label{the-arrow-paradox}

\begin{quote}
The third is \ldots{} that the flying arrow is at rest, which result
follows from the assumption that time is composed of moments \ldots{} .
he says that if everything when it occupies an equal space is at rest,
and if that which is in locomotion is always in a now, the flying arrow
is therefore motionless. (Aristotle Physics, 239b.30)
\end{quote}

\begin{quote}
Zeno abolishes motion, saying ``What is in motion moves neither in the
place it is nor in one in which it is not''. (Diogenes Laertius Lives of
Famous Philosophers, ix.72)
\end{quote}

\subsubsection{Outline of the Paradox}\label{outline-of-the-paradox}

Assume the following claims:

\begin{enumerate}
\def\labelenumi{\arabic{enumi}.}
\itemsep1pt\parskip0pt\parsep0pt
\item
  Space is finitely divisible.
\item
  Time is composed of moment.
\end{enumerate}

\begin{itemize}
\itemsep1pt\parskip0pt\parsep0pt
\item
  P1. An arrow must occupy a space equal to itself at each moment that
  it exists.
\item
  P2. If an arrow moves for, say, 1 minute, the arrow will occupy a
  space equal to itself at each moment that is it moving.
\item
  P3. An arrow that occupies a space equal to itself at a specific
  moment is not moving in that moment in that space.
\item
  P4. If an arrow is not moving
\end{itemize}

Consider an arrow, apparently in motion, at any instant. First, Zeno
assumes that it travels no distance during that moment---`it occupies an
equal space' for the whole instant. But the entire period of its motion
contains only instants, all of which contain an arrow at rest, and so,
Zeno concludes, the arrow cannot be moving.

'' a moving arrow must occupy a space equal to itself during any moment.
That is, during any moment it is at the place where it is. But places do
not move. So, if in each moment, the arrow is occupying a space equal to
itself, then the arrow is not moving in that moment because it has no
time in which to move; it is simply there at the place. The same holds
for any other moment during the so-called ``flight'' of the arrow. So,
the arrow is never moving. Similarly, nothing else moves. The source for
Zeno's argument is Aristotle (Physics, Book VI, chapter 5, 239b5-32).

\subsubsection{The Standard Solution}\label{the-standard-solution}

The standard solution to the paradox uses the ``at-at'' theory of
motion:

\begin{itemize}
\item
  \begin{enumerate}
  \def\labelenumi{(\roman{enumi})}
  \itemsep1pt\parskip0pt\parsep0pt
  \item
    being in motion involve being at different places at different
    times, and (ii) being at rest involves being motionless at a
    particular point at a particular time.
  \end{enumerate}
\end{itemize}

This theory asks us to distinguish two things:

\begin{enumerate}
\def\labelenumi{\alph{enumi})}
\itemsep1pt\parskip0pt\parsep0pt
\item
  being in motion in or during an instant.
\item
  being in motion at an instant.
\end{enumerate}

The at-at theory accepts that the arrow cannot move during an instant,
but claims that the arrow can still move at an instant. It does so by
occupying different locations before and after that instant.

If this is correct, the difference between rest and motion has to do
with what is happening at nearby moments and has nothing to do with what
is happening during a moment. The arrow counts as moving at an instant
because it occupies different locations before and after that instant.
The arrow counts as being at rest at an instant because it is not
located at different locations before and after that instant.

The instant must be part of a period in which the arrow is continuously
in motion.

instantaneous motion from instantaneous rest.

The Arrow Paradox seems especially strong to someone who would say that
motion is an intrinsic property of an instant, being some propensity or
disposition to be elsewhere.

Calculus: speed of an object at an instant (instantaneous velocity) is
the time derivative of the object's position.

This means the object's speed is the limit of its speeds during
arbitrarily small intervals of time containing the instant.

The object's speed is the limit of its speed over an interval as the
length of the interval tends to zero.

The derivative of position x with respect to time t, namely dx/dt, is
the arrow's speed, and it has non-zero values at specific places at
specific instants during the flight.

The speed during an instant or in an instant, which is what Zeno is
calling for, would be 0/0 and so be undefined.

Using these modern concepts, Zeno cannot successfully argue that at each
moment the arrow is at rest or that the speed of the arrow is zero at
every instant. Therefore, advocates of the Standard Solution conclude
that Zeno's Arrow Paradox has a false, but crucial, assumption and so is
unsound.

\subsubsection{Response}\label{response}

Stuff before and a moment determines if its moving. That's odd.

Suppose God were to wipe the arrow out of existence completely an
instant after it moved. This would mean that it was not moving. But
notice here that we are assuming that the future is fine in the future.

\end{document}
