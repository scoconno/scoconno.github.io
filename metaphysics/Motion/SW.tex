\documentclass[]{article}
\usepackage{amssymb,amsmath}
\usepackage{ifxetex,ifluatex}
\usepackage{fixltx2e} % provides \textsubscript
\ifnum 0\ifxetex 1\fi\ifluatex 1\fi=0 % if pdftex
  \usepackage[T1]{fontenc}
  \usepackage[utf8]{inputenc}
\else % if luatex or xelatex
  \ifxetex
    \usepackage{mathspec}
    \usepackage{xltxtra,xunicode}
  \else
    \usepackage{fontspec}
  \fi
  \defaultfontfeatures{Mapping=tex-text,Scale=MatchLowercase}
  \newcommand{\euro}{€}
\fi
% use upquote if available, for straight quotes in verbatim environments
\IfFileExists{upquote.sty}{\usepackage{upquote}}{}
% use microtype if available
\IfFileExists{microtype.sty}{%
\usepackage{microtype}
\UseMicrotypeSet[protrusion]{basicmath} % disable protrusion for tt fonts
}{}
\ifxetex
  \usepackage[setpagesize=false, % page size defined by xetex
              unicode=false, % unicode breaks when used with xetex
              xetex]{hyperref}
\else
  \usepackage[unicode=true]{hyperref}
\fi
\hypersetup{breaklinks=true,
            bookmarks=true,
            pdfauthor={},
            pdftitle={Short Writing Assignment 2},
            colorlinks=true,
            citecolor=blue,
            urlcolor=blue,
            linkcolor=magenta,
            pdfborder={0 0 0}}
\urlstyle{same}  % don't use monospace font for urls
\setlength{\parindent}{0pt}
\setlength{\parskip}{6pt plus 2pt minus 1pt}
\setlength{\emergencystretch}{3em}  % prevent overfull lines
\setcounter{secnumdepth}{0}

\title{Short Writing Assignment 2}
\date{}

\begin{document}
\maketitle

\subsection{HARDCOPIES!}\label{hardcopies}

Due to a shoulder injury, I need to refrain from typing. Please submit
this assignment to me as a hard-copy in class on October 12th.

\subsection{Plagiarism}\label{plagiarism}

Please review the plagiarism policy on the syllabus. It is critical that
you prepare your assignment by yourself. Use only the assigned readings
and handouts---it will take you less time to work through these
materials than to find and read other sources. I will be checking for
significant overlaps between submissions as well as checking answers
against Wikipedia, internet search results, standard essay sites, etc.
If you include material in your essay without citing it, you will
receive 0 for the assignment. A second violation will result in a 0 for
the course, a report to the Dean, and a petition for a note to be added
to your permanent academic record.

\subsection{Late Submissions}\label{late-submissions}

Per the policies outlined in the syllabus, late work will not be
accepted. As the policies also state, there are no make-ups or extra
credit opportunities. Any request for special treatment will be ignored.
If you foresee difficulties submitting work on time, either because of
personal or work commitments, then you should start this paper early and
submit it early.

\subsection{Grading}\label{grading}

Please find the rubric and explanation of it
\href{/Teaching/Grading/}{here}.

\subsection{Resources}\label{resources}

Please find links to writing resources \href{/Teaching/Resources/}{here}

\subsection{Word Count}\label{word-count}

Your submission must be 500-750 words long. Essays shorter than 500
words or longer than 750 words will lose points.

\subsection{Prompt}\label{prompt}

Zeno uses his Stadium Paradox to argue that motion is impossible in
discrete space, i.e., a space in which there is a smallest spatial
distance. Assume that space is discrete. This essay has two parts.

\begin{enumerate}
\def\labelenumi{\arabic{enumi}.}
\itemsep1pt\parskip0pt\parsep0pt
\item
  Explain the Stadium Paradox\\
\item
  Try solve the paradox while maintaining the assumption that space is
  discrete.
\end{enumerate}

\subsection{Further Instruction}\label{further-instruction}

\begin{itemize}
\itemsep1pt\parskip0pt\parsep0pt
\item
  Please use diagrams, illustrations, etc. to explain the paradox. They
  help!
\item
  We didn't discuss possible solutions in class. This was by design. The
  second part of this exercise is asking you to think for yourself. It's
  ok if your solution ultimately won't work. I want you to try!
\item
  Assume that your audience is bright, but not familiar with the topic.
  You will need to explain the terminology you introduce.
\end{itemize}

\end{document}
