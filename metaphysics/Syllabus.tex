\documentclass[article,oneside]{memoir}

%%% custom style file with standard settings for xelatex and biblatex. Note that when [minion] is present, we assume you have minion pro installed for use with pdflatex.
%\usepackage[minion]{org-preamble-pdflatex} 

%%% alternatively, use xelatex instead
\usepackage{org-preamble-xelatex} 



\usepackage{org-preamble-xelatex} 
\usepackage{ulem}
\usepackage{lscape}
\usepackage{multirow}
\def\myauthor{Author}
\def\mytitle{Title}
\def\mycopyright{\myauthor}
\def\mykeywords{}
\def\mybibliostyle{plain}
\def\mybibliocommand{}
\def\mysubtitle{}
\def\myaffiliation{NJCU}
\def\myaddress{Phil 1}
\def\myemail{soconnor@njcu.edu}
\def\myweb{\href{http://scottoconnor.org/metaphysics}{http://scottoconnor.org/metaphysics}}
\def\myphone{}
\def\myversion{}
\def\myrevision{}
\def\myaffiliation{NJCU}
\def\myauthor{Dr. Scott O'Connor}
\def\mykeywords{}
\def\mysubtitle{Syllabus}
\def\mytitle{{\normalsize \myweb \newline} \HUGE Metaphysics}



\begin{document}

%%% If using xelatex and not pdflatex
%%% xelatex font choices
\defaultfontfeatures{}
\defaultfontfeatures{Scale=MatchLowercase}    
% You will need to buy these fonts, change the names to fonts you own, or comment out if not using xelatex.      
\setromanfont[Mapping=tex-text]{Minion Pro} 
\setsansfont[Mapping=tex-text]{Myriad Pro} 
\setmonofont[Mapping=tex-text,Scale=0.8]{Georgia} 

%% blank label items; hanging bibs for text
%% Custom hanging indent for vita items
\def\ind{\hangindent=1 true cm\hangafter=1 \noindent}
\def\labelitemi{$\cdot$}
%\renewcommand{\labelitemii}{~}

%% RCS info string for version tracking
\chapterstyle{article-3}  % alternative styles are defined in latex-custom-kjh/needs-memoir/
%\pagestyle{kjh}

\title{\LARGE \mytitle}     
\author{\Large\myauthor \newline \footnotesize\texttt{\noindent Office hours: \href{http://scottoconnor.org/contact/office/}{http://scottoconnor.org/contact/office/}}}

\published{\textbf{Phil 205 (2993), 3 Credits, Fall 2022, M\&W @ 9:55 pm., Rossey 304.} }

\maketitle

% \thispagestyle{kjhgit}

% Copyright Page
%\textcopyright{} \mycopyright


%
% Main Content
%

\section{Disclaimer}
 This syllabus is subject to change at the discretion of the faculty. Students will be notified of such changes ahead of time via Blackboard. 

\section{Copyright}
The materials used in this class, including, but not limited to, lectures, exams, quizzes, and homework assignments are copyright protected works.  Any unauthorized copying of the class materials or recording of lectures is a violation of federal law and may result in disciplinary actions being taken against the student.  Additionally, the sharing of class materials without the specific, express approval of the instructor may be a violation of the University's Student Honor Code and an act of academic dishonesty, which could result in further disciplinary action.  This includes, among other things, uploading class materials to websites for the purpose of sharing those materials with other current or future students. 

\section{Course Description and Objectives}

Most of us believe that change exists, that babies are born, that trees grow, and that planes fly in the sky. However, the existence of change has long been doubted. Some deny that any kind of change exists. Other deny that certain kinds of change like motion exist. These denials seem radical. Surely babies are born, trees do grow, and planes really do fly in the sky. Nevertheless, the arguments that change does not exist are powerful and our goal will be to understand and assess them. 


\section{Learning Objectives}

Upon completing this course, students will be able to (i) read
philosophical texts, (ii) clearly and charitably explain viewpoints that
are not their own, (iii) think critically and philosophically, (iv)
write well-structured prose in which they clearly state a thesis and
persuasively defend it, (v) demonstrate an understanding of several core
metaphysical issues.

\section{Readings}

There is a required textbook for this course. All other readings are available on the course website:
\begin{itemize}
\item `Metaphysics: An Introduction', Alyssa Ney, Routledge, 2014
\end{itemize}
This course is part of the Barnes \& Noble First Day Complete program.  The bookstore has provided you with a convenient package for physical books and any digital materials that have been integrated into Blackboard for this course. 
You should have received an email from the bookstore confirming your materials and asking you to select how you would like to receive your printed materials (pick up at bookstore or delivery to your home). If you haven't done so already, please make sure you select your fulfillment preference through the link below so the bookstore can prepare your materials.\begin{itemize}

\item To view your materials: \\ \href{https://sso.bncollege.com/bes-sp/bessso/saml/njcuedu/aip/logon}{https://sso.bncollege.com/bes-sp/bessso/saml/njcuedu/aip/logon}
\item For more information on Barnes \& Noble First Day Complete, please visit: \\ \href{https://www.njcu.edu/academics/academic-success-resources/barnes-noble-first-day-complete-students}{https://www.njcu.edu/academics/academic-success-resources/barnes-noble-first-day-complete-students}   
\end{itemize}
All readings will be posted on the course website. You do not need to purchase anything for this course. 


\section{Course Website}

There is both a Blackboard site and website for this course (link on first page). All assignments will be submitted through Blackboard. Readings, notes, etc. will be posted on the course website. Note that Blackboard difficulties are rare and automatically reported to instructors. Under no circumstance will a student's report of a Blackboard difficulty be reason for an extension. It is your responsibility to contact Blackboard support for help.


\section{Requirements}

\begin{itemize}
\item \textit{Workload:} expect to spend an average of 6 hours per week completing the readings and assignments. NJCU abides by the Federal and State definitions of a credit hour and adopts a policy consistent with the Carnegie Unit. A three-credit class represents 112.5 hours total of work. See \href{http://scottoconnor.org/resources/Credit.pdf}{here} for more details.

%\item \textit{Attendance:} Roll call will be taken. 0.5 points will be awarded per class up to a maximum of 10 points. Points will not be awarded during weeks 1 \& 2.  

\item \textit{1 logic problem set} submitted through Blackboard. 


\item \textit{5 short essays (250--500 words)} submitted through Blackboard. 

\item \textit{2 long essays (1000--1250 words)} submitted through Blackboard. 


\item \textit{Course evaluations} completed online. 3 points extra credit for successful completion and screenshot of completed page sent through Blackboard. 

\item \textit{Course video} completed online. 5 points extra credit for successful completion of a short reflection video. Details will be announced in class.  

\item \textit{Grade Distribution:} logic problem set---10 points; short essays---10 points each (50 total); long essay 1---20 points; long essay 2---20 points. 

\item \textit{Grade Breakdown:}

 \begin{tabular}{ | l | l | p{2cm} | l | l | }
    \hline 
96--100 & A  & &  77--79 &  C+ \\  
90--95 & A- & &  73--76 & C \\
87-89 & B+ &  &  70--72 & C- \\ 
83--86 & B  & &  60--69 & D\\
80--82 & B - & & 0--59 & F\\ \hline
    \end{tabular}


\end{itemize}





\section{Policies}

\begin{itemize}

\item \textbf{Student Responsibility:} This syllabus outlines the required text, assignments, requirements, and policies for this course. By taking this course, you agree to read this syllabus and be bound by those requirements and policies. 

 \item \textit{Academic Integrity:} All the work you turn in (including papers, drafts, and discussion board posts) must be written by you specifically for this course. It must originate with you in form and content with all contributory sources fully and specifically acknowledged. Being a student at NJCU requires you to follow \href{http://scottoconnor.org/resources/Plagiarism.pdf}{NJCU's Academic Integrity Policy.} Penalties for violations are as follows: 1st infraction will result in a 0 for the assignment.  2nd infraction will result in a 0 for the entire course \& application for permanent record on student's transcript. (Repeated violations can lead to expulsion from NJCU). 

%\item \textit{Attendance:} You are considered absent if you are (i) not present during roll call, (ii) leave early, (iii) leave without permission, or (iv) leave for an extended period of time. No excuses. No exceptions.

\item \textit{Communication:} To comply with Federal Privacy Laws (FERPA) and NJCU policies, all communication will be through Blackboard and/or official NJCU e-mail. Check Blackboard daily. For further information see \href{http://scottoconnor.org/contact/}{http://scottoconnor.org/contact/}.

\item \textit{Conduct:} Distracting and disrespectful behaviors, including but not limited to eating, leaving your seat, talking out of turn, and aggression are prohibited. Penalties include, but are not limited to, a loss of participation points for the day of violation. Repeat offenders will be reported to the Dean of Students. 

\item \textit{Electronic devices:} Use of electronic device, including, but not limited, to smartphones, dictaphones, tablets, and laptops, is prohibited. Recording a lecture is in violation of Copyright. Penalties include, but are not limited to, a loss of participation points for the day of violation. Repeat offenders will be reported to the Dean of Students.

\item \textit{Format for Written Work:} Submit work to Blackboard as either a pdf, rtf, or doc file. Blackboard will not allow any other format. All work must be typed and neatly presented. 


\item \textit{Grading:} Grades will be available within 1--2 weeks of an assignment being submitted. See: \href{http://scottoconnor.org/resources/grading}{http://scottoconnor.org/resources/grading} for further information.





\item \textit{Late work \& Make-up Policy:} See the assignment schedule below. Late work will be accepted up until the final day of this course. However, comments will not be given on late work and grades will not be returned on late work until the course concludes. 

\item \textit{Statement for students with disabilities:} If you are a student
with a disability and wish to receive consideration for reasonable
accommodations, please register with the Office of Specialized Services
and Supplemental Instruction (OSS/SI). To begin this process, complete
the registration form available on the OSS/SI website at
\href{http://www.njcu.edu/oss}{http://www.njcu.edu/oss}
(listed under Student Resources-Forms). Contact OSS/SI at 201-200-2091
or visit the office in Karnoutsos Hall, Room 102 for additional
information.

%\item \textit{Turnitin:} Students agree that by taking this course all assignments are subject to submission for textual similarity review to Turnitin.com. Assignments submitted to Turnitin.com will be included as source documents in Turnitin.com's restricted access database solely for the purpose of detecting plagiarism in such documents.  The terms that apply to the University’s use of the Turnitin.com service are described on the Turnitin.com web site.  For further information about Turnitin, please visit: http://www.turnitin.com 



\end{itemize}



\section{Weekly Course Schedule}
Dates refer to the first day of the week. Complete the readings before the first class of the week. All handouts and readings not in the textbook (TEXTBOOK) will be available on the website. Changes to the syllabus will be announced in class and \emph{via} your NJCU email address. All assignments must be submitted through Blackboard by Monday, 10:00am. No late work accepted. No exceptions. 


\begin{description}
\item[Introduction:]  What  is Metaphysics? 
\begin{description}
\item [9/7:] Introduction
\begin{enumerate}
\item `Logic for Metaphysics', ch.1. TEXTBOOK, \emph{read very slowly}
\end{enumerate}
\end{description}

\item[Module 1:] Generation \& Destruction
\begin{description}
\item [9/12:] Parmenides' Challenge
\begin{enumerate}
\item \emph{On Nature}, Parmenides
\item `An Introduction to Ontology', ch.1, pp.30--36, TEXTBOOK
\end{enumerate}

\item [9/19:] Potential and Actual Being 
\begin{enumerate}
\item \textbf{Logic Problem Set due Blackboard}
\item \emph{Physics}\ Bk 1, ch.7--8, Aristotle
\item `Explaining Nature and the Nature of Explanation', Christopher Shields

 
\end{enumerate}
\end{description}

\item[Module 2:] Ontology
\begin{description}

\item [9/26:] Do Properties Exist? 

\begin{enumerate}
\item `Abstract Entities', ch.2, pp.60--71, 77--81, TEXTBOOK
\end{enumerate}

\item[10/3: ] Does Race Exist? 

\begin{enumerate}
\item \textbf{Short essay 1 due}

\item `The Metaphysics of Race', ch.10, TEXTBOOK

\end{enumerate}

\end{description}


\item[Module 3:] Space \& Motion
\begin{description}
\item[10/10:] Absolute vs. Relational Theories of Space
\begin{enumerate}
\item `Space, Absolute and Relational', Tim Maudlin
\item `Incongruent Counterparts and Higher Dimensions', James Van Cleve
\end{enumerate}

\item[10/17:] The Fourth Dimension
\begin{enumerate} 
\item \textbf{Short essay 2 due}
\item `Flatland', selections, Edwin Abbott 
\item `The Fourth Dimension: an Excerpt from The Ambidextrous Universe', Martin Gardner
\end{enumerate}
\end{description}

\begin{description}
\item [10/24:] Zeno's Paradoxes
\begin{enumerate}
\item \emph{Physics}\ Bk 6, Aristotle
\item `A Contemporary Look at Zeno’s Paradoxes: an Excerpt from Space, Time, and Motion.', Wesley C. Salmon. 
\item `The Paradoxes of Motion and the Possibility of Change', E.J. Lowe, pp. 288-306
\end{enumerate}
\end{description}


\item[Module 5:] Time 
\begin{description}
\item [10/31:] Does Time Pass?
\begin{enumerate}
\item \textbf{Short essay 3 due}
\item `Time', ch.5, pp.138--146, TEXTBOOK
\item `Albert's Time', Falk
\end{enumerate}

\item[11/7:] Does Time Exist?
\begin{enumerate}
\item `SlaughterHouse Five', selections, Kurt Vonnegut
\item `Time', ch.5, pp.146--162, TEXTBOOK
\item `Time: an Excerpt from The Nature of Existence', J. McT. E. McTaggart.
\end{enumerate}
\end{description}

\item[Module 6:] Material Objects and Change
\begin{description}
\item[11/14:] The Puzzles 
\begin{enumerate}
\item \textbf{Long essay 1 due}

\item `Material Objects', ch.3, pp.89--103, TEXTBOOK; `Persistence', ch.6., pp.170--172, TEXTBOOK
\item `Of Confused Subjects Which Are Equivalent to Two Subjects:. An Excerpt from The Port-Royal Logic', Antoine Arnauld and Pierre Nicole. 
\end{enumerate}
\item[11/21/] Composition 
\begin{enumerate}
\item \textbf{Short essay 4 due}
\item `Material Objects', ch.3., pp.103--117, TEXTBOOK
\end{enumerate}
\item[11/28/] Persistence
\begin{enumerate}
\item `Persistence', ch.6., pp.173--189, TEXTBOOK
 \item `Identity, Ostension, and Hypostasis', W. V. O. Quine.
\item `In Defense of Stages: Postscript B to ``Survival and Identity'' ', David Lewis.
\end{enumerate}
\end{description}
\item[Module 7:] Time Travel
\begin{description}
\item[12/5:] Is Time Travel Possible?
\begin{enumerate}
\item \textbf{Short essay 5 due}
\item `All You Zombies', Robert Heinlein 
\item `Time', ch.5, pp.162--167, TEXTBOOK
\item `Interfering with History', Robin Le Poidevin 
\end{enumerate}

\item[12/12:] Time Travel: Continued (No class on Wed).
\begin{enumerate}
\item `The Paradoxes of Time Travel', David Lewis.
\item \textbf{No class on Wednesday}
\end{enumerate}

\end{description}

\item[Concluding:] Long Essay
\begin{description}
\item[12/19/17:] No classes. Submit work on \textbf{Tuesday} by midnight.
\begin{enumerate}
\item No exam.

\end{enumerate}
\end{description}
\end{description} 



%% Uncomment if you want a printed bibliography.
%\printbibliography 

\end{document}