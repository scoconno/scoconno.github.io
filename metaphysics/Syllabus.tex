\documentclass[article,oneside]{memoir}

%%% custom style file with standard settings for xelatex and biblatex. Note that when [minion] is present, we assume you have minion pro installed for use with pdflatex.
%\usepackage[minion]{org-preamble-pdflatex} 

%%% alternatively, use xelatex instead
\usepackage{org-preamble-xelatex} 



\def\myauthor{Author}
\def\mytitle{Title}
\def\mycopyright{\myauthor}
\def\mykeywords{}
\def\mybibliostyle{plain}
\def\mybibliocommand{}
\def\mysubtitle{}
\def\myaffiliation{NJCU}
\def\myaddress{}
\def\myemail{soconnor@njcu.edu}
\def\myweb{\href{http://scottoconnor.org/Teaching/Metaphysics}{http://scottoconnor.org/Teaching/Metaphysics}}
\def\myphone{}
\def\myversion{}
\def\myrevision{}
\def\myaffiliation{NJCU}
\def\myauthor{Dr. Scott O'Connor}
\def\mykeywords{}
\def\mysubtitle{Syllabus}
\def\mytitle{{\normalsize Phil 205 (2982), 3 Credits, Fall 2015, M\&W @ 4:00 p.m., K358. \newline} \HUGE Metaphysics}


\begin{document}

%%% If using xelatex and not pdflatex
%%% xelatex font choices
\defaultfontfeatures{}
\defaultfontfeatures{Scale=MatchLowercase}    
% You will need to buy these fonts, change the names to fonts you own, or comment out if not using xelatex.      
\setromanfont[Mapping=tex-text]{Georgia} 
\setsansfont[Mapping=tex-text]{Georgia} 
\setmonofont[Mapping=tex-text,Scale=0.8]{Georgia} 

%% blank label items; hanging bibs for text
%% Custom hanging indent for vita items
\def\ind{\hangindent=1 true cm\hangafter=1 \noindent}
\def\labelitemi{$\cdot$}
%\renewcommand{\labelitemii}{~}

%% RCS info string for version tracking
\chapterstyle{article-3}  % alternative styles are defined in latex-custom-kjh/needs-memoir/
\pagestyle{kjh}

\title{\LARGE\mytitle}     
\author{\Large\myauthor \newline \footnotesize\texttt{\noindent\myweb}}
\date{9/01/2015--12/21/2015}

\published{\,}

\maketitle

% \thispagestyle{kjhgit}

% Copyright Page
%\textcopyright{} \mycopyright


%
% Main Content
%

\section{Course Description and Objectives}

Most of us believe that change exists, that babies are born, that trees grow, and that planes fly in the sky. However, the existence of change has long been doubted. Some deny that any kind of change exists. Other deny that certain kinds of change like motion exist. These denials seem radical. Surely babies are born, trees do grow, and planes really do fly in the sky. Nevertheless, the arguments that change does not exist are powerful and our goal will be to understand and assess them. 


\section{Learning Objectives}

Upon completing this course, students will be able to (i) read
philosophical texts, (ii) clearly and charitably explain viewpoints that
are not their own, (iii) think critically and philosophically, (iv)
write well-structured prose in which they clearly state a thesis and
persuasively defend it, (v) demonstrate an understanding of several core
metaphysical issues.

\section{Required Textbook}

The following textbook must purchased or rented by 9/14/2015:

\begin{itemize}
\item
\href{http://www.amazon.com/Metaphysics-Questions-Peter-van-Inwagen/dp/1405125861/ref=sr_1_1?ie=UTF8&qid=1440685163&sr=8-1&keywords=metaphysics+big+questions}{`Metaphysics: The Big Questions', ed. Van Inwagen, 2nd edition.} (Available from the campus book store and online retailers)
\end{itemize}

\section{Course Website}
There is both a Blackboard site and website for this course (link on first page). Clicking the first link on the left panel within the Blackboard site will bring you to the course website. All assignments will be submitted through Blackboard. Readings, notes, etc. will be posted on the course website. Note that Blackboard difficulties are rare and automatically reported to instructors. Under no circumstance will a student's report of a Blackboard difficulty be reason for an extension. It is your responsibility to contact Blackboard support for help.


\section{Requirements}

\begin{itemize}
\item \textit{Workload:} Expect to spend an average of 5--6 hours per week  completing the readings and assignments.

\item \textit{Attendance:} Roll call will be taken. 0.5 point will be awarded per class up to a maximum of 10 points. Points will not be awarded during weeks 1 \& 2. 

\item \textit{3 Discussion Questions} submitted through Blackboard. Sample answers will be made public to the entire class after the due date. 

\item \textit{1 Essay} submitted through Blackboard. 
 
\item \textit{1 Final project} comprising a proposal, short presentation, and written submission in two drafts.



\item \textit{Grade Distribution:} Attendance--0.5 point per class (10 total); 3 Discussion Questions---10 points each (30 total); Essay--30 points; Final Project--10 for presentation, 10 for draft 1, 20 for written submission (40 points total)

\item \textit{Grade Breakdown:}

 \begin{tabular}{ | l | l | p{2cm} | l | l | }
    \hline 
96--110 & A  & &  77--79 &  C+ \\  
90--95 & A- & &  73--76 & C \\
87-89 & B+ &  &  70--72 & C- \\ 
83--86 & B  & &  60--69 & D\\
80--82 & B - & & 0--59 & F\\ \hline
    \end{tabular}


\end{itemize}





\section{Policies}

\begin{itemize}

\item \textbf{Student Responsibility:} This syllabus outlines the required text, assignments, requirements, and policies for this course. By taking this course, you agree to read this syllabus and be bound by those requirements and policies. 

 \item \textit{Academic Integrity:} All the work you turn in (including papers, drafts, and discussion board posts) must be written by you specifically for this course. It must originate with you in form and content with all contributory sources fully and specifically acknowledged. Being a student at NJCU requires you to follow \href{http://www.njcu.edu/uploadedFiles/About_NJCU/Governance_and_Organization/University_Senate/Policies/Academic\%20INTEGRITY\%20POLICY\%20FINAL\%202-04.pdf}{NJCU's Academic Integrity Policy.} Penalties for violations are as follows: 1st infraction will result in a 0 for the assignment.  2nd infraction will result in a 0 for the entire course \& application for permanent record on student's transcript. (Repeated violations can lead to expulsion from NJCU). 

\item \textit{Attendance:} You are considered absent if you are (i) not present during roll call, (ii) leave early, (iii) leave without permission, or (iv) leave for an extended period of time. No excuses. No exceptions.



\item \textit{Communication:} To comply with Federal Privacy Laws (FERPA) and NJCU policies, all communication will be through Blackboard and/or official NJCU e-mail. Check both your NJCU e-mail and Blackboard daily. For further information see \href{http://scoconno.github.io/Contact/}{http://scoconno.github.io/Contact/}.

\item \textit{Conduct:} Distracting and disrespectful behaviors, including but not limited to eating, leaving your seat, talking out of turn, and aggression are prohibited. Penalties include, but are not limited to, a loss of attendance points for the day of violation. Repeat offenders will be reported to the Dean of Students. 

\item \textit{Electronic devices:} Use of electronic device, including, but not limited, to smartphones, dictaphones, tablets, and laptops, is prohibited. Recording a lecture is in violation of Copyright. Penalties include, but are not limited to, a loss of attendance points for the day of violation. Repeat offenders will be reported to the Dean of Students.

\item \textit{Format for Written Work:} Submit work to Blackboard either as a rich text or Microsoft Word file. All work must be typed. Your papers should be in 12-point Times New Roman font, double-spaced with margins set to one inch on all sides. If hard copies are requested, please staple or paperclip copies of papers and drafts.



\item \textit{Grading:} Grades will be available within 1 week of an assignment being submitted. See: \href{http://scoconno.github.io/Teaching/Grading}{http://scoconno.github.io/Teaching/Grading} for further information.


\item \textit{Late work \& Make-up Policy:} See the assignment schedule below. No make-ups or late work accepted under any circumstances. No exceptions. But note that there are 110 points available with 96+ being required for an A.


\item \textit{Statement for students with disabilities:} If you are a student with a disability and wish to receive consideration for reasonable accommodations, please register with the Office of Specialized Services and Supplemental Instruction (OSS/SI). To begin this process, complete the registration form available on the OSS/SI website at
\href{http://www.njcu.edu/Specialized_Services.aspx}{www.njcu.edu/Specialized\_Services.aspx}
(listed under Student Resources-Forms). Contact OSS/SI at 201-200-2091
or visit the office in Karnoutsos Hall, Room 102 for additional
information.

\end{itemize}



\section{Weekly Course Schedule}
Dates refer to the first day of the week. Complete the readings before the first class of the week. Readings marked with a `**' can be found on the course website. Handouts can be found under the relevant modules on the course website. All other listed readings can be found in the required textbook. Changes to the syllabus will be announced in class and \emph{via} your NJCU email address.


\begin{description}
\item[Module 0:] \href{http://scoconno.github.io/Teaching/Metaphysics/Intro}{Introduction}
\begin{enumerate}
\item \textit{08/31} Introduction
\end{enumerate}
\item[Module 1:] \href{http://scoconno.github.io/Teaching/Metaphysics/Space}{Space} 
\begin{enumerate}
\item \textit{09/07} 1, 2, and 3 Dimensions
\begin{enumerate}
\item **`Flatland', Edwin A. Abbott
\end{enumerate}

\item \textit{09/14} Absolute vs. Relative Theories of Space
\begin{enumerate}
\item **`Flatland', continued.
\item `The Fourth Dimension: an Excerpt from The Ambidextrous Universe', Martin Gardner
\item `Incongruent Counterparts and Higher Dimensions', James Van Cleve
\end{enumerate}
\item \textit{09/21} Continued
\begin{enumerate}
\item Continued
\end{enumerate}
\end{enumerate}
\item[Module 2:] \href{http://scoconno.github.io/Teaching/Metaphysics/Motion}{Motion}
\begin{enumerate}
\item \textit{09/28} Zeno's Paradoxes
\begin{enumerate}
\item `A Contemporary Look at Zeno’s Paradoxes: an Excerpt from Space, Time, and Motion.', Wesley C. Salmon. 
\item **`The Paradoxes of Motion and the Possibility of Change', E.J. Lowe, pp. 288-297
\end{enumerate}
\item \textit{10/05} Continued
\begin{enumerate}
\item **`The Paradoxes of Motion and the Possibility of Change', E.J. Lowe, pp. 300-306
\end{enumerate}
\end{enumerate}
\item[Module 3:] \href{http://scoconno.github.io/Teaching/Metaphysics/Time}{Time}
\begin{enumerate}
\item \textit{10/12} Two Theories
\begin{enumerate}
\item **`SlaughterHouse Five', by Kurt Vonnegut (online)
\end{enumerate}
\item  \textit{10/19} Does Time Exist?
\begin{enumerate}
\item `Time: an Excerpt from The Nature of Existence', J. McT. E. McTaggart.
\item `McTaggart’s Arguments against the Reality of Time: an Excerpt from Examination of McTaggart’s Philosophy', C. D. Broad.
\item `The Myth of Passage', D. C. Williams.
\end{enumerate}

\item \textit{10/26} The Direction of Time
\begin{enumerate}
\item **`The Arrows of Time', Robin Le Poidevin 
\end{enumerate}
\end{enumerate}
\item[Module 4:] \href{http://scoconno.github.io/Teaching/Metaphysics/Change}{Change}
\begin{enumerate}
\item \textit{11/02} Puzzles about persistence
\begin{enumerate}
\item `Of Confused Subjects Which Are Equivalent to Two Subjects:. An Excerpt from The Port-Royal Logic', Antoine Arnauld and Pierre Nicole. 
\item **`Change and Identity', Michael Rea, pp.102--111 
\item `The Paradox of Increase', Eric T. Olson (optional)
\end{enumerate}
\item \textit{11/09} Solutions, \textbf{no class 11/11}
\begin{enumerate}
 \item `Identity, Ostension, and Hypostasis', W. V. O. Quine.
\item `In Defense of Stages: Postscript B to ``Survival and Identity'' ', David Lewis.
\end{enumerate}
\end{enumerate}
\item[Module 5:] \href{http://scoconno.github.io/Teaching/Metaphysics/TT}{Time Travel}
\begin{enumerate}
\item \textit{11/16} Is Time Travel Possible?
\begin{enumerate}
\item **`All You Zombies', Robert Heinlein 
\item **`Interfering with History', Robin Le Poidevin 
\end{enumerate}
\item \textit{11/23} Time Travel: Continued.
\begin{enumerate}
\item `The Paradoxes of Time Travel', David Lewis.
\end{enumerate}
\end{enumerate}
\item[Module 6:] \href{http://scoconno.github.io/Teaching/Metaphysics/Personal}{Personal Identity}
\begin{enumerate}
\item \textit{11/30} Final Projects, \textbf{Mon. class will attend Philosophy Colloquium}
\item \textit{12/07} Final Projects
\item \textit{12/14} Final  Projects (no class on Wed)
\end{enumerate}
\end{description}

\section{ Assignment Schedule}
Dates refer to the due date. All assignments must be submitted through Blackboard by 1:00pm. No late work accepted. No make-ups. No exceptions. 

\begin{enumerate}
\item \textit{09/28/2015,} SW1
\item \textit{10/12/2015,} SW2
\item \textit{11/02/2015,} SW3\footnote{Change made 10/25}
\item \textit{11/23/2015,} Essay 
\item \textit{12/11/2015,} Final Projects--First Draft Due\footnote{Change made 11/28}
\item \textit{12/21/2015,} Final Projects--Final Draft
\end{enumerate}




%% Uncomment if you want a printed bibliography.
%\printbibliography 

\end{document}