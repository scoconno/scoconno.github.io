\documentclass[oneside]{article}
 \headheight = 25pt
\footskip = 20pt
\usepackage[T1]{fontenc}
\renewcommand{\rmdefault}{ppl}

\usepackage{fancyhdr}
 \pagestyle{fancy}
 \lhead{\textbf{\textsc{\small Scott O'Connor\\The Philosophy of Time (Phil1111)}}}
 \chead{}
 \rhead{\LARGE\textbf{\textsc{Travel Guide}}}
 \lfoot{\footnotesize{\thepage}}
 \cfoot{}
 \rfoot{\footnotesize{\today}}
\tolerance=700

\begin{document}
\thispagestyle{fancy}

\section*{Travel Guide}

In `Slaughterhouse-Five', Kurt Vonnegut describes an alien race called the Tralfamadorians. The Tralfamadorians experience time in a radically different way from how humans experience time. They claim that they see all of time as a large landscape; just as you can see the lake in front of the mountain, they claim that the see `all together' every event laid out before and after each other. They struggle to explain this to the main character, Billy Pilgrim, and struggle to understand just how Billy experiences his time. Imagine you are a tour guide introducing Tralfmadorian to Earth, and humans to Tralfmadore. You can imagine, if you like, that you do this on the space ship which carries them to their respective destinations. First, explain to the human visitors how the inhabitants understand and experience time. Then  explain to a Tralfmadorian visitor to Earth how the inhabitants there understand and experience time. 


\subsection*{Further Instruction}

\begin{itemize} 
\item Your audience are tourists! Some of these tourists are likely children. Write very simple and clearly!  
\item Feel free to use diagrams. 
\item Aim for 5 minutes on each part. 5 minutes is about 1-2 pages at most. 
\end{itemize}
\section*{Assignments}

\subsection*{Read for Tuesday}
\begin{itemize}
\item `The Unreality of Time', by J.M.E McTaggart (READER).
\item  `Does Time Pass', TRAVELS, ch.8. Think about the questions at the end of the chapter. We'll discuss these in class.
\end{itemize}

\subsection*{Style Exercises} 
\begin{itemize}
\item Lesson 8. Exercises 8.1 and 8.2. Do the even numbers in each. This is your last set of exercises. Yay!!! 
\end{itemize}
\end{document}
