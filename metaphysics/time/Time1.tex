\documentclass[oneside]{article}
 \headheight = 25pt
\footskip = 20pt
\usepackage{graphicx}

\usepackage{mdwlist}
\usepackage[T1]{fontenc}
\renewcommand{\rmdefault}{ppl}
\usepackage{fancyhdr}
 \pagestyle{fancy}
 \lhead{\textbf{\textsc{\small Scott O'Connor\\Metaphysics}}}
 \chead{}
 \rhead{\large\textbf{\textsc{Time 1}}}
 \lfoot{\footnotesize{\thepage}}
 \cfoot{}
 \rfoot{\footnotesize{\today}}
\tolerance=700


\begin{document}
\thispagestyle{fancy}

\subsection*{Introduction}

\begin{quote}
For what is time? Who can easily and briefly explain it? Who even in thought can comprehend it, even to the pronouncing of a word concerning it? But what in speaking do we refer to more familiarly and knowingly than time? And certainly we understand when we speak of it; we understand also when we hear it spoken of by another. What, then, is time? If no one ask of me, I know; if I wish to explain to him who asks, I know not (Augustine of Hippo). 
\end{quote}


\subsection*{The A-Series and the B-Series}

\begin{itemize}
\item Events are ordered in time. The A-series and the B-series are different theories about the nature of this ordering. 
\item A-properties are one-place relations: \emph{being past, being two days past, being present, being future, being 1 year in the future}.
\begin{itemize}
\item Time flows by events changing in their A-properties. 
\item Example: in 1992 the event of Obama winning the election had the property of being a future event, an event that will happen. At that time, it did not have the property of being a present event, an event that was currently happening. Nor did it have the property of being a past event, an event that had occurred. This event briefly acquired the property of being present in 2008 before losing that property and becoming a past event. 
\end{itemize}
\item B-properties are two-place relations: \emph{earlier than, later than, simultaneous with}
\begin{itemize}
\item B-properties are relations between events, e.g., the event of Obama winning the election was later than the event of George Bush winning the election.
\item Events never change their B-properties, e.g., it is eternally true that Obama won the election after George Bush won the election.
\item Consider a spatial analogy....volunteers please.
\end{itemize}
\item A-Series: both A and B properties exist. 
\item B-Series: A properties cannot exist. Only B-properties exist. 
\end{itemize}

\subsection*{Objects at Different Times}

Distinguish two senses of `x exists now'. 1) `x is present'. 2) `x is in the domain of all things that exist.' 

\begin{description}
\item[Presentism] is the view that only present objects exist. If we were to make an accurate list of all the things that exist, there would not be a single non-present object on the list. Thus, you and the Taj Mahal would be on the list, but neither Socrates nor any future Martian outposts would be included. The same goes for any other putative object that lacks the property of being present. All such objects are unreal, according to Presentism.
\item[Eternalism] is the view that objects from both the past and the future exist just as much as present objects. According to Eternalism, non-present objects like Socrates and future Martian outposts exist right now, even though they are not currently present. We may not be able to see them at the moment, on this view, and they may not be in the same space-time vicinity that we find ourselves in right now, but they should nevertheless be on the list of all existing things.
\item[The Growing Universe Theory] is the view that only objects that are either past or present, but not objects that are future, exist. 
\end{description}






\end{document} 

